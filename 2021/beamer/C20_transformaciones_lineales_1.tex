%\documentclass{beamer} 
\documentclass[handout]{beamer} % sin pausas
\usetheme{CambridgeUS}

\usepackage{etex}
\usepackage{t1enc}
\usepackage[spanish,es-nodecimaldot]{babel}
\usepackage{latexsym}
\usepackage[utf8]{inputenc}
\usepackage{verbatim}
\usepackage{multicol}
\usepackage{amsgen,amsmath,amstext,amsbsy,amsopn,amsfonts,amssymb}
\usepackage{amsthm}
\usepackage{calc}         % From LaTeX distribution
\usepackage{graphicx}     % From LaTeX distribution
\usepackage{ifthen}
%\usepackage{makeidx}
\input{random.tex}        % From CTAN/macros/generic
\usepackage{subfigure} 
\usepackage{tikz}
\usepackage[customcolors]{hf-tikz}
\usetikzlibrary{arrows}
\usetikzlibrary{matrix}
\tikzset{
	every picture/.append style={
		execute at begin picture={\deactivatequoting},
		execute at end picture={\activatequoting}
	}
}
\usetikzlibrary{decorations.pathreplacing,angles,quotes}
\usetikzlibrary{shapes.geometric}
\usepackage{mathtools}
\usepackage{stackrel}
%\usepackage{enumerate}
\usepackage{enumitem}
\usepackage{tkz-graph}
\usepackage{polynom}
\polyset{%
	style=B,
	delims={(}{)},
	div=:
}
\renewcommand\labelitemi{$\circ$}
\setlist[enumerate]{label={(\arabic*)}}
%\setbeamertemplate{background}[grid][step=8 ] %cuadriculado
\setbeamertemplate{itemize item}{$\circ$}
\setbeamertemplate{enumerate items}[default]
\definecolor{links}{HTML}{2A1B81}
\hypersetup{colorlinks,linkcolor=,urlcolor=links}


\newcommand{\Id}{\operatorname{Id}}
\newcommand{\img}{\operatorname{Im}}
\newcommand{\nuc}{\operatorname{Nu}}
\newcommand{\im}{\operatorname{Im}}
\renewcommand\nu{\operatorname{Nu}}
\newcommand{\la}{\langle}
\newcommand{\ra}{\rangle}
\renewcommand{\t}{{\operatorname{t}}}
\renewcommand{\sin}{{\,\operatorname{sen}}}
\newcommand{\Q}{\mathbb Q}
\newcommand{\R}{\mathbb R}
\newcommand{\C}{\mathbb C}
\newcommand{\K}{\mathbb K}
\newcommand{\F}{\mathbb F}
\newcommand{\Z}{\mathbb Z}
\newcommand{\N}{\mathbb N}
\newcommand\sgn{\operatorname{sgn}}
\renewcommand{\t}{{\operatorname{t}}}
\renewcommand{\figurename }{Figura}

%
% Ver http://joshua.smcvt.edu/latex2e/_005cnewenvironment-_0026-_005crenewenvironment.html
%

\renewenvironment{block}[1]% environment name
{% begin code
	\par\vskip .2cm%
	{\color{blue}#1}%
	\vskip .2cm
}%
{%
	\vskip .2cm}% end code


\renewenvironment{alertblock}[1]% environment name
{% begin code
	\par\vskip .2cm%
	{\color{red!80!black}#1}%
	\vskip .2cm
}%
{%
	\vskip .2cm}% end code


\renewenvironment{exampleblock}[1]% environment name
{% begin code
	\par\vskip .2cm%
	{\color{blue}#1}%
	\vskip .2cm
}%
{%
	\vskip .2cm}% end code




\newenvironment{exercise}[1]% environment name
{% begin code
	\par\vspace{\baselineskip}\noindent
	\textbf{Ejercicio (#1)}\begin{itshape}%
		\par\vspace{\baselineskip}\noindent\ignorespaces
	}%
	{% end code
	\end{itshape}\ignorespacesafterend
}


\newenvironment{definicion}[1][]% environment name
{% begin code
	\par\vskip .2cm%
	{\color{blue}Definición #1}%
	\vskip .2cm
}%
{%
	\vskip .2cm}% end code

\newenvironment{observacion}[1][]% environment name
{% begin code
	\par\vskip .2cm%
	{\color{blue}Observación #1}%
	\vskip .2cm
}%
{%
	\vskip .2cm}% end code

\newenvironment{ejemplo}[1][]% environment name
{% begin code
	\par\vskip .2cm%
	{\color{blue}Ejemplo #1}%
	\vskip .2cm
}%
{%
	\vskip .2cm}% end code

\newenvironment{ejercicio}[1][]% environment name
{% begin code
	\par\vskip .2cm%
	{\color{blue}Ejercicio #1}%
	\vskip .2cm
}%
{%
	\vskip .2cm}% end code


\renewenvironment{proof}% environment name
{% begin code
	\par\vskip .2cm%
	{\color{blue}Demostración}%
	\vskip .2cm
}%
{%
	\vskip .2cm}% end code



\newenvironment{demostracion}% environment name
{% begin code
	\par\vskip .2cm%
	{\color{blue}Demostración}%
	\vskip .2cm
}%
{%
	\vskip .2cm}% end code

\newenvironment{idea}% environment name
{% begin code
	\par\vskip .2cm%
	{\color{blue}Idea de la demostración}%
	\vskip .2cm
}%
{%
	\vskip .2cm}% end code

\newenvironment{solucion}% environment name
{% begin code
	\par\vskip .2cm%
	{\color{blue}Solución}%
	\vskip .2cm
}%
{%
	\vskip .2cm}% end code



\newenvironment{lema}[1][]% environment name
{% begin code
	\par\vskip .2cm%
	{\color{blue}Lema #1}\begin{itshape}%
		\par\vskip .2cm
	}%
	{% end code
	\end{itshape}\vskip .2cm\ignorespacesafterend
}

\newenvironment{proposicion}[1][]% environment name
{% begin code
	\par\vskip .2cm%
	{\color{blue}Proposición #1}\begin{itshape}%
		\par\vskip .2cm
	}%
	{% end code
	\end{itshape}\vskip .2cm\ignorespacesafterend
}

\newenvironment{teorema}[1][]% environment name
{% begin code
	\par\vskip .2cm%
	{\color{blue}Teorema #1}\begin{itshape}%
		\par\vskip .2cm
	}%
	{% end code
	\end{itshape}\vskip .2cm\ignorespacesafterend
}


\newenvironment{corolario}[1][]% environment name
{% begin code
	\par\vskip .2cm%
	{\color{blue}Corolario #1}\begin{itshape}%
		\par\vskip .2cm
	}%
	{% end code
	\end{itshape}\vskip .2cm\ignorespacesafterend
}

\newenvironment{propiedad}% environment name
{% begin code
	\par\vskip .2cm%
	{\color{blue}Propiedad}\begin{itshape}%
		\par\vskip .2cm
	}%
	{% end code
	\end{itshape}\vskip .2cm\ignorespacesafterend
}

\newenvironment{conclusion}% environment name
{% begin code
	\par\vskip .2cm%
	{\color{blue}Conclusión}\begin{itshape}%
		\par\vskip .2cm
	}%
	{% end code
	\end{itshape}\vskip .2cm\ignorespacesafterend
}







\newenvironment{definicion*}% environment name
{% begin code
	\par\vskip .2cm%
	{\color{blue}Definición}%
	\vskip .2cm
}%
{%
	\vskip .2cm}% end code

\newenvironment{observacion*}% environment name
{% begin code
	\par\vskip .2cm%
	{\color{blue}Observación}%
	\vskip .2cm
}%
{%
	\vskip .2cm}% end code


\newenvironment{obs*}% environment name
	{% begin code
		\par\vskip .2cm%
		{\color{blue}Observación}%
		\vskip .2cm
	}%
	{%
		\vskip .2cm}% end code

\newenvironment{ejemplo*}% environment name
{% begin code
	\par\vskip .2cm%
	{\color{blue}Ejemplo}%
	\vskip .2cm
}%
{%
	\vskip .2cm}% end code

\newenvironment{ejercicio*}% environment name
{% begin code
	\par\vskip .2cm%
	{\color{blue}Ejercicio}%
	\vskip .2cm
}%
{%
	\vskip .2cm}% end code

\newenvironment{propiedad*}% environment name
{% begin code
	\par\vskip .2cm%
	{\color{blue}Propiedad}\begin{itshape}%
		\par\vskip .2cm
	}%
	{% end code
	\end{itshape}\vskip .2cm\ignorespacesafterend
}

\newenvironment{conclusion*}% environment name
{% begin code
	\par\vskip .2cm%
	{\color{blue}Conclusión}\begin{itshape}%
		\par\vskip .2cm
	}%
	{% end code
	\end{itshape}\vskip .2cm\ignorespacesafterend
}








\title[Clase 20 - Transformaciones lineales 1]{Álgebra/Álgebra II \\ Clase 20 - Transformaciones lineales 1}

\author[]{}
\institute[]{\normalsize FAMAF / UNC
	\\[\baselineskip] ${}^{}$
	\\[\baselineskip]
} 
\date[10/11/2020]{10 de noviembre de 2020}



\begin{document}
\begin{frame}
\maketitle
\end{frame}

\begin{frame}
    \begin{definicion}
    Una \textit{transformación lineal} entre dos espacios vectoriales $V$ y $W$ es una función $T:V\longrightarrow W$ tal que \pause
    \begin{enumerate}
     \item Preserva la suma: 
     $$
     T(v+v')=T(v)+T(v')
     \quad\forall\,v,v'\in V
     $$\pause
     \item Preserva el producto por escalares
     $$
     T(\lambda v)=\lambda T(v)
     \quad\forall\,v\in V,\,\lambda\in\R
     $$ 
    \end{enumerate}
    
    \end{definicion}
    \pause
    \begin{observacion}
        $T:V\longrightarrow W$  es transformación lineal  $\Leftrightarrow$ 
        $$
        T(v+\lambda v')=T(v)+\lambda T(v')\quad\forall\,v,v'\in V, \lambda \in \K.
        $$
    
    \end{observacion}
    
\end{frame}

\begin{frame}


Ya conocemos algunas transformaciones lineales. Por ejemplo:\pause
\vskip .4cm
\begin{itemize}
    \item La derivada: 
    $$(f+cg)'=f'+cg'$$\pause
    \item La integral: $$\int_a^b (f+cg) \,dx=\int_a^b f \,dx+ c \int_a^b g \,dx.$$\pause
    \item La multiplicación por matrices: $$A(v+\lambda v')=Av+\lambda Av'$$
   \end{itemize}

\


\end{frame}

\begin{frame}

Un isomorfismo lineal es una transformación lineal que es biyectiva.
\pause
\


Por ejemplo,  si definimos 
\begin{align*}
&T:\R_{<n}[x]\longrightarrow\R^n,\\ &T(a_{n-1}x^{n-1}+\cdots+a_0)=(a_{n-1}, ...,, a_0) 
\end{align*}

Esta identificación es un isomorfismo.  
\vskip .4cm\pause
Otro isomorfismo que hemos mencionado repetidas veces es 
\begin{equation*}
    \begin{array}{rccc}
        S: &\K^{2 \times 2}& \to &\K^4 \\
    &\begin{bmatrix}
        a_{11} & a_{12} \\ a_{21}&a_{22}
    \end{bmatrix}    &\mapsto &( a_{11}, a_{12}, a_{21},a_{22})
    \end{array}
\end{equation*}

\begin{teorema} Sean $V$ y $W$ dos espacios vectoriales de dimensión finita tal que $\dim(V) = \dim(W)$ entonces $V$ y $W$ son isomorfos.
\end{teorema}

\end{frame}



\begin{frame}{Ejemplos de transformaciones lineales}



\begin{ejemplo*}
        Sea $T : \K^3 \to \K^2$ definida por
        $$
        T(x_1,x_2,x_3) = (2x_1 - x_3, -x_1+3x_2+x_3).
        $$
        Entonces, $T$  es una transformación lineal. 
        
        \vskip .4cm\pause

        La demostración es rutinaria y  parte de un resultado más general. 
        
        \vskip .4cm\pause

    Observar que si 
    $$
     A = \begin{bmatrix}
     2&0&-1 \\ -1&3&1
     \end{bmatrix},
    $$
    entonces
    $$
    \begin{bmatrix}
    2&0&-1 \\ -1&3&1
    \end{bmatrix} 
    \begin{bmatrix}
    x_1\\x_2\\x_3
    \end{bmatrix} =
    \begin{bmatrix}
    2x_1 - x_3 \\ -x_1+3x_2+x_3
    \end{bmatrix}.
    $$
 
    \end{ejemplo*}   

\end{frame}

\begin{frame}
    \begin{observacion}\label{obs-tl-1.5} 	Sea $T: \K^n \to \K^m$. En  general si $T(x_1,\ldots,x_n)$ en cada coordenada tiene una combinación lineal de los $x_1,\ldots,x_n$,  entonces $T$ es una transformación lineal. Mas precisamente, si $T$ está definida por
        \begin{align*}
        T(x_1,\ldots,x_n) &= (a_{11}x_1+\cdots + a_{1n}x_n,\, \ldots\,,a_{m1}x_1+\cdots + a_{mn}x_n )\\
        &=(\sum_{j=1}^n a_{1j} x_j,\ldots,\sum_{j=1}^n a_{mj} x_j),
        \end{align*}
        con $a_{ij} \in \K$, entonces $T$  es lineal. 
    \end{observacion}\pause
    \begin{proof} Ejercicio. \pause Lo  vemos  en  p. \ref{obs-nu-im}. \qed
    \end{proof}
\end{frame}

\begin{frame}{Ejemplos (continuado)}

    \begin{ejemplo}

Sea $C^1$ el espacio de funciones derivables. Entonces la derivada es una transformación lineal pues:
$$
(f+g)'=f'+g'\quad\mbox{y}\quad
(\lambda f)'=\lambda f'
$$

\end{ejemplo}

\pause

\begin{ejemplo}

Sea $V$ un espacio vectorial. La función \textit{identidad $\Id:V\longrightarrow V$},
$$
\Id(v)=v\quad\forall\, v\in V,
$$
es una transformación lineal. 
\end{ejemplo}
\end{frame}


\begin{frame}{(contra)Ejemplos}
\begin{exampleblock}{Ejemplo}
No todas las funciones son transformaciones lineales. La función $f(x)=x^2$ de $\R$ en $\R$ no es lineal. Probamos esto dando un ejemplo concreto donde no se verifique algunas de las propiedades. Por ejemplo:
$$
(1+1)^2=4\neq2= 1^2+1^2.
$$


\end{exampleblock}
 \vskip 3cm
\end{frame}

\begin{frame}



 Podemos dar una definición equivalente de transformación lineal que reúna las dos condiciones en sólo una. 
 \vskip .4cm
Algo similar a la definición de subespacio.
\pause
\vskip .4cm
  \begin{definicion}

Una \textit{transformación lineal} entre dos espacios vectoriales $V$ y $W$ es una función $T:V\longrightarrow W$ tal que
\vskip .2cm
\begin{itemize}
 \item $
 T(\lambda v+v')=\lambda T(v)+T(v')
 \quad\forall\,v,v'\in V, \lambda\in\R
 $
\end{itemize}
\end{definicion}
\vskip .4cm
Podemos usar cualquiera de las dos definiciones para decidir si una función es transformación lineal o no.


\end{frame}


\begin{frame}


\begin{block}{Observación}
Sea $T:V\longrightarrow W$ una transformación lineal. Entonces $T(0)=0$. \pause
\end{block}
\begin{demostracion}
    \vskip -0.8cm
\begin{align*}
T(0)&=T(0+0)&&\\
T(0)&=T(0)+T(0)&&\text{(linealidad de $T$)}\\
-T(0) + T(0)&=- T(0) + T(0)+T(0)&&\\
0&= 0 + T(0) = T(0).&& \\
\end{align*}\qed 
\end{demostracion}

\vskip 2cm

\end{frame}


\begin{frame}

Entre otras cosas esta propiedad, es útil como ``test'' para verificar si una función \textit{no} es transformación lineal. 
\pause
\
\begin{exampleblock}{Ejemplo}
Sea $V$ un espacio vectorial y $v_0\in V$ un vector no nulo. Entonces la función $f:V\longrightarrow V$ dada por 
$$
f(v)=v+v_0\quad\forall v\in V
$$
no es lineal dado que
$$
f(0)=0+v_0=v_0\neq0.
$$
\end{exampleblock}


\end{frame}

\begin{frame}
\begin{observacion}
Las transformaciones lineales preservan combinaciones lineales, es decir si $T:V\longrightarrow W$ es una transformación lineal, $v_1, ..., v_k\in V$ y $\lambda_1, ..., \lambda_k\in\R$, entonces
$$
T(\lambda_1v_1+\cdots+\lambda_kv_k)=\lambda_1T(v_1)+\cdots+\lambda_kT(v_k)
$$
\end{observacion}\pause
\begin{block}{Esquema de la demostración}\pause
    La demostración sigue por inducción y aplicando la definición de t. lineal. 
\begin{itemize}
    \item \textbf{Caso base.}  $T(\lambda_1v_1)=\lambda_1T(v_1)$. Lo cual es cierto porque es una de las condiciones de la definición de transformación lineal.
    \item \textbf{Paso inductivo.} 
    \begin{align*}
        T(\lambda_1v_1+\cdots+\lambda_kv_k) &= T(\lambda_1v_1)+T(v_2+\cdots+\lambda_kv_k)&&\text{(T es t.l.)}\\
        &=  \lambda_1T(v_1)+\cdots+\lambda_kT(v_k)&&\text{(C.B. e HI)}
    \end{align*}
\end{itemize}
\qed
\end{block}

\end{frame}




\begin{frame}
\begin{definicion}
Sea $T:V\longrightarrow W$ una transformación lineal.
\begin{itemize}\pause
\item La \textit{imagen de $T$} es el subconjunto de $W$
 \begin{align*}
 \img(T)=\left\{T(v)\mid v\in V\right\}=\left\{w\in W\mid T(v)=w\right\}
 \end{align*}\pause
 \item El \textit{núcleo de $T$} es el subconjunto de $V$
 \begin{align*}
 \nu(T)=\left\{v\in V\mid T(v)=0\right\}
 \end{align*}
\end{itemize}
\end{definicion}
\end{frame}

\begin{frame}

\begin{block}{Observación}
\begin{itemize}\pause
\item $\img(T)$ se define como la imagen de cualquier función. \pause

\item $\nu(T)$ serían las raíces de la transformación. \pause

\item $\nu(T)$ es definido de forma implícita al igual que la segunda expresión de $\img(T)$. \pause

\item La primera expresión de $\img(T)$ es de forma explícita o paramétrica, donde el parámetro es un vector.

\end{itemize}

\begin{block}{Notación}

    Si $T : V \to W$ transformación lineal denotamos
$$
          T(V) := \{T(v): v \in V \} = \im(T).
$$
    
\end{block}
 
\end{block}

\end{frame}

\begin{frame}
El núcleo y la imagen son importante entre otras cosas por lo siguiente. 
\vskip .4cm \pause
\begin{block}{Teorema}
Sea $T:V\longrightarrow W$ una transformación lineal. Entonces 
\begin{itemize}\pause
\item $\img(T)$ es un subespacio vectorial de $W$.\pause
 \item $\nu(T)$ es un subespacio vectorial de $V$.
\end{itemize}
\end{block}

A continuación haremos la demostración.
\end{frame}

\begin{frame}

    \begin{block}{Demostración: $\nu(T)$ es subespacio}\vskip .4cm\pause

        \begin{itemize}
            \item  $\nuc(T) \ne \emptyset$ pues $T(0) =0$ y por lo tanto $0 \in \nuc(T)$.\pause
            \vskip .4cm
            \item  Si $v,w \in V$ tales que $T(v) =0$ y $T(w)=0$,  entonces,\pause
            \vskip .4cm

            \begin{itemize}
                \item  $T(v+w)= T(v)+T(w) =0$  $\Rightarrow$ $v+w \in \nuc(T)$. \vskip .4cm\pause

                \item  Si  $\lambda \in \K$,  entonces $T(\lambda v) = \lambda T(v) = \lambda.0 =0$  $\Rightarrow$  $\lambda v \in \nuc(T)$.
            \end{itemize}
        \end{itemize}

    \end{block}

\end{frame}


\begin{frame}

    \begin{block}{Demostración: $\im(T)$ es subespacio}\pause
        \vskip .4cm
        \begin{itemize}
            \item  $\im(T) \ne \emptyset$,  pues $0 = T(0) \in \im(T)$.  \vskip .4cm\pause
            \item   Si $T(v_1),T(v_2) \in \img(T)$ y $\lambda \in \K$,  entonces \vskip .4cm\pause
            \begin{itemize}
                \item   $T(v_1) + T(v_2) = T(v_1+v_2) \in \im(T)$. \vskip .4cm\pause
                \item $\lambda T(v_1) = T(\lambda v_1) \in \im(T)$.
            \end{itemize}
        \end{itemize}

    \end{block}
\end{frame}


\begin{frame}

    \begin{lema}
    Sea $T:V\longrightarrow W$ una transformación lineal con $V$ de dimensión finita. Sea $\{v_1, ..., v_k\}$ una base de $V$. Entonces $\{T(v_1), ..., T(v_k)\}$ genera a $\im(T)$ y por lo tanto $\im(T)$ es de dimensión finita.
    \end{lema}\pause
    \begin{proof}\pause
        Por hipotesis: $V=\langle v_1, ..., v_k\rangle=\left\{\lambda_1v_1+\cdots+\lambda_k v_k\mid \lambda_1, ..., \lambda_k\in\K\right\}$.
        \begin{align*}
        \mbox{Luego, }\,\im(T)&=\left\{T(v)\mid v\in V\right\}\\
        &=\left\{T(\lambda_1v_1+\cdots+\lambda_kv_k)\mid \lambda_1, ..., \lambda_k\in\K\right\}\\
        &=\left\{\lambda_1T(v_1)+\cdots+\lambda_kT(v_k)\mid \lambda_1, ..., \lambda_k\in\K\right\}\\
        &=\langle T(v_1), ..., T(v_k)\rangle.
        \end{align*}
        Entonces $\im(T)$ es generado por $S=\{T(v_1), ..., T(v_k)\}$. Por Teorema 3.3.9, existe un subconjunto $\mathcal B$ de $S$ que es base de $\im(T)$. En particular, $\im(T)$ es de dimensión finita. \qed     
    \end{proof}
    
    \end{frame}

\begin{frame}

    Sea $T:V\longrightarrow W$ una transformación lineal y supongamos que $V$ es de dimensión finita. 

    \vskip .4cm\pause

    Como $\nu(T)$  es un subespacio de un espacio  dimensión $< \infty$ $\Rightarrow$ $\dim(\nu T)< \infty$. 

    \vskip .4cm\pause

    Por el lema anterior  $\dim(\im T)< \infty$.

    \vskip .4cm

\begin{exampleblock}{Definición}
Sea $T:V\longrightarrow W$ una transformación lineal y supongamos que $V$ es de dimensión finita. Entonces
\begin{itemize}\pause
 \item El \textit{rango de $T$} es la dimensión de $\im(T)$.\pause
 \item La \textit{nulidad de $T$} es la dimensión de $\nu(T)$.
\end{itemize}

\end{exampleblock}




\end{frame}











\begin{frame}

Todas las transformaciones lineales entre $\R^n$ y $\R^m$ son de la forma ``multiplicar por una matriz''.
\pause
\

Más aún,  toda transformación lineal entre espacios vectoriales de dimensión finita se puede expresar de esta forma.
\pause
\

Así que analicemos un poco más en detalle este tipo de transformaciones.

\vskip 3 cm
\end{frame}


\begin{frame}

\begin{block}{Observación}
Sea $A\in\R^{m\times n}$ y consideramos la función 
\begin{align*}
\begin{array}{rccc}
    T : &\R^n &\to &\R^m \\
        &v &\mapsto &Av.
\end{array}
\end{align*}
Entonces $T$ es una transformación lineal. \pause
\end{block}
\begin{proof}\pause
    Debemos ver que $T$ respeta suma y producto por escalares.
\end{proof}

\pause
Sean $v_1,v_2\in\R^n$ y $\lambda\in\R$ entonces
\begin{align*}
T(v_1+\lambda v_2)=A(v_1+\lambda v_2)=
Av_1+\lambda Av_2=T(v_1)+\lambda T(v_2).
\end{align*}
\qed


\end{frame}

\begin{frame}

    \begin{definicion}
        Sea $A\in\R^{m\times n}$ y sea $T$ la transformación lineal 
        \begin{align*}
            \begin{array}{rccc}
                T : &\R^n &\to &\R^m \\
                    &v &\mapsto &Av.
            \end{array}
            \end{align*}\pause
            Diremos que $T$  es \textit{la transformación lineal asociada a $A$} o  \textit{la transformación lineal inducida por $A$}. 
            
            \vskip .4cm\pause

            Muchas veces denotaremos a esta transformación lineal con el mismo símbolo que la matriz, es decir, en este caso con $A$.  
    \end{definicion}
    
\end{frame}

\begin{frame}

\begin{ejemplo*}
Consideremos la matriz $A=\begin{bmatrix}   1&1&1\\   2&2&2 \end{bmatrix}$.

\vskip .2cm

Entonces si $v =(x,y,z)$, 
\begin{align*}
A(v) &= \begin{bmatrix} 1&1&1\\2&2&2 \end{bmatrix} \begin{bmatrix}  x\\y\\z \end{bmatrix}
= \begin{bmatrix}  x+y+z\\2x+2y+2z \end{bmatrix}           
\end{align*}

\
\pause
En particular,
$(1,-1,0)\in\nu(A)$ pues 
\vskip .2cm 
$A(1,-1,0)=(1 + (-1) +0, 2\cdot 1 + 2 \cdot(-1) + 2 \cdot 0) = (0,0)$
\vskip .2cm 
y
\begin{align*}
A(1,0,0)&=(1,2)\in\im(A)\\
A(0,1,\pi)&=(1+\pi,2+2\pi)\in\im(A)
\end{align*}
\end{ejemplo*}

\end{frame}




\begin{frame}
    \begin{observacion}\label{obs-nu-im} 	Sea $T: \K^n \to \K^m$ definida por
        \begin{align*}
        T(x_1,\ldots,x_n) &= (a_{11}x_1+\cdots + a_{1n}x_n,\, \ldots\,,a_{m1}x_1+\cdots + a_{mn}x_n )
        \end{align*}
        con $a_{ij} \in \K$, \pause entonces 
        \begin{equation*}
            T(x) = \begin{bmatrix*}
                a_{11}& a_{12} & \cdots &a_{1n} \\
                a_{21}& a_{22} & \cdots &a_{2n} \\
                \vdots&\vdots&\ddots&\vdots \\
                a_{m1}& a_{m2} & \cdots &a_{mn}
            \end{bmatrix*}
            \begin{bmatrix}
                x_1 \\ x_2 \\ \vdots \\ x_n
            \end{bmatrix}
        \end{equation*} 
        \pause
        Es decir,  $T$ es la transformación lineal inducida por la matriz $A = [a_{ij}]$. 
        \vskip .4cm\pause
        Esto demuestra la observación de la p. \ref{obs-tl-1.5}.
    \end{observacion}
\end{frame}


\begin{frame}



\begin{block}{Proposición}

Sea $A\in\R^{m\times n}$ y $T:\R^n\longrightarrow\R^m$ la transformación lineal asociada. Entonces
\begin{itemize}\pause
 \item El núcleo de $T$ es el conjunto de soluciones del sistema homogéneo $AX=0$
 \pause
 \item La imagen de $T$ es el conjunto de los $b\in\R^m$ para los cuales el sistema $AX=b$ tiene solución
\end{itemize}
\end{block}\pause
\begin{demostracion}\pause
    
\end{demostracion}
Se demuestra fácilmente escribiendo las definiciones de los respectivos subconjuntos.
\begin{align*}
v\in\nu(T)\Leftrightarrow Av=0\Leftrightarrow v\,\mbox{ es solución de $AX=0$.}
\end{align*}
\begin{align*}
b\in\im(T)&\Leftrightarrow\exists v\in\R^n\,\mbox{ tal que } Av=b 
\Leftrightarrow\,\mbox{$AX=b$ tienen solución}.
\end{align*}
\qed


\end{frame}



\begin{frame}
    \begin{ejemplo*}
        Sea $T: \R^3 \to \R^4$, definida
        $$
        T(x,y,z) = (x +y ,\,x +2y +z,\,3y +3z,\,2x +4y +2z).
        $$
        \begin{enumerate}
            \item Describir $\nuc(T)$  en forma paramétrica y dar una base.
            \item Describir $\img(T)$  en forma paramétrica y  dar una base. 
        \end{enumerate}
        \end{ejemplo*}\pause
        \begin{solucion}\pause
            La matriz asociada a esta transformación lineal  es 
            \begin{equation*}
                A = \begin{bmatrix}
                1&1&0\\1&2&1\\0&3&3\\2&4&2
                \end{bmatrix}
            \end{equation*}
        \end{solucion}
        \end{frame}


        

        \begin{frame}
            Debemos encontrar la descripción paramétrica de
            \begin{align*}
                \nuc(T) &= \{v=(x,y,z):   A.{v}=0\}\\
                \img(T) &= \{y= (y_1,y_2,y_3,y_4): \text{ tal que } \exists v \in \R^3, A.{v} = {y}  \}
                \end{align*}
           
          
            En  ambos casos, la solución depende de resolver el sistema de ecuaciones cuya matriz asociada es $A$:
            \begin{align*}
            \left[\begin{array}{@{}*{3}{c}|c@{}}1&1&0&y_1\\1&2&1&y_2\\0&3&3&y_3\\2&4&2&y_4 \end{array}\right]
            &\underset{F_4-2F_1}{\stackrel{F_2 -F_1}{\longrightarrow}} 
            \left[\begin{array}{@{}*{3}{c}|c@{}}1&1&0&y_1\\0&1&1&-y_1+y_2\\0&3&3&y_3\\0&2&2&-2y_1 +y_4 \end{array}\right]\\
            &\underset{F_4-2F_2}{\underset{F_3-3F_2}{\stackrel{F_1 -F_2}{\longrightarrow}} } 
            \left[\begin{array}{@{}*{3}{c}|c@{}}1&0&-1&2y_1-y_2\\0&1&1&-y_1+y_2\\0&0&0&3y_1-3y_2+y_3\\0&0&0&-2y_2 +y_4 \end{array}\right],
            \end{align*}
        \end{frame}


        

        \begin{frame}
            \begin{equation*}\label{eq-gen}
            T(x,y,z) = (y_1,y_2,y_3,y_4) \quad\Leftrightarrow \quad
            \begin{array}{rl}
            x -z &= 2y_1-y_2\\ 
            y +z &= -y_1+y_2\\
            0&=3y_1-3y_2+y_3 \\
            0&= -2y_2 +y_4
            \end{array}
            \end{equation*}
            Si hacemos $y_1 = y_2 = y_3 = y_4 = 0$, entonces las soluciones del sistema describen el núcleo de $T$, es decir
            \begin{align*}
            \nuc(T) &= \{(x,y,z):x-z=0, y+z =0 \} = \{(s,-s,s):s \in \R \} \\
            &= \{s(1,-1,1):s \in \R \}
            \end{align*}
            que es la forma paramétrica. 
            
            \vskip .4cm 
            
            Una base del núcleo de $T$  es $\{(1,-1,1)\}$. 
        \end{frame}


        

        \begin{frame}
            
            \begin{equation*}\label{eq-gen}
                T(x,y,z) = (y_1,y_2,y_3,y_4) \quad\Leftrightarrow \quad
                \begin{array}{rl}
                x -z &= 2y_1-y_2\\ 
                y +z &= -y_1+y_2\\
                0&=3y_1-3y_2+y_3 \\
                0&= -2y_2 +y_4
                \end{array}
                \end{equation*} 
            Luego, 
            \begin{align*}
                \img(T) &=  \{(y_1,y_2,y_3,y_4): \text{ tal que $0=3y_1-3y_2+y_3$ y $0= -2y_2 +y_4$} \} 
            \end{align*}
            
            
            Resolviendo este sistema, obtenemos
            \begin{align*}
                \img(T) &=  \{(-\frac13 s + \frac12 t, \frac 12 t, s,t): s,t \in \R \}\\
                &=  \{s(-\frac13,0,1,0)+t(\frac12,\frac12,0,1): s,t \in \R \}
            \end{align*}
            .
            
            Luego $\{(-\frac13,0,1,0),(\frac12,\frac12,0,1) \}$ es una base de $\img(T)$. \qed
            
           
\end{frame}


\end{document}
