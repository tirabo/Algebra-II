%\documentclass{beamer} 
\documentclass[handout]{beamer} % sin pausas
\usetheme{CambridgeUS}

\usepackage{etex}
\usepackage{t1enc}
\usepackage[spanish,es-nodecimaldot]{babel}
\usepackage{latexsym}
\usepackage[utf8]{inputenc}
\usepackage{verbatim}
\usepackage{multicol}
\usepackage{amsgen,amsmath,amstext,amsbsy,amsopn,amsfonts,amssymb}
\usepackage{amsthm}
\usepackage{calc}         % From LaTeX distribution
\usepackage{graphicx}     % From LaTeX distribution
\usepackage{ifthen}
%\usepackage{makeidx}
\input{random.tex}        % From CTAN/macros/generic
\usepackage{subfigure} 
\usepackage{tikz}
\usepackage[customcolors]{hf-tikz}
\usetikzlibrary{arrows}
\usetikzlibrary{matrix}
\tikzset{
	every picture/.append style={
		execute at begin picture={\deactivatequoting},
		execute at end picture={\activatequoting}
	}
}
\usetikzlibrary{decorations.pathreplacing,angles,quotes}
\usetikzlibrary{shapes.geometric}
\usepackage{mathtools}
\usepackage{stackrel}
%\usepackage{enumerate}
\usepackage{enumitem}
\usepackage{tkz-graph}
\usepackage{polynom}
\polyset{%
	style=B,
	delims={(}{)},
	div=:
}
\renewcommand\labelitemi{$\circ$}

%\setbeamertemplate{background}[grid][step=8 ]
\setbeamertemplate{itemize item}{$\circ$}
\setbeamertemplate{enumerate items}[default]
\definecolor{links}{HTML}{2A1B81}
\hypersetup{colorlinks,linkcolor=,urlcolor=links}


\newcommand{\Id}{\operatorname{Id}}
\newcommand{\img}{\operatorname{Im}}
\newcommand{\nuc}{\operatorname{Nu}}
\newcommand{\im}{\operatorname{Im}}
\renewcommand\nu{\operatorname{Nu}}
\newcommand{\la}{\langle}
\newcommand{\ra}{\rangle}
\renewcommand{\t}{{\operatorname{t}}}
\renewcommand{\sin}{{\,\operatorname{sen}}}
\newcommand{\Q}{\mathbb Q}
\newcommand{\R}{\mathbb R}
\newcommand{\C}{\mathbb C}
\newcommand{\K}{\mathbb K}
\newcommand{\F}{\mathbb F}
\newcommand{\Z}{\mathbb Z}
\newcommand{\N}{\mathbb N}
\newcommand\sgn{\operatorname{sgn}}
\renewcommand{\t}{{\operatorname{t}}}
\renewcommand{\figurename }{Figura}

%
% Ver http://joshua.smcvt.edu/latex2e/_005cnewenvironment-_0026-_005crenewenvironment.html
%

\renewenvironment{block}[1]% environment name
{% begin code
	\par\vskip .2cm%
	{\color{blue}#1}%
	\vskip .2cm
}%
{%
	\vskip .2cm}% end code


\renewenvironment{alertblock}[1]% environment name
{% begin code
	\par\vskip .2cm%
	{\color{red!80!black}#1}%
	\vskip .2cm
}%
{%
	\vskip .2cm}% end code


\renewenvironment{exampleblock}[1]% environment name
{% begin code
	\par\vskip .2cm%
	{\color{blue}#1}%
	\vskip .2cm
}%
{%
	\vskip .2cm}% end code




\newenvironment{exercise}[1]% environment name
{% begin code
	\par\vspace{\baselineskip}\noindent
	\textbf{Ejercicio (#1)}\begin{itshape}%
		\par\vspace{\baselineskip}\noindent\ignorespaces
	}%
	{% end code
	\end{itshape}\ignorespacesafterend
}


\newenvironment{definicion}[1][]% environment name
{% begin code
	\par\vskip .2cm%
	{\color{blue}Definición #1}%
	\vskip .2cm
}%
{%
	\vskip .2cm}% end code

\newenvironment{observacion}[1][]% environment name
{% begin code
	\par\vskip .2cm%
	{\color{blue}Observación #1}%
	\vskip .2cm
}%
{%
	\vskip .2cm}% end code

\newenvironment{ejemplo}[1][]% environment name
{% begin code
	\par\vskip .2cm%
	{\color{blue}Ejemplo #1}%
	\vskip .2cm
}%
{%
	\vskip .2cm}% end code

\newenvironment{ejercicio}[1][]% environment name
{% begin code
	\par\vskip .2cm%
	{\color{blue}Ejercicio #1}%
	\vskip .2cm
}%
{%
	\vskip .2cm}% end code


\renewenvironment{proof}% environment name
{% begin code
	\par\vskip .2cm%
	{\color{blue}Demostración}%
	\vskip .2cm
}%
{%
	\vskip .2cm}% end code



\newenvironment{demostracion}% environment name
{% begin code
	\par\vskip .2cm%
	{\color{blue}Demostración}%
	\vskip .2cm
}%
{%
	\vskip .2cm}% end code

\newenvironment{idea}% environment name
{% begin code
	\par\vskip .2cm%
	{\color{blue}Idea de la demostración}%
	\vskip .2cm
}%
{%
	\vskip .2cm}% end code

\newenvironment{solucion}% environment name
{% begin code
	\par\vskip .2cm%
	{\color{blue}Solución}%
	\vskip .2cm
}%
{%
	\vskip .2cm}% end code



\newenvironment{lema}[1][]% environment name
{% begin code
	\par\vskip .2cm%
	{\color{blue}Lema #1}\begin{itshape}%
		\par\vskip .2cm
	}%
	{% end code
	\end{itshape}\vskip .2cm\ignorespacesafterend
}

\newenvironment{proposicion}[1][]% environment name
{% begin code
	\par\vskip .2cm%
	{\color{blue}Proposición #1}\begin{itshape}%
		\par\vskip .2cm
	}%
	{% end code
	\end{itshape}\vskip .2cm\ignorespacesafterend
}

\newenvironment{teorema}[1][]% environment name
{% begin code
	\par\vskip .2cm%
	{\color{blue}Teorema #1}\begin{itshape}%
		\par\vskip .2cm
	}%
	{% end code
	\end{itshape}\vskip .2cm\ignorespacesafterend
}


\newenvironment{corolario}[1][]% environment name
{% begin code
	\par\vskip .2cm%
	{\color{blue}Corolario #1}\begin{itshape}%
		\par\vskip .2cm
	}%
	{% end code
	\end{itshape}\vskip .2cm\ignorespacesafterend
}

\newenvironment{propiedad}% environment name
{% begin code
	\par\vskip .2cm%
	{\color{blue}Propiedad}\begin{itshape}%
		\par\vskip .2cm
	}%
	{% end code
	\end{itshape}\vskip .2cm\ignorespacesafterend
}

\newenvironment{conclusion}% environment name
{% begin code
	\par\vskip .2cm%
	{\color{blue}Conclusión}\begin{itshape}%
		\par\vskip .2cm
	}%
	{% end code
	\end{itshape}\vskip .2cm\ignorespacesafterend
}







\newenvironment{definicion*}% environment name
{% begin code
	\par\vskip .2cm%
	{\color{blue}Definición}%
	\vskip .2cm
}%
{%
	\vskip .2cm}% end code

\newenvironment{observacion*}% environment name
{% begin code
	\par\vskip .2cm%
	{\color{blue}Observación}%
	\vskip .2cm
}%
{%
	\vskip .2cm}% end code


\newenvironment{obs*}% environment name
	{% begin code
		\par\vskip .2cm%
		{\color{blue}Observación}%
		\vskip .2cm
	}%
	{%
		\vskip .2cm}% end code

\newenvironment{ejemplo*}% environment name
{% begin code
	\par\vskip .2cm%
	{\color{blue}Ejemplo}%
	\vskip .2cm
}%
{%
	\vskip .2cm}% end code

\newenvironment{ejercicio*}% environment name
{% begin code
	\par\vskip .2cm%
	{\color{blue}Ejercicio}%
	\vskip .2cm
}%
{%
	\vskip .2cm}% end code

\newenvironment{propiedad*}% environment name
{% begin code
	\par\vskip .2cm%
	{\color{blue}Propiedad}\begin{itshape}%
		\par\vskip .2cm
	}%
	{% end code
	\end{itshape}\vskip .2cm\ignorespacesafterend
}

\newenvironment{conclusion*}% environment name
{% begin code
	\par\vskip .2cm%
	{\color{blue}Conclusión}\begin{itshape}%
		\par\vskip .2cm
	}%
	{% end code
	\end{itshape}\vskip .2cm\ignorespacesafterend
}






\title[Clase 13 - Determinante 2]{Álgebra/Álgebra II \\ Clase 13 - Determinante 2}

\author[]{}
\institute[]{\normalsize FAMAF / UNC
	\\[\baselineskip] ${}^{}$
	\\[\baselineskip]
}
\date[13/10/2020]{13 de octubre de 2020}


\begin{document}

\begin{frame}
\maketitle
\end{frame}

\begin{frame}{Resumen}
	En esta clase veremos 

	\vskip .4cm
\pause
	\begin{itemize}
		\item  Determinantes de matrices elementales.\pause
		\item Si $A$ es invertible,  determinaremos el determinante de la inversa.\pause
		\item Determinante del producto de matrices.

	\end{itemize}

	\vskip .4cm
	\pause
	El tema de esta clase  está contenido de la sección la sección 2.8  del apunte de clase ``Álgebra II / Álgebra - Notas del teórico''.
\end{frame}


\begin{frame}

Recordemos
\vskip.4cm
\begin{teorema}
\pause
    Sea $A\in\K^{n\times n}$.

    \begin{enumerate}
        \item[{\color{blue} E1.}]  Si $c\in\K$ no nulo, 
        $$
        A \stackrel{{cF_i}}{\longrightarrow} B \qquad \Rightarrow   \qquad     \det(B)=c\det(A).
        $$\pause
        \item[{\color{blue} E2.}]
        Si $ 1 \le s , t \le n$ con $s\ne t$ y  $t\in\K$: 
        $$
        A \stackrel{{F_r+tF_s}}{\longrightarrow} B  \qquad \Rightarrow   \qquad \det(B)=\det(A).
        $$\pause
          
        \item[{\color{blue} E3.}]  
        $$
        A \stackrel{{F_r \leftrightarrow F_s}}{\longrightarrow} B  \qquad \Rightarrow   \qquad \det(B)=-\det(A).$$ 
    \end{enumerate}

\end{teorema}
    
\end{frame}


\begin{frame}
    \begin{corolario}\label{cor-filas-nulas} $A  \in \K^{n \times n}$.
		\begin{enumerate}
			\item Si $A$ tiene dos filas iguales,  entonces $\det A=0$.
			\item Si $A$ tiene una fila nula, entonces $\det A =0$.
		\end{enumerate}
	\end{corolario}\pause
	\begin{proof}\pause
        1.Si  $F_r = F_s$ con $r\ne s$. Entonces,
        $$
        A \stackrel{F_r \leftrightarrow F_s}{\longrightarrow} A \qquad \stackrel{\text{T. E3}}{\Rightarrow} \qquad \det(A) = -\det(A).
        $$
        Luego $\det(A) =0$.
\vskip .2cm
        2. Si $F_r = 0$, 
        $$
        A  \stackrel{2F_r }{\longrightarrow} A \qquad \stackrel{\text{T. E1}}{\Rightarrow} \qquad \det(A) = 2\det(A). 
        $$
        Luego $\det(A) =0$.
        \qed
        
	\end{proof}
	
\end{frame}

\begin{frame}
    	
    \begin{corolario} Sea $E=e(\Id_n)$, matriz elemental en $\K^{n \times n}$. \pause
        \begin{enumerate}
            \item[{\color{blue} E1.}]  Si $c\in\K$ no nulo, 
            $$
            \Id_n \stackrel{{cF_i}}{\longrightarrow} E \qquad \Rightarrow   \qquad     \det(E)=c.
            $$\pause
            \item[{\color{blue} E2.}]
            Si $ 1 \le s , t \le n$ con $s\ne t$ y  $t\in\K$: 
            $$
            \Id_n \stackrel{{F_r+tF_s}}{\longrightarrow} E  \qquad \Rightarrow   \qquad \det(E)=1.
            $$
              \pause
            \item[{\color{blue} E3.}]  
            $$
            \Id_n \stackrel{{F_r \leftrightarrow F_s}}{\longrightarrow} E  \qquad \Rightarrow   \qquad \det(E)=-1.$$ 
        \end{enumerate}
\end{corolario}

\pause
\begin{demostracion}
	Se  demuestra trivialmente considerando que en todos los casos $E=e(\Id_n)$ donde $e$  es una operación elemental por fila,  considerando  que $\det(\Id_n)=1$  y aplicando los teoremas E1, E2 y E3. \qed
\end{demostracion}
\end{frame}



\begin{frame}
    \begin{teorema} Sea $A  \in \K^{n \times n}$ y $E$ una matriz elemental $n \times n$. Entonces
        \begin{equation}\label{det-prod-elem-matr}
        \det (EA) = \det E \det A.
        \end{equation}  
    \end{teorema}\pause
    \begin{proof} \pause
       En todos los casos $EA = e(A)$ donde $e$  es una operación elemental por fila.
       \vskip .2cm
       
        (E1) Si $c \not=0$  y $\Id_n \stackrel{cF_r }{\longrightarrow}E$, tenemos $\det(E)=c$ y
        \begin{equation*}
            \det(EA) = \det(e(A)){=} c \cdot \det(A) {=} \det(E) \det(A). 
        \end{equation*}
       
        
        (E2) Si $\Id_n \stackrel{F_r + cF_s}{\longrightarrow}E$, luego $\det(E)=1$ y
        \begin{equation*}
            \det(EA) = \det(e(A)){=}  \det(A) {=} \det(E) \det(A). 
        \end{equation*}
    
        (E3) Ejercicio. \qed
    \end{proof}
\end{frame}



\begin{frame}

       	
    \begin{corolario}\label{cor-det-elem} Sea $A = E_1 E_2 \cdots E_k B$ donde  $E_1, E_2, \ldots, E_k$  son matrices elementales. 
        Entonces, 
        \begin{equation*}
            \det(A) = \det(E_1) \det(E_2) \cdots \det(E_k)\det(B).
        \end{equation*}
\end{corolario}\pause
\vskip -.6cm
\begin{demostracion}\pause
    \vskip -.6cm
$$
\det(A) = \det(E_1(E_2 \cdots E_k B)) = \det(E_1)\det(E_2 \cdots E_k B),
$$
y así sucesivamente (inducción). \qed
\end{demostracion}
    	\vskip .4cm\pause
    \begin{corolario} Sea $A = E_1 E_2 \cdots E_k$ producto de matrices elementales en  $\K^{n \times n}$. 
        Entonces, 
        \begin{equation*}
            \det(A) = \det(E_1) \det(E_2) \cdots \det(E_k).
        \end{equation*}
\qed
\end{corolario}



\end{frame}





\begin{frame}
    
\begin{teorema}
    $A \in \K^{n \times n}$ es invertible si y solo si $\det(A) \ne 0$.
\end{teorema}
\pause
\begin{demostracion}\pause
    ($\Rightarrow$) 
    
    $A$ invertible $\Rightarrow$ $A = E_1 E_2 \cdots E_k$  $\Rightarrow$  $\det(A) = \det(E_1) \det(E_2) \ldots \det(E_k)$. Como el determinante de matrices elementales en no nulo, $\det(A) \ne 0$.
\vskip .4cm
    ($\Leftarrow$) 
    
    Sean $E_1, E_2, \ldots, E_k$ matrices elementales tales que $R = E_1 E_2 \cdots E_k A$ y $R$ es MERF. Luego,
    \begin{equation*}
        \det(R) = \det(E_1) \det(E_2) \cdots \det(E_k) \det(A). 
    \end{equation*}
   


\end{demostracion}

\end{frame}



\begin{frame}
    Como los determinantes de matrices  elementales son no nulos
    \begin{equation*}
        \frac{\det(R)}{\det(E_1) \det(E_2) \cdots \det(E_k) } = \det(A). \tag{*}
    \end{equation*}

    
    Supongamos que $R$ no es la identidad.
    \vskip .3cm
    Entonces $\det(R) =0$ (ver clase pasada) $\stackrel{\text{(*)}}{\Rightarrow}$  $\det(A)=0$, absurdo. 
    \vskip .3cm
    Luego, $R= \Id_n$  $\Rightarrow$ $A$ es equivalente por filas a $\Id_n$  $\Rightarrow$  $A$ invertible.

    \qed
    \vskip 2cm
\end{frame}



\begin{frame}
    \begin{teorema} Sean $A,B \in \K^{n \times n}$, entonces 
        $$\det (A B) = \det(A)\det(B).$$ 
    \end{teorema}\pause
	\begin{proof}\pause
       - Si $A$ invertible $\Rightarrow$  $A= E_1\cdots E_k$ y $AB =  E_1\cdots E_kB$, luego por los corolarios de {\color{blue}p. \ref{cor-det-elem}} $\det(AB) =  \det(E_1)\cdots \det(E_k)\det(B) = \det(A)\det(B)$.
       \vskip .3cm
       - Si $A$ no invertible $\Rightarrow$  $A= E_1\cdots E_kR$ y $R$ MERF con la última fila nula.        
       \vskip .2cm
       Luego,  
       \begin{itemize}
           \item $RB$ tiene la última fila nula $\Rightarrow$ $\det(RB) = 0$ (corolario {\color{blue}p. \ref{cor-filas-nulas}} ).
           \item $\det(AB) = \det(E_1\cdots E_kRB) = \det(E_1\cdots E_k)\det(RB) = 0$.
           \item Como $A$ no invertible $\Rightarrow$ $\det(A) = 0$ $\Rightarrow$ $\det(A)\det(B) = 0$.
       \end{itemize}  
        \qed
	\end{proof}

\end{frame}



\begin{frame}
    
    \begin{corolario} San $A$, $B$ matrices $n \times n$, entonces
        \begin{itemize}
            \item $\det(A^m) = \det(A)^m$, para $m \in \N$.
            \item $\det (AB) = \det(BA)$.
        \end{itemize}
	\end{corolario}\pause
    \begin{proof}\pause
        $\circ$  $\det(A^m) = \det(A \cdot A^{m-1})  =  \det(A )\cdot \det(A^{m-1})$ y se demuestra por inducción.
        \vskip .4cm
        $\circ$   $\det (AB) = \det(A)\det(B) = \det(B)\det(A) = \det(BA)$.
        \qed
	\end{proof}
\vskip 3cm
\end{frame}




\begin{frame}

    \begin{exampleblock}{Definición}
    Sea $A\in\R^{m\times n}$. La \textit{transpuesta de $A$} es la matriz {$A^{\t}\in\R^{n\times m}$} cuyas entradas son definidas por
    $$
    [A^\t]_{ij}=[A]_{ji}
    $$
    \end{exampleblock}\pause

    \begin{ejemplo}
        Sea 
        \begin{equation*}
            A = \begin{bmatrix}
                1&2&3\\
                4&5&6\\
                7&8&9\\
            \end{bmatrix}
            \qquad \Rightarrow \qquad
            A^{\t} = \begin{bmatrix}
                1&4&7\\
                2&5&8\\
                3&6&9\\
            \end{bmatrix}
        \end{equation*}
        \vskip .3cm
    ($  [A^\t]_{12}= [A]_{21} = 4$, $  [A^\t]_{13}= [A]_{31} = 3$,  etc.)
    \end{ejemplo}
    \end{frame}
    
    \begin{frame}
        Para matrices cuadradas, en general: 
    \begin{align*}
    A&=
    \left[
    \begin{array}{ccccc}
     a_{11} & \cdots & a_{1i} & \cdots & a_{1n}\\
     \vdots & & \vdots & & \vdots\\
     a_{i1} & \cdots &  a_{ii} & \cdots & a_{in}\\  
     \vdots & & \vdots & & \vdots\\
     a_{n1} & \cdots & a_{ni} & \cdots & a_{nn}
    \end{array}
    \right] 
    \end{align*}
    
    \
    \pause Entonces, 
    \
    
    \begin{align*}
    A^\t=
    \left[
    \begin{array}{ccccc}
     a_{11} & \cdots & a_{i1} & \cdots & a_{n1}\\
     \vdots & & \vdots & & \vdots\\
     a_{1i} & \cdots & a_{ii} & \cdots & a_{ni}\\  
     \vdots & & \vdots & & \vdots\\
     a_{1n} & \cdots & a_{in} & \cdots & a_{nn}
    \end{array}
    \right] 
    \end{align*} 
    \end{frame}
    
    \begin{frame}
    \begin{teorema}
    El determinante de una matriz cuadrada  es igual al determinante de su transpuesta. 
    \vskip.2cm
    Es decir, si $A$ matriz $n \times n$,
    $$
    \det(A)=\det(A^\t).
    $$
    \end{teorema}
    \vskip.3cm\pause
    Pueden ver la demostración en las notas de curso.
     \vskip .3cm
   \begin{block}{Idea de la demostración}
   
   No es difícil ver que 
   \begin{itemize}
   	 \item para $E$ matriz elemental $\det(E^\t) = \det(E)$,
   	 \item $(A_1 \cdot A_2 \cdots A_k)^\t = A_k^\t \cdots A_2^\t \cdot A_1^\t$.
   \end{itemize}
   Se sigue entonces de $A = E_1 \cdots E_s R$,  con $E_i$  elementales y $R$ MERF.
   \end{block} 
    \end{frame}
    

    \begin{frame}
        Sea 
        \begin{align*}
        A= \left[
        \begin{array}{ccccc}
         a_{11} & \cdots & 0  & \cdots & 0\\
         \vdots & & \vdots & & \vdots\\
         a_{i1} & \cdots &  a_{ii} & \cdots & 0\\  
         \vdots & & \vdots & & \vdots\\
         a_{n1} & \cdots & a_{ni} & \cdots & a_{nn}
        \end{array}
        \right] 
        \end{align*}
        
        \
        triangular inferior. Entonces,
        \
        
        \begin{align*}
            A^\t =
        \left[
        \begin{array}{ccccc}
         a_{11} & \cdots & a_{i1} & \cdots & a_{n1}\\
         \vdots & & \vdots & & \vdots\\
         0 & \cdots & a_{ii} & \cdots & a_{ni}\\  
         \vdots & & \vdots & & \vdots\\
         0 & \cdots & 0 & \cdots & a_{nn}
        \end{array}
        \right] 
        \end{align*}
        
        es triangular superior.
        \end{frame}
        

    
    \begin{frame}
    \begin{proposicion}
    
    El determinante de una matriz triangular inferior es igual al producto de los elementos de la diagonal.
     
    \end{proposicion}
    
\begin{demostracion}
    La transpuesta de una una matriz triangular inferior es una matriz triangular superior.
    
    \
    
    Entonces la proposición es una consecuencia del teorema anterior y la proposición referida al determinante de una triangular superior.

    \qed
\end{demostracion}

     
    \end{frame}
    
   
    
    \begin{frame}
    
    
    \begin{observacion}
    La transpuesta transforma filas en columnas y columnas en filas 
    \end{observacion}
    
    \
    
    Gracias a esta observación podemos deducir como cambia el determinante de una matriz al aplicarle ``operaciones elementales por columna''
    \vskip 4cm
    \end{frame}
    
    
    \begin{frame}
    \begin{exampleblock}{Ejemplo}
    Si una matriz tiene una columna con muchos ceros,  podemos intercambiarla con la primera fila.
    \end{exampleblock}
    
    \begin{align*}
    A=\left[
    \begin{array}{cccc}
    1&5&6&1\\
    2&0&7&1\\
    3&0&8&1\\
    4&0&1&1
    \end{array}
    \right]
    \rightarrow
    B=
    \left[
    \begin{array}{cccc}
    5&1&6&1\\
    0&2&7&1\\
    0&3&8&1\\
    0&4&1&1
    \end{array}
    \right]
    \end{align*}
    Entonces
    \begin{align*}
    \det(A)=-\det(B)=-5\det B(1|1)
    \end{align*}
    \end{frame}
    
    \begin{frame}
    \begin{exampleblock}{Ejemplo}
    Si una matriz tiene una fila con muchos ceros, entonces intercambio esta con la primer fila, luego transpongo y calculo el determinante.
    \end{exampleblock}
    
    \begin{align*}
    A= \left[
    \begin{array}{cccc}
    1&2&3&4\\
    5&0&0&0\\
    6&7&8&9\\
    1&1&1&1
    \end{array}
    \right]
    \rightarrow
    B=
     \left[
    \begin{array}{cccc}
    5&0&0&0\\
    1&2&3&4\\
    6&7&8&9\\
    1&1&1&1
    \end{array}
    \right]
    \rightarrow
    B^\t=
     \left[
    \begin{array}{cccc}
    5&1&6&1\\
    0&2&7&1\\
    0&3&8&1\\
    0&4&1&1
    \end{array}
    \right]
    \end{align*}
    Entonces
    \begin{align*}
    \det(A)=-\det(B)=-\det(B^\t)=-5\det B^\t(1|1)
    \end{align*}
    \end{frame}
    
   
    
    
    \begin{frame}
    
    El determinante se puede calcular desarrollando por cualquier columna o fila.
    
    \begin{teorema}
        Sa $A$ matriz $n \times n$, entonces el determinante
    \begin{itemize}
     \item se puede calcular el determinante por la columna $j$ así:
    \begin{align*}
    \det(A)=\sum_{i=1}^n(-1)^{i+j}a_{ij}\det A(i|j),
    \end{align*}
     \item se puede calcular el determinante por la fila $i$ así:
    \begin{align*}
    \det(A)=\sum_{j=1}^n(-1)^{i+j}a_{ij}\det A(i|j)
    \end{align*}
    \end{itemize}
    \end{teorema}
    (la diferencia entre ambas fórmulas es la variable de la sumatoria)
    \end{frame}
    
    \begin{frame}
    La demostración de este teorema (que no la haremos),  se basa en dos resultados que ya mencionamos. 
    \vskip .2cm
    \begin{itemize}
        \item[(A)]  Teorema E3:
        $$
        A \stackrel{{F_r \leftrightarrow F_s}}{\longrightarrow} B  \qquad \Rightarrow   \qquad \det(B)=-\det(A).$$ 
        \item[(B)] $\det(A^\t) = det(A)$, 
    \end{itemize}
    y un resultado que no es difícil demostrar:
    \begin{itemize}

        \item[(C)]
            $ A \stackrel{{C_r \leftrightarrow C_s}}{\longrightarrow} B  \qquad \Leftrightarrow   \qquad A^\t \stackrel{{F_r \leftrightarrow F_s}}{\longrightarrow} B^\t
             .$
            \end{itemize}
            Luego,
            \begin{itemize}
        \item[(D)]  $\det(B) \stackrel{(B)}{=} \det(B^\t) \stackrel{(A)}{=} -det(A^\t) \stackrel{(B)}{=} -\det(A)$.
    \end{itemize}
   
\vskip .3cm
    Usando estos resultados, y un poco de manipulación de índices, se obtiene una demostración del teorema. 
   
    \end{frame}
    
\end{document}
    
    \begin{frame}{Matriz de Vandermonde}
    La \textit{matriz de Vandermonde} es una matriz $n\times n$ de la siguiente forma
    \[
    \mathtt V = \begin{bmatrix}
    1 & 1 & \cdots & 1\\
    \lambda_1 & \lambda_2 & \cdots & \lambda_n\\
    \lambda_1^2 & \lambda_2^2 & \cdots & \lambda_n^2\\
    \vdots &\vdots & \ddots& \vdots\\
    \lambda_1^{n-1} & \lambda_2^{n-1} & \cdots & \lambda_n^{n-1}\end{bmatrix}.
    \]
    donde $\lambda_1$, ..., $\lambda_n$ son escalares.
    
    \
    
    También se suele llamar matriz de Vandermonde a la transpuesta de esta.
    
    \
    
    Esta matriz aparece naturalmente al tratar de aproximar funciones mediante polinomios
    \end{frame}
    
    \begin{frame}
    \begin{exampleblock}{Problema}
    Sean $\lambda_1$, ..., $\lambda_n$, $y_1$, ..., $y_n$ escalares. Hallar de ser posible un polinomio
    $$
    p(x)=a_{n-1}x^{n-1}+a_{n-2}x^{n-}+\cdots a_{1}x+a_0
    $$
    tal que 
    $$
    p(\lambda_1)=y_1,\quad p(\lambda_2)=y_2,\,..., \quad p(\lambda_n)=y_n
    $$
    \end{exampleblock}
    
    Desarrollando los polinomios evaluados en los $\lambda$'s, el problema se traduce en resolver el siguiente sistema de ecuaciones
    \begin{align*}
    \left\{
    \begin{array}{l}
    a_0+a_{1}\lambda_1+\cdots+a_{n-2}\lambda_1^{n-2}+ a_{n-1}\lambda_1^{n-1}=y_1\\
    a_0+a_{1}\lambda_2+\cdots+a_{n-2}\lambda_2^{n-2}+ a_{n-1}\lambda_2^{n-1}=y_2\\
    \vdots\\
    a_0+a_{1}\lambda_n+\cdots+a_{n-2}\lambda_n^{n-2}+ a_{n-1}\lambda_n^{n-1}=y_n
    \end{array}
    \right.
    \end{align*}
    donde los coeficientes $a_0$, ..., $a_{n-1}$ son las incógnitas
    \end{frame}
    
    \begin{frame}
    Si escribimos este sistema en forma matricial aparece la matriz de Vandermonde transpuesta
    \begin{align*}
    {\mathtt V}^\t
     \left[
    \begin{array}{c}
    a_0\\ \vdots\\a_{n-1}
    \end{array}
    \right]
    =
     \left[
    \begin{array}{c}
    y_1\\ \vdots\\y_{n}
    \end{array}
    \right]
    \end{align*}
    
    Notemos que si ${\mathtt V}^\t$ es invertible, el sistema tiene solución única dada por
    \begin{align*}
     \left[
    \begin{array}{c}
    a_0\\ \vdots\\a_{n-1}
    \end{array}
    \right]
    =
    ({\mathtt V}^\t)^{-1}
     \left[
    \begin{array}{c}
    y_1\\ \vdots\\y_{n}
    \end{array}
    \right]
    \end{align*}
    
    Un ejercicio del Práctico 4 da una fórmula para el determinante de ${\mathtt V}$ que permite decidir cuando es invertible
    \end{frame}
    
    \begin{frame}
    \begin{exampleblock}{Ejercicio}
    $$
    \det(\mathtt V)=\Pi_{1\leq i<j\leq n}(\lambda_j-\lambda_i)
    $$
    Entonces $\mathtt V$ es invertible si y sólo si todos los $\lambda_i$'s son distintos
    \end{exampleblock}
     
    La demostración es por inducción en $n$. Hay que hacer operaciones elementales por filas (y columnas) para reducir el caso $n$ al caso $n-1$ y aplicar la hipótesis inductiva.
    \end{frame}
    
    \begin{frame}
    Primero se aplican una a una las siguientes operaciones:
    \begin{itemize}
     \item La última fila menos $\lambda_1$ por la penúltima
     \item La penúltima fila menos $\lambda_1$ por la antepenúltima
     \item ... La fila $r$ menos $\lambda_1$ por la fila $r-1$  ...
     \item La segunda fila menos $\lambda_1$ por la primera
    \end{itemize}
    
    Así obtenemos la siguiente matriz
    \[
    A = \begin{pmatrix}
    1 & 1 & \cdots & 1\\
    0 & \lambda_2-\lambda_1 & \cdots & \lambda_n-\lambda_1\\
    0 & (\lambda_2-\lambda_1)\lambda_2 & \cdots & (\lambda_n-\lambda_1)\lambda_n\\
    \vdots &\vdots & \ddots& \vdots\\
    0 & (\lambda_2-\lambda_1)\lambda_2^{n-2} & \cdots & (\lambda_n-\lambda_1)\lambda_n^{n-2}\end{pmatrix}.
    \]
    Por el tipo de operaciones que aplicamos resulta que
    $$
    \det\mathtt{V}=\det A=\det A(1|1)
    $$
    
    \end{frame}
    
    \begin{frame}
    Notemos que la columna $j$ de $A(1|1)$ es la columna $j$ de una matriz de Vandermonde $\mathtt{V}_{n-1}$ de tamaño $n-1$
    \begin{align*}
    (\lambda_j-\lambda_1)
     \left[
    \begin{array}{c}
    1\\ \lambda_2\\ \vdots\\ \lambda_n
    \end{array}
    \right]
    \end{align*}
    
    En otras palabras, $A(1|1)$ se obtiene a partir de $\mathtt{V}_{n-1}$ multiplicando la columna $j$ por $(\lambda_j-\lambda_1)$. Entonces
    \begin{align*}
    \det\mathtt{V}&=\det A=\det A(1|1)\\
    &=(\lambda_2-\lambda_1)\cdots(\lambda_2-\lambda_1)\det\mathtt{V}_{n-1}\\
    &=(\lambda_2-\lambda_1)\cdots(\lambda_2-\lambda_1)
    \Pi_{2\leq i<j\leq n}(\lambda_j-\lambda_i)\\
    &=\Pi_{1\leq i<j\leq n}(\lambda_j-\lambda_i)
    \end{align*}
    \end{frame}
    



\end{document}

