%\documentclass{beamer} 
\documentclass[handout]{beamer} % sin pausas
\usetheme{CambridgeUS}

\usepackage{etex}
\usepackage{t1enc}
\usepackage[spanish,es-nodecimaldot]{babel}
\usepackage{latexsym}
\usepackage[utf8]{inputenc}
\usepackage{verbatim}
\usepackage{multicol}
\usepackage{amsgen,amsmath,amstext,amsbsy,amsopn,amsfonts,amssymb}
\usepackage{amsthm}
\usepackage{calc}         % From LaTeX distribution
\usepackage{graphicx}     % From LaTeX distribution
\usepackage{ifthen}
%\usepackage{makeidx}
\input{random.tex}        % From CTAN/macros/generic
\usepackage{subfigure} 
\usepackage{tikz}
\usepackage[customcolors]{hf-tikz}
\usetikzlibrary{arrows}
\usetikzlibrary{matrix}
\tikzset{
	every picture/.append style={
		execute at begin picture={\deactivatequoting},
		execute at end picture={\activatequoting}
	}
}
\usetikzlibrary{decorations.pathreplacing,angles,quotes}
\usetikzlibrary{shapes.geometric}
\usepackage{mathtools}
\usepackage{stackrel}
%\usepackage{enumerate}
\usepackage{enumitem}
\usepackage{tkz-graph}
\usepackage{polynom}
\polyset{%
	style=B,
	delims={(}{)},
	div=:
}
\renewcommand\labelitemi{$\circ$}

%\setbeamertemplate{background}[grid][step=8 ]
\setbeamertemplate{itemize item}{$\circ$}
\setbeamertemplate{enumerate items}[default]
\definecolor{links}{HTML}{2A1B81}
\hypersetup{colorlinks,linkcolor=,urlcolor=links}


\definecolor{airforceblue}{rgb}{0.36, 0.54, 0.66}
\hfsetfillcolor{airforceblue!30}
\hfsetbordercolor{blue!10}


\newcommand{\Id}{\operatorname{Id}}
\newcommand{\img}{\operatorname{Im}}
\newcommand{\nuc}{\operatorname{Nu}}
\newcommand{\im}{\operatorname{Im}}
\renewcommand\nu{\operatorname{Nu}}
\newcommand{\la}{\langle}
\newcommand{\ra}{\rangle}
\renewcommand{\t}{{\operatorname{t}}}
\renewcommand{\sin}{{\,\operatorname{sen}}}
\newcommand{\Q}{\mathbb Q}
\newcommand{\R}{\mathbb R}
\newcommand{\C}{\mathbb C}
\newcommand{\K}{\mathbb K}
\newcommand{\F}{\mathbb F}
\newcommand{\Z}{\mathbb Z}
\newcommand{\N}{\mathbb N}
\newcommand\sgn{\operatorname{sgn}}
\renewcommand{\t}{{\operatorname{t}}}
\renewcommand{\figurename }{Figura}

%
% Ver http://joshua.smcvt.edu/latex2e/_005cnewenvironment-_0026-_005crenewenvironment.html
%

\renewenvironment{block}[1]% environment name
{% begin code
	\par\vskip .2cm%
	{\color{blue}#1}%
	\vskip .2cm
}%
{%
	\vskip .2cm}% end code


\renewenvironment{alertblock}[1]% environment name
{% begin code
	\par\vskip .2cm%
	{\color{red!80!black}#1}%
	\vskip .2cm
}%
{%
	\vskip .2cm}% end code


\renewenvironment{exampleblock}[1]% environment name
{% begin code
	\par\vskip .2cm%
	{\color{blue}#1}%
	\vskip .2cm
}%
{%
	\vskip .2cm}% end code




\newenvironment{exercise}[1]% environment name
{% begin code
	\par\vspace{\baselineskip}\noindent
	\textbf{Ejercicio (#1)}\begin{itshape}%
		\par\vspace{\baselineskip}\noindent\ignorespaces
	}%
	{% end code
	\end{itshape}\ignorespacesafterend
}


\newenvironment{definicion}[1][]% environment name
{% begin code
	\par\vskip .2cm%
	{\color{blue}Definición #1}%
	\vskip .2cm
}%
{%
	\vskip .2cm}% end code

\newenvironment{observacion}[1][]% environment name
{% begin code
	\par\vskip .2cm%
	{\color{blue}Observación #1}%
	\vskip .2cm
}%
{%
	\vskip .2cm}% end code

\newenvironment{ejemplo}[1][]% environment name
{% begin code
	\par\vskip .2cm%
	{\color{blue}Ejemplo #1}%
	\vskip .2cm
}%
{%
	\vskip .2cm}% end code

\newenvironment{ejercicio}[1][]% environment name
{% begin code
	\par\vskip .2cm%
	{\color{blue}Ejercicio #1}%
	\vskip .2cm
}%
{%
	\vskip .2cm}% end code


\renewenvironment{proof}% environment name
{% begin code
	\par\vskip .2cm%
	{\color{blue}Demostración}%
	\vskip .2cm
}%
{%
	\vskip .2cm}% end code



\newenvironment{demostracion}% environment name
{% begin code
	\par\vskip .2cm%
	{\color{blue}Demostración}%
	\vskip .2cm
}%
{%
	\vskip .2cm}% end code

\newenvironment{idea}% environment name
{% begin code
	\par\vskip .2cm%
	{\color{blue}Idea de la demostración}%
	\vskip .2cm
}%
{%
	\vskip .2cm}% end code

\newenvironment{solucion}% environment name
{% begin code
	\par\vskip .2cm%
	{\color{blue}Solución}%
	\vskip .2cm
}%
{%
	\vskip .2cm}% end code



\newenvironment{lema}[1][]% environment name
{% begin code
	\par\vskip .2cm%
	{\color{blue}Lema #1}\begin{itshape}%
		\par\vskip .2cm
	}%
	{% end code
	\end{itshape}\vskip .2cm\ignorespacesafterend
}

\newenvironment{proposicion}[1][]% environment name
{% begin code
	\par\vskip .2cm%
	{\color{blue}Proposición #1}\begin{itshape}%
		\par\vskip .2cm
	}%
	{% end code
	\end{itshape}\vskip .2cm\ignorespacesafterend
}

\newenvironment{teorema}[1][]% environment name
{% begin code
	\par\vskip .2cm%
	{\color{blue}Teorema #1}\begin{itshape}%
		\par\vskip .2cm
	}%
	{% end code
	\end{itshape}\vskip .2cm\ignorespacesafterend
}


\newenvironment{corolario}[1][]% environment name
{% begin code
	\par\vskip .2cm%
	{\color{blue}Corolario #1}\begin{itshape}%
		\par\vskip .2cm
	}%
	{% end code
	\end{itshape}\vskip .2cm\ignorespacesafterend
}

\newenvironment{propiedad}% environment name
{% begin code
	\par\vskip .2cm%
	{\color{blue}Propiedad}\begin{itshape}%
		\par\vskip .2cm
	}%
	{% end code
	\end{itshape}\vskip .2cm\ignorespacesafterend
}

\newenvironment{conclusion}% environment name
{% begin code
	\par\vskip .2cm%
	{\color{blue}Conclusión}\begin{itshape}%
		\par\vskip .2cm
	}%
	{% end code
	\end{itshape}\vskip .2cm\ignorespacesafterend
}







\newenvironment{definicion*}% environment name
{% begin code
	\par\vskip .2cm%
	{\color{blue}Definición}%
	\vskip .2cm
}%
{%
	\vskip .2cm}% end code

\newenvironment{observacion*}% environment name
{% begin code
	\par\vskip .2cm%
	{\color{blue}Observación}%
	\vskip .2cm
}%
{%
	\vskip .2cm}% end code


\newenvironment{obs*}% environment name
	{% begin code
		\par\vskip .2cm%
		{\color{blue}Observación}%
		\vskip .2cm
	}%
	{%
		\vskip .2cm}% end code

\newenvironment{ejemplo*}% environment name
{% begin code
	\par\vskip .2cm%
	{\color{blue}Ejemplo}%
	\vskip .2cm
}%
{%
	\vskip .2cm}% end code

\newenvironment{ejercicio*}% environment name
{% begin code
	\par\vskip .2cm%
	{\color{blue}Ejercicio}%
	\vskip .2cm
}%
{%
	\vskip .2cm}% end code

\newenvironment{propiedad*}% environment name
{% begin code
	\par\vskip .2cm%
	{\color{blue}Propiedad}\begin{itshape}%
		\par\vskip .2cm
	}%
	{% end code
	\end{itshape}\vskip .2cm\ignorespacesafterend
}

\newenvironment{conclusion*}% environment name
{% begin code
	\par\vskip .2cm%
	{\color{blue}Conclusión}\begin{itshape}%
		\par\vskip .2cm
	}%
	{% end code
	\end{itshape}\vskip .2cm\ignorespacesafterend
}








\title[Clase 18 - Espacios vectoriales 4]{Álgebra/Álgebra II \\ Clase 18 - Espacios vectoriales 4}

\author[]{}
\institute[]{\normalsize FAMAF / UNC
	\\[\baselineskip] ${}^{}$
	\\[\baselineskip]
} 
\date[03/11/2020]{3 de noviembre de 2020}



\begin{document}

\begin{frame}
\maketitle
\end{frame}

\begin{frame}{Resumen}
	
    En esta clase veremos que todo espacio vectorial tiene una base,  que es un conjunto de generadores mínimo. En  el caso que este conjunto sea finito, todo otro conjunto de generadores mínimo tendrá el  mismo número de elementos,  y este número será llamado \textit{dimensión}. 

    \vskip .2cm

    Los temas de la clase se ordenan de la siguiente forma: 

    \begin{itemize}
     \item Definición de independencia lineal. 
     \item Definición de base (un conjunto linealmente independiente que genera el espacio).
     \item Ejemplos de bases de espacios vectoriales.  
     \item Propiedades de las bases y dimensión. 
    \end{itemize}
   
    
    \vskip .2cm
    
   El tema de esta clase  está contenido de la sección 3.3 y 3.4 del apunte de clase ``Álgebra II / Álgebra - Notas del teórico''.
    \end{frame}
    


    \begin{frame}{Independencia lineal}
        \begin{definicion}
            Sea $V$ un espacio vectorial sobre $\K$. Un subconjunto $S$ de $V$ se dice \textit{linealmente independiente} (o simplemente, \textit{LI} o \textit{independiente}) si no es linealmente dependiente. 

\vskip .8cm
        \end{definicion}
        \begin{observacion}
            Si el conjunto $S$ tiene solo un número finito de vectores $v_1,\ldots,v_n$, se dice,
            a veces, que los $v_1,\ldots,v_n$ son \textit{independientes} o \textit{LI}, en vez de decir que $S$ es independiente.
        \end{observacion}
            
        
    \end{frame}  
    


   
    \begin{frame}
       
        \begin{observacion}
            Sea $V$ un espacio vectorial sobre $\K$ y $v_1,\ldots,v_n \in V$. 
            \begin{equation*}
                v_1,\ldots,v_n \text{ son LD} \; \Leftrightarrow\; \exists \lambda_i\text{'s} \in \K, \text{ alguno no nulo, t.q. }\;  \lambda_1v_1+\cdots+\lambda_nv_n=0.
            \end{equation*}
        \end{observacion}

        \begin{observacion} Por definición, un conjunto $v_1,\ldots,v_n $ es LI si se cumple:
            \begin{enumerate}
                \item[(a)] $\forall \,\lambda_i \in \K$ tal que $\lambda_j \ne 0$ para algún $j$\; $\Rightarrow$ \;  $\lambda_1v_1+\cdots+\lambda_nv_n\not=0$, 
                
                o bien
                
                \item[(b)]si $\lambda_i \in \K$ tal que $\lambda_1v_1+\cdots+\lambda_nv_n=0$ \; $\Rightarrow$ \;$0=\lambda_1=\cdots=\lambda_n$
            \end{enumerate}

            \vskip .2cm

            El enunciado (a) se deduce  negando  la definición de linealmente dependiente.
            
            \vskip .2cm 
            El enunciado  (b) es el contrarrecíproco  de (a).
            
       
            
        \end{observacion}
    \end{frame}  
    
   
\begin{frame}
     
\begin{ejemplo}
    En $\R^3$  los vectores $(1,-1,1)$ y $(-1,1,1)$ son LI, pues si $\lambda_1(1,-1,1)+\lambda_2(-1,1,1) =0$,  entonces $0= (\lambda_1,-\lambda_1,\lambda_1)+(-\lambda_2,\lambda_2,\lambda_2) =  (\lambda_1-\lambda_2,-\lambda_1+\lambda_2,\lambda_1+\lambda_2)$, 
    
    y esto es cierto si 
    \begin{equation*}
        \begin{array}{rcl}
        \lambda_1-\lambda_2 &=& 0 \\
        -\lambda_1+\lambda_2 &=& 0 \\
        \lambda_1+\lambda_2 &=& 0 
        \end{array}.
    \end{equation*} 
    Luego $\lambda_1 = \lambda_2$ y $\lambda_1 = -\lambda_2$, por lo tanto $\lambda_1 = \lambda_2 =0$. Es decir,  hemos visto que 
    $$
    \lambda_1(1,-1,1)+\lambda_2(-1,1,1) =0 \quad \Rightarrow \quad\lambda_1 = \lambda_2 =0,
    $$
    y, por lo tanto,  $(1,-1,1)$ y $(-1,1,1)$ son LI. \qed
\end{ejemplo}
\end{frame}  
    
   
    \begin{frame}
     
\begin{ejemplo} Sea $\K$  cuerpo. En $\K^3$ los vectores
    \begin{align*}
    v_1 &= (\;\;3,\;0,-3) \\
    v_2 &= (-1,\;1,\;\;2) \\
    v_3 &= (\;\;4,\;2,-2) \\
    v_4 &= (\;\;2,\;1,\,\;\;1)
    \end{align*}
    
    son linealmente dependientes, pues
    $$
    2v_1+2v_2 -v_3 +0.v_4 =0.
    $$
    Por otro lado, los vectores
    \begin{align*}
    e_1 &= (1,0,0) \\
    e_2 &= (0,1,0) \\
    e_3 &= (0,0,1) 
    \end{align*}
    son linealmente independientes. \qed
\end{ejemplo}

    \end{frame}  
    
   
    \begin{frame}
        Las siguientes afirmaciones son consecuencias casi inmediatas de la definición.
        \begin{enumerate}
            \item Todo conjunto que contiene un conjunto linealmente dependiente es linealmente dependiente.
            
            {\color{blue} Dem.} En el conjunto ``más chico''  hay una c. l.  no trivial que lo anula,  luego,  en el ``más grande'' también. \qed
            \item  Todo subconjunto de un conjunto linealmente independiente es linealmente independiente.
            
            {\color{blue} Dem.} Si el subconjnuto tiene una c. l.  no trivial que lo anula,  el conjunto también. \qed
            \item  Todo conjunto que contiene el vector $0$ es linealmente dependiente.
            
            {\color{blue} Dem.} En efecto, $1.0 = 0$. \qed
        \end{enumerate}
    \end{frame}  
    
   
    \begin{frame}
     
\begin{observacion}
    En  general,  en $\K^m$, si queremos determinar si  $v_1,\ldots,v_n$ es LI, planteamos la ecuación  
    \begin{equation*}
    \lambda_1v_1+\cdots+\lambda_nv_n=(0,\ldots,0).
    \end{equation*}
    \vskip .4cm
    
    Viendo esta ecuación  coordenada a coordenada, es equivalente a un sistema de $m$ ecuaciones lineales con  $n$ incógnitas (que son $\lambda_1,\ldots,\lambda_n$). 
    \vskip .4cm
    
    Si  la única solución es la trivial entonces $v_1,\ldots,v_n$ es LI. 
    \vskip .4cm
    
    Si hay alguna solución no trivial, entonces $v_1,\ldots,v_n$ es LD. 
\end{observacion}
    \end{frame}  
    
   
    \begin{frame}
        \begin{definicion}
            Sea $V$ un espacio vectorial. Una \textit{base}\index{base de un espacio vectorial} de $V$ es un conjunto $\mathcal B \subseteq V$ tal que
            \begin{enumerate}
                \item $\mathcal B$ genera a $V$, y
                \item $\mathcal B$ es LI.
            \end{enumerate}
            \vskip .6cm 
             El espacio $V$ es de \textit{dimensión finita} si tiene una base finita,  es decir con  un número finito de elementos.
        \end{definicion}
    \end{frame}  
    
   
    \begin{frame}
     
\begin{block}{ Ejemplo: base canónica de $\K^n$} Sea el espacio vectorial $\K^n$ y sean
    \begin{equation*}
    \begin{array}{rcl}
    e_1 &=& (1,0,0,\ldots,0) \\
    e_2 &=& (0,1,0,\ldots,0) \\
    &&\qquad.\,.\,.\,.\,.\,.\,\\ 
    e_n&=& (0,0,0,\ldots,1)
    \end{array}
    \end{equation*}
    ($e_i$ es el vector con todas sus coordenadas iguales a cero,  excepto  la coordenada $i$ que vale 1). Entonces veamos que  $e_1,\ldots,e_n$ es una base de $\K^n$.
\vskip .2cm
     1. Si $(x_1,\ldots,x_n) \in \K^n$,  entonces
        $$
        (x_1,\ldots,x_n) = x_1e_1+\cdots+x_ne_n.
        $$
        Por lo tanto, $e_1,\ldots,e_n$ genera a  $\K^n$.
\end{block}
 
    \end{frame}  
    
   
    \begin{frame}
        2. Si 
        $$
        x_1e_1+\cdots+x_ne_n =0,
        $$
        entonces
        \begin{align*}
            (0,\ldots,0) &= x_1(1,0,\ldots,0)+ x_2(0,1,\ldots,0)+\cdots+x_n(0,0,\ldots,1)\\ 
            &=  (x_1,0,\ldots,0)+(0,x_2,\ldots,0)+\cdots+(0,0,\ldots,x_n)\\ &= (x_1,x_2,\ldots,x_n).
        \end{align*}
        Luego, $x_1= x_2=\cdots=x_n =0$. 
        
        Por lo tanto $e_1,\ldots,e_n$ es LI.
        
        \vskip .4cm  
        Para $1 \le i \le n$, al vector $e_i$ se lo denomina el \textit{$i$-ésimo vector canónico}  y a la base $\mathcal B_n = \left\{e_1,\ldots,e_n \right\}$ se la denomina la \textit{base canónica}\index{base canónica} de $\K^n$. \qed
    \end{frame}  
    
   
    \begin{frame}
        \begin{block}{Ejemplo: vectores columna de una matriz invertible}
            Sea $P$ una matriz $n \times n$ invertible con elementos en el cuerpo $\K$. Sean $C_1,\ldots,C_n$ son los vectores columna de $P$. 


            Entonces, $\mathcal B = \{C_1,\ldots,C_n\}$, es una base de $\K^n$.
        \end{block}
        \begin{demostracion}
            Si $X = (x_1,\ldots,x_n) \in \K^n$, lo podemos ver como vector columna y 
            $$
            PX=x_1C_1+\cdots+x_nC_n.
            $$
            $PX=0$ tiene solo solución $X= 0$ $\Rightarrow$  $\mathcal B =\{C_1,\ldots,C_n\}$ es LI. 
            \vskip .4cm
            ¿Por qué generan $\K^n$? Sea $Y \in \K^n$, si $X = P^{-1} Y$, entonces $Y = PX$, esto es
            $$
            Y=x_1C_1+\cdots+x_nC_n.
            $$
            Así, $\{C_1,\ldots,C_n\}$ es una base de $\K^n$. \qed
        \end{demostracion}
    \end{frame}  
    
   
    \begin{frame}
     
        \begin{block}{Ejemplo: polinomios de grado $\le n-1$}
    Sea $\K_n[x]$  el conjunto de polinomios de grado menor  que $n$ con coeficientes en $\K$:
    $$
    \K_n[x] = \left\{a_0 + a_1 x + a_2x^2+\cdots+a_{n-1}x^{n-1}: a_0,\ldots,a_{n-1} \in \K  \right\}.
    $$
    Entonces $1,x,x^2,\ldots,x^{n-1}$  es una base de $\K_n[x]$.
    
    \vskip .4cm 

    Es claro que los $1,x,x^2,\ldots,x^{n-1}$ generan $\K_n[x]$. 
    
    \vskip .4cm 

    Por otro lado, si  $\lambda_0 + \lambda_1 x + \lambda_2x^2+\cdots+\lambda_{n-1}x^{n-1} =0$, tenemos que $\lambda_0=\lambda_1 = \lambda_2 =\cdots =\lambda_{n-1} =0$.
\end{block}
    \end{frame}  
    

    \begin{frame}
        
        \begin{block}{Ejemplo: polinomios (base infinita)}
            Sea $\K[x]$  el conjunto de polinomios  con coeficientes en $\K$:
            $$
            \K[x] = \left\{a_0 + a_1 x + a_2x^2+\cdots+a_{n}x^{n}: n \in \N, a_0,\ldots,a_{n} \in \K  \right\}.
            $$
            Entonces $\mathcal B = \{1,x,x^2,\ldots,x^i, \ldots\} = \{x^i: i \in \N_0\}$  es una base de $\K[x]$.
            
            \vskip .4cm 
        
            Es claro que los $x^i$ generan $\K[x]$. 
            
            \vskip .4cm 
        
            Por otro lado, supongamos $\mathcal B$ se LD,  luego existe un subconjunto  \textit{finito} $S$  de $\mathcal B$ con el cual puedo hacer una c.l.  no trivial que de $0$. 
            
            \vskip .4cm 

            Sea $n$ tal que $S \subset   \{1,x,x^2,\ldots,x^n\}$. Entonces existen $\lambda_i$ no todos nulos tal que 
             $\lambda_0 + \lambda_1 x + \lambda_2x^2+\cdots+\lambda_{n-1}x^{n} =0$. Absurdo.

             \vskip .4cm 

             Por lo tanto $\mathcal B$ es base.
        \end{block}
    \end{frame}

   
    \begin{frame}
        \begin{block}{Ejemplo: base canónica de $M_{m \times n}(\K)$}
            Sean $1 \le i \le m$, $1\le j \le n$ y $E_{ij} \in M_{m \times n}(\K)$ definida por
            \begin{equation*}
                [E_{ij}]_{kl} = \left\{ 
                \begin{array}{lll}
                1& &\text{si $i=k$ y $j=l$,} \\
                0& &\text{otro caso}. 
                \end{array}
                 \right.
            \end{equation*}
            Es decir $E_{ij}$  es la matriz cuyas entradas son todas iguales a 0,  excepto la entrada $ij$ que vale 1. En el caso $2 \times 2$  tenemos la matrices
            \begin{equation*}
                E_{11} = \begin{bmatrix} 1&0\\0&0\end{bmatrix}, \quad
                E_{12} = \begin{bmatrix} 0&1\\0&0\end{bmatrix}, \quad
                E_{21} = \begin{bmatrix} 0&0\\1&0\end{bmatrix}, \quad
                E_{22} = \begin{bmatrix} 0&0\\0&1\end{bmatrix}.
            \end{equation*}
            Volviendo al caso general,  
            $$
            \mathcal B = \{ E_{ij}: 1 \le i \le m, 1\le j \le n\}
            $$
            (son $mn$ vectores) es una base de  $M_{m \times n}(\K)$ y se la denomina la \textit{base canónica} de  $M_{m \times n}(\K)$.
            
            \vskip.4cm  
            La demostración es análoga al caso $\K^n$.

        \end{block}
    \end{frame}  
    
   
    \begin{frame}
     

Si $S$ es un conjunto finito denotemos $|S|$  al \textit{cardinal} de  $S$ es decir, la cantidad de elementos de $S$. 

\vskip .4cm

\begin{block}{Preguntas}
    \vskip .4cm
    \begin{itemize}
        \item Dado $V$ espacio vectorial ¿existe una base de  $V$?

        {\color{blue} Respuesta:} sí. La respuesta la da la teoría de conjuntos (Lema de Zorn). 
        \vskip .4cm
        \item Sea $V$ espacio vectorial y $\mathcal B$, $ \mathcal B'$ bases finitas de $V$ ¿Es  $|\mathcal B| =  |\mathcal B'|$?

        {\color{blue} Respuesta:} sí. Es lo que veremos más adelante. 
    \end{itemize}
\end{block}

\end{frame}  
    
\begin{frame}{¿Todo espacio vectorial  tiene una base ``explícita''?}
    \begin{itemize}
        \item Vimos en los ejemplos de las páginas anteriores bases de distintos espacios vectoriales.
        \vskip.4cm  
        \item Vimos  que hay bases finitas y bases infinitas, pero todas la bases que consideramos eran explícitas. 
        \vskip.4cm  
        \item Por  el Lema de Zorn existe una base $\mathcal B$  de $F(\R) = \{f: \R \to \R\}$.
        \vskip.4cm  
        \item ¿Se puede dar en forma relativamente explícita una base de $F(\R)$?
        \vskip.4cm  
        \item Respuesta: NO. 
    \end{itemize}
\end{frame}

   
    \begin{frame}




        \begin{teorema}\label{indep-menorigual-gen}
            Sea $V$ un espacio vectorial generado por un conjunto finito de vectores $w_1,\ldots,w_m$. Entonces 
            $$
            S \subset V \text{ es LI } \Rightarrow |S| \le m.
            $$
        \end{teorema}
        \begin{proof} 
            Para demostrar este teorema es suficiente probar el contrarrecíproco del enunciado, es decir:
            $$
            \text{si }|S| > m \Rightarrow S \text{ es LD},
            $$
            Sea  $S = \{v_1,\ldots,v_n\}$ con $n >m$.  
            
            \vskip .2cm 
            Como  $w_1,\ldots,w_m$ generan $V$, existen escalares $a_{ij}$ en $\K$ tales que
            \begin{equation*}
                v_j = \sum_{i=1}^{m}a_{ij}w_i, \qquad (1 \le j \le n).
            \end{equation*}
        \end{proof}
          
        
            \end{frame}  
            
           
            \begin{frame}
                Probaremos ahora que existen $x_1,\ldots,x_n \in \K$ no todos nulos, tal que $x_1v_1 + \cdots+x_nv_n =0$ ($S$ es LD). 
                \vskip .2cm
                Ahora bien, para cualesquiera $x_1,\ldots,x_n \in \K$ tenemos
                \begin{align*}
                    x_1v_1 + \cdots+x_nv_n &= \sum_{j=1}^{n} x_jv_j& \\
                    & = \sum_{j=1}^{n}x_j \sum_{i=1}^{m}a_{ij}w_i& \\
                    & = \sum_{j=1}^{n} \sum_{i=1}^{m}(x_ja_{ij})w_i& \\ 
                    & = \sum_{i=1}^{m}(\sum_{j=1}^{n} x_ja_{ij})w_i.& 
                \end{align*}
               
            \end{frame}  
            
           
            \begin{frame}

                Es decir,  para cualesquiera $x_1,\ldots,x_n \in \K$ tenemos
                \begin{align*}
                    x_1v_1 + \cdots+x_nv_n &= \sum_{i=1}^{m}(\underbrace{\sum_{j=1}^{n} x_ja_{ij}}_{c_i})w_i.\tag{*}
                \end{align*}

                Si cada coeficiente $c_i$ es nulo $\Rightarrow$ $  x_1v_1 + \cdots+x_nv_n=0$. 
                
                \vskip .4cm 


                Vamos a ver ahora que $\exists$  $x_1,\ldots,x_n$ no todos nulos tal $c_i =0, \forall i$. 
                
                \vskip .4cm 
                
                
                Esto se debe a que el sistema de ecuaciones 
                \begin{equation*}
                    \sum_{j=1}^{n} x_ja_{ij} = 0, \qquad (1 \le i \le m) 
                \end{equation*}
                tiene $m$ ecuaciones  y $n$ incógnitas con  $n > m$ $\Rightarrow$ existen soluciones no triviales (quedan variable libres). 
                
                \vskip .4cm
                
              
            \end{frame}  

            \begin{frame}
                Es decir, existen escalares $x_1,\ldots,x_n \in \K$ no todos nulos, tal que $$c_i = \sum_{j=1}^{n} x_ja_{ij} = 0,\quad  (1 \le i \le m)$$ y, por $(*)$, tenemos que  
                $$x_1v_1 + \cdots+x_nv_n =0.$$ 
                
                Esto quiere decir que los $v_1,\ldots,v_n$ son LD.
                \qed

                \vskip 3cm
            \end{frame}
            
           
           
  


\end{document}