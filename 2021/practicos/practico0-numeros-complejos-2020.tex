\documentclass[12pt]{amsart}
\usepackage{amssymb}
\usepackage{enumerate}
\usepackage{amsmath}
\usepackage{geometry}
\geometry{ a4paper, total={210mm,297mm}, left=2cm, right=2cm, top=1.5cm, bottom=2.5cm, }
\usepackage{graphicx}
\usepackage{fancyhdr}
\usepackage{multicol}
\usepackage{enumitem}

\pagestyle{fancy}



\begin{document}

%\title{Pr\'actico 1}

\noindent {\tiny \'Algebra / \'Algebra II \hfill Segundo Cuatrimestre 2020}

%\maketitle

\centerline{\Large{Pr\' actico 0}}

\

\centerline{\textsc{N\'{u}meros complejos}}


\bigbreak

\subsection*{Objetivos}

\begin{itemize}
 \item Familiarizarse con los n\'umeros complejos.
 \item Aprender a operar con n\'umeros complejos (sumar, multiplicar, c\'alcular inversos, conjugados y normas).
\end{itemize}

\subsection*{Ejercicios}


\begin{enumerate}

\item Expresar los siguientes n{\'u}meros complejos en la forma $a +i b$.
Hallar el m{\'o}dulo y conjugado de cada uno de ellos, y graficarlos.

\begin{multicols}{3}
\begin{enumerate}
\item $(-1+i) (3-2i)$
\item $i^{131} - i^9 +1$
\item $\frac {1+i}{1+2i} + \frac{1-i}{1-2i}$
\end{enumerate}
\end{multicols}

\


\item Encontrar n\'umeros reales $x$ e $y$ tales que $3x+2yi-xi+5y = 7 + 5i$


%\
%
%\item Sean $a,b\in\mathbb{C}$. Decidir si existe $z \in \mathbb{C}$ tal que:
%\begin{enumerate}
%  \item $a \, \operatorname{Im}(z)=2$.  ?`Es \'unico?
%  \item $z^2=b$. ?`Es \'unico? ?`Para qu\'e valores de $b$ resulta $z$ ser un n\'umero real?
%  \item $z$ es imaginario puro y $z^2=4$.
%  \item $z$ es imaginario puro y $z^2=-4$.
%\end{enumerate}

\

\item Probar que si $z \in \mathbb{C}$ tiene m\'odulo $1$ entonces $z + z^{-1} \in \mathbb{R}.$

\

\item Probar que si $a\in \mathbb{R}\backslash \{0\}$ entonces el polinomio $x^2+a^2$ tiene siempre dos ra\'ices complejas distintas.


\end{enumerate}

%============================================================
\subsection*{Ejercicios de repaso} Si ya hizo los ejercicios anteriores continue a la siguiente gu\'ia. Los ejercicios que siguen son similares a los anteriores y le pueden servir para practicar antes de los ex\'amenes.
%============================================================

\

\begin{enumerate}[resume]

\item Expresar los siguientes n{\'u}meros complejos en la forma $a +i b$. Hallar el m{\'o}dulo y conjugado de cada uno de ellos, y graficarlos.

$$\textrm{(a)}\; (\cos\theta - i\sin\theta)^{-1},\; 0\leq\theta<2\pi, \quad \qquad
\textrm{(b)} \; 3 i(1 + i)^4, \quad \qquad
\textrm{(c)} \; \dfrac{1+i}{1-i}$$


\

  \item Sea $z=2+\frac{1}{2}i$, calcular
 \begin{multicols}{3}
\begin{enumerate}
\item $\dfrac{(z+i)(z-i)}{z^2+1}$.
\item $z-2 + \dfrac{1}{z-2}$.
\item $\left|\dfrac{1}{z-i}\right|^2$.
\end{enumerate}
\end{multicols}

\

\item Sea $z\in\mathbb{C}$. Calcular $\frac{1}{z}+\frac{1}{\overline{z}} - \frac{1}{|z|^2}$.

%\
%
%\item Mostrar que las soluciones de la ecuaci\'on $z^4 + (-4+2i)z^2 - 1 =0$ son exactamente
%$-1+\sqrt {1-i}$, $-1-\sqrt {1-i}$, $1+\sqrt {1-i}$ y $1-\sqrt {1-i}$.

\

\item\label{desT} (Desigualdad triangular) Sean $w$ y $z$ n\'umeros complejos. Probar que
\begin{eqnarray*}
  |w + z| \leq |w| + |z|,
\end{eqnarray*}
y la igualdad se cumple si y s\'olo si $w= r\cdot z$ para alg\'un n\'umero real $r \geq 0$.
En general, sean $z_1,z_2,\ldots, z_n$ n\'umeros complejos. Probar que
\begin{eqnarray*}
  \left|\sum_{k=1}^{n} z_k\right| \leq \sum_{k=1}^{n} |z_k|.
\end{eqnarray*}

\

\item Sean $w$ y $z$ n\'umeros complejos. Entonces
\begin{eqnarray*}
  ||w|-|z|| \leq |w-z|.
\end{eqnarray*}

\end{enumerate}


\end{document}
