
\begin{chapter}{N\'umeros complejos}\label{chap-num-compl}
    
    \begin{section}{Cuerpos}\label{seccion-cuerpos}
        
        En el cuatrimestre pasado se ha visto el concepto de cuerpo, del cual haremos un repaso (ver también  \href{https://es.wikipedia.org/wiki/Cuerpo\_(matemáticas)}{https://es.wikipedia.org/wiki/Cuerpo\_(matemáticas)}).
        
        
        \begin{definicion}
            Un  conjunto $\K$ es un \textit{cuerpo}\index{cuerpo} si es un anillo de división conmutativo, es decir, un anillo conmutativo con unidad en el que todo elemento distinto de cero es invertible respecto del producto. Por tanto,  un cuerpo es un conjunto $\K$ en el que se han definido dos operaciones, '$+$' y '$\cdot$', llamadas \textit{adición} y \textit{multiplicación} respectivamente, que cumplen las siguientes propiedades. En la siguiente lista de axiomas $a$, $b$, $c$ denotan elementos arbitrarios de $\K$, y $0$ y $1$ denotan elementos especiales de $\K$ que cumplen las propiedades especificadas más abajo.
            \begin{enumerate}
                \item[\textbf{I1.}] $a+b$ y $a\cdot b$ pertenecen a ${\K}$.
                \item[\textbf{I2.}] {\em Conmutatividad.}\, $a+b = b+a$; $ab=ba$. 
                \item[\textbf{I3.}] {\em Asociatividad.}\, $(a+b)+c = a+(b+c)$;\; $(a\cdot b)\cdot c = a\cdot (b\cdot c)$. 
                \item[\textbf{I4.}] {\em Existencia de elemento neutro.}\, Existen números $0$, $1 \in \K$ con $0\not=1$ tal que $a+0=a$; $a\cdot 1=a$. 
                \item[\textbf{I5.}] {\em Distributividad.}\, $a\cdot (b+c)=a\cdot b+a\cdot c$. 
                \item[\textbf{I6.}] {\em Existencia del inverso aditivo.}\, Por cada $a$ en ${\K}$ existe un único  $-a$ en ${\K}$ tal que $a+(-a)=0$. 
                \item[\textbf{I7.}] {\em Existencia de inverso multiplicativo.}\, Si $a$ es distinto de 0, existe un único elemento $a^{-1} \in \K$  tal que $a\cdot a^{-1}=1$. 
            \end{enumerate}
        \end{definicion}
        
        Muchas veces denotaremos el producto yuxtaponiendo los elementos,  es decir $ab := a\cdot b$, para $a,b \in \K$. Debido a la ley de asociatividad para la suma (axioma \textbf{I3}) $(a+b)+c$ es igual a $a+(b+c)$ y por lo tanto podemos eliminar los paréntesis sin ambigüedad. Es decir, denotamos
        $$
        a+b+c := (a+b)+c = a+(b+c).
        $$
        De forma análoga, usaremos la notación
        $$
        abc = (ab)c = a(bc).
        $$
        Debido a la ley de conmutatividad (axioma \textbf{I2}), es claro que del axioma \textbf{I4} se deduce que  $0+a=a+0=a$ y $1a = a1=a$. Análogamente,  por  \textbf{I2} e  \textbf{I6} obtenemos que  $-a+a =   
        a+(-a)=0$, y por \textbf{I6} que  $a a^{-1} = a^{-1}a=1$.
        
        
        
        Todos los axiomas corresponden a propiedades familiares de los cuerpos que ya conocemos,  como ser el cuerpo de los números reales, denotado $\R$ y el cuerpo de los números racionales (fracciones),  denotado $\Q$. De ellas pueden deducirse la mayoría de las reglas comunes a los cuerpos. Por ejemplo, podemos \textit{definir} la operación de sustracción diciendo que $a-b$ es lo mismo que $a+(-b)$; y deducir las reglas elementales por ejemplo,
        \begin{equation*}
        a-(-b) = a+b, \qquad -(-a) = a.
        \end{equation*}	
        También podemos deducir
        \begin{equation*}
        (ab)^{-1} = a^{-1}b^{-1}
        \end{equation*}
        con tal que $a$ y $b$ sean diferentes de cero. Otras reglas útiles incluyen	
        \begin{equation*}
        -a = (-1)a 
        \end{equation*}					
        y más generalmente
        \begin{equation*}
        - (ab) = (-a) b = a  (-b),
        \end{equation*}	
        y  también 
        \begin{equation*}
        ab = (-a) (-b),
        \end{equation*}
        así como
        \begin{equation*}
        a\cdot 0 = 0,
        \end{equation*}
        todas reglas familiares de la aritmética elemental.
    \end{section}
    
    
    \begin{subsection}{Un cuerpo finito}
        A modo de ejemplo, y para entrenar la intuición de que un cuerpo no necesariamente tiene un número infinito de elementos, consideremos el conjunto con dos elementos $\F_2=\{0,1\}$. Definimos la suma $+\colon\F_2\times \F_2\to \F_2$ mediante la regla
        \begin{align*}
        0+0&=0, & 0+1&=1, & 1+0&=1, & 1+1&=0
        \end{align*}
        y el producto $\cdot \colon\F_2\times \F_2\to \F_2$ como 
        \begin{align*}
        0\cdot 0&=0, & 0\cdot 1&=0, & 1\cdot 0&=0, & 1\cdot 1&=1.
        \end{align*}
        Dejamos como ejercicio para el lector comprobar que estas operaciones así definidas satisfacen los axiomas \textbf{I1.} a \textbf{I7.} y por lo tanto $\F_2$ es un cuerpo, con dos elementos.
        
        \begin{obs*}
            El lector suspicaz reconocerá en estas operaciones a la suma y el producto definidos en el conjunto $\Z_2=\{\overline{0},\overline{1}\}$ de congruencias módulo 2  definido en Álgebra 1/Matemática Discreta I. En efecto, resultados desarrollados en ese curso permiten demostrar que los conjuntos $\Z_p$, con $p$ primo, son ejemplos de cuerpos, en este caso con $p$ elementos.
        \end{obs*}
        
    \end{subsection}
    
    
    \begin{section}{N\'umeros complejos}\label{seccion-numeros-complejos}
        La ecuación polinómica $x^2 + 1 =0$ (¿cuál es el número que elevado  al cuadrado y adicionado 1 da 0?) no tiene solución dentro del cuerpo de los números reales,  pues todos sabemos que  $x^2 \ge 0$ para todo $x \in \R$ y por lo tanto $x^2 + 1 >0$ $\forall\; x \in \R$. Podemos extender $\R$ a otro cuerpo,  de tal forma que \textit{toda} ecuación polinómica con coeficientes en $\R$ tenga solución. 
        
        \begin{definicion} 
            Los \textit{números complejos}\index{números complejos} es el conjunto $\C$  de los pares ordendados $(a,b)$,  denotados $a+ib$, con $a, b$  en $\R$, con las operaciones '$+$' y '$\cdot$', definidas
            \begin{align}
            (a+ib)+ (c+id) &:= (a+c) + i(c+d), \label{sumacompleja} \\
            (a+ib) \cdot (c+id) &:= (ac -bd) + i(ad+bc). \label{productocomplejo}
            \end{align}
            Al número complejo $i = 0 + i\cdot 1$ lo llamamos el \textit{imaginario puro}.  Si $z= a + ib$  es un número complejo,  diremos que $a$ es la \textit{parte real} de $z$ y  la denotamos $a = \operatorname{Re} z$. Por otro lado,  $b$ es la \textit{parte  imaginaria} de $z$ que es denotada $b = \operatorname{Im} z$.
        \end{definicion}
        
        Es claro  que $z=a+ib$ es igual a $w = c+id$ si coinciden su parte real e imaginaria, es decir
        \begin{equation*}
        a+ bi = c+ di\quad \Leftrightarrow\quad a=c \;\wedge\; b = d.
        \end{equation*}
        
        Podemos  ver a $\R$ contenido en $\C$,  con la correspondencia $a \to a + i \cdot 0$ y  observamos que si  nos restringimos a $\R$, tenemos las reglas de adición y  multiplicación usuales.  
        
        La definición de la suma de dos números complejos no debería sorprendernos, pues es la suma ``coordenada a coordenada''. La definición del producto se basa en que deseamos que $i^2 = -1$,  es decir que $i$  sea la solución de la ecuación polinómica $x^2 + 1 =0$,   y que el producto sea distributivo. 	
        
        Primero, comprobemos que  $i^2 = -1$. Esto es debido a que
        \begin{equation*}
        i^2 = (0 + i\cdot 1)(0 + i\cdot 1) = (0\cdot 0 - 1 \cdot 1) + i(0\cdot 1 + 1 \cdot 0) = -1,
        \end{equation*} 
        y por lo tanto $i^2 + 1 = -1+1 = 0$.  
        
        
        Sean $0 = 0 + i\cdot 0, 1 = 1 + i\cdot 0 \in \C$,  es fácil comprobar que son los elementos neutros de la suma y el producto,  respectivamente. Por otro lado, si $z = a + ib$,  entonces $-z = -a -ib$ es el opuesto aditivo de $z$. 
        El inverso multiplicativo es un poco más complicado. Primero observemos que dado $a+ib \in \C$,
        \begin{equation*}
        (a+ ib)(a-ib) = aa -b(-b) = a^2 + b^2 \in \R. 
        \end{equation*}
        Supongamos que $a+ib\ne0$,  encontremos  a partir  de las reglas de adición y multiplicación la inversa de $z$. Sea $c+id$ tal que $(a+ib)(c+id)=1$, luego
        \begin{equation*}
        c + id = \frac{1}{a+ib} = \frac{1}{a+ib}\,\frac{a-ib}{a-ib} = \frac{a-ib}{(a+ib)(a-ib)} = 
        \frac{a-ib}{ a^2 + b^2} = \frac{a}{ a^2 + b^2} - i\frac{b}{ a^2 + b^2}
        \end{equation*}  
        (observar que como $a+ib\ne0$,  entonces $a^2 + b^2 >0$.)
        
        Usando lo anterior,  y un poco más de trabajo, obtenemos
        
        \begin{proposicion}
            Sean $0 = 0 + i\cdot 0, 1 = 1 + i\cdot 0\in \C$. Entonces, $\C$ con las operaciones '$+$' y '$\cdot$', definidas en (\ref{sumacompleja}) y (\ref{productocomplejo}),  respectivamente, es un cuerpo con elementos neutros $0$ y $1$, y
            \begin{align*}
            -(a+ib) &= -a -ib \\
            (a+ib)^{-1} &= 	\frac{a-ib}{ a^2 + b^2}, \qquad \text{para $a+ib \ne 0$}.
            \end{align*}
        \end{proposicion}
        \begin{proof}
            Ejercicio.
        \end{proof}

        Hemos definido los números complejos como pares ordenados y como tales es posible representarlos en el plano $\R \times \R$:
        
        
        
        \begin{figure}[h]
        	\centering
            \begin{tikzpicture}
            \draw[->] (-4.0,0) -- (4.0,0) node[right] {}; % eje x
            \draw[->] (0,-3) -- (0,3) node[above] {}; % eje y
            \draw[fill] (2.5,1.5) circle [radius=0.05];
            \node [right] at (2.5,1.5) {$a+ ib$};
            \node [below] at (2.5,-3pt) {$a$};
            \node [left] at (-3pt,1.5) {$b$};
            \draw (2.5,-3pt) -- (2.5,3pt);
            \draw (-3pt, 1.5) -- (3pt, 1.5);
            \draw [dashed] (0,1.5) -- (2.5,1.5);
            \draw [dashed] (2.5,0) -- (2.5,1.5);
            \end{tikzpicture}
            
            
            \caption{Representación gráfica de los números complejos. }
        \end{figure}
        
        
        \begin{figure}[h]
        	\centering
            \begin{tikzpicture}
            \draw[->] (-4.0,0) -- (4.0,0) node[right] {}; % eje x
            \draw[->] (0,-3) -- (0,3) node[above] {}; % eje y
            \foreach \x in {-4,...,-1}
            \draw (\x,3pt) -- (\x,-3pt)
            node[anchor=north] {\x};
            \foreach \x in {1,...,4}
            \draw (\x,3pt) -- (\x,-3pt)
            node[anchor=north] {\x};
            \foreach \y in {-3,...,-1}
            \draw (3pt,\y) -- (-3pt,\y) 
            node[anchor=east] {\y}; 
            \foreach \y in {1,...,3}
            \draw (3pt,\y) -- (-3pt,\y) 
            node[anchor=east] {\y}; 
            \draw[fill] (2,1) circle [radius=0.05];
            \node [right] at (2,1) {$2+ i$};
            \draw [dashed] (0,1) -- (2,1);
            \draw [dashed] (2,0) -- (2,1);
            \draw[fill] (-1,2.5) circle [radius=0.05];
            \node [left] at (-1,2.5) {$-1+i \,2.5$};
            \draw [dashed] (0,2.5) -- (-1,2.5);
            \draw [dashed] (-1,0) -- (-1,2.5);
            \draw[fill] (-2.5,-2.5) circle [radius=0.05];
            \node [below] at (-2.5,-2.5) {$-2.5-i\,2.5$};
            \draw [dashed] (0,-2.5) -- (-2.5,-2.5);
            \draw [dashed] (-2.5,0) -- (-2.5,-2.5);
            \end{tikzpicture}
            
            
            \caption{Ejemplos de la representación gráfica de los números complejos. }
        \end{figure}
        
        Por el teorema de Pitágoras, la distancia del  número complejo $a+ib$ al $0$ es $\sqrt{a^2+b^2}$.
        
        \begin{definicion} Sea $z = a + ib \in \C$. El \textit{módulo} de $z$ es
            \begin{equation*}
            |z| = \sqrt{a^2+b^2}.
            \end{equation*}
            El  \textit{conjugado} de $z$ es
            \begin{equation*}
            \bar z = a-ib.
            \end{equation*}
        \end{definicion} 
        
        \begin{ejemplo*}
            $|4+3i| = \sqrt{4^2+3^2} = \sqrt{25} =5$, $\overline{4+3i} = 4-3i$.
        \end{ejemplo*}

        \begin{proposicion} Sean $z$ y $w$ números complejos.
            \begin{enumerate}
                \item $z\bar{z} = |z|^2$.
                \item Si $z \ne 0$, $z^{-1} = \displaystyle\frac{\overline{z}}{|z|^2}$.
                \item  $\overline{z+w} = \overline{z} + \overline{w}$.
                \item\label{itm-cmplx}  $\overline{zw} = \overline{z}\;  \overline{w}$.
            \end{enumerate}
            \begin{proof}
                Son comprobaciones rutinarias. Para ejemplificar, hagamos la demostración  de \ref{itm-cmplx}.
                
                Si $z = a + bi$ y $w = c +di$, entonces $(a+bi) (c+di) = (ac -bd) + (ad+bc)i$. Por lo tanto,
                \begin{equation*}
                \overline{zw} = (ac -bd) - (ad+bc)i.
                \end{equation*} 
                Como $\overline{z} = a - bi$ y $\overline{w} = c -di$,
                \begin{equation*}
                \overline{z}\;  \overline{w} = (ac -(-b)(-d)) + (a(-d)+b(-c) )i = (ac -bd) - (ad+bc)i.
                \end{equation*} 
                Por lo tanto $	\overline{zw} = \overline{z}\;  \overline{w}$.
            \end{proof}	
        \end{proposicion}
        
        
        \begin{ejercicio*}
            Determinar el número complejo $2- 3i + \displaystyle\frac{i}{1-i}$.
        \end{ejercicio*}
        \begin{proof}[Solución] 
            El  ejercicio nos pide que escribamos el número en el formato $a + bi$. En general, para eliminar un cociente donde el divisor tiene parte imaginaria no nula, multiplicamos arriba y abajo por el conjugado del divisor, como $z\overline{z} \in \R$, obtenemos un divisor real. En  el ejemplo: 
            \begin{align*}
            2+ 3i + \frac{i}{1-i} &= 2+3i +\frac{i}{1-i}\times\frac{1+i}{1+i}  = 2+3i +\frac{i(1+i)}{(1-i)(1+i)}\\ 
            &=  2+3i +\frac{i-1}{2} = 2+3i +\frac{i}{2}-\frac{1}{2} =\frac{3}{2}+i\frac{7}{2}
            \end{align*}
        \end{proof}
        

        \textbf{Un poco de trigonometría}. Recordemos que dado un punto $p=(x,y)$ en el plano, la recta que une el origen con $p$ determina un ángulo $\theta$ con el eje $x$ y entonces 
        \begin{equation*}
        x = r\sin(\theta) , \qquad y =  r\cos(\theta)  
        \end{equation*}  
        donde $r$ es la longitud del segmento determinado por $(0,0)$ y $(x,y)$. En  el  lenguaje  de los números complejos, si $z = a +bi$ y $\theta$  el ángulo determinado por $z$ y  el eje horizontal, entonces 
        \begin{equation*}
        a = |z|\sin(\theta) , \qquad b =   |z|\cos(\theta), 
        \end{equation*}  
        es decir
        \begin{equation}\label{forma-polar}
        z = |z|(\cos(\theta)+i\sin(\theta)).
        \end{equation}
        Si $z \in \C$, la fórmula (\ref{forma-polar}) e llamada la \textit{forma polar}\index{forma polar} de $z$ y $\theta$ es llamado  el  \textit{argumento de $z$}. 

        \textbf{Notación exponencial.} Otra notación para representar a los números complejos es la \textit{notación exponencial}, \index{notación exponencial}en la cual se denota
        \begin{equation}\label{eq-formula-de-euler}
        e^{i\theta} = \cos(\theta) + i\sin(\theta).
        \end{equation}
        Por lo tanto si $z \in \C$ y $\theta$  es el argumento de  $z$,
        \begin{equation*}
        z = r e^{i\theta} 
        \end{equation*} 
        donde  $r = |z|$. No  perder de vista,  que la notación exponencial no es más que una notación (por ahora). 
        
        \begin{proposicion}
            Sean $z_1 = r_1 e^{i\theta_1}$, $z_2 = r_2 e^{i\theta_2}$,  entonces
            $$
            z_1 z_2 =  r_1r_2 \,e^{i(\theta_1+ \theta_2)}.
            $$
        \end{proposicion}
        \begin{proof}
            $z_1 = r_1(\cos(\theta_1)+i\sin(\theta_1))$, $z_2 = r_2(\cos(\theta_2)+i\sin(\theta_2))$, luego
            \begin{align*}
            z_1z_2 &= r_1r_2(\cos(\theta_1)+i\sin(\theta_1))(\cos(\theta_2)+i\sin(\theta_2)) \\
            &= r_1r_2(\cos(\theta_1)\cos(\theta_2)+i\cos(\theta_1)\sin(\theta_2)+i\sin(\theta_1)\cos(\theta_2)+i^2\sin(\theta_1)\sin(\theta_2)) \\
            &= r_1r_2((\cos(\theta_1)\cos(\theta_2)-\sin(\theta_1)\sin(\theta_2))+i(\sin(\theta_1)\cos(\theta_2) +\cos(\theta_1)\sin(\theta_2))) \\
            &\overset{(*)}= r_1r_2(\cos(\theta_1+\theta_2) + i\sin(\theta_1+\theta_2)) =  r_1r_2e^{i(\theta_1+ \theta_2)}.
            \end{align*}
            La igualdad ($*$) se debe a las tradicionales fórmulas trigonométrica del coseno y  seno de la suma de ángulos.
        \end{proof}
        
        
        \begin{observacion*}[Identidad de Euler] Los alumnos que conozcan las series de Taylor reconocerán inmediatamente la fórmula
            \begin{equation*}
                e^x = \sum_{n=0}^{\infty} \frac{1}{n!}x^n,
            \end{equation*} 
            donde $x$ es un número real. Ahora bien, remplacemos $x$ por $i\theta$ y obtenemos
            \begin{align*}
                e^{i\theta} &= \sum_{n=0}^{\infty} \frac{1}{n!}(i\theta)^n \\
                &=  \sum_{k=0}^{\infty} \frac{1}{(2k)!}(i\theta)^{2k}  + \sum_{k=0}^{\infty} \frac{1}{(2k+1)!}(i\theta)^{2k+1}. \tag{*}
            \end{align*}
        
        No es difícil ver que $i^{2k} = (-1)^k$ y por lo tanto $i^{2k+1} = i^{2k}\cdot i = (-1)^ki $. Por lo tanto, por (*), 
        \begin{align*}
            e^{i\theta} &=  \sum_{k=0}^{\infty} \frac{(-1)^k}{(2k)!}\theta^{2k}  + i\sum_{k=0}^{\infty} \frac{ (-1)^k}{(2k+1)!}\theta^{2k+1} \\
            &=\cos(\theta) + i \sen(\theta), 
        \end{align*}
        recuperando así  la fórmula (\ref{eq-formula-de-euler}), llamada \emph{fórmula de Euler.}\index{fórmula de Euler} Observemos que especializando la fórmula en $\pi$ obtenemos 
        \begin{equation*}
                e^{i\pi} = \cos(\pi) + i\sin(\pi) = -1.
        \end{equation*}
        Escrito de otra forma
        \begin{equation}
            e^{i\pi} -1 =0.
        \end{equation}
        Esta última expresión es denominada la \emph{identidad de Euler}\index{identidad de Euler} y es considerada una de las fórmulas más relevantes de la matemática, pues comprende las cinco constantes matemáticas más importantes. 
        
        Las cinco constantes son:
        \begin{enumerate}
            \item El número $\mathbf 0$.
            \item El numero $\mathbf 1$.
            \item El número $\mathbf \pi$, número irracional  que es la relación entre la circunferencia de un círculo y su diámetro. Es aproximadamente $3.14159\ldots$.
            \item El número $\mathbf e$, también un número irracional. Es la base de los logaritmos naturales y surge naturalmente a través del estudio del interés compuesto y el cálculo. El número $e$ está presente en una gran cantidad de ecuaciones importantes. Es aproximadamente $2.71828\ldots$.
            \item El número $\mathbf i$, el más fundamental de los números imaginarios.
            
        \end{enumerate}
        
        
        \end{observacion*}
        
        
        
    \end{section}	
    \end{chapter}