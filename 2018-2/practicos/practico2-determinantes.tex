\documentclass[11pt,spanish,makeidx]{amsbook}
\tolerance=10000
\renewcommand{\baselinestretch}{1.3}

\usepackage{t1enc}
%usepackage[spanish]{babel}
\usepackage{latexsym}
\usepackage[utf8]{inputenc}
\usepackage{verbatim}
\usepackage{multicol}
\usepackage{amsgen,amsmath,amstext,amsbsy,amsopn,amsfonts,amssymb}
\usepackage{calc}         % From LaTeX distribution
\usepackage{graphicx}     % From LaTeX distribution
\usepackage{ifthen}       % From LaTeX distribution
%\usepackage{subfigure}    % From CTAN/macros/latex/contrib/supported/subfigure
\usepackage{pst-all}      % From PSTricks
\usepackage{pst-poly}     % From pstricks/contrib/pst-poly
\usepackage{pst-tools}
\usepackage{multido}      % From PSTricks
\input{random.tex}        % From CTAN/macros/generic

\tolerance=10000
\renewcommand{\baselinestretch}{1.3}
\oddsidemargin -0.25cm \evensidemargin -.5cm \topmargin 0cm \textheight 23cm
\textwidth 16.5cm

\newcommand\im{\operatorname{Im}}
\renewcommand\nu{\operatorname{Nu}}
\newcommand\Q{\mathbb{Q}}
\newcommand\R{\mathbb{R}}
\newcommand\C{\mathbb{C}}
\newcommand\tr{\operatorname{tr}}



\begin{document}

\begin{center} {\large \bf \'Algebra y \'Algebra II - Segundo Cuatrimestre 2018 \\  Pr\'actico 2 - Determinantes } \end{center}
%============================================================
\vskip .4cm


\begin{enumerate}
	
	\item Calcular el determinante de las siguientes matrices.
	\begin{align*}
	&A=\begin{bmatrix} 4&7\\ 5&3\end{bmatrix},
	&&B=\begin{bmatrix} -3&2&4\\ 1&-1&2\\ -1&4&0\end{bmatrix},
	&&C=\begin{bmatrix} 2&3&1&1\\ 0&2&-1&3 \\ 0&5&1&1 \\1&1&2&5\end{bmatrix}.
	\end{align*}
	
	\ 
	
	\item Determinar para qu\'e valores de $c\in\R$ las siguientes matrices son invertibles.
	\begin{align*}
	&A=\begin{bmatrix} 0&c&-c\\ -1&2&-1\\c&-c&c\end{bmatrix},
	&&B=\begin{bmatrix}4& c&3\\c&2&c\\ 5&c&4 \end{bmatrix},
	&&C=\begin{bmatrix} 1&c&-1\\ c&1&1\\0&1&c\end{bmatrix}.
	\end{align*}
	
	\ 
	
	\item Sean $A$, $B$ y $C$ matrices $n\times n$, tales que $\det A=-1$, $\det B=2$ y $\det C=3$. 
	Calcular $\det(A^2BC^TB^{-1})$ y $\det(B^2C^{-1}AB^{-1}C^T)$.
	
	\ 
	
	\item
	Usar la matriz de cofactores para calcular la inversa de las matrices:
	\[
	A=\begin{bmatrix} -2&3&2\\ 6&0&3\\4&1&-1\end{bmatrix},\qquad
	B=\begin{bmatrix}\cos(t)& 0& \sin(t)\\ 0&1&0 \\ \sin(t)&0&\cos(t) \end{bmatrix}.
	\]


%\
%
%\item
%Usar la regla de Cramer para resolver los siguientes sistemas de
%ecuaciones lineales.
%\begin{align*}
%&\begin{cases}
%x+y+z=11\\
%2x-6y -z= 0\\
%3x+4y+2z=0
%\end{cases}
%&&\begin{cases}
%3x-2y=7\\
%3y-2z=6 \\
%3z-2x=-1.
%\end{cases}
%&&\begin{cases}
%2ix-z=4\\
%x-2y+2z=7+i \\
%3x+2y=1.
%\end{cases}
%\end{align*}

\end{enumerate}
\vskip .3cm

\medskip
\centerline{EJERCICIOS ADICIONALES}
\medskip

\begin{enumerate}


		\item Calcular el determinante de las siguientes matrices.
		\begin{align*}
		&A=\begin{bmatrix} -2&3&2&-6\\ 0&4&4&-5\\ 5&-6&-3&2\\ -3&7&0&0 \end{bmatrix},\quad
		&&B=\begin{bmatrix} 2&0&0&0\\ 0&0&3&0\\ 0&-1&0&0\\ 0&0&0&4\end{bmatrix},\quad
		&&C=\begin{bmatrix}
		1&-1&2&0&0\\ 3&1&4&0&0\\ 2&-1&5&0&0 \\0&0&0&2&1\\ 0&0&0&-1&4
		\end{bmatrix}.
		\end{align*}
	
\ 		
		
		\item Sabiendo que
		$
		\det \begin{bmatrix} a&b&c\\ p&q&r\\
		x&y&z\end{bmatrix}=-1
		$,
		calcular
		$
		\det \begin{bmatrix} -2a&-2b&-2c\\ 2p+x&2q+y&2r+z\\
		3x&3y&3z\end{bmatrix}.
		$
	
\ 	
		
		\item
		Probar que
		$$
		\det\begin{bmatrix}
		1+x_1 & x_2 & x_3 & \cdots & x_n \\
		x_1 & 1+x_2 & x_3 & \cdots & x_n \\
		x_1 & x_2 & 1+x_3 & \cdots & x_n \\
		\vdots & \vdots & \vdots &\ddots& \vdots \\
		x_1 & x_2 & x_3 & \cdots & 1+x_n
		\end{bmatrix}
		= 1+x_1+x_2 + \cdots x_n.
		$$
	
\ 
	
		\item
		Una matriz $A$ compleja $n \times n$ se dice {\it antisim\'etrica}
		si $A^t=-A$.
		\begin{enumerate}
			\item Probar que si $n$ es impar y $A$ es antisim\'etrica, entonces
			$\det(A)=0$.
			\item Para cada $n$ par encontrar una matriz $A$ antisim\'etrica
			$n \times n$ tal que $\det(A)\not=0$.
		\end{enumerate}
		
	
	
\end{enumerate}

\end{document}