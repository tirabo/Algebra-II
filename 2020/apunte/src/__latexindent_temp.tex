 
    \begin{teorema}[Teorema fundamental del álgebra]\label{th-fundamental-del-algebra}
        Todo polinomio no constante con coeficientes complejos tiene al menos una raíz compleja. Es decir si 
        \begin{equation*}
            \text{$p(x) = a_nx^n+ a_{n-1}x^{n-1} +\cdots+a_0$, con $a_i \in \mathbb{C}$,  $a_n\ne 0$ y $n\ge 1$,}
        \end{equation*}
        entonces existe $\alpha \in \mathbb{C}$ tal que $p(\alpha)=0$.
    \end{teorema}

    Pese a llamarse ``Teorema fundamental del álgebra'', este resultado  no suele demostrarse en los cursos de álgebra, pues su demostración requiere del uso de análisis matemático.
    
    Si $\alpha$ es raíz de $p$, un polinomio de grado $n$, por  el teorema del resto,  $p(x) = (x-\alpha)p_1(x)$, con  $p_1$ un polinomio de grado $n-1$.  Aplicando inductivamente este procedimiento, podemos deducir:
    
    \begin{corolario} Si $p$ es un polinomio de de grado $n\ge 1$ con coeficientes en $\C$,  entonces
        \begin{equation*}
            p(x)= c(x-\alpha_1)(x-\alpha_2)\ldots(x-\alpha_n),
        \end{equation*}
        con $c,\alpha_i \in \C$.
    \end{corolario}