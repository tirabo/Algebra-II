%\documentclass{beamer} 
\documentclass[handout]{beamer} % sin pausas
\usetheme{CambridgeUS}

\usepackage{etex}
\usepackage{t1enc}
\usepackage[spanish,es-nodecimaldot]{babel}
\usepackage{latexsym}
\usepackage[utf8]{inputenc}
\usepackage{verbatim}
\usepackage{multicol}
\usepackage{amsgen,amsmath,amstext,amsbsy,amsopn,amsfonts,amssymb}
\usepackage{amsthm}
\usepackage{calc}         % From LaTeX distribution
\usepackage{graphicx}     % From LaTeX distribution
\usepackage{ifthen}
%\usepackage{makeidx}
\input{random.tex}        % From CTAN/macros/generic
\usepackage{subfigure} 
\usepackage{tikz}
\usepackage[customcolors]{hf-tikz}
\usetikzlibrary{arrows}
\usetikzlibrary{matrix}
\tikzset{
	every picture/.append style={
		execute at begin picture={\deactivatequoting},
		execute at end picture={\activatequoting}
	}
}
\usetikzlibrary{decorations.pathreplacing,angles,quotes}
\usetikzlibrary{shapes.geometric}
\usepackage{mathtools}
\usepackage{stackrel}
%\usepackage{enumerate}
\usepackage{enumitem}
\usepackage{tkz-graph}
\usepackage{polynom}
\polyset{%
	style=B,
	delims={(}{)},
	div=:
}
\renewcommand\labelitemi{$\circ$}
\setlist[enumerate]{label={(\arabic*)}}
%\setbeamertemplate{background}[grid][step=8 ] % para hacer cuadriculado
\setbeamertemplate{itemize item}{$\circ$}
\setbeamertemplate{enumerate items}[default]
\definecolor{links}{HTML}{2A1B81}
\hypersetup{colorlinks,linkcolor=,urlcolor=links}


\newcommand{\Id}{\operatorname{Id}}
\newcommand{\img}{\operatorname{Im}}
\newcommand{\nuc}{\operatorname{Nu}}
\newcommand{\im}{\operatorname{Im}}
\renewcommand\nu{\operatorname{Nu}}
\newcommand{\la}{\langle}
\newcommand{\ra}{\rangle}
\renewcommand{\t}{{\operatorname{t}}}
\renewcommand{\sin}{{\,\operatorname{sen}}}
\newcommand{\Q}{\mathbb Q}
\newcommand{\R}{\mathbb R}
\newcommand{\C}{\mathbb C}
\newcommand{\K}{\mathbb K}
\newcommand{\F}{\mathbb F}
\newcommand{\Z}{\mathbb Z}
\newcommand{\N}{\mathbb N}
\newcommand\sgn{\operatorname{sgn}}
\renewcommand{\t}{{\operatorname{t}}}
\renewcommand{\figurename }{Figura}

%
% Ver http://joshua.smcvt.edu/latex2e/_005cnewenvironment-_0026-_005crenewenvironment.html
%

\renewenvironment{block}[1]% environment name
{% begin code
	\par\vskip .2cm%
	{\color{blue}#1}%
	\vskip .2cm
}%
{%
	\vskip .2cm}% end code


\renewenvironment{alertblock}[1]% environment name
{% begin code
	\par\vskip .2cm%
	{\color{red!80!black}#1}%
	\vskip .2cm
}%
{%
	\vskip .2cm}% end code


\renewenvironment{exampleblock}[1]% environment name
{% begin code
	\par\vskip .2cm%
	{\color{blue}#1}%
	\vskip .2cm
}%
{%
	\vskip .2cm}% end code




\newenvironment{exercise}[1]% environment name
{% begin code
	\par\vspace{\baselineskip}\noindent
	\textbf{Ejercicio (#1)}\begin{itshape}%
		\par\vspace{\baselineskip}\noindent\ignorespaces
	}%
	{% end code
	\end{itshape}\ignorespacesafterend
}


\newenvironment{definicion}[1][]% environment name
{% begin code
	\par\vskip .2cm%
	{\color{blue}Definición #1}%
	\vskip .2cm
}%
{%
	\vskip .2cm}% end code

\newenvironment{observacion}[1][]% environment name
{% begin code
	\par\vskip .2cm%
	{\color{blue}Observación #1}%
	\vskip .2cm
}%
{%
	\vskip .2cm}% end code

\newenvironment{ejemplo}[1][]% environment name
{% begin code
	\par\vskip .2cm%
	{\color{blue}Ejemplo #1}%
	\vskip .2cm
}%
{%
	\vskip .2cm}% end code

\newenvironment{ejercicio}[1][]% environment name
{% begin code
	\par\vskip .2cm%
	{\color{blue}Ejercicio #1}%
	\vskip .2cm
}%
{%
	\vskip .2cm}% end code


\renewenvironment{proof}% environment name
{% begin code
	\par\vskip .2cm%
	{\color{blue}Demostración}%
	\vskip .2cm
}%
{%
	\vskip .2cm}% end code



\newenvironment{demostracion}% environment name
{% begin code
	\par\vskip .2cm%
	{\color{blue}Demostración}%
	\vskip .2cm
}%
{%
	\vskip .2cm}% end code

\newenvironment{idea}% environment name
{% begin code
	\par\vskip .2cm%
	{\color{blue}Idea de la demostración}%
	\vskip .2cm
}%
{%
	\vskip .2cm}% end code

\newenvironment{solucion}% environment name
{% begin code
	\par\vskip .2cm%
	{\color{blue}Solución}%
	\vskip .2cm
}%
{%
	\vskip .2cm}% end code



\newenvironment{lema}[1][]% environment name
{% begin code
	\par\vskip .2cm%
	{\color{blue}Lema #1}\begin{itshape}%
		\par\vskip .2cm
	}%
	{% end code
	\end{itshape}\vskip .2cm\ignorespacesafterend
}

\newenvironment{proposicion}[1][]% environment name
{% begin code
	\par\vskip .2cm%
	{\color{blue}Proposición #1}\begin{itshape}%
		\par\vskip .2cm
	}%
	{% end code
	\end{itshape}\vskip .2cm\ignorespacesafterend
}

\newenvironment{teorema}[1][]% environment name
{% begin code
	\par\vskip .2cm%
	{\color{blue}Teorema #1}\begin{itshape}%
		\par\vskip .2cm
	}%
	{% end code
	\end{itshape}\vskip .2cm\ignorespacesafterend
}


\newenvironment{corolario}[1][]% environment name
{% begin code
	\par\vskip .2cm%
	{\color{blue}Corolario #1}\begin{itshape}%
		\par\vskip .2cm
	}%
	{% end code
	\end{itshape}\vskip .2cm\ignorespacesafterend
}

\newenvironment{propiedad}% environment name
{% begin code
	\par\vskip .2cm%
	{\color{blue}Propiedad}\begin{itshape}%
		\par\vskip .2cm
	}%
	{% end code
	\end{itshape}\vskip .2cm\ignorespacesafterend
}

\newenvironment{conclusion}% environment name
{% begin code
	\par\vskip .2cm%
	{\color{blue}Conclusión}\begin{itshape}%
		\par\vskip .2cm
	}%
	{% end code
	\end{itshape}\vskip .2cm\ignorespacesafterend
}







\newenvironment{definicion*}% environment name
{% begin code
	\par\vskip .2cm%
	{\color{blue}Definición}%
	\vskip .2cm
}%
{%
	\vskip .2cm}% end code

\newenvironment{observacion*}% environment name
{% begin code
	\par\vskip .2cm%
	{\color{blue}Observación}%
	\vskip .2cm
}%
{%
	\vskip .2cm}% end code


\newenvironment{obs*}% environment name
	{% begin code
		\par\vskip .2cm%
		{\color{blue}Observación}%
		\vskip .2cm
	}%
	{%
		\vskip .2cm}% end code

\newenvironment{ejemplo*}% environment name
{% begin code
	\par\vskip .2cm%
	{\color{blue}Ejemplo}%
	\vskip .2cm
}%
{%
	\vskip .2cm}% end code

\newenvironment{ejercicio*}% environment name
{% begin code
	\par\vskip .2cm%
	{\color{blue}Ejercicio}%
	\vskip .2cm
}%
{%
	\vskip .2cm}% end code

\newenvironment{propiedad*}% environment name
{% begin code
	\par\vskip .2cm%
	{\color{blue}Propiedad}\begin{itshape}%
		\par\vskip .2cm
	}%
	{% end code
	\end{itshape}\vskip .2cm\ignorespacesafterend
}

\newenvironment{conclusion*}% environment name
{% begin code
	\par\vskip .2cm%
	{\color{blue}Conclusión}\begin{itshape}%
		\par\vskip .2cm
	}%
	{% end code
	\end{itshape}\vskip .2cm\ignorespacesafterend
}







\newcommand{\nc}{\newcommand}


%%%%%%%%%%%%%%%%%%%%%%%%%LETRAS

\nc{\FF}{{\mathbb F}} \nc{\NN}{{\mathbb N}} \nc{\QQ}{{\mathbb Q}}
\nc{\PP}{{\mathbb P}} \nc{\DD}{{\mathbb D}} \nc{\Sn}{{\mathbb S}}
\nc{\uno}{\mathbb{1}} \nc{\BB}{{\mathbb B}} \nc{\An}{{\mathbb A}}

\nc{\ba}{\mathbf{a}} \nc{\bb}{\mathbf{b}} \nc{\bt}{\mathbf{t}}
\nc{\bB}{\mathbf{B}}

\nc{\cP}{\mathcal{P}} \nc{\cU}{\mathcal{U}} \nc{\cX}{\mathcal{X}}
\nc{\cE}{\mathcal{E}} \nc{\cS}{\mathcal{S}} \nc{\cA}{\mathcal{A}}
\nc{\cC}{\mathcal{C}} \nc{\cO}{\mathcal{O}} \nc{\cQ}{\mathcal{Q}}
\nc{\cB}{\mathcal{B}} \nc{\cJ}{\mathcal{J}} \nc{\cI}{\mathcal{I}}
\nc{\cM}{\mathcal{M}} \nc{\cK}{\mathcal{K}}

\nc{\fD}{\mathfrak{D}} \nc{\fI}{\mathfrak{I}} \nc{\fJ}{\mathfrak{J}}
\nc{\fS}{\mathfrak{S}} \nc{\gA}{\mathfrak{A}}
%%%%%%%%%%%%%%%%%%%%%%%%%LETRAS


\title[Clase 24 -  Matriz de una transformación lineal 2]{Álgebra/Álgebra II \\ Clase 24 - Matriz de una transformación lineal 2}

\author[]{}
\institute[]{\normalsize FAMAF / UNC
	\\[\baselineskip] ${}^{}$
	\\[\baselineskip]
} 
\date[24/11/2020]{24 de noviembre de 2020}


\begin{document}

\begin{frame}
\maketitle
\end{frame}




\begin{frame}

\begin{block}{Proposición 3.5.1}
Sean $V$ y $W$ espacios vectoriales de dimensión finita con bases ordenadas $\cB=\{v_1, ..., v_n\}$ y $\cB'=\{w_1, ..., w_m\}$, respectivamente. 

Si $T:V\longrightarrow W$ es una transformación lineal, entonces
\begin{align*}
[T(v)]_{\cB'}=[T]_{\cB\cB'}[v]_{\cB}
\end{align*}
para todo $v\in V$.
\end{block}
\pause
En palabras, el vector de coordenadas de $T(v)$ en la base $\cB'$ es igual a multiplicar la matriz de $T$ en las bases $\cB$ y $\cB'$ por el vector de coordenadas de $v$ en la base $\cB$.

\



\end{frame}

\begin{frame}
La demostración se puede ver en el apunte de clase.
\pause
\

La demostración sigue los mismos pasos del siguiente ejemplo pero escribiendo todo en forma más abstracta: con letras y subíndices en lugar de números concretos.
\end{frame}



\begin{frame}
	\begin{ejemplo}
		Sean las bases ordenadas de $\R^3$
		$$\cB = \{(1,0,-1),(0,1,-1),(1,1,0)\}\quad \textbf{ y } \quad\cB' = \{e_1,e_2,e_3\}.$$ 
		Sea $T: \R^3 \to \R^3$ definida
		$$
		T(x,y,z) = (x-2y+z, x+y, -x-y+z),
		$$ 
		y sea $v=(-4,4,6)$. Veamos  que 
	\begin{align*}
		[T(v)]_{\cB'}=[T]_{\cB\cB'}[v]_{\cB}
		\end{align*}
	\end{ejemplo}\pause
	\begin{solucion}\pause
		Primero debemos calcular (1) $[T(v)]_{\cB'}$,\; (2) $[v]_{\cB}$\; y \;(3) $[T]_{\cB\cB'}$.
	\end{solucion}	
\end{frame}


\begin{frame}
	Aplicando la definición de $T$, calculamos
	$$
	T(-4,4,6) = (-6,0,6) = -6(1,0,0) +0(0,1,0) + 6(0,0,1).
	$$
	Es decir 
	\begin{equation*}
		[T(v)]_{\cB'} = \begin{bmatrix}
			-6 \\0\\6
		\end{bmatrix}.\tag{1}
	\end{equation*}
	\pause
	Por otro  lado
	\begin{align*}
		(-4,4,6) = -7(1,0,-1) +1(0,1,-1)+3(1,1,0),
	\end{align*}
	Luego
	\begin{equation*}
		[v]_{\cB} = \begin{bmatrix}
			-7 \\1\\3
		\end{bmatrix}.\tag{2}
	\end{equation*}
\end{frame}


\begin{frame}
	
	\begin{align*}
	\text{Finalmente, \quad}	T(1,0,-1) &= (0,1,-2)\\ 
		T(0,1,-1) &= (-3,1,-2)\\
		T(1,1,0) &= (-1,2,-2)\\
	\end{align*}
	\vskip -.8cm\pause Luego
	\begin{equation*}
		[T]_{\cB \cB' } = \begin{bmatrix}
			0&-3&1 \\1&1&2\\-2&-2&-2
		\end{bmatrix}.\tag{3}
	\end{equation*}
	\pause
	$\Rightarrow$
	\pause
	\begin{align*}
		[T]_{\cB\cB'}[v]_{\cB} &= 
		\begin{bmatrix} 0&-3&1 \\1&1&2\\-2&-2&-2\end{bmatrix} \begin{bmatrix}-7 \\1\\3\end{bmatrix}
		=   \begin{bmatrix}	-6 \\0\\6\end{bmatrix}\\
		&= [T(v)]_{\cB'}.
		\end{align*}\qed
\end{frame}



\begin{frame}
¿Por  qué se  cumple esta ``mágica'' igualdad? 
\vskip .4cm\pause
Escribiendo $v$ como combinación lineal de la base $\cB'$ y luego  calculando $T(v)$ se devela  el ``misterio''.\pause En  nuestro caso 
$$
v = (-4,4,6) = -7(1,0,-1)+ 1(0,1,-1)+3(1,1,0),
$$
Luego,
	\begin{align*}
		T(-4,4,6) &= T(-7(1,0,-1)+ 1(0,1,-1)+3(1,1,0)) \\ \noalign{\vskip.2cm}
		&= -7T(1,0,-1)+ 1T(0,1,-1)+3T(1,1,0) \\\noalign{\vskip.2cm}
		&= -7(0,1,2) +1(-3,1,-2) +3(-1,2,-2) \\\noalign{\vskip.2cm}
		&=
		\begin{bmatrix*}[l]
			-7\cdot 0 &+& 1\cdot (-3) &+&3 \cdot (-1) \\
			-7\cdot 1 &+& 1\cdot 1 &+&3 \cdot 2\\ 
			-7\cdot 2 &+& 1\cdot (-2) &+&3 \cdot (-2)
		\end{bmatrix*} =  [T(v)]_{\cB'}
	\end{align*}	
	\vskip .4cm 
	


\end{frame}

\begin{frame}
	Repasando \pause
	$$
	[T]_{\cB \cB'} = \begin{bmatrix} 0&-3&1 \\1&1&2\\-2&-2&-2\end{bmatrix},
	$$\pause
	$$
	[v]_{\cB} = \begin{bmatrix}-7 \\1\\3\end{bmatrix}
	$$\pause
	$$
	[T(v)]_{\cB'} =\begin{bmatrix*}[l]
		-7\cdot 0 &+& 1\cdot (-3) &+&3 \cdot (-1) \\
		-7\cdot 1 &+& 1\cdot 1 &+&3 \cdot 2\\ 
		-7\cdot 2 &+& 1\cdot (-2) &+&3 \cdot (-2)
	\end{bmatrix*}
	$$\pause
	Es decir 
	\begin{equation*}
		\underbrace{\begin{bmatrix} 0&-3&1 \\1&1&2\\-2&-2&-2\end{bmatrix}}_{[T]_{\cB \cB' }} 
		\underbrace{\begin{bmatrix}-7 \\1\\3\end{bmatrix}}_{[v]_{\cB}}= 
		\underbrace{
			\begin{bmatrix*}[l]
				-7\cdot 0 &+& 1\cdot (-3) &+&3 \cdot (-1) \\
				-7\cdot 1 &+& 1\cdot 1 &+&3 \cdot 2\\ 
				-7\cdot 2 &+& 1\cdot (-2) &+&3 \cdot (-2)
			\end{bmatrix*}}_{ [T(v)]_{\cB'}}
	\end{equation*}
\end{frame}



\begin{frame}
	\begin{observacion}
		Con sólo conocer cuanto vale la transformación en una base conocemos cuanto vale en todo el espacio.
\vskip .4cm
En efecto, a la matriz de la transformación la armamos calculando la transformación en los vectores de una base. Y la proposición anterior nos dice que para calcular la transformación en un vector cualquier debemos multiplicar por esa matriz.
	\end{observacion}
	\pause
También vale  lo siguiente.
\pause
\begin{teorema}\label{th-tl-definida-en-base}
	Sean $V$ un espacio vectorial de dimensión finita sobre el cuerpo $\K$ y $\{v_1,\ldots,v_n\}$  una base ordenada de $V$. Sean $W$ un espacio vectorial sobre el mismo cuerpo y $\{w_1,\ldots,w_n\}$, vectores cualesquiera de $W$. Entonces existe una única transformación  lineal $T$ de $V$ en $W$ tal que
	\begin{equation*}
	T(v_j) = w_j, \quad j=1,\ldots,n.
	\end{equation*}
\end{teorema}

\end{frame}

\begin{frame}
	\begin{corolario}\label{cor-cambio-de-base}
		Sea $V$ un espacio vectorial de dimensión finita sobre el cuerpo $\K$, sean $\mathcal B$, $\mathcal B'$  bases ordenadas de $V$. Entonces 
		\begin{equation*}
			[v]_{\mathcal B} = [\Id]_{\mathcal B' \mathcal B}\, [v]_{\mathcal B'}, \quad \forall v \in V.
		\end{equation*}
	\end{corolario}\pause
	\begin{proof}\pause
		Por la proposición 3.5.1 tenemos que 
		$$
		[\Id]_{\mathcal B' \mathcal B}  [v]_{\mathcal B'} = [\Id (v)]_{\mathcal B} = [v]_{\mathcal B}.
		$$
	\end{proof}
	\pause
	\begin{definicion}
		Sea $V$ un espacio vectorial de dimensión finita sobre el cuerpo $\K$ y sean $\mathcal B$ y $\mathcal B'$ bases ordenadas de $V$. La matriz $P =[\Id]_{\mathcal B' \mathcal B}$  es llamada la \textit{matriz de cambio de base} \index{matriz!de cambio de base} de la base $\mathcal B'$  a la base $\mathcal B$. 
	\end{definicion}

	

\end{frame}


\begin{frame}

\begin{teorema} [4.5.3]
Sean $V$, $W$ y $Z$ espacios vectoriales de dimensión finita con bases $\cB$, $\cB'$ y $\cB''$, respectivamente.

\vskip .2cm

Sean $T:V\longrightarrow W$ y $U:W\longrightarrow Z$ transformaciones lineales.

\vskip .2cm

Entonces la matriz de la tranfomación lineal $$UT:V\longrightarrow Z,$$ es decir la composición de $T$ con $U$, satisface
\begin{align*}
[UT]_{\cB\cB''}=[U]_{\cB'\cB''}[T]_{\cB\cB'}
\end{align*}
\end{teorema}
(multiplicación de matrices)
\end{frame}


\begin{frame}
	\begin{corolario}\label{cor-inversa-matriz-cambio-de-base} Sea $V$ un espacio vectorial de dimensión finita sobre el cuerpo $\K$ y sean $\mathcal B$ y $\mathcal B'$ bases ordenadas de $V$. La matriz de cambio de base  $P =[\Id]_{\mathcal B' \mathcal B}$ es invertible y su  inversa es $P^{-1} =[\Id]_{\mathcal B \mathcal B'}$
	\end{corolario}\pause
	\begin{proof}\pause
		\begin{equation*}
			[\Id]_{\mathcal B \mathcal B'} P =[\Id]_{\mathcal B \mathcal B'}[\Id]_{\mathcal B' \mathcal B} = [\Id]_{\mathcal B'} = \Id.
		\end{equation*}

		Luego $=[\Id]_{\mathcal B \mathcal B'} = P^{-1}$.
	\end{proof}\qed

\end{frame}


\begin{frame}
\begin{block}{Notación}
Si $T:V\longrightarrow V$ es Una transformación lineal que va de un espacio en si mismo, diremos que $T$ es un \textit{operador lineal en $V$}.
\pause
\vskip .4cm

Si $\cB$ es una base de $V$, \textit{$[T]_{\cB}$} denota la matriz de $T$ en la base $\cB$ y $\cB$, o sea si la base de salida y llegada es la misma, entonces usamos un sólo subíndice.
\end{block}
\pause
\vskip .4cm
\begin{corolario}[4.5.3]
Sea $V$ un espacio vectorial de dimensión finita con base $\cB$ y $U,T:V\longrightarrow V$ dos transformaciones lineales. Entonces
\begin{enumerate}
 \item $[UT]_{\cB}=[U]_{\cB}[T]_{\cB}$
 \item $T$ es un isomorfismo si y sólo si $[T]_\cB$ es una matriz invertible. En tal caso 
 $$
 [T^{-1}]_{\cB}=[T]_{\cB}^{-1}
 $$
\end{enumerate}
\end{corolario}

\end{frame}


\begin{frame}

Los espacios vectoriales no tienen una base ``natural'' es decir una que es más importante que otras. Cuando trabajamos con bases estamos haciendo una elección y hay infinitas elecciones posible. \pause

\vskip .4cm

El siguiente teorema nos dice como se relacionan las matrices de una transformación lineal respecto a distintas bases.\pause

\vskip .4cm

      
\begin{teorema}\label{th-cambio-de-base}
	Sea $V$ un espacio vectorial de dimensión finita sobre el cuerpo $\K$ y sean
	$$
	\mathcal B = \{v_1,\ldots,v_n \}, \qquad \mathcal B' = \{w_1,\ldots,w_n \}
	$$
	bases ordenadas de $V$. Sea $T$ es un operador lineal sobre V. Entonces, si $P$ es la matriz de cambio de base de $\mathcal B'$ a $\mathcal B$, se cumple que   
	\begin{equation*}
	[T]_{\mathcal B'} = P^{-1}[T]_{\mathcal B}  P.
	\end{equation*}\pause
	Es decir
	\begin{equation*}\label{eq-cambio de base}
		[T]_{\mathcal B'} = [\Id]_{\mathcal B \mathcal B'} [T]_{\mathcal B} [\Id]_{\mathcal B' \mathcal B}.
	\end{equation*}
\end{teorema}


\end{frame}

\begin{frame}

	\begin{proof} \pause Tenemos que $T = \Id T$ y $T =T  \Id$, luego 
		\begin{align*}
			[T]_{\mathcal B'\mathcal B'} &=  [\Id T]_{\mathcal B'\mathcal B'}& \qquad& \\\noalign{\vskip .2cm}
			&= [\Id]_{\mathcal B \mathcal B'} [T]_{\mathcal B' \mathcal B}& &(\text{teorema 4.5.3})  \\\noalign{\vskip .2cm}
			&= [\Id]_{\mathcal B \mathcal B'} [T \Id]_{\mathcal B' \mathcal B}& &\\\noalign{\vskip .2cm}
			&= [\Id]_{\mathcal B \mathcal B'} [T]_{\mathcal B\mathcal B} [\Id]_{\mathcal B' \mathcal B}& &(\text{teorema 4.5.3}).
		\end{align*} 
		
		Por lo tanto  $ [T]_{\mathcal B'} = [\Id]_{\mathcal B \mathcal B'} [T]_{\mathcal B} [\Id]_{\mathcal B' \mathcal B} = P^{-1} [T]_{\mathcal B\mathcal B}P$.
		\vskip .2cm
		\qed
	\end{proof}

\end{frame}

\begin{frame}
	Las fórmulas
	\begin{align}
		&[T]_{\mathcal B'} = [\Id]_{\mathcal B \mathcal B'} [T]_{\mathcal B} [\Id]_{\mathcal B' \mathcal B} \tag{*}\\
		&[\Id]_{\mathcal B \mathcal B'} [\Id]_{\mathcal B' \mathcal B}= \Id \tag{**} \\
		&[v]_{\mathcal B} = [\Id]_{\mathcal B' \mathcal B} [v]_{\mathcal B'} \tag{***}
	\end{align}
	son importantes por si mismas y debemos recordarlas.\pause

	\vskip .4cm

	Como ya dijimos, la matriz $P =[\Id]_{\mathcal B' \mathcal B} $  es llamada la {matriz de cambio de base} de la base $\mathcal B'$  a la base $\mathcal B$. \pause
	\vskip .4cm
	La matriz de cambio de base nos permite calcular los cambios de  coordenadas de los vectores y los cambio de base de las transformaciones lineales.
	\vskip .4cm


\end{frame}



\begin{frame}

\begin{observacion} Sea $T: \K^n \to \K^n$ operador lineal, $\cB = \{v_1,\ldots,v_n\}$ base ordenada y $\cC$ la base canónica,  entonces 
$$
[T]_{\cB\cC} = \begin{bmatrix}
	Tv_1 & Tv_2 & \cdots &Tv_n
\end{bmatrix}.
$$
\end{observacion}\pause
	

\begin{observacion}
Pudimos probar el teorema de cambio de base usando adecuadamente el teorema 4.5.3, es decir la fórmula
\begin{align*}
	[UT]_{\cB\cB''}=[U]_{\cB'\cB''}[T]_{\cB\cB'}
\end{align*}\pause
Con igual argumento podemos deducir otras igualdades que son útiles para armar todas las matrices a partir de matrices asociadas a bases canónicas, que, como  dijimos en la observación anterior, es fácil calcularlas. 
\end{observacion}
\end{frame}


\begin{frame}

\begin{observacion}
Sea $T:\R^n\longrightarrow\R^n$ una transformación lineal.

\vskip .4cm

Sean $\cB$ y $\cB'$ bases de $\R^n$.

\vskip .4cm

Entonces 
\begin{align*}
[T]_{\cB'\cB}=[\Id]_{\cC\cB}\,[T]_{\cC\cC}\,[\Id]_{\cB'\cC}=[\Id]_{\cB\cC}^{-1}\,[T]_{\cC\cC}\,[\Id]_{\cB'\cC}
\end{align*}\pause

\vskip .4cm

En palabras: para ir de $\cB'$ a $\cB$ con $T$, primero vamos de $\cB'$ a $\cC$, despues de $\cC$ a $\cC$ con $T$ y finalmente vamos de $\cC$ a $\cB$.\pause

\vskip .4cm

Las matrices $[\Id]_{\cB\cC}$ y $[\Id]_{\cB'\cC}$ son fáciles de calcular, ubicamos los vectores de $\cB$ y $\cB'$ como columnas. Similarmente, la matriz de $T$ en la base canónica también es fácil de calcular.
\end{observacion}
\end{frame}

\begin{frame}

\begin{observacion}
Sean $\cB$ y $\cB'$ dos bases de $\R^n$.

Entonces la matriz de cambio de base de $\cB'$ a $\cB$ es 
\begin{align*}
[\Id]_{\cB'\cB}=[\Id]_{\cC\cB}\,[\Id]_{\cB'\cC}=[\Id]_{\cB\cC}^{-1}\,[\Id]_{\cB'\cC}
\end{align*}\pause

\

En palabras, 
``para ir de $\cB'$ a $\cB$, primero vamos de $\cB'$ a $\cC$ y despues vamos de $\cC$ a $\cB$''.

\

Las matrices $[\Id]_{\cB\cC}$ y $[\Id]_{\cB'\cC}$ son fáciles de calcular, ponemos los vectores de $\cB$ y $\cB'$ como columnas.
\end{observacion}
\end{frame}



\end{document}



