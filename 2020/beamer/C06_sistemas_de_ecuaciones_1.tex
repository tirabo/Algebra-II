%\documentclass{beamer} 
\documentclass[handout]{beamer} % sin pausas
\usetheme{CambridgeUS}

\usepackage[utf8]{inputenc}%esto permite (en Windows) escribir directamente 
\usepackage{graphicx}
\usepackage{array}
\usepackage{tikz} 
\usetikzlibrary{shapes,arrows,babel,decorations.pathreplacing}
\usepackage{verbatim} 
\usepackage{xcolor} 
\usepackage{amsgen,amsmath,amstext,amsbsy,amsopn,amsfonts,amssymb}
\usepackage{amsthm}
\usepackage{tikz}
\usepackage{tkz-graph}
\usepackage{mathtools}
\usepackage{xcolor}

%\setbeamertemplate{background}[grid][step=8 ]
\setbeamertemplate{itemize item}{$\circ$}
\setbeamertemplate{enumerate items}[default]

\definecolor{links}{HTML}{2A1B81}
\hypersetup{colorlinks,linkcolor=,urlcolor=links}

\newcommand{\img}{\operatorname{Im}}
\newcommand{\nuc}{\operatorname{Nu}}
\renewcommand\nu{\operatorname{Nu}}
\newcommand{\la}{\langle}
\newcommand{\ra}{\rangle}
\renewcommand{\t}{{\operatorname{t}}}
\renewcommand{\sin}{{\,\operatorname{sen}}}
\newcommand{\Q}{\mathbb Q}
\newcommand{\R}{\mathbb R}
\newcommand{\C}{\mathbb C}
\newcommand{\K}{\mathbb K}
\newcommand{\F}{\mathbb F}
\newcommand{\Z}{\mathbb Z}

\renewcommand{\figurename }{Figura}


\setbeamercolor{block}{fg=red, bg=red!40!white}
\setbeamercolor{block example}{use=structure,fg=black,bg=white!20!white}

\renewenvironment{block}[1]% environment name
{% begin code
	\par\vskip .2cm%
	{\color{blue}#1}%
	\vskip .2cm
}%
{%
	\vskip .2cm}% end code


\renewenvironment{alertblock}[1]% environment name
{% begin code
	\par\vskip .2cm%
	{\color{red!80!black}#1}%
	\vskip .2cm
}%
{%
	\vskip .2cm}% end code


\newenvironment{exercise}[1]% environment name
{% begin code
	\par\vspace{\baselineskip}\noindent
	\textbf{Ejercicio (#1)}\begin{itshape}%
		\par\vspace{\baselineskip}\noindent\ignorespaces
	}%
	{% end code
	\end{itshape}\ignorespacesafterend
}


\newenvironment{definicion}% environment name
{% begin code
	\par\vskip .2cm%
	{\color{blue}Definición}%
	\vskip .2cm
}%
{%
	\vskip .2cm}% end code

\newenvironment{observacion}% environment name
{% begin code
	\par\vskip .2cm%
	{\color{blue}Observación}%
	\vskip .2cm
}%
{%
	\vskip .2cm}% end code

\newenvironment{ejemplo}% environment name
{% begin code
	\par\vskip .2cm%
	{\color{blue}Ejemplo}%
	\vskip .2cm
}%
{%
	\vskip .2cm}% end code

\newenvironment{ejercicio}% environment name
{% begin code
	\par\vskip .2cm%
	{\color{blue}Ejercicio}%
	\vskip .2cm
}%
{%
	\vskip .2cm}% end code


\renewenvironment{proof}% environment name
{% begin code
	\par\vskip .2cm%
	{\color{blue}Demostración}%
	\vskip .2cm
}%
{%
	\vskip .2cm}% end code



\newenvironment{demostracion}% environment name
{% begin code
	\par\vskip .2cm%
	{\color{blue}Demostración}%
	\vskip .2cm
}%
{%
	\vskip .2cm}% end code

\newenvironment{idea}% environment name
{% begin code
	\par\vskip .2cm%
	{\color{blue}Idea de la demostración}%
	\vskip .2cm
}%
{%
	\vskip .2cm}% end code

\newenvironment{solucion}% environment name
{% begin code
	\par\vskip .2cm%
	{\color{blue}Solución}%
	\vskip .2cm
}%
{%
	\vskip .2cm}% end code



\newenvironment{lema}% environment name
{% begin code
	\par\vskip .2cm%
	{\color{blue}Lema}\begin{itshape}%
		\par\vskip .2cm
	}%
	{% end code
	\end{itshape}\vskip .2cm\ignorespacesafterend
}

\newenvironment{proposicion}% environment name
{% begin code
	\par\vskip .2cm%
	{\color{blue}Proposición}\begin{itshape}%
		\par\vskip .2cm
	}%
	{% end code
	\end{itshape}\vskip .2cm\ignorespacesafterend
}

\newenvironment{teorema}% environment name
{% begin code
	\par\vskip .2cm%
	{\color{blue}Teorema}\begin{itshape}%
		\par\vskip .2cm
	}%
	{% end code
	\end{itshape}\vskip .2cm\ignorespacesafterend
}


\newenvironment{corolario}% environment name
{% begin code
	\par\vskip .2cm%
	{\color{blue}Corolario}\begin{itshape}%
		\par\vskip .2cm
	}%
	{% end code
	\end{itshape}\vskip .2cm\ignorespacesafterend
}

\newenvironment{propiedad}% environment name
{% begin code
	\par\vskip .2cm%
	{\color{blue}Propiedad}\begin{itshape}%
		\par\vskip .2cm
	}%
	{% end code
	\end{itshape}\vskip .2cm\ignorespacesafterend
}

\newenvironment{conclusion}% environment name
{% begin code
	\par\vskip .2cm%
	{\color{blue}Conclusión}\begin{itshape}%
		\par\vskip .2cm
	}%
	{% end code
	\end{itshape}\vskip .2cm\ignorespacesafterend
}



%%%%%%%%%%%%%%%%%%%%%%%%%%%%%%%%%%%%%%%%%%%%%%%%%%%%%%%

\newcommand{\nc}{\newcommand}


%%%%%%%%%%%%%%%%%%%%%%%%%LETRAS
\nc{\RR}{{\mathbb R}} \nc{\CC}{{\mathbb C}} \nc{\ZZ}{{\mathbb Z}}
\nc{\FF}{{\mathbb F}} \nc{\NN}{{\mathbb N}} \nc{\QQ}{{\mathbb Q}}
\nc{\PP}{{\mathbb P}} \nc{\DD}{{\mathbb D}} \nc{\Sn}{{\mathbb S}}
\nc{\uno}{\mathbb{1}} \nc{\BB}{{\mathbb B}} \nc{\An}{{\mathbb A}}

\nc{\ba}{\mathbf{a}} \nc{\bb}{\mathbf{b}} \nc{\bt}{\mathbf{t}}
\nc{\bB}{\mathbf{B}}

\nc{\cP}{\mathcal{P}} \nc{\cU}{\mathcal{U}} \nc{\cX}{\mathcal{X}}
\nc{\cE}{\mathcal{E}} \nc{\cS}{\mathcal{S}} \nc{\cA}{\mathcal{A}}
\nc{\cC}{\mathcal{C}} \nc{\cO}{\mathcal{O}} \nc{\cQ}{\mathcal{Q}}
\nc{\cB}{\mathcal{B}} \nc{\cJ}{\mathcal{J}} \nc{\cI}{\mathcal{I}}
\nc{\cM}{\mathcal{M}} \nc{\cK}{\mathcal{K}}

\nc{\fD}{\mathfrak{D}} \nc{\fI}{\mathfrak{I}} \nc{\fJ}{\mathfrak{J}}
\nc{\fS}{\mathfrak{S}} \nc{\gA}{\mathfrak{A}}
%%%%%%%%%%%%%%%%%%%%%%%%%LETRAS




%%%%%%%%%%%%%%%%%yetter drinfield
\newcommand{\ydg}{{}_{\ku G}^{\ku G}\mathcal{YD}}
\newcommand{\ydgdual}{{}_{\ku^G}^{\ku^G}\mathcal{YD}}
\newcommand{\ydf}{{}_{\ku F}^{\ku F}\mathcal{YD}}
\newcommand{\ydgx}{{}_{\ku \Gx}^{\ku \Gx}\mathcal{YD}}

\newcommand{\ydgxy}{{}_{\ku \Gy}^{\ku \Gx}\mathcal{YD}}

\newcommand{\ydixq}{{}_{\ku \ixq}^{\ku \ixq}\mathcal{YD}}

\newcommand{\ydl}{{}^H_H\mathcal{YD}}
\newcommand{\ydll}{{}_{K}^{K}\mathcal{YD}}
\newcommand{\ydh}{{}^H_H\mathcal{YD}}
\newcommand{\ydhdual}{{}^{H^*}_{H^*}\mathcal{YD}}


\newcommand{\ydha}{{}^H_A\mathcal{YD}}
\newcommand{\ydhhaa}{{}^{\Hx}_{\Ay}\mathcal{YD}}


\newcommand{\wydh}{\widehat{{}^H_H\mathcal{YD}}}
\newcommand{\ydvh}{{}^{\ac(V)}_{\ac(V)}\mathcal{YD}}
\newcommand{\ydrh}{{}^{R\# H}_{R\# H}\mathcal{YD}}
\newcommand{\ydho}{{}^{H^{\dop}}_{H^{\dop}}\mathcal{YD}}
\newcommand{\ydhsw}{{}^{H^{\sw}}_{H^{\sw}}\mathcal{YD}}

\newcommand{\ydhlf}{{}^H_H\mathcal{YD}_{\text{loc fin}}}
\newcommand{\ydholf}{{}^{H^{\dop}}_{H^{\dop}}\mathcal{YD}_{\text{loc fin}}}

\nc{\yd}{\mathcal{YD}}

\newcommand{\ydsn}{{}^{\ku{\Sn_n}}_{\ku{\Sn_n}}\mathcal{YD}}
\newcommand{\ydsnd}{{}^{\ku^{\Sn_n}}_{\ku^{\Sn_n}}\mathcal{YD}}
\nc{\ydSn}[1]{{}^{\Sn_{#1}}_{\Sn_{#1}}\yd}
\nc{\ydSndual}[1]{{}^{\ku^{\Sn_{#1}}}_{\ku^{\Sn_{#1}}}\yd}
\newcommand{\ydstres}{{}^{\ku^{\Sn_3}}_{\ku^{\Sn_3}}\mathcal{YD}}
%%%%%%%%%%%%%%%%%yetter drinfield


%%%%%%%%%%%%%%%%%%%%%%%%%%%%%Operatorename
\newcommand\Irr{\operatorname{Irr}}
\newcommand\id{\operatorname{id}}
\newcommand\ad{\operatorname{ad}}
\newcommand\Ad{\operatorname{Ad}}
\newcommand\Ext{\operatorname{Ext}}
\newcommand\tr{\operatorname{tr}}
\newcommand\gr{\operatorname{gr}}
\newcommand\grdual{\operatorname{gr-dual}}
\newcommand\Gr{\operatorname{Gr}}
\newcommand\co{\operatorname{co}}
\newcommand\car{\operatorname{car}}
\newcommand\rk{\operatorname{rg}}
\newcommand\ord{\operatorname{ord}}
\newcommand\cop{\operatorname{cop}}
\newcommand\End{\operatorname{End}}
\newcommand\Hom{\operatorname{Hom}}
\newcommand\Alg{\operatorname{Alg}}
\newcommand\Aut{\operatorname{Aut}}
\newcommand\Int{\operatorname{Int}}
\newcommand\Id{\operatorname{Id}}
\newcommand\qAut{\operatorname{qAut}}
\newcommand\Map{\operatorname{Map}}
\newcommand\Jac{\operatorname{Jac}}
\newcommand\Rad{\operatorname{Rad}}
\newcommand\Rep{\operatorname{Rep}}
\newcommand\Ker{\operatorname{Ker}}
\newcommand\Img{\operatorname{Im}}
\newcommand\Ind{\operatorname{Ind}}
\newcommand\Comod{\operatorname{Comod}}
\newcommand\Reg{\operatorname{Reg}}
\newcommand\Pic{\operatorname{Pic}}
\newcommand\Emb{\operatorname{Emb}}
\newcommand\op{\operatorname{op}}
\newcommand\Perm{\operatorname{Perm}}
\newcommand\Res{\operatorname{Res}}
\newcommand\res{\operatorname{res}}
\newcommand{\sop}{\operatorname{Supp}}
\newcommand\Cent{\operatorname{Cent}}
\newcommand\sgn{\operatorname{sgn}}
\nc{\GL}{\operatorname{GL}}
%%%%%%%%%%%%%%%%%%%%%%%%%%%%%Operatorename

%%%%%%%%%%%%%%%%%%%%%%%%%%%%%Usuales de hopf
\nc{\D}{\Delta} 
\nc{\e}{\varepsilon}
\nc{\adl}{\ad_\ell}
\nc{\ot}{\otimes}
\nc{\Ho}{H_0} 
\nc{\GH}{G(H)} 
\nc{\coM}{\mathcal{M}^\ast(2,k)} 
\nc{\PH}{\cP(H)}
\nc{\Ftwist}{\overset{\curvearrowright}F}
\nc{\rep}{{\mathcal Rep}(H)}
\newcommand{\deltad}{_*\delta}
\newcommand{\B}{\mathfrak{B}}
\newcommand{\wB}{\widehat{\mathfrak{B}}}
\newcommand{\Cg}[1]{C_{G}(#1)}
\nc{\hmh}{{}_H\hspace{-1pt}{\mathcal M}_H}
\nc{\hm}{{}_H\hspace{-1pt}{\mathcal M}}
\renewcommand{\_}[1]{_{\left( #1 \right)}}
\renewcommand{\^}[1]{^{\left( #1 \right)}}
%%%%%%%%%%%%%%%%%%%%%%%%%%%%%Usuales de hopf

%%%%%%%%%%%%%%%%%%%%%%%%%%%%%%%%Usuales
\nc{\im}{\mathtt{i}}
\renewcommand{\Re}{{\rm Re}}
\renewcommand{\Im}{{\rm Im}}
\nc{\Tr}{\mathrm{Tr}} 
\nc{\cark}{char\,k} 
\nc{\ku}{\Bbbk} 
\newcommand{\fd}{finite dimensional}
%%%%%%%%%%%%%%%%%%%%%%%%%%%%%%%%Usuales

%%%%%%%%%%%%%%%%%%%%%%%%Especiales para liftings de duales de Sn
\newcommand{\xij}[1]{x_{(#1)}}
\newcommand{\yij}[1]{y_{(#1)}}
\newcommand{\Xij}[1]{X_{(#1)}}
\newcommand{\dij}[1]{\delta_{#1}}
\newcommand{\aij}[1]{a_{(#1)}}
\newcommand{\hij}[1]{h_{(#1)}}
\newcommand{\gij}[1]{g_{(#1)}}
\newcommand{\eij}[1]{e_{(#1)}}
\newcommand{\fij}[1]{f_{#1}}
\newcommand{\tij}[1]{t_{(#1)}}
\newcommand{\Tij}[1]{T_{(#1)}}
\newcommand{\mij}[1]{m_{(#1)}}
\newcommand{\Lij}[1]{L_{(#1)}}
\newcommand{\mdos}[2]{m_{(#1)(#2)}}
\newcommand{\mtres}[3]{m_{(#1)(#2)(#3)}}
\newcommand{\mcuatro}{m_{\textsf{top}}}
\newcommand{\trid}{\triangleright}
\newcommand{\link}{\sim_{\ba}}
%%%%%%%%%%%%%%%%%%%%%%%%Especiales para liftings de duales de Sn


%%%%%%%%%%%beaamer%%%%%%%%%%%%%%%%%
% Header: Secciones una arriba de la otra.
% \usetheme{Copenhagen}
% \usetheme{Warsaw}

% Header: Secciones una al lado de la otra. Feo footer.
% Los circulitos del header son medio chotos tambien.
% \usetheme[compress]{Ilmenau}
% Muy bueno: difuminado, sin footer:
%\usetheme{Frankfurt}
% agrego footer como el de Copenhagen.
%\useoutertheme[footline=authortitle,subsection=false]{miniframes}



\title[Clase 6 - Sistemas de ecuaciones lineales]{Álgebra/Álgebra II \\ Clase 6 -Sistemas de ecuaciones lineales 1}
%\author[C. Olmos / A. Tiraboschi]{Carlos Olmos / Alejandro Tiraboschi}
\institute[]{\normalsize FAMAF / UNC
	\\[\baselineskip] ${}^{}$
	\\[\baselineskip]
}
\date[10/09/2020]{10 de septiembre de 2020}


%\titlegraphic{\includegraphics[width=0.2\textwidth]{logo_gimp100.pdf}}

% Converted to PDF using ImageMagick:
% # convert logo_gimp100.png logo_gimp100.pdf


\begin{document}

\begin{frame}
\maketitle
\end{frame}


\begin{frame}

\begin{block}{Objetivo}
En las próximas clases aprenderemos a resolver sistemas de ecuaciones lineales sobre $\RR$ usando el \textit{método de Gauss.} 
\end{block}

\pause
 
 \vskip .4cm
Con este fin, veremos en esta clase\pause
\begin{itemize}
 \item[$\circ$] La definición de sistemas de ecuaciones lineales. \pause
  \item[$\circ$] Resolveremos algunos sistemas de ecuaciones concretos  usando el  método de \textit{eliminación de variables.}\pause
  \item[$\circ$] Justificaremos el método de eliminación de variables.
\end{itemize}
\pause
\vskip .4cm


El método de Gauss no es más que una forma de utilizar el método de eliminación de variables de manera sistemática y algorítmica. 
\pause
\vskip .4cm

El  método de Gauss será visto en las próximas clases. 

\end{frame}



\begin{frame}{El problema general}
			
	El problema a resolver será el siguiente: buscamos números  $x_1,\ldots,x_n$ en el cuerpo $\K$ ($= \R$ o $\C$)  que satisfagan las siguientes condiciones
	\vskip .3cm
	\begin{equation}\label{sist-eq}
	\begin{matrix}
	a_{11}x_1& + &a_{12}x_2& + &\cdots& + &a_{1n}x_n &= &y_1\\
	\vdots&  &\vdots& &&  &\vdots \\
	a_{m1}x_1& + &a_{m2}x_2& + &\cdots& + &a_{mn}x_n &=&y_m
	\end{matrix} \tag{*}
	\end{equation}
	\vskip .3cm
	donde $y_1, \ldots,y_m$ y $a_{ij}$ ($1 \le i \le m$, $1 \le j \le n$) son números en $\K$.
	
	
	\vskip .5cm\pause
	
	Se dice que las ecuaciones (*) forman un  \textit{sistema de ecuaciones lineales} de $m$ ecuaciones y $n$ incógnitas.
	\vskip .3cm
	\begin{itemize}
		\item  El sistema es \textit{homogeneo} si $y_i=0$ para todo $i$.  
		\item  El sistema es \textit{no homogeneo} si $y_i\not=0$ para algún $i$.  
	\end{itemize}
	
\end{frame}


\begin{frame}
	
		Los siguientes son sistemas de 2 ecuaciones lineales con 2 incógnitas:
		\begin{align*}
		\text{(1)}\qquad & 
		\begin{matrix}
		2x_1 + 8x_2 &= 0 &  \\
		2x_1 + x_2 &=  1& 
		\end{matrix}  \\
		&
		\\
		\text{(2)}\qquad & 
		\begin{matrix}
		2x_1 + x_2 &= 0 &  \\
		2x_1 - x_2 &=  1& 
		\end{matrix} \\
		&
		\\
		\text{(3)}\qquad & 
		\begin{matrix}
		2x + y &= 1 &  \\
		4x +2 y &=  2& 
		\end{matrix} 
		\end{align*}
		

\end{frame}

\begin{frame}



\begin{block}{Problema 1}
Encontrar las soluciones $(x,y,z)$ del sistema de ecuaciones:
\begin{equation*}
\begin{matrix}
x &  & +2z & = 1 \\
x& -3y & +3z & =2 \\
2x& -y & +5z & =3
\end{matrix}
\end{equation*}
Es decir, queremos encontrar los n\'umeros reales $x$, $y$ y $z$ que satisfagan las ecuaciones anteriores.
\end{block}

\vskip .5cm\pause
 
\begin{solucion}
Veremos que la única solución es $(x,y,z)=(-1,0,1)$. 
\end{solucion}


\end{frame}

\begin{frame}

Supongamos que $(x,y,z)$ es una solución de nuestro sistema


 
Entonces también vale que: 
\begin{equation*}
\begin{matrix}
&x& -3y & +3z & = & 2 \\
(-1)\cdot&(x &  & +2z) & = &(-1)\cdot1 \\
\hline
&& -3y & +z & = & 1  
\end{matrix}
\end{equation*}

 \pause
Por lo tanto $(x,y,z)$ también es solución del sistema
\begin{equation*}
\begin{matrix}
x &  & +2z & = 1 \\
& -3y & +z & = 1   \\
2x& -y & +5z & =3
\end{matrix}
\end{equation*}
\end{frame}


\begin{frame}
Dado que $(x,y,z)$ es solución del sistema
\begin{equation*}
\begin{matrix}
x &  & +2z & = 1 \\
& -3y & +z & = 1   \\
2x& -y & +5z & =3
\end{matrix}
\end{equation*}

 
Entonces también vale que: 
\begin{equation*}
\begin{matrix}
&2x& -y & +5z & = &3 \\
(-2)\cdot&(x &  & +2z) & = &(-2)\cdot1 \\
\hline
& & -y & + z & = & 1  
\end{matrix}
\end{equation*}
\pause
 
Por lo tanto $(x,y,z)$ también es solución del sistema
\begin{equation*}
\begin{matrix}
x &  & +2z & = 1 \\
& -3y & +z & = 1   \\
& -y & +z & =1
\end{matrix}
\qquad\mbox{equivalentemente}\qquad
\begin{matrix}
x &  & +2z & = 1 \\
& -y & +z & =1\\
& -3y & +z & = 1 
\end{matrix}
\end{equation*}
\end{frame}

\begin{frame}
Dado que $(x,y,z)$ es solución del sistema
\begin{equation*}
\begin{matrix}
x &  & +2z & = 1 \\
& -y & +z & =1\\
& -3y & +z & = 1 
\end{matrix}
\end{equation*}

 
Entonces también vale que: 
\begin{equation*}
\begin{matrix}
& -3y & +z & = & 1 \\
(-3)\cdot&( -y & +z) & = &(-3)\cdot1 \\
\hline
&   & -2z & = & -2  
\end{matrix}
\end{equation*}

 \pause
Por lo tanto $(x,y,z)$ también es solución del sistema
\begin{equation*}
\begin{matrix}
x &  & +2z & = 1 \\
& -y & +z & =1 \\
&    & -2z & = -2
\end{matrix}
\qquad\mbox{equivalentemente}\qquad
\begin{matrix}
x &  & +2z & = 1 \\
& -y & +z & =1 \\
&    & z & = 1
\end{matrix}
\end{equation*}
\end{frame}

\begin{frame}
Dado $(x,y,z)$ es solución del sistema
\begin{equation*}
\begin{matrix}
x &  & +2z & = 1 \\
& -y & +z & =1 \\
&    & z & = 1
\end{matrix}
\end{equation*}

 
Entonces también vale que: 
\begin{equation*}
\begin{matrix}
 &x & +2z & = & 1 \\
(-2)\cdot&(&z) & = &(-2)\cdot1 \\
\hline
&   & x & = & -1    
\end{matrix}
\qquad\mbox{y}\qquad
\begin{matrix}
&-y & +z & = & 1 \\
(-1)\cdot&(&z) & = &(-1)\cdot1 \\
\hline
&   & -y & = & 0
\end{matrix}
\end{equation*}
\pause
 
Por lo tanto $(x,y,z)$ también es solución del sistema
\begin{equation*}
\begin{matrix}
x &  &  & = -1 \\
& -y &  & =0 \\
&    & z & = 1
\end{matrix}
\qquad\mbox{equivalentemente}\qquad
\begin{matrix}
x & = -1 \\
y & =0 \\
z & = 1
\end{matrix}
\end{equation*}
\end{frame}

\begin{frame}
En resumen, supusimos que $(x,y,z)$ es una solución del sistema
\begin{equation*}
\begin{matrix}
x &  & +2z & = 1 \\
x& -3y & +3z & =2 \\
2x& -y & +3z & =1
\end{matrix}
\end{equation*}
 
y probamos que 
\begin{equation*}
x=-1\quad y=0,\quad z=1. 
\end{equation*}\qed

 \pause
 \vskip .4cm
\textbf{Comprobemos.} Si reemplazamos en el sistema  $x$, $y$ y $z$ por estos valores
\begin{equation*}
\begin{matrix}
(-1) &  & +2\cdot(1) & = 1 \\
(-1)& -3\cdot 0 & +3\cdot (1) & =2 \\
2\cdot(-1)& -0 & +3 \cdot (1) & =1
\end{matrix}
\end{equation*}
vemos que verifican las igualdades del sistema.
\end{frame}




\begin{frame}
Podría suceder que el sistema no tenga solución como en el siguiente caso.
\begin{block}{Problema 2}
Encontrar las soluciones $(x,y,z)$ del sistema de ecuaciones:
\begin{equation*}
\begin{matrix}
x &  & +2z & = 1 \\
x& -3y & +3z & =2 \\
2x& -3y & +5z & =4
\end{matrix}
\end{equation*}
\end{block}

 \pause
\begin{solucion}
Veremos  que el sistema no tiene solución.
\end{solucion}

 
\end{frame}

\begin{frame}
Supongamos que $(x,y,z)$ es una solución de nuestro sistema
\begin{equation*}
\begin{matrix}
x &  & +2z & = 1 \\
x& -3y & +3z & =2 \\
2x& -3y & +5z & =4
\end{matrix}
\end{equation*}

 
Entonces también vale que: 
\begin{equation*}
\begin{matrix}
&x& -3y & +3z & = & 2 \\
(-1)\cdot&(x &  & +2z) & = &(-1)\cdot1 \\
\hline
&& -3y & +z & = & 1  
\end{matrix}
\end{equation*}
\pause
 
Por lo tanto $(x,y,z)$ también es solución del sistema
\begin{equation*}
\begin{matrix}
x &  & +2z & = 1 \\
& -3y & +z & = 1   \\
2x& -3y & +5z & =4
\end{matrix}
\end{equation*}
 
\end{frame}


\begin{frame}

Dado que $(x,y,z)$ es solución del sistema
\begin{equation*}
\begin{matrix}
x &  & +2z & = 1 \\
& -3y & +z & = 1   \\
2x& -3y & +5z & =4
\end{matrix}
\end{equation*}

 
Entonces también vale que: 
\begin{equation*}
\begin{matrix}
&2x& -3y & +5z & = &4 \\
(-2)\cdot&(x &  & +2z) & = &(-2)\cdot1 \\
\hline
& & -3y & + z & = & 2  
\end{matrix}
\end{equation*}

 \pause
Por lo tanto $(x,y,z)$ también es solución del sistema
\begin{equation*}
\begin{matrix}
x &  & +2z & = 1 \\
& -3y & +z & = 1   \\
& -3y & +z & = 2
\end{matrix}
\end{equation*}
\end{frame}



\begin{frame}
Dado que $(x,y,z)$ es solución del sistema
\begin{equation*}
\begin{matrix}
x &  & +2z & = 1 \\
& -3y & +z & = 1   \\
& -3y & +z & = 2
\end{matrix}
\end{equation*}

 
Entonces también vale que: 
\begin{equation*}
\begin{matrix}
&& -3y & +z & = &2 \\
(-1)\cdot&(& -3y & +z) & = &(-1)\cdot1 \\
\hline
& &  & 0 & = & 1  
\end{matrix}
\end{equation*}

 \pause
Esta igualdad es un absurdo, el cual provino de suponer que nuestro sistema tenía solución. \qed
\end{frame}



\begin{frame}

Un sistema también puede tener infinitas soluciones.

\begin{block}{Problema 3}
Encontrar las soluciones $(x,y,z)$ del sistema de ecuaciones:
\begin{equation*}
\begin{matrix}
x &  & +2z & = 1 \\
x& -3y & +3z & =2 \\
2x& -3y & +5z & =3
\end{matrix}
\end{equation*}
\end{block}

 \pause
\begin{solucion}
Veremos que el conjunto de soluciones del sistema es
\begin{equation*}
\left\{(-2z+1,\frac{z-1}{3},z): z\in\mathbb R\right\}.
\end{equation*}
Es decir, todas las soluciones son de la forma 
\begin{equation*}
x=-2z+1\,\mbox{ e }\,y=\frac{z-1}{3}\,\mbox{donde $z\in\R$.} 
\end{equation*}
\end{solucion}
\end{frame}




\begin{frame}
	
Supongamos que $(x,y,z)$ es una solución de nuestro sistema
\begin{equation*}
\begin{matrix}
x &  & +2z & = 1 \\
x& -3y & +3z & =2 \\
2x& -3y & +5z & =3
\end{matrix}
\end{equation*}

 
Entonces también vale que: 
\begin{equation*}
\begin{matrix}
&x& -3y & +3z & = & 2 \\
(-1)\cdot&(x &  & +2z) & = &(-1)\cdot1 \\
\hline
&& -3y & +z & = & 1  
\end{matrix}
\end{equation*}

  \pause
Por lo tanto $(x,y,z)$ también es solución del sistema
\begin{equation*}
\begin{matrix}
x &  & +2z & = 1 \\
& -3y & +z & = 1   \\
2x& -3y & +5z & =3
\end{matrix}
\end{equation*}
\end{frame}



\begin{frame}

Dado que $(x,y,z)$ es solución del sistema
\begin{equation*}
\begin{matrix}
x &  & +2z & = 1 \\
& -3y & +z & = 1   \\
2x& -3y & +5z & =3
\end{matrix}
\end{equation*}



Entonces también vale que: 
\begin{equation*}
\begin{matrix}
&2x& -3y & +5z & = &3 \\
(-2)\cdot&(x &  & +2z) & = &(-2)\cdot1 \\
\hline
& & -3y & + z & = & 1  
\end{matrix}
\end{equation*}

 \pause
 
Por lo tanto $(x,y,z)$ también es solución del sistema
\begin{equation*}
\begin{matrix}
x &  & +2z & = 1 \\
& -3y & +z & = 1   \\
& -3y & +z & = 1
\end{matrix}
\qquad\mbox{equivalentemente}\qquad
\begin{matrix}
x &  & +2z & = 1 \\
& -3y & +z & = 1 
\end{matrix}
\end{equation*}
\end{frame}


\begin{frame}
Dado que $(x,y,z)$ es solución del sistema 
\begin{equation*}
\begin{matrix}
x &  & +2z & = 1 \\
& -3y & +z & = 1 
\end{matrix}
\end{equation*}
 
podemos despejar $x$ e $y$ en función de $z$.  \pause Esto es,
\begin{align*}
x&=-2z+1\\
y&=\frac{z-1}{3}
\end{align*}
y no tenemos ninguna condición sobre $z$.
\end{frame}



\begin{frame}
En resumen, supusimos que $(x,y,z)$ es una solución del sistema
\begin{equation*}
\begin{matrix}
x &  & +2z & = 1 \\
x& -3y & +3z & =2 \\
2x& -3y & +5z & =3
\end{matrix}
\end{equation*}
 
y probamos que 
\begin{equation*}
x=-2z+1\,\mbox{ e }\,y=\frac{z-1}{3}.
\end{equation*}\qed

 
  \pause
\vskip .4cm
\textbf{Comprobemos.} Si reemplazamos $x$, $y$ y $z$ por estos valores
\begin{equation*}
\begin{matrix}
(-2z+1) &  & +2z & = 1 \\
(-2z+1)& -3\cdot(\displaystyle\frac{z-1}{3}) & +3z & =2 \\
2\cdot(-2z+1)& -3\cdot(\displaystyle\frac{z-1}{3}) & +5z & =3
\end{matrix}
\end{equation*}
vemos que verifican las igualdades del sistema.

 

\end{frame}

\begin{frame}{Justificación  del método de eliminación de incógnitas}
		\begin{proposicion}\label{sist-impl}
		Sean $c_1,\ldots,c_m$ en $\K$. Si $(x_1,\ldots,x_n) \in \K^n$  es solución del sistema de ecuaciones
		\begin{equation*}
		\begin{matrix}
		a_{11}x_1& + &a_{12}x_2& + &\cdots& + &a_{1n}x_n &= &y_1\\
		\vdots&  &\vdots& &&  &\vdots \\
		a_{m1}x_1& + &a_{m2}x_2& + &\cdots& + &a_{mn}x_n &=&y_m.
		\end{matrix}
		\end{equation*}
		entonces $(x_1,\ldots,x_n)$ también es solución de la ecuación
		\begin{equation*}
		\sum_{i=1}^m c_i(a_{i1}x_1 + a_{i2}x_2 + \cdots + a_{in}x_n) = \sum_i c_iy_i.
		\end{equation*}
	\end{proposicion}
 
 \color{gray}{(se usa en p. \ref{prop-equiv})}
 
\end{frame}


\begin{frame}
		\begin{demostracion} \pause
		Por hipótesis
		\begin{equation*}
		a_{i1}x_1 + a_{i2}x_2 + \cdots + a_{in}x_n = y_i,\; \text{ para } 1 \le i \le m.
		\end{equation*}
		Luego, 
		\begin{equation*}
		\sum_{i=1}^m c_i(a_{i1}x_1 + a_{i2}x_2 + \cdots + a_{in}x_n) = \sum_i c_iy_i.
		\end{equation*}
		\qed
	\end{demostracion}

\pause
\begin{observacion}
	La ecuación anterior se puede reescribir:
	\begin{equation*}
	\left(\sum_{i=1}^{m}c_{i}a_{i1}\right)x_1 + \cdots +  	\left(\sum_{i=1}^{m}c_{i}a_{in}\right)x_n = \sum_{i=1}^{m}	c_{i}y_{i}.
	\end{equation*}
	Es decir es una nueva ecuación lineal con $n$ incógnitas.
\end{observacion}
\end{frame}

\begin{frame}
	La idea de hacer combinaciones lineales de ecuaciones es fundamental en el proceso de eliminación de incógnitas.
		\vskip .4cm \pause
	\begin{definicion}
		Decimos que dos sistemas de ecuaciones lineales son \textit{equivalentes}\index{sistemas lineales equivalentes} si cada ecuación de un sistema es combinación lineal del otro.
	\end{definicion}
	 \pause
	\vskip .4cm
	\begin{teorema}
		Dos sistemas de ecuaciones lineales equivalentes tienen las mismas soluciones.
	\end{teorema}
	\vskip .4cm
\end{frame}

\begin{frame}
	\begin{proof} \pause
		Sea 
		\begin{equation}\label{sist-01} \tag{*}
		\begin{matrix}
		a_{11}x_1& + &a_{12}x_2& + &\cdots& + &a_{1n}x_n &= &y_1\\
		\vdots&  &\vdots& &&  &\vdots \\
		a_{m1}x_1& + &a_{m2}x_2& + &\cdots& + &a_{mn}x_n &=&y_m
		\end{matrix}
		\end{equation}
		\vskip .4cm
		equivalente a
		
		\begin{equation}\label{sist-02} \tag{**}
		\begin{matrix}
		b_{11}x_1& + &b_{12}x_2& + &\cdots& + &b_{1n}x_n &= &z_1\\
		\vdots&  &\vdots& &&  &\vdots \\
		b_{k1}x_1& + &b_{k2}x_2& + &\cdots& + &b_{kn}x_n &=&z_k,
		\end{matrix}.
		\end{equation}
		\vskip .4cm

	\end{proof}
		\vskip .4cm
	\end{frame}

\begin{frame}
	Esto quiere decir  que 
	
	\begin{enumerate}
		\item las ecuaciones  de (\ref{sist-02}) se obtienen a partir de combinaciones lineales de las ecuaciones del sistema  (\ref{sist-01}), y
		\item las ecuaciones  de (\ref{sist-01}) se obtienen a partir de combinaciones lineales de las ecuaciones del sistema  (\ref{sist-02}).
	\end{enumerate} \pause
		\vskip .4cm
		Luego, por proposición de la diapositiva \ref{sist-impl}:\label{prop-equiv}
			\vskip .4cm
		\begin{enumerate}
			\item si  $(x_1,\ldots,x_n)$  es solución de  (\ref{sist-01}), también será solución de cada una de las ecuaciones de (\ref{sist-02}) y por lo tanto solución  del sistema  (\ref{sist-02}), y 
			\item si  $(x_1,\ldots,x_n)$  es solución de  (\ref{sist-02}), también será solución de cada una de las ecuaciones de (\ref{sist-01}) y por lo tanto solución  del sistema  (\ref{sist-01}). 
		\end{enumerate}
		
		\qed
		
\end{frame}


\begin{frame}
	\begin{observacion}
		Las combinaciones lineales que hemos utilizado  en los tres sistemas que hemos trabajado son \pause
		\begin{itemize}
			\item sumar a una ecuación una constante por otra, \pause
			\item multiplicar una ecuación por una constante no nula, y \pause
			\item permutar ecuaciones.  
		\end{itemize}
		
		 \pause
		\vskip .4cm
		Veremos que estas operaciones son ``reversibles'',  es decir, así como  haciendo estas  operaciones en la ecuaciones   llegamos de 
		
		\begin{equation*}
		\begin{matrix}
		x &  & +2z & = 1 \\
		x& -3y & +3z & =2 \\
		2x& -y & +5z & =3
		\end{matrix}
		\qquad \text{ a } 
		\qquad 
		\begin{matrix}x=-1\\ y=0 \\ z=1.
		\end{matrix}
		\end{equation*}
	\end{observacion}
	
\end{frame}


\begin{frame}
	Haciendo  las ``operaciones inversas'' (que son del mismo tipo)  podemos llegar de  
		\begin{equation*}
\begin{matrix}x=-1\\ y=0 \\ z=1.
\end{matrix}
	\qquad \text{ a } 
	\qquad 
		\begin{matrix}
	x &  & +2z & = 1 \\
	x& -3y & +3z & =2 \\
	2x& -y & +5z & =3
	\end{matrix}.
	\end{equation*}
	
		\vskip .4cm
	 \pause	
	Luego,  ambos sistemas son equivalentes y, por lo tanto, tiene las mismas soluciones.
	
		\vskip .4cm
			\vskip .4cm
				\vskip .4cm
				
\end{frame}



\begin{frame}{Conclusiones}
	 \pause
	\begin{itemize}
		\item[$\circ$]Un sistema de ecuaciones puede tener una, ninguna o infinitas soluciones. \pause
		
		\item[$\circ$]Hemos cambiado nuestro sistema inicial haciendo combinaciones lineales de las ecuaciones. \pause
		
		\item[$\circ$]El nuevo sistema es más sencillo en el sentido que: \pause
		\begin{itemize}
			\item Cada ecuación tiene menos incógnitas.
			\item Las soluciones quedan descriptas explícitamente.
		\end{itemize}
		 \pause
		\item[$\circ$]Las soluciones del nuevo sistema son las soluciones de nuestro sistema original.
		
	\end{itemize}
\end{frame}

\end{document}
