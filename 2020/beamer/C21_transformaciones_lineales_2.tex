%\documentclass{beamer} 
\documentclass[handout]{beamer} % sin pausas
\usetheme{CambridgeUS}

\usepackage{etex}
\usepackage{t1enc}
\usepackage[spanish,es-nodecimaldot]{babel}
\usepackage{latexsym}
\usepackage[utf8]{inputenc}
\usepackage{verbatim}
\usepackage{multicol}
\usepackage{amsgen,amsmath,amstext,amsbsy,amsopn,amsfonts,amssymb}
\usepackage{amsthm}
\usepackage{calc}         % From LaTeX distribution
\usepackage{graphicx}     % From LaTeX distribution
\usepackage{ifthen}
\input{random.tex}        % From CTAN/macros/generic
\usepackage{subfigure} 
\usepackage{tikz}
\usepackage[customcolors]{hf-tikz}
\usetikzlibrary{arrows}
\usetikzlibrary{matrix}
\tikzset{
	every picture/.append style={
		execute at begin picture={\deactivatequoting},
		execute at end picture={\activatequoting}
	}
}
\usetikzlibrary{decorations.pathreplacing,angles,quotes}
\usetikzlibrary{shapes.geometric}
\usepackage{mathtools}
\usepackage{stackrel}
%\usepackage{enumerate}
\usepackage{enumitem}
\usepackage{tkz-graph}
\usepackage{polynom}
\polyset{%
	style=B,
	delims={(}{)},
	div=:
}
\renewcommand\labelitemi{$\circ$}
\setlist[enumerate]{label={(\arabic*)}}
%\setbeamertemplate{background}[grid][step=8 ]
\setbeamertemplate{itemize item}{$\circ$}
\setbeamertemplate{enumerate items}[default]
\definecolor{links}{HTML}{2A1B81}
\hypersetup{colorlinks,linkcolor=,urlcolor=links}


\newcommand{\Id}{\operatorname{Id}}
\newcommand{\img}{\operatorname{Im}}
\newcommand{\nuc}{\operatorname{Nu}}
\newcommand{\im}{\operatorname{Im}}
\renewcommand\nu{\operatorname{Nu}}
\newcommand{\la}{\langle}
\newcommand{\ra}{\rangle}
\renewcommand{\t}{{\operatorname{t}}}
\renewcommand{\sin}{{\,\operatorname{sen}}}
\newcommand{\Q}{\mathbb Q}
\newcommand{\R}{\mathbb R}
\newcommand{\C}{\mathbb C}
\newcommand{\K}{\mathbb K}
\newcommand{\F}{\mathbb F}
\newcommand{\Z}{\mathbb Z}
\newcommand{\N}{\mathbb N}
\newcommand\sgn{\operatorname{sgn}}
\renewcommand{\t}{{\operatorname{t}}}
\renewcommand{\figurename }{Figura}

%
% Ver http://joshua.smcvt.edu/latex2e/_005cnewenvironment-_0026-_005crenewenvironment.html
%

\renewenvironment{block}[1]% environment name
{% begin code
	\par\vskip .2cm%
	{\color{blue}#1}%
	\vskip .2cm
}%
{%
	\vskip .2cm}% end code


\renewenvironment{alertblock}[1]% environment name
{% begin code
	\par\vskip .2cm%
	{\color{red!80!black}#1}%
	\vskip .2cm
}%
{%
	\vskip .2cm}% end code


\renewenvironment{exampleblock}[1]% environment name
{% begin code
	\par\vskip .2cm%
	{\color{blue}#1}%
	\vskip .2cm
}%
{%
	\vskip .2cm}% end code




\newenvironment{exercise}[1]% environment name
{% begin code
	\par\vspace{\baselineskip}\noindent
	\textbf{Ejercicio (#1)}\begin{itshape}%
		\par\vspace{\baselineskip}\noindent\ignorespaces
	}%
	{% end code
	\end{itshape}\ignorespacesafterend
}


\newenvironment{definicion}[1][]% environment name
{% begin code
	\par\vskip .2cm%
	{\color{blue}Definición #1}%
	\vskip .2cm
}%
{%
	\vskip .2cm}% end code

\newenvironment{observacion}[1][]% environment name
{% begin code
	\par\vskip .2cm%
	{\color{blue}Observación #1}%
	\vskip .2cm
}%
{%
	\vskip .2cm}% end code

\newenvironment{ejemplo}[1][]% environment name
{% begin code
	\par\vskip .2cm%
	{\color{blue}Ejemplo #1}%
	\vskip .2cm
}%
{%
	\vskip .2cm}% end code

\newenvironment{ejercicio}[1][]% environment name
{% begin code
	\par\vskip .2cm%
	{\color{blue}Ejercicio #1}%
	\vskip .2cm
}%
{%
	\vskip .2cm}% end code


\renewenvironment{proof}% environment name
{% begin code
	\par\vskip .2cm%
	{\color{blue}Demostración}%
	\vskip .2cm
}%
{%
	\vskip .2cm}% end code



\newenvironment{demostracion}% environment name
{% begin code
	\par\vskip .2cm%
	{\color{blue}Demostración}%
	\vskip .2cm
}%
{%
	\vskip .2cm}% end code

\newenvironment{idea}% environment name
{% begin code
	\par\vskip .2cm%
	{\color{blue}Idea de la demostración}%
	\vskip .2cm
}%
{%
	\vskip .2cm}% end code

\newenvironment{solucion}% environment name
{% begin code
	\par\vskip .2cm%
	{\color{blue}Solución}%
	\vskip .2cm
}%
{%
	\vskip .2cm}% end code



\newenvironment{lema}[1][]% environment name
{% begin code
	\par\vskip .2cm%
	{\color{blue}Lema #1}\begin{itshape}%
		\par\vskip .2cm
	}%
	{% end code
	\end{itshape}\vskip .2cm\ignorespacesafterend
}

\newenvironment{proposicion}[1][]% environment name
{% begin code
	\par\vskip .2cm%
	{\color{blue}Proposición #1}\begin{itshape}%
		\par\vskip .2cm
	}%
	{% end code
	\end{itshape}\vskip .2cm\ignorespacesafterend
}

\newenvironment{teorema}[1][]% environment name
{% begin code
	\par\vskip .2cm%
	{\color{blue}Teorema #1}\begin{itshape}%
		\par\vskip .2cm
	}%
	{% end code
	\end{itshape}\vskip .2cm\ignorespacesafterend
}


\newenvironment{corolario}[1][]% environment name
{% begin code
	\par\vskip .2cm%
	{\color{blue}Corolario #1}\begin{itshape}%
		\par\vskip .2cm
	}%
	{% end code
	\end{itshape}\vskip .2cm\ignorespacesafterend
}

\newenvironment{propiedad}% environment name
{% begin code
	\par\vskip .2cm%
	{\color{blue}Propiedad}\begin{itshape}%
		\par\vskip .2cm
	}%
	{% end code
	\end{itshape}\vskip .2cm\ignorespacesafterend
}

\newenvironment{conclusion}% environment name
{% begin code
	\par\vskip .2cm%
	{\color{blue}Conclusión}\begin{itshape}%
		\par\vskip .2cm
	}%
	{% end code
	\end{itshape}\vskip .2cm\ignorespacesafterend
}







\newenvironment{definicion*}% environment name
{% begin code
	\par\vskip .2cm%
	{\color{blue}Definición}%
	\vskip .2cm
}%
{%
	\vskip .2cm}% end code

\newenvironment{observacion*}% environment name
{% begin code
	\par\vskip .2cm%
	{\color{blue}Observación}%
	\vskip .2cm
}%
{%
	\vskip .2cm}% end code


\newenvironment{obs*}% environment name
	{% begin code
		\par\vskip .2cm%
		{\color{blue}Observación}%
		\vskip .2cm
	}%
	{%
		\vskip .2cm}% end code

\newenvironment{ejemplo*}% environment name
{% begin code
	\par\vskip .2cm%
	{\color{blue}Ejemplo}%
	\vskip .2cm
}%
{%
	\vskip .2cm}% end code

\newenvironment{ejercicio*}% environment name
{% begin code
	\par\vskip .2cm%
	{\color{blue}Ejercicio}%
	\vskip .2cm
}%
{%
	\vskip .2cm}% end code

\newenvironment{propiedad*}% environment name
{% begin code
	\par\vskip .2cm%
	{\color{blue}Propiedad}\begin{itshape}%
		\par\vskip .2cm
	}%
	{% end code
	\end{itshape}\vskip .2cm\ignorespacesafterend
}

\newenvironment{conclusion*}% environment name
{% begin code
	\par\vskip .2cm%
	{\color{blue}Conclusión}\begin{itshape}%
		\par\vskip .2cm
	}%
	{% end code
	\end{itshape}\vskip .2cm\ignorespacesafterend
}








\title[Clase 21 - Transformaciones lineales 2]{Álgebra/Álgebra II \\ Clase 21 - Transformaciones lineales 2}

\author[]{}
\institute[]{\normalsize FAMAF / UNC
	\\[\baselineskip] ${}^{}$
	\\[\baselineskip]
} 
\date[12/11/2020]{12 de noviembre de 2020}



\begin{document}
\begin{frame}
\maketitle
\end{frame}




\begin{frame}
Los dos teoremas que vamos a ver aquí son muy fuertes, en el sentido que dan mucha información por si solos y que además serán de utilidad para estudiar transformaciones inyectivas, suryectivas y biyectivas.
\pause

\

Las demostraciones son elegantes, en el sentido que sólo requieren que razonemos pegando algunas ideas y resultados pero sin trabajar en cuentas largas y tediosas.
\pause

\

Las demostraciones no son difíciles, pero requieren concentración y maduración de ideas y conceptos; hacer ejercicios ayuda a asimilarlos.
\pause
\


\end{frame}


\begin{frame}
El siguiente resultado relaciona las dimensiones del núcleo y la imagen. 
\vskip .4cm \pause
\begin{teorema}
Sea $T:V\longrightarrow W$ una transformación lineal. Si $V$ es de dimensión finita entonces
\begin{align*}
\dim V=\dim\nu(T)+\dim\im(T)
\end{align*}
\end{teorema}
\pause

\vskip .6 cm 
Observar que este resultado relaciona en forma general dos subespacios que ``viven'' en espacios diferentes. 

\begin{itemize}
	\item $\nu(T) \in V$,
	\item $\im(T) \in W$.
	\item Si $n = \dim V$,
\end{itemize}
tenemos que $n = \dim\nu(T)+\dim\im(T)$ \textit{cualquiera} sea $T$. 



\end{frame}

\begin{frame}

\begin{proof}
	

Sea $\{v_1, ..., v_k\}$ una base del $\nu(T)$ \quad ($\Rightarrow$ $\dim\nu T =k$).

\ \pause

Sea $\{v_1, ..., v_k, {w_1}, ..., {w_m}\}$ una base de $V$ obtenida completando la base de $\nu(T)$ \quad ($\Rightarrow$ $\dim V =k+m$).

\ \pause

Si probamos que $\{T({w_1}), ..., T({w_m})\}$ es una base de $\im(T)$ el teorema queda demostrado.\pause 
\vskip .4 cm 
Pues, de ser así, deducimos que
\begin{align*}
\dim V&= k +m \\
&=|\{v_1, ..., v_k\}|\quad+\quad|\{{w_1}, ..., {w_m}\}|\\
\noalign{\smallskip}
&=\dim\nu(T)\quad+\quad|\{T({w_1}), ..., T({w_m})\}|\\
\noalign{\smallskip}
&=\dim\nu(T)\quad+\quad\dim\im(T)
\end{align*}

Esto lo probaremos en las siguientes pantallas.
\end{proof}
\end{frame}

\begin{frame}


	Queremos ver que $\{T({w_1}), ..., T({w_m})\}$ genera $\im(T)$ y es LI.
	\pause
\vskip .4cm 

\begin{block}{$\{T({w_1}), ..., T({w_m})\}$ genera $\im(T)$:}\pause
\vskip .2cm
	Sea $w \in \im(T)$ $\Rightarrow$ $w = T(v)$, para algún $v \in V$.
	\vskip .2cm
	Como $v =\mu_1v_1+\cdots+\mu_kv_k+\lambda_1{w_1}+\cdots +\lambda_m{w_m}$
	\vskip .2cm
	$\Rightarrow$
	\begin{align*}
		w = T(v) &= T(\mu_1v_1+\cdots+\mu_kv_k+\lambda_1{w_1}+\cdots +\lambda_m{w_m}) \\
		\noalign{\vskip .3cm}
		&= \mu_1\underbrace{T(v_1)}_{0}+\cdots+\mu_k\underbrace{T(v_k)}_{0}+\lambda_1T({w_1})+\cdots +\lambda_mT({w_m}) \\
		&= \lambda_1T({w_1})+\cdots +\lambda_mT({w_m}) \\
		\end{align*}

	Luego $w \in \la T({w_1}), ..., T({w_m})\ra$. 
		
\end{block}


\end{frame}


\begin{frame}
 
	\begin{block}{$\{T({w_1}), ..., T({w_m})\}$ es LI:}\pause
		Sea $\lambda_1, ..., \lambda_m$ tales que
		\begin{align*}
		\lambda_1T({w_1})+\cdots +\lambda_mT({w_m})=0 \tag{*}
		\end{align*}
		debemos ver que  $\lambda_1=\cdots=\lambda_m=0$. \pause
	
	\
	Ahora bien 
	$$
	T(\lambda_1{w_1}+\cdots +\lambda_m{w_m})=\lambda_1T({w_1})+\cdots +\lambda_mT({w_m})\stackrel{(*)}{=}0.
	$$
	\pause
	
	Es decir $\lambda_1{w_1}+\cdots +\lambda_m{w_m} \in \nu(T) = \la v_1,\dots,v_k \ra$.\pause

	\ 

	$\Rightarrow$\pause

	$$\lambda_1{w_1}+\cdots +\lambda_m{w_m} = \mu_1v_1+\cdots+\mu_kv_k.$$

		\end{block}

\end{frame}

\begin{frame}

	Luego
	\begin{align*}
		0 = -\mu_1v_1-\cdots-\mu_kv_k+\lambda_1{w_1}+\cdots +\lambda_m{w_m}
	\end{align*}

	\pause
Dado que $\{v_1, ..., v_k, {w_1}, ..., {w_m}\}$ es LI, la igualdad
\begin{align*}
-\mu_1v_1-\cdots-\mu_kv_k+\lambda_1{w_1}+\cdots +\lambda_m{w_m}=0
\end{align*}
implica que 
\begin{align*}
\mu_1=\cdots=\mu_k=\lambda_1=\cdots=\lambda_m=0
\end{align*} 

Luego 
\begin{equation*}
	\lambda_1T({w_1})+\cdots +\lambda_mT({w_m})=0 \quad \Rightarrow \quad \lambda_1=\cdots=\lambda_m=0, 
\end{equation*}
y por lo tanto  $T({w_1}),\ldots ,\lambda_mT({w_m})$ es LI. 


\qed 
\end{frame}


\begin{frame}

El siguiente lema es importante por si mismo y además será necesario más adelante.
\pause

\begin{block}{Lema}
Sea $A\in\R^{m\times n}$ y $R$ la MERF equivalente a $A$. Entonces 
$$
\begin{matrix}
	\dim\{\text{soluciones del sistema homogéneo $AX=0$}\} \\
	\| \\
	|\text{variables libres de }RX=0|. 
\end{matrix}
$$
\vskip .4cm 

\end{block}
\pause



A continuación damos una idea de la demostración.
\end{frame}


\begin{frame}

\begin{block}{Idea de la demostración}\pause

Sea $r$ el número de filas no nulas de $R$ y $k_1,\ldots, k_r$ las columnas donde aparecen los  $1$'s principales.  \pause

\vskip .4cm 

Entonces, $k_1 < k_2 < \cdots< k_r$ y el sistema de ecuaciones asociado a $R$ es:
\begin{equation*}
\begin{matrix}
&x_{k_1}& + &\sum_{j \not= k_1,\ldots, k_r} b_{1j}\,x_j&= &0\\
&x_{k_2}& + &\sum_{j \not= k_1,\ldots, k_r} b_{2j}\,x_j&= &0\\
& \vdots& &  &\vdots \\
&x_{k_r}& + &\sum_{j \not= k_1,\ldots, k_r} b_{rj}\,x_j&= &0\\
\end{matrix}
\end{equation*}  

\vskip .4cm\pause

Sean $x_{j_{1}},x_{j_{1}},\ldots,x_{j_{n-r}}$ las $n-r$ variables libres 

(es decir los $x_j$ con $j \not= k_1,\ldots, k_r$)  

\vskip .4cm

Luego, 
\end{block} 
\end{frame}


\begin{frame}
	
	\begin{equation*}
		\begin{matrix}
		&x_{k_1} &= &-  \sum_{i=1}^{n-r} b_{1j_i}\,x_{j_i}\\
		&x_{k_2} &= &- \sum_{i=1}^{n-r} b_{2{j_i}}\,x_{j_i}\\
		& \vdots & &\vdots  \\
		&x_{k_r} &= &-  \sum_{i=1}^{n-r} b_{r{j_i}}\,x_{j_i}\\
		\end{matrix}
		\end{equation*}  \pause


Es decir, el subespacio formado por las soluciones de $AX=0$, consta de $n$-uplas
$(x_1, x_2, \ldots, x_n)$,
donde 
\begin{itemize}
	\item $x_k = x_{j_i}$, para algún $i=1,\ldots, n-r$, o
	\item  $x_k =$ c.l. de los  $x_{j_i}$.
\end{itemize}\pause

Por lo tanto, 
$$
W = \{\sum_{i=1}^{n-r} x_{j_i} w_i: x_{j_1}, \ldots, x_{j_{n-r}} \in \K\}
$$
para algunos $w_1,\ldots,w_{n-r}$ que son LI. \pause

\vskip .4cm 
Luego  $\dim(W) = n-r$. \qed 

\end{frame}

\begin{frame}


\begin{exampleblock}{Definición}
Sea $A\in\R^{m\times n}$.  \pause
\begin{itemize}
 \item El \textit{rango fila} de $A$ es la dimensión del subespacio de $\R^n$ generado por las filas de $A$.\pause
 \item El \textit{rango columna} de $A$ es la dimensión del subespacio de $\R^m$ generado por las columnas de $A$.
\end{itemize}
\end{exampleblock}\pause



\begin{teorema}
Sea $A\in\R^{m\times n}$. El rango fila de $A$ es igual al rango columna de $A$.
\end{teorema}\pause

Notar que si $n\neq m$, estamos comparando subespacios de distintos espacios vectoriales.

\

\end{frame}

\begin{frame}

\begin{proof}
	


Consideremos la transformación lineal $T:\R^n\longrightarrow\R^m$ dada por la multiplicación por $A$. \pause

\

Es decir, $T(v)=Av$ para todo $v\in\R^n$\pause

\

La demostración consiste en comparar el núcleo y la imagen de $T$ con los espacios fila y columna de $A$.
\end{proof}
\vskip 1cm \pause
Primero, el espacio columna de $A$ es igual a la imagen de $T$. 

\
\end{frame}

\begin{frame}


Esto es por la forma en que multiplicamos matrices:\pause
\begin{align*}
A=&\left[
\begin{array}{cccc}
\mid& \mid& &\mid\\
v_1 & v_2 & \cdots &v_k\\
\mid& \mid& &\mid
\end{array}
\right] \qquad \Rightarrow\\
&
T(e_i) =
A\left[
\begin{array}{c}
0\\\vdots \\ 1 \\ \vdots \\0
\end{array}
\right]=
v_i
\end{align*}\pause
Luego \quad $T(\lambda_1,\ldots,\lambda_n) =T(\sum \lambda_i e_i)= \lambda_1v_1+\cdots+\lambda_nv_n$\pause
\vskip .2cm
$\Rightarrow$
\vskip .2cm
$\im(T) = \langle v_1,\ldots,v_n \rangle$.  Entonces, rango columna de $A$ $=$  $\dim\im(T)$.


\end{frame}


\begin{frame}

Segundo, por el Corolario 2.5.2 las filas no nulas de la MERF equivalente a $A$ forman una base del espacio fila de $A$. \pause

\

Por lo tanto, el rango fila de $A$ es igual a la cantidad de $1$'s principales de la MERF. O dicho de otro modo,
\vskip .2cm
\begin{itemize}
 \item Rango fila de $A$ es igual a $n$ menos la cantidad de variables libres.

 ($n$ es la cantidad de columnas de $A$.)
\end{itemize}

\vskip .4cm\pause

Por otro lado, el $\nu(T)$ es igual al conjunto de soluciones de $AX=0$. 
\vskip .2cm\pause
Entonces, por el lema anterior,
\begin{itemize}
	\vskip .2cm
 \item $\dim\nu(T)$ es igual a la cantidad de variables libres
\end{itemize}

\end{frame}




\begin{frame}
En resumen: sea $r$ la cantidad de variables libres:\pause
\vskip .2cm
\begin{enumerate}
 \item Rango columna de $A$ es igual a $\dim\im(T)$\pause
 \vskip .2cm
 \item Rango fila de $A$ es igual a $n$ menos la cantidad de variables libres: $n-r$ . \pause
 \vskip .2cm
 \item $\dim\nu(T)$ es igual a la cantidad de variables libres: $r$.\pause
 \vskip .2cm
 \item $\dim\R^n=n$.
\end{enumerate}
\vskip .2cm\pause

Por lo tanto, por el teorema de la dimensión, 
\begin{align*}
	\dim(\K^n) &= \dim\nu(T) + \dim\im(T)& &\\
	n &= r + \text{rgcol}(A)& &\text{(por (4), (3) y (1))}\\
	n &= n -  \text{rgfil}(A) + \text{rgcol}(A)& &\text{(por (2))}\\
	0 &=-  \text{rgfil}(A) + \text{rgcol}(A).&&
\end{align*}\qed

\end{frame}




\end{document}
