%\documentclass{beamer} 
\documentclass[handout]{beamer} % sin pausas
\usetheme{CambridgeUS}

\usepackage{etex}
\usepackage{t1enc}
\usepackage[spanish,es-nodecimaldot]{babel}
\usepackage{latexsym}
\usepackage[utf8]{inputenc}
\usepackage{verbatim}
\usepackage{multicol}
\usepackage{amsgen,amsmath,amstext,amsbsy,amsopn,amsfonts,amssymb}
\usepackage{amsthm}
\usepackage{calc}         % From LaTeX distribution
\usepackage{graphicx}     % From LaTeX distribution
\usepackage{ifthen}
%\usepackage{makeidx}
\input{random.tex}        % From CTAN/macros/generic
\usepackage{subfigure} 
\usepackage{tikz}
\usepackage[customcolors]{hf-tikz}
\usetikzlibrary{arrows}
\usetikzlibrary{matrix}
\tikzset{
	every picture/.append style={
		execute at begin picture={\deactivatequoting},
		execute at end picture={\activatequoting}
	}
}
\usetikzlibrary{decorations.pathreplacing,angles,quotes}
\usetikzlibrary{shapes.geometric}
\usepackage{mathtools}
\usepackage{stackrel}
%\usepackage{enumerate}
\usepackage{enumitem}
\usepackage{tkz-graph}
\usepackage{polynom}
\polyset{%
	style=B,
	delims={(}{)},
	div=:
}
\renewcommand\labelitemi{$\circ$}

%\setbeamertemplate{background}[grid][step=8 ] %cuadriculado
\setbeamertemplate{itemize item}{$\circ$}
\setbeamertemplate{enumerate items}[default]
\definecolor{links}{HTML}{2A1B81}
\hypersetup{colorlinks,linkcolor=,urlcolor=links}


\newcommand{\Id}{\operatorname{Id}}
\newcommand{\img}{\operatorname{Im}}
\newcommand{\nuc}{\operatorname{Nu}}
\newcommand{\im}{\operatorname{Im}}
\renewcommand\nu{\operatorname{Nu}}
\newcommand{\la}{\langle}
\newcommand{\ra}{\rangle}
\renewcommand{\t}{{\operatorname{t}}}
\renewcommand{\sin}{{\,\operatorname{sen}}}
\newcommand{\Q}{\mathbb Q}
\newcommand{\R}{\mathbb R}
\newcommand{\C}{\mathbb C}
\newcommand{\K}{\mathbb K}
\newcommand{\F}{\mathbb F}
\newcommand{\Z}{\mathbb Z}
\newcommand{\N}{\mathbb N}
\newcommand\sgn{\operatorname{sgn}}
\renewcommand{\t}{{\operatorname{t}}}
\renewcommand{\figurename }{Figura}

%
% Ver http://joshua.smcvt.edu/latex2e/_005cnewenvironment-_0026-_005crenewenvironment.html
%

\renewenvironment{block}[1]% environment name
{% begin code
	\par\vskip .2cm%
	{\color{blue}#1}%
	\vskip .2cm
}%
{%
	\vskip .2cm}% end code


\renewenvironment{alertblock}[1]% environment name
{% begin code
	\par\vskip .2cm%
	{\color{red!80!black}#1}%
	\vskip .2cm
}%
{%
	\vskip .2cm}% end code


\renewenvironment{exampleblock}[1]% environment name
{% begin code
	\par\vskip .2cm%
	{\color{blue}#1}%
	\vskip .2cm
}%
{%
	\vskip .2cm}% end code




\newenvironment{exercise}[1]% environment name
{% begin code
	\par\vspace{\baselineskip}\noindent
	\textbf{Ejercicio (#1)}\begin{itshape}%
		\par\vspace{\baselineskip}\noindent\ignorespaces
	}%
	{% end code
	\end{itshape}\ignorespacesafterend
}


\newenvironment{definicion}[1][]% environment name
{% begin code
	\par\vskip .2cm%
	{\color{blue}Definición #1}%
	\vskip .2cm
}%
{%
	\vskip .2cm}% end code

\newenvironment{observacion}[1][]% environment name
{% begin code
	\par\vskip .2cm%
	{\color{blue}Observación #1}%
	\vskip .2cm
}%
{%
	\vskip .2cm}% end code

\newenvironment{ejemplo}[1][]% environment name
{% begin code
	\par\vskip .2cm%
	{\color{blue}Ejemplo #1}%
	\vskip .2cm
}%
{%
	\vskip .2cm}% end code

\newenvironment{ejercicio}[1][]% environment name
{% begin code
	\par\vskip .2cm%
	{\color{blue}Ejercicio #1}%
	\vskip .2cm
}%
{%
	\vskip .2cm}% end code


\renewenvironment{proof}% environment name
{% begin code
	\par\vskip .2cm%
	{\color{blue}Demostración}%
	\vskip .2cm
}%
{%
	\vskip .2cm}% end code



\newenvironment{demostracion}% environment name
{% begin code
	\par\vskip .2cm%
	{\color{blue}Demostración}%
	\vskip .2cm
}%
{%
	\vskip .2cm}% end code

\newenvironment{idea}% environment name
{% begin code
	\par\vskip .2cm%
	{\color{blue}Idea de la demostración}%
	\vskip .2cm
}%
{%
	\vskip .2cm}% end code

\newenvironment{solucion}% environment name
{% begin code
	\par\vskip .2cm%
	{\color{blue}Solución}%
	\vskip .2cm
}%
{%
	\vskip .2cm}% end code



\newenvironment{lema}[1][]% environment name
{% begin code
	\par\vskip .2cm%
	{\color{blue}Lema #1}\begin{itshape}%
		\par\vskip .2cm
	}%
	{% end code
	\end{itshape}\vskip .2cm\ignorespacesafterend
}

\newenvironment{proposicion}[1][]% environment name
{% begin code
	\par\vskip .2cm%
	{\color{blue}Proposición #1}\begin{itshape}%
		\par\vskip .2cm
	}%
	{% end code
	\end{itshape}\vskip .2cm\ignorespacesafterend
}

\newenvironment{teorema}[1][]% environment name
{% begin code
	\par\vskip .2cm%
	{\color{blue}Teorema #1}\begin{itshape}%
		\par\vskip .2cm
	}%
	{% end code
	\end{itshape}\vskip .2cm\ignorespacesafterend
}


\newenvironment{corolario}[1][]% environment name
{% begin code
	\par\vskip .2cm%
	{\color{blue}Corolario #1}\begin{itshape}%
		\par\vskip .2cm
	}%
	{% end code
	\end{itshape}\vskip .2cm\ignorespacesafterend
}

\newenvironment{propiedad}% environment name
{% begin code
	\par\vskip .2cm%
	{\color{blue}Propiedad}\begin{itshape}%
		\par\vskip .2cm
	}%
	{% end code
	\end{itshape}\vskip .2cm\ignorespacesafterend
}

\newenvironment{conclusion}% environment name
{% begin code
	\par\vskip .2cm%
	{\color{blue}Conclusión}\begin{itshape}%
		\par\vskip .2cm
	}%
	{% end code
	\end{itshape}\vskip .2cm\ignorespacesafterend
}







\newenvironment{definicion*}% environment name
{% begin code
	\par\vskip .2cm%
	{\color{blue}Definición}%
	\vskip .2cm
}%
{%
	\vskip .2cm}% end code

\newenvironment{observacion*}% environment name
{% begin code
	\par\vskip .2cm%
	{\color{blue}Observación}%
	\vskip .2cm
}%
{%
	\vskip .2cm}% end code


\newenvironment{obs*}% environment name
	{% begin code
		\par\vskip .2cm%
		{\color{blue}Observación}%
		\vskip .2cm
	}%
	{%
		\vskip .2cm}% end code

\newenvironment{ejemplo*}% environment name
{% begin code
	\par\vskip .2cm%
	{\color{blue}Ejemplo}%
	\vskip .2cm
}%
{%
	\vskip .2cm}% end code

\newenvironment{ejercicio*}% environment name
{% begin code
	\par\vskip .2cm%
	{\color{blue}Ejercicio}%
	\vskip .2cm
}%
{%
	\vskip .2cm}% end code

\newenvironment{propiedad*}% environment name
{% begin code
	\par\vskip .2cm%
	{\color{blue}Propiedad}\begin{itshape}%
		\par\vskip .2cm
	}%
	{% end code
	\end{itshape}\vskip .2cm\ignorespacesafterend
}

\newenvironment{conclusion*}% environment name
{% begin code
	\par\vskip .2cm%
	{\color{blue}Conclusión}\begin{itshape}%
		\par\vskip .2cm
	}%
	{% end code
	\end{itshape}\vskip .2cm\ignorespacesafterend
}








\title[Clase 17 - Espacios vectoriales 3]{Álgebra/Álgebra II \\ Clase 17 - Espacios vectoriales 3}

\author[]{}
\institute[]{\normalsize FAMAF / UNC
	\\[\baselineskip] ${}^{}$
	\\[\baselineskip]
} 
\date[29/10/2020]{29 de octubre de 2020}



\begin{document}

\begin{frame}
\maketitle
\end{frame}

\begin{frame}{Resumen}
	
    En esta clase veremos 
    \begin{itemize}
        \item como determinar un subespacio vectorial de $\K^n$ en forma implícita a partir de vectores que lo generan.

        \end{itemize}\pause
        \vskip .4cm 
Además,  veremos que

        \begin{itemize}
             \item la intersección de subespacios vectoriales es subespacio vectorial, \pause
             \item la suma de subespacios vectoriales es subespacio vectorial, \pause
             \item propiedades de las suma e intersección de subespacios, y\pause
             \item dependencia lineal. 
            \end{itemize}
   
            \pause
    \
    
   El tema de esta clase  está contenido de la sección 3.3 del apunte de clase ``Álgebra II / Álgebra - Notas del teórico''.
    \end{frame}
    


    \begin{frame}{Determinación ``implícita'' de un subespacio de $\K^n$}
 
        Nos interesa tener una manera de decidir rápidamente si un vector esta en el subespacio generado o no.
        \pause

        \
        
        Una forma sencilla de verificar si un vector pertenece a un subespacio $W \subseteq\K^n$ es obtener la descripción del  subespacio por un sistema de ecuaciones lineales homogéneas,  es decir 
        \begin{equation*}
           W = \{v \in \K^n: Av =0\},
        \end{equation*} 
        o equivalentemente
        \begin{equation*}
            v \in W \quad \Leftrightarrow \quad Av =0.
        \end{equation*} 
        
        Entonces comprobar si un vector $v$ pertenece o no a $W$  se reduce a calcular $Av$. 
        \pause

        \
        
        (Ejercicios 7 y 9 del Pr\'actico 6)
        \vskip 2cm
        \end{frame}
        
        
        \begin{frame}
        Ejemplificaremos con los siguientes vectores en $\R^4$:
        \begin{align*}
        v_1=(3,1,2,-1),&\quad
        v_2=(6,2,4,-2),\\
        v_3=(3,0,1,1),&\quad
        v_4=(15,3,8,-1)
        \end{align*}
        \pause

        \begin{exampleblock}{Problema}\pause
        
        Caracterizar mediante ecuaciones el subespacio $\langle v_1, v_2, v_3, v_4\rangle$.
       
        \pause
        
        \
        
        En otras palabras, queremos describir implícitamente el conjunto de los $b=(b_1,b_2,b_3,b_4)
        \in\R^4$ tales que $b\in\langle v_1, v_2, v_3, v_4\rangle$.
        
        \

        \pause
        O sea, los $b=(b_1,b_2,b_3,b_4)
        \in\R^4$ tales que 
        \begin{equation*}
            b=\lambda_1v_1+\lambda_2v_2+\lambda_3v_3+\lambda_4v_4 \tag{*}
        \end{equation*}
        
        
        con $\lambda_1,\lambda_2,\lambda_3,\lambda_4\in\R$.
        \end{exampleblock}
        \end{frame}
        

        \begin{frame}
            Planteemos la fórmula (*) en coordenadas, pero es conveniente hacerlo con vectores columna :
            \begin{align*}
                \lambda_1 \begin{bmatrix*}[r] 
                    3\\
        1\\
        2\\
        -1
        \end{bmatrix*}+
        \lambda_2\begin{bmatrix*}[r] 
        6\\
        2\\
        4\\
        -2
         \end{bmatrix*}+
        \lambda_3
        \begin{bmatrix*}[r] 
        3\\
        0\\
        1\\
        1
    \end{bmatrix*}+\lambda_4
        \begin{bmatrix*}[r] 
        15\\
        3\\
        8\\
        -1
    \end{bmatrix*}=
        \begin{bmatrix*}[r] 
        b_1\\
        b_2\\
        b_3\\
        b_4
    \end{bmatrix*}
            \end{align*}
            
\pause Luego 

\begin{equation*}
    \begin{bmatrix*}[r]
        3\lambda_1+6\lambda_2+3\lambda_3+15\lambda_4\\
    \lambda_1+2\lambda_2+3\lambda_4\\
    2\lambda_1+4\lambda_2+ \lambda_3+8\lambda_4\\
    -x-2\lambda_2+\lambda_3-\lambda_4
    \end{bmatrix*}
    = 
    \begin{bmatrix}
        b_1\\ b_2 \\b_3 \\ b_4 
    \end{bmatrix}
\end{equation*}

        \end{frame}

        \begin{frame}
        
        En forma de producto de matrices podemos reescribirla asi:
        
        \begin{equation*}
            \begin{bmatrix*}[r]    
        3&6&3&15\\
        1&2&0&3\\
        2&4&1&8\\
        -1&-2&1&-1
    \end{bmatrix*}
        \begin{bmatrix*}[r]
        \lambda_1\\
        \lambda_2\\
        \lambda_3\\
        \lambda_4
    \end{bmatrix*}
        =
        \begin{bmatrix*}[r]
        b_1\\
        b_2\\
        b_3\\
        b_4 
    \end{bmatrix*}
    \end{equation*}
        
    \pause  En forma de sistema de ecuaciones esto es:
     
        \begin{equation*}
            \begin{cases}
                3\lambda_1+6\lambda_2+3\lambda_3+15\lambda_4 = b_1\\
            \lambda_1+2\lambda_2+3\lambda_4 = b_2\\
            2\lambda_1+4\lambda_2+ \lambda_3+8\lambda_4 = b_3\\
            -x-2\lambda_2+\lambda_3-\lambda_4 = b_4
            \end{cases}\tag{**}
        \end{equation*}

        \pause
        
        \begin{block}{Conclusión}
        $b\in\langle v_1,v_2,v_3,v_4\rangle$ si y s\'olo si el sistema anterior tiene soluci\'on.
        \end{block}
        
        \end{frame}
        
        \begin{frame}
        
        El  sistema (**) tiene solución si el siguiente sistema la tiene  (es cambio de notación solamente)
  
        \begin{equation*}
            \begin{cases}
                3x+6y+3z+15w=b_1\\
                x+2y+3w=b_2\\
                2x+4y+z+8w=b_3\\
                -x-2y+z-w=b_4
            \end{cases}
        \end{equation*}
        \vskip .4cm\pause
        Este es exactamente el Ejercicio 2 de la Tarea 2. Entonces la respuesta a nuestro problema es
        \vskip .3cm
        \begin{block}{Respuesta}
            \vskip -.8cm 
        \begin{multline*}
            \langle v_1, v_2, v_3, v_4\rangle=\\\{(b_1,b_2,b_3,b_4)\in\mathbb{R}^4\mid b_1+3b_2-3b_3=0,b_1-6b_2-3b_4=0\}.
        \end{multline*}
            
        
        \end{block}
        
        \end{frame}
        
        \begin{frame}
        Notemos que podemos repetir todo el razonamiento anterior para cualesquiera vectores $v_1, ..., v_k$ en cualquier $\R^n$ y cualquier $b\in\R^n$.\pause
        
        \
        
        Sólo hay que tener presente que multiplicar una matriz por un vector columna es lo mismo que hacer una combinación lineal de las columnas de la matriz:\pause
        
        \
        
        Es decir, si
        $$
        A=\left[
        \begin{array}{cccc}
        \mid& \mid& &\mid\\
        v_1 & v_2 & \cdots &v_k\\
        \mid& \mid& &\mid
        \end{array}
        \right]
        ,$$ 
        entonces
        \begin{align*}
        A\left[
        \begin{array}{c}
        \lambda_1\\\vdots\\\lambda_k
        \end{array}
        \right]=
        \lambda_1v_1+\cdots+\lambda_kv_k
        \end{align*}
        \end{frame}
        
        \begin{frame}
        
      \begin{block}{Conclusión}
          
         Sean $v_1, ..., v_k\in\K^n$ y $A\in\K^{n\times k}$ la matriz cuyas  columnas son los vectores $v_1, ..., v_k$, es decir
        $$
        A=\left[
        \begin{array}{cccc}
        \mid& \mid& &\mid\\
        v_1 & v_2 & \cdots &v_k\\
        \mid& \mid& &\mid
        \end{array}
        \right]
        .$$\pause
        
        Entonces
        \begin{itemize}
         \item El subespacio vectorial  $\langle v_1, ..., v_k\rangle$ es igual al conjunto de los $b\in\K^n$ para los cuales el sistema $AX=b$ tiene solución.
         \vskip .2cm
         \item Las ecuaciones vienen dadas por las filas nulas de la MERF equivalente a $A$. En particular, si no tiene filas nulas entonces $\langle v_1, ..., v_k\rangle=\K^n$ porque el sistema $AX=b$ siempre tiene solución.
        \end{itemize}
    \end{block}
        \end{frame}
        
  



    \begin{frame}{Intersección y suma de subespacios vectoriales}
    
        \begin{teorema}\label{th-interseccion}
            Sea $V$ un espacio vectorial sobre $\K$. Entonces la intersección de subespacios vectoriales es un subespacio vectorial. 
        \end{teorema}  \pause
    \begin{proof}  \pause
        Veamos el caso de la intersección de dos subespacios. 
        \vskip .2cm
        Debemos probar que si $W_1$, $W_2$ subespacios $\Rightarrow$ $W_1 \cap W_2$  es subespacio. 
        \vskip .2cm
        Observemos:\quad  $w \in W_1 \cap W_2 \quad \Leftrightarrow\quad w \in W_1 \;\wedge\; w \in W_2$.
 \begin{align*}
    \text{ Sea  $\lambda \in \K$. } u,v \in  W_1 \cap W_2\quad &\Rightarrow \quad u,v \in W_1 \;\wedge\; u,v \in W_2 \\
    &\Rightarrow \quad u+\lambda v \in W_1 \;\wedge\; u+\lambda v\in W_2\qquad\qquad\qquad\\
    &\Rightarrow\quad  u+\lambda v  \in W_1 \cap W_2. 
 \end{align*}
 
Luego $W_1 \cap W_2$  es subespacio.
\qed
    \end{proof}
    \end{frame}
    

    
\begin{frame}
    \begin{ejemplo}
        Sean 
        \begin{align*}
            W_1 = \left\{ (x,y,z):  -3x + y + 2z = 0\right\}
        \end{align*}
        y
        \begin{align*}
             W_2 = \left\{ (x,y,z):  x - y + 2z = 0\right\}.
        \end{align*}
        Encontrar generadores de $W_1 \cap W_2$. 
    \end{ejemplo}  \pause
    \begin{solucion}  \pause
        Es claro que
        \begin{equation*}
            W_1 \cap W_2 = \left\{ (x,y,z):  -3x + y + 2z = 0 \;\wedge \;  x - y + 2z = 0\right\}.
        \end{equation*}
    \end{solucion}

    \end{frame}

    \begin{frame}
      Por lo tanto debemos resolver el sistema de ecuaciones
       \begin{equation*}
           \begin{cases}
            -3x + y + 2z = 0 \\  x - y + 2z = 0
           \end{cases}
       \end{equation*}\pause
       Reduzcamos la matriz del sistema a una MRF:
       \begin{align*}
           \begin{bmatrix}
            -3&1&2 \\  1&-1&2
           \end{bmatrix} &
           \stackrel{F_1+3F_2}{\longrightarrow}
           \begin{bmatrix}
            0&-2&8 \\  1&-1&2
           \end{bmatrix} 
           \stackrel{F_1/(-2)}{\longrightarrow}
           \begin{bmatrix}
            0&1&-4 \\  1&-1&2
           \end{bmatrix} 
           &\stackrel{F_2+F_1}{\longrightarrow}
           \begin{bmatrix}
            0&1&-4 \\  1&0&-2
           \end{bmatrix} 
       \end{align*}\pause

       Por lo tanto, $x_2 -4x_3 =0$ y $ x_1 -2x_3 =0$,  es decir  $x_2 = 4x_3$ y $ x_1 =2x_3 $. 
       \vskip .2cm\pause
       Luego, 
       \begin{equation*}
        W_1 \cap W_2 = \left\{ (2t,4t,t):  \t \in \R\right\} =  \left\{ t(2,4,1):  \t \in \R\right\}.
       \end{equation*}\pause

       La respuesta es entonces: $(2,4,1)$ es generador $ W_1 \cap W_2$. 




    \qed

\end{frame}
          

            

    \begin{frame}
        \begin{teorema}
            Sea $V$ un espacio vectorial sobre $\K$ y sean $v_1,\ldots,v_k \in V$. Entonces,  la intersección de todos los subespacios vectoriales que contienen  a $v_1,\ldots,v_k$ es igual a $\langle v_1,\ldots,v_k \rangle$.	
        \end{teorema}	  \pause
        \begin{proof}  \pause
            Denotemos 
            \begin{itemize}
                \item  $U= \bigcap$ de todos los subespacios vectoriales $\supseteq$ $\{ v_1,\ldots,v_k\}$.
            \end{itemize}
           \vskip .3cm  \pause
            Probaremos que  $U = \langle v_1,\ldots,v_k \rangle$ con la doble inclusión,  es decir probando que 
            $$U \subseteq  \langle v_1,\ldots,v_k \rangle \quad \text{y} \quad  \langle v_1,\ldots,v_k \rangle \subseteq U.$$
        \end{proof}
                       
    \end{frame}
    
            

    \begin{frame}
            
        ($U\subseteq\langle v_1, ..., v_k\rangle$) \;   \pause
        \vskip .3cm
        Primero, $U\subseteq\langle v_1, ..., v_k\rangle$ vale puesto que  $\langle v_1, ..., v_k\rangle$ es un subespacio que contiene a $\{v_1, ..., v_k\}$.
        \pause
\vskip .6cm
            
            
            
($\langle v_1,\ldots,v_k \rangle \subseteq U$) \;   \pause
\vskip .3cm
$U$  es intersección de subespacios $\Rightarrow$ (teor. p. \ref{th-interseccion}) $U$ es un subespacio.  
\vskip .3cm  \pause
Luego, $\{v_1, ..., v_k\}\subset U$  $\Rightarrow$ $\lambda_1v_1+\cdots+\lambda_kv_k\in U$, \; $\forall\,\lambda_1, ..., \lambda_k\in\K$.
\vskip .3cm  \pause
Por lo tanto $\langle v_1, ..., v_k\rangle\subseteq U.$ \qed
       
            
    \end{frame}
    
      
    \begin{frame}
        \begin{observacion}
            Si $V$ es un $\K$-espacio vectorial, $S$ y $T$ subespacios de $V$.   \pause
            \vskip .3cm
            Entonces $S \cup T$ \textit{no es necesariamente un subespacio} de $V$. 
            \vskip .3cm  \pause
            
        
            En efecto, consideremos en $\R^2$ los subespacios 
            $$S = \R(1,0)\quad \text{ y } T = \R(0,1).$$ 	
            \pause
            \begin{itemize}
                \item  $(1,0)\in  S$ y $(0,1) \in  T$ $\Rightarrow$  $(1,0), (0,1) \in  S \cup T$. \vskip .2cm
                \pause             \item Ahora bien $(1,0) + (0,1) = (1,1) \not\in S \cup T$, puesto que $(1,1) \not\in S$ y $(1,1) \not\in T$.
            \end{itemize}
        \end{observacion}
    \end{frame}
                

    \begin{frame}
        \begin{definicion} Sea $V$ un espacio vectorial sobre $\K$ y sean $S_1,\ldots,S_k$ subconjuntos  de $V$.
            definimos 
            \begin{equation*}
                S_1+  \cdots +S_k := \left\{s_1+\cdots+s_k: s_i \in S_i, 1 \le i \le k \right\},
            \end{equation*}
            el conjunto \textit{suma de los  $S_1,\ldots,S_k$.}
        \end{definicion}	
        \pause
        \vskip .8cm
        \begin{teorema}
            Sea $V$ un espacio vectorial sobre $\K$ y sean $W_1,\ldots,W_k$ subespacios  de $V$. Entonces $W= W_1+\cdots+W_k$ es un subespacio de $V$.
       \end{teorema}  \pause
       \begin{demostracion}  \pause
           Ejercicio (ver apunte). \qed
       \end{demostracion}
    \end{frame}

    

    \begin{frame}
        \begin{proposicion}
            Sea $V$ un espacio vectorial sobre $\K$ y sean $v_1,\ldots,v_r$ elementos de   de $V$. Entonces
            \begin{equation*}
                \langle v_1,\ldots,v_r \rangle = \langle v_1 \rangle+ \cdots + \langle v_r \rangle.
            \end{equation*}
        \end{proposicion}  \pause
        \begin{proof}  \pause
            Probemos el resultado viendo que los dos conjuntos se incluyen mutuamente.  \pause
            
            ($\subseteq$) Sea $w \in \langle v_1,\ldots,v_r \rangle$, luego $w = \lambda_1 v_1 +\cdots+ \lambda_r v_r$. Como $ \lambda_i v_i \in \langle v_i \rangle$, $1 \le i \le r$ ,  tenemos que  $w \in \langle v_1 \rangle+ \cdots + \langle v_r \rangle$.  En  consecuencia, $\langle v_1,\ldots,v_r \rangle \subseteq \langle v_1 \rangle+ \cdots + \langle v_r \rangle$.   \pause
            
            ($\supseteq$) Si $w \in \langle v_1 \rangle+ \cdots + \langle v_r \rangle$, entonces $w = w_1 + \cdots+w_r$ con $w_i \in \langle v_i\rangle$ para todo $i$. Por lo tanto, $w_i = \lambda_i v_i$ para algún $\lambda_i \in \K$ y  $w = \lambda_1 v_1 +\cdots+ \lambda_r v_r \in \langle v_1,\ldots,v_r \rangle $. En  consecuencia, $\langle v_1 \rangle+ \cdots + \langle v_r \rangle \subseteq \langle v_1,\ldots,v_r \rangle$. \qedhere
        \end{proof}
    
    \end{frame}
    
            
\begin{comment}

    \begin{frame}{Ejemplos}

        Veremos una serie de ejemplos de subespacios,  suma e intersección de subespacios.
\vskip .4cm 
\pause
\begin{ejemplo}
    Sea $V= \K^5$. Consideremos los vectores
            \begin{equation*}
                v_1 = (1,2,0,3,0), \qquad v_2 = (0,0,1,4,0), \qquad v_3 = (0,0,0,0,1),
            \end{equation*}
            y sea $W=  \operatorname{gen}\{v_1,v_2,v_3\}$. 
            \vskip .4cm
            Ahora bien, $w \in W$ $\Leftrightarrow$ $w = \lambda_1 v_1+\lambda_2 v_2+\lambda_3 v_3$ $\Leftrightarrow$
            \begin{align*}
                w &= \lambda_1 (1,2,0,3,0)+\lambda_2 (0,0,1,4,0)+\lambda_3 (0,0,0,0,1) \\
                &=  (\lambda_1,2\lambda_1,0,3\lambda_1,0)+ (0,0,\lambda_2,4\lambda_2,0)+ (0,0,0,0,\lambda_3) \\ &=(\lambda_1,2\lambda_1,\lambda_2,3\lambda_1+4\lambda_2,\lambda_3)
            \end{align*} 
           
\end{ejemplo}
\end{frame}
        
\begin{frame}
    Es decir 
    \begin{equation*}
        W = \left\{(\lambda_1,2\lambda_1,\lambda_2,3\lambda_1+4\lambda_2,\lambda_3)\in \K^5: \lambda_1,\lambda_2,\lambda_3 \in \K \right\}.
    \end{equation*}
    Luego,  también podríamos escribir
    \begin{equation*}
        W = \left\{(x_1,x_2,x_3,x_4,x_5)\in \K^5: x_2 = 2x_1, x_4 = 3x_1+4x_3 \right\}.
    \end{equation*}

    \qed

    \vskip 4cm
\end{frame}
\end{comment}

        
\begin{comment}
    \begin{frame}
    \begin{ejemplo}
        Encontrar la intersección entre el subespacio de los $(x_1,x_2,x_3) \in \R^3$ que satisfacen la ecuación $ -3x_1 + x_2 + 3x_3 = 0$ el subespacio generado por los vectores $(-5,4,3)$ y $(-4,4,4)$.
    \end{ejemplo}  \pause
    \begin{solucion}  \pause
        Denotemos $W_1$ al subespacio de los $(x_1,x_2,x_3) \in \R^3$ que satisfacen la ecuación $-3x_1 + x_2 + 3x_3 = 0$.
        \vskip .4cm
        Denotemos $W_2 =  \operatorname{gen}\{(-5,4,3),(-4,4,4)\}$. Luego,
        \begin{align*}
            W_2 &= \{ s(-5,4,3)+t(-4,4,4): s,t \in \R \} \\
            &= \{ (-5s-4t,4s+4t,3s+4t): s,t \in \R \}. 
        \end{align*}
    \end{solucion}
    \end{frame}
    
            

    \begin{frame} 
    Por lo tanto $(x_1,x_2,x_3) \in W_1 \cap W_2 \Leftrightarrow$
    \begin{equation*}
 (x_1,x_2,x_3) = (-5s-4t,4s+4t,3s+4t) \;\wedge\; -3x_1 + x_2 + 3x_3 = 0.
    \end{equation*}
    \vskip -.8cm
    \begin{align*}
     \text{Es decir,\quad} \Leftrightarrow\;  -3&(-5s-4t) + (4s+4t) + 3(3s+4t) = 0 \Leftrightarrow \\
        &15s +12t +4s +4t +9s +9t = 0  \Leftrightarrow \\
        &28s + 28 t =0  \Leftrightarrow \\
        &s +t = 0 \Leftrightarrow t=-s.
    \end{align*}
    \vskip -.8cm
    \begin{align*}
        \text{ Luego, \qquad\qquad}
        (x_1,x_2,x_3)& \in W_1 \cap W_2 \Leftrightarrow \\
        (x_1,x_2,x_3)& = (-5s+4s,4s-4s,3s-4s)  \Leftrightarrow \\
        (x_1,x_2,x_3)& = (-s,0,-s) \Leftrightarrow  \\
        (x_1,x_2,x_3)& = (-1,0,-1)s.
    \end{align*}  
    Por lo tanto $W_1 \cap W_2 =  \operatorname{gen}\{(-1,0,-1)\}$. \qed
    \end{frame}
    
    
\begin{frame}
    \begin{ejemplo}
        Sea $V = M_2(\C)$ y sean 
            \begin{equation*}
                W_1 = \left\{\begin{bmatrix} x_1&x_2\\x_3&0 \end{bmatrix}: x_1,x_2,x_3 \in \C \right\}, 
                \qquad W_2 = \left\{\begin{bmatrix} y_1&0\\0&y_2 \end{bmatrix}: y_1,y_2 \in \C \right\}.
            \end{equation*} 
            Determinar $W_1 + W_2$ y $W_1 \cap W_2$. 
    \end{ejemplo}  \pause

    \begin{solucion}  \pause
            Es claro que cada uno de estos conjuntos es un subespacio, pues,
            $$
            W_1 = \C \begin{bmatrix} 1&0\\0&0 \end{bmatrix} +
            \C \begin{bmatrix} 0&1\\0&0 \end{bmatrix} +
            \C \begin{bmatrix} 1&0\\0&0 \end{bmatrix}, \quad 
            W_2 = \C \begin{bmatrix} 1&0\\0&0 \end{bmatrix} +
            \C \begin{bmatrix} 0&0\\0&1 \end{bmatrix}. 
            $$
            Entonces, $W_1 + W_2 = V$. En  efecto, sea 
            $\begin{bmatrix} a&b\\c&d\end{bmatrix} \in V$, entonces
          
        \end{solucion}
  
\end{frame}


\begin{frame}
    $$
    \begin{bmatrix} a&b\\c&d\end{bmatrix} =  \begin{bmatrix} x_1&x_2\\x_3&0 \end{bmatrix} + 
    \begin{bmatrix} y_1&0\\0&y_2 \end{bmatrix} =
    \begin{bmatrix} x_1+y_1&x_2\\x_3&y_2 \end{bmatrix},
    $$
    y esto se cumple  tomando $x_1 = a, y_1 =0,  x_2 = b, x_3= c, y_2 =d$.
            \vskip .6cm
            Por otro lado
            \begin{align*}
                W_1 \cap W_2 &= \left\{\begin{bmatrix} a&b\\c&d\end{bmatrix}:  \begin{bmatrix} a&b\\c&d\end{bmatrix} \in W_1, \begin{bmatrix} a&b\\c&d\end{bmatrix} \in W_2\right\} \\
                &= \left\{\begin{bmatrix} a&b\\c&d\end{bmatrix}:  \begin{matrix}
                (a =x_1, b = x_2, c = x_3,d =0)\wedge \\(a = y_1, b=c=0, d= y_2)
                \end{matrix} \right\} \\
                &= \left\{\begin{bmatrix} a&0\\0&0\end{bmatrix}:  a \in \C\right\} .
            \end{align*}

\qed

 \end{frame}

\end{comment}
    \begin{frame}{Dependencia lineal}
        \begin{definicion}
            Sea $V$ un espacio vectorial sobre $\K$. Un subconjunto $S$ de $V$ se dice \textit{linealmente dependiente} o simplemente, \textit{LD} o \textit{dependiente}, si existen vectores distintos $v_1,\ldots,v_n \in S$  y escalares $\lambda_1,\ldots,\lambda_n$ de $\K$, no todos nulos, tales que 	
            \begin{equation*}
                \lambda_1v_1+\cdots+\lambda_nv_n=0.
            \end{equation*}
        \end{definicion}\pause
        \vskip .3cm
        \begin{observacion}
            Si el conjunto $S =\{v_1,\ldots,v_n\}$ , podemos reinterpretar la definición: 
             $v_1,\ldots,v_n$ son \textit{linealmente dependientes} o \textit{LD} si existen escalares $\lambda_1,\ldots,\lambda_n$ de $\K$, algún $\lambda_i \ne 0$, tales que 	
             \begin{equation*}
                 \lambda_1v_1+\cdots+\lambda_nv_n=0.
             \end{equation*}
        \end{observacion}
            
        
    \end{frame}  
    
    


    \begin{frame}
        \begin{itemize}
            \item En  la clase pasada vimos el concepto de que las combinaciones lineales de un conjunto de vectores generan un subespacio vectorial.\pause
            \item Dado un subespacio vectorial: ¿Cuál es el  número mínimo de vectores que generan el subespacio?
            \item En  general, dado un espacio vectorial ¿Cuál es el  número mínimo de vectores que generan el espacio y que propiedades tienen esos generadores?\pause
   
        \end{itemize}
   \vskip 1cm\pause
        Estas preguntas serán respondidas en la clase siguiente,  pero ahora veremos algunas herramientas que nos permitan prepararnos para estos resultados.   
       \end{frame}  
       
      





    \begin{frame}

        \begin{block}{Proposición}
        Sea $V$ un espacio vectorial y $v_1, ..., v_n\in V$. Entonces $v_1, ..., v_n$ son LD si y solo si alguno de ellos es combinación lineal de los otros.
        \end{block}\pause
        
        \begin{demostracion}\pause
            $(\Rightarrow)$ \pause Supongamos que son LD. Entonces $\lambda_1v_1+\cdots+\lambda_n v_n=0$ donde alg\'un escalar es no nulo. Digamos que tal escalar es $\lambda_i$. Podemos entonces despejar $v_i$, es decir escribirlo como combinación lineal de los otros:
            \begin{align*}
                v_i=-\frac{\lambda_1}{\lambda_i}v_1-\cdots-\frac{\lambda_{i-1}}{\lambda_{i-1}}v_{i-1}-\frac{\lambda_{i+1}}{\lambda_{i+1}}v_{i+1}-\cdots-\frac{\lambda_n}{\lambda_i}v_n
                \end{align*}
        
        \vskip .4cm\pause
 
         $(\Leftarrow)$ \pause 
         Supongamos que $v_i$ es combinaci\'on lineal de los otros, es decir 
        \begin{align*}
        &v_i=\lambda_1v_1+\cdots+\lambda_n\lambda_nv_n\\
        \Rightarrow&0=\lambda_1v_1+\cdots-v_i+\cdots+\lambda_n\lambda_iv_n
        \end{align*}
        como $-1\neq0$ esta multiplicando a $v_i$, la \'ultima igualdad dice que los vectores son LD.\qed
        \end{demostracion}
    
        \end{frame}
    



\end{document}