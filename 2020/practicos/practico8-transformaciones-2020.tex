\documentclass[12pt]{amsart}
\usepackage{amssymb}
\usepackage{enumerate}
\usepackage{amsmath}
\usepackage{geometry}
\geometry{ a4paper, total={210mm,297mm}, left=2cm, right=2cm, top=1.5cm, bottom=2.5cm, }
\usepackage{graphicx}
\usepackage{fancyhdr}
\usepackage{multicol}
\usepackage{enumitem}

\pagestyle{fancy}



\begin{document}

%\title{Pr\'actico 1}

\noindent {\tiny \'Algebra / \'Algebra II \hfill Segundo Cuatrimestre 2020}

%\maketitle

\centerline{\Large{Pr\' actico 9}}

\

\centerline{\textsc{Transformaciones lineales}}

\subsection*{Objetivos}

\begin{itemize}
 \item Aprender a calcular coordenadas de vectores
 \item Aprender a la matriz de una transformaci\'on lineal
 \item Aprender a calcular la matriz de una transformaci\'con respecto a las bases can\'onicas.
 \item Aprender a calcular el n\'ucleo y la imagen de una transformaci\'on.
 \item Aprender a verificar si una transformaci\'on lineal es inyectiva, sobreyectiva o un isomorfismo.
 \item Familiarizarse con el teorema sobre la dimensi\'on del n\'ucleo y la imagen.
\end{itemize}


\bigbreak 


\subsection*{Ejercicios}

Algunos ejercicios tienen ayuda, las que hemos puesto al final del archivo para que los puedan pensar un poco antes de leerlas.

\begin{enumerate}[resume, topsep=5pt,itemsep=5pt]
\item Decidir si las siguientes funciones son transformaciones lineales entre los respectivos espacios vectoriales sobre $\mathbb{R}$.
\begin{enumerate}[resume, topsep=5pt,itemsep=5pt]
 \item La traza $\operatorname{Tr}:\mathbb{R}^{n\times n}\longrightarrow\mathbb{R}$ (recordar Ejercicios 7, Pr\'actico 3) 
 \item $T:\mathbb{C}\longrightarrow\mathbb{C}$, $T(z)=\overline{z}$.
 \item $T:\mathbb{R}[x]\longrightarrow\mathbb{R}[x]$, $T(p(x))=x\,p(x)$.
 \item $T:\mathbb{R}^2\longrightarrow\mathbb{R}$, $T(x,y)=xy$
 \item $T:\mathbb{R}^2\longrightarrow\mathbb{R}^3$, $T(x,y)=(x,y,1)$
 \item El determinante $\operatorname{det}:\mathbb{R}^{n\times n}\longrightarrow\mathbb{R}$
\end{enumerate}

\item Calcular la matriz con respecto a la base can\'onica de las siguientes transformaciones. Es decir, encontrar una matriz del tama\~no apropiado de tal modo que las transformaciones sean iguales a multiplicar por dicha matriz. (Recordar Observaci\'on 3.1.2 de Garcia-Tiraboschi y la secci\'on ``Transformaciones de $\mathbb{R}^n$ a $\mathbb{R}^m$'' de la clase te\'orica ``Transformaciones lineales 1'')
\begin{enumerate}[topsep=5pt,itemsep=5pt]
 \item $T:\mathbb{R}^3\longrightarrow\mathbb{R}$, $T(x,y,z)=x+2y+3z$
 \item $T:\mathbb{R}^3\longrightarrow\mathbb{R}^2$, $T(x,y,z)=(x+2y+3z, y-z)$
 \item $T:\mathbb{R}^3\longrightarrow\mathbb{R}^3$, $T(x,y,z)=(x+2y+3z, y-z,0)$
 \item $T:\mathbb{R}^2 \longrightarrow \mathbb{R}^3$, \ $T(x,y)=(x-y,x+y,2x+3y)$.
\end{enumerate}

\item Para cada una de las transformaciones lineales del ejercicio anterior describir impl\'ictamente el n\'ucleo y dar un conjunto de generadores de la imagen.

\item Sea $T: \mathbb{R}^4 \to \mathbb{R}^5$ dada por $T(v) = Av$ donde $A$ es la siguiente matriz
	$$
	A=\left(\begin{matrix}
	0& 2& 0&1\\   1& 3& 0&1\\  -1&-1&0&0\\3&0&3&0\\2&1&1&0 \end{matrix}
	\right)
	$$
	\begin{enumerate}[topsep=5pt,itemsep=5pt]
		\item Decir cu\'ales de los siguientes vectores est\'an en el n\'ucleo:
		$(1,2,3,4)$, $(1,-1,-1,2)$, $(1,0,2,1)$.
		\item Decir cu\'ales de los siguientes vectores est\'an en la imagen:
		$(2,3,-1,0,1)$, $(1,1,0,3,1)$, $(1,0,2,1,0)$.
		\item Dar una base del n\'ucleo y de la imagen. 
		\item Dar la dimensi\'on del n\'ucleo y de la imagen.
		\item Describir el n\'ucleo y la imagen impl\'icitamente.
	\end{enumerate}
	
	
    \item Sea $T:\mathbb{R}^3\longrightarrow\mathbb{R}^3$ una transformaci\'on lineal tal que $T(e_1)=(1,2,3)$, $T(e_2)=(-1,0,5)$ y $T(e_3)=(-2,3,1)$. Calcular $T(2,3,8)$ y $T(0,1,-1)$. M\'as generalmente, calcular $T(x,y,z)$ para todo $(x,y,z)\in\mathbb{R}^3$ (es decir, que $T$ quede definida de manera parecida a las del ejercicio (2)). 
    
	\item Encontrar en cada caso, cuando sea posible, una matriz $A\in\mathbb{R}^{3\times 3}$ tal que la transformaci\'on lineal $T:\mathbb{R}^3\longrightarrow\mathbb{R}^3$, $T(v)=Av$, satisfaga las condiciones exigidas.
	Cuando no sea posible, explicar por qu\'e no es posible.
	\begin{enumerate}[ topsep=5pt,itemsep=5pt]
		\item $\operatorname{dim} \operatorname{Im}(T)=2$ y $\operatorname{dim}\operatorname{Nu}(T)=2$.
		\item $T$ inyectiva y $T(e_1)=(1,1,1)$, $T(e_2)=(1,0,1)$ y $T(e_3)=(0,1,0)$
		\item $T$ sobreyectiva y $T(e_1)=(1,1,1)$, $T(e_2)=(1,0,1)$ y $T(e_3)=(0,1,0)$
		\item $T(e_1)=(1,1,0)$, $T(e_2)=(0,1,1)$ y $T(e_3)=(1,0,1)$
		\item $(1,1,0)\in\operatorname{Im}(T)$ y $(0,1,1)\in\operatorname{Nu}(T)$  
		\item $\operatorname{dim} \operatorname{Im}(T)=1$
	\end{enumerate}
   \item Sea $V$ un espacio vectorial no nulo y $T:V\longrightarrow\mathbb{R}$ probar que $T=0$ \'o $T$ es sobreyectiva.
   
      
\end{enumerate}


\

\subsection*{M\'as ejercicios}
% Si ya hizo los ejercicios anteriores continue con la siguiente gu\'ia. Los ejercicios que siguen son similares y le pueden servir para practicar antes de los ex\'amenes.

\begin{enumerate}[resume, topsep=5pt,itemsep=5pt]
  \item Verificar, en cada una de las transformaciones lineales de este pr\'actico, si son inyectivas, sobreyectivas o isomorfismos.
  \item Sea $T: \mathbb{R}^3\longrightarrow\mathbb{R}[x]$ una transformaci\'on lineal tal que $T(e_1)=x^2+2x+3$, $T(e_2)=-x^2+5$ y $T(e_3)=-2x^2+3x+1$. Calcular $T(2,3,8)$ y $T(0,1,-1)$. M\'as generalmente, calcular $T(a,b,c)$ para todo $(a,b,c)\in\mathbb{R}^3$. 
   \item Sea $V$ un espacio vectorial de dimensi\'on finita y $T:V\longrightarrow V$ una transformaci\'on lineal. Probar las siguientes afirmaciones.
   \begin{enumerate}
    \item $\operatorname{Nu}(T)\subseteq\operatorname{Nu}(T^2)$
    \item $\operatorname{Nu}(T)\neq\operatorname{Im}(T)$ si $\dim(V)$ es impar.
    \end{enumerate}

\item Decidir si las siguientes afirmaciones son verdaderas o falsas. Justificar.

\begin{enumerate}
\item Existe una transformaci\' on lineal $T : \mathbb R^3 \to \mathbb R^2$ tal que $T(1, 0,-1) = (1, -1)$ y $T(-1, 0, 1) = (1, 0)$.
\item Existe una transformaci\' on lineal $T : \mathbb R^3 \to \mathbb R^2$ tal que $T(1, 0,-1) = (1, -1)$ y $T(-1, 0, 1) = (-1, 1)$.
\item  Si $T : \mathbb R^9 \to \mathbb R^7$ es una transformaci\' on lineal, entonces $\dim \operatorname{Nu}(T) \geq  2$.
\item Sea $T : V \to W$ una transformaci\' on lineal tal que $T(v_i) = w_i$, para $i = 1, \dots , n$. Si $\{w_1, \dots , w_n\}$ genera $W$, entonces
$\{v_1, \dots , v_n\}$ genera $V$.
\item Existe una transformaci\' on lineal $T : \mathbb R^2 \to \mathbb R^5$ tal que los vectores $(1, 0, -1, 0, 0)$, $(1, 1, -1, 0, 0)$ y $(1, 0, -1, 2, 1)$ pertenecen a la imagen de $T$.
\item Existe una transformaci\' on lineal sobreyectiva $T : \mathbb R^5 \to \mathbb R^4$ tal que los vectores $(1, 0, 1, -1, 0)$ y $(0, 0, 0, -1, 2)$
pertenecen al n\' ucleo de $T$.
\end{enumerate}


\end{enumerate}


%\subsection*{Ejercicios un poco m\'as dif\'iciles} 

%Si ya hizo los primeros ejercicios ya sabe lo que tiene que saber. Los %siguientes ejercicios le pueden servir si esta muy aburridx con la cuarentena. 

%\


%\begin{enumerate}
%\item 
%\end{enumerate}

%\subsection*{Ayudas}
\end{document} 
