\documentclass[12pt]{amsart}
\usepackage{amssymb}
\usepackage{enumerate}
\usepackage{amsmath}
\usepackage{geometry}
\geometry{ a4paper, total={210mm,297mm}, left=2cm, right=2cm, top=1.5cm, bottom=2.5cm, }
\usepackage{graphicx}
\usepackage{fancyhdr}
\usepackage{multicol}
%\usepackage{enumitem}

\pagestyle{fancy}



\begin{document}

%\title{Pr\'actico 1}

\noindent {\tiny \'Algebra / \'Algebra II \hfill Segundo Cuatrimestre 2020}

%\maketitle

\centerline{\Large{Pr\' actico 9}}

\

\centerline{\textsc{Coordenadas y matrices de transformaciones lineales}}

\subsection*{Objetivos}

\begin{itemize}
 \item Aprender a calcular coordenadas y la matriz de cambio de base.
 \item Aprender a calcular la matriz de una transformaci\'on lineal.
 \item Saber decidir si una transformaci\'on lineal es diagonalizable.
 \item Aprender a construir transformaciones lineales que satisfagan las propiedades solicitadas.
\end{itemize}


\bigbreak 


\subsection*{Ejercicios}


\begin{enumerate}
\item\label{otras bases} Sean $\mathcal{C}_n$, $n=2,3$, las bases can\'onicas de $\mathbb{R}^2$ y $\mathbb{R}^3$ respectivamente. Sean
	$\mathcal{B}_2=\{(1,0),(1,1)\}$ y $\mathcal{B}_3=\{(1,0,0),(1,1,0),(1,1,1)\}$ bases de $\mathbb{R}^2$ y $\mathbb{R}^3$, respectivamente.
	\begin{enumerate}
		\item Escribir la matriz de cambio de base $P_{{\mathcal{C}_n},{\mathcal{B}_n}}$ de $\mathcal{C}_n$ a $\mathcal{B}_n$, $n=2,3$.
		\item Escribir la matriz de cambio de base $P_{{\mathcal{B}_n},{\mathcal{C}_n}}$ de $\mathcal{B}_n$ a $\mathcal{C}_n$, $n=2,3$.
		\item ?`Qu\'e relaci\'on hay entre $P_{{\mathcal{C}_n},{\mathcal{B}_n}}$ y $P_{{\mathcal{B}_n},{\mathcal{C}_n}}$?
		\item Determinar los vectores de $\mathbb{R}^2$ y $\mathbb{R}^3$ que tienen coordenadas $(1,4)$ y $(1,1,1)$ en las bases $\mathcal{B}_2$ y $\mathcal{B}_3$, respectivamente.
		\item Determinar las coordenadas de $(2,3)$ y $(1,2,3)$ en las bases $\mathcal{B}_2$ y $\mathcal{B}_3$, respectivamente.
	\end{enumerate}
	
	\item \label{lineales1} 
	Sean $\mathcal{C}_n, \mathcal{B}_n$ como en el ejercicio anterior y consideremos las  transformaciones lineales
	\begin{itemize}
	 	\item $T:\mathbb{R}^2 \longrightarrow \mathbb{R}^3$, \ $T(x,y)=(x-y,x+y,2x+3y)$.
		\item $S:\mathbb{R}^3 \longrightarrow \mathbb{R}^2$, \ $S(x,y,z)=(x-y+z,2x-y+2z)$.
	\end{itemize}
	\begin{enumerate}
		\item Dar las matrices de las transformaciones respecto de las bases $\mathcal{B}_n$ y
		$\mathcal{C}_n$.
		\item Dar las matrices de las transformaciones  respecto de las bases $\mathcal{C}_n$ y
		$\mathcal{B}_n$.
		\item Dar las matrices de las transformaciones  respecto de las bases $\mathcal{B}_n$ y
		$\mathcal{B}_n$.
		\item Calcular $[TS]_{\mathcal{B}_3}$ y verificar que es igual al producto $[T]_{\mathcal{C}_2\mathcal{B}_3}[S]_{\mathcal{B}_3\mathcal{C}_2}$.
	\end{enumerate}
	    \item Para cada una de las siguientes transformaciones lineales, hallar sus autovalores,
	y para cada uno de ellos, dar una base de autovectores del espacio propio asociado. Luego, decir si la
	transformaci\'on considerada es o no  diagonalizable.
	\begin{enumerate}
		\item $T:\mathbb{R}^2\to \mathbb{R}^2$, \ $T(x,y)=(y,0)$.
		\item $T:\mathbb{R}^3\to \mathbb{R}^3$, \ $T(x,y,z)=(x+2z,-x-y+z,x+2y+z)$.
		\item $T:\mathbb{R}^3\to \mathbb{R}^3$, \ $T(x,y,z)=(4x+y+5z,4x-y+3z,-12x+y-11z)$.
        \item $T:\mathbb{R}^4\to \mathbb{R}^4$, \ $T(x,y,z,w)=(2x-y,x+4y,z+3w,z-w)$.
	\end{enumerate}
		
\item Definir en cada caso una transformaci\'on lineal $T:\mathbb{R}^3\longrightarrow\mathbb{R}^3$ que satisfaga las condiciones requeridas. ?`Es posible definir m\'as de una transformaci\'on lineal?
\begin{enumerate}
 \item $(1,0,0)\in \operatorname{Nu}(T)$ 
 \item $(1,0,0)\in \operatorname{Im}(T)$ 
 \item $(1,1,0) \in  \operatorname{Im}(T)$ y $(0,1,1),(1,2,1)\in\operatorname{Nu}(T)$
 \item $(1,1,0)\in \operatorname{Im}(T) \cap \operatorname{Nu}(T)$ y $(0,1,1)$ es autovector con autovalor $2$.
 \item Los vectores de la base $\mathcal{B}_3$ son autovectores con autovalores $1$, $2$ y $3$ respectivamente.
 \item La imagen de $T$ es el subespacio generado por $(1,0,-1)$ y $(1,2,2)$
 \item El n\'ucleo de $T$ est\' a generado por los vectores $(1,1,0)$, $(1,0,0)$ y $(2,1,0)$.
\end{enumerate}

\item Sea $V$ un espacio vectorial con base $\mathcal{B}=\{v_1, ..., v_n\}$ y $A=(a_{ij})\in\mathbb{R}^{n\times n}$ una matriz. Sea $\mathcal{B}'=\{v_1', ..., v_n'\}$ donde
\begin{align*}
v_j'=\sum_{i=1}^na_{ij}v_i\,\mbox{ para todo $1\leq j\leq n$}. 
\end{align*}

Probar que $\mathcal{B}'$ es una base de $V$ si y s\'olo si $A$ es inversible. En tal caso determinar la matriz de cambio de base de la base $\mathcal{B}'$ a la base $\mathcal{B}$ y viceversa.
\end{enumerate}


%\

%\subsection*{M\'as ejercicios}
% Si ya hizo los ejercicios anteriores continue con la siguiente gu\'ia. Los ejercicios que siguen son similares y le pueden servir para practicar antes de los ex\'amenes.


%\begin{enumerate}

%\item


%\end{enumerate}


%\subsection*{Ejercicios un poco m\'as dif\'iciles} 

%Si ya hizo los primeros ejercicios ya sabe lo que tiene que saber. Los %siguientes ejercicios le pueden servir si esta muy aburridx con la cuarentena. 

%\


%\begin{enumerate}
%\item 
%\end{enumerate}

%\subsection*{Ayudas}
\end{document} 
