\documentclass[12pt]{amsart}
\usepackage{amssymb}
\usepackage{enumerate}
\usepackage{amsmath}
\usepackage{geometry}
\geometry{ a4paper, total={210mm,297mm}, left=2cm, right=2cm, top=1.5cm, bottom=2.5cm, }
\usepackage{graphicx}
\usepackage{fancyhdr}
\usepackage{multicol}
\usepackage{enumitem}

\newcommand{\este}{($\star$) \; }

\pagestyle{fancy}



\begin{document}
	
\noindent {\tiny \'Algebra / \'Algebra II \hfill Segundo Cuatrimestre 2020}
	
\centerline{\Large{Pr\' actico 7}}
	
\
	
\centerline{\textsc{Bases y dimensi\' on}}
	
\
	
\noindent \textbf{Objetivos.} 
	
\begin{itemize}
\item Familiarizarse con los conceptos de conjunto de generadores e independencia lineal, base y dimensi\' on de un espacio vectorial.
		
\item Aprender a caracterizar los subespacios de $\mathbb R^n$ por generadores y de manera impl\' \i cita.

\item Dado un subespacio $W$ de $\mathbb R^n$, aprender a extraer una base de cualquier conjunto de generadores de $W$ y a completar cualquier subconjunto linealmente independiente de $W$ a una base.

\end{itemize}
	
	\
	
	\noindent \textbf{Ejercicios.} 
	
\begin{enumerate}

\item\label{practicos anteriores} Dar un conjunto de generadores para los siguientes subespacios vectoriales.

\begin{enumerate}
\item Los conjuntos de soluciones de los sistemas homog\'eneos del Ejercicio 5-Pr\'actica 1.
\item Los conjuntos descriptos en el Ejercicio 7-Pr\'actica 1.
\end{enumerate}

\

\item Sean $u=(1,1)$, $v=(1,0)$, $w=(0,1)$ y $z=(3,4)$ vectores de $\mathbb{R}^2$.
\begin{enumerate}
\item Escribir $z$ como combinaci\'on lineal de $u,v$ y $w$, con coeficientes todos no nulos.
\item Escribir $z$ como combinaci\'on lineal de $u$ y $v$.
\item Escribir $z$ como combinaci\'on lineal de $u$ y $w$.
\item Escribir $z$ como combinaci\'on lineal de $v$ y $w$.
\end{enumerate}

\

(En este ejercicio vemos como un vector se puede escribir de muchas maneras como combinaci\'on lineal de vectores dado. Esto pasa porque $u,v,w$ son LD en este ejercicio)

\
		
\item Sean $p$, $q$ y $r$ los polinomios $p=(1-x)(x+2)$, $q=x^2-1$ y $r=x(x^2-1)$.

\

\begin{enumerate}
			\item Escribir, si es posible, el polinomio $x$ como combinaci\'on lineal de $p,q$ y $r$.
			\item Elegir $a$ tal que el polinomio $x$ se pueda escribir como combinaci\'on lineal de $p,q$ y $2x^2+a$.
			\item Escribir, si es posible, el polinomio $x^3+x^2+x+1$ como combinaci\'on lineal de $p,q$ y $r$.
			\item Describir todos los polinomios de grado menor o igual que $3$ que son combinaci\' on lineal de $p,q$ y $r$.
 \end{enumerate}


\

\item\label{(5)}  En cada uno de los casos que siguen caracterizar con ecuaciones al subespacio vectorial dado por generadores.
    
    \
    
\begin{enumerate}
\item ${\left\langle(1,0,3),(0,1,-2)\right\rangle}\subseteq \mathbb{R}^3$.
\item ${\left\langle(1,2,0,1),(0,-1,-1,0),(2,3,-1,4)\right\rangle}\subseteq \mathbb{R}^4$.
\end{enumerate}

\

\item\label{son LI} En cada caso, determinar si el subconjunto indicado es linealmente independiente.

\

\begin{enumerate}
	\item $\{ (1,0,-1), (1,2,1), (0,-3,2) \}\subseteq \mathbb{R}^3$.
	
	\
	
	\item $\left\{  \begin{bmatrix} 1 & 0 & 2 \\ 0 & -1 & -3 \\ \end{bmatrix}, \quad
	\begin{bmatrix} 1 & 0 & 1 \\ -2 & 1 & 0 \\ \end{bmatrix}, \quad
	\begin{bmatrix} 1 & 2 & 3 \\ 3 & 2 & 1 \\ \end{bmatrix} \right\}\subseteq M_{2\times 3}(\mathbb{R})$.
\end{enumerate}

\

\item Extender, de ser posible, los siguientes conjuntos a una base de los respectivos espacios vectoriales

\begin{enumerate}
	\item Los conjuntos del Ejercicio \eqref{son LI}
	\item $\{ (1,2,0,0),(1,0,1,0) \}\subset\mathbb{R}^4$.
	\item $\{ (1,2,1,1),(1,0,1,1),(3,2,3,3)\}\subset\mathbb{R}^4$.
\end{enumerate}

\

\item  Exhibir una base y calcular la dimensi\'on de los siguientes subespacios:

\

\begin{enumerate}
    \item Los subespacios del Ejercicio \eqref{practicos anteriores}
	\item $W = \{(x,y,z,w,u) \in \mathbb{R}^5 \ : \ y = x - z,\, w = x + z,\,  u = 2x - 3z \}$.
	\item $W = \langle (1, 0, -1, 1),  (1, 2, 1, 1), (0, 1, 1, 0), (0, -2, -2, 0) \rangle \subseteq \mathbb R^4$.
	\item Matrices triangulares superiores $2\times 2$ y $3\times 3$.
	\item Matrices triangulares superiores $n\times n$ para cualquier $n\in\mathbb{N}$.
\end{enumerate}

\

\item En este ejercicio no es necesario hacer ninguna cuenta. Es l\'ogica y comprender bien la definici\'on de LI y LD. Probar las siguientes afirmaciones. 
\begin{enumerate}
\item Todo subconjunto de un conjunto LI es LI.
\item Todo conjunto que contiene un subconjunto LD es tambi\'en LD.
\item Todo conjunto que contiene al vector 0 es LD.
\item Un conjunto es LI si y s\'olo si todos sus subconjuntos \emph{finitos} son LI.
\item Probar que un conjunto de vectores $\{v_1, \dots, v_n\}$ es LD si y s\'olo si alguno de los vectores est\'a en el generado por los otros, esto es:  existe $i,\ 1 \leq i \leq n$ tal que $v_i \in \langle v_1, \dots, v_{i-1}, v_{i+1}, \dots, v_n\rangle$.
\end{enumerate}

\

\item\label{matrices} Sean
	$
	A_1=\begin{bmatrix}
	1&-2&0&3&7\\
	2&1&-3&1&1
	\end{bmatrix}$ y $A_2=\begin{bmatrix}
	3&2&0&0&3\\
	1&0&-3&1&0 \\
	-1&1&-3&1&-2
	\end{bmatrix}
	$.
	
	\begin{enumerate}
    \item Sean $W_1$ y $W_2$ los espacios soluci{\'o}n de los sistemas
	homog{\'e}neos asociados a $A_1$ y $A_2$, respectivamente.  Describir impl{\'\i}citamente $W_1\cap W_2$.
	\item Sean $V_1$ y $V_2$ los subespacios de $\mathbb{R}^5$ generado por las filas de $A_1$ y $A_2$, respectivamente. Dar un conjunto de generadores de $V_1+V_2$.
	\end{enumerate}



\
	
\item\label{todo} Sea $V=\mathbb{R}^6$, y sean $W_1$ y $W_2$ los siguientes subespacios de $V$:
	\begin{align*}
	W_1 &= \{ (u,v,w,x,y,z)\ : \ u+v+w=0,\, x+y+z=0\},  \\
	W_2 &= \left\langle{(1,-1,1,-1,1,-1),(1,2,3,4,5,6),(1,0,-1,-1,0,1),(2,1,0,0,0,0)}\right\rangle.
	\end{align*}
	\begin{enumerate}
		\item  Determinar $W_1 \cap W_2$, y describirlo por generadores y con ecuaciones.
		\item  Determinar $W_1+W_2$, y describirlo por generadores y con ecuaciones.
		\item  Decir cu\'ales de los siguientes vectores est\'an en $W_1\cap W_2$ y cu\'ales en $W_1+W_2$:
		\[ (1,1,-2,-2,1,1),\ (0,0,0,1,0,-1),\ (1,1,1,0,0,0),\ (3,0,0,1,1,3),\ (-1,2,5,6,5,4). \]
		\item Para los vectores $v$ del punto anterior que est\' en en $W_1+W_2$,  hallar $w_1\in W_1$ y $w_2\in W_2$ tales que $v=w_1+w_2$.
	\end{enumerate}

\

\item Dar un ejemplo de un conjunto de 3 vectores en $\mathbb{R}^3$ que sean LD, y tales que dos cualesquiera de ellos sean LI.

\

\item  Probar que si $\alpha$, $\beta$ y $\gamma$ son vectores LI en el $\mathbb{R}$-espacio vectorial $V$, entonces $\alpha +\beta$, $\alpha +\gamma$ y $\beta +\gamma $ tambi\'en son LI.

\

\item Decidir si las siguientes afirmaciones son verdaderas o falsas. Justificar

\begin{enumerate}
	\item Sean $W_1$ y $W_2$ subespacios no nulos de $\mathbb{R}^2$.  Si $W_1 \cap W_2$ contiene un vector no nulo, entonces $W_1 = W_2$.
	\item Sean $W_1$ y $W_2$ subespacios de dimensi\'on $2$ de $\mathbb{R}^3$ entonces $W_1 \cap W_2$ contiene un vector no nulo.
	\item Si $v$ no pertenece al subespacio generado por $\{\alpha_1,\dots ,\alpha_n\}$ entonces $\{\alpha_1,\dots,\alpha_n,v\}$ es un conjunto linealmente independiente.
	\item Todo conjunto de $3$ vectores en $\mathbb{R}^4$ se extiende a una base.
\end{enumerate}

% \item ?`Cual es la dimensi\'on de $\mathbb{C}^n$ cuando se lo considera como $\mathbb{R}$-espacio vectorial?.
% 
% \

\end{enumerate}

%============================================================
\subsection*{M\'as ejercicios}
%============================================================

\begin{enumerate}[resume]
\item Dar un conjunto de generadores  de los autoespacios del Ejercicio 2-Pr\'actica 4.

\

\item  Hallar $a, b, c\in \mathbb{R}$ tales que $(-1,2,1)=a(1,1,1)+b(1,-1,0)+c(2,1,-1)$.

\

\item
\begin{enumerate}
	\item Hallar escalares $a, b \in \mathbb R$ tales que $1+2i=a(1+i)+b(1-i)$.
	\item  Hallar escalares $w, z \in \mathbb C$ tales que $1+2i=z(1+i)+w(1-i)$.
\end{enumerate}
% 
% \item Sean $u=(-1,1)$, $v=(i,i)$, $w=(2,-i)$ y $z=(1,1+i)$.
% 		\begin{enumerate}
% 			\item Escribir $z$ como combinaci\'on lineal de $u,v$ y $w$, con coeficientes todos no nulos.
% 			\item Escribir $z$ como combinaci\'on lineal de $u$ y $v$.
% 			\item Escribir $z$ como combinaci\'on lineal de $u$ y $w$.
% 			\item Escribir $z$ como combinaci\'on lineal de $v$ y $w$.
% 		\end{enumerate}
% 		
% 
\

\item  Exhibir una base y calcular la dimensi\'on de los siguientes subespacios:

\

\begin{enumerate}
    \item $W=\{(x,y,z) \in \mathbb{R}^3 \ : \ z = x + y \}$.
	\item $W = \langle (-1, 1, 1, -1, 1),  (0, 0, 1, 0, 0), (2, -1, 0, 2, -1), (1, 0, 1, 1, 0) \rangle \subseteq \mathbb R^5$.
\end{enumerate}


\


\item  Repetir el ejercicio \eqref{(5)} con los subespacios:

\

\begin{enumerate}
	\item ${\left\langle(1,1,0,0),(0,1,1,0),(0,0,1,1)\right\rangle}\subseteq \mathbb{R}^4$.
	\item ${\left\langle 1+x+x^2,\, x-x^2+x^3,\, 1-x,\, 1-x^2,\, x-x^2,\, 1+x^4\right\rangle}\subseteq \mathbb{R}[x]$.
\end{enumerate}

\

    \item Sea  $S=\{v_1,v_2,v_3,v_4\}\subset\mathbb R^4$, donde
$$v_1=(-1,0,1,2), \quad v_2=(3,4,-2,5), \quad v_3=(0,4,1,11), \quad v_4=(1,4,0,9).$$
\begin{enumerate}
	\item  Describir impl{\'\i}citamente al subespacio  $W= \langle \, S\, \rangle,$ es decir,
	hallar un sistema de ecuaciones lineales homog\'eneo, para el cual su espacio de soluciones sea exactamente $W$.
	\item Si $W_1 = \langle \, v_1,v_2,v_3+v_4\, \rangle $ y $W_2 = \langle \, v_3,v_4\, \rangle $,
	describir $W_1\cap W_2$ impl{\'\i}citamente.
\end{enumerate}

\	

\item 	Expresar ${\mathbb  R}^2$ como suma de dos subespacios no nulos.

\
 \item Describir impl{\'\i}citamente $W_1+W_2$ del Ejercicio \eqref{matrices}.

\

\item Calcular la dimensi\'on y exhibir una base de los siguientes subespacios.
\begin{enumerate}
	\item $W = \{ p(x)=a+bx+cx^2+dx^3\in P_4 \ : \ a+d=b+c \}$.
\item $W= \{ p(x)\in P_4 \ : \ p'(0)=0 \}$.
	\item $S = \{A \in \mathbb{R}^{n\times n} \ : \ A = A^t\}$.
	\item $S = \{A \in \mathbb{C}^{n\times n} \ : \ A = \bar{A^t}\}$ (considerado como $\mathbb{R}$-subespacio de $\mathbb{C}^{n\times n}$).
\end{enumerate}


\end{enumerate}

\subsection*{Ejercicios un poco m\'as dif\' \i ciles} 

Si ya hizo los primeros ejercicios ya sabe lo que tiene que saber. Los siguientes ejercicios le pueden servir si esta muy aburridx con la cuarentena. 


\begin{enumerate}[resume]
	\item
	Mostrar dos complementos distintos del subespacio generado por $(1,2)$ en ${\mathbb{R}}^2$. Ver la definici\'on de complemento en la Subsecci\'on 2.4 de Garcia-Tirboschi.

	\
	
\item Sean $W_1$ y $W_2$ los subespacios del Ejercicio\eqref{todo}. Ver la Subsecci\'on 2.4 de Garcia-Tirboschi para responder lo siguiente.

\begin{enumerate}
\item  ?`Es la suma $W_1+W_2$ directa?
\item  Dar un complemento de $W_1$.
\item  Dar un complemento de $W_2$.
\end{enumerate}

\item 
\begin{enumerate}
	\item Probar que si \ $p_i(x), i = 1, \ldots , n$, son polinomios en $\mathbb{R}[x]$ tales que sus grados son todos distintos entonces $\{p_1(x), \ldots ,p_n(x)\}$ es un conjunto LI en $\mathbb{R}[x]$.
	\item Probar que $\{1 ,1 + x, (1 +x)^2\}$ es una base del subespacio $P_3(\mathbb R)$  de los polinomios de grado $\leq 3$ junto al polinomio nulo.
	\item Probar que  $P_3(\mathbb R)$ está generado por el conjunto $\{1, 2 + 2x, 1 - x + x^2, 2 - x^2\}$.
	?`Es este conjunto una base de $P_3(\mathbb R)$?
\end{enumerate}

\

\item Determinar si los siguientes conjuntos de $F(\mathbb{R})$ son linealmente independientes.

\begin{enumerate}
	\item $\{1,{\rm sen}(x),\cos(x)\}$.
	\item $\{1,{\rm sen}^2(x),\cos^2(x)\}$.
\end{enumerate}

\
\item En cada caso extender los conjuntos LI dados a una base del espacio vectorial correspondiente de dos maneras distintas.

\


\begin{enumerate}
	\item $\left\{ \begin{bmatrix} 0 & -1 \\ 1 & 0 \end{bmatrix}, \begin{bmatrix} 0 & -1 \\ 0 & 0 \end{bmatrix} \right\}
	\subseteq M_{2\times 2}(\mathbb{R})$.
	
	\
	
	\item $\{x, x-2x^2, 1-x+x^2\} \subseteq P_4(\mathbb{R})$.
\end{enumerate}

\

\item Sea $\{f_1,...,f_n\}$ un conjunto LI de funciones {\it pares} de $\mathbb{R}$ en $\mathbb{R}$ (i.e., $f(x)=f(-x)$ para todo $x$) y sea $\{g_1,...,g_m\}$ un conjunto LI de funciones {\it impares}
de $\mathbb{R}$ en $\mathbb{R}$ (i.e., $f(-x)=-f(x)$ para todo $x$).
Probar que $\{f_1,...,f_n,g_1,...,g_m\}$ es LI.

\end{enumerate}

\end{document} 
