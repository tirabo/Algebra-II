\documentclass[12pt]{amsart}
\usepackage{amssymb}
\usepackage{enumerate}
\usepackage{amsmath}
\usepackage{geometry}
\geometry{ a4paper, total={210mm,297mm}, left=2cm, right=2cm, top=1.5cm, bottom=2.5cm, }
\usepackage{graphicx}
\usepackage{fancyhdr}
\usepackage{multicol}
\usepackage{enumitem}

\newcommand{\este}{($\star$) \; }

\pagestyle{fancy}



\begin{document}
	
	\noindent {\tiny \'Algebra / \'Algebra II \hfill Segundo Cuatrimestre 2020}
	
	\centerline{\Large{Pr\' actico 5}}
	
	\
	
	\centerline{\textsc{Autovalores y autovectores}}
	
	\
	
	\noindent \textbf{Objetivos.} 
	
\begin{itemize}
\item Familiarizarse con las nociones de autovalor y autovector de una matriz cuadrada.

\item Aprender a calcular el polinomio caracter\' \i stico y los autovalores y autoespacios de una matriz.
\end{itemize}
	
	\
	
	\noindent \textbf{Ejercicios.} Los ejercicios con el s\'imbolo $\textcircled{a}$ tienen una ayuda al final del archivo para que recurran a ella despu\'es de pensar un poco.

\begin{enumerate} 


\item Para cada una de las siguientes matrices, hallar sus autovalores reales, y para cada autovalor, dar una descripci\' on param\'etrica del autoespacio asociado sobre $\mathbb{R}$. 


$$\textrm{(a)\; } \left[\begin{matrix}0 & 1\\ 0 & 0 \end{matrix} \right]
\quad 
\textrm{(b)\; } 
\left[\begin{matrix}0 & -1\\ -1 & 0 \end{matrix} \right]
\quad \textrm{(c)\; } \left[\begin{matrix}2 & 0 & 0\\ -1 & 1& -1\\ 0 & 0 & 2 \end{matrix} \right] , 
\quad 
\textrm{(d)\; } \begin{bmatrix} \lambda & 0 & 0 \\ 1 & \lambda & 0\\ 0 & 1 & \lambda \\ \end{bmatrix}, \; \lambda\in \mathbb R.$$


$$(e)\begin{bmatrix} 0 & -1 \\ 1 & 0 \end{bmatrix}
\quad
\textrm{(f)\; } \left[\begin{matrix}1 & 0& 0 \\ 0 & \cos\theta & \operatorname{sen}\theta\\ 0 & -\operatorname{sen}\theta & \cos\theta \end{matrix} \right], 0\leq \theta<2\pi
$$

\item Calcular los autovalores complejos de las matrices (e) y (f) del ejercicio anterior y para cada autovalor, dar una descripci\' on param\'etrica del autoespacio asociado sobre $\mathbb{C}$. 

\

\item[{\it Observaci\'on:}] Es oportuno destacar algunos fen\'omenos que podemos observar en los ejerecicios (1)-(2).

\

\begin{enumerate}
 \item Una matriz con coeficientes reales puede no tener autovalores reales pero s\'i complejos (matriz $(e)$) o tener ambos (matriz $(f)$).
 
 \
 
 \item La cantidad de autovalores distintos es menor o igual al tama\~no de la matriz. Incluso puede tener un s\'olo autovalor y ser muy grande (matriz $(d)$ y m\'as generalmente la matriz $(e)$ del Ejercicio \eqref{mas}) o tener tantos como el tama\~no (matriz $(b)$ y $(f)$).
 
 \
 
 \item Para describir param\'etricamente los autoespacios podemos necesitar distintas cantidad de par\'ametros para los distintos autovalores (la matriz $(c)$). Esta cantidad de par\'ametros es lo que llamaremos {\it dimensi\'on}. 
\end{enumerate}


\

%\begin{enumerate}
%\item $T:\mathbb{R}^2\to \mathbb{R}^2$, \ $T(x,y)=(y,0)$.
%\item $T:\mathbb{R}^3\to \mathbb{R}^3$, \ $T(x,y,z)=(x+2z,-x-y+z,x+2y+z)$.
%\item $T:\mathbb{R}^3\to \mathbb{R}^3$, \ %$T(x,y,z)=(4x+y+5z,4x-y+3z,-12x+y-11z)$.
%\item $T:\mathbb{R}^4\to \mathbb{R}^4$, \ $T(x,y,z,w)=(2x-y,x+4y,z+3w,z-w)$.
%\end{enumerate}
		

\
	
\item Sea $A$ una matriz $2\times 2$.

\

\begin{enumerate} 	
\item Probar que el polinomio caracter\'istico de $A$ es \ $X^2-\operatorname{Tr}(A)X+\det(A)$.

\

\item Si $A$ no es invertible, probar que los autovalores de  $A$ son $0$ y $\operatorname{Tr}(A)$.
\end{enumerate}

\

\item Decidir si las siguientes afirmaciones son verdaderas o falsas. Justificar.

\

\begin{enumerate}
\item Existe una matriz invertible $A$ tal que $0$ es autovalor de $A$.

\

\item  Si $A$ es invertible todo autovector de $A$ es autovector de $A^{-1}$.

\

\item Si $A$ es una matriz nilpotente entonces $0$ es el \' unico autovalor de $A$.
\end{enumerate}

\

\item Sea $A$ una matriz $n\times n$ y sea $f(x) = ax^2+bx+c$ un polinomio. Sea $f(A)$ la matriz $n \times n$ definida por
$$f(A) = a A^2+bA+c\operatorname{Id}_n.$$
Probar que todo autovector de $A$ con autovalor $\lambda$ es autovector de $f(A)$ con autovalor $f(\lambda)$.

\

\item  Calcular el polinomio caracter\'istico de las matrices.
\begin{align*}
A_1=\begin{bmatrix} 0 & -a_0 \\ 1 & -a_1 
\end{bmatrix}\quad\mbox{y}\quad
A_2=\begin{bmatrix} 0 & 0 & -a_0 \\ 1 & 0 & -a_1 \\ 0 & 1 & -a_2 
\end{bmatrix}
\end{align*}
donde $a_0, a_1, a_2$ son escalares.


Estas matrices forman parte de una familia de matrices similares. Es decir, para cada $n\in\mathbb{N}$ podr\'iamos definir una matriz $A_{n-1}$ de tama\~no $n\times n$ con una forma parecida. ?`Te imaginas c\'omo ser\'ia $A_3$, $A_4$, etc.? La respuesta est\'a en el ejercicio \eqref{matriz de un polinomio}.




\

\item\label{tr det}$\textcircled{a}$ Sea $A$ una matriz $n \times n$. Probar que el t\'ermino independiente del		polinomio caracter\'istico de $A$ es igual $(-1)^n\det(A)$ y que el coeficiente de grado $(n-1)$ es igual a $-\operatorname{Tr}(A)$.



	\end{enumerate}

\

\subsection*{M\'as ejercicios}
Si ya hizo los ejercicios anteriores continue con la siguiente gu\'ia. Los ejercicios que siguen son similares y le pueden servir para practicar antes de los ex\'amenes.

\

\begin{enumerate}[resume]

\item\label{mas} Repetir los Ejercicio 1 y 2 con las matrices 

$$\textrm{(a)\; }\begin{bmatrix} 2 & 3 \\ -1 & 1
\end{bmatrix}, \qquad
\textrm{(b)\; }\begin{bmatrix} -9 & 4 & 4 \\ -8 & 3 & 4 \\ -16 & 8 & 7 \end{bmatrix}, \qquad \textrm{(c)\; } \left[\begin{matrix}4 & 4 & -12\\ 1 & -1 & 1\\ 5 & 3 & -11 \end{matrix} \right]$$

$$
\textrm{(d)\; } \left[\begin{matrix}2 & 1 & 0 & 0\\ -1 & 4 & 0 & 0\\ 0 & 0 & 1 & 1 \\ 0 & 0 & 3 & -1\end{matrix} \right], 
\quad \textrm{(e)\; } \begin{bmatrix} \lambda & 0 & 0 & \dots & 0  \\ 1 & \lambda & 0 &\dots & 0  \\ 0 & 1 & \lambda&  \dots & 0  \\ \vdots & \vdots & \quad & \ddots & \vdots\\ 0 &  0&   \dots & 1  & \lambda \end{bmatrix}, \; \lambda\in \mathbb R.
$$
\

\item Sea $A$ una matriz $n\times n$ y sea $f(x) = a_0 + a_1 x + \dots + a_nx^n$, $n \geq 1$, $a_n \neq 0$, un polinomio. Sea $f(A)$ la matriz $n \times n$ definida por
$$f(A) = a_0 \operatorname{Id}_n + a_1 A + \dots + a_n A^n.$$

Probar que todo autovector de $A$ con autovalor $\lambda$ es autovector de $f(A)$ con autovalor $f(\lambda)$.

\

\item Probar que hay una \'unica matriz $2\times 2$, $A$, con coeficientes reales tal que $(1,1)$ es autovector de autovalor $2$, y $(-2,1)$ es autovector de autovalor $1$. 

\

\item\label{matriz de un polinomio} $\textcircled{a}$ Sean $a_0, ..., a_{n-1}$ escalares. Calcular el polinomio caracter\'istico de
\begin{align*}
\begin{bmatrix} 0 & 0 & 0 &\dots & 0 & -a_0 \\ 1 & 0 & 0&  \dots & 0  & -a_1 \\ 0 & 1 & 0&  \dots & 0  & -a_2 \\ \vdots & \vdots & \ddots & \quad  & \vdots\\ 0 & 0 & 0 & \dots & 1  & -a_{n-1} 
\end{bmatrix}.		
 \end{align*}
Deducir que dado un polinomio $p(x)$ siempre existe una matriz $A$ tal que $\chi_A(x)=p(x)$.

\

\item\label{complejos} $\textcircled{a}$ Sea $A$ una matriz $n \times n$ con coeficientes en $\mathbb C$. Probar que si $c_1,\dots,c_n \in \mathbb C$ son los autovalores de $A$
(posiblemente repetidos), entonces se cumple que:



\begin{enumerate}
	\item $\det(A)=c_1\cdots c_n$.
	
	\
	
	\item $\operatorname{Tr}(A)=c_1+\cdots+c_n$.
\end{enumerate}

\end{enumerate}

\subsection*{Ayudas}

\eqref{tr det} Evaluar el polinomio caracter\'istico en un valor apropiado para obtener el t\'ermino independiente. Desarrollar el determinante por la primera columna y hacer inducci\'on, prestando atenci\'on a la potencias de $x$.

\

\eqref{matriz de un polinomio} Desarrollar el determinante por la primera fila y hacer inducci\'on.

\

\eqref{complejos} Usar el Ejercicio \eqref{tr det}.
\end{document} 
