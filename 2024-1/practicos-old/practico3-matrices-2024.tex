% PDFLaTeX
\documentclass[a4paper,12pt,twoside,spanish,reqno]{amsbook}
%%%---------------------------------------------------
\usepackage[math]{kurier}

\usepackage{etex}
\usepackage{t1enc}
\usepackage{latexsym}
\usepackage[utf8]{inputenc}
\usepackage{verbatim}
\usepackage{multicol}
\usepackage{amsgen,amsmath,amstext,amsbsy,amsopn,amsfonts,amssymb}
\usepackage{amsthm}
\usepackage{calc}         % From LaTeX distribution
\usepackage{graphicx}     % From LaTeX distribution
\usepackage{ifthen}
\input{random.tex}   
\usepackage{tikz}
\usetikzlibrary{arrows}
\usetikzlibrary{matrix}
\usepackage{mathtools}
\usepackage{stackrel}
\usepackage{enumitem}
\usepackage{tkz-graph}

\usepackage{enumitem} 
\usepackage[compatibility=false]{caption} % para usar subcaption
\usepackage{subcaption} % para poner varias imagenes juntas
\usetikzlibrary{arrows.meta}
\usepackage{hyperref}
\hypersetup{ 
    colorlinks=true,
    linkcolor=blue,
    filecolor=magenta,      
    urlcolor=cyan,
}
\usepackage{hypcap}
\numberwithin{equation}{section}
% http://www.texnia.com/archive/enumitem.pdf (para las labels de los enumerate)
\renewcommand\labelitemi{$\circ$}
\setlist[enumerate, 1]{label={(\arabic*)}}
\setlist[enumerate, 2]{label=\emph{\alph*)}}


\newcommand{\rta}{\noindent\textsc{Solución: }} 

\newcommand{\img}{\operatorname{Im}}
\newcommand{\nuc}{\operatorname{Nu}}
\newcommand\im{\operatorname{Im}}
\renewcommand\nu{\operatorname{Nu}}
\newcommand{\la}{\langle}
\newcommand{\ra}{\rangle}
\renewcommand{\t}{{\operatorname{t}}}
\renewcommand{\sin}{{\,\operatorname{sen}}}
\newcommand{\Q}{\mathbb Q}
\newcommand{\R}{\mathbb R}
\newcommand{\C}{\mathbb C}
\newcommand{\K}{\mathbb K}
\newcommand{\F}{\mathbb F}
\newcommand{\Z}{\mathbb Z}
\newenvironment{amatrix}[1]{%
  \left[\begin{array}{@{}*{#1}{c}|c@{}}
}{%
  \end{array}\right]
}

%%% FORMATOS %%%%%%%%%%%%%%%%%%%%%%%%%%%%%%%%%%%%%%%%%%%%%%%%%%%%%%%%%%%%%%%%%%%%%
\tolerance=10000
\renewcommand{\baselinestretch}{1.3}
\usepackage[a4paper, top=3cm, left=3cm, right=2cm, bottom=2.5cm]{geometry}
\usepackage{setspace}
%\setlength{\parindent}{0,7cm}% tamaño de sangria.
\setlength{\parskip}{0,4cm} % separación entre parrafos.
\renewcommand{\baselinestretch}{0.90}% separacion del interlineado
%%%%%%%%%%%%%%%%%%%%%%%%%%%%%%%%%%%%%%%%%%%%%%%%%%%%%%%%%%%%%%%%%%%%%%%%%%%%%%%%%%%
%\end{comment}
%%% FIN FORMATOS  %%%%%%%%%%%%%%%%%%%%%%%%%%%%%%%%%%%%%%%%%%%%%%%%%%%%%%%%%%%%%%%%%

\begin{document}
    \baselineskip=0.55truecm %original
    
    
    {\bf \begin{center} Práctico 3 \\ Álgebra  II -- Año 2024/1 \\ FAMAF \end{center}}

%\title{Pr\'actico 1}


\centerline{\textsc{\'Algebra de matrices}}

\subsection*{Objetivos}

\begin{itemize}
 \item Familiarizarse con las matrices y sus operaciones de suma y multiplicaci\'on, Ejercicios \ref{ej} -- \ref{ej:multiplicar por columna}.
 \vskip .2cm
 \item Familiarizarse con la notaci\'on de sub\'indices para las entradas de matrices, Ejercicios \ref{ej:multiplicar por columna} y \ref{traza}.
 \vskip .2cm
 \item Aprender la noci\'on de matriz inversa y c\'omo c\'alcularla, Ejercicios \ref{ej:inversas} -- \ref{nilpotene - id}.
 \vskip .2cm
 \item Usar matrices para la resoluci\'on de sistemas de ecuaciones, Ejercicios \ref{sol homog es subesp} -- \ref{ej:sistemas ABX}.
\end{itemize}

\bigbreak

\subsection*{Ejercicios}

Los ejercicios con el símbolo $\textcircled{a}$ tienen una ayuda al final del archivo para que recurran a ella despu\'es de pensar un poco.

\begin{enumerate}[topsep=6pt,itemsep=.4cm]

%%%%%%%%%%%%%%%%%%%%%%%%%%%%%%%%%%%%%%%%%%%%%%%%%%%%%%%%%%%%%%%%%%%%%%%%%%%%%%%%%%%%%%%%%%%%%%%%%%%%%%%%%%%%%%%%%%%%%%%%%%%%%%%%%%%%%%%%%%%%%%%%%%%%%%%%

\item\label{ej} Sean
$$
A= \begin{bmatrix} 1&-2&0\\ 1&-2&1\\ 1&-2&-1\end{bmatrix},\quad
\quad B= \begin{bmatrix}1&1&2\\ -2&0&-1\\ 1&3&5 \end{bmatrix},
\quad\quad C=\begin{bmatrix}1&-1&1\\ 2&0&1\\ 3&0&1 \end{bmatrix}.
$$

Verificar que $A(BC)=(AB)C$, es decir que vale la asociatividad del producto.


\item\label{ej2} Determinar cu\'al de las siguientes matrices es $A$, cu\'al es $B$ y cu\'al es $C$ de modo tal que sea posible realizar el producto $ABC$ y verificar que $A(BC)=(AB)C$.
\begin{equation*}
\begin{bmatrix} 2 & -1 & 1 \\ 1 & 2 &
1\end{bmatrix},\qquad
\begin{bmatrix} 3 \\ 1 \\ -1\end{bmatrix}, \qquad
\begin{bmatrix} 1 & -1 \end{bmatrix}.
\end{equation*}


\item Calcular $A^2$ y $A^3$ para la matriz \
$
A=\begin{bmatrix}
3 & 4\\ 6 & 8
\end{bmatrix}.
$


\item\label{ejemplos 2x2} $\textcircled{a}$ Dar ejemplos de matrices no nulas $A$ y $B$ de orden $2\times2$ tales que
\begin{multicols}{2}
\begin{enumerate}[topsep=5pt,itemsep=5pt]
 \item $A^2=0$ (dar dos ejemplos).
 \item $AB\neq BA$.
 \item $A^2=-\operatorname{I}_2$.
 \item $A^2=A\neq\operatorname{I}_2$.
\end{enumerate}
\end{multicols}


\item\label{2x2 central} $\textcircled{a}$ Sea $A \in\mathbb{R}^{2\times 2}$ tal que $AB=BA$ para toda $B\in\mathbb{R}^{2\times 2}$. Probar que $A$ es un m\'ultiplo de $\operatorname{I}_2$.


\item  Para cada $n\in\mathbb{N}$, con $n\geq 2$, hallar una matriz no nula $A\in\mathbb{R}^{n\times n}$ tal que $A^n=0$ pero $A^{n-1}\neq0$.


\item\label{eq:binomio} $\textcircled{a}$ Dar condiciones necesarias y suficientes sobre matrices $A$ y $B$ de tama\~{n}o $n\times n$ para que
\begin{multicols}{2}
	\begin{enumerate}
		\item $(A + B)^2 = A^2 + 2AB + B^2$.
		\item $A^2 - B^2 = (A - B)(A + B)$.
	\end{enumerate}
\end{multicols}

%%%%%%%%%%%%%%%%%%%%%%%%%%%%%%%%%%%%%%%%%%%%%%%%%%%%%%%%%%%%%%%%%%%%%%%%%%%%%%%%%%%%%%%%%%%%%%%%%%%%%%%%%%%%%%%%%%%%%%%%%%%%%%%%%%%%%%%%%%%%%%%%%%%%%%%%

\item\label{ej:multiplicar por columna} $\textcircled{a}$ Sean
\begin{align*}
v=\begin{bmatrix} v_1 \\ \vdots \\ v_n
\end{bmatrix}\in\mathbb{R}^{n\times1}
\quad\mbox{y}\quad A=\begin{bmatrix} \mid& \mid& &\mid\\ C_1 & C_2 & \cdots &C_n\\ \mid& \mid& &\mid\end{bmatrix}
\in\mathbb{R}^{m\times n},
\end{align*}
es decir, $C_1, ..., C_n$ denotan las columnas de $A$. Probar que $Av=\sum_{j=1}^nv_jC_j$.


\item\label{traza} Si $A$ es una matriz cuadrada $n\times n$, se define la {\it \textbf{traza}} de $A$
como $\operatorname{Tr}(A)=\displaystyle{\sum_{i=1}^n} a_{ii}$.
\begin{enumerate}[topsep=5pt,itemsep=5pt]
 \item Calcular la traza de las matrices del ejercicio  \ref{ej:inversas}.
 \item\label{ej:traza}$\textcircled{a}$ Probar que si $A,B\in\mathbb{R}^{n\times n}$ y $c\in\mathbb{R}$ entonces
 \begin{align*}
 \operatorname{Tr}(A+cB)=\operatorname{Tr}(A)+c\operatorname{Tr}(B)
 \quad\mbox{y}\quad
 \operatorname{Tr}(AB)=\operatorname{Tr}(BA).
 \end{align*}
\end{enumerate}

%%%%%%%%%%%%%%%%%%%%%%%%%%%%%%%%%%%%%%%%%%%%%%%%%%%%%%%%%%%%%%%%%%%%%%%%%%%%%%%%%%%%%%%%%%%%%%%%%%%%%%%%%%%%%%%%%%%%%%%%%%%%%%%%%%%%%%%%%%%%%%%%%%%%%%%%


\item\label{ej:inversas} Para cada una de las siguientes matrices, usar operaciones elementales
por fila para decidir si son invertibles y hallar la matriz inversa cuando sea posible.
\begin{equation*}
\begin{bmatrix} 3 & -1 & 2 \\ 2 & 1 & 1 \\ 1 & -3 & 0\end{bmatrix},\qquad
\begin{bmatrix} -1 & -1 &4 \\ 1 & 3 & 8 \\ 1 & 2 & 5\end{bmatrix},\qquad
\begin{bmatrix} 1 & 1 & 1 & 2 \\ 1 & -3 & 3 & -8 \\ -2 & 1 & 2 & -2 \\ 1 & 2 & 1 & 4 \end{bmatrix},\qquad
\begin{bmatrix} 1 & -3 & 5 \\ 2 & -3 & 1 \\ 0 & -1 & 3 \end{bmatrix}.
\end{equation*}
(para que hagan menos cuentas: las matrices $3\times3$ aparecieron en el Pr\'actico 2).


\item Sea $A$ la primera matriz del ejercicio anterior.
Hallar matrices elementales $E_1,E_2,\dots,E_k$ tales que $E_kE_{k-1}\cdots E_2E_1A=\operatorname{I}_3$.


\item ?` Es cierto que si $A$ y $B$ son matrices invertibles entonces $A+B$ es una matriz invertible? Justificar su respuesta.

\item\label{nilpotene - id} $\textcircled{a}$ Una matriz $A\in\mathbb{R}^{n\times n}$ se dice \emph{nilpotente} si $A^k=0$ para alg\'un $k\in\mathbb{N}$.
Probar que si una matriz $A$ es nilpotente, entonces  $\operatorname{I}_n - A$  es invertible.

%%%%%%%%%%%%%%%%%%%%%%%%%%%%%%%%%%%%%%%%%%%%%%%%%%%%%%%%%%%%%%%%%%%%%%%%%%%%%%%%%%%%%%%%%%%%%%%%%%%%%%%%%%%%%%%%%%%%%%%%%%%%%%%%%%%%%%%%%%%%%%%%%%%%%%%%

\item\label{sol homog es subesp} Sean  $v$ y $w$ dos soluciones del sistema homog\'eneo $AX=0$. Probar que $v+tw$ tambi\'en es soluci\'on para todo $t\in\mathbb{K}$.

\item Sea $v$ una soluci\'on del sistema $AX=Y$ y $w$ una soluci\'on del sistema homog\'eneo $AX=0$. Probar que $v+tw$ tambi\'en es soluci\'on del sistema $AX=Y$ para todo $t\in\mathbb{K}$.

\item Probar que si el sistema homog\'eneo  $AX=0$ posee alguna soluci\'on no trivial, entonces el sistema $AX=Y$ no tiene
soluci\'on o tiene al menos dos soluciones distintas.

\item Supongamos que los sistemas $AX=Y$ y $AX=Z$ tienen soluci\'on. Probar que el sistema $AX=Y+tZ$ tambi\'en tiene soluci\'on para todo $t\in\mathbb{K}$.

\item Sean $A$ una matriz invertible $n\times n$, y $B$ una matriz $n\times m$.  Probar que los sistemas $BX=Y$ y $ABX=AY$ tienen las mismas soluciones.

\item\label{ej:sistemas ABX} $\textcircled{a}$
Sean $A$ y $B$ matrices $r\times n$ y $n\times m$ respectivamente.
Probar que:
\begin{enumerate}[topsep=5pt,itemsep=5pt]
	\item  Si $m>n$, entonces el sistema $ABX=0$ tiene soluciones no triviales.
	\item  Si $r>n$, entonces existe un $Y$, $r\times 1$, tal que $ABX=Y$
	no tiene soluci\'on.
\end{enumerate}

%%%%%%%%%%%%%%%%%%%%%%%%%%%%%%%%%%%%%%%%%%%%%%%%%%%%%%%%%%%%%%%%%%%%%%%%%%%%%%%%%%%%%%%%%%%%%%%%%%%%%%%%%%%%%%%%%%%%%%%%%%%%%%%%%%%%%%%%%%%%%%%%%%%%%%%%

\end{enumerate}




\subsection*{Ejercicios de repaso}
Si ya hizo los ejercicios anteriores continue con la siguiente gu\'ia. Los ejercicios que siguen son similares y le pueden servir para practicar antes de los ex\'amenes.



\begin{enumerate}[resume, topsep=6pt, itemsep=.4cm]


\item\label{ej: distributiva} $\textcircled{a}$ Probar que si $A\in\mathbb{R}^{m\times n}$ y $B,C\in\mathbb{R}^{n\times p}$ entonces
$A(B+ C)=AB + AC$.

\item Probar que si $A,B\in\mathbb{R}^{m\times n}$ y $C\in\mathbb{R}^{n\times p}$ entonces
$(A+B)C = AC + BC$.

\item Sea $v=[v_1 \cdots v_m]\in\mathbb{R}^{1\times m}$ y $A\in\mathbb{R}^{m\times n}$. Probar que $vA=\sum_{i=1}^m v_iF_i$, donde $F_1, ..., F_m$ denotan las filas de $A$.

\item Sea $D=(d_{ij})\in\mathbb{R}^{n\times n}$ una matriz diagonal y $A=(a_{ij})\in\mathbb{R}^{m\times n}$. Probar que \\ $AD=(d_{jj}a_{ij})\in\mathbb{R}^{m\times n}$.

\item
Probar las siguientes afirmaciones:
\begin{enumerate}[topsep=5pt,itemsep=5pt]
\item Si $A,B\in\mathbb{R}^{n\times n}$ son matrices diagonales, entonces $AB=BA$.
\item Si $A=c \operatorname{I}_n$ para alg\'un $c \in \mathbb{R}$, entonces $AB=BA$ para toda $B\in\mathbb{R}^{n\times n}$.
\end{enumerate}

\item Probar que si $A\in\mathbb{R}^{n\times n}$ es una matriz diagonal tal que $\operatorname{Tr} (A^2)=0$, entonces $A=0$.

\item Sea $A$ matriz $2\times 2$  tal que $\operatorname{Tr}(A)=0$ y $\operatorname{Tr}(A^2)=0$.

\begin{enumerate}[topsep=5pt,itemsep=5pt]
	\item  Probar que $A^2 = 0$.
	\item   >Es cierta la rec\'iproca?
	%	\item  >Puede generalizar el resultado?
\end{enumerate}


\item Probar que si $A$ y $B$ son matrices $n \times n$ \emph{que conmutan entre s\' \i}, entonces para todo $k \in \mathbb N \cup \{0\}$ se cumple que:
$$(A+B)^k = \sum_{j = 0}^k \binom{k}{j} \, A^j \, B^{k-j}.$$

\item Sea $A\in\mathbb{R}^{m\times n}$. La \emph{matriz traspuesta} de $A$ es la matriz $A^t\in\mathbb{R}^{n\times m}$ tal que
$(A^t)_{ij}=A_{ji}$, $1\le i\le n$, $1\le j\le m$.
\begin{enumerate}[topsep=5pt,itemsep=5pt]
 \item Dar las matrices traspuestas de las matrices $A$, $B$ y $C$ de los ejercicios \ref{ej} y \ref{ej2}.
 \item Probar que si $A,B\in\mathbb{R}^{m\times n}$, $C\in\mathbb{R}^{n\times p}$ y $c\in\mathbb{R}$ entonces
$$(A+cB)^t=A^{t} + c B^t, \quad\quad (BC)^t= C^t B^t.$$

\item Probar que si $D\in\mathbb{R}^{n\times n}$ es invertible, entonces $D^t$ tambi\'en lo es y $(D^{t})^{-1}=(D^{-1})^{t}$.
\end{enumerate}

\item Una matriz $A$ se dice {\it sim\'etrica} si $A^t=A$. Una matriz $B$ se dice {\it antisim\'etrica} si $B^t=-B$. Probar que toda matriz se puede expresar como la suma de una matriz sim\'etrica y una antisim\'etrica.

\item Decidir si las siguientes afirmaciones son verdaderas o falsas. Justificar.
\begin{enumerate}[topsep=5pt,itemsep=5pt]

\item Si $A$ y $B$ son matrices cuadradas tales $AB=BA$ pero ninguna es m\'ultiplo de la otra, entonces $A$ o $B$ es diagonal.

\item Existen una matriz $3\times 2$, $A$,  y una matriz $2\times 3$, $B$, tales que $AB$ es una matriz invertible.

\item Existen una matriz $2\times 3$, $A$,  y una matriz $3\times 2$, $B$, tales que $AB$ es una matriz invertible.

\end{enumerate}

\end{enumerate}



\subsection*{Ayudas}

\vskip .4cm\

\ref{ejemplos 2x2} Probar con algunos $0$ y $1$ en las entradas.

\vskip .4cm

\ref{2x2 central} Elegir matrices $B$ apropiadas con muchos ceros y \textbf{un} $1$.

\vskip .4cm

\ref{eq:binomio} El objetivo del ejercicio es completar los puntos suspensivos en la siguiente frase:


``$(A+B)^2=A^2+2AB+B^2$ si y s\'olo si $A$ y $B$ satisfacen ..... ''

Desarrollen el cuadrado de la suma $A+B$ usando que el producto de matrices es distributivo y vean que les ``sobra'' para obtener la f\'ormula del binomio.  Misma idea para el item (b).

\vskip .4cm

\ref{ej:multiplicar por columna} Usar la notaci\'on de sub\'indices para las entradas de matrices.

\vskip .4cm

\ref{traza}\,\ref{ej:traza} Usar la notaci\'on de sub\'indices para las entradas de matrices.

\vskip .4cm

\ref{nilpotene - id} Pensar en la f\'ormula de $\sum_{i=0}^na^i$ vista en \'Algebra I/Matem\'atica Discreta I.

\vskip .4cm

\ref{ej:sistemas ABX} Recordar el Ejercicio 11 del Pr\'actico 2.

\vskip .4cm

\ref{ej: distributiva} Usar la notaci\'on de sub\'indices para las entradas de matrices.

\end{document}
