%\documentclass{beamer} % descomentar para tener pausas
\documentclass[handout]{beamer} % descomentar para no tener pausas
\usetheme{CambridgeUS}
%\setbeamertemplate{background}[grid][step=8 ] % cuadriculado

\usepackage{etex}
\usepackage{t1enc}
\usepackage[spanish,es-nodecimaldot]{babel}
\usepackage{latexsym}
\usepackage[utf8]{inputenc}
\usepackage{verbatim}
\usepackage{multicol}
\usepackage{amsgen,amsmath,amstext,amsbsy,amsopn,amsfonts,amssymb}
\usepackage{amsthm}
\usepackage{calc}         % From LaTeX distribution
\usepackage{graphicx}     % From LaTeX distribution
\usepackage{ifthen}
%\usepackage{makeidx}
\input{random.tex}        % From CTAN/macros/generic
\usepackage{subfigure} 
\usepackage{tikz}
\usepackage[customcolors]{hf-tikz}
\usetikzlibrary{arrows}
\usetikzlibrary{matrix}
\tikzset{
	every picture/.append style={
		execute at begin picture={\deactivatequoting},
		execute at end picture={\activatequoting}
	}
}
\usetikzlibrary{decorations.pathreplacing,angles,quotes}
\usetikzlibrary{shapes.geometric}
\usepackage{mathtools}
\usepackage{stackrel}
%\usepackage{enumerate}
\usepackage{enumitem}
\usepackage{tkz-graph}
\usepackage{polynom}
\polyset{%
	style=B,
	delims={(}{)},
	div=:
}
\renewcommand\labelitemi{$\circ$}
\usepackage{nicematrix} %https://ctan.dcc.uchile.cl/macros/latex/contrib/nicematrix/nicematrix.pdf

%\setbeamertemplate{background}[grid][step=8 ]
\setbeamertemplate{itemize item}{$\circ$}
\setbeamertemplate{enumerate items}[default]
\definecolor{links}{HTML}{2A1B81}
\hypersetup{colorlinks,linkcolor=,urlcolor=links}


\newcommand{\Id}{\operatorname{Id}}
\newcommand{\img}{\operatorname{Im}}
\newcommand{\nuc}{\operatorname{Nu}}
\newcommand{\im}{\operatorname{Im}}
\renewcommand\nu{\operatorname{Nu}}
\newcommand{\la}{\langle}
\newcommand{\ra}{\rangle}
\renewcommand{\t}{{\operatorname{t}}}
\renewcommand{\sin}{{\,\operatorname{sen}}}
\newcommand{\Q}{\mathbb Q}
\newcommand{\R}{\mathbb R}
\newcommand{\C}{\mathbb C}
\newcommand{\K}{\mathbb K}
\newcommand{\F}{\mathbb F}
\newcommand{\Z}{\mathbb Z}
\newcommand{\N}{\mathbb N}
\newcommand\sgn{\operatorname{sgn}}
\renewcommand{\t}{{\operatorname{t}}}
\renewcommand{\figurename }{Figura}

%
% Ver http://joshua.smcvt.edu/latex2e/_005cnewenvironment-_0026-_005crenewenvironment.html
%

\renewenvironment{block}[1]% environment name
{% begin code
	\par\vskip .2cm%
	{\color{blue}#1}%
	\vskip .2cm
}%
{%
	\vskip .2cm}% end code


\renewenvironment{alertblock}[1]% environment name
{% begin code
	\par\vskip .2cm%
	{\color{red!80!black}#1}%
	\vskip .2cm
}%
{%
	\vskip .2cm}% end code


\renewenvironment{exampleblock}[1]% environment name
{% begin code
	\par\vskip .2cm%
	{\color{blue}#1}%
	\vskip .2cm
}%
{%
	\vskip .2cm}% end code




\newenvironment{exercise}[1]% environment name
{% begin code
	\par\vspace{\baselineskip}\noindent
	\textbf{Ejercicio (#1)}\begin{itshape}%
		\par\vspace{\baselineskip}\noindent\ignorespaces
	}%
	{% end code
	\end{itshape}\ignorespacesafterend
}


\newenvironment{definicion}[1][]% environment name
{% begin code
	\par\vskip .2cm%
	{\color{blue}Definición #1}%
	\vskip .2cm
}%
{%
	\vskip .2cm}% end code

\newenvironment{observacion}[1][]% environment name
{% begin code
	\par\vskip .2cm%
	{\color{blue}Observación #1}%
	\vskip .2cm
}%
{%
	\vskip .2cm}% end code

\newenvironment{ejemplo}[1][]% environment name
{% begin code
	\par\vskip .2cm%
	{\color{blue}Ejemplo #1}%
	\vskip .2cm
}%
{%
	\vskip .2cm}% end code

\newenvironment{ejercicio}[1][]% environment name
{% begin code
	\par\vskip .2cm%
	{\color{blue}Ejercicio #1}%
	\vskip .2cm
}%
{%
	\vskip .2cm}% end code


\renewenvironment{proof}% environment name
{% begin code
	\par\vskip .2cm%
	{\color{blue}Demostración}%
	\vskip .2cm
}%
{%
	\vskip .2cm}% end code



\newenvironment{demostracion}% environment name
{% begin code
	\par\vskip .2cm%
	{\color{blue}Demostración}%
	\vskip .2cm
}%
{%
	\vskip .2cm}% end code

\newenvironment{idea}% environment name
{% begin code
	\par\vskip .2cm%
	{\color{blue}Idea de la demostración}%
	\vskip .2cm
}%
{%
	\vskip .2cm}% end code

\newenvironment{solucion}% environment name
{% begin code
	\par\vskip .2cm%
	{\color{blue}Solución}%
	\vskip .2cm
}%
{%
	\vskip .2cm}% end code



\newenvironment{lema}[1][]% environment name
{% begin code
	\par\vskip .2cm%
	{\color{blue}Lema #1}\begin{itshape}%
		\par\vskip .2cm
	}%
	{% end code
	\end{itshape}\vskip .2cm\ignorespacesafterend
}

\newenvironment{proposicion}[1][]% environment name
{% begin code
	\par\vskip .2cm%
	{\color{blue}Proposición #1}\begin{itshape}%
		\par\vskip .2cm
	}%
	{% end code
	\end{itshape}\vskip .2cm\ignorespacesafterend
}

\newenvironment{teorema}[1][]% environment name
{% begin code
	\par\vskip .2cm%
	{\color{blue}Teorema #1}\begin{itshape}%
		\par\vskip .2cm
	}%
	{% end code
	\end{itshape}\vskip .2cm\ignorespacesafterend
}


\newenvironment{corolario}[1][]% environment name
{% begin code
	\par\vskip .2cm%
	{\color{blue}Corolario #1}\begin{itshape}%
		\par\vskip .2cm
	}%
	{% end code
	\end{itshape}\vskip .2cm\ignorespacesafterend
}

\newenvironment{propiedad}% environment name
{% begin code
	\par\vskip .2cm%
	{\color{blue}Propiedad}\begin{itshape}%
		\par\vskip .2cm
	}%
	{% end code
	\end{itshape}\vskip .2cm\ignorespacesafterend
}

\newenvironment{conclusion}% environment name
{% begin code
	\par\vskip .2cm%
	{\color{blue}Conclusión}\begin{itshape}%
		\par\vskip .2cm
	}%
	{% end code
	\end{itshape}\vskip .2cm\ignorespacesafterend
}







\newenvironment{definicion*}% environment name
{% begin code
	\par\vskip .2cm%
	{\color{blue}Definición}%
	\vskip .2cm
}%
{%
	\vskip .2cm}% end code

\newenvironment{observacion*}% environment name
{% begin code
	\par\vskip .2cm%
	{\color{blue}Observación}%
	\vskip .2cm
}%
{%
	\vskip .2cm}% end code


\newenvironment{obs*}% environment name
	{% begin code
		\par\vskip .2cm%
		{\color{blue}Observación}%
		\vskip .2cm
	}%
	{%
		\vskip .2cm}% end code

\newenvironment{ejemplo*}% environment name
{% begin code
	\par\vskip .2cm%
	{\color{blue}Ejemplo}%
	\vskip .2cm
}%
{%
	\vskip .2cm}% end code

\newenvironment{ejercicio*}% environment name
{% begin code
	\par\vskip .2cm%
	{\color{blue}Ejercicio}%
	\vskip .2cm
}%
{%
	\vskip .2cm}% end code

\newenvironment{propiedad*}% environment name
{% begin code
	\par\vskip .2cm%
	{\color{blue}Propiedad}\begin{itshape}%
		\par\vskip .2cm
	}%
	{% end code
	\end{itshape}\vskip .2cm\ignorespacesafterend
}

\newenvironment{conclusion*}% environment name
{% begin code
	\par\vskip .2cm%
	{\color{blue}Conclusión}\begin{itshape}%
		\par\vskip .2cm
	}%
	{% end code
	\end{itshape}\vskip .2cm\ignorespacesafterend
}








\title[Clase 14 - Dimensión. Subespacios]{Álgebra/Álgebra II \\ Clase 14- Dimensión. Subespacios}

\author[]{}
\institute[]{\normalsize FAMAF / UNC
	\\[\baselineskip] ${}^{}$
	\\[\baselineskip]
}
\date[14/05/2024]{14 de mayo de 2024}



\begin{document}

\begin{frame}
\maketitle
\end{frame}




    
    \begin{frame}
        
    Recordemos este importante resultado de la clase anterior:
    \vskip.6cm
\pause
    Sea $V$ un espacio vectorial  y $T \subset V$, finito tal que  $\la T \ra = V$. Sea $S \subset V$.
    \vskip.4cm
    Entonces 
        \begin{equation*} \label{eq-1}
            \la T \ra = V, \quad  S \text{ es LI } \Rightarrow |S| \le |T|. \tag{P1}
        \end{equation*}
        
        
        \vskip.4cm

        El contrarrecíproco también nos resultará de utilidad

        \begin{equation*}\label{eq-2}
            \la T \ra = V, \quad  |S| > |T| \Rightarrow S \text{ es LD}.\tag{P2}
        \end{equation*}





    \end{frame} 
                    



\begin{frame}{Dimensiones de subespacios}
\pause
\begin{itemize}
    \item Si $A$ matriz $m \times n$, en donces $W = \{x: Ax=0\}$ es un subespacio.
    \vskip .6cm\pause
    \item ¿Cuál es la dimensión de $W$? ¿Qué relación tiene con $R$, la MRF equivalente a $A$?
    \vskip .6cm\pause
    \item Veremos que si $r$  es la cantidad  de filas no  nulas de $R$, entonces $\dim(W) = n-r$. 
\end{itemize}

\vskip 3cm

\end{frame}


\begin{frame}
        
\begin{ejemplo}
    Encontrar una base del subespacio 
    $$
    W = \left\{(x,y,z,w) \in \mathbb{R}: \quad\begin{array}{rcl}
    x-y -3z +\;\;w &=& 0 \\ y +5z +3w &=& 0
    \end{array} \right\}.
    $$
\end{ejemplo}\pause
\begin{solucion}\pause
    $W$  está definido implícitamente y usando el método de Gauss podemos describirlo paramétricamente, pues:
    \begin{equation*}
    \begin{bmatrix}1&-1&-3&1 \\ 0&1&5&3  \end{bmatrix}
    \stackrel{F_1+F_2}{\longrightarrow} 
    \begin{bmatrix}1&0&2&4 \\ 0&1&5&3  \end{bmatrix}.
    \end{equation*}
\end{solucion}
\end{frame}


\begin{frame}
Por lo tanto, el sistema de ecuaciones que define $W$ es equivalente a 
\begin{equation*}
\begin{array}{rcl}
x  +2z +4w &=& 0 \\ y +5z +3w &=& 0,
\end{array}
\end{equation*}
es decir 
\begin{equation*}
\begin{array}{rcl}
x  &=& -2z - 4w  \\ y &=& -5z -3w ,
\end{array}
\end{equation*}
y entonces
\begin{align*}
W &= \left\{(-2z -4w,-5z -3w,z,w) : z,w\in \mathbb{R} \right\} \\
&= \left\{(-2,-5,1,0)z+(-4, -3,0,1)w : z,w\in \mathbb{R} \right\}\\
&= \langle (-2,-5,1,0),(-4, -3,0,1)\rangle.
\end{align*}
Concluimos entonces que $(-2,-5,1,0),(-4, -3,0,1)$  es una base de $W$ y, por lo tanto,  su dimensión es 2.\qed
\end{frame}

\begin{frame}
\begin{proposicion}
Sea $A$ matriz $m \times n$ y sea $W = \{x: Ax=0\}$. 
\vskip .2cm
Sea $R$ una MRF equivalentes por filas a $A$ y  sea $r$ la cantidad de filas no nulas de $R$. 
\vskip .2cm
Entonces $\dim(W) = n - r$.
\end{proposicion}

\begin{demostracion} Es posible hacer este demostración con las herramientas actuales. Sin embargo,  haremos una demostración mucho más conceptual de  este hecho  cuando veamos transformaciones lineales. 
    \qed   
    
\end{demostracion}

\end{frame}


\begin{frame}
\begin{definicion}\label{espacio-fila-columna}  Sea $A = [a_{ij}] \in M_{m \times n}(\K)$.
    \vskip .4cm\pause
    \begin{itemize}
        \item  El \textit{vector fila $i$} es el vector  $(a_{i1},\ldots,a_{in}) \in \K^n$.
        \vskip .4cm\pause
        \item  El \textit{espacio fila}\index{matriz!espacio fila} de $A$ es el subespacio de $\K^n$ generado por los $m$ vectores fila de $A$. \pause
        \vskip .4cm
        \item El vector columna $j$ es el vector $(a_{1j},\ldots,a_{mj}) \in \K^m$.
        \vskip .4cm\pause
        \item El \textit{espacio columna}\index{matriz!espacio columna} de $A$ es el subespacio de $\K^m$ generado por los $n$ vectores columna de $A$
    \end{itemize}
\end{definicion}

\end{frame}


\begin{frame}

\begin{ejemplo*}
    Sea
    $$
    A = \begin{bmatrix}
    1&2&0&3&0\\ 0&0&1&4&0 \\0&0&0&0&1
    \end{bmatrix}.
    $$
    El vector fila 1 es $(1,2,0,3,0)$, el vector columna 4 es $(3,4,0)$,  etc.
\vskip .3cm
    Sea $W$ el espacio fila de $A$.  entonces

    \begin{equation*}
        W = \la (1,2,0,3,0), (0,0,1,4,0), (0,0,0,0,1)\ra
    \end{equation*}

    Sea $U$  el espacio columna de $A$. Entonces:

    \begin{equation*}
        U = \la (1,0,0), (2,0,0), (0,1,0), (3,4,0), (0,0,1)\ra = \R^3.
    \end{equation*}
\end{ejemplo*}
\end{frame}



\begin{frame}
        
\begin{teorema}
    Sean $A$ matriz $m \times n$ con coeficientes en $\K$, $P$ matriz $m\times m$ invertible y $B =PA$. Entonces el el espacio fila de $A$ es igual al espacio fila de $B$.
\end{teorema}\pause
\begin{proof}\pause

    Sea $W_1$ espacio fila de $A$ y $W_2$ espacio fila de $B$.
    \vskip .2cm 

    Sea $A= [a_{ij}]$, $P =[p_{ij}]$ y $B = [b_{ij}]$. Como  $B= PA$, tenemos que la fila $i$ de $B$ es
    \begin{align*}
        (b_{i1},\ldots,b_{in})&= (F_i(P).C_1(A),\ldots,F_i(P).C_n(A)) \\
        &= (\sum_{j=1}^{m} p_{ij}a_{j1}, \ldots, \sum_{j=1}^{m} p_{ij}a_{jn}) \\
        &= \sum_{j=1}^{m} p_{ij}(a_{j1}, \ldots,a_{jn}). \tag{*}
    \end{align*}

\end{proof}
\end{frame}


\begin{frame}




\begin{itemize}
    \item por (*) cada vector fila de $B$ se puede obtener como combinación lineal de los vectores fila de $A$.
    \vskip .2cm 
    \item Por lo tanto el espacio fila de $B$ está incluido en el espacio fila de $A$: $W_2 \subset W_1$.
    \vskip .4cm 
    \item $P$ invertible $\Rightarrow$ $\exists P^{-1}$.
    \vskip .2cm 
    \item $P^{-1}B = P^{-1}P A = A$.
    \vskip .2cm 
    \item Un razonamiento análogo al (*) de la página anterior $\Rightarrow$ espacio fila de $A$ está incluido en el espacio fila de $B$: $W_1 \subset W_2$. 
\end{itemize}

$$
W_2 \subset W_1 \quad \wedge \quad W_1 \subset W_2 \qquad \Rightarrow \qquad W_1 = W_2.  \qed
$$
\end{frame}


\begin{frame}

\begin{corolario}\label{subesp-merf}
    Sean $A$ matriz $m \times n$ y $R$ la MRF equivalente por filas a $A$. Entonces, 
    \begin{enumerate}
        \item  el espacio fila de $A$ es igual al espacio fila de $R$,
        \item las filas no nulas de $R$ forman una base del espacio fila de $A$. 
    \end{enumerate}
\end{corolario}\pause
\begin{proof}\pause

(1)   $R$ la MRF equivalente por filas a $A$ $\Rightarrow$  $R=PA$ con  $P$ invertible $\stackrel{Teor. ant.}{\Rightarrow}$ espacio fila de $A$ $=$ espacio fila de $B$. 

\vskip .4cm
(2) $R$ es MRF $\Rightarrow$  cada fila no nula comienza con un 1 y en esa coordenada  todas las demás filas tienen un 0 $\Rightarrow$ las filas no nulas de $R$ son LI $\Rightarrow$ las filas no nulas de $R$ son base.

\qed

\end{proof}

\end{frame}


\begin{frame}

\begin{corolario}\label{inv-impl-filasgen}
    Sean $A$ matriz $n \times n$. Entonces, $A$ es invertible si y sólo si las filas de $A$ son una base de $\K^n$.
\end{corolario}\pause
\begin{proof}\pause
Si $A$ es invertible entonces la MERF de $A$ es la identidad, por lo tanto  el espacio fila de $A$ genera $\K^n$.
\vskip .4cm
Por otro lado, si el espacio fila de $A$  genera $\K^n$, el espacio fila de  la  MERF es $\K^n$ y por lo tanto  la MERF de $A$ es la identidad y en consecuencia $A$ es invertible.
\vskip .4cm
Hemos probado que $A$ es invertible si y sólo si las $n$ filas de $A$ generan $\K^n$. 
\vskip .4cm
Como $\dim \K^n = n$,  todo conjunto de $n$ generadores es una base. \qed
\end{proof}

\end{frame}


\begin{frame}{Bases de subespacios}

El corolario de la p. \ref{subesp-merf} nos permite encontrar fácilmente la dimensión de un subespacio de $\K^n$ generado explícitamente por $m$ vectores.\vskip .2cm\pause
\begin{itemize}
\item Sea $v_1,\ldots,v_m \in \K^n$ y $W = \langle v_1,\ldots,v_m\rangle$,\pause
\item Consideramos la matriz 
$$
A = \begin{bmatrix}
v_1 \\ v_2 \\ \vdots \\ v_m
\end{bmatrix}
$$\pause
\item Calculamos $R$, una MRF equivalente por filas a $A$. \pause
\item $W$ $=$ espacio fila de $R$.\pause
\item Si $R$ tiene $r$ filas no nulas, las $r$ filas no nulas son una base de $W$.\pause
\item   Por consiguiente, $\dim W = r$. 
\end{itemize}


\end{frame}


\begin{frame}
\begin{ejemplo*}\label{ej-4.5}
    Encontrar una base  de $W= \langle (1,0,1), (1,-1,0), (5,-3,2)\rangle$. 
\end{ejemplo*}\pause
\begin{solucion}\pause
    Formemos la matriz cuyas filas son los vectores que generan $W$,  es decir 
    $$
    A = \begin{bmatrix} 1&0&1 \\ 1&-1&0 \\ 5&-3&2 \end{bmatrix}.
    $$
    Entonces
    \begin{equation*}
    \begin{bmatrix}1&0&1 \\ 1&-1&0 \\ 5&-3&2  \end{bmatrix}
    \underset{F_3-5F_1}{\stackrel{F_2- F_1}{\longrightarrow}} 
    \begin{bmatrix}1&0&1 \\ 0&-1&-1 \\ 0&-3&-3\end{bmatrix}
    \stackrel{-F_2}{\longrightarrow} 
    \begin{bmatrix}1&0&1 \\ 0&1&1 \\ 0&-3&-3\end{bmatrix}
    \stackrel{F_3 - 3F_2}{\longrightarrow}
    \begin{bmatrix}1&0&1 \\ 0&1&1 \\ 0&0&0\end{bmatrix}.
    \end{equation*}
    Por lo tanto, $\dim W =2$ y $(1,0,1), (0,1,1)$ es una base de  $W$. \qed
\end{solucion}
\end{frame}


\begin{frame}{Subconjuntos LI de un sistema de generadores}
\pause
\begin{itemize}
    \item Dada un conjunto de generadores de un subespacio $W$ de $\K^n$ ``sabemos'' encontrar una base de $W$.\vskip .4cm\pause
    \item Esa base de $W$,  en general, utiliza otros  vectores (no  necesariamente los generadores).\vskip .4cm\pause
    \item Veremos a continuación que dado $S= \{v_1, \ldots, v_m\}$ y $W= \la S \ra$, podemos encontrar fácilmente un subconjunto de $S$ base de $W$.    \vskip .4cm
\end{itemize}
\vskip 2cm
\end{frame}


\begin{frame}
\begin{teorema}
    Sea $v_1,\ldots, v_r$ vectores en $\K^n$ y $W = \langle  v_1,\ldots, v_r \rangle$. 
    \vskip .4cm
    Sea $A$ la matriz formada por las filas $v_1,\ldots, v_r$ y $R$ una MRF equivalente por filas a $A$ que se obtiene \textbf{sin} el uso de permutaciones de filas. 
    \vskip .4cm
    Si $i_1,i_2,\ldots,i_s$  filas no nulas de $R$ $\Rightarrow$ $v_{i_1},v_{i_2},\ldots,v_{i_s}$  base de $W$.
\end{teorema}\pause

\begin{proof}\pause
    Se hará por inducción sobre $r$. 
    \vskip .2cm
    Si $r = 1$ es trivial ver que vale la afirmación. 
    \vskip .2cm
    
    Supongamos que tenemos el resultado probado para $r-1$ (hipótesis inductiva).
    


\end{proof}
\end{frame}


\begin{frame}
        
Sea $W' = \langle  v_1,\ldots, v_{r-1} \rangle$ y sea $A'$ la matriz formada por las $r-1$ filas $v_1,\ldots, v_{r-1}$. 		
Sea $R'$ la MRF equivalente por filas a $A'$ que se obtiene sin usar permutaciones de filas. Por hipótesis inductiva, si $i_1,i_2,\ldots,i_s$ son las filas no nulas de $R'$,  entonces $v_{i_1},v_{i_2},\ldots,v_{i_s}$ es una base de $W'$.

Sea
\begin{equation*}
    R_0 = \begin{bmatrix}
    R' \\ v_r
    \end{bmatrix}.
\end{equation*}
Si $v_r \in W'$, entonces  $v_{i_1},v_{i_2},\ldots,v_{i_s}$ es una base de $W$ y 
\begin{equation*}
R = \begin{bmatrix}
R' \\ 0
\end{bmatrix}
\end{equation*}
es la MRF de $A$.

Si $v_r \not\in W'$, entonces  $v_{i_1},v_{i_2},\ldots,v_{i_s}, v_r$ es una base de $W$  y la MRF de $A$ tiene la última fila no nula.  \qed

\end{frame}


\begin{frame}
Finalmente, terminaremos la clase  con un teorema que resume algunas equivalencias respecto a matrices invertibles.

\begin{teorema}
Sea $A$ matriz $n \times n$ con coeficientes en $\K$. Entonces son equivalentes\pause
\begin{enumerate} 
    \item $A$ es invertible.\pause
    \item $A$  es equivalente por filas a $\Id_n$.\pause
    \item $A$ es producto de matrices elementales.\pause
    \item El sistema $AX=Y$ tiene una única solución para toda matriz $Y$ de orden $n \times 1$. \pause
    \item El sistema homogéneo $AX=0$ tiene una única solución trivial.\pause
    \item $\det A \ne 0$.\pause
    \item Las filas de $A$ son LI.\pause
    \item Las columnas de $A$ son LI.
\end{enumerate}
\end{teorema}
\end{frame}



\end{document}



\begin{frame}

\begin{definicion}
    Si $V$ es un espacio vectorial de dimensión finita, una \textit{base ordenada}\index{base ordenada} de $V$ es una sucesión finita de vectores linealmente independiente y que genera $V$.
\end{definicion}
\vskip .4cm
\pause
La diferencia entre la definición de ``base'' y la de ``base ordenada'',  es que en la última es  importante el orden de los vectores de la base. 
\vskip .6cm\pause
Se incurrirá en un pequeño abuso de notación y se escribirá
$$
\mathcal{B} = \{v_1,\ldots,v_n\}
$$
diciendo que $\mathcal{B}$ es una base ordenada de $V$.
\vskip 2cm

\end{frame}


\begin{frame}
\begin{proposicion}
    Sea $V$  espacio vectorial de dimensión finita y sea $\mathcal{B} = \{v_1,\ldots,v_n\}$ una base ordenada de $V$. Entonces, para cada $v \in V$,  existen únicos $x_1,\ldots,x_n \in \K$ tales que $$v =   x_1v_1 + \cdots +x_nv_n.$$
\end{proposicion}\pause
\begin{proof}\pause
    Como $v_1,\ldots,v_n$  generan $V$,  es claro que existen $x_1,\ldots,x_n \in \K$ tales que 
    $$v =   x_1v_1 + \cdots +x_nv_n.$$
    Sean $y_1,\ldots,y_n \in \K$ tales que $$v =   y_1v_1 + \cdots +y_nv_n.$$
    
    
    
\end{proof}


\end{frame}


\begin{frame}

Veremos que $x_i = y_i$ para $1 \le i \le n$.

\vskip .4cm

Como $v =  \sum_{i=1}^{n} x_iv_i$ y $v =  \sum_{i=1}^{n} y_iv_i$,  restando miembro a miembro obtenemos 
$$
0 =   \sum_{i=1}^{n} (x_i-y_i)v_i.
$$
Ahora bien,  $v_1,\ldots,v_n$ son  LI, por lo tanto todos los coeficientes de la ecuación anterior son nulos.
\vskip .2cm
Es decir $x_i-y_i=0$ para $1 \le i \le n$.
\vskip .2cm
Entonces $x_i = y_i$ para $1 \le i \le n$.

\qed \vskip 2cm
\end{frame}
            


\end{document}

