%\documentclass{beamer} 
\documentclass[handout]{beamer} % sin pausas
\usetheme{CambridgeUS}

\usepackage{etex}
\usepackage{t1enc}
\usepackage[spanish,es-nodecimaldot]{babel}
\usepackage{latexsym}
\usepackage[utf8]{inputenc}
\usepackage{verbatim}
\usepackage{multicol}
\usepackage{amsgen,amsmath,amstext,amsbsy,amsopn,amsfonts,amssymb}
\usepackage{amsthm}
\usepackage{calc}         
\usepackage{graphicx}     
\usepackage{ifthen}
%\usepackage{makeidx}
\input{random.tex}        
\usepackage{subfigure} 
\usepackage{tikz}
\usepackage[customcolors]{hf-tikz}
\usetikzlibrary{arrows}
\usetikzlibrary{matrix}
\tikzset{
    every picture/.append style={
        execute at begin picture={\deactivatequoting},
        execute at end picture={\activatequoting}
    }
}
\usetikzlibrary{decorations.pathreplacing,angles,quotes}
\usetikzlibrary{shapes.geometric}
\usepackage{mathtools}
\usepackage{stackrel}
\usepackage{enumitem}
\usepackage{tkz-graph}
\usepackage{polynom}
\polyset{%
    style=B,
    delims={(}{)},
    div=:
}
\renewcommand\labelitemi{$\circ$}

%\setbeamertemplate{background}[grid][step=8 ]
\setbeamertemplate{itemize item}{$\circ$}
\setbeamertemplate{enumerate items}[default]
\definecolor{links}{HTML}{2A1B81}
\hypersetup{colorlinks,linkcolor=,urlcolor=links}

%%
% Ver http://joshua.smcvt.edu/latex2e/_005cnewenvironment-_0026-_005crenewenvironment.html
%

\renewenvironment{block}[1]% environment name
{% begin code
	\par\vskip .2cm%
	{\color{blue}#1}%
	\vskip .2cm
}%
{%
	\vskip .2cm}% end code


\renewenvironment{alertblock}[1]% environment name
{% begin code
	\par\vskip .2cm%
	{\color{red!80!black}#1}%
	\vskip .2cm
}%
{%
	\vskip .2cm}% end code


\renewenvironment{exampleblock}[1]% environment name
{% begin code
	\par\vskip .2cm%
	{\color{blue}#1}%
	\vskip .2cm
}%
{%
	\vskip .2cm}% end code




\newenvironment{exercise}[1]% environment name
{% begin code
	\par\vspace{\baselineskip}\noindent
	\textbf{Ejercicio (#1)}\begin{itshape}%
		\par\vspace{\baselineskip}\noindent\ignorespaces
	}%
	{% end code
	\end{itshape}\ignorespacesafterend
}


\newenvironment{definicion}[1][]% environment name
{% begin code
	\par\vskip .2cm%
	{\color{blue}Definición #1}%
	\vskip .2cm
}%
{%
	\vskip .2cm}% end code

\newenvironment{observacion}[1][]% environment name
{% begin code
	\par\vskip .2cm%
	{\color{blue}Observación #1}%
	\vskip .2cm
}%
{%
	\vskip .2cm}% end code

\newenvironment{ejemplo}[1][]% environment name
{% begin code
	\par\vskip .2cm%
	{\color{blue}Ejemplo #1}%
	\vskip .2cm
}%
{%
	\vskip .2cm}% end code

\newenvironment{ejercicio}[1][]% environment name
{% begin code
	\par\vskip .2cm%
	{\color{blue}Ejercicio #1}%
	\vskip .2cm
}%
{%
	\vskip .2cm}% end code


\renewenvironment{proof}% environment name
{% begin code
	\par\vskip .2cm%
	{\color{blue}Demostración}%
	\vskip .2cm
}%
{%
	\vskip .2cm}% end code



\newenvironment{demostracion}% environment name
{% begin code
	\par\vskip .2cm%
	{\color{blue}Demostración}%
	\vskip .2cm
}%
{%
	\vskip .2cm}% end code

\newenvironment{idea}% environment name
{% begin code
	\par\vskip .2cm%
	{\color{blue}Idea de la demostración}%
	\vskip .2cm
}%
{%
	\vskip .2cm}% end code

\newenvironment{solucion}% environment name
{% begin code
	\par\vskip .2cm%
	{\color{blue}Solución}%
	\vskip .2cm
}%
{%
	\vskip .2cm}% end code



\newenvironment{lema}[1][]% environment name
{% begin code
	\par\vskip .2cm%
	{\color{blue}Lema #1}\begin{itshape}%
		\par\vskip .2cm
	}%
	{% end code
	\end{itshape}\vskip .2cm\ignorespacesafterend
}

\newenvironment{proposicion}[1][]% environment name
{% begin code
	\par\vskip .2cm%
	{\color{blue}Proposición #1}\begin{itshape}%
		\par\vskip .2cm
	}%
	{% end code
	\end{itshape}\vskip .2cm\ignorespacesafterend
}

\newenvironment{teorema}[1][]% environment name
{% begin code
	\par\vskip .2cm%
	{\color{blue}Teorema #1}\begin{itshape}%
		\par\vskip .2cm
	}%
	{% end code
	\end{itshape}\vskip .2cm\ignorespacesafterend
}


\newenvironment{corolario}[1][]% environment name
{% begin code
	\par\vskip .2cm%
	{\color{blue}Corolario #1}\begin{itshape}%
		\par\vskip .2cm
	}%
	{% end code
	\end{itshape}\vskip .2cm\ignorespacesafterend
}

\newenvironment{propiedad}% environment name
{% begin code
	\par\vskip .2cm%
	{\color{blue}Propiedad}\begin{itshape}%
		\par\vskip .2cm
	}%
	{% end code
	\end{itshape}\vskip .2cm\ignorespacesafterend
}

\newenvironment{conclusion}% environment name
{% begin code
	\par\vskip .2cm%
	{\color{blue}Conclusión}\begin{itshape}%
		\par\vskip .2cm
	}%
	{% end code
	\end{itshape}\vskip .2cm\ignorespacesafterend
}







\newenvironment{definicion*}% environment name
{% begin code
	\par\vskip .2cm%
	{\color{blue}Definición}%
	\vskip .2cm
}%
{%
	\vskip .2cm}% end code

\newenvironment{observacion*}% environment name
{% begin code
	\par\vskip .2cm%
	{\color{blue}Observación}%
	\vskip .2cm
}%
{%
	\vskip .2cm}% end code


\newenvironment{obs*}% environment name
	{% begin code
		\par\vskip .2cm%
		{\color{blue}Observación}%
		\vskip .2cm
	}%
	{%
		\vskip .2cm}% end code

\newenvironment{ejemplo*}% environment name
{% begin code
	\par\vskip .2cm%
	{\color{blue}Ejemplo}%
	\vskip .2cm
}%
{%
	\vskip .2cm}% end code

\newenvironment{ejercicio*}% environment name
{% begin code
	\par\vskip .2cm%
	{\color{blue}Ejercicio}%
	\vskip .2cm
}%
{%
	\vskip .2cm}% end code

\newenvironment{propiedad*}% environment name
{% begin code
	\par\vskip .2cm%
	{\color{blue}Propiedad}\begin{itshape}%
		\par\vskip .2cm
	}%
	{% end code
	\end{itshape}\vskip .2cm\ignorespacesafterend
}

\newenvironment{conclusion*}% environment name
{% begin code
	\par\vskip .2cm%
	{\color{blue}Conclusión}\begin{itshape}%
		\par\vskip .2cm
	}%
	{% end code
	\end{itshape}\vskip .2cm\ignorespacesafterend
}




\newcommand{\img}{\operatorname{Im}}
\newcommand{\nuc}{\operatorname{Nu}}
\renewcommand\nu{\operatorname{Nu}}
\newcommand{\la}{\langle}
\newcommand{\ra}{\rangle}
\renewcommand{\t}{{\operatorname{t}}}
\renewcommand{\sin}{{\,\operatorname{sen}}}
\newcommand{\Q}{\mathbb Q}
\newcommand{\R}{\mathbb R}
\newcommand{\C}{\mathbb C}
\newcommand{\K}{\mathbb K}
\newcommand{\F}{\mathbb F}
\newcommand{\Z}{\mathbb Z}
\newcommand{\N}{\mathbb N}
\renewcommand{\figurename }{Figura}




\renewenvironment{block}[1]%%% environment name
{% begin code
    \par\vskip .2cm%
    {\color{blue}#1}%
    \vskip .2cm
}%
{%
    \vskip .2cm}% end code


\renewenvironment{alertblock}[1]% environment name
{% begin code
    \par\vskip .2cm%
    {\color{red!80!black}#1}%
    \vskip .2cm
}%
{%
    \vskip .2cm}% end code


\renewenvironment{exampleblock}[1]% environment name
{% begin code
    \par\vskip .2cm%
    {\color{blue}#1}%
    \vskip .2cm
}%
{%
    \vskip .2cm}% end code




\newenvironment{exercise}[1]% environment name
{% begin code
    \par\vspace{\baselineskip}\noindent
    \textbf{Ejercicio (#1)}\begin{itshape}%
        \par\vspace{\baselineskip}\noindent\ignorespaces
    }%
    {% end code
    \end{itshape}\ignorespacesafterend
}


\newenvironment{definicion}% environment name
{% begin code
    \par\vskip .2cm%
    {\color{blue}Definición}%
    \vskip .2cm
}%
{%
    \vskip .2cm}% end code

\newenvironment{observacion}% environment name
{% begin code
    \par\vskip .2cm%
    {\color{blue}Observación}%
    \vskip .2cm
}%
{%
    \vskip .2cm}% end code

\newenvironment{ejemplo}% environment name
{% begin code
    \par\vskip .2cm%
    {\color{blue}Ejemplo}%
    \vskip .2cm
}%
{%
    \vskip .2cm}% end code

\newenvironment{ejercicio}% environment name
{% begin code
    \par\vskip .2cm%
    {\color{blue}Ejercicio}%
    \vskip .2cm
}%
{%
    \vskip .2cm}% end code


\renewenvironment{proof}% environment name
{% begin code
    \par\vskip .2cm%
    {\color{blue}Demostración}%
    \vskip .2cm
}%
{%
    \vskip .2cm}% end code



\newenvironment{demostracion}% environment name
{% begin code
    \par\vskip .2cm%
    {\color{blue}Demostración}%
    \vskip .2cm
}%
{%
    \vskip .2cm}% end code

\newenvironment{idea}% environment name
{% begin code
    \par\vskip .2cm%
    {\color{blue}Idea de la demostración}%
    \vskip .2cm
}%
{%
    \vskip .2cm}% end code

\newenvironment{solucion}% environment name
{% begin code
    \par\vskip .2cm%
    {\color{blue}Solución}%
    \vskip .2cm
}%
{%
    \vskip .2cm}% end code



\newenvironment{lema}% environment name
{% begin code
    \par\vskip .2cm%
    {\color{blue}Lema}\begin{itshape}%
        \par\vskip .2cm
    }%
    {% end code
    \end{itshape}\vskip .2cm\ignorespacesafterend
}

\newenvironment{proposicion}% environment name
{% begin code
    \par\vskip .2cm%
    {\color{blue}Proposición}\begin{itshape}%
        \par\vskip .2cm
    }%
    {% end code
    \end{itshape}\vskip .2cm\ignorespacesafterend
}

\newenvironment{teorema}% environment name
{% begin code
    \par\vskip .2cm%
    {\color{blue}Teorema}\begin{itshape}%
        \par\vskip .2cm
    }%
    {% end code
    \end{itshape}\vskip .2cm\ignorespacesafterend
}


\newenvironment{corolario}% environment name
{% begin code
    \par\vskip .2cm%
    {\color{blue}Corolario}\begin{itshape}%
        \par\vskip .2cm
    }%
    {% end code
    \end{itshape}\vskip .2cm\ignorespacesafterend
}

\newenvironment{propiedad}% environment name
{% begin code
    \par\vskip .2cm%
    {\color{blue}Propiedad}\begin{itshape}%
        \par\vskip .2cm
    }%
    {% end code
    \end{itshape}\vskip .2cm\ignorespacesafterend
}

\newenvironment{conclusion}% environment name
{% begin code
    \par\vskip .2cm%
    {\color{blue}Conclusión}\begin{itshape}%
        \par\vskip .2cm
    }%
    {% end code
    \end{itshape}\vskip .2cm\ignorespacesafterend
}





%%%%%%%%%%%%%%%%%%%%%%%%%%%%%%%%%%%%%%%%%%%%%%%%%%%%%%%

\newcommand{\nc}{\newcommand}


%%%%%%%%%%%%%%%%%%%%%%%%%LETRAS
\nc{\RR}{{\mathbb R}} \nc{\CC}{{\mathbb C}} \nc{\ZZ}{{\mathbb Z}}
\nc{\FF}{{\mathbb F}} \nc{\NN}{{\mathbb N}} \nc{\QQ}{{\mathbb Q}}
\nc{\PP}{{\mathbb P}} \nc{\DD}{{\mathbb D}} \nc{\Sn}{{\mathbb S}}
\nc{\uno}{\mathbb{1}} \nc{\BB}{{\mathbb B}} \nc{\An}{{\mathbb A}}
%%%%%%%%%%%%%%%%%%%%%%%%%LETRAS


\newcommand{\Id}{\operatorname{Id}}
\newcommand\sgn{\operatorname{sgn}}




\title[Clase 12 - Determinante 1]{Álgebra/Álgebra II \\ Clase 12 - Determinante 1}

\author[]{}
\institute[]{\normalsize FAMAF / UNC
    \\[\baselineskip] ${}^{}$
    \\[\baselineskip]
}
\date[08/10/2020]{8 de octubre de 2020}


\begin{document}

\begin{frame}
\maketitle
\end{frame}

\begin{frame}{Resumen}
    En esta clase veremos 

    \vskip .4cm\pause

    \begin{itemize}
        \item La definición  de determinante de una matriz cuadrada.\pause
        \item Cálculo de determinantes $2 \times 2$ y $3 \times 3$.\pause
        \item Determinantes de matrices triangulares y diagonales.\pause
        \item Cálculo de determinante de matrices $n \times n$ utilizando operaciones elementales por fila.
    \end{itemize}

    \vskip .4cm
    \pause
    El tema de esta clase  está contenido de la sección la sección 2.8  del apunte de clase ``Álgebra II / Álgebra - Notas del teórico''.
\end{frame}


\begin{frame}{Introducción}

    El \textit{determinante} es una función que a cada matriz cuadrada $n \times n$ con coeficientes en $\K$,  le asocia un elemento de $\K$.\pause
\vskip .8cm
    \begin{center}
        \begin{tikzpicture}
        \draw (-1.3,0) node {$\det:$};
        \draw (0,0) node {$\K^n \times \K^n$};
        \draw[thick,->] (1,0) -- (5,0) ;
        \draw (6,0) node {$\K$};
        \draw (0,-1) node {$[a_{ij}]$};
        \draw[thick,->] (1,-1) -- (5,-1) ;
        \draw (6,-1) node {$\det([a_{ij}])$};
        \end{tikzpicture}
    \end{center} 

    \vskip 1cm

\end{frame}




\begin{frame}{Formas de definir determinante}
    \pause
\begin{itemize}

    \item Una forma de definir determinante es con una fórmula cerrada que usa el \textit{grupo de permutaciones.} Esta forma de definir determinante está fuera del alcance de este curso. 
    \vskip .2cm\pause
    \item  La forma  que usaremos nosotros para definir determinante es mediante una definición recursiva.\vskip .2cm\pause
    \item Es decir para calcular el determinante de una matriz $n \times n$, usaremos el cálculo  del determinante para matrices $n-1 \times n-1$, \vskip .2cm\pause
    \begin{itemize}
        \item[$\circ$] que a su vez se calcula usando el determinante de matrices $n-2 \times n-2$ \vskip .2cm
        \begin{itemize}\pause
            \item [$\circ$] y así sucesivamente hasta llegar al caso base, que es el caso  de matrices $1 \pause\times 1$.
        \end{itemize}
    \end{itemize}

\end{itemize}

\end{frame}



\begin{frame}

El determinante, permite, entre otras cosas \pause
\vskip .3cm
\begin{itemize}
    \item determinar si  una matriz cuadrada es invertible,\pause
    \item dar una fórmula cerrada para la inversa de una matriz invertible.\pause
\end{itemize}


\vskip .4cm


Para la definición del determinante de una matriz cuadrada $A$ usaremos \textit{submatrices} de la matriz $A$. 
\pause
\vskip .3cm
\begin{definicion}
Sea $A\in\K^{n\times n}$ y sean $i,j$ tales que $1\leq i,j\leq n$. Entonces 
$${A(i|j)}\in\K^{n-1\times n-1}$$ es la matriz que se obtine eliminando la fila $i$ y la columna $j$ de $A$. 
\end{definicion}


\end{frame}




\begin{frame}[fragile]
\begin{equation*}
    \begin{tikzpicture}
        \draw (-5,0) node {$A(i|j)=$};
    \end{tikzpicture}
    \left[\begin{tikzpicture}[baseline=(A.center)]
    \tikzset{node style ge/.style={circle}}
    \tikzset{BarreStyle/.style =   {opacity=.4,line width=4 mm,line cap=round,color=#1}}
    \tikzset{SignePlus/.style =   {above left,,opacity=1}}
    \tikzset{SigneMoins/.style =   {below left,,opacity=1}}
    % les matrices
    
    \matrix (A) [matrix of math nodes, nodes = {node style ge},,column sep=0 mm] 
    { a_{11} & \cdots & \cdots & a_{1j} & \cdots & a_{1n}\\
    \vdots & && \vdots & & \vdots\\
    a_{i1} & \cdots& \cdots & a_{ij}& \cdots & a_{in}\\  
    \vdots & && \vdots & & \vdots\\
    \vdots & && \vdots & & \vdots\\
    a_{m1} & \cdots& \cdots & a_{mj} & \cdots & a_{mn}\\
    };
    \draw [BarreStyle=blue] (A-3-1.west)  to (A-3-6.east) ;
    \draw [BarreStyle=red]  (A-1-4.north) to (A-6-4.south);
    \end{tikzpicture}
    \right]
\end{equation*}
(tacho  la fila $i$ y la columna $j$.)
\end{frame}






\begin{frame}

Más precisamente, 
\vskip.4cm
\begin{equation*}
    {A(i|j)}=\left[
\begin{array}{cccrlcc}
 a_{11} & \cdots& \cdots & a_{1j-1} & a_{1j+1} & \cdots & a_{1n}\\
 \vdots & && \vdots\;\;\; {}^{}& {}^{}\;\;\;\vdots  & & \vdots\\
  a_{i-11} & \cdots& \cdots & a_{i-1j-1} & a_{i-1j+1} & \cdots & a_{i-1n}\\  
a_{i+11} & \cdots& \cdots & a_{i+1j-1} & a_{i+1j+1}& \cdots & a_{i+1n}\\
 \vdots & & &\vdots \;\;\; {}^{} & {}^{}\;\;\;\vdots  & & \vdots\\
 \vdots & & &\vdots \;\;\; {}^{} & {}^{}\;\;\;\vdots  & & \vdots\\
 a_{m1} & \cdots& \cdots & a_{mj-1} &a_{mj+1} & \cdots & a_{mn}\\ 
\end{array}
\right]
\end{equation*}
\end{frame}

\begin{frame}

    \begin{exampleblock}{Ejemplos}
Si 
$A=
\left[
\begin{array}{ccc}
 1 & 2 & 3\\
 4 & 5 & 6\\
 7 & 8 & 9
\end{array}
\right]
$, entonces 
$A(1|2)=
\left[
\begin{array}{cc}
4 & 6\\
 7 & 9
\end{array}
\right] 
$, $A(1|1)=
\left[
\begin{array}{cc}
5 & 6\\
 8 & 9
\end{array}
\right]
$.
\pause
\vskip .9cm
Si 
$A=
\begin{bmatrix}
    -1 & 3 & 2 & 5\\
     -2& 7 &  2& 6&\\
     -3&  2&  1& 11&\\
     -4&  5&  3& -3&
\end{bmatrix}
$, entonces 
$A(3|2)=
\begin{bmatrix}
    -1 &  2 & 5\\
     -2&   2& 6&\\
     -4&   3& -3&
\end{bmatrix}
$.
\end{exampleblock}


\end{frame}



\begin{frame}
\begin{exampleblock}{Definición}
Sea $n\in\NN$ y $A=[a_{ij}]\in\K^{n\times n}$, entonces el {determinante de $A$}, denotado {$\det(A)$} ó {$|A|$}, se define como
\vskip .2cm
\begin{enumerate}
 \item Si $n=1$, $\det A=a_{11}$
 \item Si $n>1$, 
 \begin{align*}
    \det A&=  a_{11}\det A(1|1) - a_{21}\det A(2|1) + \cdots + (-1)^{1+n}  a_{n1}\det A(n|1),
\end{align*}\pause
o
\definecolor{airforceblue}{rgb}{0.36, 0.54, 0.66}
\hfsetfillcolor{airforceblue!30}
\hfsetbordercolor{blue!10}
\begin{equation*}
    \tikzmarkin{b}(0.2,-0.6)(-0.2,0.8)  \det A=   \sum_{i=1}^{n} (-1)^{1+i}\,a_{i1}\,\det A(i|1).
    \tikzmarkend{b}
\end{equation*}

   
\end{enumerate}
\end{exampleblock}

\vskip .2cm


\end{frame}



\begin{frame}
    

    Si  $1 \le i,j \le n$, 
    \vskip 0.2cm
    \begin{itemize}
        \item  $\det A(i|j)$ :  \textit{menor $i,j$ de $A$}.\pause
        \item  $C_{ij}:= (-1)^{i+j} \det A(i|j)$ : \textit{cofactor $i,j$ de $A$.} 
        
        Luego 
        \begin{equation*}
            \det A = \sum_{i=1}^{n} \,a_{i1}\,C_{i1} .
        \end{equation*}
    \end{itemize}

    \vskip 0.6cm
Este es el \textit{cálculo del determinante por desarrollo de la primera columna} pues para calcular el determinante estamos usando los coeficientes de la primera columna: $$a_{11}, a_{21}, a_{31}, \ldots, a_{n1}.$$





\vskip 2cm
\end{frame}

\begin{frame}{Determinante $2\times 2$}

    Calculemos el determinante de las matrices $2 \times 2$. Sea 
    $$A=\begin{bmatrix}a&b\\c&d\end{bmatrix},$$  \pause entonces
    \begin{align*}
        \det A &= a \det [d] - c \det [b] \\
        &= ad-bc.
    \end{align*}
    \vskip .1cm 
    \pause
    \begin{exampleblock}{Observación}
        Si 
        $A=
        \left[
        \begin{array}{cc}
        a_{11}&a_{12}\\a_{21}&a_{22}
        \end{array}
        \right]
        $, entonces 
        $\det(A)=a_{11} a_{22} - a_{12}a_{21}$
        \end{exampleblock}
(solo es notación),
    
\end{frame}




\begin{frame}
    \begin{exampleblock}{Ejemplo}
        Si 
        $A=
        \left[
        \begin{array}{cc}
        1&2\\3&4
        \end{array}
        \right]
        $, entonces 
        $\det(A)=1\cdot 4-2\cdot 3=-2$
        \end{exampleblock}
        \vskip .8cm    
    
    \pause
    
\begin{exampleblock}{Ejemplo}
    Si 
    $$A=
    \left[
    \begin{array}{cc}
    \operatorname{cos}\theta&
    -\operatorname{sen}\theta\\
    \operatorname{sen}\theta&
    \operatorname{cos}\theta
    \end{array}
    \right]
    ,$$ 
    entonces
    $$
    \det(A)=\operatorname{cos}^2\theta+\operatorname{sen}^2\theta=1.
    $$
    \end{exampleblock}
    

 

\end{frame}


\begin{frame}
    Hemos visto en la clase anterior que $A \in \R^{2 \times 2}$ es invertible si y solo si $ ad-bc \not=0$,  es decir 
    \begin{equation*}
    \text{\textit{$A$ es invertible si y solo si $ \det A \not=0$.}}
    \end{equation*}
   
    \vskip .2cm\pause
    Más aún, la fórmula de la inversa de $A$ es
    \begin{equation*}
        A^{-1} = \dfrac{1}{\det(A)}
        \begin{bmatrix}
            d&c\\b&a
        \end{bmatrix}.
    \end{equation*} \pause
    Se puede reescribir como
    \begin{equation*}
        A^{-1} = \dfrac{1}{\det(A)}
        \begin{bmatrix*}[c]C_{11}&C_{12}\\C_{21}&C_{22}\end{bmatrix*}.
    \end{equation*} 
    \vskip .4cm\pause
    Esta fórmula se puede generalizar a matrices $n \times n$.

\end{frame}



\begin{frame}{Determinantes ${3 \times 3}$.}

Sea 
\begin{equation*}
A=\begin{bmatrix}a_{11}&a_{12}&a_{13}\\a_{21}&a_{22}&a_{23}\\a_{31}&a_{32}&a_{33}\end{bmatrix},
\end{equation*}
entonces
\pause

\begin{multline*}
    \det A = a_{11}\left|\begin{matrix}a_{22}&a_{23}\\a_{32}&a_{33}\end{matrix}\right|
    - a_{21}\left|\begin{matrix}a_{12}&a_{13}\\a_{32}&a_{33}\end{matrix}\right|
    + a_{31}\left|\begin{matrix}a_{12}&a_{13}\\a_{22}&a_{23}\end{matrix}\right|\\
    = a_{11}(a_{22}a_{33}- a_{23}a_{32})
    - a_{21}(a_{12}a_{33}-a_{13}a_{32}) 
    + a_{31}(a_{12}a_{23} - a_{13}a_{22}) \\
    =
    a_{11}a_{22}a_{33}- a_{11}a_{23}a_{32} 
    - a_{12}a_{21}a_{33}+ a_{13}a_{21}a_{32}+ a_{12}a_{23}a_{31}
    - a_{13}a_{22}a_{31}.
\end{multline*}
        
        
    
\end{frame}

\begin{frame}

    \begin{observacion}
    Observar que  el determinante de una matriz $3 \times 3$ es una sumatoria de seis términos cada uno de los cuales es de la forma $\pm a_{1\,i_1}a_{2\,i_2}a_{3\,i_3}$ e $i_1i_2i_3$ puede ser cualquier permutación de $123$. 
    \vskip .2cm\pause
    Es decir 
    \begin{equation*}
        |A| = \sum_{\sigma \in \mathbb S_3} \pm a_{1\,\sigma(1)}a_{2\,\sigma(2)}a_{3\,\sigma(3)},
    \end{equation*}
    donde $\mathbb S_3$ son las permutaciones de 3 elementos. 
    \end{observacion}

    \pause


    \vskip .4cm

    La fórmula 
        \begin{equation*}
        |A| =a_{11}a_{22}a_{33}- a_{11}a_{23}a_{32} 
        - a_{12}a_{21}a_{33}+ a_{13}a_{21}a_{32}+ a_{12}a_{23}a_{31}
        - a_{13}a_{22}a_{31},
        \end{equation*} 
        no es fácil de recordar, pero existe un procedimiento sencillo que nos permite obtenerla.
        
        \vskip 1cm

\end{frame}


\begin{frame}[fragile]
     Cálculo de $|A|$:
        \begin{enumerate}
            \item a la matriz original le agregamos las dos primeras filas al final, \pause
            \item sumamos  cada ``producto'' de las diagonales descendentes y \pause
            \item restamos cada ``producto'' de las diagonales ascendentes.\pause
        \end{enumerate}
            
        %prueba y otra 
        \begin{equation*}
        \begin{tikzpicture}[baseline=(A.center)]
        \tikzset{node style ge/.style={circle}}
        \tikzset{BarreStyle/.style =   {opacity=.4,line width=0.5 mm,line cap=round,color=#1}}
        \tikzset{SignePlus/.style =   {above left,,opacity=1}}
        \tikzset{SigneMoins/.style =   {below left,,opacity=1}}
        % les matrices
        \matrix (A) [matrix of math nodes, nodes = {node style ge},,column sep=0 mm] 
        { a_{11} & a_{12} & a_{13}  \\
            a_{21} & a_{22} & a_{23}  \\
            a_{31} & a_{32} & a_{33}  \\
            a_{11} & a_{12} & a_{13} \\
            a_{21} & a_{22} & a_{13}\\
        };
        
        \draw [BarreStyle=blue] (A-1-1.north west) node[SignePlus=blue] {$+$} to (A-3-3.south east) ;
        \draw [BarreStyle=blue] (A-2-1.north west) node[SignePlus=blue] {$+$} to (A-4-3.south east) ;
        \draw [BarreStyle=blue] (A-3-1.north west) node[SignePlus=blue] {$+$} to (A-5-3.south east) ;
        \draw [BarreStyle=red]  (A-3-1.south west) node[SigneMoins=red] {$-$} to (A-1-3.north east);
        \draw [BarreStyle=red]  (A-4-1.south west) node[SigneMoins=red] {$-$} to (A-2-3.north east);
        \draw [BarreStyle=red]  (A-5-1.south west) node[SigneMoins=red] {$-$} to (A-3-3.north east);
        \end{tikzpicture}
        \end{equation*}
        
        
\end{frame}

\begin{frame}
    Es decir,
        \begin{enumerate}
            \item[(a)]  se suman  $a_{11}a_{22}a_{33}$, $a_{21}a_{32}a_{13}$, $a_{31}a_{12}a_{23}$,  y
            \item[(b)]  se restan  $a_{31}a_{22}a_{13}$,  $a_{11}a_{32}a_{23}$,  $a_{21}a_{12}a_{33}$, 
        \end{enumerate}
        \pause
        obteniéndose nuevamente 

        \begin{equation*}
            |A| =a_{11}a_{22}a_{33}- a_{11}a_{23}a_{32} 
            - a_{12}a_{21}a_{33}+ a_{13}a_{21}a_{32}+ a_{12}a_{23}a_{31}
            - a_{13}a_{22}a_{31}.
            \end{equation*} 

        \vskip 2cm

        
\end{frame}

\begin{frame}[fragile]

\begin{exampleblock}{Ejemplo}
Si 
$$A=
\left[
\begin{array}{ccc}
1&2&3\\
4&5&6\\
7&8&9
\end{array}
\right].
$$ \pause
Calculamos:
\begin{equation*}
    \begin{tikzpicture}[baseline=(A.center)]
    \tikzset{node style ge/.style={circle}}
    \tikzset{BarreStyle/.style =   {opacity=.4,line width=0.5 mm,line cap=round,color=#1}}
    \tikzset{SignePlus/.style =   {above left,,opacity=1}}
    \tikzset{SigneMoins/.style =   {below left,,opacity=1}}
    % les matrices
    \matrix (A) [matrix of math nodes, nodes = {node style ge},,column sep=0 mm]
    { 1&2&3\\
    4&5&6\\
    7&8&9\\
    1&2&3\\
    4&5&6\\
    };
    
    \draw [BarreStyle=blue] (A-1-1.north west) to (A-3-3.south east) ;
    \draw [BarreStyle=blue] (A-2-1.north west) to (A-4-3.south east) ;
    \draw [BarreStyle=blue] (A-3-1.north west) to (A-5-3.south east) ;
    \draw [BarreStyle=red]  (A-3-1.south west) to (A-1-3.north east);
    \draw [BarreStyle=red]  (A-4-1.south west) to (A-2-3.north east);
    \draw [BarreStyle=red]  (A-5-1.south west) to (A-3-3.north east);
    \draw (1.8,1.45) node {$- 7\cdot 5\cdot 3$};
    \draw (1.8,0.90) node {$- 1\cdot 8\cdot 6 $};
    \draw (1.8,0.35) node {$ - 4\cdot 2\cdot 9$};
    \draw (1.8,-0.20) node {$+1\cdot 5\cdot 9 $};
    \draw (1.8,-0.85) node {$+ 4\cdot 8\cdot 3 $};
    \draw (1.8,-1.45) node {$+ 7\cdot 2\cdot 6 $};
    \end{tikzpicture}
    \end{equation*}
  
    Entonces  $\det(A) =45+96+84-105-48-72=0$.



\end{exampleblock}

\end{frame}



\begin{frame}

\begin{ejemplo}
\end{ejemplo}

Calculemos el determinante de una matriz $4\times 4$

\begin{align*}
\left|
\begin{array}{cccc}
1&1&1&1\\
1&2&3&0\\
0&1&2&3\\
2&2&1&1
\end{array}
\right|
=&1\left|
\begin{array}{ccc}
2&3&0\\
1&2&3\\
2&1&1
\end{array}
\right|
-1
\left|
\begin{array}{ccc}
1&1&1\\
1&2&3\\
2&1&1
\end{array}
\right|
\\
&\qquad\qquad\qquad\qquad
+0
\left|
\begin{array}{ccc}
1&1&1\\
2&3&0\\
2&1&1
\end{array}
\right|
-2
\left|
\begin{array}{ccc}
1&1&1\\
2&3&0\\
1&2&3
\end{array}
\right|
\\
=&13-3+0-8\\
=&2.
\end{align*}


\end{frame}

\begin{frame}

\begin{alertblock}{Observación}

Las fórmulas para $n=2$ y $n=3$ es un caso particular de la fórmula 

$$
\det(A)=\sum_{\sigma\in\mathbb{S}_n}\sgn(\sigma)\, a_{1\sigma(1)}\,a_{2\sigma(2)}\cdots a_{n\sigma(n)}
$$

válida para todo $n$, la cual tiene $n!$ términos.


\


($\mathbb{S}_n$ es el conjunto de permutaciones de $\{1, ..., n\}$ y $\sgn:\mathbb{S}_n\rightarrow\{1,-1\}$ es una función, llamada \textit{función signo}.)
 
\end{alertblock}
\vskip 1.5 cm

\end{frame}


\begin{frame}

\begin{block}{Observación}
    \pause
    \begin{itemize}
        \item Del modo que lo presentamos, el determinante no es más que una fórmula que le aplicamos a una matriz. Pero aquí sólo estamos viendo el producto final de años y años de estudio.
        \vskip .2cm\pause
        \item  De hecho, el determinante existió antes que las matrices y se lo utilizaba para ``determinar'' cuando un sistema de $n$ ecuaciones con $n$ incógnitas tiene solución única (si y sólo si el determinante es no nulo). 
        \vskip .2cm\pause
        \item También tiene otras aplicaciones.
        \vskip .2cm\pause
        \item  Pueden googlear ``determinante'' para saber más o leer la página de Wikipedia.
    \end{itemize}



\end{block}
 
\end{frame}

\begin{frame}

Viendo la fórmula del determinante
$$
\det(A)=a_{11}\det A(1|1) - a_{21}\det A(2|1) + \cdots + (-1)^{1+n}  a_{n1}\det A(n|1)
$$
notamos que mientras más ceros tenga la primera columna (o sea, más $a_{i1}$'s iguales a $0$), menos cuentas deberemos hacer.

\

Por ejemplo, si $A$ es triangular superior o una MERF.
\end{frame}

\begin{frame}{Determinante de una matriz triangular superior}

\begin{block}{Proposición}
El determinante de una matriz triangular superior es el producto de los elementos de la diagonal
\begin{align*}
\left|
\begin{array}{ccccccc}
a_{11}& a_{12} & \cdots & & &  & a_{1n}\\ 
0 & a_{22} & \cdots & & &  & a_{2n}\\
 \vdots & 0 & \ddots & & &  & \vdots\\
  & \vdots &  & & &&  \\
 & &  & & &\ddots& \vdots \\
 0 & 0 & \cdots & & & 0& a_{nn}\\ 
\end{array}
\right|={a_{11}\cdot a_{22}\cdots a_{nn}}
\end{align*}
\end{block}

(esto aplica también a las matrices diagonal)

\end{frame}


\begin{frame}

        \begin{proof} Podemos demostrar el resultado por inducción sobre $n$. 
            \vskip .2cm\pause
        Si $n=1$,  es decir si $A = [a_{11}]$, el determinante vale $a_{11}$. 
        \vskip .2cm\pause
        Sea $n>1$,  observemos que $A(1|1)$ es también triangular superior con valores $a_{22},\ldots,a_{nn}$  en la diagonal principal. Por HI:
        $$
        \det(A(1|1)) = a_{22}.\ldots.a_{nn}.
        $$
        \vskip .2cm\pause
        Por definición de determinante observamos que el desarrollo por la primera  columna solo tiene un término ($a_{11}$) en la primera posición. 
        \vskip .2cm\pause
        Por lo tanto, 
            \begin{equation*}
            \det(A) = a_{11} \det(A(1|1)) \stackrel{\text{(HI)}}{=} a_{11}.(a_{22}.\ldots.a_{nn}).
            \end{equation*}
            \qed 
        \end{proof}
        
\end{frame}

\begin{frame}{Casos particulares}
    \pause
\begin{block}{Corolario 1}
\center{$\det(\Id_n)=1$} 
\end{block}
\pause
\begin{block}{Corolario 2}
Si $R\in\K^{n\times n}$ es una MERF, entonces
$$
\det(R)=\begin{cases}
         1&\mbox{si $R$ no tiene filas nulas}\\
         0&\mbox{si $R$ tiene filas nulas}
        \end{cases}
$$
\end{block}

\vspace{3cm}


\end{frame}


\begin{frame}

Volvamos a la fórmula del determinante
$$
\det(A)=a_{11}\det A(1|1) - a_{21}\det A(2|1) + \cdots + (-1)^{1+n}  a_{n1}\det A(n|1)
$$
y a la observación de que con más ceros en la primera columna menos cuentas deberemos hacer.
\pause

\

Con las operaciones elementales por filas podemos anular las entradas no nulas como lo hacíamos para transformar una matriz en MERF.
\pause


\

Entonces deberíamos analizar como estas operaciones afectan en el cálculo del determinante. 

\end{frame}

\begin{frame}

\begin{block}{Teorema E1}{\em

Sea $A\in\K^{n\times n}$ y $c\in\K$ no nulo. Sea $A \stackrel{{cF_i}}{\longrightarrow} B$  y $E$ la matriz elemental tal que $B=EA$. Entonces 
$$\det(B)=c\det(A) = \det(E)\det(A).$$}
\end{block}
\vskip -.6cm\pause
\begin{exampleblock}{Ejemplo (verificar cuentas)}
$$
A=\begin{bmatrix}
    1&2\\3&4 
\end{bmatrix}
\stackrel{{10F_1}}{\longrightarrow}
B=\begin{bmatrix}
10&20\\3&4 
\end{bmatrix},
$$
\vskip .1cm
y
\vskip .1cm
$$
\det
\begin{bmatrix}
10&20\\3&4 
\end{bmatrix}
=-20
=10\det
\begin{bmatrix}
1&2\\3&4 
\end{bmatrix}.
$$
\end{exampleblock}


\end{frame}


\begin{frame}

    \begin{block}{Teorema E2}{\em

Sean $A\in\K^{n\times n}$ y $ 1 \le r,s \le n$ con $r\ne s$. Sea $t\in\K$, $A \stackrel{{F_r+tF_s}}{\longrightarrow} B$ y $E$ la matriz elemental tal que $B=EA$. Entonces 
$$\det(B)=\det(A)= \det(E)\det(A).$$
}
\end{block}
\vskip -.6cm\pause
\begin{exampleblock}{Ejemplo (verificar cuentas)}
$$
A=\begin{bmatrix}
    1&2\\3&4 
\end{bmatrix}
\stackrel{{F_2+10F_1}}{\longrightarrow}
B=\begin{bmatrix}
1&2\\13&24 
\end{bmatrix},
$$

\vskip .2cm
y
\vskip .2cm


$$
\det
\begin{bmatrix}
    1&2\\13&24 
\end{bmatrix}
=-2
=\det
\begin{bmatrix}
1&2\\3&4 
\end{bmatrix}.
$$
\end{exampleblock}



\end{frame}

\begin{frame}
    \begin{block}{Teorema E3}{\em

Sean $A\in\K^{n\times n}$. Sea $A \stackrel{{F_r \leftrightarrow F_s}}{\longrightarrow} B$ y $E$ la matriz elemental tal que $B=EA$.
entonces
{$$\det(B)=-\det(A)= \det(E)\det(A)$$}
}
\end{block}
\vskip -.6cm\pause
\begin{exampleblock}{Ejemplo (verificar cuentas)}
$$
A=\begin{bmatrix}
    1&2\\3&4 
\end{bmatrix}
\stackrel{{F_2 \leftrightarrow F_1}}{\longrightarrow}
B=\begin{bmatrix}
    3&4\\ 1&2
\end{bmatrix},
$$

\vskip .2cm
y
\vskip .2cm


$$
\det
\begin{bmatrix}
    3&4 \\1&2
\end{bmatrix}
=2
=-\det
\begin{bmatrix}
1&2\\3&4 
\end{bmatrix}.
$$
\end{exampleblock}






\end{frame}

\begin{frame}



    \begin{block}{Observación}\pause
        \begin{itemize}
            \item  En todos los casos $\det(B)=k\det(A)$ para algún $k\in\K$ no nulo.\pause
            \vskip .3cm
            \item  A partir de lo anterior podemos plantear una estrategia general para calcular el determinante.
        \end{itemize}\pause
    \end{block}
    \vskip .6cm

\begin{corolario} Sea $A$ matriz $n \times n$ y $C$ matriz que se obtiene de $A$ a partir de operaciones elementales de fila $e_1,\ldots,e_t$ tal que $C = e_t e_{t-1} \cdots e_1 A$. Luego $$\det(C) = k_t k_{t-1} \cdots k_1 \det(A)$$ donde $k_i$ es el escalar que corresponde a la operación elemental $e_i$.
\end{corolario}

\vskip 2cm


\end{frame}

\begin{frame}



\begin{exampleblock}{Estrategia para calcular el determinante de $A\in\K^{n\times n}$}
    \pause
\begin{enumerate}
 \item Realizando $\ell$ operaciones elementales de fila obtenemos a partir de $A$  una matriz triangular superior $C$.\pause
 
 \item Luego, 
 $$
 \det(C)=k_\ell\cdots k_1\det(A)
 $$\pause
 
 \item Como $C$  es triangular superior, el determinante de $C$ es el producto de la diagonal.
 
 \item Entonces, podemos despejar
 $$
 {\det(A)=\frac{1}{k_\ell\cdots k_1}\det(C)}
 $$
\end{enumerate}
\end{exampleblock}





\end{frame}

\begin{frame}
\begin{exampleblock}{Problema}
Calcular el determinante de  
$
A=\left[
\begin{array}{rrr}
0&2&3\\
2&-1&7\\
1&3&0
\end{array}
\right]
$
\end{exampleblock}
\pause
\begin{solucion}\pause

Primero, le aplicamos operaciones elementales a $A$ hasta obtener una matriz triangular superior.

\begin{align*}
A=&\left[
\begin{array}{rrr}
0&2&3\\
2&-1&7\\
1&3&0
\end{array}
\right]
\overset{{F_1\leftrightarrow F_3}}{\longrightarrow}
\left[
\begin{array}{rrr}
1&3&0\\
2&-1&7\\
0&2&3
\end{array}
\right]
\overset{{F_2-2F_1}}{\longrightarrow}
\left[
\begin{array}{rrr}
1&3&0\\
0&-7&7\\
0&2&3
\end{array}
\right]
\\
\\
&
\overset{{-\frac{1}{7}F_2}}{\longrightarrow}
\left[
\begin{array}{rrr}
1&3&0\\
0&1&-1\\
0&2&3
\end{array}
\right]
\overset{{F_3-2F_2}}{\longrightarrow}
\left[
\begin{array}{rrr}
1&3&0\\
0&1&-1\\
0&0&5
\end{array}
\right]=C
\end{align*}


\end{solucion}

\end{frame}

\begin{frame}

Entonces 

$$C=e_4(e_3(e_2(e_1(A))))$$

donde $e_1$, $e_2$, $e_3$ y $e_4$ denotan las operaciones elementales aplicadas en cada paso
Ahora calculamos el determinante de $C$ usando Teoremas Ei:
\pause
\begin{align*}
5&=\det \left[
\begin{array}{rrr}
1&3&0\\
0&1&-1\\
0&0&5
\end{array}
\right]
=\det\left(e_4(e_3(e_2(e_1(A))))\right)
\end{align*}
{Donde $e_4 = F_3-2F_2$, $e_3 = -\frac{1}{7}F_2$, $e_2 = F_2-2F_1$, $e_1 =F_1\leftrightarrow F_3$,}


\end{frame}


\begin{frame}

    Luego, 

    \begin{align*}
        5&\quad\;= \det\left(e_4(e_3(e_2(e_1(A))))\right)\\
        \\
        &\stackrel{\text{Teorema E2}}{=}\det\left(e_3(e_2(e_1(A)))\right)
        \stackrel{\text{Teorema E1}}{=}-\frac{1}{7}\det\left(e_2(e_1(A))\right)
        \\
        \\
        &\stackrel{\text{Teorema E2}}{=}
        -\frac{1}{7}\det\left(e_1(A)\right)
        \stackrel{\text{Teorema E3}}{=}-\frac{1}{7}\cdot(-1)\cdot\det\left(A\right)
        \end{align*}

\vskip.4cm
\pause
De esta igualdad podemos despejar $\det(A)$

$$
{\det A=7\cdot 5=35.}
$$

\qed

\end{frame}




\begin{frame}

Recordemos
\vskip.4cm
\begin{teorema}
\pause
    Sea $A\in\K^{n\times n}$.

    \begin{enumerate}
        \item[{\color{blue} E1.}]  Si $c\in\K$ no nulo, 
        $$
        A \stackrel{{cF_i}}{\longrightarrow} B \qquad \Rightarrow   \qquad     \det(B)=c\det(A).
        $$\pause
        \item[{\color{blue} E2.}]
        Si $ 1 \le s , t \le n$ con $s\ne t$ y  $t\in\K$: 
        $$
        A \stackrel{{F_r+tF_s}}{\longrightarrow} B  \qquad \Rightarrow   \qquad \det(B)=\det(A).
        $$\pause

        \item[{\color{blue} E3.}]  
        $$
        A \stackrel{{F_r \leftrightarrow F_s}}{\longrightarrow} B  \qquad \Rightarrow   \qquad \det(B)=-\det(A).$$ 
    \end{enumerate}

\end{teorema}
    
\end{frame}


\begin{frame}
    \begin{corolario}\label{cor-filas-nulas} $A  \in \K^{n \times n}$.
        \begin{enumerate}
            \item Si $A$ tiene dos filas iguales,  entonces $\det A=0$.
            \item Si $A$ tiene una fila nula, entonces $\det A =0$.
        \end{enumerate}
    \end{corolario}\pause

           
    \begin{corolario}\label{cor-det-elem} Sea $A = E_1 E_2 \cdots E_k B$ donde  $E_1, E_2, \ldots, E_k$  son matrices elementales. 
        Entonces, 
        \begin{equation*}
            \det(A) = \det(E_1) \det(E_2) \cdots \det(E_k)\det(B).
        \end{equation*}
\end{corolario}\pause
\vskip -.6cm
\begin{demostracion}\pause
    \vskip -.6cm
$$
\det(A) = \det(E_1(E_2 \cdots E_k B)) = \det(E_1)\det(E_2 \cdots E_k B),
$$
y así sucesivamente (inducción). \qed
\end{demostracion}
        \vskip .4cm\pause
    \begin{corolario} Sea $A = E_1 E_2 \cdots E_k$ producto de matrices elementales en  $\K^{n \times n}$. 
        Entonces, 
        \begin{equation*}
            \det(A) = \det(E_1) \det(E_2) \cdots \det(E_k).
        \end{equation*}
\qed
\end{corolario}



\end{frame}





\begin{frame}
    
\begin{teorema}
    $A \in \K^{n \times n}$ es invertible si y solo si $\det(A) \ne 0$.
\end{teorema}
\pause
\begin{demostracion}\pause
    ($\Rightarrow$) 
    
    $A$ invertible $\Rightarrow$ $A = E_1 E_2 \cdots E_k$  $\Rightarrow$  $\det(A) = \det(E_1) \det(E_2) \ldots \det(E_k)$. Como el determinante de matrices elementales en no nulo, $\det(A) \ne 0$.
\vskip .4cm
    ($\Leftarrow$) 
    
    Sean $E_1, E_2, \ldots, E_k$ matrices elementales tales que $R = E_1 E_2 \cdots E_k A$ y $R$ es MERF. Luego,
    \begin{equation*}
        \det(R) = \det(E_1) \det(E_2) \cdots \det(E_k) \det(A). 
    \end{equation*}
   


\end{demostracion}

\end{frame}


\begin{frame}
    Como los determinantes de matrices  elementales son no nulos
    \begin{equation*}
        \frac{\det(R)}{\det(E_1) \det(E_2) \cdots \det(E_k) } = \det(A). \tag{*}
    \end{equation*}

    
    Supongamos que $R$ no es la identidad.
    \vskip .3cm
    Entonces $\det(R) =0$ (ver clase pasada) $\stackrel{\text{(*)}}{\Rightarrow}$  $\det(A)=0$, absurdo. 
    \vskip .3cm
    Luego, $R= \Id_n$  $\Rightarrow$ $A$ es equivalente por filas a $\Id_n$  $\Rightarrow$  $A$ invertible.

    \qed
    \vskip 2cm
\end{frame}



\begin{frame}
    \begin{teorema} Sean $A,B \in \K^{n \times n}$, entonces 
        $$\det (A B) = \det(A)\det(B).$$ 
    \end{teorema}\pause
    \begin{proof}\pause
       - Si $A$ invertible $\Rightarrow$  $A= E_1\cdots E_k$ y $AB =  E_1\cdots E_kB$, luego por los corolarios de {\color{blue}p. \ref{cor-det-elem}} $\det(AB) =  \det(E_1)\cdots \det(E_k)\det(B) = \det(A)\det(B)$.
       \vskip .3cm
       - Si $A$ no invertible $\Rightarrow$  $A= E_1\cdots E_kR$ y $R$ MERF con la última fila nula.        
       \vskip .2cm
       Luego,  
       \begin{itemize}
           \item $RB$ tiene la última fila nula $\Rightarrow$ $\det(RB) = 0$ (corolario {\color{blue}p. \ref{cor-filas-nulas}} ).
           \item $\det(AB) = \det(E_1\cdots E_kRB) = \det(E_1\cdots E_k)\det(RB) = 0$.
           \item Como $A$ no invertible $\Rightarrow$ $\det(A) = 0$ $\Rightarrow$ $\det(A)\det(B) = 0$.
       \end{itemize}  
        \qed
    \end{proof}

\end{frame}



\begin{frame}
    
    \begin{corolario} San $A$, $B$ matrices $n \times n$, entonces
        \begin{itemize}
            \item $\det(A^m) = \det(A)^m$, para $m \in \N$.
            \item $\det (AB) = \det(BA)$.
        \end{itemize}
    \end{corolario}\pause
    \begin{proof}\pause
        $\circ$  $\det(A^m) = \det(A \cdot A^{m-1})  =  \det(A )\cdot \det(A^{m-1})$ y se demuestra por inducción.
        \vskip .4cm
        $\circ$   $\det (AB) = \det(A)\det(B) = \det(B)\det(A) = \det(BA)$.
        \qed
    \end{proof}
\vskip 3cm
\end{frame}




\begin{frame}

    \begin{exampleblock}{Definición}
    Sea $A\in\R^{m\times n}$. La \textit{transpuesta de $A$} es la matriz {$A^{\t}\in\R^{n\times m}$} cuyas entradas son definidas por
    $$
    [A^\t]_{ij}=[A]_{ji}
    $$
    \end{exampleblock}\pause

    \begin{ejemplo}
        Sea 
        \begin{equation*}
            A = \begin{bmatrix}
                1&2&3\\
                4&5&6\\
                7&8&9\\
            \end{bmatrix}
            \qquad \Rightarrow \qquad
            A^{\t} = \begin{bmatrix}
                1&4&7\\
                2&5&8\\
                3&6&9\\
            \end{bmatrix}
        \end{equation*}
        \vskip .3cm
    ($  [A^\t]_{12}= [A]_{21} = 4$, $  [A^\t]_{13}= [A]_{31} = 3$,  etc.)
    \end{ejemplo}
    \end{frame}
    
    \begin{frame}
        Para matrices cuadradas, en general: 
    \begin{align*}
    A&=
    \left[
    \begin{array}{ccccc}
     a_{11} & \cdots & a_{1i} & \cdots & a_{1n}\\
     \vdots & & \vdots & & \vdots\\
     a_{i1} & \cdots &  a_{ii} & \cdots & a_{in}\\  
     \vdots & & \vdots & & \vdots\\
     a_{n1} & \cdots & a_{ni} & \cdots & a_{nn}
    \end{array}
    \right] 
    \end{align*}
    
    \
    \pause Entonces, 
    \
    
    \begin{align*}
    A^\t=
    \left[
    \begin{array}{ccccc}
     a_{11} & \cdots & a_{i1} & \cdots & a_{n1}\\
     \vdots & & \vdots & & \vdots\\
     a_{1i} & \cdots & a_{ii} & \cdots & a_{ni}\\  
     \vdots & & \vdots & & \vdots\\
     a_{1n} & \cdots & a_{in} & \cdots & a_{nn}
    \end{array}
    \right] 
    \end{align*} 
    \end{frame}
    
    \begin{frame}
    \begin{teorema}
    El determinante de una matriz cuadrada  es igual al determinante de su transpuesta. 
    \vskip.2cm
    Es decir, si $A$ matriz $n \times n$,
    $$
    \det(A)=\det(A^\t).
    $$
    \end{teorema}
    \vskip.3cm\pause
    Pueden ver la demostración en las notas de curso.
     \vskip .3cm
   \begin{block}{Idea de la demostración}
   
   No es difícil ver que 
   \begin{itemize}
        \item para $E$ matriz elemental $\det(E^\t) = \det(E)$,
        \item $(A_1 \cdot A_2 \cdots A_k)^\t = A_k^\t \cdots A_2^\t \cdot A_1^\t$.
   \end{itemize}
   Se sigue entonces de $A = E_1 \cdots E_s R$,  con $E_i$  elementales y $R$ MERF.
   \end{block} 
    \end{frame}
    

    \begin{frame}
        Sea 
        \begin{align*}
        A= \left[
        \begin{array}{ccccc}
         a_{11} & \cdots & 0  & \cdots & 0\\
         \vdots & & \vdots & & \vdots\\
         a_{i1} & \cdots &  a_{ii} & \cdots & 0\\  
         \vdots & & \vdots & & \vdots\\
         a_{n1} & \cdots & a_{ni} & \cdots & a_{nn}
        \end{array}
        \right] 
        \end{align*}
        
        \
        triangular inferior. Entonces,
        \
        
        \begin{align*}
            A^\t =
        \left[
        \begin{array}{ccccc}
         a_{11} & \cdots & a_{i1} & \cdots & a_{n1}\\
         \vdots & & \vdots & & \vdots\\
         0 & \cdots & a_{ii} & \cdots & a_{ni}\\  
         \vdots & & \vdots & & \vdots\\
         0 & \cdots & 0 & \cdots & a_{nn}
        \end{array}
        \right] 
        \end{align*}
        
        es triangular superior.
        \end{frame}
        


    
        \begin{frame}
            \begin{proposicion}
            
            El determinante de una matriz triangular inferior es igual al producto de los elementos de la diagonal.
             
            \end{proposicion}
            
        \begin{demostracion}
            La transpuesta de una una matriz triangular inferior es una matriz triangular superior.
            
            \
            
            Entonces la proposición es una consecuencia del teorema anterior y la proposición referida al determinante de una triangular superior.
        
            \qed
        \end{demostracion}
        
             
            \end{frame}
            
           
            
            \begin{frame}
            
            
            \begin{observacion}
            La transpuesta transforma filas en columnas y columnas en filas 
            \end{observacion}
            
            \
            
            Gracias a esta observación podemos deducir como cambia el determinante de una matriz al aplicarle ``operaciones elementales por columna''
            \vskip 4cm
            \end{frame}
            
            
            \begin{frame}
            \begin{exampleblock}{Ejemplo}
            Si una matriz tiene una columna con muchos ceros,  podemos intercambiarla con la primera fila.
            \end{exampleblock}
            
            \begin{align*}
            A=\left[
            \begin{array}{cccc}
            1&5&6&1\\
            2&0&7&1\\
            3&0&8&1\\
            4&0&1&1
            \end{array}
            \right]
            \rightarrow
            B=
            \left[
            \begin{array}{cccc}
            5&1&6&1\\
            0&2&7&1\\
            0&3&8&1\\
            0&4&1&1
            \end{array}
            \right]
            \end{align*}
            Entonces
            \begin{align*}
            \det(A)=-\det(B)=-5\det B(1|1)
            \end{align*}
            \end{frame}
            
            \begin{frame}
            \begin{exampleblock}{Ejemplo}
            Si una matriz tiene una fila con muchos ceros, entonces intercambio esta con la primer fila, luego transpongo y calculo el determinante.
            \end{exampleblock}
            
            \begin{align*}
            A= \left[
            \begin{array}{cccc}
            1&2&3&4\\
            5&0&0&0\\
            6&7&8&9\\
            1&1&1&1
            \end{array}
            \right]
            \rightarrow
            B=
             \left[
            \begin{array}{cccc}
            5&0&0&0\\
            1&2&3&4\\
            6&7&8&9\\
            1&1&1&1
            \end{array}
            \right]
            \rightarrow
            B^\t=
             \left[
            \begin{array}{cccc}
            5&1&6&1\\
            0&2&7&1\\
            0&3&8&1\\
            0&4&1&1
            \end{array}
            \right]
            \end{align*}
            Entonces
            \begin{align*}
            \det(A)=-\det(B)=-\det(B^\t)=-5\det B^\t(1|1)
            \end{align*}
            \end{frame}
            
           
            
            
            \begin{frame}
            
            El determinante se puede calcular desarrollando por cualquier columna o fila.
            
            \begin{teorema}
                Sa $A$ matriz $n \times n$, entonces el determinante
            \begin{itemize}
             \item se puede calcular el determinante por la columna $j$ así:
            \begin{align*}
            \det(A)=\sum_{i=1}^n(-1)^{i+j}a_{ij}\det A(i|j),
            \end{align*}
             \item se puede calcular el determinante por la fila $i$ así:
            \begin{align*}
            \det(A)=\sum_{j=1}^n(-1)^{i+j}a_{ij}\det A(i|j)
            \end{align*}
            \end{itemize}
            \end{teorema}
            (la diferencia entre ambas fórmulas es la variable de la sumatoria)
            \end{frame}
            
            \begin{frame}
            La demostración de este teorema (que no la haremos),  se basa en dos resultados que ya mencionamos. 
            \vskip .2cm
            \begin{itemize}
                \item[(A)]  Teorema E3:
                $$
                A \stackrel{{F_r \leftrightarrow F_s}}{\longrightarrow} B  \qquad \Rightarrow   \qquad \det(B)=-\det(A).$$ 
                \item[(B)] $\det(A^\t) = det(A)$, 
            \end{itemize}
            y un resultado que no es difícil demostrar:
            \begin{itemize}
        
                \item[(C)]
                    $ A \stackrel{{C_r \leftrightarrow C_s}}{\longrightarrow} B  \qquad \Leftrightarrow   \qquad A^\t \stackrel{{F_r \leftrightarrow F_s}}{\longrightarrow} B^\t
                     .$
                    \end{itemize}
                    Luego,
                    \begin{itemize}
                \item[(D)]  $\det(B) \stackrel{(B)}{=} \det(B^\t) \stackrel{(A)}{=} -det(A^\t) \stackrel{(B)}{=} -\det(A)$.
            \end{itemize}
           
        \vskip .3cm
            Usando estos resultados, y un poco de manipulación de índices, se obtiene una demostración del teorema. 
           
            \end{frame}

\end{document}

