%\documentclass{beamer} 
\documentclass[handout]{beamer} % sin pausas
\usetheme{CambridgeUS}

\usepackage{etex}
\usepackage{t1enc}
\usepackage[spanish,es-nodecimaldot]{babel}
\usepackage{latexsym}
\usepackage[utf8]{inputenc}
\usepackage{verbatim}
\usepackage{multicol}
\usepackage{amsgen,amsmath,amstext,amsbsy,amsopn,amsfonts,amssymb}
\usepackage{amsthm}
\usepackage{calc}         % From LaTeX distribution
\usepackage{graphicx}     % From LaTeX distribution
\usepackage{ifthen}
%\usepackage{makeidx}
\input{random.tex}        % From CTAN/macros/generic
\usepackage{subfigure} 
\usepackage{tikz}
\usepackage[customcolors]{hf-tikz}
\usetikzlibrary{arrows}
\usetikzlibrary{matrix}
\tikzset{
	every picture/.append style={
		execute at begin picture={\deactivatequoting},
		execute at end picture={\activatequoting}
	}
}
\usetikzlibrary{decorations.pathreplacing,angles,quotes}
\usetikzlibrary{shapes.geometric}
\usepackage{mathtools}
\usepackage{stackrel}
%\usepackage{enumerate}
\usepackage{enumitem}
\usepackage{tkz-graph}
\usepackage{polynom}
\polyset{%
	style=B,
	delims={(}{)},
	div=:
}
\renewcommand\labelitemi{$\circ$}
\usepackage{nicematrix} %https://ctan.dcc.uchile.cl/macros/latex/contrib/nicematrix/nicematrix.pdf

%\setbeamertemplate{background}[grid][step=8 ]
\setbeamertemplate{itemize item}{$\circ$}
\setbeamertemplate{enumerate items}[default]
\definecolor{links}{HTML}{2A1B81}
\hypersetup{colorlinks,linkcolor=,urlcolor=links}


\newcommand{\Id}{\operatorname{Id}}
\newcommand{\img}{\operatorname{Im}}
\newcommand{\nuc}{\operatorname{Nu}}
\newcommand{\im}{\operatorname{Im}}
\renewcommand\nu{\operatorname{Nu}}
\newcommand{\la}{\langle}
\newcommand{\ra}{\rangle}
\renewcommand{\t}{{\operatorname{t}}}
\renewcommand{\sin}{{\,\operatorname{sen}}}
\newcommand{\Q}{\mathbb Q}
\newcommand{\R}{\mathbb R}
\newcommand{\C}{\mathbb C}
\newcommand{\K}{\mathbb K}
\newcommand{\F}{\mathbb F}
\newcommand{\Z}{\mathbb Z}
\newcommand{\N}{\mathbb N}
\newcommand\sgn{\operatorname{sgn}}
\renewcommand{\t}{{\operatorname{t}}}
\renewcommand{\figurename }{Figura}

%
% Ver http://joshua.smcvt.edu/latex2e/_005cnewenvironment-_0026-_005crenewenvironment.html
%

\renewenvironment{block}[1]% environment name
{% begin code
	\par\vskip .2cm%
	{\color{blue}#1}%
	\vskip .2cm
}%
{%
	\vskip .2cm}% end code


\renewenvironment{alertblock}[1]% environment name
{% begin code
	\par\vskip .2cm%
	{\color{red!80!black}#1}%
	\vskip .2cm
}%
{%
	\vskip .2cm}% end code


\renewenvironment{exampleblock}[1]% environment name
{% begin code
	\par\vskip .2cm%
	{\color{blue}#1}%
	\vskip .2cm
}%
{%
	\vskip .2cm}% end code




\newenvironment{exercise}[1]% environment name
{% begin code
	\par\vspace{\baselineskip}\noindent
	\textbf{Ejercicio (#1)}\begin{itshape}%
		\par\vspace{\baselineskip}\noindent\ignorespaces
	}%
	{% end code
	\end{itshape}\ignorespacesafterend
}


\newenvironment{definicion}[1][]% environment name
{% begin code
	\par\vskip .2cm%
	{\color{blue}Definición #1}%
	\vskip .2cm
}%
{%
	\vskip .2cm}% end code

\newenvironment{observacion}[1][]% environment name
{% begin code
	\par\vskip .2cm%
	{\color{blue}Observación #1}%
	\vskip .2cm
}%
{%
	\vskip .2cm}% end code

\newenvironment{ejemplo}[1][]% environment name
{% begin code
	\par\vskip .2cm%
	{\color{blue}Ejemplo #1}%
	\vskip .2cm
}%
{%
	\vskip .2cm}% end code

\newenvironment{ejercicio}[1][]% environment name
{% begin code
	\par\vskip .2cm%
	{\color{blue}Ejercicio #1}%
	\vskip .2cm
}%
{%
	\vskip .2cm}% end code


\renewenvironment{proof}% environment name
{% begin code
	\par\vskip .2cm%
	{\color{blue}Demostración}%
	\vskip .2cm
}%
{%
	\vskip .2cm}% end code



\newenvironment{demostracion}% environment name
{% begin code
	\par\vskip .2cm%
	{\color{blue}Demostración}%
	\vskip .2cm
}%
{%
	\vskip .2cm}% end code

\newenvironment{idea}% environment name
{% begin code
	\par\vskip .2cm%
	{\color{blue}Idea de la demostración}%
	\vskip .2cm
}%
{%
	\vskip .2cm}% end code

\newenvironment{solucion}% environment name
{% begin code
	\par\vskip .2cm%
	{\color{blue}Solución}%
	\vskip .2cm
}%
{%
	\vskip .2cm}% end code



\newenvironment{lema}[1][]% environment name
{% begin code
	\par\vskip .2cm%
	{\color{blue}Lema #1}\begin{itshape}%
		\par\vskip .2cm
	}%
	{% end code
	\end{itshape}\vskip .2cm\ignorespacesafterend
}

\newenvironment{proposicion}[1][]% environment name
{% begin code
	\par\vskip .2cm%
	{\color{blue}Proposición #1}\begin{itshape}%
		\par\vskip .2cm
	}%
	{% end code
	\end{itshape}\vskip .2cm\ignorespacesafterend
}

\newenvironment{teorema}[1][]% environment name
{% begin code
	\par\vskip .2cm%
	{\color{blue}Teorema #1}\begin{itshape}%
		\par\vskip .2cm
	}%
	{% end code
	\end{itshape}\vskip .2cm\ignorespacesafterend
}


\newenvironment{corolario}[1][]% environment name
{% begin code
	\par\vskip .2cm%
	{\color{blue}Corolario #1}\begin{itshape}%
		\par\vskip .2cm
	}%
	{% end code
	\end{itshape}\vskip .2cm\ignorespacesafterend
}

\newenvironment{propiedad}% environment name
{% begin code
	\par\vskip .2cm%
	{\color{blue}Propiedad}\begin{itshape}%
		\par\vskip .2cm
	}%
	{% end code
	\end{itshape}\vskip .2cm\ignorespacesafterend
}

\newenvironment{conclusion}% environment name
{% begin code
	\par\vskip .2cm%
	{\color{blue}Conclusión}\begin{itshape}%
		\par\vskip .2cm
	}%
	{% end code
	\end{itshape}\vskip .2cm\ignorespacesafterend
}







\newenvironment{definicion*}% environment name
{% begin code
	\par\vskip .2cm%
	{\color{blue}Definición}%
	\vskip .2cm
}%
{%
	\vskip .2cm}% end code

\newenvironment{observacion*}% environment name
{% begin code
	\par\vskip .2cm%
	{\color{blue}Observación}%
	\vskip .2cm
}%
{%
	\vskip .2cm}% end code


\newenvironment{obs*}% environment name
	{% begin code
		\par\vskip .2cm%
		{\color{blue}Observación}%
		\vskip .2cm
	}%
	{%
		\vskip .2cm}% end code

\newenvironment{ejemplo*}% environment name
{% begin code
	\par\vskip .2cm%
	{\color{blue}Ejemplo}%
	\vskip .2cm
}%
{%
	\vskip .2cm}% end code

\newenvironment{ejercicio*}% environment name
{% begin code
	\par\vskip .2cm%
	{\color{blue}Ejercicio}%
	\vskip .2cm
}%
{%
	\vskip .2cm}% end code

\newenvironment{propiedad*}% environment name
{% begin code
	\par\vskip .2cm%
	{\color{blue}Propiedad}\begin{itshape}%
		\par\vskip .2cm
	}%
	{% end code
	\end{itshape}\vskip .2cm\ignorespacesafterend
}

\newenvironment{conclusion*}% environment name
{% begin code
	\par\vskip .2cm%
	{\color{blue}Conclusión}\begin{itshape}%
		\par\vskip .2cm
	}%
	{% end code
	\end{itshape}\vskip .2cm\ignorespacesafterend
}








\title[Clase 12 - Subespacios vectoriales]{Álgebra/Álgebra II \\ Clase 12- Subespacios vectoriales.}

\author[]{}
\institute[]{\normalsize FAMAF / UNC
	\\[\baselineskip] ${}^{}$
	\\[\baselineskip]
}
\date[30/04/2024]{30 de abril de 2024}



\begin{document}

\begin{frame}
\maketitle
\end{frame}


\begin{frame}{Resumen}

    En esta clase veremos que
    \begin{itemize}
    \item más ejemplos de subespacios, 
        \item combinaciones lineales de vectores,
        \item vectores generadores de subespacios,  y
        \item Determinación implícita de un subespacio de $\mathbb{K}^n$ a
        partir de generadores
        \item Intersección y suma de subespacios vectoriales . 
    \end{itemize}
    

\

El tema de esta clase  está contenido de la sección 4.2 del apunte de clase ``Álgebra II / Álgebra - Notas del teórico''.
\end{frame}


    \begin{frame}{Ejemplos de subespacio vectoriales}
    
    
        \begin{enumerate}     
            \item[4.] Sean $V=\K^n$ y $1\le j \le n$. Definimos 
            $$
            W = \left\{ (x_1,x_2,\ldots,x_n): x_i \in \K\; (1 \le i \le n), x_j =0\right\}.
            $$
            Es decir $W$  es el subconjunto de $V$ de todas las $n$-tuplas con la coordenada $j$ igual a 0. Por ejemplo  si $j=1$ 
            $$
            W = \left\{ (0,x_2,\ldots,x_n): x_i \in \K \;(2 \le i \le n)\right\}.
            $$
            \pause
            Veamos que este último es un subespacio.
            
            Si $(0,x_2,\ldots,x_n), (0,y_2,\ldots,y_n) \in W$ y  $\lambda \in \K$,  entonces
            $$(0,x_2,\ldots,x_n)+ \lambda(0,y_2,\ldots,y_n) = (0,x_2+\lambda y_2,\ldots,x_n+\lambda y_n) \in W.$$
            
            
            La demostración para $j >1$ es completamente análoga. 
        \end{enumerate}

    
    \end{frame}
    

    
    
    

    \begin{frame}
        \begin{enumerate}     
            \item[5.] Sea $\operatorname{Sim}_n(\K) = \left\{A \in M_n(\K): A^\t = A\right\}$.    \pause
            \vskip .2cm 
            Es claro que: \quad   $A \in \operatorname{Sim}_n(\K)$ $\Leftrightarrow$  $[A]_{ij} = [A]_{ji}\; \forall i,j $.     \pause
            \vskip .3cm
            \begin{proposicion}
                $A \in \operatorname{Sim}_n(\K)$ es subespacio de $M_n(\K)$
            \end{proposicion}    \pause
            \begin{demostracion}    \pause
            Sean $A=[a_{ij}]$, $B= [b_{ij}]$ tales que $A=A^\t$ y $B=B^\t$ y sea  $\lambda \in \K$, entonces debemos verificar que:\quad  $A+ \lambda B \in  \operatorname{Sim}_n(\K)$.
                \begin{align*}
                    [(A+\lambda B)^\t]_{ij} &=  [(A+\lambda B)]_{ji}&& (\text{definición de transpuesta}) \\
                    &= [A]_{ji} + \lambda [B]_{ji}&& (\text{def. de suma y prod. por escalar}) \\
                    &= [A]_{ij} + \lambda [B]_{ij}&& (\text{$A$ y $B$ simétricas}) \\
                    &= [A+ \lambda B]_{ij}&& (\text{def. de suma y prod. por escalar}) \\
                \end{align*}
        Luego $A+\lambda B\in \operatorname{Sim}_n(\K) $.\qed
            \end{demostracion}
        \end{enumerate}
    \end{frame}
    
    

    
    

    \begin{frame}
        \begin{enumerate}     
        \item[6.] El  conjunto $\R[x] = \left\{P(x): P(x) \text{ es polinomio en $\R$} \right\}$, es subespacio de $\operatorname{F}(\R)$, pues $\R[x] \subset \operatorname{F}(\R)$ y las operaciones de suma y producto por un escalar son cerradas en $\R[x]$.
        \vskip .4cm     \pause
        \item[7.] Sea $\operatorname{C}(\R)$ las funciones continuas de $\R$ en $\R$. Entonces, $\operatorname{C}(\R)$ es subespacio de $\operatorname{F}(\R)$.     \pause
        \begin{demostracion}
            Sean $f,g$ funciones continuas, es decir $\lim_{x \to a} f(x) = f(a)$ y $\lim_{x \to a} g(x) = g(a)$, $\forall a \in \R$. Sea $\lambda \in \R$.  Por las propiedades de los límites 
            $$
            \lim_{x \to a} (f + \lambda g)(x) =  \lim_{x \to a} f + \lambda \lim_{x \to a} g(x) = f(a) + \lambda g(a) = 
            (f + \lambda g)(a)
            $$\qed
        \end{demostracion} 
        
        \pause
        De forma análoga, el conjunto $\R[x]$ es subespacio de $\operatorname{C}(\R)$.
        \end{enumerate}
    \end{frame}
    
\begin{frame}{Combinaciones lineales}
    \begin{definicion}
        Sea $V$ espacio vectorial sobre $\K$ y $v_1,\ldots,v_n$ vectores en $V$. Dado $v \in V$, diremos que\textit{ $v$  es combinación lineal de los $v_1,\ldots,v_n$ }\index{combinación lineal} si existen escalares $\lambda_1,\ldots,\lambda_n$ en $\K$,  tal que 
        $$
        v = \lambda_1v_1+\cdots+\lambda_nv_n.
        $$
    \end{definicion}
    
    \vskip 2cm
\end{frame}


\begin{frame}
    

\begin{ejemplo}
    
    Sean $v_1 = (1,0)$, $v_2 = (0,1)$ en $\C^2$ ¿es $v = (i,2)$ combinación lineal de $v_1,v_2$?    \pause La respuesta es sí, pues 
        $$
        v = iv_1+2v_2.
        $$    \pause
        Observar además que es la única combinación lineal posible, pues si 
        $$
        v = \lambda_1v_1+ \lambda_2 v_2,
        $$
        entonces
        $$
        (i,2) = (\lambda_1,0)+(0,\lambda_2) = (\lambda_1,\lambda_2),
        $$
        luego $\lambda_1=i$ y $\lambda_2= 2$.
        

\end{ejemplo}
\end{frame}


\begin{frame}

\begin{ejemplo}
    Puede ocurrir que un vector sea combinación lineal de otros vectores de varias formas diferentes. Por ejemplo,   si   $v = (i,2)$ y $v_1 = (1,0)$, $v_2 = (0,1)$, $v_3 = (1,1)$,  tenemos que
        \begin{align*}
            v &= iv_1+2v_2+0v_3,\quad\quad\quad\text{y también}\\
            v &= (i-1)v_1 + v_2 + v_3.  
        \end{align*}
\end{ejemplo}
\pause
\begin{ejemplo}
        Sean $(0,1,0)$, $(0,1,1)$ en $\C^3$ ¿es $(1,1,0)$ combinación lineal de $(0,1,0)$, $(0,1,1)$? La respuesta es no, pues si 
        $$
        (1,1,0) = \lambda_1(0,1,0)+ \lambda_2(0,1,1) = (0,\lambda_1,0)+ (0,\lambda_2,\lambda_2) = (0,\lambda_1+\lambda_2,\lambda_2),
        $$ 
        luego, la primera coordenada nos dice que $1=0$, lo cual es absurdo. 
        
\end{ejemplo}
    
    
\end{frame}


\begin{frame}
    \begin{ejemplo}\label{vector-en-subespacio}
        Demostrar que $(7, 5, 4)$  es combinación lineal de los vectores $(1,-5,2), (1,-1,1)$ y escribir la combinación lineal explícita.
    \end{ejemplo} 
    
    \begin{solucion}
        
    
    Planteamos la ecuación:
    \begin{align*}
        (7, 5, 4) &= \lambda_1(1,-5,2)+\lambda_2 (1,-1,1)
        \\
        &= (\lambda_1+\lambda_2,-5\lambda_1-\lambda_2,2\lambda_1+\lambda_2).
    \end{align*}
    Por consiguiente,  esta ecuación se resuelve con el siguiente sistema de ecuaciones
    \begin{align*}
        \lambda_1+\lambda_2 &= 7 \\
        -5\lambda_1-\lambda_2&= 5 \\
        2\lambda_1+\lambda_2 &= 4.
    \end{align*}

    \end{solucion}
    \end{frame}


    \begin{frame}
    Ahora bien, usando el método de Gauss
    \begin{align*}
        &\left[\begin{array}{rr|r}
        1 & 1 & 7 \\ -5 & -1 & 5 \\	2 & 1 & 4
        \end{array}\right] 
        \underset{F3-2F_1}{\stackrel{F_2+5F_1}{\longrightarrow}} 
        \left[\begin{array}{rr|r}
        1 & 1 & 7 \\ 0 & 4 & 40 \\	0 & -1 & -10
        \end{array}\right]\stackrel{F_2/4}{\longrightarrow}
        \left[\begin{array}{rr|r}
        1 & 1 & 7 \\ 0 & 1 & 10 \\	0 & -1 & -10
        \end{array}\right] \\
        &\stackrel{F_3+F_2}{\longrightarrow}
        \left[\begin{array}{rr|r}
        1 & 1 & 7 \\ 0 & 1 & 10 \\	0 & 0 & 0
        \end{array}\right] \stackrel{F_1-F_2}{\longrightarrow}
        \left[\begin{array}{rr|r}
        1 & 0 & -3 \\ 0 & 1 & 10 \\	0 & 0 & 0
        \end{array}\right].
    \end{align*}	
        Luego $\lambda_1= -3$ y $\lambda_2 = -10$,  es decir,
        \begin{align*}
        (7, 5, 4) &= -3(1,-5,2)+10 (1,-1,1).
        \end{align*} 

        \qed


\end{frame}
    
\begin{comment}

\begin{frame}

\definecolor{airforceblue}{rgb}{0.36, 0.54, 0.66}
\hfsetfillcolor{airforceblue!30}
\hfsetbordercolor{blue!10}


    \begin{block}{Observación (muy importante)}\label{obs-muy-importante}
        ¿Es $v =(b_1,\ldots,b_m) \in \K^m$  c. l. de vectores $v_1,\ldots,v_n \in \K^m$? 
        \vskip .3cm     \pause
        Sea $v_i = (a_{1i},\ldots,a_{mi})$ $(1 \le i \le n)$, entonces $v = \lambda_1v_1 + \cdots +\lambda_nv_n$ $\Rightarrow$
        \begin{align*}
            (b_1,\ldots,b_m) &= \lambda_1(a_{11},\ldots,a_{m1}) + \cdots +\lambda_n(a_{1n},\ldots,a_{mn}) \\
            &= (\lambda_1a_{11} + \cdots+ \lambda_na_{1n}, \,\ldots,\, \lambda_1a_{m1} + \cdots+ \lambda_na_{mn}).
        \end{align*}    \pause
        Luego, 
        \begin{center}
            \tikzmarkin{a}(0.2,-0.3)(-0.2,0.5) {$v$  es combinación lineal de los vectores $v_1,\ldots,v_n \in \K^m$} \tikzmarkend{a}
        \end{center}
            si y sólo si tiene solución el siguiente sistema de ecuaciones: 
        \begin{equation*}
            \tikzmarkin{b}(0.2,-0.8)(-0.2,1) 
        \begin{matrix}
        a_{11}\lambda_1& + &a_{12}\lambda_2& + &\cdots& + &a_{1n}\lambda_n &= &b_1\\
        \vdots&  &\vdots& &&  &\vdots \\
        a_{m1}\lambda_1& + &a_{m2}\lambda_2& + &\cdots& + &a_{mn}\lambda_n &=&b_m.
        \end{matrix}
        \tikzmarkend{b}
        \end{equation*}
        
    \end{block}
    
    
\end{frame}
    

\begin{frame}
    
    \begin{ejemplo} Demostrar que $(5,12,5)$  es combinación lineal de los vectores $(1,-5,2), (0,1,-1), (1,2,-1)$. 
    \end{ejemplo}    \pause
        \vskip .2cm
        
        \begin{solucion}  \pause
            Planteamos la ecuación:
        \begin{align*}
            (5,12,5) &= \lambda_1(1,-5,2)+\lambda_2 (0,1,-1)+\lambda_3 (1,2,-1) \\
            &= (\lambda_1,-5\lambda_1,2\lambda_1)+ (0,\lambda_2,-\lambda_2)+(\lambda_3 ,2\lambda_3 ,-\lambda_3 )	\\
            &= (\lambda_1+\lambda_3,-5\lambda_1+\lambda_2+2\lambda_3,2\lambda_1-\lambda_2-\lambda_3).
        \end{align*}  \pause
        Por consiguiente,  esta ecuación se resuelve con el siguiente sistema de ecuaciones
        \vskip -.6cm
        \begin{align*}
            \lambda_1+\lambda_3 &= 5 \\
            -5\lambda_1+\lambda_2+2\lambda_3 &= 12 \\
            2\lambda_1-\lambda_2-\lambda_3 &= 5.
        \end{align*}
        \end{solucion}
        
    


\end{frame}


\begin{frame}
    Ahora bien, usando el método de Gauss
    \begin{multline*}
\left[\begin{array}{rrr|r}1 & 0 & 1 &  5 \\ -5 & 1 & 2 &  12 \\	2 & -1 & -1 &  5  \end{array}\right]
\stackrel[F_3 -2F_1]{F_2 + 5F_1}{\longrightarrow} 
\left[\begin{array}{rrr|r}1 & 0 & 1 &  5 \\ 0 & 1 & 7 &  37 \\	0 & -1 & -3 &  -5  \end{array}\right]
\\ \stackrel{F_3+F_1}{\longrightarrow}
\left[\begin{array}{rrr|r}1 & 0 & 1 &  5 \\ 0 & 1 & 7 &  37 \\	0 & 0 & 4 & 32  \end{array}\right]
\stackrel{F_3/4}{\longrightarrow} 
\left[\begin{array}{rrr|r}1 & 0 & 1 & 5 \\ 0 & 1 & 7 & 37 \\	0 & 0 & 1& 8  \end{array}\right]
\\ \stackrel[F_2 -7F_3]{F_1 - F_3}{\longrightarrow}
\left[\begin{array}{rrr|r}1 & 0 & 0 & -3 \\ 0 & 1 & 0 &  -19 \\	0 & 0 & 1& 8  \end{array}\right].
\end{multline*}	  \pause
Luego $\lambda_1= -3$, $\lambda_2 = -19$ y $\lambda_3=8$,  es decir
\begin{align*}
(5,12,5) &= -3(1,-5,2)-19 (0,1,-1)+8 (1,2,-1).\qed
\end{align*}
\end{frame}
                    
\begin{frame}
    \begin{proposicion}
        Sea $W$ subespacio de $V$ y $w_1,\ldots,w_k \in W$,  entonces cualquier combinación lineal de los $w_1,\ldots,w_k$ pertenece a $W$.
    \end{proposicion}  \pause

    \begin{demostracion}  \pause
        Debemos probar que, para cualesquiera $\lambda_1,\ldots,\lambda_k \in \K$, se cumple que  $\lambda_1w_1+\cdots+\lambda_kw_k \in W$. 
            \vskip .2cm
            
            Ahora bien,  como $W$ es subespacio, $\lambda_iw_i \in W$ para $1\le i \le k$. 
            
            \vskip .2cm
            Por un argumento inductivo, como $W$  es subespacio, no es difícil probar que la suma de $k$ términos en $W$ es un elemento de $W$, por lo tanto $\lambda_1w_1+\cdots+\lambda_kw_k \in W$. 
            
            \qed
    \end{demostracion}
            
\end{frame}

\end{comment}                    
    

\begin{frame}

\begin{teorema}
    Sea $V$ un espacio vectorial sobre $\K$ y sean $v_1,\ldots,v_k \in V$. Entonces
    $$
    W = \{\lambda_1v_1+\cdots+\lambda_kv_k: \lambda_1,\ldots,\lambda_k \in \K \}
    $$
    es un subespacio vectorial. Es decir,  el conjunto de las combinaciones lineales de $v_1,\ldots,v_k$ es un subespacio vectorial.
    \end{teorema}  \pause
    \begin{proof}  \pause
                Sean $\lambda_1v_1+\cdots+\lambda_kv_k$, $\mu_1v_1+\cdots+\mu_kv_k$ dos combinaciones lineales de $v_1,\ldots,v_k$ y $\lambda \in \K$, entonces 
            \begin{align*}
                (\lambda_1v_1+\cdots+\lambda_kv_k)&+\lambda (\mu_1v_1+\cdots+\mu_kv_k) \\&=  \lambda_1v_1+\lambda\mu_1v_1+\cdots+\lambda_kv_k+\lambda\mu_kv_k\\
                &= (\lambda_1+\lambda\mu_1)v_1+\cdots+(\lambda_k+\lambda\mu_k)v_k,
            \end{align*}
            que es  una combinación lineal de  $v_1,\ldots,v_k$ y por lo tanto pertenece a $W$. 
            
            \qed
    \end{proof}
    
\end{frame}

        

\begin{frame}

\begin{definicion}
    Sea $V$ un espacio vectorial sobre $\K$ y sean $v_1,\ldots,v_k \in V$. Al  subespacio vectorial $	W = \{\lambda_1v_1+\cdots+\lambda_kv_k: \lambda_1,\ldots,\lambda_k \in \K \}$ de las combinaciones lineales de $v_1,\ldots,v_k$ se lo denomina \textit{subespacio generado por $v_1,\ldots,v_k$}\index{subespacio generado} y se lo denota  
    \begin{equation*}
        W \; = \; \langle v_1,\ldots,v_k \rangle \; = \; \operatorname{gen}\left\{ v_1,\ldots,v_k\right\}  \; = \;  \operatorname{span}\left\{ v_1,\ldots,v_k\right\}.
    \end{equation*}  \pause
    Además, en este caso, diremos que el conjunto $S = \left\{ v_1,\ldots,v_k \right\}$ \textit{genera} al subespacio $W$ o que los vectores $v_1,\ldots,v_k$ \textit{generan} $W$. 
    \vskip .2cm 
\end{definicion}  \pause
    \begin{observacion}
        Un caso especial,  que será de suma importancia,  es el caso en que consideramos todo $V$. 

        \vskip .2cm
        \pause
        Estudiaremos en las clases que siguen conjuntos de generadores de $V$ llamados \textit{bases}, que tienen la propiedad de que todo vector de $V$  se escribe de una única forma como c.l.  de los generadores.  
        
    \end{observacion}

\end{frame}



\begin{frame}{Determinación ``implícita'' de un subespacio de $\K^n$}
    
    En  general,  si queremos  averiguar si  un vector concreto $(b_1,b_2, \ldots,b_m) \in \K^m$ es combinación lineal de vectores $v_1,\ldots,v_n \in \K^m$,  debemos plantear la ecuación
\begin{equation*}
    (b_1,b_2, \ldots,b_m) = \lambda_1v_1+\cdots+\lambda_nv_n, \tag{*}
\end{equation*}
y resolver el sistema correspondiente, así como lo hicimos en el ejemplo de  la página \ref{vector-en-subespacio}.

\vskip .4cm

Es decir, si 
$$
W = \{\lambda_1v_1+\cdots+\lambda_nv_n: \lambda_1,\ldots,\lambda_n \in \K \},
$$
queremos averiguar si el vector $(b_1,b_2, \ldots,b_m) \in \K^m$  pertenece a $W$ o, equivalentemente,  si es combinación lineal de $v_1,\ldots,v_n$.:

    \end{frame}
    

\begin{frame}
    
Ahora, si
\begin{align*}
    v_1&=(a_{11},a_{21},\ldots,a_{m1}),\\
    v_2&=(a_{12},a_{22},\ldots,a_{m2}),\\
    &\vdots\\
    v_n&=(a_{1n},a_{2n},\ldots,a_{mn}),
\end{align*}
la ecuación (*) de la página anterior se traduce en el sistema de ecuaciones
\begin{equation*}
    \begin{matrix}
    a_{11}\lambda_1& + &a_{12}\lambda_2& + &\cdots& + &a_{1n}\lambda_n &= &b_1\\
    \vdots&  &\vdots& &&  &\vdots \\
    a_{m1}\lambda_1& + &a_{m2}\lambda_2& + &\cdots& + &a_{mn}\lambda_n &=&b_m.
    \end{matrix}
\end{equation*}

\end{frame}


\begin{frame}
    De la matriz ampliada original $\left[\begin{array}{r|r}A&b\end{array}\right]$ podemos obtener una MERF equivalente $\left[\begin{array}{r|r}A'&b'\end{array}\right]$ y el sistema asociado. 
    
    \vskip .4cm

    \begin{equation*}
        \begin{matrix}
        &x_{k_1}& + &\sum_{j \not= k_1,\ldots, k_r} a'_{1j}\,x_j&= &b'_1\\
        &x_{k_2}& + &\sum_{j \not= k_1,\ldots, k_r} a'_{2j}\,x_j&= &b'_2\\
        & \vdots& &  &\vdots \\
        &x_{k_r}& + &\sum_{j \not= k_1,\ldots, k_r} a'_{rj}\,x_j&= &b'_r\\
        &&  &0&= &b'_{r+1}\\
        &&  &\vdots &\\
        &&  &0&= &b'_{m},\\
        \end{matrix}
    \end{equation*} 
    donde $k_1 < k_2 < \cdots < k_r$.
    
\end{frame}

\begin{frame}
    Por lo tanto, el sistema tiene solución si y solo si  $b'_{r+1} = \cdots = b'_m =0$. Luego, 
    \begin{equation*}
        W = \{(b_1,b_2, \ldots,b_m) \in \K^m: b'_{r+1} = \cdots = b'_m =0\}.
    \end{equation*}
    Las ecuaciones $ b'_{r+1} = \cdots = b'_m =0$ son las ecuaciones implícitas que definen a $W$ y nos permiten decidir rápidamente si un vector pertenece o no a $W$, simplemente viendo si sus coordenadas satisfacen las ecuaciones.

\end{frame}

\begin{frame}
    \begin{ejemplo}
        Caracterizar mediante ecuaciones el subespacio del subespacio generado por 
        \begin{align*}
            v_1=(3,1,2,-1),&\quad
            v_2=(6,2,4,-2),\\
            v_3=(3,0,1,1),&\quad
            v_4=(15,3,8,-1).
            \end{align*}
    \end{ejemplo}

    \pause
    
    \begin{solucion}
        

    
    En otras palabras, queremos describir implícitamente el conjunto de los $b=(b_1,b_2,b_3,b_4)
    \in\R^4$ tales que $b\in\langle v_1, v_2, v_3, v_4\rangle$.
    
    \

    \pause
    O sea, los $b=(b_1,b_2,b_3,b_4)
    \in\R^4$ tales que 
    \begin{equation*}
        b=\lambda_1v_1+\lambda_2v_2+\lambda_3v_3+\lambda_4v_4 \tag{*}
    \end{equation*}
    
    
    con $\lambda_1,\lambda_2,\lambda_3,\lambda_4\in\R$.

    \end{solucion}

    \end{frame}
    

    \begin{frame}
        Planteemos la fórmula (*) en como un sistema de ecuaciones. Obtenemos:
        \begin{align*}
            b_1&=3\lambda_1+6\lambda_2+3\lambda_3+15\lambda_4\\
            b_2&=\lambda_1+2\lambda_2+3\lambda_4\\
            b_3&=2\lambda_1+4\lambda_2+ \lambda_3+8\lambda_4\\
            b_4&=-\lambda_1-2\lambda_2+\lambda_3-\lambda_4
        \end{align*}
        
        \pause Luego, escrito como producto de matrices, el sistema es

        \begin{equation*}
            \begin{bmatrix*}[r]    
        3&6&3&15\\
        1&2&0&3\\
        2&4&1&8\\
        -1&-2&1&-1
    \end{bmatrix*}
        \begin{bmatrix*}[r]
        \lambda_1\\
        \lambda_2\\
        \lambda_3\\
        \lambda_4
    \end{bmatrix*}
        =
        \begin{bmatrix*}[r]
        b_1\\
        b_2\\
        b_3\\
        b_4 
    \end{bmatrix*}
    \end{equation*}

    \end{frame}

    \begin{frame}
    
    Resolvamos  el sistema anterior.  \pause

    \begin{align*}
        &\left[\begin{array}{cccc|c}
        3&6&3&15&b_1\\
        1&2&0&3&b_2\\
        2&4&1&8&b_3\\
        -1&-2&1&-1&b_4
        \end{array}\right] 
        \underset{F_4+F_2}{\underset{F_3-2F_2}{\stackrel{F1-3F_2}{\longrightarrow}}} 
        \left[\begin{array}{cccc|c}
        0&0&3&6&b_1-3b_2\\
        1&2&0&3&b_2\\
        0&0&1&2&b_3-2b_2\\
        0&0&1&2&b_4+b_2
        \end{array}\right] \\
        &\underset{F_4-F_3}{\stackrel{F_1-3F_3}{\longrightarrow}}
        \left[\begin{array}{cccc|c}
        0&0&0&0&b_1+3b_2-3b_3\\
        1&2&0&3&b_2\\
        0&0&1&2&b_3-2b_2\\
        0&0&0&0&3b_2-b_3+b_4
        \end{array}\right].
    \end{align*}
    \pause Luego el sistema tiene solución si y solo si $b_1+3b_2-3b_3=0$ y $3b_2-b_3+b_4=0$. Por lo tanto, el subespacio que estamos buscando es
    \begin{multline*}
        \langle v_1, v_2, v_3, v_4\rangle=\\\{(b_1,b_2,b_3,b_4)\in\mathbb{R}^4\mid b_1+3b_2-3b_3=0,3b_2-b_3+b_4=0\}.\qed
    \end{multline*}
    
    \end{frame}
    
    
    
    \begin{frame}
    Notemos que podemos repetir todo el razonamiento anterior para cualesquiera vectores $v_1, ..., v_k$ en cualquier $\R^n$ y cualquier $b\in\R^n$.\pause
    
    \
    
    Sólo hay que tener presente que multiplicar una matriz por un vector columna es lo mismo que hacer una combinación lineal de las columnas de la matriz:\pause
    
    \
    
    Es decir, si
    $$
    A=\left[
    \begin{array}{cccc}
    \mid& \mid& &\mid\\
    v_1 & v_2 & \cdots &v_k\\
    \mid& \mid& &\mid
    \end{array}
    \right]
    ,$$ 
    entonces
    \begin{align*}
    A\left[
    \begin{array}{c}
    \lambda_1\\\vdots\\\lambda_k
    \end{array}
    \right]=
    \lambda_1v_1+\cdots+\lambda_kv_k
    \end{align*}
    \end{frame}
    
    \begin{frame}
    
    \begin{block}{Conclusión}
        
        Sean $v_1, ..., v_k\in\K^n$ y $A\in\K^{n\times k}$ la matriz cuyas  columnas son los vectores $v_1, ..., v_k$, es decir
    $$
    A=\left[
    \begin{array}{cccc}
    \mid& \mid& &\mid\\
    v_1 & v_2 & \cdots &v_k\\
    \mid& \mid& &\mid
    \end{array}
    \right]
    .$$\pause
    
    Entonces
    \begin{itemize}
        \item El subespacio vectorial  $\langle v_1, ..., v_k\rangle$ es igual al conjunto de los $b\in\K^n$ para los cuales el sistema $AX=b$ tiene solución.
        \vskip .2cm
        \item Las ecuaciones vienen dadas por las filas nulas de la MERF equivalente a $A$. En particular, si no tiene filas nulas entonces $\langle v_1, ..., v_k\rangle=\K^n$ porque el sistema $AX=b$ siempre tiene solución.
    \end{itemize}
\end{block}
    \end{frame}
    




\begin{frame}{Intersección y suma de subespacios vectoriales}

    \begin{teorema}\label{th-interseccion}
        Sea $V$ un espacio vectorial sobre $\K$. Entonces la intersección de subespacios vectoriales es un subespacio vectorial. 
    \end{teorema}  \pause
\begin{proof}  \pause
    Veamos el caso de la intersección de dos subespacios. 
    \vskip .2cm
    Debemos probar que si $W_1$, $W_2$ subespacios $\Rightarrow$ $W_1 \cap W_2$  es subespacio. 
    \vskip .2cm
    Observemos:\quad  $w \in W_1 \cap W_2 \quad \Leftrightarrow\quad w \in W_1 \;\wedge\; w \in W_2$.
\begin{align*}
\text{ Sea  $\lambda \in \K$. } u,v \in  W_1 \cap W_2\quad &\Rightarrow \quad u,v \in W_1 \;\wedge\; u,v \in W_2 \\
&\Rightarrow \quad u+\lambda v \in W_1 \;\wedge\; u+\lambda v\in W_2\qquad\qquad\qquad\\
&\Rightarrow\quad  u+\lambda v  \in W_1 \cap W_2. 
\end{align*}

Luego $W_1 \cap W_2$  es subespacio.
\qed
\end{proof}
\end{frame}



\begin{frame}
\begin{ejemplo}
    Sean 
    \begin{align*}
        W_1 = \left\{ (x,y,z):  -3x + y + 2z = 0\right\}
    \end{align*}
    y
    \begin{align*}
            W_2 = \left\{ (x,y,z):  x - y + 2z = 0\right\}.
    \end{align*}
    Encontrar generadores de $W_1 \cap W_2$. 
\end{ejemplo}  \pause
\begin{solucion}  \pause
    Es claro que
    \begin{equation*}
        W_1 \cap W_2 = \left\{ (x,y,z):  -3x + y + 2z = 0 \;\wedge \;  x - y + 2z = 0\right\}.
    \end{equation*}
\end{solucion}

\end{frame}

\begin{frame}
Por lo tanto debemos resolver el sistema de ecuaciones
\begin{equation*}
    \begin{cases}
    -3x + y + 2z = 0 \\  x - y + 2z = 0
    \end{cases}
\end{equation*}\pause
Reduzcamos la matriz del sistema a una MRF:
\begin{align*}
    \begin{bmatrix}
    -3&1&2 \\  1&-1&2
    \end{bmatrix} &
    \stackrel{F_1+3F_2}{\longrightarrow}
    \begin{bmatrix}
    0&-2&8 \\  1&-1&2
    \end{bmatrix} 
    \stackrel{F_1/(-2)}{\longrightarrow}
    \begin{bmatrix}
    0&1&-4 \\  1&-1&2
    \end{bmatrix} 
    &\stackrel{F_2+F_1}{\longrightarrow}
    \begin{bmatrix}
    0&1&-4 \\  1&0&-2
    \end{bmatrix} 
\end{align*}\pause

Por lo tanto, $x_2 -4x_3 =0$ y $ x_1 -2x_3 =0$,  es decir  $x_2 = 4x_3$ y $ x_1 =2x_3 $. 
\vskip .2cm\pause
Luego, 
\begin{equation*}
W_1 \cap W_2 = \left\{ (2t,4t,t):  \t \in \R\right\} =  \left\{ t(2,4,1):  \t \in \R\right\}.
\end{equation*}\pause

La respuesta es entonces: $(2,4,1)$ es generador $ W_1 \cap W_2$. 




\qed

\end{frame}
    

    

\begin{frame}
\begin{teorema}
    Sea $V$ un espacio vectorial sobre $\K$ y sean $v_1,\ldots,v_k \in V$. Entonces,  la intersección de todos los subespacios vectoriales que contienen  a $v_1,\ldots,v_k$ es igual a $\langle v_1,\ldots,v_k \rangle$.	
\end{teorema}	  \pause
\begin{proof}  \pause
    Denotemos 
    \begin{itemize}
        \item  $U= \bigcap$ de todos los subespacios vectoriales $\supseteq$ $\{ v_1,\ldots,v_k\}$.
    \end{itemize}
    \vskip .3cm  \pause
    Probaremos que  $U = \langle v_1,\ldots,v_k \rangle$ con la doble inclusión,  es decir probando que 
    $$U \subseteq  \langle v_1,\ldots,v_k \rangle \quad \text{y} \quad  \langle v_1,\ldots,v_k \rangle \subseteq U.$$
\end{proof}
                
\end{frame}

    

\begin{frame}
    
($U\subseteq\langle v_1, ..., v_k\rangle$) \;   \pause
\vskip .3cm
Primero, $U\subseteq\langle v_1, ..., v_k\rangle$ vale puesto que  $\langle v_1, ..., v_k\rangle$ es un subespacio que contiene a $\{v_1, ..., v_k\}$.
\pause
\vskip .6cm
    
    
    
($\langle v_1,\ldots,v_k \rangle \subseteq U$) \;   \pause
\vskip .3cm
$U$  es intersección de subespacios $\Rightarrow$ (teor. p. \ref{th-interseccion}) $U$ es un subespacio.  
\vskip .3cm  \pause
Luego, $\{v_1, ..., v_k\}\subset U$  $\Rightarrow$ $\lambda_1v_1+\cdots+\lambda_kv_k\in U$, \; $\forall\,\lambda_1, ..., \lambda_k\in\K$.
\vskip .3cm  \pause
Por lo tanto $\langle v_1, ..., v_k\rangle\subseteq U.$ \qed

    
\end{frame}


\begin{frame}
\begin{observacion}
    Si $V$ es un $\K$-espacio vectorial, $S$ y $T$ subespacios de $V$.   \pause
    \vskip .3cm
    Entonces $S \cup T$ \textit{no es necesariamente un subespacio} de $V$. 
    \vskip .3cm  \pause
    

    En efecto, consideremos en $\R^2$ los subespacios 
    $$S = \R(1,0)\quad \text{ y } T = \R(0,1).$$ 	
    \pause
    \begin{itemize}
        \item  $(1,0)\in  S$ y $(0,1) \in  T$ $\Rightarrow$  $(1,0), (0,1) \in  S \cup T$. \vskip .2cm
        \pause             \item Ahora bien $(1,0) + (0,1) = (1,1) \not\in S \cup T$, puesto que $(1,1) \not\in S$ y $(1,1) \not\in T$.
    \end{itemize}
\end{observacion}
\end{frame}
        

    
\begin{frame}
    \begin{definicion} Sea $V$ un espacio vectorial sobre $\K$ y sean $S_1,\ldots,S_k$ subconjuntos  de $V$.
        definimos 
        \begin{equation*}
            S_1+  \cdots +S_k := \left\{s_1+\cdots+s_k: s_i \in S_i, 1 \le i \le k \right\},
        \end{equation*}
        el conjunto \textit{suma de los  $S_1,\ldots,S_k$.}
    \end{definicion}	
    \pause
    \vskip .8cm
    \begin{teorema}
        Sea $V$ un espacio vectorial sobre $\K$ y sean $W_1,\ldots,W_k$ subespacios  de $V$. Entonces $W= W_1+\cdots+W_k$ es un subespacio de $V$.
    \end{teorema}  \pause
    \begin{demostracion}  \pause
        Ejercicio (ver apunte). \qed
    \end{demostracion}
    \end{frame}
    
    
    
    \begin{frame}
    \begin{proposicion}
        Sea $V$ un espacio vectorial sobre $\K$ y sean $v_1,\ldots,v_r$ elementos de   de $V$. Entonces
        \begin{equation*}
            \langle v_1,\ldots,v_r \rangle = \langle v_1 \rangle+ \cdots + \langle v_r \rangle.
        \end{equation*}
    \end{proposicion}  \pause
    \begin{proof}  \pause
        Probemos el resultado viendo que los dos conjuntos se incluyen mutuamente.  \pause
        
        ($\subseteq$) Sea $w \in \langle v_1,\ldots,v_r \rangle$, luego $w = \lambda_1 v_1 +\cdots+ \lambda_r v_r$. Como $ \lambda_i v_i \in \langle v_i \rangle$, $1 \le i \le r$ ,  tenemos que  $w \in \langle v_1 \rangle+ \cdots + \langle v_r \rangle$.  En  consecuencia, $\langle v_1,\ldots,v_r \rangle \subseteq \langle v_1 \rangle+ \cdots + \langle v_r \rangle$.   \pause
        
        ($\supseteq$) Si $w \in \langle v_1 \rangle+ \cdots + \langle v_r \rangle$, entonces $w = w_1 + \cdots+w_r$ con $w_i \in \langle v_i\rangle$ para todo $i$. Por lo tanto, $w_i = \lambda_i v_i$ para algún $\lambda_i \in \K$ y  $w = \lambda_1 v_1 +\cdots+ \lambda_r v_r \in \langle v_1,\ldots,v_r \rangle $. En  consecuencia, $\langle v_1 \rangle+ \cdots + \langle v_r \rangle \subseteq \langle v_1,\ldots,v_r \rangle$. \qedhere
    \end{proof}
    
    \end{frame}
    

\end{document}

