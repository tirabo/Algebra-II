%\documentclass{beamer} 
\documentclass[handout]{beamer} % sin pausas
\usetheme{CambridgeUS}

\usepackage{etex}
\usepackage{t1enc}
\usepackage[spanish,es-nodecimaldot]{babel}
\usepackage{latexsym}
\usepackage[utf8]{inputenc}
\usepackage{verbatim}
\usepackage{multicol}
\usepackage{amsgen,amsmath,amstext,amsbsy,amsopn,amsfonts,amssymb}
\usepackage{amsthm}
\usepackage{calc}         % From LaTeX distribution
\usepackage{graphicx}     % From LaTeX distribution
\usepackage{ifthen}
%\usepackage{makeidx}
\input{random.tex}        % From CTAN/macros/generic
\usepackage{subfigure} 
\usepackage{tikz}
\usepackage[customcolors]{hf-tikz}
\usetikzlibrary{arrows}
\usetikzlibrary{matrix}
\tikzset{
	every picture/.append style={
		execute at begin picture={\deactivatequoting},
		execute at end picture={\activatequoting}
	}
}
\usetikzlibrary{decorations.pathreplacing,angles,quotes}
\usetikzlibrary{shapes.geometric}
\usepackage{mathtools}
\usepackage{stackrel}
%\usepackage{enumerate}
\usepackage{enumitem}
\usepackage{tkz-graph}
\usepackage{polynom}
\polyset{%
	style=B,
	delims={(}{)},
	div=:
}
\renewcommand\labelitemi{$\circ$}
\usepackage{nicematrix} %https://ctan.dcc.uchile.cl/macros/latex/contrib/nicematrix/nicematrix.pdf

%\setbeamertemplate{background}[grid][step=8 ]
\setbeamertemplate{itemize item}{$\circ$}
\setbeamertemplate{enumerate items}[default]
\definecolor{links}{HTML}{2A1B81}
\hypersetup{colorlinks,linkcolor=,urlcolor=links}


\newcommand{\Id}{\operatorname{Id}}
\newcommand{\img}{\operatorname{Im}}
\newcommand{\nuc}{\operatorname{Nu}}
\newcommand{\im}{\operatorname{Im}}
\renewcommand\nu{\operatorname{Nu}}
\newcommand{\la}{\langle}
\newcommand{\ra}{\rangle}
\renewcommand{\t}{{\operatorname{t}}}
\renewcommand{\sin}{{\,\operatorname{sen}}}
\newcommand{\Q}{\mathbb Q}
\newcommand{\R}{\mathbb R}
\newcommand{\C}{\mathbb C}
\newcommand{\K}{\mathbb K}
\newcommand{\F}{\mathbb F}
\newcommand{\Z}{\mathbb Z}
\newcommand{\N}{\mathbb N}
\newcommand\sgn{\operatorname{sgn}}
\renewcommand{\t}{{\operatorname{t}}}
\renewcommand{\figurename }{Figura}

%
% Ver http://joshua.smcvt.edu/latex2e/_005cnewenvironment-_0026-_005crenewenvironment.html
%

\renewenvironment{block}[1]% environment name
{% begin code
	\par\vskip .2cm%
	{\color{blue}#1}%
	\vskip .2cm
}%
{%
	\vskip .2cm}% end code


\renewenvironment{alertblock}[1]% environment name
{% begin code
	\par\vskip .2cm%
	{\color{red!80!black}#1}%
	\vskip .2cm
}%
{%
	\vskip .2cm}% end code


\renewenvironment{exampleblock}[1]% environment name
{% begin code
	\par\vskip .2cm%
	{\color{blue}#1}%
	\vskip .2cm
}%
{%
	\vskip .2cm}% end code




\newenvironment{exercise}[1]% environment name
{% begin code
	\par\vspace{\baselineskip}\noindent
	\textbf{Ejercicio (#1)}\begin{itshape}%
		\par\vspace{\baselineskip}\noindent\ignorespaces
	}%
	{% end code
	\end{itshape}\ignorespacesafterend
}


\newenvironment{definicion}[1][]% environment name
{% begin code
	\par\vskip .2cm%
	{\color{blue}Definición #1}%
	\vskip .2cm
}%
{%
	\vskip .2cm}% end code

\newenvironment{observacion}[1][]% environment name
{% begin code
	\par\vskip .2cm%
	{\color{blue}Observación #1}%
	\vskip .2cm
}%
{%
	\vskip .2cm}% end code

\newenvironment{ejemplo}[1][]% environment name
{% begin code
	\par\vskip .2cm%
	{\color{blue}Ejemplo #1}%
	\vskip .2cm
}%
{%
	\vskip .2cm}% end code

\newenvironment{ejercicio}[1][]% environment name
{% begin code
	\par\vskip .2cm%
	{\color{blue}Ejercicio #1}%
	\vskip .2cm
}%
{%
	\vskip .2cm}% end code


\renewenvironment{proof}% environment name
{% begin code
	\par\vskip .2cm%
	{\color{blue}Demostración}%
	\vskip .2cm
}%
{%
	\vskip .2cm}% end code



\newenvironment{demostracion}% environment name
{% begin code
	\par\vskip .2cm%
	{\color{blue}Demostración}%
	\vskip .2cm
}%
{%
	\vskip .2cm}% end code

\newenvironment{idea}% environment name
{% begin code
	\par\vskip .2cm%
	{\color{blue}Idea de la demostración}%
	\vskip .2cm
}%
{%
	\vskip .2cm}% end code

\newenvironment{solucion}% environment name
{% begin code
	\par\vskip .2cm%
	{\color{blue}Solución}%
	\vskip .2cm
}%
{%
	\vskip .2cm}% end code



\newenvironment{lema}[1][]% environment name
{% begin code
	\par\vskip .2cm%
	{\color{blue}Lema #1}\begin{itshape}%
		\par\vskip .2cm
	}%
	{% end code
	\end{itshape}\vskip .2cm\ignorespacesafterend
}

\newenvironment{proposicion}[1][]% environment name
{% begin code
	\par\vskip .2cm%
	{\color{blue}Proposición #1}\begin{itshape}%
		\par\vskip .2cm
	}%
	{% end code
	\end{itshape}\vskip .2cm\ignorespacesafterend
}

\newenvironment{teorema}[1][]% environment name
{% begin code
	\par\vskip .2cm%
	{\color{blue}Teorema #1}\begin{itshape}%
		\par\vskip .2cm
	}%
	{% end code
	\end{itshape}\vskip .2cm\ignorespacesafterend
}


\newenvironment{corolario}[1][]% environment name
{% begin code
	\par\vskip .2cm%
	{\color{blue}Corolario #1}\begin{itshape}%
		\par\vskip .2cm
	}%
	{% end code
	\end{itshape}\vskip .2cm\ignorespacesafterend
}

\newenvironment{propiedad}% environment name
{% begin code
	\par\vskip .2cm%
	{\color{blue}Propiedad}\begin{itshape}%
		\par\vskip .2cm
	}%
	{% end code
	\end{itshape}\vskip .2cm\ignorespacesafterend
}

\newenvironment{conclusion}% environment name
{% begin code
	\par\vskip .2cm%
	{\color{blue}Conclusión}\begin{itshape}%
		\par\vskip .2cm
	}%
	{% end code
	\end{itshape}\vskip .2cm\ignorespacesafterend
}







\newenvironment{definicion*}% environment name
{% begin code
	\par\vskip .2cm%
	{\color{blue}Definición}%
	\vskip .2cm
}%
{%
	\vskip .2cm}% end code

\newenvironment{observacion*}% environment name
{% begin code
	\par\vskip .2cm%
	{\color{blue}Observación}%
	\vskip .2cm
}%
{%
	\vskip .2cm}% end code


\newenvironment{obs*}% environment name
	{% begin code
		\par\vskip .2cm%
		{\color{blue}Observación}%
		\vskip .2cm
	}%
	{%
		\vskip .2cm}% end code

\newenvironment{ejemplo*}% environment name
{% begin code
	\par\vskip .2cm%
	{\color{blue}Ejemplo}%
	\vskip .2cm
}%
{%
	\vskip .2cm}% end code

\newenvironment{ejercicio*}% environment name
{% begin code
	\par\vskip .2cm%
	{\color{blue}Ejercicio}%
	\vskip .2cm
}%
{%
	\vskip .2cm}% end code

\newenvironment{propiedad*}% environment name
{% begin code
	\par\vskip .2cm%
	{\color{blue}Propiedad}\begin{itshape}%
		\par\vskip .2cm
	}%
	{% end code
	\end{itshape}\vskip .2cm\ignorespacesafterend
}

\newenvironment{conclusion*}% environment name
{% begin code
	\par\vskip .2cm%
	{\color{blue}Conclusión}\begin{itshape}%
		\par\vskip .2cm
	}%
	{% end code
	\end{itshape}\vskip .2cm\ignorespacesafterend
}








\title[Clase 11 - Espacios vectoriales]{Álgebra/Álgebra II \\ Clase 11- Espacios vectoriales. Subespacios vectoriales.}

\author[]{}
\institute[]{\normalsize FAMAF / UNC
	\\[\baselineskip] ${}^{}$
	\\[\baselineskip]
}
\date[25/04/2024]{25 de abril de 2024}



\begin{document}

\begin{frame}
\maketitle
\end{frame}

\begin{frame}{Resumen}
	
    En esta clase\pause
    \begin{itemize}
     \item definiremos espacios vectoriales,\pause
     \item daremos ejemplos de espacios vectoriales\pause
     \item definiremos subespacios vectoriales y veremos algunos ejemplos.
    \end{itemize}
   
    \vskip.8cm
    \
    \pause
   El tema de esta clase  está contenido de la sección 4.1 y comienzo de la 4.2 del apunte de clase ``Álgebra II / Álgebra - Notas del teórico''.
    \end{frame}
    
    \begin{frame}

        La materia en general gira alrededor del problema de
        \begin{itemize}\pause
         \item resolver sistemas homogéneos de ecuaciones lineales y\pause
         \item  caracterizar el  conjunto de soluciones como subconjunto de $\R^n$.
        \end{itemize}
        
        
        \vskip .8cm
        \pause
        Anteriormente introdujimos dos operaciones en $\R^n$:
        \begin{itemize}
         \item los vectores de $\R^n$ se pueden sumar y multiplicar por escalares,
        \end{itemize}
     y   vimos que los conjuntos de soluciones son invariantes por estas operaciones. Dicho de otro modo
        \begin{itemize}\pause
         \item Las soluciones de un sistema homogéneo de ecuaciones lineales se pueden sumar y multiplicar por escalares.
        \end{itemize}
        
       \end{frame}
        


\begin{frame}

Estas son álgunas de las preguntas que responderemos en esta parte de la materia
\pause
\begin{block}{Preguntas}
\begin{enumerate}
    \item ?`Podremos generar todas las soluciones de un sistema homogéneo sumando y multiplicando por escalares algunas pocas soluciones?\pause
    \item ?`Cuál es la mínima cantidad de soluciones que generan todas las soluciones?\pause
    \item ?`Cómo podemos representar cada solución usando el conjunto generador?
\end{enumerate}
\end{block}

\end{frame}

\begin{frame}

Por otro lado, hay otras estructuras matemáticas que tienen suma y producto por escalar 
\begin{itemize}\pause
    \item Matrices\pause
    \item Polinomios\pause
    \item Funciones
\end{itemize}

\

Las operaciones satisfacen las mismas propiedades que las operaciones en $\R^n$
\begin{itemize}\pause
    \item asociatividad\pause
    \item conmutatividad\pause
    \item distributividad\pause
    \item neutro y opuesto
\end{itemize}


\end{frame}

\begin{frame}
Entonces estudiaremos todas estas estructuras en abstracto, sin distinguir si son vectores, matrices, polinomios, funciones o lo que fuere.
\vskip 1cm\pause
Lo importante son las operaciones y las propiedades que satisfacen.




\end{frame}  


\begin{frame}
\begin{definicion}

Un \textit{espacio vectorial (sobre $\K$)} o un \textit{$\K$-espacio vectorial} es un conjunto $V$ que tiene dos operaciones que satisfacen ciertos axiomas. Llamaremos a los elementos de $V$ \textit{vectores}. 
\end{definicion}\pause
    \vskip .2cm 
\textbf{Operaciones}
\begin{itemize}
\item Suma de vectores: Dados $v,w\in V$ podemos formar el vector $v+w\in V$ ($+: V \times V \to V $).\pause
\item Producto por escalares: Dado $v\in V$ y $\lambda\in\K$ podemos formar el vector $\lambda\cdot v\in V$ ($\cdot: \K \times V \to V$).
\end{itemize}\pause
\vskip .2cm 

\textbf{Axiomas}\pause
\begin{itemize}
    \item $+$ es conmutativa, asociativa, existe neutro y opuesto\pause
    \item $\cdot$ es asociativa, distributiva y tiene neutro.
\end{itemize}
    
\end{frame}



\begin{frame}

    Explícitamente, sean $u,v,w\in V$ y $\lambda,\mu\in\K$, los axiomas son\pause
    \vskip .3cm
    \begin{tabular}{lll}
        \textbf{S1.}&$v + w = w + v$ &(\textit{$+$ conmutativa}) \\
        &&\\
        \textbf{S2.}& $(v+ w)+ u = v + (w+u)$ &(\textit{$+$ asociativa}). \\
        &&\\
        \textbf{S3.}& $\exists !$ vector $0$, tal que  $0+ v = v$\quad& (\textit{neutro de la $+$}). \\
        &&\\
        \textbf{S4.}& $\exists !$ $-v$ tal que  $v + (-v) =0$ &(\textit{opuesto}) \\
        &&\\
        \textbf{P1.}& $1\cdot v=v$ para todo $v \in V$& (\textit{neutro de $\cdot$}) \\
        &&\\
        \textbf{P2.}& $\lambda\cdot (\mu \cdot v) = (\lambda\mu )\cdot v$& (\textit{$\cdot$ asociativo}). \\
        &&\\
        \textbf{D1.}& $\lambda\cdot (v+w) = \lambda \cdot v +\lambda \cdot w$& (\textit{propiedad distributiva 1}) \\
        &&\\
        \textbf{D2.}& $(\lambda+\mu )\cdot v = \lambda \cdot v + \mu \cdot  v$ & (\textit{propiedad distributiva 2})
    \end{tabular}
    
\end{frame}

\begin{frame}

\begin{block}{Convenciones}\pause
\begin{itemize}
    \item $\lambda v=\lambda\cdot v$\pause
    \item $-v$ se llama el \textit{opuesto} de $v$\pause
    \item Gracias a la asociatividad de $+$ y $\cdot$ podemos obviar los paréntesis\pause
    \item $w-v=w+(-v)$, en palabras ``$w$ menos $v$'' significa ``$w$ más el opuesto de $v$''
    
    También denotamos $-v + w = (-v) +w$.
\end{itemize}
\end{block}


\end{frame}

\begin{frame}

\begin{block}{Ejemplo}\pause
Podemos comprobar  que $\R$  es un $\R$-espacio vectorial con la suma y la multiplicación usuales  viendo que los axiomas de espacios vectoriales son un subconjunto de los axiomas de los números reales. 
\end{block}
\vskip .4cm

\begin{block}{Ejemplo} Respecto a los número complejos:
    \begin{itemize}
        \item $\C$  es un $\C$-espacio vectorial. 
        \item $\C$ es un $\R$-espacio vectorial. 
        \item $\R$ \textit{no} es un $\C$-espacio vectorial con la suma y multiplicación usuales. ($i\cdot 1 = i \not\in \R$). 
    \end{itemize}
\end{block}
    
\end{frame}

    
\begin{frame}
    \begin{ejemplo}
        $\R^n$  es un $\R$-espacio vectorial con 
        \begin{align*}
            (x_1,\ldots,x_n) + (y_1,\ldots,y_n) &= (x_1+y_1,\ldots,x_n+y_n)&&(x_i,y_i \in \R) \\
            \lambda \cdot (x_1,\ldots,x_n) &= (\lambda x_1,\ldots,\lambda x_n)&&(\lambda \in \R).
        \end{align*}

        El  hecho  de que $\R^n$  es un $\R$-espacio vectorial con estas operaciones  fue probado en clases anteriores. 
        
        \vskip 3cm
    \end{ejemplo}
\end{frame}   
\begin{frame}

\begin{block}{Ejemplo}\pause
El conjunto de matrices $\K^{m\times n}$ es un espacio vectorial con las operaciones que definimos previamente.
\end{block}
\pause
\begin{block}{}
Si $A,B\in\K^{m\times n}$ y $\lambda\in\K$ entonces
\begin{itemize}
    \item $A+B$ es la matriz con entradas $[A+B]_{ij}=[A]_{ij}+[B]_{ij}$
    \item $\lambda\cdot A$ es la matriz con entradas $[\lambda A]_{ij}=\lambda[A]_{ij}$
\end{itemize} 
\end{block}
\pause
Ya hemos visto que estas operaciones satisfacen los axiomas de la definición. En particular
\pause
\begin{block}{}
\begin{itemize}
    \item El elemento neutro $0$ es la matriz con todas las coordenadas iguales a cero,\pause
    \item El opuesto de $A$ es la matriz $(-1)\cdot A$
\end{itemize} 
\end{block}
\end{frame}


\begin{frame}

\begin{block}{Ejemplo}
El conjunto de vectores filas $\K^{1\times n}$  (o columnas $\K^{n\times 1}$) es un espacio vectorial con las operaciones que hemos definido anteriormente en esta clase.\pause
\begin{itemize}
    \item La suma coordenada a coordenada
    \item La multiplicación coordenada a coordenada
\end{itemize}



\end{block}
\pause
\begin{block}{}
Es un caso particular de las matrices. 
\end{block}

    
\end{frame}



\begin{frame}

\begin{block}{Ejemplo}
El conjunto de polinomios sobre $\K$
\begin{align*}
\K[x]=\{a_nx^n+\cdots +a_1 x+a_0\mid n\in\N,\, a_n, ..., a_0\in\R\} 
\end{align*}
con la suma y multiplicación que ya conocen:\pause
\end{block}

\begin{block}{}
\begin{itemize}
    \item Suma coeficiente a coeficiente
    \begin{multline*}
    (a_nx^n+\cdots +a_1 x+a_0)+(b_nx^n+\cdots +b_1 x+b_0)=\\
    = (a_n+b_n)x^n+\cdots +(a_1+b_1) x+(a_0+b_0)
    \end{multline*}

    \pause

    \item Multiplicación coeficiente a coeficiente
    \begin{multline*}
    \lambda\cdot (a_nx^n+\cdots +a_1 x+a_0)
    =(\lambda a_n)x^n+\cdots +(\lambda a_1) x+(\lambda a_0)
    \end{multline*}
\end{itemize}
    
\end{block}

    
\end{frame}




\begin{frame}


\begin{block}{}

\begin{itemize}
    \item El neutro es el polinomio  $0$.
    \vskip .4cm\pause
    \item El opuesto del polinomio $a_nx^n+\cdots +a_1 x+a_0$ es el polinomio 
    \begin{align*}
    (-a_n)x^n+\cdots +(-a_1) x+(- a_0) = -a_nx^n -\cdots -a_1 x- a_0.
    \end{align*}
    \end{itemize}
    
    \pause
\end{block}
\begin{block}{Observación}
    \begin{itemize}
        \item Si $x^i$ no aparece en la expresión de un polinomio quiere decir que respectivo coeficiente $a_i$ es cero. Por ejemplo:
        \begin{align*}
        x^2+1=x^2+0x+1 
        \end{align*}\pause
    \item Para sumar polinomios no es necesario que tengan el mismos grado. Por ejemplo:
    \begin{align*}
    (x^2+1)+(x^5+2x^2+5x+2)=x^5+3x^2+5x+3
    \end{align*}
    
    
    \end{itemize}
    
    
    
    \end{block}
    
\end{frame}


\begin{frame}

\begin{block}{Ejemplo}

Sea $X$ un conjunto. El \textit{espacio vectorial de funciones de $X$ a $\R$} es el conjunto
\begin{align*}
\R^X=\left\{\mbox{las funciones }f:X\longrightarrow\R\right\}
\end{align*}
con la suma y producto por escalar ``punto a punto''. 
\vskip .4cm\pause


\end{block}

\begin{block}{}
    Es decir, si $f, g \in \R^X$ y $\lambda \in \R$, 
    \vskip .4cm
\begin{itemize}
    \item $f+g:X\longrightarrow\R$ es la función definida por
    \begin{align*}
    (f+g)(x)=f(x)+g(x)
    \end{align*}\pause
\item $\lambda\cdot f:X\longrightarrow\R$ es la función definida por
    \begin{align*}
    (\lambda\cdot f)(x)=\lambda f(x)
    \end{align*}
\end{itemize}

    
\end{block}

    
\end{frame}

\begin{frame}

Si $f,g:X\longrightarrow\R$ y $\lambda\in\R$ entonces
\begin{itemize}
    \item el opuesto de $f$ es $-f:X\longrightarrow\R$, la función definida por
    \begin{align*}
    (-f)(x)=-f(x)
    \end{align*}\pause
\item El elemento neutro es la función constante igual a cero, es decir $f(x) =0$ para todo $x \in X$. la cual denotamos $0$
\end{itemize}

\vskip .4cm \pause
\begin{block}{Observación}
Si $X=\R$ entonces la suma y el producto por escalar es la misma definición que se usa en Análisis Matemático I.

\vskip .2cm 
En este caso se suele denotar $F(\R)=\R^\R$
\end{block}


\end{frame}


\begin{frame}

\begin{block}{Ejemplo}

El conjunto de los números reales positivos
$\R_{>0}=(0,\infty)$ es un espacio vectorial con las siguientes operaciones:
\begin{itemize}
    \item $x\oplus y=x\cdot y$ ($\oplus$ es la multiplicación)
    \item $\lambda\odot x=x^\lambda$ ($\odot$ es la potenciación)


    \item El ``neutro'' es el $1$: $1\oplus x=1\cdot x=x$
    \item El ``opuesto'' es el inverso: $x^{-1}\oplus x=x^{-1}\cdot x=1$
\end{itemize}
\end{block}
\pause
\begin{observacion}
    \begin{itemize}
        \item Definición  
        $$x^\lambda := e^{\lambda\ln(x)} = \sum_{n=0}^\infty \frac{(\lambda\ln(x))^n}{n!}.$$
        \item Probemos \textbf{D2}: \pause
        $$ (\lambda + \mu )\odot x = x^{\lambda + \mu} = x^{\lambda}x^{\mu} = 
        x^{\lambda}\oplus x^{\mu} =  \lambda\odot x \oplus \mu\odot x.\qed
        $$
    \end{itemize}
    
\end{observacion}
\end{frame}


\begin{frame}
\begin{block}{Proposición}
Sea $V$ un espacio vectorial sobre $\K$. Entonces\pause
\begin{enumerate}
    \item $\lambda\cdot 0=0$ para todo $\lambda\in \K$\pause
    \item $0\cdot v=0$ para todo $v\in V$\pause
    \item Si $\lambda\cdot v=0$ entonces $\lambda=0$ ó $v=0$\pause
    \item $(-1)\cdot v=-v$, en palabras, $-1$ por $v$ es igual al opuesto de $v$
\end{enumerate}
\end{block}
\pause
\begin{block}{Observación}
La demostración es similar  a las propiedades análogas de los números reales  o los números enteros dado que lo único que usamos son los axiomas.
\end{block}

\end{frame}

\begin{frame}
\begin{block}{Demostración 1.}\pause
\begin{itemize}
    \item $\lambda\cdot 0=0$ para todo $\lambda\in \R$ 
\end{itemize}
\begin{align*}
\lambda\cdot 0&= \lambda\cdot (0+0)&\quad (\mbox{axioma elemento neutro})\\
\lambda\cdot 0&=\lambda\cdot 0+\lambda\cdot 0
&\quad (\mbox{axioma distributividad})\\
\Rightarrow 0&=\lambda\cdot0
&\quad (\mbox{sumando el opuesto de $\lambda\cdot 0$})
\end{align*}

\end{block}
\pause
\begin{block}{Demostración 2.}\pause
\begin{itemize}
    \item $0\cdot v=0$ para todo $v\in V$
\end{itemize}
    es similar a la anterior. 
\end{block}
\end{frame}


\begin{frame}

    \begin{block}{Demostración 3.}\pause
\begin{itemize}
    \item Si $\lambda\cdot v=0$ entonces $\lambda=0$ ó $v=0$

\end{itemize}
\vskip .2cm
Si $\lambda=0$ no hay nada que demostrar. 
\vskip .2cm
Supongamos que $\lambda\neq0$. Sea $\lambda^{-1}\in\R$ su inverso multiplicativo.
\begin{align*}
\lambda^{-1}\cdot 0&=\lambda^{-1}\cdot(\lambda\cdot v)&&\mbox{(por hipótesis)}\\
0&=(\lambda^{-1}\lambda)\cdot v&\quad&\mbox{(asociatividad)}\\
0&=1\cdot v&&\\
0&=v&\quad&\mbox{(axioma neutro)}
\end{align*}
\end{block}
\end{frame}

\begin{frame}

    \begin{block}{Demostración 4.}\pause
\begin{itemize}
    \item $(-1)\cdot v=-v$, en palabras, $-1$ por $v$ es igual al opuesto de $v$
\end{itemize}
\begin{align*}
0&=0\cdot v&&\text{(por 2.)}\\
0&=(1+(-1))v&&\text{}\\
0&=1\cdot v+(-1)\cdot v&&\text{(distributividad)}\\
0&=v+(-1)\cdot v&&\text{(elemento neutro $\cdot$)}\\
-v+0&=-v+ v+(-1)\cdot v&&\text{(sumo $-v$ a ambos miembros)}\\
-v&= 0 +(-1)\cdot v&&\text{(elemento neutro $+$ y opuesto)}\\
-v&=(-1)\cdot v&&\text{(elemento neutro $+$)}
\end{align*}
\end{block} \qed
\end{frame}

\begin{frame}{Subespacios vectoriales}

    \pause
    \begin{definicion}
        Sea $V$ un espacio vectorial sobre $\K$. diremos que $W \subset V$ es \textit{subespacio de $V$}\index{subespacio} si $W \not= \emptyset$ y
        \begin{enumerate}
            \item[(a)] si para cualesquiera $w_1,w_2 \in W$, se cumple que $w_1+w_2 \in W$ y
            \item[(b)] si $\lambda \in \K$ y  $w \in W$, entonces $\lambda w \in W$.
        \end{enumerate}
    \end{definicion}
\vskip .4cm\pause
    \begin{observacion} Si $W$ subespacio de  $V$.
        \begin{itemize}
            \item $0 \in W$.
            \item Si $w \in W$,  entonces $-w \in W$.  
        \end{itemize}
        
    \end{observacion}
\end{frame}

\begin{frame}

\begin{block}{Demostración $0 \in W$.}\pause
    Como $W \ne \emptyset$, existe $w \in W$. Por la condición (b), $0\cdot w \in W$. Ahora bien,  hemos visto que $0\cdot w =0$, por lo tanto $0 \in W$.
\end{block}
\pause
\begin{block}{Demostración $-w \in W$.}\pause
    Por la condición (b), $(-1)\cdot w \in W$. Ahora bien,  hemos visto que $(-1)\cdot w =-w$, por lo tanto $-w \in W$.
\end{block}
\vskip 2cm


\end{frame}

\begin{frame}
    \begin{observacion}
        Sea $W \subset V$,   $W \ne \emptyset$. Entonces
        \begin{center}
            $W$ subespacio de $V$\quad $\Leftrightarrow$\quad  $u + \lambda w \in W$,\;  $\forall\,u,w \in W,\lambda \in \K$.
        \end{center}
    \end{observacion}
    \pause
    \begin{demostracion}\pause

    ($\Rightarrow$) \pause
    \begin{itemize}
        \item  Por (b) de la definición, $\lambda w \in W$.
        \item  Como $u \in W$ y  $\lambda w \in W$, por (a)  de la definición  $u + \lambda w \in W$.
    \end{itemize}
    \pause

    ($\Leftarrow$) \pause
    
    \begin{itemize}
        \item[(a)] Sean $w_1 , w_2 \in W$, luego $w_1 + 1 \cdot w_2 = w_1 +w_2 \in W$.
        \item[(b)] Sea $\lambda \in \K$ y $w \in W$, entonces $0 + \lambda w = \lambda w \in W$.
    \end{itemize}
    \qed
    

    

    \end{demostracion}
\end{frame}

\begin{frame}

    \begin{teorema}
        Sea $V$ un espacio vectorial sobre $\K$ y $W$ subespacio de $V$. Entonces $W$ con las operaciones suma y producto por escalares de $V$ es un espacio vectorial.
    \end{teorema}\pause
    \begin{proof} \pause
        Para que $W$ sea espacio vectorial sus operaciones deben satisfacer los axiomas de la definición de espacio vectorial. 

        \vskip .2cm
        
        $0 \in W$ y si   $w \in W$ $\Rightarrow$  $-w \in W$.
        
        \vskip .2cm         
        Teniendo en cuenta estos dos hechos,  y que las operaciones en $V$ satisfacen los axiomas de la definición (y por lo tanto en $W$ también),  queda demostrado que $W$, con las operaciones heredadas de $V$, es espacio vectorial.\qed  
        
    \end{proof}

\end{frame}

\begin{frame}{Ejemplos}
    \begin{enumerate}\pause
        \item Sea $V$ un $\K$-espacio vectorial, entonces $\{0\}$ (que se denota $0$) y $V$ son subespacios vectoriales de $V$. Suelen ser llamados los \textit{subespacios triviales}\index{subespacios triviales} de $V$.
        \vskip .4cm \pause
        \item Sea $V$ un $\K$-espacio vectorial y sea $v \in V$, entonces
        $$
        W = \{\mu v: \mu \in \K \}
        $$
        es un subespacio vectorial.
        \vskip .3cm 
        En  efecto, si $\mu_1v,\mu_2v \in W$, con $\mu_1,\mu_2 \in \K$,  entonces 
        $$\mu_1v + \lambda \mu_2v = (\mu_1+\lambda\mu_2)v \in W, $$
        para todo $\lambda\in \K$.
        \pause
        \vskip .4cm 
        El subespacio $W$ suele ser denotado $\K v$.
    \end{enumerate}
\end{frame}



\begin{frame}{Ejemplos}
    \begin{enumerate}     

        \item[3.] Sea $A \in M_{m \times n}(\K)$. Si $x = (x_1,\ldots,x_n) \in \K^n$,  entonces $Ax$ denota
        $$
        Ax := A\begin{bmatrix} x_1 \\ \vdots \\ x_n\end{bmatrix}.
        $$
        Sea 
        $$
        W = \left\{x \in \K^n: Ax=0 \right\}.
        $$
        Es decir, $W$  es el subconjunto de las soluciones del sistema $Ax=0$. 
        \vskip .2cm
        
        Entonces, $W$  es un subespacio de $\K^n$:\pause sean $x,y \in W$ y $\lambda \in \K$, es decir $Ax=0$, $Ay=0$ y $\lambda \in \K$,  entonces 
        $$
        A(x+\lambda y) = Ax + A(\lambda y) = Ax + \lambda Ay = 0 + \lambda \cdot 0 =0.$$ 
    \end{enumerate}
\end{frame}

\begin{frame}
    Es decir 
    \vskip .4cm
    \textit{El conjunto de soluciones de un sistema de ecuaciones homogéneo es un subespacio vectorial de $\K^m$,}
    \vskip .4cm\pause
    En particular, 
    \vskip .2cm
    \begin{itemize}
        \item Las rectas en el plano que pasan por el origen son subespacios de $\R^2$.
        \vskip .2cm
        \item Los planos en el espacio que pasan por el origen son subespacios de $\R^3$.
    \end{itemize}
    


\end{frame}




\end{document}

