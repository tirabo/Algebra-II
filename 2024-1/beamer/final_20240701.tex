% PDFLaTeX \documentclass[a4paper,12pt,twoside,spanish]{amsbook} %\documentclass[a4paper,11pt,twoside]{book}
\documentclass[a4paper,11pt,twoside,spanish]{amsbook}

%%%---------------------------------------------------
   

\usepackage{etex}
\tolerance=10000
\renewcommand{\baselinestretch}{1.1}

\renewcommand{\familydefault}{\sfdefault} % la font por default es sans serif

% Para hacer el  indice en linea de comando hacer 
% makeindex main
%% En http://www.tug.org/pracjourn/2006-1/hartke/hartke.pdf hay ejemplos de packages de fonts libres, como los siguientes:
%\usepackage{cmbright}
%\usepackage{pxfonts}
%\usepackage[varg]{txfonts}
%\usepackage{ccfonts}
%\usepackage[math]{iwona}
%\usepackage[math]{kurier}


\usepackage{t1enc}
%\usepackage[spanish]{babel}
\usepackage{latexsym}
\usepackage[utf8]{inputenc}
\usepackage{verbatim}
\usepackage{multicol}
\usepackage{amsgen,amsmath,amstext,amsbsy,amsopn,amsfonts,amssymb}
\usepackage{amsthm}
\usepackage{calc}         % From LaTeX distribution
\usepackage{graphicx}     % From LaTeX distribution
\usepackage{ifthen}
\input{random.tex}        % From CTAN/macros/generic
\usepackage{subfigure} 
\usepackage{tikz}
\usetikzlibrary{arrows}
\usetikzlibrary{matrix}
%\usetikzlibrary{graphs}
%\usepackage{tikz-3dplot} %for tikz-3dplot functionality
%\usepackage{pgfplots}
\usepackage{mathtools}
\usepackage{stackrel}
\usepackage{enumerate}
\usepackage{tkz-graph}
%\usepackage{makeidx}
\usepackage{enumitem}
\makeindex

%%%----------------------------------------------------------------------------
\usepackage[a4paper, top=3cm, left=3cm, right=2cm, bottom=2.5cm]{geometry}
%% CONTROLADORES DE.
% Tamaño de la hoja de impresión.
% Tamaños de los laterales del documento. 
%%%%%%%%%%%%%%%%%%%%%%%%%%%%%%%%%%%%%%%%%%%%%%%%%%%%%%%%%%%%%%%%%%%%%%%%%%%%%%%%%
%%% \theoremstyle{plain} %% This is the default
%\oddsidemargin 0.0in \evensidemargin -1.0cm \topmargin 0in
%\headheight .3in \headsep .2in \footskip .2in
%\setlength{\textwidth}{16cm} %ancho para apunte
%\setlength{\textheight}{21cm} %largo para apunte
%%%%\leftmargin 2.5cm
%%%%\rightmargin 2.5cm
%\topmargin 0.5 cm
%%%%%%%%%%%%%%%%%%%%%%%%%%%%%%%%%%%%%%%%%%%%%%%%%%%%%%%%%%%%%%%%%%%%%%%%%%%%%%%%%%%

\usepackage{hyperref}
\hypersetup{
	colorlinks=true,
	linkcolor=blue,
	filecolor=magenta,      
	urlcolor=cyan,
}
\usepackage{hypcap}


\renewcommand{\thesection}{\thechapter.\arabic{section}}
\renewcommand{\thesubsection}{\thesection.\arabic{subsection}}

\newtheorem{teorema}{Teorema}[section]
\newtheorem{proposicion}[teorema]{Proposición}
\newtheorem{corolario}[teorema]{Corolario}
\newtheorem{lema}[teorema]{Lema}
\newtheorem{propiedad}[teorema]{Propiedad}

\theoremstyle{definition}

\newtheorem{definicion}{Definición}[section]
\newtheorem{ejemplo}{Ejemplo}[section]
\newtheorem{problema}{Problema}[section]
\newtheorem{ejercicio}{Ejercicio}[section]
\newtheorem{ejerciciof}{}[section]

\theoremstyle{remark}
\newtheorem{observacion}{Observación}[section]
\newtheorem{nota}{Nota}[section]

\renewcommand{\partname }{Parte }
\renewcommand{\indexname}{Indice }
\renewcommand{\figurename }{Figura }
\renewcommand{\tablename }{Tabla }
\renewcommand{\proofname}{Demostración}
\renewcommand{\appendixname }{}
\renewcommand{\contentsname }{Contenidos }
\renewcommand{\chaptername }{}
\renewcommand{\bibname }{Bibliograf\'\i a }

\newcommand{\img}{\operatorname{Im}}
\newcommand{\nuc}{\operatorname{Nu}}
\newcommand\im{\operatorname{Im}}
\renewcommand\nu{\operatorname{Nu}}
\newcommand{\la}{\langle}
\newcommand{\ra}{\rangle}
\renewcommand{\t}{{\operatorname{t}}}
\renewcommand{\sin}{{\,\operatorname{sen}}}
\newcommand{\Q}{\mathbb Q}
\newcommand{\R}{\mathbb R}
\newcommand{\C}{\mathbb C}
\newcommand{\K}{\mathbb K}
\newcommand{\F}{\mathbb F}
\newcommand{\Z}{\mathbb Z}

\newcommand{\parcial}[2]{
	\begin{center}
		{\Large {Álgebra/Álgebra II/Álgebra Lineal} - 2024/1} \vskip.4cm
		{\Large Parcial #1 - Tema #2}\vskip .4cm
\end{center}}

\newcommand{\recuperatorio}[1]{
	\begin{center}
		{\Large {Álgebra/Álgebra II/Álgebra Lineal} - 2024/1} \vskip.4cm
		{\Large Parcial #1 -- Recuperatorio }\vskip .4cm
\end{center}}

\newcommand{\final}[1]{
	\begin{center}
		{\Large {Álgebra/Álgebra II/Álgebra Lineal}} \vskip.4cm
		{\Large Examen Final -- #1}\vskip .4cm
\end{center}}
\renewenvironment{ejercicio}% environment name
{% begin code
	\par\vskip .5cm%
	{\noindent\color{blue}Ejercicio}%
	\vskip .2cm
}%
{%
	\vskip .2cm}% end code


\newenvironment{ejercicios}% environment name
{% begin code
	\par\vskip .5cm%
	{\noindent\color{blue}Ejercicios}%
	\vskip .2cm
}%
{%
	\vskip .2cm}% end code
	
	
\newenvironment{ejercicios-teoricos}% environment name
{% begin code
	\par\vskip .5cm%
	{\noindent\color{blue}Ejercicios teóricos}%
	\vskip .2cm
}%
{%
	\vskip .2cm}% end code
	
\newenvironment{ejercicios-practicos}% environment name
{% begin code
	\par\vskip .5cm%
	{\noindent\color{blue}Ejercicios prácticos}%
	\vskip .2cm
}%
{%
	\vskip .2cm}% end code


\newenvironment{solucion}% environment name
{% begin code
	\par\vskip .2cm%
	{\noindent\color{blue}Solución}%
	\vskip .2cm
}%
{%
	\vskip .2cm}% end code



\pagenumbering{gobble}

\begin{document}
	
	%\frame{\titlepage} 
	
	

	\final{1/7/2024}
	
	
	
	
	
	\noindent{Nombres y  apellidos: }
	\vskip 0.1cm
	\noindent{Correo UNC: }\hskip 8cm Carrera: 
	
	\vskip .4cm
	
	\noindent\textbf{Importante}
	\begin{itemize}
		\item Todos los resultados deben estar debidamente justificados, mostrando paso a paso la obtención de los mismos. 
		\item No podés usar calculadora, celular o cualquier otro dispositivo electrónico mientras estés haciendo el
		examen.
		\item Los alumnos en Condición Regular no deben resolver el ítem (b) del Ejercicio 3: el puntaje  del mismo se les sumará automáticamente por revestir esta condición.
	\end{itemize}
	
	{\noindent\color{blue}Ejercicios teóricos}%
	\vskip .2cm
	
	\begin{enumerate}
				\item {(10\%)} Demostrar que si $A,B$ son  matrices $n \times n$ invertibles, entonces $AB$ es una matriz invertible y decir cual es la inversa de $AB$.
				\item {(10\%)} Sea $T$ transformación lineal, entonces $T$ es monomorfismo si y sólo si $\operatorname{Nu}(T) = 0$.
	\end{enumerate}
		

	{\noindent\color{blue}Ejercicios}%
	\vskip .2cm

		\begin{enumerate}[resume]
			
		
		\item 
			\begin{enumerate}
			\item {(5\%)} Describir de manera paramétrica el conjunto solución del sistema homogéneo:
			\begin{align*}
				\begin{cases}
					2y  + z = 0\\
					-x+ y+2z = 0\\
					x + 3y  = 0
				\end{cases}
			\end{align*}
			
			\item {(5\%)} ({\bf solo alumnos libres}) Indicar cuál es la MERF asociada al sistema anterior. Debe justificar la respuesta.
			
			\end{enumerate}
			
			\vskip .2cm
		
		\item{(10\%)} Sea $A=\begin{bmatrix}
			1 & -1 & 2\\3 & 2 & 1\\0 & 1 & -2
		\end{bmatrix}$ la matriz
			\begin{enumerate} 
			\item Calcular el determinante de $A$.
			\item Calcular la inversa de $A$. Usar el método explicado en clase, con operaciones de filas. 
			\end{enumerate}
		\vskip .2cm 
		
		\item{(20\%)} Sean $W_1$ y $W_2$ los siguientes subespacios de $\mathbb{R}^3$:
		\begin{align*}
			W_1 &= \{ (x,y,z)\in\mathbb{R}^3\ : \ x+y-2z=0\},  \\
			W_2 &= {\left\langle(1,-1,1),(2,1,-2),(3,0,-1)\right\rangle}.
		\end{align*}
		\begin{enumerate}
			\item Dar una base de $W_1$ (justificar).
			\item  Determinar $W_1 \cap W_2$, y describirlo por generadores y con ecuaciones.
			%\item  Determinar $W_1+W_2$, y describirlo por generadores y con ecuaciones.
			%\item  ?`Es la suma $W_1+W_2$ directa?
			%        \item  Dar un complemento de $W_1$.
			%        \item  Dar un complemento de $W_2$.
		\end{enumerate}
	
		\vskip .2cm
		
		\item{(20\%)}  Sea $T:P_{2} \longrightarrow P_{3}$, la transformación lineal definida $$T(ax+b)=(2a+3b)x^{2} +(a+b)x+ a-b.$$ 
			\begin{enumerate}
			\item Dar una base del núcleo y la imagen de $T$.
			\item Dar la matriz de $T$ en las bases $\mathcal{B}_2=\{1,x+1\}$ y $\mathcal{B}_3=\{1,1+x,1+x+x^{2}\}$ de $P_{2}$ y $P_{3}$, respectivamente.
			\end{enumerate}
		\vskip .4cm 
		
		\begin{center}
			{\color{blue} \Large \textbf{Continúa atrás.}}
		\end{center}
		\newpage
		\item {(10\%)} 
		Sea $A=\begin{bmatrix}7 & 1\\2 & 8\end{bmatrix}$.
			\begin{enumerate}
			\item Calcular los autovalores de $A$.
			
			\item Describir paramétricamente los autoespacios asociados a los autovalores de $A$, y decidir si $A$ es diagonalizable.
			\end{enumerate}
			
		\vskip .2cm 
		\item {(10\%)} Sean $T: \R^{3} \to W$ y $S: W \to \R^{3}$ dos transformaciones lineales entre espacios vectoriales tal que $S \circ T(x,y,z) = (x+y,y+z,x+z)$.
		\begin{enumerate}
		\item {(2\%)} Probar que $Nu(S \circ T) =0$.
		\item {(3\%)} Probar que  $S\circ T$ es biyectiva.
		\item {(5\%)} Probar que $T$  es inyectiva.
		\end{enumerate}

		\end{enumerate}

	
	\vskip1cm
	
	\scriptsize
	
	\begin{tabular}{|c|c|c|c|cc|}
		\hline
		Ejercicio & \hspace{0.5cm} 1\hspace{0.5cm}  &\hspace{0.5cm}  2\hspace{0.5cm} & \hspace{0.5cm} 3\hspace{0.5cm} & \hspace{0.5cm} 4\hspace{0.5cm}&\\ \hline
		&  & & & &\\
		\% & &  & & & \\
		&  & & & &\\ \hline
	\end{tabular}
	\hspace*{0.5cm}
	\begin{tabular}{|c|}
		\hline
		\hspace{0.0cm} {\bf Total \%}\hspace{0.0cm}   \\ \hline
		\\
		\\
		\\ \hline
	\end{tabular}
	\hspace*{0.5cm}
	\begin{tabular}{|c|}
		\hline
		\hspace{0.3cm} {\bf Nota}\hspace{0.3cm}   \\ \hline
		\\
		\\
		\\ \hline
	\end{tabular}
	
		\begin{tabular}{|c|c|c|c|cc|}
		\hline
		Ejercicio & \hspace{0.5cm} 5\hspace{0.5cm}  &\hspace{0.5cm}  6\hspace{0.5cm} & \hspace{0.5cm} 7\hspace{0.5cm} & \hspace{0.5cm} 8\hspace{0.5cm}&\\ \hline
		&  & & & &\\
		\% & &  & & & \\
		&  & & & &\\ \hline
	\end{tabular}
	
\end{document}

