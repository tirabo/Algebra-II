\documentclass{beamer} % descomentar para tener pausas
%\documentclass[handout]{beamer} % descomentar para no tener pausas
\usetheme{CambridgeUS}
\setbeamertemplate{background}[grid][step=8 ] % cuadriculado

\usepackage[utf8]{inputenc}%esto permite (en Windows) escribir directamente 
\usepackage{graphicx}
\usepackage{array}
\usepackage{tikz} 
\usetikzlibrary{shapes,arrows,babel,decorations.pathreplacing}
\usepackage{verbatim} 
\usepackage{xcolor} 
\usepackage{amsgen,amsmath,amstext,amsbsy,amsopn,amsfonts,amssymb}
\usepackage{amsthm}
\usepackage{tikz}
\usepackage{tkz-graph}
\usepackage{mathtools}
\usepackage{xcolor}
\usepackage{soul}
%https://en.wikibooks.org/wiki/LaTeX/Colors
\newcommand{\mathcolorbox}[2]{\colorbox{#1}{$\displaystyle #2$}}
\newcommand{\hlfancy}[2]{\sethlcolor{#1}\hl{#2}}


%\setbeamertemplate{background}[grid][step=8 ]
\setbeamertemplate{itemize item}{$\circ$}
\setbeamertemplate{enumerate items}[default]
\definecolor{links}{HTML}{2A1B81}
\hypersetup{colorlinks,linkcolor=,urlcolor=links}
\setbeamercolor{block}{fg=red, bg=red!40!white}
\setbeamercolor{block example}{use=structure,fg=black,bg=white!20!white}


\newcommand{\img}{\operatorname{Im}}
\newcommand{\nuc}{\operatorname{Nu}}
\renewcommand\nu{\operatorname{Nu}}
\newcommand{\la}{\langle}
\newcommand{\ra}{\rangle}
\renewcommand{\t}{{\operatorname{t}}}
\renewcommand{\sin}{{\,\operatorname{sen}}}
\newcommand{\Q}{\mathbb Q}
\newcommand{\R}{\mathbb R}
\newcommand{\C}{\mathbb C}
\newcommand{\K}{\mathbb K}
\newcommand{\F}{\mathbb F}
\newcommand{\Z}{\mathbb Z}
\renewcommand{\figurename }{Figura}




\renewenvironment{block}[1]% environment name
{% begin code
    \par\vskip .2cm%
    {\color{blue}#1}%
    \vskip .2cm
}%
{%
    \vskip .2cm}% end code


\renewenvironment{alertblock}[1]% environment name
{% begin code
    \par\vskip .2cm%
    {\color{red!80!black}#1}%
    \vskip .2cm
}%
{%
    \vskip .2cm}% end code


\renewenvironment{exampleblock}[1]% environment name
{% begin code
    \par\vskip .2cm%
    {\color{blue}#1}%
    \vskip .2cm
}%
{%
    \vskip .2cm}% end code




\newenvironment{exercise}[1]% environment name
{% begin code
    \par\vspace{\baselineskip}\noindent
    \textbf{Ejercicio (#1)}\begin{itshape}%
        \par\vspace{\baselineskip}\noindent\ignorespaces
    }%
    {% end code
    \end{itshape}\ignorespacesafterend
}


\newenvironment{definicion}% environment name
{% begin code
    \par\vskip .2cm%
    {\color{blue}Definición}%
    \vskip .2cm
}%
{%
    \vskip .2cm}% end code

\newenvironment{observacion}% environment name
{% begin code
    \par\vskip .2cm%
    {\color{blue}Observación}%
    \vskip .2cm
}%
{%
    \vskip .2cm}% end code

\newenvironment{ejemplo}% environment name
{% begin code
    \par\vskip .2cm%
    {\color{blue}Ejemplo}%
    \vskip .2cm
}%
{%
    \vskip .2cm}% end code

\newenvironment{ejercicio}% environment name
{% begin code
    \par\vskip .2cm%
    {\color{blue}Ejercicio}%
    \vskip .2cm
}%
{%
    \vskip .2cm}% end code


\renewenvironment{proof}% environment name
{% begin code
    \par\vskip .2cm%
    {\color{blue}Demostración}%
    \vskip .2cm
}%
{%
    \vskip .2cm}% end code



\newenvironment{demostracion}% environment name
{% begin code
    \par\vskip .2cm%
    {\color{blue}Demostración}%
    \vskip .2cm
}%
{%
    \vskip .2cm}% end code

\newenvironment{idea}% environment name
{% begin code
    \par\vskip .2cm%
    {\color{blue}Idea de la demostración}%
    \vskip .2cm
}%
{%
    \vskip .2cm}% end code

\newenvironment{solucion}% environment name
{% begin code
    \par\vskip .2cm%
    {\color{blue}Solución}%
    \vskip .2cm
}%
{%
    \vskip .2cm}% end code



\newenvironment{lema}% environment name
{% begin code
    \par\vskip .2cm%
    {\color{blue}Lema}\begin{itshape}%
        \par\vskip .2cm
    }%
    {% end code
    \end{itshape}\vskip .2cm\ignorespacesafterend
}

\newenvironment{proposicion}% environment name
{% begin code
    \par\vskip .2cm%
    {\color{blue}Proposición}\begin{itshape}%
        \par\vskip .2cm
    }%
    {% end code
    \end{itshape}\vskip .2cm\ignorespacesafterend
}

\newenvironment{teorema}% environment name
{% begin code
    \par\vskip .2cm%
    {\color{blue}Teorema}\begin{itshape}%
        \par\vskip .2cm
    }%
    {% end code
    \end{itshape}\vskip .2cm\ignorespacesafterend
}


\newenvironment{corolario}% environment name
{% begin code
    \par\vskip .2cm%
    {\color{blue}Corolario}\begin{itshape}%
        \par\vskip .2cm
    }%
    {% end code
    \end{itshape}\vskip .2cm\ignorespacesafterend
}

\newenvironment{propiedad}% environment name
{% begin code
    \par\vskip .2cm%
    {\color{blue}Propiedad}\begin{itshape}%
        \par\vskip .2cm
    }%
    {% end code
    \end{itshape}\vskip .2cm\ignorespacesafterend
}

\newenvironment{conclusion}% environment name
{% begin code
    \par\vskip .2cm%
    {\color{blue}Conclusión}\begin{itshape}%
        \par\vskip .2cm
    }%
    {% end code
    \end{itshape}\vskip .2cm\ignorespacesafterend
}





%%%%%%%%%%%%%%%%%%%%%%%%%%%%%%%%%%%%%%%%%%%%%%%%%%%%%%%

\newcommand{\nc}{\newcommand}


%%%%%%%%%%%%%%%%%%%%%%%%%LETRAS
\nc{\RR}{{\mathbb R}} \nc{\CC}{{\mathbb C}} \nc{\ZZ}{{\mathbb Z}}
\nc{\FF}{{\mathbb F}} \nc{\NN}{{\mathbb N}} \nc{\QQ}{{\mathbb Q}}
\nc{\PP}{{\mathbb P}} \nc{\DD}{{\mathbb D}} \nc{\Sn}{{\mathbb S}}
\nc{\uno}{\mathbb{1}} \nc{\BB}{{\mathbb B}} \nc{\An}{{\mathbb A}}

\nc{\ba}{\mathbf{a}} \nc{\bb}{\mathbf{b}} \nc{\bt}{\mathbf{t}}
\nc{\bB}{\mathbf{B}}

\nc{\cP}{\mathcal{P}} \nc{\cU}{\mathcal{U}} \nc{\cX}{\mathcal{X}}
\nc{\cE}{\mathcal{E}} \nc{\cS}{\mathcal{S}} \nc{\cA}{\mathcal{A}}
\nc{\cC}{\mathcal{C}} \nc{\cO}{\mathcal{O}} \nc{\cQ}{\mathcal{Q}}
\nc{\cB}{\mathcal{B}} \nc{\cJ}{\mathcal{J}} \nc{\cI}{\mathcal{I}}
\nc{\cM}{\mathcal{M}} \nc{\cK}{\mathcal{K}}

\nc{\fD}{\mathfrak{D}} \nc{\fI}{\mathfrak{I}} \nc{\fJ}{\mathfrak{J}}
\nc{\fS}{\mathfrak{S}} \nc{\gA}{\mathfrak{A}}
%%%%%%%%%%%%%%%%%%%%%%%%%LETRAS




%%%%%%%%%%%%%%%%%yetter drinfield
\newcommand{\ydg}{{}_{\ku G}^{\ku G}\mathcal{YD}}
\newcommand{\ydgdual}{{}_{\ku^G}^{\ku^G}\mathcal{YD}}
\newcommand{\ydf}{{}_{\ku F}^{\ku F}\mathcal{YD}}
\newcommand{\ydgx}{{}_{\ku \Gx}^{\ku \Gx}\mathcal{YD}}

\newcommand{\ydgxy}{{}_{\ku \Gy}^{\ku \Gx}\mathcal{YD}}

\newcommand{\ydixq}{{}_{\ku \ixq}^{\ku \ixq}\mathcal{YD}}

\newcommand{\ydl}{{}^H_H\mathcal{YD}}
\newcommand{\ydll}{{}_{K}^{K}\mathcal{YD}}
\newcommand{\ydh}{{}^H_H\mathcal{YD}}
\newcommand{\ydhdual}{{}^{H^*}_{H^*}\mathcal{YD}}


\newcommand{\ydha}{{}^H_A\mathcal{YD}}
\newcommand{\ydhhaa}{{}^{\Hx}_{\Ay}\mathcal{YD}}


\newcommand{\wydh}{\widehat{{}^H_H\mathcal{YD}}}
\newcommand{\ydvh}{{}^{\ac(V)}_{\ac(V)}\mathcal{YD}}
\newcommand{\ydrh}{{}^{R\# H}_{R\# H}\mathcal{YD}}
\newcommand{\ydho}{{}^{H^{\dop}}_{H^{\dop}}\mathcal{YD}}
\newcommand{\ydhsw}{{}^{H^{\sw}}_{H^{\sw}}\mathcal{YD}}

\newcommand{\ydhlf}{{}^H_H\mathcal{YD}_{\text{loc fin}}}
\newcommand{\ydholf}{{}^{H^{\dop}}_{H^{\dop}}\mathcal{YD}_{\text{loc fin}}}

\nc{\yd}{\mathcal{YD}}

\newcommand{\ydsn}{{}^{\ku{\Sn_n}}_{\ku{\Sn_n}}\mathcal{YD}}
\newcommand{\ydsnd}{{}^{\ku^{\Sn_n}}_{\ku^{\Sn_n}}\mathcal{YD}}
\nc{\ydSn}[1]{{}^{\Sn_{#1}}_{\Sn_{#1}}\yd}
\nc{\ydSndual}[1]{{}^{\ku^{\Sn_{#1}}}_{\ku^{\Sn_{#1}}}\yd}
\newcommand{\ydstres}{{}^{\ku^{\Sn_3}}_{\ku^{\Sn_3}}\mathcal{YD}}
%%%%%%%%%%%%%%%%%yetter drinfield


%%%%%%%%%%%%%%%%%%%%%%%%%%%%%Operatorename
\newcommand\Irr{\operatorname{Irr}}
\newcommand\id{\operatorname{id}}
\newcommand\ad{\operatorname{ad}}
\newcommand\Ad{\operatorname{Ad}}
\newcommand\Ext{\operatorname{Ext}}
\newcommand\tr{\operatorname{tr}}
\newcommand\gr{\operatorname{gr}}
\newcommand\grdual{\operatorname{gr-dual}}
\newcommand\Gr{\operatorname{Gr}}
\newcommand\co{\operatorname{co}}
\newcommand\car{\operatorname{car}}
\newcommand\rk{\operatorname{rg}}
\newcommand\ord{\operatorname{ord}}
\newcommand\cop{\operatorname{cop}}
\newcommand\End{\operatorname{End}}
\newcommand\Hom{\operatorname{Hom}}
\newcommand\Alg{\operatorname{Alg}}
\newcommand\Aut{\operatorname{Aut}}
\newcommand\Int{\operatorname{Int}}
\newcommand\Id{\operatorname{Id}}
\newcommand\qAut{\operatorname{qAut}}
\newcommand\Map{\operatorname{Map}}
\newcommand\Jac{\operatorname{Jac}}
\newcommand\Rad{\operatorname{Rad}}
\newcommand\Rep{\operatorname{Rep}}
\newcommand\Ker{\operatorname{Ker}}
\newcommand\Img{\operatorname{Im}}
\newcommand\Ind{\operatorname{Ind}}
\newcommand\Comod{\operatorname{Comod}}
\newcommand\Reg{\operatorname{Reg}}
\newcommand\Pic{\operatorname{Pic}}
\newcommand\textitb{\operatorname{Emb}}
\newcommand\op{\operatorname{op}}
\newcommand\Perm{\operatorname{Perm}}
\newcommand\Res{\operatorname{Res}}
\newcommand\res{\operatorname{res}}
\newcommand{\sop}{\operatorname{Supp}}
\newcommand\Cent{\operatorname{Cent}}
\newcommand\sgn{\operatorname{sgn}}
\nc{\GL}{\operatorname{GL}}
%%%%%%%%%%%%%%%%%%%%%%%%%%%%%Operatorename

%%%%%%%%%%%%%%%%%%%%%%%%%%%%%Usuales de hopf
\nc{\D}{\Delta} 
\nc{\e}{\varepsilon}
\nc{\adl}{\ad_\ell}
\nc{\ot}{\otimes}
\nc{\Ho}{H_0} 
\nc{\GH}{G(H)} 
\nc{\coM}{\mathcal{M}^\ast(2,k)} 
\nc{\PH}{\cP(H)}
\nc{\Ftwist}{\overset{\curvearrowright}F}
\nc{\rep}{{\mathcal Rep}(H)}
\newcommand{\deltad}{_*\delta}
\newcommand{\B}{\mathfrak{B}}
\newcommand{\wB}{\widehat{\mathfrak{B}}}
\newcommand{\Cg}[1]{C_{G}(#1)}
\nc{\hmh}{{}_H\hspace{-1pt}{\mathcal M}_H}
\nc{\hm}{{}_H\hspace{-1pt}{\mathcal M}}
\renewcommand{\_}[1]{_{\left[ #1 \right]}}
\renewcommand{\^}[1]{^{\left[ #1 \right]}}
%%%%%%%%%%%%%%%%%%%%%%%%%%%%%Usuales de hopf

%%%%%%%%%%%%%%%%%%%%%%%%%%%%%%%%Usuales
\nc{\im}{\mathtt{i}}
\renewcommand{\Re}{{\rm Re}}
\renewcommand{\Im}{{\rm Im}}
\nc{\Tr}{\mathrm{Tr}} 
\nc{\cark}{char\,k} 
\nc{\ku}{\Bbbk} 
\newcommand{\fd}{finite dimensional}
%%%%%%%%%%%%%%%%%%%%%%%%%%%%%%%%Usuales

%%%%%%%%%%%%%%%%%%%%%%%%Especiales para liftings de duales de Sn
\newcommand{\xij}[1]{x_{(#1)}}
\newcommand{\yij}[1]{y_{(#1)}}
\newcommand{\Xij}[1]{X_{(#1)}}
\newcommand{\dij}[1]{\delta_{#1}}
\newcommand{\aij}[1]{a_{(#1)}}
\newcommand{\hij}[1]{h_{(#1)}}
\newcommand{\gij}[1]{g_{(#1)}}
\newcommand{\eij}[1]{e_{(#1)}}
\newcommand{\fij}[1]{f_{#1}}
\newcommand{\tij}[1]{t_{(#1)}}
\newcommand{\Tij}[1]{T_{(#1)}}
\newcommand{\mij}[1]{m_{(#1)}}
\newcommand{\Lij}[1]{L_{(#1)}}
\newcommand{\mdos}[2]{m_{(#1)(#2)}}
\newcommand{\mtres}[3]{m_{(#1)(#2)(#3)}}
\newcommand{\mcuatro}{m_{\textsf{top}}}
\newcommand{\trid}{\triangleright}
\newcommand{\link}{\sim_{\ba}}








\title[Clase 08 - Álgebra de matrices 3]{Álgebra/Álgebra II \\ Clase 08 - Álgebra de matrices 3}

\author[]{}
\institute[]{\normalsize FAMAF / UNC
    \\[\baselineskip] ${}^{}$
    \\[\baselineskip]
}
\date[11/04/2024]{11 de abril de 2024}

%\titlegraphic{\includegraphics[width=0.2\textwidth]{logo_gimp100.pdf}}

% Converted to PDF using ImageMagick:
% # convert logo_gimp100.png logo_gimp100.pdf


\begin{document}

\begin{frame}
\maketitle
\end{frame}

\begin{frame}{Resumen}
    En esta clase veremos:
    \pause
    \vskip .4cm

    \begin{itemize}
        \item toda matriz elemental es invertible. \pause
        \item Toda matriz invertible es producto de matrices elementales. \pause
        \item Estudiaremos la forma de calcular la inversa de una matriz (cuando  existe) con operaciones elementales. \pause
        \item Finalmente, probarémos que los sistemas de ecuaciones cuya matriz es invertible tienen una única solución.
    \end{itemize}

    \vskip .4cm \pause
    
    El tema de esta clase  está contenido de la sección la sección 2.7  del apunte de clase ``Álgebra II / Álgebra - Notas del teórico''.
\end{frame}

\begin{frame}
        
    \begin{teorema}
        Una matriz elemental es invertible.
    \end{teorema} \pause
    \begin{proof} \pause
        Sea $E$ la matriz elemental que se obtiene a partir de $\Id_n$ por la operación elemental $e$. Sea $e'$ la operación elemental inversa  y $E' = e'(\Id_n)$. Entonces 
        \begin{align*}
            EE' &= e(e'(\Id_n)) = \Id_n, \\
            E'E &= e'(e(\Id_n)) = \Id_n.
        \end{align*}
        Luego  $E' = E^{-1}$. \qed
    \end{proof}    
    
    \vskip 2cm
\end{frame}



\begin{frame}
        Es fácil encontrar explícitamente la matriz inversa de una matríz elemental, por ejemplo, en el caso $2 \times 2$ tenemos:
        \pause
        \vskip .4cm
    \begin{enumerate}
        \item Si $c \not=0$,
        \begin{equation*}
            \begin{bmatrix} c& 0\\ 0&1\end{bmatrix}^{-1}=\begin{bmatrix} 1/c& 0\\ 0&1\end{bmatrix}
            \;\text{ y }\; \begin{bmatrix} 1& 0\\ 0&c\end{bmatrix}^{-1}=\begin{bmatrix} 1& 0\\ 0&1/c\end{bmatrix},
        \end{equation*} \pause
        \item si  $c \in \K$, ,
        \begin{equation*}
            \begin{bmatrix} 1& 0\\ c&1\end{bmatrix}^{-1}=\begin{bmatrix} 1& 0\\ -c&1\end{bmatrix}
            \;\text{ y }\; \begin{bmatrix} 1& c\\ 0&1\end{bmatrix}^{-1}=\begin{bmatrix} 1& -c\\ 0&1\end{bmatrix}.
        \end{equation*} \pause
        \item Finalmente, 
        \begin{equation*}
            \begin{bmatrix} 0& 1\\ 1&0\end{bmatrix} ^{-1}=     \begin{bmatrix} 0& 1\\ 1&0\end{bmatrix}.
        \end{equation*}
    \end{enumerate}
\end{frame}


\begin{frame}
    \begin{lema} Sea $R \in \K^{n \times n}$. Entonces
\begin{equation*}
    \text{$R$ es MERF } \wedge \text{ $R$ es invertible} \quad \Rightarrow \quad R = \Id_n.
\end{equation*}
    \end{lema}     \pause
    \begin{demostracion} \pause
        Supongamos que la $r_{11} =0$. Como $R$ es MERF $\Rightarrow$ $C_1=0$. 
        \vskip .2cm
        Sea $t \not=0$ en $\K$. Como $C_1=0$, 
        \begin{equation*}
            R\begin{bmatrix}
                t \\ 0 \\ \vdots \\0
            \end{bmatrix} =
              \begin{bmatrix}
                0 & * & \cdots &*\\ 0 & * & \cdots &*\\ \vdots \\0& * & \cdots &*
            \end{bmatrix}
            \begin{bmatrix}
                t \\ 0 \\ \vdots \\0
            \end{bmatrix} 
            = 
            \begin{bmatrix}
                0 \\ 0 \\ \vdots \\0
            \end{bmatrix} 
        \end{equation*}

    
    \end{demostracion}
\end{frame}

\begin{frame}
    Luego, como $R$ tiene inversa:
    \begin{equation*}
        \begin{bmatrix}
            t \\ 0 \\ \vdots \\0
        \end{bmatrix}  = 
        \Id 
        \begin{bmatrix}
            t \\ 0 \\ \vdots \\0
        \end{bmatrix}  = 
        R^{-1}R
        \begin{bmatrix}
            t \\ 0 \\ \vdots \\0
        \end{bmatrix}  =
        R^{-1}\begin{bmatrix}
            0 \\ 0 \\ \vdots \\0
        \end{bmatrix} 
        = 
        \begin{bmatrix}
            0 \\ 0 \\ \vdots \\0
        \end{bmatrix} 
    \end{equation*}

Esto nos dice que $t=0$, lo cual es absurdo pues habíamos partido  de $t \not=0$. Por lo tanto 
\begin{equation*}
    R = \begin{bmatrix}
        1 & * & \cdots &*\\ 0 & * & \cdots &*\\ \vdots \\0& * & \cdots &*
    \end{bmatrix}.
\end{equation*} 

Por inducción podemos probar que $R= \Id_n$. \qed

\end{frame}


\begin{frame}
        \begin{teorema}\label{mtrx-inv-equiv} Sea $A$ matriz $n \times n$ con coeficientes en $\K$. Las siguientes afirmaciones son equivalentes
        \begin{enumerate}
            \item[\textit{i})] $A$ es invertible, \pause
            \item[\textit{ii})] $A$  es equivalente por filas a $\Id_n$,  \pause
            \item[\textit{iii})] $A$ es producto de matrices elementales. \pause
        \end{enumerate}
    \end{teorema} \pause
    \begin{proof} \pause

        \textit{i}) $\Rightarrow$ \textit{ii})\; Sea $R$ una MERF que se obtiene de a $A$.  \pause
        \begin{itemize}
            \item Existen $E_1,\ldots,E_k$ matrices elementales tal que $E_1,\ldots,E_kA = R$.
            \pause
            \item Como $E_1,\ldots,E_k$, $A$ invertibles $\Rightarrow$ $E_1,\ldots,E_kA = R$ invertible. 
            \pause
            \item $R$ es MERF e invertible $\Rightarrow$  $R=\Id_n$.
            \pause
            \item $A$  es equivalente por filas a $R=\Id_n$.
        \end{itemize}
        

    \end{proof}    
\end{frame}


\begin{frame}
            
    \textit{ii}) $\Rightarrow$ \textit{iii})\; 
    \begin{itemize}
        \item $A$  es equivalente por filas a $\Id_n$  $\Rightarrow$
        \pause
        existen $E_1,\ldots,E_k$ matrices  elementales  tal que $E_1\cdots E_kA =\Id_n$.
        \item Sean  $F_1,\ldots,F_k$ las inversas de $E_1,\ldots,E_k$, respectivamente  $\Rightarrow$ 
        $$
        (E_1\cdots E_k)^{-1} = E_k^{-1}\cdots E_1^{-1} = F_k\cdots F_1,
        $$
        luego
        $$F_k\cdots F_1 E_1\cdots E_k A =\Id_n A= A.$$ \pause
        \item Como $E_1\cdots E_kA =\Id_n$ $\Rightarrow$ $F_k\cdots F_1 = F_k\cdots F_1 \Id_n = A$.  \pause
        Es decir 
        $$
        A = F_k\cdots F_1.
        $$
        
    \end{itemize}
    
    \vskip .4cm \pause
    
    \textit{iii}) $\Rightarrow$ \textit{i}) \; Sea $A = E_1E_2,\ldots,E_k$ donde $E_i$  es una matriz elemental ($i=1,\ldots,k$). Como cada $E_i$ es invertible,  el producto de ellos es invertible,  por lo tanto $A$ es invertible.\qed
\end{frame}



\begin{frame}
    
    \begin{corolario}
        Sean $A$ y $B$ matrices $m \times n$. Entonces,   $B$ es equivalente por filas a $A$ si y sólo si existe matriz invertible $P$ de orden $m \times m$ tal que $B =PA$ . 
    \end{corolario} \pause
    \begin{proof} \pause
        ($\Rightarrow$)
        \begin{itemize}
            \item  $B$ es equivalente por filas a $A$ $\Rightarrow$ $\exists \,E_1,\cdots, E_k$ elementales tal que $B =  E_1\ldots E_kA$. \pause
            \item  Sea  $P = E_1\ldots E_k$, luego   $B =PA$. \pause
            \item Cada $E_i$ es invertible $\Rightarrow$ $P =E_1\ldots E_k$ es invertible. \pause
        \end{itemize}
        ($\Leftarrow$)   \pause
        \begin{itemize}
            \item Sea  $P$  matriz invertible tal que $B =PA$. \pause
            \item Como $P$ es invertible $\Rightarrow$ $P = E_1\ldots E_k$, producto de matrices elementales. \pause
            \item Por lo tanto,$B =PA = E_1\ldots E_kA$ es equivalente por filas a $A$. \qed
        \end{itemize}

    \end{proof}    
    
    
\end{frame}



\begin{frame}
    \begin{corolario}
        Sea $A$ matriz $n \times n$. Sean $e_1,\ldots,e_k$  operaciones elementales por filas tal que 
        \begin{equation*}
            e_1(e_{2}(\cdots(e_k(A))\cdots)) =\Id_n.\tag{*}
        \end{equation*} \pause
        Entonces, $A$ invertible y  
        \begin{equation*}
            e_1(e_{2}(\cdots(e_k(\Id_n))\cdots)) =A^{-1}. \tag{**}
        \end{equation*} 
    \end{corolario} \pause
    \vskip -.6cm 
    \begin{proof}  \pause
        \begin{itemize}
            \item (*) $\Rightarrow$ $A$ es equivalente por filas a $\Id_{n}$ $\Rightarrow$  $A$ es invertible.  \pause
            \item Sean las matrices elementales  $E_i = e_i(\Id_n)$ para $i=1,\ldots,k$,  entonces $E_1E_2\ldots E_kA = \Id_n$, por lo tanto, \pause
            \begin{align*}
            E_1E_2\ldots E_kA A^{-1}&= \Id_nA^{-1} \quad \Leftrightarrow \\
            E_1E_2\ldots E_k\Id_n&= A^{-1} \quad \Leftrightarrow \\
            e_1(e_{2}(\cdots(e_k(\Id_n))\cdots)) &=A^{-1}.
            \end{align*}
        \end{itemize}
        \qed
    \end{proof}
\end{frame}

\begin{frame}


Este último corolario nos provee un método sencillo para calcular la inversa de una matriz cuadrada  $A$ invertible.
\vskip .4cm \pause

    \begin{enumerate}
        \item Aplicando operaciones elementales $e_1,\ldots,e_k$ encontramos $R = \Id_n$ la MERF  de $A$. \pause
        \item Aplicando las operaciones elementales $e_1,\ldots,e_k$ a $\Id_n$, obtenemos $A^{-1}$ la inversa de $A$.
    \end{enumerate}
\vskip .4cm \pause
    Para facilitar el cálculo es  conveniente comenzar con $A$ e $\Id_n$ e ir aplicando paralelamente las operaciones elementales por fila.
    \vskip .4cm \pause
    En las próxima filminas veremos un ejemplo.

\end{frame}



\begin{frame}
    \begin{ejemplo}
        Calculemos la inversa (si tiene) de 
        \begin{equation*}
        A=\begin{bmatrix}2&-1\\1&3 \end{bmatrix}.
        \end{equation*}
    \end{ejemplo} \pause
    \begin{solucion}  \pause Trataremos de reducir por filas a $A$ y todas las operaciones elementales las haremos en paralelo partiendo de la matriz identidad: \pause
        \begin{align*}
        [A|\Id] = &\left[\begin{array}{cc|cc}2&-1 &  1&0\\1&3& 0&1\end{array}\right] 
        \stackrel{F_1\leftrightarrow F_2}{\longrightarrow} 
        \left[\begin{array}{cc|cc}1&3& 0&1\\2&-1 &  1&0 \end{array}\right] \\
        &\stackrel{F_2-2 F_1}{\longrightarrow}
        \left[\begin{array}{cc|cc}1&3& 0&1\\0&-7 &  1&-2 \end{array}\right]
        \stackrel{F_2/(-7)}{\longrightarrow}
        \left[\begin{array}{cc|cc}1&3& 0&1\\0&1 &  -1/7&2/7\end{array}\right] \\
        &\stackrel{F_1-3 F_2}{\longrightarrow}
        \left[\begin{array}{cc|cc}1&0&  3/7&1/7\\0&1 &  -1/7&2/7 \end{array}\right].
    \end{align*}
        \vskip .2cm
        
    \end{solucion}
\end{frame}



\begin{frame}
    Luego,  $A$ es invertible y su inversa es  
        \vskip .2cm
        \begin{equation*}
        A^{-1}=\begin{bmatrix}3/7&1/7\\-1/7&2/7 \end{bmatrix}.
        \end{equation*}
        \qed
        \vskip .2cm\vskip .2cm
        \pause
    \begin{itemize}
        \item El lector desconfiado  podrá comprobar, haciendo el producto de matrices, que $AA^{-1} = A^{-1}A=\Id_2$.
    \end{itemize}
    \vskip 2cm
\end{frame}



\begin{frame}
            
    \begin{teorema}
        Sea $A$ matriz $n \times n$ con coeficientes en $\K$. Entonces,  las siguientes afirmaciones son equivalentes.  \pause
        \vskip .2cm
        \begin{enumerate}
            \item[\textit{i})] $A$ es invertible. \pause
            \item[\textit{ii})] El sistema $AX=Y$ tiene una única solución para toda matriz $Y$ de orden $n \times 1$.  \pause
            \item[\textit{iii})] El sistema homogéneo $AX=0$ tiene una única solución ($X=0$). \pause
        \end{enumerate}
    \end{teorema} \pause
    \begin{proof} \pause
        \textit{i}) $\Rightarrow$  \textit{ii}) Sea $X_0$ solución del sistema $AX=Y$, luego
        \begin{equation*}
        AX_0=Y  \quad \Rightarrow \quad  A^{-1}AX_0 = A^{-1}Y  \quad \Rightarrow \quad  X_0 = A^{-1}Y.
        \end{equation*}
        Es decir, $X_0$ es único (siempre igual  a $A^{-1}Y$).  
        \pause
        \vskip .2cm
            
        
    \end{proof}
    
\end{frame}



\begin{frame}

    \textit{ii}) $\Rightarrow$  \textit{iii}) Es trivial, tomando $Y =0$.        
    \vskip .4cm

    \textit{iii}) $\Rightarrow$  \textit{i})  La hipótesis es $AX=0 \Rightarrow X=0$.
    \vskip .2cm
    \begin{itemize}
        \item Sea $R$ la MERF equivalente a $A$.\vskip .2cm \pause
        \item Si $R$ tiene una filas nulas hay variables que no corresponden a 1's principales, luego esas variables son libres. Por lo tanto, el sistema $AX =0$ tiene más de una solución, contradiciendo la hipótesis.\vskip .2cm \pause
        \item  Por lo tanto, $R$ no tiene filas nulas.\vskip .2cm \pause
        \item Como $R$ es una matriz cuadrada y es MERF, tenemos que $R=\Id_n$.\vskip .2cm \pause
        \item Luego $A$ es equivalente por filas a $\Id_n$ $\Rightarrow$ $A$ es invertible.\vskip .2cm 
    \end{itemize}
    \qed
    \vskip 0.8 cm
\end{frame}



\begin{frame}
    \begin{corolario}
        Sea $A$ una matriz $n \times n$ con coeficientes  en $\K$. 
        \begin{enumerate}
            \item Si  existe $B$ matriz $n \times n$ tal que $BA=\Id_n$,   entonces $AB = \Id_n$. 
            \pause
            ($A$ tiene inversa a izquierda $\Rightarrow$ es invertible).
            \item Si  existe $C$ matriz $n \times n$ tal que $AC=\Id_n$,   entonces $CA=\Id_n$. 
            
            ($A$ tiene inversa a derecha $\Rightarrow$ es invertible).
        \end{enumerate}
    \end{corolario}     \pause
    \begin{proof} \pause
        1. Sea $B$ tal que   $BA=\Id_n$. El sistema $AX=0$ tiene una única solución, pues 
        $$AX_0=0 \quad\Rightarrow\quad BAX_0=B0=0  \quad\Rightarrow\quad\Id_nX_0 = 0 \quad\Rightarrow\quad X_0=0.$$ \pause
        Luego, $A$  es invertible (y su inversa es $B$). 
        \vskip .4cm \pause
        2. Sea  $C$  tal que $AC=\Id_n$. Luego $A$ es la inversa a izquierda de $C$. Por  lo que demostramos más arriba,  $C$ es invertible y  su inversa es $A$, es decir $AC=\Id_n$ y $CA=\Id_n$, luego $C$  es invertible. \qed
    \end{proof}    
\end{frame}

\begin{frame}
Elsiguiente teorema reune algunos resultados ya demostrados.

\vskip .3cm
    \begin{block}{Teorema}
    
    Sea $A\in\K^{n\times n}$. Las siguientes afirmaciones son equivalentes \pause
    \begin{enumerate}
     \item $A$ es invertible, \pause
     \item $A$ es equivalente por fila a $\Id_n$, \pause
     \item el sistema $AX=0$ tiene solución única (la trivial), \pause
     \item el sistema $AX=Y$ tiene solución única para todo $Y\in\K^n$ \ (la solución es $A^{-1}Y$), \pause
     \item $A$ es el producto de matrices elementales, \pause
     \item existe $B$  matriz $n \times n$ tal que $BA=\Id$, \pause
     \item existe $C$  matriz $n \times n$ tal que $AC=\Id$, \pause
     \item $\det(A)\neq0$ (esto  lo veremos en próximas clases).
    \end{enumerate}
    
    \end{block}
     
    \end{frame}

\begin{frame}{Matrices invertibles $2 \times 2$}

        Dados $a,b,c,d \in \mathbb R$, determinaremos cuando la matriz $A = \begin{bmatrix*} a&b\\c&d\end{bmatrix*}$  es invertible y en ese caso,  cual es su inversa. 
\pause
    \begin{solucion}\pause
        
Para poder aplicar el método de Gauss-Jordan debemos analizar dos casos:  $a\not=0$ y   $a=0$.\pause

\vskip .3cm
\textbf{Caso 1.}  $a\not=0$.\pause

Como $a\not=0$, entonces 
\begin{equation*}
\begin{bmatrix}a&b\\c&d\end{bmatrix} \stackrel{F_1/a}{\longrightarrow}
\begin{bmatrix}1&\dfrac{b}{a}\\[6pt]c&d\end{bmatrix} \stackrel{F_2 -cF_1}{\longrightarrow}
\begin{bmatrix}1&\dfrac{b}{a}\\[6pt]0&d- c\dfrac{b}{a}\end{bmatrix} =
\begin{bmatrix}1&\dfrac{b}{a}\\[6pt]0&\dfrac{ad-bc}{a}\end{bmatrix}
\end{equation*}   
\end{solucion} 
    
\end{frame}




\begin{frame}
    \textbf{Caso 1.1} $a\not=0$ y  $ad-bc=0$.\pause

    Si $ad-bc=0$,  entonces la matriz se encuentra reducida por filas y la última fila es $0$, luego en ese caso no es invertible. 
    \vskip .4cm 
    \pause
    \textbf{Caso 1.2} $a\not=0$ y  $ad-bc\not=0$.\pause

    Entonces
    \begin{equation*}
    \begin{bmatrix}1&\dfrac{b}{a}\\[8pt]0&\dfrac{ad-bc}{a}\end{bmatrix} \stackrel{a/(ad-bc)\,F_2}{\longrightarrow}
    \begin{bmatrix}1&\dfrac{b}{a}\\[6pt]0&1\end{bmatrix}
    \stackrel{F1-b/a\,F_2}{\longrightarrow}
    \begin{bmatrix}1&0\\0&1\end{bmatrix}.
    \end{equation*} \pause
    Luego, en este  caso $a\not=0$, $ad-bc\not=0$ hemos reducido por filas la matriz $A$  a la identidad y por lo tanto $A$  es invertible. 
    
    
    

\end{frame}


\begin{frame}
    Además, podemos encontrar $A^{-1}$ aplicando a $\Id_2$ las mismas operaciones elementales que reducían $A$ a la identidad:
    \pause
    {\footnotesize
    \begin{align*}
    \begin{bmatrix}1&0\\0&1\end{bmatrix} 
    &\stackrel{F_1/a}{\longrightarrow}
    \begin{bmatrix}\dfrac1a&0\\[8pt]0&1\end{bmatrix} 
    \stackrel{F_2 -cF_1}{\longrightarrow}
    \begin{bmatrix*}[r]\dfrac1a&0\\[8pt]-\dfrac{c}{a}&1\end{bmatrix*}
    \stackrel{a/(ad-bc)\,F_2}{\longrightarrow}
    \begin{bmatrix*}[c]\dfrac1a&0\\[8pt]-\dfrac{c}{ad-bc}&\dfrac{a}{ad-bc}\end{bmatrix*}
    \stackrel{F1-b/a\,F_2}{\longrightarrow}
    \\&\longrightarrow            
    \begin{bmatrix*}[c]\dfrac1a+\dfrac{bc}{a(ad-bc)}&-\dfrac{b}{ad-bc}\\[8pt]-\dfrac{c}{ad-bc}&\dfrac{a}{ad-bc}\end{bmatrix*} 
    =
    \begin{bmatrix*}[c]\dfrac{d}{ad-bc}&-\dfrac{b}{ad-bc}\\[8pt]-\dfrac{c}{ad-bc}&\dfrac{a}{ad-bc}\end{bmatrix*} .
    \end{align*}  }
    \pause
    Concluyendo, en el caso $a\not=0$, $ad-bc\not=0$, $A$  es invertible y 

    \begin{equation}\label{inv-2x2}
    A^{-1} = \dfrac{1}{ad-bc}
    \begin{bmatrix*}[c]d&-b\\-c&a\end{bmatrix*}.
    \end{equation}
\end{frame}



\begin{frame}
    \textbf{Caso 2.} $a=0$.\pause
    \vskip .4cm
    Si $c=0$ o $b=0$ , entonces la matriz no es invertible, pues en ambos casos nos quedan matrices que no pueden ser reducidas por fila a la identidad. 
    \vskip .4cm
    \pause
    Luego la matriz puede ser invertible si $bc\not=0$ y en este caso la reducción por filas es:
    \begin{equation*}
    \begin{bmatrix}0&b\\c&d\end{bmatrix} \stackrel{F_1\leftrightarrow F_2}{\longrightarrow}
    \begin{bmatrix}c&d\\0&b\end{bmatrix} \stackrel{F_1/c}{\longrightarrow}
    \begin{bmatrix}1&\dfrac{d}{c}\\[6pt]0&b\end{bmatrix} \stackrel{F_2/b}{\longrightarrow}
    \begin{bmatrix}1&\dfrac{d}{c}\\[6pt]0&1\end{bmatrix}
    \stackrel{F_1 - d/c F_2}{\longrightarrow}
    \begin{bmatrix}1&0\\0&1\end{bmatrix}.
    \end{equation*}   
    \vskip 2cm
\end{frame}



\begin{frame}
    Luego $A$  es invertible y aplicando estas mismas operaciones elementales a la identidad  obtenemos la inversa:\pause
    \begin{equation*}
    \begin{bmatrix}1&0\\0&1\end{bmatrix} \stackrel{F_1\leftrightarrow F_2}{\longrightarrow}
    \begin{bmatrix}0&1\\1&0\end{bmatrix} \stackrel{F_1/c}{\longrightarrow}
    \begin{bmatrix}0&\dfrac{1}{c}\\[6pt]1&0\end{bmatrix} \stackrel{F_2/b}{\longrightarrow}
    \begin{bmatrix}0&\dfrac{1}{c}\\[6pt]\dfrac{1}{b}&0\end{bmatrix}
    \stackrel{F_1 - d/c F_2}{\longrightarrow}
    \begin{bmatrix}-\dfrac{d}{bc}&\dfrac{1}{c}\\[6pt]\dfrac{1}{b}&0\end{bmatrix}.
    \end{equation*}  \pause
    Es decir, en el caso  que $a=0$, entonces $A$ invertible si  $b c\not=0$ y su inversa es
    \begin{equation}\label{eq-amenos12}
    A^{-1} = \begin{bmatrix}-\dfrac{d}{bc}&\dfrac{1}{c}\\[6pt]\dfrac{1}{b}&0\end{bmatrix} = 
    \dfrac{1}{-bc}
    \begin{bmatrix*}[c]d&-b\\-c&0\end{bmatrix*}.
    \end{equation}\pause
    Es decir, la expresión de la inversa es igual a (\ref{inv-2x2}) (considerando que  $a=0$).
    
\end{frame}



\begin{frame}
    
    Reuniendo los dos casos: de (\ref{inv-2x2}) y (\ref{eq-amenos12}) se deduce:



    \begin{equation*}
        \begin{bmatrix*} a&b\\c&d\end{bmatrix*} \; \text{ es invertible} \quad  \Leftrightarrow \quad  ad-bc\not=0,
    \end{equation*}
    
      y en ese caso,  su inversa viene dada por 

    \begin{equation*}
    \begin{bmatrix*} a&b\\c&d\end{bmatrix*}^{-1} =  \dfrac{1}{ad-bc}
    \begin{bmatrix*}[c]d&-b\\-c&a\end{bmatrix*}
    \end{equation*} 
    
    \vskip .4cm
    \pause
    \begin{observacion}
        Definiremos $\det(A) := ad-bc$. Luego, \pause
        \begin{itemize}
            \item $A$ invertible  si y solo si $\det(A) \ne 0$.\pause
            \item Veremos en las próximas clases que el uso de determinantes permitirá establecer la generalización de este resultado para matrices $n \times n$ con $n\ge 1$.
        \end{itemize}
    \end{observacion}    
    
\end{frame}

\begin{frame}

Terminaremos la clase con un cálculo de matriz inversa.\pause
\vskip .4cm
    \begin{ejemplo}
        Calcular la inversa (si tiene) de la matriz 
        $$A=\begin{bmatrix} 1&-1&2\\ 3&2&4\\ 0&1&-2    \end{bmatrix}.$$ 
    \end{ejemplo}\vskip -.4cm\pause
    \begin{solucion}\vskip -.4cm\pause
        \begin{align*} 
        &\left[\begin{array}{rrr|rrr}    1&-1&2&1&0&0\\ 3&2&4&0&1&0\\ 0&1&-2&0&0&1 \end{array}\right]
        \stackrel{F_2-3 F_1}{\longrightarrow}
        \left[\begin{array}{rrr|rrr}    1&-1&2
        &1&0&0\\ 0&5&-2&-3&1&0\\ 0&1&-2&0&0&1 \end{array}\right]
        \stackrel{F_2\leftrightarrow F_3}{\longrightarrow} \\
        &\longrightarrow 
        \left[\begin{array}{rrr|rrr}    1&-1&2&1&0&0\\ 0&1&-2&0&0&1 \\ 0&5&-2&-3&1&0 \end{array}\right]
        \stackrel{F_1 + F_2}{\longrightarrow}
        \left[\begin{array}{rrr|rrr}    1&0&0&1&0&1\\ 0&1&-2&0&0&1 \\ 0&5&-2&-3&1&0 \end{array}\right]
        \stackrel{F_3-5F_2}{\longrightarrow} 
    \end{align*}
        
    \end{solucion}
\end{frame}


\begin{frame}
    \begin{align*} 
        &\longrightarrow
        \left[\begin{array}{rrr|rrr}    1&0&0&1&0&1\\ 0&1&-2&0&0&1 \\ 0&0&8&-3&1&-5 \end{array}\right]
        \stackrel{F_3/8}{\longrightarrow}
        \left[\begin{array}{rrr|rrr}    1&0&0&1&0&1\\ 0&1&-2&0&0&1 \\ 0&0&1&-\frac38&\frac18&-\frac58 \end{array}\right]
        \stackrel{F_2+2F_3}{\longrightarrow} \\
        &\longrightarrow
        \left[\begin{array}{rrr|rrr}    1&0&0&1&0&1\\ 0&1&0&-\frac34&\frac14&-\frac14 \\ 0&0&1&-\frac38&\frac18&-\frac58 \end{array}\right].
    \end{align*}
    \pause
        Por lo tanto \pause
        \begin{equation*}
        A^{-1} =     \begin{bmatrix*}[r]    1&0&1\\ -\frac34&\frac14&-\frac14 \\ -\frac38&\frac18&-\frac58 \end{bmatrix*}
        \end{equation*}
\qed
\end{frame}


\end{document}


















