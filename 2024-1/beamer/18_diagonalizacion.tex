%\documentclass{beamer} % descomentar para tener pausas
\documentclass[handout]{beamer} % descomentar para no tener pausas
\usetheme{CambridgeUS}
%\setbeamertemplate{background}[grid][step=8 ] % cuadriculado

\usepackage{etex}
\usepackage{t1enc}
\usepackage[spanish,es-nodecimaldot]{babel}
\usepackage{latexsym}
\usepackage[utf8]{inputenc}
\usepackage{verbatim}
\usepackage{multicol}
\usepackage{amsgen,amsmath,amstext,amsbsy,amsopn,amsfonts,amssymb}
\usepackage{amsthm}
\usepackage{calc}         % From LaTeX distribution
\usepackage{graphicx}     % From LaTeX distribution
\usepackage{ifthen}
%\usepackage{makeidx}
\input{random.tex}        % From CTAN/macros/generic
\usepackage{subfigure} 
\usepackage{tikz}
\usepackage[customcolors]{hf-tikz}
\usetikzlibrary{arrows}
\usetikzlibrary{matrix}
\tikzset{
    every picture/.append style={
        execute at begin picture={\deactivatequoting},
        execute at end picture={\activatequoting}
    }
}
\usetikzlibrary{decorations.pathreplacing,angles,quotes}
\usetikzlibrary{shapes.geometric}
\usepackage{mathtools}
\usepackage{stackrel}
%\usepackage{enumerate}
\usepackage{enumitem}
\usepackage{tkz-graph}
\usepackage{polynom}
\polyset{%
    style=B,
    delims={(}{)},
    div=:
}
\renewcommand\labelitemi{$\circ$}
\setlist[enumerate]{label={(\arabic*)}}
%\setbeamertemplate{background}[grid][step=8 ] % cuadriculado
\setbeamertemplate{itemize item}{$\circ$}
\setbeamertemplate{enumerate items}[default]
\definecolor{links}{HTML}{2A1B81}
\hypersetup{colorlinks,linkcolor=,urlcolor=links}


\newcommand{\Id}{\operatorname{Id}}
\newcommand{\img}{\operatorname{Im}}
\newcommand{\nuc}{\operatorname{Nu}}
\newcommand{\im}{\operatorname{Im}}
\renewcommand\nu{\operatorname{Nu}}
\newcommand{\la}{\langle}
\newcommand{\ra}{\rangle}
\renewcommand{\t}{{\operatorname{t}}}
\renewcommand{\sin}{{\,\operatorname{sen}}}
\newcommand{\Q}{\mathbb Q}
\newcommand{\R}{\mathbb R}
\newcommand{\C}{\mathbb C}
\newcommand{\K}{\mathbb K}
\newcommand{\F}{\mathbb F}
\newcommand{\Z}{\mathbb Z}
\newcommand{\N}{\mathbb N}
\newcommand\sgn{\operatorname{sgn}}
\renewcommand{\t}{{\operatorname{t}}}
\renewcommand{\figurename }{Figura}

%
% Ver http://joshua.smcvt.edu/latex2e/_005cnewenvironment-_0026-_005crenewenvironment.html
%

\renewenvironment{block}[1]% environment name
{% begin code
	\par\vskip .2cm%
	{\color{blue}#1}%
	\vskip .2cm
}%
{%
	\vskip .2cm}% end code


\renewenvironment{alertblock}[1]% environment name
{% begin code
	\par\vskip .2cm%
	{\color{red!80!black}#1}%
	\vskip .2cm
}%
{%
	\vskip .2cm}% end code


\renewenvironment{exampleblock}[1]% environment name
{% begin code
	\par\vskip .2cm%
	{\color{blue}#1}%
	\vskip .2cm
}%
{%
	\vskip .2cm}% end code




\newenvironment{exercise}[1]% environment name
{% begin code
	\par\vspace{\baselineskip}\noindent
	\textbf{Ejercicio (#1)}\begin{itshape}%
		\par\vspace{\baselineskip}\noindent\ignorespaces
	}%
	{% end code
	\end{itshape}\ignorespacesafterend
}


\newenvironment{definicion}[1][]% environment name
{% begin code
	\par\vskip .2cm%
	{\color{blue}Definición #1}%
	\vskip .2cm
}%
{%
	\vskip .2cm}% end code

\newenvironment{observacion}[1][]% environment name
{% begin code
	\par\vskip .2cm%
	{\color{blue}Observación #1}%
	\vskip .2cm
}%
{%
	\vskip .2cm}% end code

\newenvironment{ejemplo}[1][]% environment name
{% begin code
	\par\vskip .2cm%
	{\color{blue}Ejemplo #1}%
	\vskip .2cm
}%
{%
	\vskip .2cm}% end code

\newenvironment{ejercicio}[1][]% environment name
{% begin code
	\par\vskip .2cm%
	{\color{blue}Ejercicio #1}%
	\vskip .2cm
}%
{%
	\vskip .2cm}% end code


\renewenvironment{proof}% environment name
{% begin code
	\par\vskip .2cm%
	{\color{blue}Demostración}%
	\vskip .2cm
}%
{%
	\vskip .2cm}% end code



\newenvironment{demostracion}% environment name
{% begin code
	\par\vskip .2cm%
	{\color{blue}Demostración}%
	\vskip .2cm
}%
{%
	\vskip .2cm}% end code

\newenvironment{idea}% environment name
{% begin code
	\par\vskip .2cm%
	{\color{blue}Idea de la demostración}%
	\vskip .2cm
}%
{%
	\vskip .2cm}% end code

\newenvironment{solucion}% environment name
{% begin code
	\par\vskip .2cm%
	{\color{blue}Solución}%
	\vskip .2cm
}%
{%
	\vskip .2cm}% end code



\newenvironment{lema}[1][]% environment name
{% begin code
	\par\vskip .2cm%
	{\color{blue}Lema #1}\begin{itshape}%
		\par\vskip .2cm
	}%
	{% end code
	\end{itshape}\vskip .2cm\ignorespacesafterend
}

\newenvironment{proposicion}[1][]% environment name
{% begin code
	\par\vskip .2cm%
	{\color{blue}Proposición #1}\begin{itshape}%
		\par\vskip .2cm
	}%
	{% end code
	\end{itshape}\vskip .2cm\ignorespacesafterend
}

\newenvironment{teorema}[1][]% environment name
{% begin code
	\par\vskip .2cm%
	{\color{blue}Teorema #1}\begin{itshape}%
		\par\vskip .2cm
	}%
	{% end code
	\end{itshape}\vskip .2cm\ignorespacesafterend
}


\newenvironment{corolario}[1][]% environment name
{% begin code
	\par\vskip .2cm%
	{\color{blue}Corolario #1}\begin{itshape}%
		\par\vskip .2cm
	}%
	{% end code
	\end{itshape}\vskip .2cm\ignorespacesafterend
}

\newenvironment{propiedad}% environment name
{% begin code
	\par\vskip .2cm%
	{\color{blue}Propiedad}\begin{itshape}%
		\par\vskip .2cm
	}%
	{% end code
	\end{itshape}\vskip .2cm\ignorespacesafterend
}

\newenvironment{conclusion}% environment name
{% begin code
	\par\vskip .2cm%
	{\color{blue}Conclusión}\begin{itshape}%
		\par\vskip .2cm
	}%
	{% end code
	\end{itshape}\vskip .2cm\ignorespacesafterend
}







\newenvironment{definicion*}% environment name
{% begin code
	\par\vskip .2cm%
	{\color{blue}Definición}%
	\vskip .2cm
}%
{%
	\vskip .2cm}% end code

\newenvironment{observacion*}% environment name
{% begin code
	\par\vskip .2cm%
	{\color{blue}Observación}%
	\vskip .2cm
}%
{%
	\vskip .2cm}% end code


\newenvironment{obs*}% environment name
	{% begin code
		\par\vskip .2cm%
		{\color{blue}Observación}%
		\vskip .2cm
	}%
	{%
		\vskip .2cm}% end code

\newenvironment{ejemplo*}% environment name
{% begin code
	\par\vskip .2cm%
	{\color{blue}Ejemplo}%
	\vskip .2cm
}%
{%
	\vskip .2cm}% end code

\newenvironment{ejercicio*}% environment name
{% begin code
	\par\vskip .2cm%
	{\color{blue}Ejercicio}%
	\vskip .2cm
}%
{%
	\vskip .2cm}% end code

\newenvironment{propiedad*}% environment name
{% begin code
	\par\vskip .2cm%
	{\color{blue}Propiedad}\begin{itshape}%
		\par\vskip .2cm
	}%
	{% end code
	\end{itshape}\vskip .2cm\ignorespacesafterend
}

\newenvironment{conclusion*}% environment name
{% begin code
	\par\vskip .2cm%
	{\color{blue}Conclusión}\begin{itshape}%
		\par\vskip .2cm
	}%
	{% end code
	\end{itshape}\vskip .2cm\ignorespacesafterend
}




\newcommand{\nc}{\newcommand}

%%%%%%%%%%%%%%%%%%%%%%%%%LETRAS

\nc{\FF}{{\mathbb F}} \nc{\NN}{{\mathbb N}} \nc{\QQ}{{\mathbb Q}}
\nc{\PP}{{\mathbb P}} \nc{\DD}{{\mathbb D}} \nc{\Sn}{{\mathbb S}}
\nc{\uno}{\mathbb{1}} \nc{\BB}{{\mathbb B}} \nc{\An}{{\mathbb A}}

\nc{\ba}{\mathbf{a}} \nc{\bb}{\mathbf{b}} \nc{\bt}{\mathbf{t}}
\nc{\bB}{\mathbf{B}}

\nc{\cP}{\mathcal{P}} \nc{\cU}{\mathcal{U}} \nc{\cX}{\mathcal{X}}
\nc{\cE}{\mathcal{E}} \nc{\cS}{\mathcal{S}} \nc{\cA}{\mathcal{A}}
\nc{\cC}{\mathcal{C}} \nc{\cO}{\mathcal{O}} \nc{\cQ}{\mathcal{Q}}
\nc{\cB}{\mathcal{B}} \nc{\cJ}{\mathcal{J}} \nc{\cI}{\mathcal{I}}
\nc{\cM}{\mathcal{M}} \nc{\cK}{\mathcal{K}}

\nc{\fD}{\mathfrak{D}} \nc{\fI}{\mathfrak{I}} \nc{\fJ}{\mathfrak{J}}
\nc{\fS}{\mathfrak{S}} \nc{\gA}{\mathfrak{A}}
%%%%%%%%%%%%%%%%%%%%%%%%%LETRAS

\title[Clase 18 -  Matriz de una transformación lineal. Diagonalización]{Álgebra/Álgebra II \\Clase 18 -  Matriz de una transformación lineal. Diagonalización}

\author[]{}
\institute[]{\normalsize FAMAF / UNC
    \\[\baselineskip] ${}^{}$
    \\[\baselineskip]
} 
\date[11/06/2024]{11 de junio de 2024}


\begin{document}

\begin{frame}
\maketitle
\end{frame}



\begin{frame}
    \begin{observacion}
        Con sólo conocer cuanto vale la transformación en una base conocemos cuanto vale en todo el espacio.
\vskip .4cm
En efecto, a la matriz de la transformación la armamos calculando la transformación en los vectores de una base. Y la proposición anterior nos dice que para calcular la transformación en un vector cualquier debemos multiplicar por esa matriz.
    \end{observacion}
    \pause
También vale  lo siguiente.
\pause
\begin{teorema}\label{th-tl-definida-en-base}
    Sean $V$ un espacio vectorial de dimensión finita sobre el cuerpo $\K$ y $\{v_1,\ldots,v_n\}$  una base ordenada de $V$. Sean $W$ un espacio vectorial sobre el mismo cuerpo y $\{w_1,\ldots,w_n\}$, vectores cualesquiera de $W$. Entonces existe una única transformación  lineal $T$ de $V$ en $W$ tal que
    \begin{equation*}
    T(v_j) = w_j, \quad j=1,\ldots,n.
    \end{equation*}
\end{teorema}

\end{frame}

\begin{frame}
    \begin{corolario}[(de la Proposición 5.6.3 de las Notas)]\label{cor-cambio-de-base}
        Sea $V$ un espacio vectorial de dimensión finita sobre el cuerpo $\K$, sean $\mathcal B$, $\mathcal B'$  bases ordenadas de $V$. Entonces 
        \begin{equation*}
            [v]_{\mathcal B} = [\Id]_{\mathcal B' \mathcal B}\, [v]_{\mathcal B'}, \quad \forall v \in V.
        \end{equation*}
    \end{corolario}\pause
    \begin{proof}\pause
        Por la Proposición 5.6.3 de las Notas tenemos que 
        $$
        [\Id]_{\mathcal B' \mathcal B}  [v]_{\mathcal B'} = [\Id (v)]_{\mathcal B} = [v]_{\mathcal B}.
        $$
    \end{proof}
    \pause
    \begin{definicion}
        Sea $V$ un espacio vectorial de dimensión finita sobre el cuerpo $\K$ y sean $\mathcal B$ y $\mathcal B'$ bases ordenadas de $V$. La matriz $P =[\Id]_{\mathcal B' \mathcal B}$  es llamada la \textit{matriz de cambio de base} \index{matriz!de cambio de base} de la base $\mathcal B'$  a la base $\mathcal B$. 
    \end{definicion}

    

\end{frame}


\begin{frame}

\begin{teorema} [(Teorema 5.6.7. de las Notas)]
Sean $V$, $W$ y $Z$ espacios vectoriales de dimensión finita con bases $\cB$, $\cB'$ y $\cB''$, respectivamente.

\vskip .2cm

Sean $T:V\longrightarrow W$ y $U:W\longrightarrow Z$ transformaciones lineales.

\vskip .2cm

Entonces la matriz de la transfomación lineal $$UT:V\longrightarrow Z,$$ es decir la composición de $T$ con $U$, satisface
\begin{align*}
[UT]_{\cB\cB''}=[U]_{\cB'\cB''}[T]_{\cB\cB'}
\end{align*}
\end{teorema}
(multiplicación de matrices)
\end{frame}


\begin{frame}
    \begin{corolario}[(Corolario 5.6.8. de las Notas)]\label{cor-inversa-matriz-cambio-de-base} Sea $V$ un espacio vectorial de dimensión finita sobre el cuerpo $\K$ y sean $\mathcal B$ y $\mathcal B'$ bases ordenadas de $V$. La matriz de cambio de base  $P =[\Id]_{\mathcal B' \mathcal B}$ es invertible y su  inversa es $P^{-1} =[\Id]_{\mathcal B \mathcal B'}$
    \end{corolario}\pause
    \begin{proof}\pause
        \begin{equation*}
            [\Id]_{\mathcal B \mathcal B'} P =[\Id]_{\mathcal B \mathcal B'}[\Id]_{\mathcal B' \mathcal B} = [\Id]_{\mathcal B'} = \Id.
        \end{equation*}

        Luego $=[\Id]_{\mathcal B \mathcal B'} = P^{-1}$.
    \end{proof}\qed

\end{frame}


\begin{frame}
\begin{block}{Notación}
Si $T:V\longrightarrow V$ es Una transformación lineal que va de un espacio en si mismo, diremos que $T$ es un \textit{operador lineal en $V$}.
\pause
\vskip .4cm

Si $\cB$ es una base de $V$, \textit{$[T]_{\cB}$} denota la matriz de $T$ en la base $\cB$ y $\cB$, o sea si la base de salida y llegada es la misma, entonces usamos un sólo subíndice.
\end{block}
\pause
\vskip .4cm
\begin{corolario}[(Corolario 5.6.9. de las Notas)]
Sea $V$ un espacio vectorial de dimensión finita con base $\cB$ y $U,T:V\longrightarrow V$ dos transformaciones lineales. Entonces
\begin{enumerate}
 \item $[UT]_{\cB}=[U]_{\cB}[T]_{\cB}$
 \item $T$ es un isomorfismo si y sólo si $[T]_\cB$ es una matriz invertible. En tal caso 
 $$
 [T^{-1}]_{\cB}=[T]_{\cB}^{-1}
 $$
\end{enumerate}
\end{corolario}

\end{frame}


\begin{frame}

Los espacios vectoriales no tienen una base ``natural'' es decir una que es más importante que otras. Cuando trabajamos con bases estamos haciendo una elección y hay infinitas elecciones posible. \pause

\vskip .4cm

El siguiente teorema nos dice como se relacionan las matrices de una transformación lineal respecto a distintas bases.\pause

\vskip .4cm

      
\begin{teorema}\label{th-cambio-de-base}
    Sea $V$ un espacio vectorial de dimensión finita sobre el cuerpo $\K$ y sean
    $$
    \mathcal B = \{v_1,\ldots,v_n \}, \qquad \mathcal B' = \{w_1,\ldots,w_n \}
    $$
    bases ordenadas de $V$. Sea $T$ es un operador lineal sobre V. Entonces, si $P$ es la matriz de cambio de base de $\mathcal B'$ a $\mathcal B$, se cumple que   
    \begin{equation*}
    [T]_{\mathcal B'} = P^{-1}[T]_{\mathcal B}  P.
    \end{equation*}\pause
    Es decir
    \begin{equation*}\label{eq-cambio de base}
        [T]_{\mathcal B'} = [\Id]_{\mathcal B \mathcal B'} [T]_{\mathcal B} [\Id]_{\mathcal B' \mathcal B}.
    \end{equation*}
\end{teorema}


\end{frame}

\begin{frame}

    \begin{proof} \pause Tenemos que $T = \Id T$ y $T =T  \Id$, luego 
        \begin{align*}
            [T]_{\mathcal B'\mathcal B'} &=  [\Id T]_{\mathcal B'\mathcal B'}& \qquad& \\\noalign{\vskip .2cm}
            &= [\Id]_{\mathcal B \mathcal B'} [T]_{\mathcal B' \mathcal B}& &(\text{teorema 4.5.3})  \\\noalign{\vskip .2cm}
            &= [\Id]_{\mathcal B \mathcal B'} [T \Id]_{\mathcal B' \mathcal B}& &\\\noalign{\vskip .2cm}
            &= [\Id]_{\mathcal B \mathcal B'} [T]_{\mathcal B\mathcal B} [\Id]_{\mathcal B' \mathcal B}& &(\text{teorema 4.5.3}).
        \end{align*} 
        
        Por lo tanto  $ [T]_{\mathcal B'} = [\Id]_{\mathcal B \mathcal B'} [T]_{\mathcal B} [\Id]_{\mathcal B' \mathcal B} = P^{-1} [T]_{\mathcal B\mathcal B}P$.
        \vskip .2cm
        \qed
    \end{proof}

\end{frame}

\begin{frame}
    Las fórmulas
    \begin{align}
        &[T]_{\mathcal B'} = [\Id]_{\mathcal B \mathcal B'} [T]_{\mathcal B} [\Id]_{\mathcal B' \mathcal B} \tag{*}\\
        &[\Id]_{\mathcal B \mathcal B'} [\Id]_{\mathcal B' \mathcal B}= \Id \tag{**} \\
        &[v]_{\mathcal B} = [\Id]_{\mathcal B' \mathcal B} [v]_{\mathcal B'} \tag{***}
    \end{align}
    son importantes por si mismas y debemos recordarlas.\pause

    \vskip .4cm

    Como ya dijimos, la matriz $P =[\Id]_{\mathcal B' \mathcal B} $  es llamada la {matriz de cambio de base} de la base $\mathcal B'$  a la base $\mathcal B$. \pause
    \vskip .4cm
    La matriz de cambio de base nos permite calcular los cambios de  coordenadas de los vectores y los cambio de base de las transformaciones lineales.
    \vskip .4cm


\end{frame}



\begin{frame}

\begin{observacion} Sea $T: \K^n \to \K^n$ operador lineal, $\cB = \{v_1,\ldots,v_n\}$ base ordenada y $\cC$ la base canónica,  entonces 
$$
[T]_{\cB\cC} = \begin{bmatrix}
    Tv_1 & Tv_2 & \cdots &Tv_n
\end{bmatrix}.
$$
\end{observacion}\pause
    

\begin{observacion}
Pudimos probar el teorema de cambio de base usando adecuadamente el teorema de cambio de bases, es decir la fórmula
\begin{align*}
    [UT]_{\cB\cB''}=[U]_{\cB'\cB''}[T]_{\cB\cB'}
\end{align*}\pause
Con igual argumento podemos deducir otras igualdades que son útiles para armar todas las matrices a partir de matrices asociadas a bases canónicas, que, como  dijimos en la observación anterior, es fácil calcularlas. 
\end{observacion}
\end{frame}


\begin{frame}

\begin{observacion}
Sea $T:\R^n\longrightarrow\R^n$ una transformación lineal.

\vskip .4cm

Sean $\cB$ y $\cB'$ bases de $\R^n$.

\vskip .4cm

Entonces 
\begin{align*}
[T]_{\cB'\cB}=[\Id]_{\cC\cB}\,[T]_{\cC\cC}\,[\Id]_{\cB'\cC}=[\Id]_{\cB\cC}^{-1}\,[T]_{\cC\cC}\,[\Id]_{\cB'\cC}
\end{align*}\pause

\vskip .4cm

En palabras: para ir de $\cB'$ a $\cB$ con $T$, primero vamos de $\cB'$ a $\cC$, despues de $\cC$ a $\cC$ con $T$ y finalmente vamos de $\cC$ a $\cB$.\pause

\vskip .4cm

Las matrices $[\Id]_{\cB\cC}$ y $[\Id]_{\cB'\cC}$ son fáciles de calcular, ubicamos los vectores de $\cB$ y $\cB'$ como columnas. Similarmente, la matriz de $T$ en la base canónica también es fácil de calcular.
\end{observacion}
\end{frame}

\begin{frame}

\begin{observacion}
Sean $\cB$ y $\cB'$ dos bases de $\R^n$.

Entonces la matriz de cambio de base de $\cB'$ a $\cB$ es 
\begin{align*}
[\Id]_{\cB'\cB}=[\Id]_{\cC\cB}\,[\Id]_{\cB'\cC}=[\Id]_{\cB\cC}^{-1}\,[\Id]_{\cB'\cC}
\end{align*}\pause

\

En palabras, 
``para ir de $\cB'$ a $\cB$, primero vamos de $\cB'$ a $\cC$ y despues vamos de $\cC$ a $\cB$''.

\

Las matrices $[\Id]_{\cB\cC}$ y $[\Id]_{\cB'\cC}$ son fáciles de calcular, ponemos los vectores de $\cB$ y $\cB'$ como columnas.
\end{observacion}
\end{frame}



\begin{frame}{Autovalores y autovectores de una transformación lineal. Diagonalización.}
\begin{itemize}
    \item Ahora veremos los autovalores y autovectores desde una perspectiva de las transformaciones lineales.
    \vskip .5cm \pause
    \item Muchos conceptos y demostraciones se repiten o son similares al caso de la matrices. 
    \vskip .5cm \pause
    \item    
    Sea $V$ espacio vectorial de dimensión finita. Un operador lineal en $V$ es \textit{diagonalizable}  si existe una base ordenada $\mathcal B= \{v_1,\ldots,v_n\}$ de $V$ y $\lambda_1,\ldots,\lambda_n \in \K$ tal que 
    \begin{equation*}\label{eq-auto-49}
        T(v_i) = \lambda_i v_i,\qquad 1\le i \le n. 
    \end{equation*}

\end{itemize}
\end{frame}


\begin{frame}
    \begin{itemize}
        \item En  general, los operadores diagonalizables permiten hacer cálculos sobre ellos en forma sencilla, por ejemplo el núcleo del  operador anterior es $\nuc(T)=\langle v_i: \lambda_i =0 \rangle$ y  su imagen es $\img(T)=\langle v_i: \lambda_i \not=0 \rangle$.\pause
        \vskip .5cm 
        \item Otra propiedad importante de los operadores diagonalizables es que la matriz de la transformación lineal en una base adecuada es diagonal (de allí viene el nombre de diagonalizable).
    \end{itemize}    \vskip .5cm \pause
    $$
        [T]_{\mathcal B} = 
        \begin{bmatrix}
        \lambda_1&0&0&\cdots&0 \\
        0&\lambda_2&0&\cdots&0\\
        0&0&\lambda_3&\cdots&0\\
        \vdots&\vdots&\vdots&&\vdots\\
        0&0&0&\cdots&\lambda_n
        \end{bmatrix}.
        $$
    \end{frame}
    



\begin{frame}

\begin{definicion}[(Definición 5.7.1 de las Notas)]

Sea $T:V\longrightarrow V$ una transformación lineal. Un \textit{autovalor de $T$} es un escalar \textit{$\lambda\in\R$} tal que existe un vector no nulo \textit{$v\in V$} con 
$$T(v)=\lambda v$$
En tal caso, se dice que \textit{$v$ es un autovector (asociado a $\lambda$)}.\pause

\

El \textit{autoespacio asociado a $\lambda$} es
\begin{align*}
V_\lambda=\{v\in V\mid T(v)=\lambda v\}=\{\mbox{autovectores asociados a $\lambda$}\}\cup \{0\} 
\end{align*}
\end{definicion}


\end{frame}



\begin{frame}
    
    \begin{block}{Lema}
        
        Sea $T:V\longrightarrow V$ una transformación lineal y $\cB$ una base de $V$. Las siguientes afirmaciones son equivalentes\pause
        \begin{itemize}
            \item $\lambda\in\R$ y $v\in V$ son autovalor y autovector de $T$\pause
            \item $v\in\operatorname{Nu}(T-\lambda\Id)$\pause
            
            \item $\lambda\in\R$ y $[v]_\cB$ son autovalor y autovector de $[T]_\cB$
        \end{itemize}
    \end{block}\pause
    
\begin{demostracion}\vskip -.8cm\pause
    \begin{align*}
        T(v)=\lambda v&
        \Leftrightarrow
        T(v)-\lambda v=(T-\lambda\Id)v=0
        \\
        \\
        T(v)=\lambda v&\Leftrightarrow \lambda[v]_\cB = [\lambda v]_\cB = [T(v)]_\cB=[T]_\cB\,[v]_\cB
    \end{align*}
\end{demostracion}    \qed
    
\end{frame}

\begin{frame}
    \begin{block}{Consecuencia}\pause
        \begin{itemize}
            \item Para calcular los autovalores y autovectores de una transformación $T$, elegimos una base $\cB$ y calculamos los autovalores y autovectores de $[T]_\cB$.\pause
            
            \item 
            Por el lema anterior, no importa que base elijamos, la matriz de una transformación lineal tiene los autovalores de la transformación lineal.  
        \vskip.2cm\pause
            También podemos probarlo directamente utilizando $[T]_{\mathcal B'} = [\Id]_{\mathcal B \mathcal B'} [T]_{\mathcal B} [\Id]_{\mathcal B' \mathcal B}$.




        \end{itemize}
    \end{block} 
\end{frame}


\begin{frame}
    \begin{teorema}
        Sea $T:V\longrightarrow V$ una tranfomación lineal. Entonces $V_\lambda$ es un subespacio vectorial.
    \end{teorema}\pause
    
    La demostración es como en el caso de matrices. 
    \pause
    
    \
    
    
    Notar que podemos definir $V_\lambda$ para cualquier $\lambda\in\R$. Así, $\lambda$ es autovalor si y sólo si $V_\lambda\neq0$ 
\end{frame}



\begin{frame}
    \begin{teorema}[(Teorema 5.7.3 de las Notas)]\label{th-autovalores-distintos}
        Sea $T:V\longrightarrow V$ una transformación lineal.
        
        \vskip .2cm
        
        Sean $v_1$, ..., $v_m$ autovectores de $T$ con autovalores $\lambda_1$, ..., $\lambda_m$, respectivamente.
        
        \vskip .2cm
        
        Si todos los autovalores son distintos, entonces los vectores $v_1$, ..., $v_m$ son LI 
    \end{teorema}\pause
    
    \begin{demostracion}\pause
        La demostración es por inducción.
        
        \vskip .2cm
        
        \textit{Caso base.} Si $m=1$, entonces vale porque los autovectores son no nulos por definición y en tal caso el conjunto $\{v_1\}$ es LI.
        
        \vskip .2cm\pause
        
        \textit{Paso  inductivo.} Para $m+1$ procedemos como sigue.
    \end{demostracion}
    
    
\end{frame}

\begin{frame}
    
    Sean $c_1, ..., c_{m+1}$ escalares tales que
    \begin{align*}
        c_1v_1+\cdots+c_mv_m+c_{m+1}v_{m+1}&=0 \tag{*}
    \end{align*}\pause\vskip -.8cm
    \begin{align*}
        T(*)\quad&\rightsquigarrow  
        c_1\lambda_1v_1+\cdots+c_m\lambda_mv_m+c_{m+1}\lambda_{m+1}v_{m+1}=0 \\
        \lambda_{m+1}\cdot (*)\quad&\rightsquigarrow 
        c_1\lambda_{m+1}v_1+\cdots+c_m\lambda_{m+1}v_m+c_{m+1}\lambda_{m+1}v_{m+1}=0
    \end{align*}\pause
    Restando miembro a miembro obtenemos
    \begin{align*}\pause
        c_1(\lambda_1-\lambda_{m+1})v_1+\cdots+c_m(\lambda_m-\lambda_{m+1})v_m&=0 
    \end{align*}\pause
    Por (HI), $c_i(\lambda_i-\lambda_{m+1})=0$ para todo $1\leq i\leq m$.
    
    \vskip .2cm\pause
    
    Dado que los autovalores son distintos, $c_i=0$ para todo $1\leq i\leq m$. Por lo tanto $c_{m+1}=0$ y los vectores son LI. \qed
\end{frame}





\begin{frame}
    \begin{definicion}
        Sea $V$ un espacio vectorial de dimensión finita y $T:V\longrightarrow V$ una transformación lineal. Se dice que $T$ es \textit{diagonalizable} si $V$ tiene una base formada por autovectores.
    \end{definicion}
    
    
\end{frame}

\begin{frame}
    
    \begin{block}{Lema}
        Sea $T:V\longrightarrow V$ una transformación lineal diagonalizable. Sea $\cB=\{v_1, ..., v_n\}$ una base de autovectores de $T$ con autovalores asociados $\lambda_1$, ..., $\lambda_n$. Entonces 
        \begin{align*}
            [T]_\cB=
            \begin{bmatrix}
                \lambda_1 & 0 & \cdots & & &  & 0\\ 
                0 & \lambda_2 & 0& \cdots & &  & 0\\
                \vdots & 0 & \ddots & & &  & \vdots\\
                & \vdots &  & & & &  \\
                & &  & & &  \ddots&0\\
                0 & 0 & \cdots & & & 0 & \lambda_{n}
            \end{bmatrix}
        \end{align*}
    \end{block}
    
    En efecto, las columnas de $[T]_\cB$ son los vectores de coordenada
    \begin{align*}
        [T(v_i)]_\cB=[\lambda_i v_i]_\cB=\lambda_i[v_i]_\cB
    \end{align*}
    y $[v_i]_\cB$ tiene todas entradas $0$ excepto un $1$ en el lugar $i$.
    
\end{frame}

\begin{frame}
    Se preguntarán cómo saber si una transformación lineal es diagonalizable.\pause
    
    \
    
    La primera respuesta es: calcular todos los autovalores y autovectores.\pause
    
    \
    
    A continuación veremos algunos criterios a tener en cuenta.\pause
\end{frame}

\begin{frame}
    
    \begin{corolario}        
        Si $V$ es un espacio vectorial de dimensión $n$ y $T:V\longrightarrow V$ es una transformación lineal con $n$ autovalores distintos entonces $T$ es diagonalizable.
        
    \end{corolario}\pause
    
    \begin{demostracion}\pause
        En efecto, cada autovalor tiene al menos un autovector. 
    
    \
    
    Elijamos un autovector $v_1$, ..., $v_n$ para cada autovalor.
    
    \
    
    Por el Teorema de la página \ref{th-autovalores-distintos} estos vectores son LI. 
    
    \
    
    Dado que son tantos como la dimensión de $V$ forman una base.

    \qed
    \end{demostracion}
    
\end{frame}

\begin{frame}
    
    \begin{corolario}
        Si $V$ es un espacio vectorial de dimensión n y $T:V\longrightarrow V$ es una transformación lineal con autovalores $\lambda_1$, ..., $\lambda_n$, todos distintos entre sí. Sean $v_1$, ..., $v_n$ autovectores asociados a $\lambda_1$, ..., $\lambda_n$, respectivamente. Entonces $\{v_1, ..., v_n\}$ es una base de $V$.
    \end{corolario}\pause
    
    \begin{demostracion}\pause
        Poe  el teorema anterior $v_1$, ..., $v_n$ son LI. Como $n= \dim(V)$, entonces  $\{v_1, ..., v_n\}$ es una base de $V$.
        
        \qed
    \end{demostracion}
    
\end{frame}


\begin{frame}
    \begin{proposicion} Sea $T: V \to V$ con  $\mathcal{B} = \{v_1,\ldots,v_n \}$  una base de autovectores con autovalores $\lambda_1,\ldots,\lambda_n$. 
    \vskip .2cm    
    Entonces $\nuc(T)=\langle v_i: \lambda_i =0 \rangle$ e  $\img(T)=\langle v_i: \lambda_i \not=0 \rangle$.            
    \end{proposicion}\pause
    \begin{proof}\pause Reordenemos:  tal que  $\lambda_i =0$ para $1 \le i \le k$ y $\lambda_i \ne 0$ para $k < i \le n$. 
    $$
    v \in V \quad \Rightarrow \quad v = x_1v_1 + \cdots+ x_k v_k+ x_{k+1} v_{k+1}+\cdots+ x_n v_n,
    $$
    y entonces
    \begin{equation}\label{eq-av-01}
        T(v) =  \lambda_{k+1}x_{k+1} v_{k+1}+\cdots+ \lambda_nx_n v_n.
    \end{equation}
    Luego 
    \begin{align*}
        T(v) =0 \quad &\Leftrightarrow \quad x_{k+1} = \cdots = x_n=0  \quad \Leftrightarrow \quad v =  x_1v_1 + \cdots+ x_k v_k \\
        \quad &\Leftrightarrow \quad v \in \langle v_i: \lambda_i =0 \rangle. 
    \end{align*}
    
    \end{proof}
\end{frame}



\begin{frame}
    
    También es claro por la ecuación (\ref{eq-av-01}) que 
    \begin{align*}
        \img(T) &= \{\lambda_{k+1}x_{k+1} v_{k+1}+\cdots+ \lambda_nx_n v_n: x_i \in \K \} \\
        &=\{\mu_{k+1} v_{k+1}+\cdots+ \mu_n v_n: \mu_i \in \K \}\\
        &= \langle v_i: \lambda_i \not=0 \rangle.
    \end{align*}
    \qed
    \vskip 3cm
\end{frame}


\begin{frame}
    
    \begin{definicion}
        Sea $V$ un espacio vectorial de dimensión finita y $T:V\longrightarrow V$ una transformación lineal.
        El \textit{polinomio característico de $T$} es el polinomio característico de la matriz $[T]_\cB$ donde $\cB$ es una base de $V$. Es decir,
        \begin{align*}
            \chi_T(x)=\det([T]_\cB-x\Id) 
        \end{align*}
    \end{definicion}\pause
    
    \begin{observacion}
        Notar que no importa que base usemos para calcular el polinomio característico dado que $[T]_{\cB'}=P^{-1}[T]_{\cB}P$.
        \begin{align*}
        \det([T]_\cB'-x\Id)  &= \det(P^{-1}[T]_\cB P-xP^{-1}P) \\
        &= P^{-1}\det([T]_\cB -x\Id)P = \det([T]_\cB-x\Id). 
        \end{align*}
    \end{observacion}
    
\end{frame}

\begin{frame}
    \begin{proposicion}
        Sea $T:V\longrightarrow V$ una transformación lineal. Entonces
        $\lambda$ es autovalor de $T$ si y sólo si $\lambda$ es raíz del polinomio característico.
    \end{proposicion}\pause
    
    \begin{block}{Corolario}
        Sea $T:V\longrightarrow V$ una transformación lineal. Supongamos que 
        \begin{align*}
            \chi_T(x)=(-1)^n (x-\lambda_1)^{d_1}\cdots (x-\lambda_m)^{d_m}
        \end{align*}
        Entonces 
        \begin{enumerate}
            \item $1\leq\dim V_{\lambda_i}\leq d_i$ para todo $i$.
            \item $T$ es diagonalizable si y sólo si $\dim V_{\lambda_i}=d_i$ para todo $i$.
        \end{enumerate}
    \end{block}
\end{frame}


\begin{frame}
    
    El punto de partida de esta sección es la siguiente simple observación.\pause
    
    \begin{observacion}
        Sea $T:V\longrightarrow W$ una tranfomación lineal. Si conocemos cuanto vale $T(v_i)$ para todos los vectores de una base $\cB=\{v_1, ..., v_n\}$ de $V$, entonces podemos calcular $T(v)$ para todo $v\in V$.
    \end{observacion} \pause
    
    Pues al ser $\cB$ una base, si $v\in V$ entonces $v=x_1v_1+\cdots+ x_nv_n$ y por lo tanto
    \begin{align*}
        T(v)=x_1T(v_1)+\cdots+ x_nT(v_n) 
    \end{align*}
    
\end{frame}



\end{document}



