%\documentclass{beamer} % descomentar para tener pausas
\documentclass[handout]{beamer} % descomentar para no tener pausas
\usetheme{CambridgeUS}
%\setbeamertemplate{background}[grid][step=8 ] % cuadriculado

\usepackage[utf8]{inputenc}%esto permite (en Windows) escribir directamente 
\usepackage{graphicx}
\usepackage{array}
\usepackage{tikz} 
\usetikzlibrary{shapes,arrows,babel,decorations.pathreplacing}
\usepackage{verbatim} 
\usepackage{xcolor} 
\usepackage{amsgen,amsmath,amstext,amsbsy,amsopn,amsfonts,amssymb}
\usepackage{amsthm}
\usepackage{tikz}
\usepackage{tkz-graph}
\usepackage{mathtools}
\usepackage{xcolor}

%\setbeamertemplate{background}[grid][step=8 ]
\setbeamertemplate{itemize item}{$\circ$}
\setbeamertemplate{enumerate items}[default]

\definecolor{links}{HTML}{2A1B81}
\hypersetup{colorlinks,linkcolor=,urlcolor=links}

\newcommand{\img}{\operatorname{Im}}
\newcommand{\nuc}{\operatorname{Nu}}
\renewcommand\nu{\operatorname{Nu}}
\newcommand{\la}{\langle}
\newcommand{\ra}{\rangle}
\renewcommand{\t}{{\operatorname{t}}}
\renewcommand{\sin}{{\,\operatorname{sen}}}
\newcommand{\Q}{\mathbb Q}
\newcommand{\R}{\mathbb R}
\newcommand{\C}{\mathbb C}
\newcommand{\K}{\mathbb K}
\newcommand{\F}{\mathbb F}
\newcommand{\Z}{\mathbb Z}

\renewcommand{\figurename }{Figura}


\setbeamercolor{block}{fg=red, bg=red!40!white}
\setbeamercolor{block example}{use=structure,fg=black,bg=white!20!white}

\renewenvironment{block}[1]% environment name
{% begin code
    \par\vskip .2cm%
    {\color{blue}#1}%
    \vskip .2cm
}%
{%
    \vskip .2cm}% end code


\renewenvironment{alertblock}[1]% environment name
{% begin code
    \par\vskip .2cm%
    {\color{red!80!black}#1}%
    \vskip .2cm
}%
{%
    \vskip .2cm}% end code


\renewenvironment{exampleblock}[1]% environment name
{% begin code
    \par\vskip .2cm%
    {\color{blue}#1}%
    \vskip .2cm
}%
{%
    \vskip .2cm}% end code




\newenvironment{exercise}[1]% environment name
{% begin code
    \par\vspace{\baselineskip}\noindent
    \textbf{Ejercicio (#1)}\begin{itshape}%
        \par\vspace{\baselineskip}\noindent\ignorespaces
    }%
    {% end code
    \end{itshape}\ignorespacesafterend
}


\newenvironment{definicion}% environment name
{% begin code
    \par\vskip .2cm%
    {\color{blue}Definición}%
    \vskip .2cm
}%
{%
    \vskip .2cm}% end code

\newenvironment{observacion}% environment name
{% begin code
    \par\vskip .2cm%
    {\color{blue}Observación}%
    \vskip .2cm
}%
{%
    \vskip .2cm}% end code

\newenvironment{ejemplo}% environment name
{% begin code
    \par\vskip .2cm%
    {\color{blue}Ejemplo}%
    \vskip .2cm
}%
{%
    \vskip .2cm}% end code

\newenvironment{ejercicio}% environment name
{% begin code
    \par\vskip .2cm%
    {\color{blue}Ejercicio}%
    \vskip .2cm
}%
{%
    \vskip .2cm}% end code


\renewenvironment{proof}% environment name
{% begin code
    \par\vskip .2cm%
    {\color{blue}Demostración}%
    \vskip .2cm
}%
{%
    \vskip .2cm}% end code



\newenvironment{demostracion}% environment name
{% begin code
    \par\vskip .2cm%
    {\color{blue}Demostración}%
    \vskip .2cm
}%
{%
    \vskip .2cm}% end code

\newenvironment{idea}% environment name
{% begin code
    \par\vskip .2cm%
    {\color{blue}Idea de la demostración}%
    \vskip .2cm
}%
{%
    \vskip .2cm}% end code

\newenvironment{solucion}% environment name
{% begin code
    \par\vskip .2cm%
    {\color{blue}Solución}%
    \vskip .2cm
}%
{%
    \vskip .2cm}% end code



\newenvironment{lema}% environment name
{% begin code
    \par\vskip .2cm%
    {\color{blue}Lema}\begin{itshape}%
        \par\vskip .2cm
    }%
    {% end code
    \end{itshape}\vskip .2cm\ignorespacesafterend
}

\newenvironment{proposicion}% environment name
{% begin code
    \par\vskip .2cm%
    {\color{blue}Proposición}\begin{itshape}%
        \par\vskip .2cm
    }%
    {% end code
    \end{itshape}\vskip .2cm\ignorespacesafterend
}

\newenvironment{teorema}% environment name
{% begin code
    \par\vskip .2cm%
    {\color{blue}Teorema}\begin{itshape}%
        \par\vskip .2cm
    }%
    {% end code
    \end{itshape}\vskip .2cm\ignorespacesafterend
}


\newenvironment{corolario}% environment name
{% begin code
    \par\vskip .2cm%
    {\color{blue}Corolario}\begin{itshape}%
        \par\vskip .2cm
    }%
    {% end code
    \end{itshape}\vskip .2cm\ignorespacesafterend
}

\newenvironment{propiedad}% environment name
{% begin code
    \par\vskip .2cm%
    {\color{blue}Propiedad}\begin{itshape}%
        \par\vskip .2cm
    }%
    {% end code
    \end{itshape}\vskip .2cm\ignorespacesafterend
}

\newenvironment{conclusion}% environment name
{% begin code
    \par\vskip .2cm%
    {\color{blue}Conclusión}\begin{itshape}%
        \par\vskip .2cm
    }%
    {% end code
    \end{itshape}\vskip .2cm\ignorespacesafterend
}





%%%%%%%%%%%%%%%%%%%%%%%%%%%%%%%%%%%%%%%%%%%%%%%%%%%%%%%

\newcommand{\nc}{\newcommand}


%%%%%%%%%%%%%%%%%%%%%%%%%LETRAS
\nc{\RR}{{\mathbb R}} \nc{\CC}{{\mathbb C}} \nc{\ZZ}{{\mathbb Z}}
\nc{\FF}{{\mathbb F}} \nc{\NN}{{\mathbb N}} \nc{\QQ}{{\mathbb Q}}
\nc{\PP}{{\mathbb P}} \nc{\DD}{{\mathbb D}} \nc{\Sn}{{\mathbb S}}
\nc{\uno}{\mathbb{1}} \nc{\BB}{{\mathbb B}} \nc{\An}{{\mathbb A}}

\nc{\ba}{\mathbf{a}} \nc{\bb}{\mathbf{b}} \nc{\bt}{\mathbf{t}}
\nc{\bB}{\mathbf{B}}

\nc{\cP}{\mathcal{P}} \nc{\cU}{\mathcal{U}} \nc{\cX}{\mathcal{X}}
\nc{\cE}{\mathcal{E}} \nc{\cS}{\mathcal{S}} \nc{\cA}{\mathcal{A}}
\nc{\cC}{\mathcal{C}} \nc{\cO}{\mathcal{O}} \nc{\cQ}{\mathcal{Q}}
\nc{\cB}{\mathcal{B}} \nc{\cJ}{\mathcal{J}} \nc{\cI}{\mathcal{I}}
\nc{\cM}{\mathcal{M}} \nc{\cK}{\mathcal{K}}

\nc{\fD}{\mathfrak{D}} \nc{\fI}{\mathfrak{I}} \nc{\fJ}{\mathfrak{J}}
\nc{\fS}{\mathfrak{S}} \nc{\gA}{\mathfrak{A}}
%%%%%%%%%%%%%%%%%%%%%%%%%LETRAS




%%%%%%%%%%%%%%%%%yetter drinfield
\newcommand{\ydg}{{}_{\ku G}^{\ku G}\mathcal{YD}}
\newcommand{\ydgdual}{{}_{\ku^G}^{\ku^G}\mathcal{YD}}
\newcommand{\ydf}{{}_{\ku F}^{\ku F}\mathcal{YD}}
\newcommand{\ydgx}{{}_{\ku \Gx}^{\ku \Gx}\mathcal{YD}}

\newcommand{\ydgxy}{{}_{\ku \Gy}^{\ku \Gx}\mathcal{YD}}

\newcommand{\ydixq}{{}_{\ku \ixq}^{\ku \ixq}\mathcal{YD}}

\newcommand{\ydl}{{}^H_H\mathcal{YD}}
\newcommand{\ydll}{{}_{K}^{K}\mathcal{YD}}
\newcommand{\ydh}{{}^H_H\mathcal{YD}}
\newcommand{\ydhdual}{{}^{H^*}_{H^*}\mathcal{YD}}


\newcommand{\ydha}{{}^H_A\mathcal{YD}}
\newcommand{\ydhhaa}{{}^{\Hx}_{\Ay}\mathcal{YD}}


\newcommand{\wydh}{\widehat{{}^H_H\mathcal{YD}}}
\newcommand{\ydvh}{{}^{\ac(V)}_{\ac(V)}\mathcal{YD}}
\newcommand{\ydrh}{{}^{R\# H}_{R\# H}\mathcal{YD}}
\newcommand{\ydho}{{}^{H^{\dop}}_{H^{\dop}}\mathcal{YD}}
\newcommand{\ydhsw}{{}^{H^{\sw}}_{H^{\sw}}\mathcal{YD}}

\newcommand{\ydhlf}{{}^H_H\mathcal{YD}_{\text{loc fin}}}
\newcommand{\ydholf}{{}^{H^{\dop}}_{H^{\dop}}\mathcal{YD}_{\text{loc fin}}}

\nc{\yd}{\mathcal{YD}}

\newcommand{\ydsn}{{}^{\ku{\Sn_n}}_{\ku{\Sn_n}}\mathcal{YD}}
\newcommand{\ydsnd}{{}^{\ku^{\Sn_n}}_{\ku^{\Sn_n}}\mathcal{YD}}
\nc{\ydSn}[1]{{}^{\Sn_{#1}}_{\Sn_{#1}}\yd}
\nc{\ydSndual}[1]{{}^{\ku^{\Sn_{#1}}}_{\ku^{\Sn_{#1}}}\yd}
\newcommand{\ydstres}{{}^{\ku^{\Sn_3}}_{\ku^{\Sn_3}}\mathcal{YD}}
%%%%%%%%%%%%%%%%%yetter drinfield


%%%%%%%%%%%%%%%%%%%%%%%%%%%%%Operatorename
\newcommand\Irr{\operatorname{Irr}}
\newcommand\id{\operatorname{id}}
\newcommand\ad{\operatorname{ad}}
\newcommand\Ad{\operatorname{Ad}}
\newcommand\Ext{\operatorname{Ext}}
\newcommand\tr{\operatorname{tr}}
\newcommand\gr{\operatorname{gr}}
\newcommand\grdual{\operatorname{gr-dual}}
\newcommand\Gr{\operatorname{Gr}}
\newcommand\co{\operatorname{co}}
\newcommand\car{\operatorname{car}}
\newcommand\rk{\operatorname{rg}}
\newcommand\ord{\operatorname{ord}}
\newcommand\cop{\operatorname{cop}}
\newcommand\End{\operatorname{End}}
\newcommand\Hom{\operatorname{Hom}}
\newcommand\Alg{\operatorname{Alg}}
\newcommand\Aut{\operatorname{Aut}}
\newcommand\Int{\operatorname{Int}}
\newcommand\Id{\operatorname{Id}}
\newcommand\qAut{\operatorname{qAut}}
\newcommand\Map{\operatorname{Map}}
\newcommand\Jac{\operatorname{Jac}}
\newcommand\Rad{\operatorname{Rad}}
\newcommand\Rep{\operatorname{Rep}}
\newcommand\Ker{\operatorname{Ker}}
\newcommand\Img{\operatorname{Im}}
\newcommand\Ind{\operatorname{Ind}}
\newcommand\Comod{\operatorname{Comod}}
\newcommand\Reg{\operatorname{Reg}}
\newcommand\Pic{\operatorname{Pic}}
\newcommand\textitb{\operatorname{Emb}}
\newcommand\op{\operatorname{op}}
\newcommand\Perm{\operatorname{Perm}}
\newcommand\Res{\operatorname{Res}}
\newcommand\res{\operatorname{res}}
\newcommand{\sop}{\operatorname{Supp}}
\newcommand\Cent{\operatorname{Cent}}
\newcommand\sgn{\operatorname{sgn}}
\nc{\GL}{\operatorname{GL}}
%%%%%%%%%%%%%%%%%%%%%%%%%%%%%Operatorename

%%%%%%%%%%%%%%%%%%%%%%%%%%%%%Usuales de hopf
\nc{\D}{\Delta} 
\nc{\e}{\varepsilon}
\nc{\adl}{\ad_\ell}
\nc{\ot}{\otimes}
\nc{\Ho}{H_0} 
\nc{\GH}{G(H)} 
\nc{\coM}{\mathcal{M}^\ast(2,k)} 
\nc{\PH}{\cP(H)}
\nc{\Ftwist}{\overset{\curvearrowright}F}
\nc{\rep}{{\mathcal Rep}(H)}
\newcommand{\deltad}{_*\delta}
\newcommand{\B}{\mathfrak{B}}
\newcommand{\wB}{\widehat{\mathfrak{B}}}
\newcommand{\Cg}[1]{C_{G}(#1)}
\nc{\hmh}{{}_H\hspace{-1pt}{\mathcal M}_H}
\nc{\hm}{{}_H\hspace{-1pt}{\mathcal M}}
\renewcommand{\_}[1]{_{\left[ #1 \right]}}
\renewcommand{\^}[1]{^{\left[ #1 \right]}}
%%%%%%%%%%%%%%%%%%%%%%%%%%%%%Usuales de hopf

%%%%%%%%%%%%%%%%%%%%%%%%%%%%%%%%Usuales
\nc{\im}{\mathtt{i}}
\renewcommand{\Re}{{\rm Re}}
\renewcommand{\Im}{{\rm Im}}
\nc{\Tr}{\mathrm{Tr}} 
\nc{\cark}{char\,k} 
\nc{\ku}{\Bbbk} 
\newcommand{\fd}{finite dimensional}
%%%%%%%%%%%%%%%%%%%%%%%%%%%%%%%%Usuales

%%%%%%%%%%%%%%%%%%%%%%%%Especiales para liftings de duales de Sn
\newcommand{\xij}[1]{x_{(#1)}}
\newcommand{\yij}[1]{y_{(#1)}}
\newcommand{\Xij}[1]{X_{(#1)}}
\newcommand{\dij}[1]{\delta_{#1}}
\newcommand{\aij}[1]{a_{(#1)}}
\newcommand{\hij}[1]{h_{(#1)}}
\newcommand{\gij}[1]{g_{(#1)}}
\newcommand{\eij}[1]{e_{(#1)}}
\newcommand{\fij}[1]{f_{#1}}
\newcommand{\tij}[1]{t_{(#1)}}
\newcommand{\Tij}[1]{T_{(#1)}}
\newcommand{\mij}[1]{m_{(#1)}}
\newcommand{\Lij}[1]{L_{(#1)}}
\newcommand{\mdos}[2]{m_{(#1)(#2)}}
\newcommand{\mtres}[3]{m_{(#1)(#2)(#3)}}
\newcommand{\mcuatro}{m_{\textsf{top}}}
\newcommand{\trid}{\triangleright}
\newcommand{\link}{\sim_{\ba}}
%%%%%%%%%%%%%%%%%%%%%%%%Especiales para liftings de duales de Sn


%%%%%%%%%%%beaamer%%%%%%%%%%%%%%%%%
% Header: Secciones una arriba de la otra.
% \usetheme{Copenhagen}
% \usetheme{Warsaw}

% Header: Secciones una al lado de la otra. Feo footer.
% Los circulitos del header son medio chotos tambien.
% \usetheme[compress]{Ilmenau}
% Muy bueno: difuminado, sin footer:
%\usetheme{Frankfurt}
% agrego footer como el de Copenhagen.
%\useoutertheme[footline=authortitle,subsection=false]{miniframes}




\title[Clase 5 - Método de Gauss-Jordan]{Álgebra/Álgebra II \\ Clase 5 -Sistemas de ecuaciones lineales 3}

\author[]{}
\institute[]{\normalsize FAMAF / UNC
    \\[\baselineskip] ${}^{}$
    \\[\baselineskip]
}
\date[26/03/2024]{26 de marzo de 2024}

%\titlegraphic{\includegraphics[width=0.2\textwidth]{logo_gimp100.pdf}}

% Converted to PDF using ImageMagick:
% # convert logo_gimp100.png logo_gimp100.pdf


\begin{document}

\begin{frame}
\maketitle
\end{frame}





\begin{frame}

En esta clase  veremos:
\begin{itemize}
 \item Qué operaciones hacer para transformar un sistema cualquiera en otro donde las ecuaciones se resuelven fácilmente.
 \item Cómo saber si el sistema tiene o no tiene solución y  en el caso de tener solución, si tiene una o infinitas.
\end{itemize}
\vskip .6cm

Todo se hará sistemáticamente pasando de sistemas de ecuaciones a matrices y reduciendo, vía el algoritmo de Gauss-Jordan,  la matriz  del sistema  a una matriz MERF.

\




\end{frame}


\begin{frame}
        \begin{definicion}
        Una matriz $A$ de $m \times n$ se llama \textit{reducida por filas}\index{matriz!reducida por filas} o \textit{MRF}\index{MRF} si 
        \begin{enumerate}
            \item[(a)] la primera entrada no nula de una fila de $A$ es 1. Este 1 es llamado \textit{1 principal}\index{1 principal de una MRF}.
            \item[(b)] Cada columna de $A$ que contiene un  1 principal tiene todos los otros elementos iguales a 0. 
        \end{enumerate} 
    \end{definicion} 
    \pause
    \begin{ejemplo} Las siguientes matrices son MRF:
        \begin{equation*}
        \begin{bmatrix}1 & 0& -1 \\ 0&1&3\\  0&0&0 \end{bmatrix}\; \qquad 
        \begin{bmatrix} 0&1&3 \\1 & 0& -1\\  0&0&0 \end{bmatrix};
        \end{equation*}
        \end{ejemplo}
        \end{frame}
    \begin{frame}
        \begin{ejemplo}
            Las siguientes matrices, no son MRF:
            \begin{equation*}
            \begin{bmatrix}1 & 0& 1 \\ 0&2&3\\  0&0&0 \end{bmatrix}\; \text{no cumple (a),} \qquad
            \begin{bmatrix}1 & 0& -1 \\ 0&1&3\\  0&0&1 \end{bmatrix}\; \text{no cumple (b)}. 
            \end{equation*}
        \end{ejemplo}
\end{frame}


\begin{frame}
    \begin{definicion}
            Una matriz $A$ de $m \times n$ es \textit{escalón reducida por fila}\index{matriz!escalón reducida por fila} o \textit{MERF}\index{MERF} si,  es  MRF y
        \begin{enumerate}
            \item[\textit{c})] todas las filas cuyas entradas son todas iguales a cero están al final de la matriz, y
            \item[\textit{d})] en dos filas consecutivas no nulas el 1 principal de la fila inferior está más a la derecha que el 1 principal de la fila superior. 
        \end{enumerate}
    \end{definicion}
\pause
    \begin{ejemplo}
    Las siguientes matrices son MERF:
    \begin{equation*}
    \begin{bmatrix} 1&0&0&2 \\ 0&1&0&5 \\ 0&0&1&4\end{bmatrix}, \qquad
    \begin{bmatrix} 1&0&0 \\ 0&1&0 \\ 0&0&0\end{bmatrix}, \qquad
    \begin{bmatrix} 0&0 \\ 0&0\end{bmatrix}, \qquad
    \begin{bmatrix} 0&1&2&0&1 \\ 0&0&0&1&0 \\ 0&0&0&0&0 \\ 0&0&0&0&0\end{bmatrix}.
    \end{equation*}
\end{ejemplo}
    
\end{frame}


\begin{frame}
        \begin{ejemplo} Las siguientes matrices son MRF, pero  no MERF:
        \begin{equation*}
        \begin{bmatrix}1 & 0& -1 \\  0&0&0\\ 0&1&3 \end{bmatrix}\;  \text{no cumple (c),} \qquad
        \begin{bmatrix} 0&1&3 \\1 & 0& -1\\  0&0&0 \end{bmatrix}\; \text{no cumple (d).} \qquad
        \end{equation*}\pause
        Las siguientes matrices, no son MRF:
        \begin{equation*}
        \begin{bmatrix}1 & 0& 1 \\ 0&2&3\\  0&0&0 \end{bmatrix}\; \text{no cumple (a),} \qquad
        \begin{bmatrix}1 & 0& -1 \\ 0&1&3\\  0&0&1 \end{bmatrix}\; \text{no cumple (b)}. 
        \end{equation*}
    \end{ejemplo}
\end{frame}



\begin{frame}
    En  general  una matriz escalón reducida por fila (MERF) tiene la siguiente forma
    
    \begin{align*}
    \left[
    \begin{matrix}
    0& \cdots & 1 & *&0&* &* &0& * &*\\
    0& \cdots &0 &\cdots & 1& *  & * & 0& * &*\\
    \vdots&  &\vdots &  &\vdots && &\vdots& &\vdots\\
    0& \cdots &0& \cdots & 0&\cdots &\cdots & 1& * &* \\
    0& \cdots &0& \cdots & 0&\cdots &\cdots & 0& \cdots &0 \\
    \vdots&  &\vdots &  &\vdots && &\vdots& &\vdots\\
    0& \cdots &0& \cdots & 0&\cdots &\cdots & 0& \cdots &0
    \end{matrix}
    \right]
    \end{align*}
    
\end{frame}

\begin{frame}
        
    \begin{ejemplo}
        Sea $\operatorname{Id}_n$ la matriz  $n \times n$ definida 
        \begin{equation*}
        [\operatorname{Id}_n]_{ij} = \left\{ \begin{matrix}
        1 && \text{si $i=j$,}\\
        0 && \text{si $i\not=j$,}
        \end{matrix}\right.\qquad \text{ o bien }\qquad \operatorname{Id}_n =
        \begin{bmatrix}
        1 & 0 & \cdots & 0 \\
        0 & 1 & \cdots & 0 \\
        \vdots & \vdots & \ddots & \vdots \\
        0 & 0 & \cdots & 1
        \end{bmatrix}
        \end{equation*}
        
        \pause
        Llamaremos a $\operatorname{Id}_n$ la \textit{matriz identidad $n \times n$}\index{matriz!identidad $n \times n$}. Observar que $\operatorname{Id}_n$ es una matriz escalón reducida por fila.
        
        
    \end{ejemplo}
\end{frame}


\begin{frame}
    
    \begin{observacion}
     Es muy fácil obtener la solución de un sistema $AX=Y$ donde $A$  es una MERF. 
    \end{observacion}
    \pause
    \begin{ejemplo}
    La solución del sistema $AX=Y$ con
    \begin{align*}
    A=\left[
    \begin{matrix} 
    1&0&0\\ 
    0&1&0\\ 
    0&0&1
    \end{matrix}
    \right]
    \quad\mbox{e}\quad
    Y=\left[
    \begin{array}{r}
    -1\\ 
    0\\ 
    1
    \end{array}
    \right]
    \end{align*}
    es $(x_1,x_2,x_3)=(-1,0,1)$.
    
    \
    \pause
    En efecto, si escribimos explícitamente el sistema la solución queda determinada automáticamente:
    \begin{align*}
    \left\{
    \begin{array}{rcl}
    x_1&=&-1\\
    x_2&=&0\\
    x_3&=&1
    \end{array}
    \right.
    \end{align*}
        \end{ejemplo}
    
\end{frame}



\begin{frame}
    
    \begin{ejemplo}
    El conjunto de soluciones del sistema $AX=Y$ con
    \begin{align*}
    A=\left[
    \begin{matrix} 
    1&0&2\\ 
    0&1&-\frac{1}{3}\\ 
    \end{matrix}
    \right]
    \quad\mbox{e}\quad
    Y=\left[
    \begin{array}{r}
    1\\ 
    -\frac{1}{3}
    \end{array}
    \right]
    \end{align*}
    es 
    \begin{align*}
    \left\{(-2x_3+1,\,\frac{1}{3}x_3-\frac{1}{3},\,x_3)\mid x_3\in\RR\right\}.
    \end{align*}
    
    \
    \pause
    En efecto, si escribimos explicitamente el sistema:
    \begin{align*}
    \left\{
    \begin{array}{rcl}
    x_1+2x_3&=&1\\
    x_2-\frac{1}{3}x_3&=&-\frac{1}{3}
    \end{array}
    \right.\qquad
    \Longrightarrow
    \qquad
    \begin{array}{rcl}
    x_1&=&-2x_3+1\\
    x_2&=&\frac{1}{3}x_3-\frac{1}{3}
    \end{array}
    \end{align*}
        \end{ejemplo}
\end{frame}

\begin{frame}
    \begin{teorema}\label{th-mrf}
        Toda matriz $m \times n$ sobre $\K$ es equivalente por fila a una matriz escalón reducida por fila.
    \end{teorema}
\pause
\begin{block}{Idea de la demostración}
        \begin{enumerate}
        \item[P1.]  Nos ubicamos en la primera fila.\pause
        \item[P2.]  Si la fila es 0 y no es la última, pasar a la fila siguiente y de nuevo {\color{blue}P2}.\pause
        \item[P3.]  Si la fila no es 0, 
        \begin{enumerate}
            \item[P3.1] si el primera entrada no nula está en  la columna $k$ y su valor es $c$, dividir la fila por $c$ (ahora la primera entrada no nula vale 1),
            \item[P3.2] con operaciones elementales del tipo $F_r+ tF_s$ hacer 0  todas las entradas en la columna $k$ (menos la de la columna actual).    
        \end{enumerate}\pause
        \item[P4.] Si la fila no es la última, pasar a la fila siguiente e ir a {\color{blue}P2}.  \pause
        \item[P5.]  Intercambiando las filas, ponemos los $1$ principal de forma escalonada y las filas nulas al final.  
    \end{enumerate}
\end{block}\qed
\end{frame}



\begin{frame}{}
    
    \begin{observacion}
        La demostración del teorema anterior nos da un algoritmo para encontrar MERF.
    \end{observacion}
    \vskip 5cm

    
\end{frame}


\begin{frame}{Método de Gauss}

Sea $AX=Y$  el sistema de ecuaciones lineales. El \textit{método de Gauss} consiste en llevar a cabo los siguientes pasos:
\vskip .2cm\pause
    \begin{itemize}
        \item Escribir $[A|Y]$ la matriz extendida del sistema \pause
        \item Reducimos la matriz  $[A|Y]$ $\leadsto$ $[R|Z]$  de tal forma   que $R$  sea  MERF (para ello utilizamos el  algoritmo P1$\cdots$P5).\pause
        \item El sistema $RX=Z$ tiene soluciones fáciles de encontrar.
    \end{itemize}

\vskip .4cm
\pause
Claramente la parte más importante del  método de Gauss es utilizar el algoritmo P1$\cdots$P5,  que llamaremos el  \textit{algoritmo de Gauss-Jordan} o  \textit{eliminación de Gauss-Jordan}.
\vskip .4cm\vskip .4cm\vskip .4cm
\end{frame}

\begin{frame}{Algoritmo de Gauss-Jordan}
    
    Sea $A$  una matriz $m \times n$. Repitamos el algoritmo:
    
    \vskip .2cm
        \begin{enumerate}
        \item[P1.]  Nos ubicamos en la primera fila.
        \item[P2.]  Si la fila es 0 y no es la última, pasar a la fila siguiente y de nuevo {\color{blue}P2}.
        \item[P3.]  Si la fila no es 0, 
        \begin{enumerate}
            \item[P3.1] si el primera entrada no nula está en  la columna $k$ y su valor es $c$, dividir la fila por $c$ (ahora la primera entrada no nula vale 1),
            \item[P3.2] con operaciones elementales del tipo $F_r+ tF_s$ hacer 0  todas las entradas en la columna $k$ (menos la de la columna actual).    
        \end{enumerate}
        \item[P4.] Si la fila no es la última, pasar a la fila siguiente e ir a {\color{blue}P2}.  
        \item[P5.]  Intercambiando las filas, ponemos los $1$ principal de forma escalonada y las filas nulas al final.  
    \end{enumerate}

\vskip .2cm    
La matriz que se obtiene es una MERF.        
    
    
\end{frame}






\begin{frame}
    \begin{ejemplo}
    Aplicaremos el algoritmo de Gauss-Jordan a la siguiente matriz:
    \begin{equation*}
        A= \left[
        \begin{array}{rrrr}
        1 & 0 & 2 & 1 \\
        1& -3& 3 & 2 \\
        2& -3& 5 & 3
        \end{array}
        \right]
    \end{equation*}
    
\pause
    
    Como la primera fila es no nulo, pasamos a P3:
    \begin{equation*}
        A\quad \stackrel{F_2 - F_1}{\longrightarrow} \quad
        \left[
        \begin{array}{rrrr}
        1 & 0 & 2 & 1 \\
        0& -3& 1 & 1 \\
        2& -3& 5 & 3
        \end{array}
        \right]
        \quad \stackrel{F_3 - 2F_1}{\longrightarrow} \quad
        \left[
        \begin{array}{rrrr}
        1 & 0 & 2 & 1 \\
        0& -3& 1 & 1 \\
        0& -3& 1 & 1
        \end{array}
        \right]
    \end{equation*}
    
        Por P4 pasamos a trabajar con la fila 2 y pasamos a P2. Como la fila no es nula hacemos P3.1
    
    \begin{equation*}
     \stackrel{-\frac{F_3}{3}}{\longrightarrow} \quad\left[
    \begin{array}{rrrr}
    1 & 0 & 2 & 1 \\
    0& 1& -\frac{1}{3} & -\frac{1}{3} \\
    0& -3& 1 & 1
    \end{array}
    \right]
    \end{equation*}
    


\end{ejemplo}
\end{frame}


 \begin{frame}
     Ahora P3.2:
     \begin{equation*}
         \stackrel{F_3 +3 F_2}{\longrightarrow} \quad
         \left[
         \begin{array}{rrrr}
         1 & 0 & 2 & 1 \\
         0& 1& -\frac{1}{3} & -\frac{1}{3} \\
         0& 0& 0 & 0
         \end{array}
         \right]
     \end{equation*}
     
     \vskip .4cm
     
         
     Esta última matriz es una MERF. En  este caso, no hizo falta usar P5,  es decir permutar filas.
     
     \vskip 2cm
 
 \end{frame}    



\begin{frame}{Ejemplo del método de Gauss}

A continuación explicaremos en $3$ pasos el método de Gauss para resolver el sistema de ecuaciones 
$$AX=Y.$$
\pause
Ejemplificaremos los pasos con el sistema
\begin{align}
\left[
\begin{array}{rrr}
1 & 0 & 2\\
1& -3& 3\\
2& -3& 5
\end{array}
\right]
\left[
\begin{array}{r}
x_1\\
x_2 \\
x_3
\end{array}
\right]
=
\left[
\begin{array}{r}
1 \\
2 \\
3
\end{array}
\right]\tag{E}
\end{align} 

\vskip 1cm

\end{frame}

\begin{frame}{Primer paso: Matriz ampliada}
\begin{itemize}
 \item Armar la matriz ampliada:\pause
 $$
{A'=\left[A|Y\right]},
 $$
 es decir, le agregamos a $A$ una columna igual a $Y$. 
\end{itemize}


\vskip .3cm
\pause
En  nuestro caso, la matriz ampliada del sistema ${(E)}$ es  
$$
\left[
\begin{array}{rrr|r}
1 & 0 & 2 & 1 \\
1& -3& 3 & 2 \\
2& -3& 5 & 3
\end{array}
\right]
$$


Observar que esta matriz es igual a la del ejemplo anterior. 

\end{frame}

\begin{frame}{Segundo paso: reducir la matriz ampliada (Gauss-Jordan)}

\begin{itemize}
 \item Usar operaciones elementales por filas para transformar la matriz ampliada $A'$ en una matriz $B'$ de la forma
 $$
 B'=\left[B|Z\right]
 $$
donde $B$ es una MERF y $Z$ es una columna.
\end{itemize}

\vskip .3cm
\pause
En  el ejemplo: 

{\footnotesize
\begin{align*}
A'=&
\left[
\begin{array}{rrr|r}
1 & 0 & 2 & 1 \\
1& -3& 3 & 2 \\
2& -3& 5 & 3
\end{array}
\right]
\overset{F_2-F_1}{\longrightarrow}
\left[
\begin{array}{rrr|r}
1 & 0 & 2 & 1 \\
0& -3& 1 & 1 \\
2& -3& 5 & 3
\end{array}
\right]
\overset{F_3-2F_1}{\longrightarrow}
\left[
\begin{array}{rrr|r}
1 & 0 & 2 & 1 \\
0& -3& 1 & 1 \\
0& -3& 1 & 1
\end{array}
\right]
\\
&
\\
&\overset{-\frac{1}{3}F_2}{\longrightarrow}
\left[
\begin{array}{rrr|r}
1 & 0 & 2 & 1 \\
0& 1& -\frac{1}{3} & -\frac{1}{3} \\
0& -3& 1 & 1
\end{array}
\right]
\overset{F_3-3F_2}{\longrightarrow}
\left[
\begin{array}{rrr|r}
1 & 0 & 2 & 1 \\
0& 1& -\frac{1}{3} & -\frac{1}{3} \\
0& 0& 0 & 0
\end{array}
\right]=B'
\end{align*}
} 


\end{frame}


\begin{frame}{Tercer paso: despejar y describir el conjunto de soluciones}
\begin{itemize}
 \item Escribir explicitamente el sistema
 $BX=Z$. 
 \item Despejar en cada ecuación la incognita correspondiente al $1$ principal.
 \item Describir el conjunto de soluciones. Hay tres opciones: tener sólo una solución; infinitas, parametrizadas por las incognitas que no corresponden a $1$'s principales; no tener solución.

\end{itemize}
\vskip .8cm \pause 
En el ejemplo:
$$
BX=Z\rightsquigarrow
\left\{
\begin{array}{l}
x_1 +2x_3 = 1 \\
x_2 -\frac{1}{3}x_3 = -\frac{1}{3}
\end{array}
\right.
\rightsquigarrow
\begin{array}{l}
x_1=-2x_3 +1 \\
x_2=\frac{1}{3}x_3-\frac{1}{3}
\end{array}
$$

\end{frame}


\begin{frame}{Tercer paso (continuación)}
    
    Entonces el conjunto de soluciones del sistema $AX=Y$ es
    
    \begin{align*}
    \left\{(-2x_3+1,\,\frac{1}{3}x_3-\frac{1}{3},\,x_3)\mid x_3\in\RR\right\}.
    \end{align*}
    
    Por ej., si $x_3=1$, entonces $(-1,0,1)$ es una solución del sistema.
    
    \vskip 1.8 cm
\end{frame}



\begin{frame}{Ejemplo}


Resolver el sistema de ecuaciones:
\begin{align*}
(E)
\left\{
\begin{array}{l}
x_2-x_3+x_4=1\\
2x_3+x_4=3\\
x_1+x_2-x_4=1\\
x_1+2x_2-x_3=2 
\end{array}
\right.
\end{align*}

\pause
\begin{block}{Solución}
El conjunto de soluciones es

\begin{align*}
\operatorname{Sol}(E)&=\left\{\left(\frac{5}{2}x_4-\frac{3}{2},-\frac{3}{2}x_4+\frac{5}{2},-\frac{1}{2}x_4+\frac{3}{2},x_4\right)\mid x_4\in\RR\right\}
\end{align*}
\end{block}
\end{frame}

\begin{frame}

Reducimos la matriz ampliada siguiendo al pie de la letra el algoritmo:\pause
{\footnotesize
    \begin{align*}
    A'=&
    \left[
    \begin{array}{rrrr|r}
    0& 1 & -1 & 1 & 1 \\
    0&0& 2& 1 & 3 \\
    1&1& 0&-1 & 1 \\
    1&2& -1& 0 & 2
    \end{array}
    \right]
        \underset{F_4-2F_1}{\stackrel{F_3- F_1}{\longrightarrow}}
    \left[
    \begin{array}{rrrr|r}
    0& 1 & -1 & 1 & 1 \\
    0&0& 2& 1 & 3 \\
    1&0& 1&-2 & 0 \\
    1&0& 1& -2 & 0
    \end{array}
    \right]
    \\
    &
    \\
    &    \overset{\frac{1}{2}F_2}{\longrightarrow}
    \left[
    \begin{array}{rrrr|r}
    0& 1 & -1 & 1 & 1 \\
    0&0& 1& {1}/{2} & {3}/{2} \\
    1&0& 1&-2 & 0 \\
    1&0& 1& -2 & 0
    \end{array}
    \right]
        \underset{F_4-F_2}{\underset{F_3-F_2}{\stackrel{F_1+ F_2}{\longrightarrow}}}
    \left[
    \begin{array}{rrrr|r}
    0& 1 & 0 & {3}/{2} &{5}/{2} \\
    0&0& 1& {1}/{2} & {3}/{2} \\
    1&0& 0&-{5}/{2} & -{3}/{2} \\
    1&0& 0& -{5}/{2} & -{3}/{2}
    \end{array}
    \right]    \\
    &    
    \\
    &    \overset{F_4-F_3}{\longrightarrow}
    \left[
    \begin{array}{rrrr|r}
    0& 1 & 0 & {3}/{2} &{5}/{2} \\
    0&0& 1& {1}/{2} & {3}/{2} \\
    1&0& 0&-{5}/{2} & -{3}/{2} \\
    0&0& 0&0 & 0
    \end{array}
    \right] =B'
    \end{align*}
} 
\end{frame}


\begin{frame}
    Si reescribimos las ecuaciones a partir de $B'$, obtenemos\pause
    \begin{align*}
    (E')
    \left\{
    \begin{array}{l}
    x_2+\frac{3}{2}x_4=\frac{5}{2}\\[1.0ex]
    x_3+\frac{1}{2}x_4=\frac{3}{2}\\[1.0ex]
    x_1-\frac{5}{2}x_4=-\frac{3}{2}
    \end{array}
    \right.
    \end{align*}
    
    Si despejamos respecto a $x_4$, obtenemos \pause
    \begin{equation*}
    x_1=\frac{5}{2}x_4-\frac{3}{2}, \qquad 
            x_2=-\frac{3}{2}x_4+\frac{5}{2}, \qquad 
        x_3=-\frac{1}{2}x_4+\frac{3}{2},
    \end{equation*}
    que es la solución del sistema. 
    
\end{frame}



\begin{frame}{Soluciones de los sistemas de ecuaciones lineales}
    
    {¿Cómo saber si el sistema tiene o no tiene solución? ¿Una o infinitas?}
    
    \pause
    Si $BX=Z$. depende de la forma de la MERF $B$ y de $Z$. 
\vskip.2cm
    Asumamos que $B\in\RR^{m\times n}$ y $Z\in\RR^{m\times 1}$ donde
    \begin{align*}
    &\begin{matrix}
    \qquad\qquad&\quad \;\;\, & k_1 &\;\; \, &k_2&\quad\;\; &\;\;\;\, &k_r& &\\
    & & \downarrow & &\downarrow& & &\downarrow&  &
    \end{matrix} \\
    B&=
    \left[
    \begin{matrix}
    0& \cdots & 1 & *&0&* &* &0& * &*\\
    0& \cdots &0 &\cdots & 1& *  & * & 0& * &*\\
    \vdots&  &\vdots &  &\vdots && &\vdots& &\vdots\\
    0& \cdots &0& \cdots & 0&\cdots &\cdots & 1& * &* \\
    0& \cdots &0& \cdots & 0&\cdots &\cdots & 0& \cdots &0 \\
    \vdots&  &\vdots &  &\vdots && &\vdots& &\vdots\\
    0& \cdots &0& \cdots & 0&\cdots &\cdots & 0& \cdots &0
    \end{matrix}
    \right]
    \quad\mbox{y}
    \qquad
    Z=\left[
    \begin{matrix}
    z_1\\
    z_2\\
    \vdots\\
    z_r\\
    z_{r+1}\\
    \vdots\\
    z_m
    \end{matrix}
    \right]
    \end{align*}
    
\end{frame}

\begin{frame}
    
    Si $ k_1,\ldots, k_r$ son las columnas que contienen 1 principales, el sistema $BX=Z$ tiene la siguiente forma
    
    \begin{equation}\label{sist-eq-hom-merf}
        \begin{aligned}
            \left\{
            \begin{matrix}
            &x_{k_1}& + &\sum_{ j \not= k_1,\ldots, k_r} b_{1j}\,x_j&=& &z_1\\
            &x_{k_2}& + &\sum_{j \not= k_1,\ldots, k_r} b_{2j}\,x_j&=& &z_2\\
            & \vdots& &  &\vdots \\
            &x_{k_r}& + &\sum_{j \not= k_1,\ldots, k_r} b_{rj}\,x_j&=& &z_r\\
            &&&0 &=& &z_{r+1}  \\
            & & & \vdots &&&\vdots \\
            &&&0 &=& &z_{m} 
            \end{matrix}
            \right.
        \end{aligned}
    \end{equation}
    
    
 
    
\end{frame}


\begin{frame}
    \begin{block}{Teorema}\label{teorema-gauss-jordan-general}
        Sea $AX=Y$ un sistema de $m$ ecuaciones lineales y $n$ incógnitas con coeficientes en $\K$ y sea $[B|Z]$ la matriz escalón reducida por fila equivalente por fila a $[A|Y]$ cuyo sistema asociado es \eqref{sist-eq-hom-merf}. 

        
        Entonces, el sistema tiene solución si y solo si  $z_{r+1} = \cdots = z_m =0$ y en ese caso las soluciones son: 
        \begin{equation}
            \begin{matrix}\label{sist-eq-hom-merf0}
                x_{k_1} &= z_1-\sum_{j \not= k_1,\ldots, k_r} b_{1j}\,x_j\\
                x_{k_2} &= z_2-\sum_{j \not= k_1,\ldots, k_r} b_{2j}\,x_j\\
                \vdots& \vdots \\
                x_{k_r}  &= z_r-\textstyle\sum_{j \not= k_1,\ldots, k_r} b_{rj}\,x_j
            \end{matrix},
        \end{equation}
        donde las $x_j$ con $j \not= k_1,\ldots, k_r$ son variables libres y pueden tomar cualquier valor en $\K$.
    \end{block}

    \label{teorema-gauss-jordan-general}
\end{frame}


\begin{frame}
    
    \begin{demostracion}
        $(\Rightarrow)$
        
        \textit{El sistema $BX=Z$ tiene solución \;$\Rightarrow$\; $z_{r+1}=z_{r+2}=\cdots z_m=0$.}
        \vskip .4cm
        Si el sistema tiene solución entonces $z_{r+1}=z_{r+2}=\cdots z_m=0$, pues si alguno de estos $z$'s fuera no nulo tendriamos un absurdo.
        \begin{align*}
        \left\{
        \begin{matrix}
        &x_{k_1}& + &\sum_{j \not= k_1,\ldots, k_r} b_{1j}\,x_j&=& &z_1\\
        &x_{k_2}& + &\sum_{j \not= k_1,\ldots, k_r} b_{2j}\,x_j&=& &z_2\\
        & \vdots& &  &\vdots \\
        &x_{k_r}& + &\sum_{j \not= k_1,\ldots, k_r} b_{rj}\,x_j&=& &z_r\\
        & & & \vdots &&&\vdots \\
        &&&\alert{0} &\alert{=}& &\alert{z_{i}\neq0}  \\
        & & & \vdots &&&\vdots
        \end{matrix}
        \right.
        \end{align*}
        
\end{demostracion}
    
\end{frame}

\begin{frame}
$(\Leftarrow)$
        
    \textit{    Si $z_{r+1}=z_{r+2}=\cdots z_m=0$ \;$\Rightarrow$\; el sistema $BX=Z$ tiene solución.    }    
        
        \vskip .4cm
        Si $z_{r+1}=z_{r+2}=\cdots z_m=0$, obtenemos 
        \begin{equation*}
        \left\{
        \begin{matrix}\label{sist-eq-hom-merf0}
        x_{k_1} &= -\sum_{j \not= k_1,\ldots, k_r} b_{1j}\,x_j + z_1\\
        x_{k_2} &= -\sum_{j \not= k_1,\ldots, k_r} b_{2j}\,x_j+ z_2\\
        \vdots& \vdots \\
        x_{k_r}  &= -\textstyle\sum_{j \not= k_1,\ldots, k_r} b_{rj}\,x_j+z_r
        \end{matrix}.
        \right.
        \end{equation*} 
        
        
        Entonces, con cualquier asignación de valores a  los $x_j$ donde $j \not= k_1,\ldots, k_r$ se obtiene  una solución del sistema.
        \qed
\end{frame}


\begin{frame}
    \begin{block}{Teorema}
        Supongamos que el sistema tiene solución y la cantidad de $1$ principales es igual a la cantidad de incognitas. Entonces el sistema $BX=Z$ tiene una única solución, la cual es $X=Z$. 
    \end{block}
    
    \begin{block}{Demostración}
        
        En este caso, el sistema $BX=Z$ tiene la siguiente forma
        \begin{align*}
        \left\{
        \begin{matrix}
        &x_{1}& &=& &z_1\\
        &x_{2}& &=& &z_2\\
        &\vdots& && &\vdots \\
        &x_n& &=& &z_n\\
        &0& &=& &z_{n+1}  \\
        &&  &\vdots& &\vdots \\
        &0& &=& &z_{m} 
        \end{matrix}
        \right.
        \end{align*} 
        y la solución queda determinada explícitamente. \qedsymbol
    \end{block}
\end{frame}

\begin{frame}
    \begin{block}{Teorema}
        Supongamos que el sistema tiene solución y hay más incognitas que $1$ principales. Entonces el sistema $BX=Z$ tiene infinitas soluciones  de la forma
        \begin{align*}
        \begin{matrix}
        x_{k_1}&=z_1-\sum_{j \not= k_1,\ldots, k_r} b_{1j}\,x_j\\
        x_{k_2}&=z_2-\sum_{j \not= k_1,\ldots, k_r} b_{2j}\,x_j\\
        \vdots&\\
        x_{k_r}&=z_r-\sum_{j \not= k_1,\ldots, k_r} b_{rj}\,x_j
        \end{matrix}
        \end{align*}
        y los $x_j$ con $j \not= k_1,\ldots, k_r$ pueden tomar cualquier valor real.
    \end{block}
    
    \begin{block}{Demostración}
        Como existe al menos un $j \not= k_1,\ldots, k_r$, variando los $x_j$ donde $j \not= k_1,\ldots, k_r$ obtenemos infinitas soluciones. \qedsymbol
    \end{block}
\end{frame}

\begin{frame}{Ejemplo}
    \begin{block}{Ejercicio}
        Dado el sistema de ecuaciones
        \begin{equation*}
            \left\{
                \begin{matrix}
                    2x_1 + x_2 - x_3 + 3x_4 &= b_1 \\
                    x_1 - 3x_2 + 2x_3 - x_4 &= b_2 \\
                    4x_1 + 2x_2 + 3x_3 + 2x_4 &= b_3 \\
                    7x_1 +4x_3 + 4x_4 &= b_4
                \end{matrix}
            \right.
        \end{equation*}
        con $b_1$, $b_2$, $b_3$, $b_4$ $\in \mathbb R$.
        \vskip .4cm
        \begin{enumerate}
            \item Determinar para qué valores de $b_1$, $b_2$, $b_3$ y $b_4$ el sistema tiene solución. 
            \item En  caso de tener solución, describir el conjunto de soluciones.
        \end{enumerate}    
    \end{block}
    
\end{frame}


\begin{frame}
    \begin{block}{Solución}
    \end{block}
    Escribimos la matriz ampliado del sistema y aplicamos el método de Gauss-Jordan:
    {\footnotesize
    \begin{align*}
        &\left[
            \begin{array}{cccc|c}
                2 & 1 & -1 & 3 & b_1 \\
                1 & -3 & 2 & -1 & b_2 \\
                4 & 2 & 3 & 2 & b_3 \\
                7 & 0 & 4 & 4 & b_4
            \end{array}
        \right]
        \underset{F_4-7F_2}{\underset{F_3-4F_2}{\stackrel{F_1-2F_2}{\longrightarrow}}}
        \left[
            \begin{array}{cccc|c}
                0 & 7 & -5 & 5 & b_1 - 2b_2 \\
                1 & -3 & 2 & -1 & b_2 \\
                0 & 14 & -5 & 6 & -4b_2 + b_3 \\
                0 & 21 & -10 & 11 &  - 7b_2 + b_4
            \end{array}
        \right]
        \\
        &{}^{}
        \\
        &\underset{F_4-3F_1}{\stackrel{F_3-2F_1}{\longrightarrow}}
        \left[
            \begin{array}{cccc|c}
                0 & 7 & -5 & 5 & b_1 - 2b_2 \\
                1 & -3 & 2 & -1 & b_2 \\
                0 & 0 & 5 & -4 &-2b_1 + b_3 \\
                0 & 0 & 5 & -4 & -3b_1 - b_2 + b_4
            \end{array}
        \right]
    \end{align*}

    }
\end{frame}



\begin{frame}

    {\footnotesize
    \begin{align*}
        &\stackrel{F_4-F_3}{\longrightarrow}
        \left[
            \begin{array}{cccc|c}
                0 & 7 & -5 & 5 & b_1 - 2b_2 \\
                1 & -3 & 2 & -1 & b_2 \\
                0 & 0 & 5 & -4 &-2b_1 + b_3 \\
                0 & 0 & 0 & 0 & -b_1 + b_2 - b_3 + b_4
            \end{array}
        \right]
        \\
        &{}^{}
        \\
        &\stackrel{F_1+F_3}{\longrightarrow}
        \left[
            \begin{array}{cccc|c}
                0 & 7 & 0 & 1 & -b_1 -2 b_2 + b_3 \\
                1 & -3 & 2 & -1 & b_2 \\
                0 & 0 & 5 & -4 &-2b_1 + b_3 \\
                0 & 0 & 0 & 0 & -b_1 - b_2 - b_3 + b_4
            \end{array}
        \right]
        \\
        &{}^{}
        \\
        &\stackrel{F_2/7}{\longrightarrow}
        \left[
            \begin{array}{cccc|c}
                0 & 1 & 0 & 1/7 & -b_1/7 -2 b_2/7 + b_3/7 \\
                1 & -3 & 2 & -1 & b_2 \\
                0 & 0 & 5 & -4 &-2b_1 + b_3 \\
                0 & 0 & 0 & 0 & -b_1 - b_2 - b_3 + b_4
            \end{array}
        \right]
    \end{align*}
    }
\end{frame}

\begin{frame}

    {\footnotesize
    \begin{align*}
        &\stackrel{F_3/5}{\longrightarrow}
        \left[
            \begin{array}{cccc|c}
                0 & 1 & 0 & 1/7 & -b_1/7 -2 b_2/7 + b_3/7 \\
                1 & -3 & 2 & -1 & b_2 \\
                0 & 0 & 1 & -4/5 &-2b_1/5 + b_3/5 \\
                0 & 0 & 0 & 0 & -b_1 - b_2 - b_3 + b_4
            \end{array}
        \right]
        \\
        &{}^{}
        \\
        &\stackrel{F_2+3F_1}{\longrightarrow}
        \left[
            \begin{array}{cccc|c}
                0 & 1 & 0 & 1/7 & -b_1/7 -2 b_2/7 + b_3/7 \\
                1 & 0 & 2 & -4/7 & -3b_1/7 + b_2/7 + 3b_3/7 \\
                0 & 0 & 1 & -4/5 &-2b_1/5 + b_3/5 \\
                0 & 0 & 0 & 0 & -b_1 - b_2 - b_3 + b_4
            \end{array}
        \right]
        \\
        &{}^{}
        \\
        &\stackrel{F_2-2F_3}{\longrightarrow}
        \left[
            \begin{array}{cccc|c}
                0 & 1 & 0 & 1/7 & -b_1/7 -2 b_2/7 + b_3/7 \\
                1 & 0 & 0 & 36/35 & -13b_1/35 + b_2/7 + b_3/35 \\
                0 & 0 & 1 & -4/5 &-2b_1/5 + b_3/5 \\
                0 & 0 & 0 & 0 & -b_1 - b_2 - b_3 + b_4
            \end{array}
        \right]
    \end{align*}
    }
\end{frame}

\begin{frame}

    El sistema de ecuaciones correspondiente a esta matriz ampliada es

    \begin{align*}
        \left\{
        \begin{matrix}
            &x_1 &+ &36/35\, x_4 &=& -13b_1/35 + b_2/7 + b_3/35 \\
            &x_2 &+ & 1/7\,x_4 &=& b_2/7 + b_3/35 \\
            &x_3 &- &4/5\, x_4 &=& -2b_1/5 + b_3/5 \\
            &&&0 &=& -b_1 - b_2 - b_3 + b_4
        \end{matrix}
        \right.
    \end{align*}

    \vskip .4cm

    Por el teorema de la página \ref{teorema-gauss-jordan-general} el sistema tiene solución si y solo si $-b_1 - b_2 - b_3 + b_4 = 0$ o equivalentemente si 
    $$
    b_4 = b_1 + b_2 + b_3.
    $$

\end{frame}

\begin{frame}
    
    En  el caso de que $b_4 = b_1 + b_2 + b_3$, si denotamos
    \begin{align*}
        c_1 &=  -13b_1/35 + b_2/7 + b_3/35 \\
        c_2 &= -b_1/7 -2 b_2/7 + b_3/7 \\
        c_3 &= -2b_1/5 + b_3/5
    \end{align*}
    entonces la soluciones del sistema son $(x_1,x_2,x_3,x_4)$ tales que
    \begin{align*}
        x_1 &= -36/35x_4 + c_1\\
        x_2 &= -1/7 x_4 + c_2 \\
        x_3 &= 4/5 x_4 + c_3
    \end{align*}
    siendo $x_4$ una variable libre.

    También podemos decir que el conjunto de soluciones es
    $$
    \left\{(-36/35x_4 + c_1, -1/7 x_4 + c_2, 4/5 x_4 + c_3, x_4) \mid x_4 \in \mathbb R\right\}.
    $$

\qed

\end{frame}



\end{document}





















