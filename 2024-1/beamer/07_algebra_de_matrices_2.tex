%\documentclass{beamer} % descomentar para tener pausas
\documentclass[handout]{beamer} % descomentar para no tener pausas
\usetheme{CambridgeUS}
%\setbeamertemplate{background}[grid][step=8 ] % cuadriculado

\usepackage[utf8]{inputenc}%esto permite (en Windows) escribir directamente 
\usepackage{graphicx}
\usepackage{array}
\usepackage{tikz} 
\usetikzlibrary{shapes,arrows,babel,decorations.pathreplacing}
\usepackage{verbatim} 
\usepackage{xcolor} 
\usepackage{amsgen,amsmath,amstext,amsbsy,amsopn,amsfonts,amssymb}
\usepackage{amsthm}
\usepackage{tikz}
\usepackage{tkz-graph}
\usepackage{mathtools}
\usepackage[customcolors]{hf-tikz}


%\setbeamertemplate{background}[grid][step=8 ]
\setbeamertemplate{itemize item}{$\circ$}
\setbeamertemplate{enumerate items}[default]

\definecolor{links}{HTML}{2A1B81}
\hypersetup{colorlinks,linkcolor=,urlcolor=links}

\newcommand{\img}{\operatorname{Im}}
\newcommand{\nuc}{\operatorname{Nu}}
\renewcommand\nu{\operatorname{Nu}}
\newcommand{\la}{\langle}
\newcommand{\ra}{\rangle}
\renewcommand{\t}{{\operatorname{t}}}
\renewcommand{\sin}{{\,\operatorname{sen}}}
\newcommand{\Q}{\mathbb Q}
\newcommand{\R}{\mathbb R}
\newcommand{\C}{\mathbb C}
\newcommand{\K}{\mathbb K}
\newcommand{\F}{\mathbb F}
\newcommand{\Z}{\mathbb Z}

\renewcommand{\figurename }{Figura}


\setbeamercolor{block}{fg=red, bg=red!40!white}
\setbeamercolor{block example}{use=structure,fg=black,bg=white!20!white}

\renewenvironment{block}[1]% environment name
{% begin code
    \par\vskip .2cm%
    {\color{blue}#1}%
    \vskip .2cm
}%
{%
    \vskip .2cm}% end code


\renewenvironment{alertblock}[1]% environment name
{% begin code
    \par\vskip .2cm%
    {\color{red!80!black}#1}%
    \vskip .2cm
}%
{%
    \vskip .2cm}% end code


\renewenvironment{exampleblock}[1]% environment name
{% begin code
    \par\vskip .2cm%
    {\color{blue}#1}%
    \vskip .2cm
}%
{%
    \vskip .2cm}% end code




\newenvironment{exercise}[1]% environment name
{% begin code
    \par\vspace{\baselineskip}\noindent
    \textbf{Ejercicio (#1)}\begin{itshape}%
        \par\vspace{\baselineskip}\noindent\ignorespaces
    }%
    {% end code
    \end{itshape}\ignorespacesafterend
}


\newenvironment{definicion}% environment name
{% begin code
    \par\vskip .2cm%
    {\color{blue}Definición}%
    \vskip .2cm
}%
{%
    \vskip .2cm}% end code

\newenvironment{observacion}% environment name
{% begin code
    \par\vskip .2cm%
    {\color{blue}Observación}%
    \vskip .2cm
}%
{%
    \vskip .2cm}% end code

\newenvironment{ejemplo}% environment name
{% begin code
    \par\vskip .2cm%
    {\color{blue}Ejemplo}%
    \vskip .2cm
}%
{%
    \vskip .2cm}% end code

\newenvironment{ejercicio}% environment name
{% begin code
    \par\vskip .2cm%
    {\color{blue}Ejercicio}%
    \vskip .2cm
}%
{%
    \vskip .2cm}% end code


\renewenvironment{proof}% environment name
{% begin code
    \par\vskip .2cm%
    {\color{blue}Demostración}%
    \vskip .2cm
}%
{%
    \vskip .2cm}% end code



\newenvironment{demostracion}% environment name
{% begin code
    \par\vskip .2cm%
    {\color{blue}Demostración}%
    \vskip .2cm
}%
{%
    \vskip .2cm}% end code

\newenvironment{idea}% environment name
{% begin code
    \par\vskip .2cm%
    {\color{blue}Idea de la demostración}%
    \vskip .2cm
}%
{%
    \vskip .2cm}% end code

\newenvironment{solucion}% environment name
{% begin code
    \par\vskip .2cm%
    {\color{blue}Solución}%
    \vskip .2cm
}%
{%
    \vskip .2cm}% end code



\newenvironment{lema}% environment name
{% begin code
    \par\vskip .2cm%
    {\color{blue}Lema}\begin{itshape}%
        \par\vskip .2cm
    }%
    {% end code
    \end{itshape}\vskip .2cm\ignorespacesafterend
}

\newenvironment{proposicion}% environment name
{% begin code
    \par\vskip .2cm%
    {\color{blue}Proposición}\begin{itshape}%
        \par\vskip .2cm
    }%
    {% end code
    \end{itshape}\vskip .2cm\ignorespacesafterend
}

\newenvironment{teorema}% environment name
{% begin code
    \par\vskip .2cm%
    {\color{blue}Teorema}\begin{itshape}%
        \par\vskip .2cm
    }%
    {% end code
    \end{itshape}\vskip .2cm\ignorespacesafterend
}


\newenvironment{corolario}% environment name
{% begin code
    \par\vskip .2cm%
    {\color{blue}Corolario}\begin{itshape}%
        \par\vskip .2cm
    }%
    {% end code
    \end{itshape}\vskip .2cm\ignorespacesafterend
}

\newenvironment{propiedad}% environment name
{% begin code
    \par\vskip .2cm%
    {\color{blue}Propiedad}\begin{itshape}%
        \par\vskip .2cm
    }%
    {% end code
    \end{itshape}\vskip .2cm\ignorespacesafterend
}

\newenvironment{conclusion}% environment name
{% begin code
    \par\vskip .2cm%
    {\color{blue}Conclusión}\begin{itshape}%
        \par\vskip .2cm
    }%
    {% end code
    \end{itshape}\vskip .2cm\ignorespacesafterend
}





%%%%%%%%%%%%%%%%%%%%%%%%%%%%%%%%%%%%%%%%%%%%%%%%%%%%%%%

\newcommand{\nc}{\newcommand}


%%%%%%%%%%%%%%%%%%%%%%%%%LETRAS
\nc{\RR}{{\mathbb R}} \nc{\CC}{{\mathbb C}} \nc{\ZZ}{{\mathbb Z}}
\nc{\FF}{{\mathbb F}} \nc{\NN}{{\mathbb N}} \nc{\QQ}{{\mathbb Q}}
\nc{\PP}{{\mathbb P}} \nc{\DD}{{\mathbb D}} \nc{\Sn}{{\mathbb S}}
\nc{\uno}{\mathbb{1}} \nc{\BB}{{\mathbb B}} \nc{\An}{{\mathbb A}}

\nc{\ba}{\mathbf{a}} \nc{\bb}{\mathbf{b}} \nc{\bt}{\mathbf{t}}
\nc{\bB}{\mathbf{B}}

\nc{\cP}{\mathcal{P}} \nc{\cU}{\mathcal{U}} \nc{\cX}{\mathcal{X}}
\nc{\cE}{\mathcal{E}} \nc{\cS}{\mathcal{S}} \nc{\cA}{\mathcal{A}}
\nc{\cC}{\mathcal{C}} \nc{\cO}{\mathcal{O}} \nc{\cQ}{\mathcal{Q}}
\nc{\cB}{\mathcal{B}} \nc{\cJ}{\mathcal{J}} \nc{\cI}{\mathcal{I}}
\nc{\cM}{\mathcal{M}} \nc{\cK}{\mathcal{K}}

\nc{\fD}{\mathfrak{D}} \nc{\fI}{\mathfrak{I}} \nc{\fJ}{\mathfrak{J}}
\nc{\fS}{\mathfrak{S}} \nc{\gA}{\mathfrak{A}}
%%%%%%%%%%%%%%%%%%%%%%%%%LETRAS




%%%%%%%%%%%%%%%%%yetter drinfield
\newcommand{\ydg}{{}_{\ku G}^{\ku G}\mathcal{YD}}
\newcommand{\ydgdual}{{}_{\ku^G}^{\ku^G}\mathcal{YD}}
\newcommand{\ydf}{{}_{\ku F}^{\ku F}\mathcal{YD}}
\newcommand{\ydgx}{{}_{\ku \Gx}^{\ku \Gx}\mathcal{YD}}

\newcommand{\ydgxy}{{}_{\ku \Gy}^{\ku \Gx}\mathcal{YD}}

\newcommand{\ydixq}{{}_{\ku \ixq}^{\ku \ixq}\mathcal{YD}}

\newcommand{\ydl}{{}^H_H\mathcal{YD}}
\newcommand{\ydll}{{}_{K}^{K}\mathcal{YD}}
\newcommand{\ydh}{{}^H_H\mathcal{YD}}
\newcommand{\ydhdual}{{}^{H^*}_{H^*}\mathcal{YD}}


\newcommand{\ydha}{{}^H_A\mathcal{YD}}
\newcommand{\ydhhaa}{{}^{\Hx}_{\Ay}\mathcal{YD}}


\newcommand{\wydh}{\widehat{{}^H_H\mathcal{YD}}}
\newcommand{\ydvh}{{}^{\ac(V)}_{\ac(V)}\mathcal{YD}}
\newcommand{\ydrh}{{}^{R\# H}_{R\# H}\mathcal{YD}}
\newcommand{\ydho}{{}^{H^{\dop}}_{H^{\dop}}\mathcal{YD}}
\newcommand{\ydhsw}{{}^{H^{\sw}}_{H^{\sw}}\mathcal{YD}}

\newcommand{\ydhlf}{{}^H_H\mathcal{YD}_{\text{loc fin}}}
\newcommand{\ydholf}{{}^{H^{\dop}}_{H^{\dop}}\mathcal{YD}_{\text{loc fin}}}

\nc{\yd}{\mathcal{YD}}

\newcommand{\ydsn}{{}^{\ku{\Sn_n}}_{\ku{\Sn_n}}\mathcal{YD}}
\newcommand{\ydsnd}{{}^{\ku^{\Sn_n}}_{\ku^{\Sn_n}}\mathcal{YD}}
\nc{\ydSn}[1]{{}^{\Sn_{#1}}_{\Sn_{#1}}\yd}
\nc{\ydSndual}[1]{{}^{\ku^{\Sn_{#1}}}_{\ku^{\Sn_{#1}}}\yd}
\newcommand{\ydstres}{{}^{\ku^{\Sn_3}}_{\ku^{\Sn_3}}\mathcal{YD}}
%%%%%%%%%%%%%%%%%yetter drinfield


%%%%%%%%%%%%%%%%%%%%%%%%%%%%%Operatorename
\newcommand\Irr{\operatorname{Irr}}
\newcommand\id{\operatorname{id}}
\newcommand\ad{\operatorname{ad}}
\newcommand\Ad{\operatorname{Ad}}
\newcommand\Ext{\operatorname{Ext}}
\newcommand\tr{\operatorname{tr}}
\newcommand\gr{\operatorname{gr}}
\newcommand\grdual{\operatorname{gr-dual}}
\newcommand\Gr{\operatorname{Gr}}
\newcommand\co{\operatorname{co}}
\newcommand\car{\operatorname{car}}
\newcommand\rk{\operatorname{rg}}
\newcommand\ord{\operatorname{ord}}
\newcommand\cop{\operatorname{cop}}
\newcommand\End{\operatorname{End}}
\newcommand\Hom{\operatorname{Hom}}
\newcommand\Alg{\operatorname{Alg}}
\newcommand\Aut{\operatorname{Aut}}
\newcommand\Int{\operatorname{Int}}
\newcommand\Id{\operatorname{Id}}
\newcommand\qAut{\operatorname{qAut}}
\newcommand\Map{\operatorname{Map}}
\newcommand\Jac{\operatorname{Jac}}
\newcommand\Rad{\operatorname{Rad}}
\newcommand\Rep{\operatorname{Rep}}
\newcommand\Ker{\operatorname{Ker}}
\newcommand\Img{\operatorname{Im}}
\newcommand\Ind{\operatorname{Ind}}
\newcommand\Comod{\operatorname{Comod}}
\newcommand\Reg{\operatorname{Reg}}
\newcommand\Pic{\operatorname{Pic}}
\newcommand\textitb{\operatorname{Emb}}
\newcommand\op{\operatorname{op}}
\newcommand\Perm{\operatorname{Perm}}
\newcommand\Res{\operatorname{Res}}
\newcommand\res{\operatorname{res}}
\newcommand{\sop}{\operatorname{Supp}}
\newcommand\Cent{\operatorname{Cent}}
\newcommand\sgn{\operatorname{sgn}}
\nc{\GL}{\operatorname{GL}}
%%%%%%%%%%%%%%%%%%%%%%%%%%%%%Operatorename

%%%%%%%%%%%%%%%%%%%%%%%%%%%%%Usuales de hopf
\nc{\D}{\Delta} 
\nc{\e}{\varepsilon}
\nc{\adl}{\ad_\ell}
\nc{\ot}{\otimes}
\nc{\Ho}{H_0} 
\nc{\GH}{G(H)} 
\nc{\coM}{\mathcal{M}^\ast(2,k)} 
\nc{\PH}{\cP(H)}
\nc{\Ftwist}{\overset{\curvearrowright}F}
\nc{\rep}{{\mathcal Rep}(H)}
\newcommand{\deltad}{_*\delta}
\newcommand{\B}{\mathfrak{B}}
\newcommand{\wB}{\widehat{\mathfrak{B}}}
\newcommand{\Cg}[1]{C_{G}(#1)}
\nc{\hmh}{{}_H\hspace{-1pt}{\mathcal M}_H}
\nc{\hm}{{}_H\hspace{-1pt}{\mathcal M}}
\renewcommand{\_}[1]{_{\left[ #1 \right]}}
\renewcommand{\^}[1]{^{\left[ #1 \right]}}
%%%%%%%%%%%%%%%%%%%%%%%%%%%%%Usuales de hopf

%%%%%%%%%%%%%%%%%%%%%%%%%%%%%%%%Usuales
\nc{\im}{\mathtt{i}}
\renewcommand{\Re}{{\rm Re}}
\renewcommand{\Im}{{\rm Im}}
\nc{\Tr}{\mathrm{Tr}} 
\nc{\cark}{char\,k} 
\nc{\ku}{\Bbbk} 
\newcommand{\fd}{finite dimensional}
%%%%%%%%%%%%%%%%%%%%%%%%%%%%%%%%Usuales

%%%%%%%%%%%%%%%%%%%%%%%%Especiales para liftings de duales de Sn
\newcommand{\xij}[1]{x_{(#1)}}
\newcommand{\yij}[1]{y_{(#1)}}
\newcommand{\Xij}[1]{X_{(#1)}}
\newcommand{\dij}[1]{\delta_{#1}}
\newcommand{\aij}[1]{a_{(#1)}}
\newcommand{\hij}[1]{h_{(#1)}}
\newcommand{\gij}[1]{g_{(#1)}}
\newcommand{\eij}[1]{e_{(#1)}}
\newcommand{\fij}[1]{f_{#1}}
\newcommand{\tij}[1]{t_{(#1)}}
\newcommand{\Tij}[1]{T_{(#1)}}
\newcommand{\mij}[1]{m_{(#1)}}
\newcommand{\Lij}[1]{L_{(#1)}}
\newcommand{\mdos}[2]{m_{(#1)(#2)}}
\newcommand{\mtres}[3]{m_{(#1)(#2)(#3)}}
\newcommand{\mcuatro}{m_{\textsf{top}}}
\newcommand{\trid}{\triangleright}
\newcommand{\link}{\sim_{\ba}}
%%%%%%%%%%%%%%%%%%%%%%%%Especiales para liftings de duales de Sn


%%%%%%%%%%%beaamer%%%%%%%%%%%%%%%%%
% Header: Secciones una arriba de la otra.
% \usetheme{Copenhagen}
% \usetheme{Warsaw}

% Header: Secciones una al lado de la otra. Feo footer.
% Los circulitos del header son medio chotos tambien.
% \usetheme[compress]{Ilmenau}
% Muy bueno: difuminado, sin footer:
%\usetheme{Frankfurt}
% agrego footer como el de Copenhagen.
%\useoutertheme[footline=authortitle,subsection=false]{miniframes}




\title[Clase 07 - Álgebra de matrices 2]{Álgebra/Álgebra II \\ Clase 07 - Álgebra de matrices 2}

\author[]{}
\institute[]{\normalsize FAMAF / UNC
    \\[\baselineskip] ${}^{}$
    \\[\baselineskip]
}
\date[09/04/2024]{09 de abril de 2024}

%\titlegraphic{\includegraphics[width=0.2\textwidth]{logo_gimp100.pdf}}

% Converted to PDF using ImageMagick:
% # convert logo_gimp100.png logo_gimp100.pdf


\begin{document}

\begin{frame}
\maketitle
\end{frame}



\begin{frame}{Resumen}
    En esta clase veremos

    \vskip .4cm\pause

    \begin{itemize}
        \item La relación entre sistemas de ecuaciones y  multiplicación de matrices.\pause
        \item Las matrices elementales, que servirán para recuperar como productos las operaciones elementales de fila de las matrices.\pause
        \item La definición de inversa de una matriz (cuando existe) y  algunas propiedades de la inversa de una matriz.

    \end{itemize}

    \vskip 2cm
\end{frame}


\begin{frame}{Sistemas de ecuaciones y multiplicación de matrices}
        Sistema de $m$  ecuaciones lineales con $n$ incógnitas: 
        \vskip .2cm
            \begin{equation*}
        \begin{matrix}
        a_{11}x_1& + &a_{12}x_2& + &\cdots& + &a_{1n}x_n &= &y_1\\
        \vdots&  &\vdots& &&  &\vdots \\
        a_{m1}x_1& + &a_{m2}x_2& + &\cdots& + &a_{mn}x_n &=&y_m
        \end{matrix}\tag{*}
        \end{equation*}
    
    \vskip .4cm
        donde $y_1, \ldots,y_m$ y $a_{i,j}$ ($1 \le i \le m$, $1 \le j \le n$) son números en $\K$. 
 
\end{frame}

\begin{frame}
    Si denotamos
    \begin{equation*}
    A = \begin{bmatrix}
    a_{11}& a_{12}& \cdots &a_{1n} \\
    \vdots&\vdots  &  &\vdots \\
    a_{m1} &a_{m2}&\cdots &a_{mn}\end{bmatrix},\qquad
    X = \begin{bmatrix}
    x_1 \\ \vdots \\ x_n 
    \end{bmatrix},
    \qquad 
    Y = \begin{bmatrix}
    y_1 \\ \vdots \\ y_n
    \end{bmatrix},
    \end{equation*}\pause
    y hacemos el producto de matrices
    \begin{equation*}
    A\cdot X = \begin{bmatrix}
    a_{11}& a_{12}& \cdots &a_{1n} \\
    \vdots&\vdots  &  &\vdots \\
    a_{m1} &a_{m2}&\cdots &a_{mn}\end{bmatrix} \begin{bmatrix} 
    x_1 \\ \vdots \\ x_n 
    \end{bmatrix}
    \end{equation*}\pause
    
    obtenemos 
    
 
\end{frame}

\begin{frame}
    \begin{equation*}
    \begin{bmatrix}
    a_{11}x_1+ a_{12}x_2+ \cdots +a_{1n}x_n \\
    \vdots \\
    a_{m1}x_1 +a_{m2}x_2+\cdots +a_{mn}x_n\end{bmatrix} =
    \begin{bmatrix}
    y_1 \\ \vdots \\ y_m
    \end{bmatrix}.
    \end{equation*}
    \vskip .4cm\pause
    Como dos matrices son iguales si y sólo si sus coeficientes son iguales, la igualdad anterior significa que recuperamos el sistema de ecuaciones original. 
    
        \vskip .4cm
        \pause
    Esto nos dice que la notación matricial antes utilizada para expresar un sistema de ecuaciones
    $$
    AX = Y
    $$\pause
    es consistente con el, ahora definido, producto de matrices
    $$
    A \cdot X = Y.
    $$
\end{frame}

\begin{frame}
    

    \begin{ejemplo}
        Sea el sistema de ecuaciones
    
        $$
        \left\{
        \begin{array}{r}
        3x+6y+3z+15w=-3\\
        x+2y+3w=1\\
        2x+4y+z+8w=0
       \end{array}
       \right.
       $$
      
     La matriz del sistema es 

     \begin{equation*} A= 
        \begin{bmatrix}
            3&6&3&15\\
            1&2&0&3\\
            2&4&1&8
           \end{bmatrix}.     
     \end{equation*}
     
     
    \end{ejemplo}

\end{frame}

\begin{frame}
    Si multiplicamos $A$ por el vector $X$ de la incógnitas obtenemos:
     \begin{equation*}
        \begin{bmatrix}
            3&6&3&15\\
            1&2&0&3\\
            2&4&1&8
           \end{bmatrix}
        \begin{bmatrix}
            x\\y\\z\\w
        \end{bmatrix}
        =     
        \begin{bmatrix*}
            3x+6y+3z+15w\\
            x+2y+3w\\
            2x+4y+z+8w
        \end{bmatrix*}.     
     \end{equation*}
     Si igualamos $AX$ al vector columna correspondiente a las constantes, obtenemos 
     \begin{equation*}AX = 
        \begin{bmatrix}
            3&6&3&15\\
            1&2&0&3\\
            2&4&1&8
           \end{bmatrix}
        \begin{bmatrix}
            x\\y\\z\\w
        \end{bmatrix}
        =     
        \begin{bmatrix*}
            3x+6y+3z+15w\\
            x+2y+3w\\
            2x+4y+z+8w
        \end{bmatrix*}
        =
        \begin{bmatrix*}
            3\\
            1\\
            0
        \end{bmatrix*}
        .     
     \end{equation*}
     Como dos matrices son iguales si y solo si sus coeficientes son iguales,  en la primera fila recuperamos la primera ecuación, en la segunda, la segunda,  etc.
\end{frame}


\begin{frame}
    \begin{block}{Pregunta}
        \begin{itemize}
            \item     ¿Cuál es la relación entre las operaciones elementales que nos llevan $A$ a una MERF y el producto de matrices?
    \end{itemize}    
    \end{block}\pause
        \vskip .2cm
        \begin{block}{Respuesta}
            \begin{itemize}
            \item Veremos que hacer una operación elemental en la matriz $A$ es lo mismo que multiplicar $A$ por una matriz elemental (a definir más abajo), \pause y que
            \item reducir por filas a $A$  se obtiene multiplicando  repetidas veces a $A$ por matrices elementales.
        \end{itemize}
        \end{block}
\end{frame}

    \begin{frame}
        
        \begin{block}{Observación}
            \begin{itemize}
                    \item En el lenguaje matricial, al vector $v = (t_1,t_2,\ldots,t_n)\in\R^n$, lo representamos como la matriz columna
                     \begin{equation*}
                         \begin{bmatrix}
                             t_1\\t_2\\\vdots\\ t_n
                         \end{bmatrix}.
                     \end{equation*}
                         \vskip .4cm\pause
            \item  Para saber si $v$,  es solución del sistema $AX=Y$ debemos verificar que vale la igualdad $Av=Y$. 
            
        \vskip .4cm
        
        
            \item  \qquad $v\in\RR^n\,\mbox{ es solución de }\, AX=Y \Longleftrightarrow     Av=Y$.

            \vskip .4cm    \pause
    
            \item  \qquad $\operatorname{Sol}(\text{*})=\left\{v\in\RR^n\mid Av=Y\right\} $.
    
        
      \end{itemize}    
    
    \end{block}
        % (como lo hacemos en un ecuación de números reales)
    \end{frame}
    
    \begin{frame}
        
        Esta observación es útil como herramienta para demostrar ciertas  propiedades de los sistemas.
        
        \vskip .4cm\pause
        
        \begin{block}{Proposición}
            Sean $v$ y $w$ soluciones del sistema homogéneo $AX=0$. Entonces $v+tw$ también es solución del sistema $AX=0$ para todo $t\in\mathbb{R}$. 
        \end{block}\pause
    
            \begin{demostracion}
                \vskip -.4cm
        \begin{align*}
            A(v+tw)&=Av+A(tw)&
            \qquad&(\mbox{{distributividad}})\\
            &=0+(At)w
            &\qquad&(\mbox{{$AV=0$ + asociatividad}})\\
            &=(tA)w
            &\qquad&(\mbox{{conmutatividad $\times$ escalares}})\\
            &=t(Aw)
            &\qquad&(\mbox{{conmutatividad $\times$ escalares}})\\
            &=t0
            &\qquad&(\mbox{{$Aw=0$}})\\
            &=0.&& 
        \end{align*}\qed
    \end{demostracion}
    
    \end{frame}
    

    
    

\begin{frame}{Matrices elementales}
        \begin{definicion} Una matriz $m \times m$  se dice  \textit{elemental}\index{matriz!elemental} si fue obtenida por medio de una única operación elemental a partir de la matriz identidad $\Id_m$.
    \end{definicion}
    \pause

\begin{ejemplo}



Las  siguientes matrices (coloreadas) son elementales:
\vskip .2cm
\hfsetfillcolor{blue!30}
\hfsetbordercolor{blue!30}

$    \begin{bmatrix}
        1&0&0 \\
        0&1&0 \\
        0&0&1
    \end{bmatrix}
    \stackrel{F_2 + 2F_1}{\longrightarrow}
    \tikzmarkin{a}(0.0,-0.6)(0.0,0.8)
    \begin{bmatrix}
        1&0&0 \\
        2&1&0 \\
        0&0&1
    \end{bmatrix}
    \tikzmarkend{a}$, \qquad
    $\begin{bmatrix}
        1&0&0&0 \\
        0&1&0&0 \\
        0&0&1&0  \\
        0&0&0&1 
    \end{bmatrix}
    \stackrel{F_1 \leftrightarrow F_4}{\longrightarrow}
    \tikzmarkin{b}(0.0,-0.8)(0.0,1)
    \begin{bmatrix}
        0&0&0&1 \\
        0&1&0&0 \\
        0&0&1&0  \\
        1&0&0&0 
    \end{bmatrix}
    \tikzmarkend{b}$, 
    

    $\begin{bmatrix}
            1&0&0 \\
            0&1&0 \\
            0&0&1
        \end{bmatrix}
        \stackrel{3F_2}{\longrightarrow}
        \tikzmarkin{c}(0.0,-0.6)(0.0,0.8)
        \begin{bmatrix}
            1&0&0 \\
            0&3&0 \\
            0&0&1
        \end{bmatrix}
        \tikzmarkend{c}
        $. 

    
\end{ejemplo}

    
\end{frame}

\begin{frame}{Matrices elementales  $2 \times 2$}
    \pause
        \begin{enumerate}
            \item[\color{blue}{E1.}] Si $c \not=0$, multiplicar por  $c$ la primera fila y multiplicar $c$ por la segunda fila son, respectivamente,
            \begin{equation*}
            \begin{bmatrix} c& 0\\ 0&1\end{bmatrix}\;\text{ y }\; \begin{bmatrix} 1& 0\\ 0&c\end{bmatrix},
            \end{equation*}
            \vskip.2cm\pause
            \item[\color{blue}{E2.}] si  $c \in \K$, sumar a la fila 2 la fila 1 multiplicada por $c$ o sumar a la fila 1 la fila 2 multiplicada por $c$ son, respectivamente,
            \begin{equation*}
            \begin{bmatrix} 1& 0\\ c&1\end{bmatrix}\;\text{ y }\; \begin{bmatrix} 1& c\\ 0&1\end{bmatrix}.
            \end{equation*}
            \vskip.2cm\pause
            \item[\color{blue}{E3.}] Finalmente, intercambiando la fila 1 por la fila 2 obtenemos la matriz
            \begin{equation*}
            \begin{bmatrix} 0& 1\\ 1&0\end{bmatrix}.
            \end{equation*}
        \end{enumerate}

\end{frame}


\begin{frame}{Matrices elementales  $m \times m$}
    \begin{enumerate}\pause
        \item[\color{blue}{E1.}] Si $c \not=0$, multiplicar por  $c$ la fila $k$ de la matriz identidad, resulta en la matriz elemental que tiene todos 1's en la diagonal, excepto en la posición $k,k$ donde vale $c$,  es decir si $e(\Id_m) = [a_{ij}]$,  entonces
        \begin{equation*}
        a_{ij} = \left\{ 
        \begin{matrix*}[l]
        1 &\text{si $i=j$ e $i\ne k$,}\\
        c &\text{si $i=j=k$,} \\
        0 \quad&\text{si $i \ne j$.}
        \end{matrix*}\right.
        \end{equation*}\pause
        Gráficamente,
        {\footnotesize
        \begin{align*}
        &\begin{matrix}
        {}^{}&{}^{}&{}^{}&{}^{}&\overset{k}{\downarrow}&{}^{}&{}^{}&{}^{}
        \end{matrix} \\
        \begin{matrix}
        {}^{}\\
        {}^{}\\
        \overset{k}{\to}\\
        {}^{}\\
        {}^{}
        \end{matrix}
        &\begin{bmatrix}
        1 & 0 &  &\cdots & 0  \\
        \vdots  & \ddots  & & & \vdots \\
        0 & \cdots &c &\cdots &0 \\
        \vdots  &   & &\ddots & \vdots \\
        0  & \cdots  & &\cdots & 1
        \end{bmatrix}\tag{E1}
        \end{align*}
    }
    \end{enumerate}
\end{frame}




\begin{frame}
    \begin{enumerate}

        \item[\color{blue}{E2.}]  si  $c \in \K$, sumar a la fila $r$  la fila $s$ multiplicada por $c$.
         
        \begin{equation*}\label{elem-tipo-2}
        a_{ij} = \left\{ 
        \begin{matrix*}[l]
        1 &\text{si $i=j$}\\
        c &\text{si $i=r$, $j=s$,} \\
        0 \quad&\text{otro caso.}
        \end{matrix*}\right.
        \end{equation*}\pause
        Gráficamente, 
            {\footnotesize
                \begin{align*}
        &\begin{matrix}
        {}^{}&{}^{}&{}^{}&{}^{}&\overset{r}{\downarrow}&{}^{}&{}^{}&{}^{}{}^{}\overset{s}{\downarrow}&{}^{}
        \end{matrix} \\
        \begin{matrix}
        {}^{}\\{}^{}\\
        \overset{r}{\to}\\
        {}^{}\\
        {}^{}\\
        {}^{}
        \end{matrix}
        &\begin{bmatrix}
        1 & 0 &  &\cdots &&& 0  \\
        \vdots  & \ddots  & & &&& \vdots \\
        0 & \cdots &1 &\cdots&c&\cdots &0 \\
        \vdots  &   & &\ddots &&& \vdots \\
        \vdots  &   & & &\ddots&& \vdots \\
        0  & \cdots  & &\cdots &&& 1
        \end{bmatrix}\tag{E2}
        \end{align*}
        }
    \end{enumerate}
\end{frame}




\begin{frame}
    \begin{enumerate}
    \item[\color{blue}{E3.}]  Finalmente, intercambiar la fila $r$ por la fila $s$ resulta ser
        \begin{equation*}\label{elem-tipo-3}
        a_{ij} = \left\{ 
        \begin{matrix*}[l]
        1 &\text{si ($i=j$, $i \ne r$, $i \ne s$) o ($i=r$, $j=s$) o ($i=s$, $j=r$) }\\
        0 \quad&\text{otro caso.}
        \end{matrix*}\right.
        \end{equation*}\pause
        Gráficamente, 
            {\footnotesize
                \begin{align*}
        &\begin{matrix}
        {}^{}&{}^{}&{}^{}&{}^{}&\overset{r}{\downarrow}&{}^{}&{}^{}&{}^{}{}^{}\overset{s}{\downarrow}&{}^{}
        \end{matrix} \\
        \begin{matrix}
        {}^{}\\{}^{}\\
        \overset{r}{\to}\\
        {}^{}\\
        \overset{s}{\to}\\{}^{}\\
        {}^{}
        \end{matrix}
        &\begin{bmatrix}
        1 & \cdots &  &\cdots &&\cdots& 0  \\
        \vdots  & \ddots  & & &&& \vdots \\
        0 & \cdots &0 &\cdots&1&\cdots &0 \\
        \vdots  &   & &\ddots &&& \vdots \\
        0  & \cdots  &1 &\cdots &0& \cdots& 0 \\
        \vdots  &   & & &&\ddots& \vdots \\
        0  & \cdots  & &\cdots &&\cdots& 1
        \end{bmatrix}\tag{E3}
        \end{align*}
        }
    \end{enumerate}
\end{frame}

\begin{frame}

    Si  $e$ es una operación elemental por filas, denotemos, como antes, $e(A)$ la matriz que se obtien a partir de $A$ haciendo  la operación elemental $e$. 
\vskip .4cm
        \begin{teorema}\label{th-mrtx-elem}
        Sea $A$ matriz $n \times n$, sea $e$ una operación elemental por fila , entonces $$e(A) = e(\Id_m)A.$$
    \end{teorema}
    \pause
    \begin{block}{Demostración}
        Hagamos en las siguientes pantallas la prueba para matrices $2 \times 2$. 
        
        \vskip .4cm\pause
        
        La prueba en general es similar.
    \end{block}
\vskip 2cm
\end{frame}

\begin{frame}
        
        \begin{enumerate}
            \item[\color{blue}{E1.}] Si $c \not=0$, y sea $e$ la operación elemental de multiplicar por  $c$ la primera fila. Entonces, $E=e(\Id_2)$  resulta en la matriz elemental
            \begin{equation*}
                E = \begin{bmatrix} c& 0\\ 0&1\end{bmatrix}.
            \end{equation*}\pause
            Ahora bien, 
            \begin{multline*}
                E A= \begin{bmatrix} c& 0\\ 0&1\end{bmatrix}
            \begin{bmatrix} a_{11}&a_{12}\\a_{21}&a_{22}\end{bmatrix} \\
            = 
            \begin{bmatrix} 
            c\,.\,a_{11} + 0 \,.\,a_{21}&c\,.\,a_{12}+0\,.\,a_{22}\\
            0\,.\,a_{11} + 1 \,.\,a_{21}&0\,.\,a_{12}+1\,.\,a_{22}\end{bmatrix} 
            =
            \begin{bmatrix} 
            c\,.\,a_{11}&c\,.\,a_{12}\\
            a_{21}&a_{22}\end{bmatrix} = e(A).
            \end{multline*}\pause
            De forma análoga se demuestra en el caso que la operación elemental sea  multiplicar la segunda fila por $c$.
        \end{enumerate}

        


\end{frame}



\begin{frame}
            \begin{enumerate}
        \item[\color{blue}{E2.}] Sea  $c \in \K$, y  sea $e$ la operación elemental de a la fila 2 le sumarle  la fila 1 multiplicada por $c$. Entonces. $E=e(\Id_2)$ resulta en la matriz elemental:
        \begin{equation*}
            E = \begin{bmatrix} 1& 0\\ c&1\end{bmatrix}.
        \end{equation*}\pause
        Ahora bien,
        \begin{equation*}
            EA= \begin{bmatrix} 1& 0\\ c&1\end{bmatrix}    \begin{bmatrix} a_{11}&a_{12}\\a_{21}&a_{22}\end{bmatrix} = 
        \begin{bmatrix} 
        a_{11} &a_{12}\\
        c\,.\,a_{11} + a_{21}&c\,.\,a_{12}+a_{22}\end{bmatrix} = e(A).
        \end{equation*}
        \vskip .4cm\pause
        La demostración es análoga si la operación elemental es sumar a la fila 1 la fila 2 multiplicada por $c$.
    \end{enumerate}
\end{frame}


\begin{frame}
                \begin{enumerate}

        \item[\color{blue}{E3.}] Finalmente, sea $e$  intercambiar la fila 1 por la fila 2. Entonces,  $E=e(\Id_2)$ es la matriz elemental
        \begin{equation*}
            E = \begin{bmatrix} 0& 1\\ 1&0\end{bmatrix}.
        \end{equation*}\pause
        Ahora bien,
        \begin{equation*}
            EA= \begin{bmatrix} 0& 1\\ 1&0\end{bmatrix}    \begin{bmatrix} a_{11}&a_{12}\\a_{21}&a_{22}\end{bmatrix} = 
        \begin{bmatrix}
        a_{21} &a_{22}\\
        a_{11} &a_{12}\end{bmatrix} = e(A).
        \end{equation*}
    \end{enumerate}\qed
\end{frame}


\begin{frame}
        \begin{corolario}
        Sean $A$ y $B$ matrices $m \times n$. Entonces 
        \begin{itemize}
            \item     $B$ equivalente por filas a $A$ si y sólo si $B=PA$ donde $P$ es  producto de matrices elementales. 
        \end{itemize}
    
        
        Más aún, 
        \begin{itemize}
            \item si  $e_1,e_2,\ldots,e_k$ son operaciones elementales de fila y $E_i=e_i(\Id)$,  entonces
            \begin{equation*}
                B = e_k(e_{k-1}(\cdots(e_1(A))\cdots)) \Rightarrow B =  E_kE_{k-1}\cdots E_1A.
            \end{equation*}
        \end{itemize}
        
    \end{corolario}
\vskip 2.5cm
\end{frame}

\begin{frame}
        \begin{block}{Demostración     ($\Rightarrow$)} \pause
        
     Si $B$ equivalente por filas a $A$  existen operaciones elementales $e_1,\ldots,e_k$ tal que 
     \begin{equation*}
     B = e_k(e_{k-1}(\cdots(e_1(A))\cdots)). 
     \end{equation*}
     \pause
Más formalmente, si 
\begin{itemize}
    \item $A_1 = e_1(A)$, 
    \item $A_i = e_i(A_{i-1})$  para $i=2,\ldots,k$, y
    \item $e_k(A_{k-1})= B$.
\end{itemize}
\pause Entonces,  si $E_i = e_i(\Id_m)$, por el teorema anterior
 \begin{itemize}
     \item $A_1 = E_1A$, 
     \item $A_i = E_iA_{i-1}$  para $i=2,\ldots,k$, y
     \item $E_kA_{k-1}= B$.
 \end{itemize}
 \pause
En otras palabras 
\begin{equation*}
    B = PA \quad \text{ con }\quad P =  E_kE_{k-1}\cdots E_1.
\end{equation*}
        

    \end{block} 


\end{frame}

\begin{frame}
    \begin{block}{Demostración ($\Leftarrow$)}\pause
                 Si $B= PA$, con  $P = E_kE_{k-1}\cdots E_1$ donde $E_i=e_i(\Id_m)$ es una matriz elemental,  entonces (razonamiento similar al anterior)
        \begin{equation*}
                B = PA = E_kE_{k-1}\cdots E_1A 
        \end{equation*}\pause
    Más formalmente, si
    \begin{itemize}
        \item $A_1 = E_1A$, 
        \item $A_i = E_iA_{i-1}$  para $i=2,\ldots,k$, y
        \item $E_kA_{k-1}= B$.
    \end{itemize}\pause
Luego, por el teorema anterior, 
\begin{itemize}
    \item $A_1 = e_1(A)$, 
    \item $A_i = e_i(A_{i-1})$  para $i=2,\ldots,k$, y
    \item $e_k(A_{k-1})= B$,
\end{itemize}
\pause
        Por lo tanto, $B$ es equivalente por filas a $A$. \qed
    \end{block}
\end{frame}

\begin{frame}{Matrices invertibles}
    \begin{definicion} Sea $A$ una  matriz $n \times n$ con coeficientes en $\K$. 
        
        \vskip .2cm
        Una matriz $B \in M_{n\times n}(\K)$  es \textit{inversa  de $A$}\index{matriz!inversa} si $BA=AB=\Id_n$. 
        \vskip .2cm
        En  ese caso,  diremos que  $A$ es \textit{invertible}.\index{matriz!invertible}
    \end{definicion}
    \vskip .4cm\pause
    \begin{ejemplo}
        La matriz $\begin{bmatrix} 2&-1\\0&1\end{bmatrix}$ tiene inversa 
        $\begin{bmatrix} \frac12&\frac12\\0&1\end{bmatrix}$ pues es fácil comprobar que 
        \begin{equation*}
        \begin{bmatrix} 2&-1\\0&1\end{bmatrix}
        \begin{bmatrix} \frac12&\frac12\\0&1\end{bmatrix} =
        \begin{bmatrix} 1&0\\0&1\end{bmatrix}\quad\text{ y } \quad
        \begin{bmatrix} \frac12&\frac12\\0&1\end{bmatrix} 
        \begin{bmatrix} 2&-1\\0&1\end{bmatrix}=
        \begin{bmatrix} 1&0\\0&1\end{bmatrix}.
        \end{equation*}
    \end{ejemplo}

\end{frame}


\begin{frame}{Preguntas (y respuestas)}
    \begin{itemize}
    \item ¿Toda matriz no nula tiene inversa?\pause
    \vskip .2cm
    Respuesta: \textbf{no}, por ejemplo
    \begin{equation*}
        \begin{bmatrix}
        0&1\\
        0&0
        \end{bmatrix}
    \end{equation*}
    \vskip .3cm\pause
    \item Si la matriz tiene una inversa ¿es única?
    \vskip .2cm\pause
        Respuesta: \textbf{sí.} Se  verá en la próxima pantalla. 
    \vskip .3cm\pause
    \item Si $BA=\Id_n$ ¿es cierto que $AB=\Id_n$? Es decir, ¿si $A$ tiene inversa a izquierda,  entonces tiene inversa a derecha y es la misma matriz?
    \vskip .2cm\pause
        Respuesta: \textbf{sí.} Se verá en la clase que viene.
    
    
    
    \end{itemize}
\end{frame}


\begin{frame}
            \begin{proposicion}
        Sea $A  \in M_{n\times n}(\K)$, 
        \begin{enumerate}
            \item    sean $B, C \in M_{n \times n}(\K)$ tales que $BA =\Id_n$ y $AC = \Id_n$, entonces $B=C$;
            \item  si $A$ invertible la inversa es única.
        \end{enumerate}
    \end{proposicion}
    \begin{proof} \pause
        1.
        \begin{equation*}
        B = B\Id_n = B(AC) = (BA)C = \Id_nC = C.
        \end{equation*}
        \pause
        2. 
        
        Sean $B$ y $C$ inversas de $A$, es decir $BA=AB=\Id_n$ y  $CA=AC=\Id_n$. En particular, $BA=\Id_n$ y $AC= \Id_n$, luego, por 1., $B=C$.  \qed
    \end{proof}
    
\end{frame}

\begin{frame}
            \begin{definicion}
        Sea $A\in M_{n\times n}(\K)$ invertible. A la única matriz inversa de $A$ la llamamos \textit{la matriz inversa de $A$} y la denotamos $A^{-1}$.
    \end{definicion}

    \pause
    \begin{ejemplo}
        Sea $A$ la matriz 
        \begin{equation*}
        \begin{bmatrix} 2&1&-2\\ 1&1&-2\\ -1&0&1
        \end{bmatrix}.
        \end{equation*}
        Entonces,  $A$ es invertible y su inversa es
        \begin{equation*}
        A^{-1} = \begin{bmatrix} 1&-1&0\\ 1&0&2\\ 1&-1&1
        \end{bmatrix}.
        \end{equation*}
        Esto se resuelve comprobando que $AA^{-1}=\Id_3$ y $A^{-1}A=\Id_3$.
    \end{ejemplo} 
\end{frame}

\begin{frame}
        \begin{teorema}\label{th-prod-inv-impl-inv}
        Sea $A$ matriz $n \times n$ con coeficientes en $\K$. Entonces
        \begin{itemize}
            \item \label{inv-itm1} si $A$ invertible,  entonces $A^{-1}$  es invertible y su inversa es $A$,  es decir 
            $$(A^{-1})^{-1}=A.$$
        \end{itemize}\pause
    \end{teorema}
    \vskip -.6cm
    \begin{demostracion}\pause
        \vskip -.3cm
        \begin{align*}
            AA^{-1}=\Id_n \quad &\Rightarrow \quad \text{la inversa a izquierda de $A^{-1}$ es $A$,} \\
            A^{-1}A=\Id_n \quad &\Rightarrow \quad \text{la inversa a derecha de $A^{-1}$ es $A$,}
        \end{align*}
        \vskip .3cm\pause
    Concluyendo: $A$  es la inversa de $A^{-1}$.
\end{demostracion} \qed    

    
\end{frame}


\begin{frame}
            \begin{teorema}\label{th-prod-inv-impl-inv}
        Sean $A$ y $B$ matrices $n \times n$ con coeficientes en $\K$. Entonces
        \begin{itemize}

            \item \label{inv-itm2} si $A$ y $B$ son invertibles, entonces $AB$ es invertible y 
            $$(AB)^{-1} = B^{-1}A^{-1}.$$
        \end{itemize}
    \end{teorema}\pause
    \vskip -.6cm
    \begin{proof}\pause
 Simplemente debemos comprobar que $B^{-1}A^{-1}$ es inversa a izquierda y derecha de $AB$:
        \begin{equation*}
        (B^{-1}A^{-1})AB = B^{-1}(A^{-1}A)B = B^{-1}\Id_nB =B^{-1}B = \Id_n.
        \end{equation*}\pause
        Análogamente,
        \begin{equation*}
        AB(B^{-1}A^{-1}) = A(BB^{-1})A^{-1} = A\Id_nA^{-1} =AA^{-1} = \Id_n.
        \end{equation*} 
    \end{proof}\qed

\end{frame}

\begin{frame}
        
    
    \begin{observacion}
        Si $A_1,\ldots,A_k$  son invertibles,  entonces $A_1\ldots A_k$ es invertible y su inversa es  $$(A_1\ldots A_k)^{-1} = A_k^{-1}\ldots A_1^{-1} .$$
        El resultado es una generalización del punto (2) del teorema anterior y su demostración se hace por inducción en $k$ (usando (2)  del teorema anterior). Se deja como ejercicio al lector. 
    \end{observacion}

    \vskip .8cm
    \pause
    \begin{observacion}
        La suma de matrices invertibles no necesariamente es invertible, por ejemplo $A+ (-A)= 0$ que no es invertible. 
    \end{observacion}    
\end{frame}

\end{document}




\begin{frame}

\begin{equation*}
    [A^n]_{jk} = \left\{ 
        \begin{matrix*}[l]
        1 &\text{si $k =j+n$}\\
        0 &\text{si $k \not=j+n$.}
        \end{matrix*}\right. \tag*{(HI)}
\end{equation*}
Luego,
\vskip .3cm
$[A^{n+1}]_{jk} = [A\cdot A^n]_{jk} = \sum_{r} a_{jr} [A^n]_{rk} = (*)$
\vskip .3cm
como $a_{jr} =1$ solo si $r = j+1$ y en las otra posiciones es $0$:
\vskip .3cm
$(*)     =  \stackrel{1}{\overbrace{a_{j,j+1}}} [A^n]_{j+1,k} =   [A^n]_{j+1,k} =(**)$
\vskip .3cm
(aca usar (HI) y seguir...)
\end{frame}




















