\documentclass{beamer} % descomentar para tener pausas
%\documentclass[handout]{beamer} % descomentar para no tener pausas
\usetheme{CambridgeUS}
\setbeamertemplate{background}[grid][step=8 ] % cuadriculado

\usepackage{etex}
\usepackage{t1enc}
\usepackage[spanish,es-nodecimaldot]{babel}
\usepackage{latexsym}
\usepackage[utf8]{inputenc}
\usepackage{verbatim}
\usepackage{multicol}
\usepackage{amsgen,amsmath,amstext,amsbsy,amsopn,amsfonts,amssymb}
\usepackage{amsthm}
\usepackage{calc}         % From LaTeX distribution
\usepackage{graphicx}     % From LaTeX distribution
\usepackage{ifthen}
%\usepackage{makeidx}
\input{random.tex}        % From CTAN/macros/generic
\usepackage{subfigure} 
\usepackage{tikz}
\usepackage[customcolors]{hf-tikz}
\usetikzlibrary{arrows}
\usetikzlibrary{matrix}
\tikzset{
    every picture/.append style={
        execute at begin picture={\deactivatequoting},
        execute at end picture={\activatequoting}
    }
}
\usetikzlibrary{decorations.pathreplacing,angles,quotes}
\usetikzlibrary{shapes.geometric}
\usepackage{mathtools}
\usepackage{stackrel}
%\usepackage{enumerate}
\usepackage{enumitem}
\usepackage{tkz-graph}
\usepackage{polynom}
\polyset{%
    style=B,
    delims={(}{)},
    div=:
}
\renewcommand\labelitemi{$\circ$}
\usepackage{nicematrix} %https://ctan.dcc.uchile.cl/macros/latex/contrib/nicematrix/nicematrix.pdf

%\setbeamertemplate{background}[grid][step=8 ]
\setbeamertemplate{itemize item}{$\circ$}
\setbeamertemplate{enumerate items}[default]
\definecolor{links}{HTML}{2A1B81}
\hypersetup{colorlinks,linkcolor=,urlcolor=links}


\newcommand{\Id}{\operatorname{Id}}
\newcommand{\img}{\operatorname{Im}}
\newcommand{\nuc}{\operatorname{Nu}}
\newcommand{\im}{\operatorname{Im}}
\renewcommand\nu{\operatorname{Nu}}
\newcommand{\la}{\langle}
\newcommand{\ra}{\rangle}
\renewcommand{\t}{{\operatorname{t}}}
\renewcommand{\sin}{{\,\operatorname{sen}}}
\newcommand{\Q}{\mathbb Q}
\newcommand{\R}{\mathbb R}
\newcommand{\C}{\mathbb C}
\newcommand{\K}{\mathbb K}
\newcommand{\F}{\mathbb F}
\newcommand{\Z}{\mathbb Z}
\newcommand{\N}{\mathbb N}
\newcommand\sgn{\operatorname{sgn}}
\renewcommand{\t}{{\operatorname{t}}}
\renewcommand{\figurename }{Figura}

%
% Ver http://joshua.smcvt.edu/latex2e/_005cnewenvironment-_0026-_005crenewenvironment.html
%

\renewenvironment{block}[1]% environment name
{% begin code
	\par\vskip .2cm%
	{\color{blue}#1}%
	\vskip .2cm
}%
{%
	\vskip .2cm}% end code


\renewenvironment{alertblock}[1]% environment name
{% begin code
	\par\vskip .2cm%
	{\color{red!80!black}#1}%
	\vskip .2cm
}%
{%
	\vskip .2cm}% end code


\renewenvironment{exampleblock}[1]% environment name
{% begin code
	\par\vskip .2cm%
	{\color{blue}#1}%
	\vskip .2cm
}%
{%
	\vskip .2cm}% end code




\newenvironment{exercise}[1]% environment name
{% begin code
	\par\vspace{\baselineskip}\noindent
	\textbf{Ejercicio (#1)}\begin{itshape}%
		\par\vspace{\baselineskip}\noindent\ignorespaces
	}%
	{% end code
	\end{itshape}\ignorespacesafterend
}


\newenvironment{definicion}[1][]% environment name
{% begin code
	\par\vskip .2cm%
	{\color{blue}Definición #1}%
	\vskip .2cm
}%
{%
	\vskip .2cm}% end code

\newenvironment{observacion}[1][]% environment name
{% begin code
	\par\vskip .2cm%
	{\color{blue}Observación #1}%
	\vskip .2cm
}%
{%
	\vskip .2cm}% end code

\newenvironment{ejemplo}[1][]% environment name
{% begin code
	\par\vskip .2cm%
	{\color{blue}Ejemplo #1}%
	\vskip .2cm
}%
{%
	\vskip .2cm}% end code

\newenvironment{ejercicio}[1][]% environment name
{% begin code
	\par\vskip .2cm%
	{\color{blue}Ejercicio #1}%
	\vskip .2cm
}%
{%
	\vskip .2cm}% end code


\renewenvironment{proof}% environment name
{% begin code
	\par\vskip .2cm%
	{\color{blue}Demostración}%
	\vskip .2cm
}%
{%
	\vskip .2cm}% end code



\newenvironment{demostracion}% environment name
{% begin code
	\par\vskip .2cm%
	{\color{blue}Demostración}%
	\vskip .2cm
}%
{%
	\vskip .2cm}% end code

\newenvironment{idea}% environment name
{% begin code
	\par\vskip .2cm%
	{\color{blue}Idea de la demostración}%
	\vskip .2cm
}%
{%
	\vskip .2cm}% end code

\newenvironment{solucion}% environment name
{% begin code
	\par\vskip .2cm%
	{\color{blue}Solución}%
	\vskip .2cm
}%
{%
	\vskip .2cm}% end code



\newenvironment{lema}[1][]% environment name
{% begin code
	\par\vskip .2cm%
	{\color{blue}Lema #1}\begin{itshape}%
		\par\vskip .2cm
	}%
	{% end code
	\end{itshape}\vskip .2cm\ignorespacesafterend
}

\newenvironment{proposicion}[1][]% environment name
{% begin code
	\par\vskip .2cm%
	{\color{blue}Proposición #1}\begin{itshape}%
		\par\vskip .2cm
	}%
	{% end code
	\end{itshape}\vskip .2cm\ignorespacesafterend
}

\newenvironment{teorema}[1][]% environment name
{% begin code
	\par\vskip .2cm%
	{\color{blue}Teorema #1}\begin{itshape}%
		\par\vskip .2cm
	}%
	{% end code
	\end{itshape}\vskip .2cm\ignorespacesafterend
}


\newenvironment{corolario}[1][]% environment name
{% begin code
	\par\vskip .2cm%
	{\color{blue}Corolario #1}\begin{itshape}%
		\par\vskip .2cm
	}%
	{% end code
	\end{itshape}\vskip .2cm\ignorespacesafterend
}

\newenvironment{propiedad}% environment name
{% begin code
	\par\vskip .2cm%
	{\color{blue}Propiedad}\begin{itshape}%
		\par\vskip .2cm
	}%
	{% end code
	\end{itshape}\vskip .2cm\ignorespacesafterend
}

\newenvironment{conclusion}% environment name
{% begin code
	\par\vskip .2cm%
	{\color{blue}Conclusión}\begin{itshape}%
		\par\vskip .2cm
	}%
	{% end code
	\end{itshape}\vskip .2cm\ignorespacesafterend
}







\newenvironment{definicion*}% environment name
{% begin code
	\par\vskip .2cm%
	{\color{blue}Definición}%
	\vskip .2cm
}%
{%
	\vskip .2cm}% end code

\newenvironment{observacion*}% environment name
{% begin code
	\par\vskip .2cm%
	{\color{blue}Observación}%
	\vskip .2cm
}%
{%
	\vskip .2cm}% end code


\newenvironment{obs*}% environment name
	{% begin code
		\par\vskip .2cm%
		{\color{blue}Observación}%
		\vskip .2cm
	}%
	{%
		\vskip .2cm}% end code

\newenvironment{ejemplo*}% environment name
{% begin code
	\par\vskip .2cm%
	{\color{blue}Ejemplo}%
	\vskip .2cm
}%
{%
	\vskip .2cm}% end code

\newenvironment{ejercicio*}% environment name
{% begin code
	\par\vskip .2cm%
	{\color{blue}Ejercicio}%
	\vskip .2cm
}%
{%
	\vskip .2cm}% end code

\newenvironment{propiedad*}% environment name
{% begin code
	\par\vskip .2cm%
	{\color{blue}Propiedad}\begin{itshape}%
		\par\vskip .2cm
	}%
	{% end code
	\end{itshape}\vskip .2cm\ignorespacesafterend
}

\newenvironment{conclusion*}% environment name
{% begin code
	\par\vskip .2cm%
	{\color{blue}Conclusión}\begin{itshape}%
		\par\vskip .2cm
	}%
	{% end code
	\end{itshape}\vskip .2cm\ignorespacesafterend
}








\title[Clase 10 - Autovalores y autovectores]{Álgebra/Álgebra II \\ Clase 10 - Autovalores y autovectores}

\author[]{}
\institute[]{\normalsize FAMAF / UNC
    \\[\baselineskip] ${}^{}$
    \\[\baselineskip]
}
\date[25/04/2024]{25 de abril de 2024}



\begin{document}

\begin{frame}
\maketitle
\end{frame}

\begin{frame}{Resumen}
    
    En esta clase definiremos
    \begin{itemize}\pause
     \item autovalor\pause
     \item autovector\pause
     \item polinomio característico
    \end{itemize}\pause
    Y explicaremos como calcularlos.
    
    \
    \pause
   El tema de esta clase  está contenido de la sección 3.6 del apunte de clase ``Álgebra II / Álgebra - Notas del teórico''.
    \end{frame}
    
        
    \begin{frame}

        Sea $A\in\K^{n\times n}$ y $v = (t_1,\ldots,t_n) \in \K^n$. Entonces, podemos ver a $v$ como una matriz columna de $n \times 1$ y multiplicar $A$ por $v$:
        \begin{equation*}
            A v = A \begin{bmatrix} t_1 \\ \vdots \\ t_n \end{bmatrix} = \begin{bmatrix} a_{11}t_1 + \cdots + a_{1n}t_n \\ \vdots \\ a_{n1}t_1 + \cdots + a_{nn}t_n \end{bmatrix}.
        \end{equation*}\pause

        \vskip .4cm

        Mirada de esta forma la multiplicación de matrices es una operación que toma una matriz $n \times n$ y un vector de $\K^n$ y devuelve otro vector de $\K^n$ y  es lo que llamaremos luego  una \textit{transformación lineal} de $\K^n$ en $\K^n$:
        $$
        A(v + \lambda w) = Av + \lambda Aw, \quad \forall v,w \in \K^n, \lambda \in \K,
        $$
        (conmuta con la  suma de vectores  y la  multiplicación por escalares). 

    \end{frame}
        


    \begin{frame}

    

    \begin{exampleblock}{Definición}
    Sea $A\in\K^{n\times n}$. Se dice que {$\lambda\in\K$} es un \textit{autovalor} de $A$ y si existe \textit{$v\in\K^{n}$} no nulo tal que 
    \begin{align*}
    A v=\lambda v .
    \end{align*}
    En ese caso decimos que $v$  es un \textit{autovector} asociado a $\lambda$
    \pause 
    \vskip .4cm
    Estudiar los autovalores y autovectores de una matriz es un problema fundamental en álgebra lineal y tiene aplicaciones en muchas áreas de la matemática y la física.

    \end{exampleblock}\pause
    \vskip .5cm
    \begin{exampleblock}{Ejemplo}
    $1$ es un autovalor de  $\Id_n$ y todo $v\in\K^n$ es un autovector asociado a $1$ pues
    \begin{align*}
    \Id_n v= v 
    \end{align*}
    \end{exampleblock}
    \end{frame}



    
    \begin{frame}
    \begin{observacion}
    El autovalor puede ser $0$ pero el autovector \textit{nunca} puede ser $0$ 
    \end{observacion}
    \vskip .8cm\pause
    
    \begin{exampleblock}{Ejemplo}
    $0$ es un autovalor de  
    $\left[\begin{array}{cc} 0&1\\0&0 \end{array}\right]$
    y $\left[\begin{array}{c}1\\0 \end{array} \right]$ es un autovector asociado a $0$ pues
    \begin{align*}
    \left[\begin{array}{cc}0&1\\0&0\end{array}\right]
    \left[\begin{array}{c}1\\0 \end{array}\right]
    =
    \left[\begin{array}{c}0\\0 \end{array}\right]=
    0\left[\begin{array}{c}1\\0 \end{array}\right]
    \end{align*}
    \end{exampleblock}
    \end{frame}
    

    \begin{frame}

    \begin{block}{Observación}
    La existencia de autovalores dependen del cuerpo donde estamos trabajando. 
    \vskip .4cm\pause
    Por ejemplo sea $A \in \R^{2\times 2}$ 
    $$
    A=\begin{bmatrix} 0&-1\\1&0\end{bmatrix}.
    $$ 
    Entonces,  $A$ no tiene autovalores reales. \pause
    \vskip .4cm
    Veremos que si permitimos autovalores complejos entonces $A$ sí tiene autovalores.
        \vskip .4cm\pause
    Etos resultados se verán en el ejemplo de la página \ref{ejemplo-autovalores-complejos}.

    \end{block}

    \end{frame}
    

    
    
    \begin{frame}
    
    \begin{exampleblock}{Definición}
    
    Dado $i\in\{1, ..., n\}$, se denota \textit{$e_i$} al vector de $\K^n$ cuyas coordenadas son todas ceros excepto la coordenada $i$ que es un $1$.\pause
    \begin{align*}
    e_i=\begin{bmatrix}0\\ \vdots\\1\\ \vdots\\0\end{bmatrix}
    \end{align*}
    El conjunto $\{e_1, ..., e_n\}$ se llama \textit{base canónica} de $\K^n$.
    \end{exampleblock}
    \pause

    \begin{exampleblock}{Ejemplo}
    En $\K^3$ la base canónica es 
    $e_1 = \begin{bmatrix} 1\\0\\0 \end{bmatrix}$, 
    $e_2 = \begin{bmatrix} 0\\1\\0\end{bmatrix}$, 
    $e_3 = \begin{bmatrix} 0\\0\\1\end{bmatrix}$.
    \end{exampleblock}

    \end{frame}
    


    \begin{frame}
    \begin{exampleblock}{Ejemplo: Matriz diagonal}
    Sea $D\in\K^{n\times n}$ una matriz diagonal con entradas $\lambda_1$, $\lambda_2$, ..., $\lambda_n$.
    \vskip .2cm
    Entonces $e_i$ es un autovector con autovalor $\lambda_i$  $\,\forall\, i\in\{1, ..., n\}$
    \end{exampleblock}   \pause
    
    \begin{demostracion}   \pause
        Recordar que la multiplicación $De_i$ se corresponde con multiplicar cada fila de $e_i$ por el elemento correspondiente de la diagonal. 
    \vskip .2cm
        Como las filas (en este caso entradas) de $e_i$ son todas nulas excepto un $1$ en la entrada $i$ queda queda
        $$
        De_i=\begin{bmatrix} 0\\ \vdots\\\lambda_i\\ \vdots\\0 \end{bmatrix}
        =\lambda_i e_i
        $$
        \qed
    \end{demostracion}
  
    \end{frame}
    

    
    \begin{frame}
    
    \begin{block}{Observación}
    
    \end{block}
    \begin{itemize}
        \item Puede haber varios autovectores con el mismo autovalor.    \pause

        \vskip .4cm

        \item Vimos esto en el ejemplo con $\Id$ y en el caso de la diagonal si tiene entradas iguales sucede lo mismo. 
        \pause
        \vskip .4cm

        \item Más aún el conjunto de todos los autovectores con un mismo autovalor es \textit{invariante por la suma y la multiplicación por escalares.}
        \pause
        \vskip .4cm

        \item En particular los múltiplos de un autovector son autovectores con el mismo autovalor.

    \end{itemize}
    
    
    \end{frame}


    \begin{frame}
    
    \begin{block}{Definición}
    Sea $A\in\K^{n\times n}$ y $\lambda\in\K$ un autovalor de $A$. El \textit{autoespacio} asociado a $\lambda$ es
    $$
    V_\lambda=
    \{v\in\K^n\mid A v=\lambda v\}.
    $$
    Es decir, \textit{$V_\lambda$} es el conjunto  formado por todos los autovectores asociados a $\lambda$ y el vector nulo.
    \end{block}
    \pause
    
    \begin{teorema}
    Si $v$ y $w$ pertenecen al autoespacio de $A$ asociado a $\lambda$, entonces $v+tw$ también pertenece a $V_\lambda$.
    \end{teorema}
    \pause
    \begin{demostracion}   \pause
     $$
    A(v+t w)=A v+t A w=\lambda v+t\lambda w=\lambda(v+t w).
    $$
    \qed
    \end{demostracion}
    
    \end{frame}


    
    \begin{frame}
    
    \begin{proposicion}
    Un autovector no puede tener dos autovalores distintos. 
    \vskip .2cm
    \pause
    Por lo tanto, autovectores con autovalores distintos son distintos.
    \end{proposicion}   \pause
    
   \begin{demostracion}   \pause
    Supongamos que $Av=\lambda v$ y $Av=\mu v$. Entonces $\lambda v=\mu v$ y por lo tanto 
    $$
    (\lambda-\mu) v=
    \begin{bmatrix}
    (\lambda-\mu)v_1\\ \vdots\\(\lambda-\mu)v_n
    \end{bmatrix}
    =
    \begin{bmatrix}
    0\\ \vdots\\0
    \end{bmatrix}
    $$
    
    Como $v\neq0$ por ser autovector, alguna de sus coordenadas es no nula. Entonces $\lambda-\mu$ tiene que ser $0$ o dicho de otro modo $\lambda=\mu$. \qed
   \end{demostracion}
    
     
    \end{frame}
    

    \begin{frame}
    
    \begin{exampleblock}{Problema}
    Hallar los autovalores de $A\in\K^{n\times n}$ y para cada autovalor, describir explícitamente o paramétricamente el autoespacio asociado
    \end{exampleblock}\pause
    
    \begin{itemize}
        \item En otras palabras nos preguntamos que $\lambda\in\K$ y  que $v\in\K^{n}$ satisfacen
        $$
        A v=\lambda v
        \Longleftrightarrow
        A v-\lambda v=0
        \Longleftrightarrow
        (A-\lambda\Id)v=0 .
        $$\pause
        \item La última igualdad es un sistema de ecuaciones lineales. Queremos ver entonces si existe un $v\in\K^{n}$ no nulo que sea solución del sistema homogéneo
        \begin{equation*}
            (A-\lambda\Id)X=0.\tag{*}
        \end{equation*}\pause
        \item  Un  sistema $BX =0$ tiene solución no trivial sii $\det(B)=0$. Por lo tanto (*) tiene  solución no trivial si y sólo si 
        $$
        \det(A-\lambda\Id)=0.
        $$ 
    \end{itemize}

    \end{frame}


    \begin{frame}
    
    \begin{conclusion}
    $\lambda\in\K$ es un autovalor de $A$ y $v\in\K^n$ es un autovector asociado a $\lambda$ si y sólo si
    \begin{itemize}
     \item $\det(A-\lambda\Id)=0$
     \item $v$ es solución del sistema homogéneo $(A-\lambda\Id)X=0$
    \end{itemize}
    \end{conclusion}
    \vskip .6cm\pause
    Esta es casi la respuesta a nuestro problema. Para dar una respuesta más operativa introducimos el siguiente polinomio.
    \end{frame}
    
    \begin{frame}
    
    \begin{exampleblock}{Definición}
    
    Sea $A\in\K^{n\times n}$. El \textit{polinomio característico} de $A$ es
    \begin{align*}
    \chi_A(x)=\det(x\Id -A)
    \end{align*}
    \end{exampleblock}\pause
    \vskip .2cm
    \begin{exampleblock}{Ejemplo}
    El polinomio  característico de $\Id_n$ es 
    \begin{align*}
    \chi_{\Id_n}(x)=(x-1)^n 
    \end{align*}
    \end{exampleblock} \vskip -.6cm\pause
    \begin{demostracion}\pause
        $x\Id-\Id=(x -1)\Id$ es una matriz diagonal con $(x-1)$ en todas las entradas de la diagonal. Entonces el determinante es el producto de la diagonal. \qed
    \end{demostracion}

    \end{frame}
    
\begin{frame}
    En general,  si $A = [a_{ij}]$ matriz $n \times n$, tenemos que
        \begin{equation*}
        \chi_A(x) = \det(x\,Id-A) = \det
        \begin{bmatrix}
        x -a_{11}&-a_{12}&\cdots&-a_{1n}\\
        -a_{21}&x-a_{22}&\cdots&-a_{2n}\\
        \vdots&\vdots&\ddots&\vdots\\
        -a_{n1}&-a_{n2}&\cdots&x-a_{nn}\\
        \end{bmatrix}
        \end{equation*} 
        y el polinomio característico de $A$ es un polinomio  de grado $n$, \pause más precisamente  
        $$
        \chi_A(x) = x^n + a_{n-1}x^{n-1}+ \cdots + a_1x + a_0.
        $$ 
        Esto se puede demostrar por inducción. 
        
\end{frame}



    \begin{frame}
        
    \begin{exampleblock}{Ejemplo}
    El polinomio  característico de 
    $A= \begin{bmatrix}  0&1\\0&0  \end{bmatrix}$ es 
    $\chi_{A}(x)=x^2$.
    \end{exampleblock}\vskip -.4cm\pause
    \begin{demostracion}\pause
        $x\Id -A=\begin{bmatrix}  x&-1\\0&x \end{bmatrix}$ es triangular superior. Entonces el determinante es el producto de la diagonal. \qed
    \end{demostracion}
 
    \vskip .2cm
    
    \begin{exampleblock}{Ejemplo}
    Si
    $
    A=
    \begin{bmatrix}  a&b\\c&d  \end{bmatrix}$, entonces $ \chi_{A}(x)=(x-a)(x-d)-bc $.
    \end{exampleblock}
    \vskip -.2cm\pause
    \begin{demostracion}\pause
        $x\Id-A=\begin{bmatrix} x-a&-b\\-c&x-d\end{bmatrix}$ y usamos la fórmula del determinante de una matriz $2\times 2$. \qed
    \end{demostracion}

    \end{frame}
    


    \begin{frame}
    
        
        \begin{proposicion}\label{autovalores}
            Sea $A\in \K^{n \times n}$.  Entonces $\lambda\in \K$ es autovalor si y sólo si $\lambda$ es raíz del polinomio característico.  
        \end{proposicion}\pause
        \begin{proof}\pause
            \begin{tabular}{rl}
    $\lambda$ es autovalor &$\Leftrightarrow$ existe $v \ne 0$ tal que $Av = \lambda v$ \\
                &\\
                &$\Leftrightarrow$ $0 = \lambda v-Av  =   \lambda \Id v -Av =  (\lambda \Id-A)v$ \\
                &\\
                &$\Leftrightarrow$ $(\lambda \Id-A)X=0$ tiene solución no trivial \\
                &\\
                &$\Leftrightarrow$ $\chi_A(\lambda) = \det(\lambda \Id-A) =0$ \\
                &\\
                &$\Leftrightarrow$ $\lambda$ es raíz del polinomio característico.  
            \end{tabular}
            
            \qed  
        \end{proof}
    
    \end{frame}
    
\begin{frame}
    \begin{block}{Método para encontrar autovalores y autovectores de $A$}
        \vskip .2cm\pause
        \begin{enumerate}
            \item Calcular $\chi_A(x) =\det(x\Id -A)$,
            \vskip .2cm\pause
            \item Encontrar las raíces $\lambda_1,\ldots,\lambda_k$ de $\chi_A(x)$.
            
            (no siempre se puede. No hay una fórmula o método  general para encontrar las raíces de polinomios de grado 5 o superior).\pause
            \vskip .2cm
            \item Para cada $i$ con $1 \le i \le k$ resolver el sistema de ecuaciones lineales:
            \begin{equation*}
                (\lambda_i\Id-A)X = 0.
            \end{equation*}
            Las soluciones no triviales  de este sistema son los autovectores con autovalor $\lambda_i$.

        \end{enumerate}
    \end{block}
\end{frame}
    
\begin{frame}
    \begin{ejemplo}
        Encontrar autovalores y autovectores de la matriz 
        \begin{equation*}
            A=\begin{bmatrix}
                3&-2 \\ 1&0
            \end{bmatrix}.
        \end{equation*}
    \end{ejemplo}\pause
    \vskip -.4cm
    \begin{solucion}\pause
        \begin{enumerate}
            \item  $\chi_A(x)= \det\left[\begin{matrix}
            x-3& 2 \\ -1 & x
            \end{matrix} \right]= x^2 -3x +2   =(x -1)(x -2)$.\pause
            \item Los autovalores de $A$ son las raíces de  $\chi_A(x)$: $1$ y $2$.\pause
            \item Debemos resolver los sistemas de ecuaciones:
            \begin{equation*}
                (A - \Id)X = 0,\qquad (A - 2\Id)X = 0.
            \end{equation*}
        \end{enumerate}
    \end{solucion}
\end{frame}

\begin{frame}
    Es decir,  debemos resolver los sistemas
    \begin{align*}
        \begin{bmatrix}
            3 - 1&-2 \\ 1&-1
        \end{bmatrix}
        \begin{bmatrix}
            x_1\\x_2
        \end{bmatrix}
        =
        \begin{bmatrix}
            0\\0
        \end{bmatrix}
        \qquad &\Rightarrow \qquad 
        \begin{bmatrix} 2&-2 \\ 1&-1 \end{bmatrix}
        \begin{bmatrix}
            x_1\\x_2
        \end{bmatrix}
        =
        \begin{bmatrix}
            0\\0
        \end{bmatrix} \tag{S1}
        \\
        &
        \\
        \begin{bmatrix}
            3 - 2&-2 \\ 1&-2
        \end{bmatrix}
        \begin{bmatrix}
            x_1\\x_2
        \end{bmatrix}
        =
        \begin{bmatrix}
            0\\0
        \end{bmatrix}
        \qquad &\Rightarrow \qquad 
        \begin{bmatrix}1&-2 \\ 1&-2\end{bmatrix}
        \begin{bmatrix}
            x_1\\x_2
        \end{bmatrix}
        =
        \begin{bmatrix}
            0\\0
        \end{bmatrix} \tag{S2}
    \end{align*}
\vskip .4cm\pause
    (S1) $\begin{bmatrix} 2&-2 \\ 1&-1 \end{bmatrix} \stackrel{F_1 -2F_2}{\longrightarrow} \begin{bmatrix} 0&0 \\ 1&-1 \end{bmatrix}$  $\Rightarrow$ $x_1-x_2=0$ $\Rightarrow$ $(t,t)$ es solución. 
    \vskip .4cm
    (S2)  $\begin{bmatrix}1&-2 \\ 1&-2\end{bmatrix} \stackrel{F_2 -F_1}{\longrightarrow} \begin{bmatrix}1&-2 \\ 0&0\end{bmatrix}$  $\Rightarrow$ $x_1-2x_2=0$ $\Rightarrow$ $(2t,t)$ es solución. 

\end{frame}


\begin{frame}
    \begin{block}{Respuesta final}
        \begin{itemize}
            \item Los autovalores de $A$ son $1$ y $2$.\pause
            \item El auto espacio correspondiente al  autovalor $1$ es
            \begin{equation*}
                V_1 = \{t(1,1): t \in \R\}.
            \end{equation*}\pause
            \item El auto espacio correspondiente al  autovalor $2$ es
            \begin{equation*}
                V_2 = \{t(2,1): t \in \R\}.
            \end{equation*}
        \end{itemize}
\qed\vskip 2cm
        
    \end{block}
\end{frame}


    
    \begin{frame}

    \begin{ejemplo} \label{ejemplo-autovalores-complejos}
    Sea $A= \begin{bmatrix}0&-1\\1&0\end{bmatrix} \in \R^2$. 
    \begin{enumerate}
        \item Encontrar los autovalores y  autovectores \textit{reales} de $A$.  
        \item Encontrar los autovalores y  autovectores \textit{complejos} de $A$.  
    \end{enumerate}
    
    \end{ejemplo}\pause
    \begin{solucion}\pause
        
        1. $x\Id-A= \begin{bmatrix}x&1\\-1&x\end{bmatrix}$, luego
        \begin{align*}
            \chi_A(x)=x^2+1.
            \end{align*}
\vskip .2cm
        El polinomio no tiene raíces reales, por lo tanto no existen autovalores reales (obviamente no hay autovectores).


    \end{solucion}
\end{frame}


    
\begin{frame}
    
    2. En  este caso, el polinomio característico se factoriza:
    \begin{align*}
    \chi_A(x)=x^2+1=(x+i)(x-i),
    \end{align*}
    y este polinomio \textit{sí} tiene raíces (complejas): $i$ y $-i$.\pause
    \vskip .2cm
    En este caso, entonces, $i$ y $-i$ son los autovalores  y es fácil ver que
    \begin{equation*}
        V_i = \left\{ \omega(i,1): \omega\in \C \right\}, \qquad  V_{-i} = \left\{ \omega(-i,1): \omega\in \C \right\}.
    \end{equation*}  
    

    Nunca está de más comprobar los resultados:
    \pause
    \begin{align*}
        \begin{bmatrix}0&-1\\1&0\end{bmatrix}\begin{bmatrix}i\\1\end{bmatrix} &= \begin{bmatrix}-1\\i\end{bmatrix} = i \begin{bmatrix}i\\1\end{bmatrix}.\\
        \\
        \begin{bmatrix}0&-1\\1&0\end{bmatrix}\begin{bmatrix}-i\\1\end{bmatrix} &= \begin{bmatrix}-1\\-i\end{bmatrix} = (-i) \begin{bmatrix}-i\\1\end{bmatrix}.
    \end{align*}

    \qed
\end{frame}


\end{document}

