%\documentclass{beamer} % descomentar para tener pausas
\documentclass[handout]{beamer} % descomentar para no tener pausas
\usetheme{CambridgeUS}
%\setbeamertemplate{background}[grid][step=8 ] % cuadriculado

\usepackage[utf8]{inputenc}%esto permite (en Windows) escribir directamente 
\usepackage{graphicx}
\usepackage{array}
\usepackage{tikz} 
\usetikzlibrary{shapes,arrows,babel,decorations.pathreplacing}
\usepackage{verbatim} 
\usepackage{xcolor} 
\usepackage{amsgen,amsmath,amstext,amsbsy,amsopn,amsfonts,amssymb}
\usepackage{amsthm}
\usepackage{tikz}
\usepackage{tkz-graph}
\usepackage{mathtools}
\usepackage{xcolor}

%\setbeamertemplate{background}[grid][step=8 ]
\setbeamertemplate{itemize item}{$\circ$}
\setbeamertemplate{enumerate items}[default]

\definecolor{links}{HTML}{2A1B81}
\hypersetup{colorlinks,linkcolor=,urlcolor=links}

\newcommand{\img}{\operatorname{Im}}
\newcommand{\nuc}{\operatorname{Nu}}
\renewcommand\nu{\operatorname{Nu}}
\newcommand{\la}{\langle}
\newcommand{\ra}{\rangle}
\renewcommand{\t}{{\operatorname{t}}}
\renewcommand{\sin}{{\,\operatorname{sen}}}
\newcommand{\Q}{\mathbb Q}
\newcommand{\R}{\mathbb R}
\newcommand{\C}{\mathbb C}
\newcommand{\K}{\mathbb K}
\newcommand{\F}{\mathbb F}
\newcommand{\Z}{\mathbb Z}

\renewcommand{\figurename }{Figura}


\setbeamercolor{block}{fg=red, bg=red!40!white}
\setbeamercolor{block example}{use=structure,fg=black,bg=white!20!white}

\renewenvironment{block}[1]% environment name
{% begin code
	\par\vskip .2cm%
	{\color{blue}#1}%
	\vskip .2cm
}%
{%
	\vskip .2cm}% end code


\renewenvironment{alertblock}[1]% environment name
{% begin code
	\par\vskip .2cm%
	{\color{red!80!black}#1}%
	\vskip .2cm
}%
{%
	\vskip .2cm}% end code


\renewenvironment{exampleblock}[1]% environment name
{% begin code
	\par\vskip .2cm%
	{\color{blue}#1}%
	\vskip .2cm
}%
{%
	\vskip .2cm}% end code


\newenvironment{exercise}[1]% environment name
{% begin code
	\par\vspace{\baselineskip}\noindent
	\textbf{Ejercicio (#1)}\begin{itshape}%
		\par\vspace{\baselineskip}\noindent\ignorespaces
	}%
	{% end code
	\end{itshape}\ignorespacesafterend
}


\newenvironment{definicion}% environment name
{% begin code
	\par\vskip .2cm%
	{\color{blue}Definición}%
	\vskip .2cm
}%
{%
	\vskip .2cm}% end code

\newenvironment{observacion}% environment name
{% begin code
	\par\vskip .2cm%
	{\color{blue}Observación}%
	\vskip .2cm
}%
{%
	\vskip .2cm}% end code

\newenvironment{ejemplo}% environment name
{% begin code
	\par\vskip .2cm%
	{\color{blue}Ejemplo}%
	\vskip .2cm
}%
{%
	\vskip .2cm}% end code

\newenvironment{ejercicio}% environment name
{% begin code
	\par\vskip .2cm%
	{\color{blue}Ejercicio}%
	\vskip .2cm
}%
{%
	\vskip .2cm}% end code


\renewenvironment{proof}% environment name
{% begin code
	\par\vskip .2cm%
	{\color{blue}Demostración}%
	\vskip .2cm
}%
{%
	\vskip .2cm}% end code



\newenvironment{demostracion}% environment name
{% begin code
	\par\vskip .2cm%
	{\color{blue}Demostración}%
	\vskip .2cm
}%
{%
	\vskip .2cm}% end code

\newenvironment{idea}% environment name
{% begin code
	\par\vskip .2cm%
	{\color{blue}Idea de la demostración}%
	\vskip .2cm
}%
{%
	\vskip .2cm}% end code

\newenvironment{solucion}% environment name
{% begin code
	\par\vskip .2cm%
	{\color{blue}Solución}%
	\vskip .2cm
}%
{%
	\vskip .2cm}% end code



\newenvironment{lema}% environment name
{% begin code
	\par\vskip .2cm%
	{\color{blue}Lema}\begin{itshape}%
		\par\vskip .2cm
	}%
	{% end code
	\end{itshape}\vskip .2cm\ignorespacesafterend
}

\newenvironment{proposicion}% environment name
{% begin code
	\par\vskip .2cm%
	{\color{blue}Proposición}\begin{itshape}%
		\par\vskip .2cm
	}%
	{% end code
	\end{itshape}\vskip .2cm\ignorespacesafterend
}

\newenvironment{teorema}% environment name
{% begin code
	\par\vskip .2cm%
	{\color{blue}Teorema}\begin{itshape}%
		\par\vskip .2cm
	}%
	{% end code
	\end{itshape}\vskip .2cm\ignorespacesafterend
}


\newenvironment{corolario}% environment name
{% begin code
	\par\vskip .2cm%
	{\color{blue}Corolario}\begin{itshape}%
		\par\vskip .2cm
	}%
	{% end code
	\end{itshape}\vskip .2cm\ignorespacesafterend
}

\newenvironment{propiedad}% environment name
{% begin code
	\par\vskip .2cm%
	{\color{blue}Propiedad}\begin{itshape}%
		\par\vskip .2cm
	}%
	{% end code
	\end{itshape}\vskip .2cm\ignorespacesafterend
}

\newenvironment{conclusion}% environment name
{% begin code
	\par\vskip .2cm%
	{\color{blue}Conclusión}\begin{itshape}%
		\par\vskip .2cm
	}%
	{% end code
	\end{itshape}\vskip .2cm\ignorespacesafterend
}






%%%%%%%%%%%%%%%%%%%%%%%%%%%%%%%%%%%%%%%%%%%%%%%%%%%%%%%







\newcommand{\nc}{\newcommand}


%%%%%%%%%%%%%%%%%%%%%%%%%LETRAS
\nc{\RR}{{\mathbb R}} \nc{\CC}{{\mathbb C}} \nc{\ZZ}{{\mathbb Z}}
\nc{\FF}{{\mathbb F}} \nc{\NN}{{\mathbb N}} \nc{\QQ}{{\mathbb Q}}
\nc{\PP}{{\mathbb P}} \nc{\DD}{{\mathbb D}} \nc{\Sn}{{\mathbb S}}
\nc{\uno}{\mathbb{1}} \nc{\BB}{{\mathbb B}} \nc{\An}{{\mathbb A}}

\nc{\ba}{\mathbf{a}} \nc{\bb}{\mathbf{b}} \nc{\bt}{\mathbf{t}}
\nc{\bB}{\mathbf{B}}

\nc{\cP}{\mathcal{P}} \nc{\cU}{\mathcal{U}} \nc{\cX}{\mathcal{X}}
\nc{\cE}{\mathcal{E}} \nc{\cS}{\mathcal{S}} \nc{\cA}{\mathcal{A}}
\nc{\cC}{\mathcal{C}} \nc{\cO}{\mathcal{O}} \nc{\cQ}{\mathcal{Q}}
\nc{\cB}{\mathcal{B}} \nc{\cJ}{\mathcal{J}} \nc{\cI}{\mathcal{I}}
\nc{\cM}{\mathcal{M}} \nc{\cK}{\mathcal{K}}

\nc{\fD}{\mathfrak{D}} \nc{\fI}{\mathfrak{I}} \nc{\fJ}{\mathfrak{J}}
\nc{\fS}{\mathfrak{S}} \nc{\gA}{\mathfrak{A}}
%%%%%%%%%%%%%%%%%%%%%%%%%LETRAS




%%%%%%%%%%%%%%%%%yetter drinfield
\newcommand{\ydg}{{}_{\ku G}^{\ku G}\mathcal{YD}}
\newcommand{\ydgdual}{{}_{\ku^G}^{\ku^G}\mathcal{YD}}
\newcommand{\ydf}{{}_{\ku F}^{\ku F}\mathcal{YD}}
\newcommand{\ydgx}{{}_{\ku \Gx}^{\ku \Gx}\mathcal{YD}}

\newcommand{\ydgxy}{{}_{\ku \Gy}^{\ku \Gx}\mathcal{YD}}

\newcommand{\ydixq}{{}_{\ku \ixq}^{\ku \ixq}\mathcal{YD}}

\newcommand{\ydl}{{}^H_H\mathcal{YD}}
\newcommand{\ydll}{{}_{K}^{K}\mathcal{YD}}
\newcommand{\ydh}{{}^H_H\mathcal{YD}}
\newcommand{\ydhdual}{{}^{H^*}_{H^*}\mathcal{YD}}


\newcommand{\ydha}{{}^H_A\mathcal{YD}}
\newcommand{\ydhhaa}{{}^{\Hx}_{\Ay}\mathcal{YD}}


\newcommand{\wydh}{\widehat{{}^H_H\mathcal{YD}}}
\newcommand{\ydvh}{{}^{\ac(V)}_{\ac(V)}\mathcal{YD}}
\newcommand{\ydrh}{{}^{R\# H}_{R\# H}\mathcal{YD}}
\newcommand{\ydho}{{}^{H^{\dop}}_{H^{\dop}}\mathcal{YD}}
\newcommand{\ydhsw}{{}^{H^{\sw}}_{H^{\sw}}\mathcal{YD}}

\newcommand{\ydhlf}{{}^H_H\mathcal{YD}_{\text{loc fin}}}
\newcommand{\ydholf}{{}^{H^{\dop}}_{H^{\dop}}\mathcal{YD}_{\text{loc fin}}}

\nc{\yd}{\mathcal{YD}}

\newcommand{\ydsn}{{}^{\ku{\Sn_n}}_{\ku{\Sn_n}}\mathcal{YD}}
\newcommand{\ydsnd}{{}^{\ku^{\Sn_n}}_{\ku^{\Sn_n}}\mathcal{YD}}
\nc{\ydSn}[1]{{}^{\Sn_{#1}}_{\Sn_{#1}}\yd}
\nc{\ydSndual}[1]{{}^{\ku^{\Sn_{#1}}}_{\ku^{\Sn_{#1}}}\yd}
\newcommand{\ydstres}{{}^{\ku^{\Sn_3}}_{\ku^{\Sn_3}}\mathcal{YD}}
%%%%%%%%%%%%%%%%%yetter drinfield


%%%%%%%%%%%%%%%%%%%%%%%%%%%%%Operatorename
\newcommand\Irr{\operatorname{Irr}}
\newcommand\id{\operatorname{id}}
\newcommand\ad{\operatorname{ad}}
\newcommand\Ad{\operatorname{Ad}}
\newcommand\Ext{\operatorname{Ext}}
\newcommand\tr{\operatorname{tr}}
\newcommand\gr{\operatorname{gr}}
\newcommand\grdual{\operatorname{gr-dual}}
\newcommand\Gr{\operatorname{Gr}}
\newcommand\co{\operatorname{co}}
\newcommand\car{\operatorname{car}}
\newcommand\rk{\operatorname{rg}}
\newcommand\ord{\operatorname{ord}}
\newcommand\cop{\operatorname{cop}}
\newcommand\End{\operatorname{End}}
\newcommand\Hom{\operatorname{Hom}}
\newcommand\Alg{\operatorname{Alg}}
\newcommand\Aut{\operatorname{Aut}}
\newcommand\Int{\operatorname{Int}}
\newcommand\Id{\operatorname{Id}}
\newcommand\qAut{\operatorname{qAut}}
\newcommand\Map{\operatorname{Map}}
\newcommand\Jac{\operatorname{Jac}}
\newcommand\Rad{\operatorname{Rad}}
\newcommand\Rep{\operatorname{Rep}}
\newcommand\Ker{\operatorname{Ker}}
\newcommand\Img{\operatorname{Im}}
\newcommand\Ind{\operatorname{Ind}}
\newcommand\Comod{\operatorname{Comod}}
\newcommand\Reg{\operatorname{Reg}}
\newcommand\Pic{\operatorname{Pic}}
\newcommand\textitb{\operatorname{Emb}}
\newcommand\op{\operatorname{op}}
\newcommand\Perm{\operatorname{Perm}}
\newcommand\Res{\operatorname{Res}}
\newcommand\res{\operatorname{res}}
\newcommand{\sop}{\operatorname{Supp}}
\newcommand\Cent{\operatorname{Cent}}
\newcommand\sgn{\operatorname{sgn}}
\nc{\GL}{\operatorname{GL}}
%%%%%%%%%%%%%%%%%%%%%%%%%%%%%Operatorename

%%%%%%%%%%%%%%%%%%%%%%%%%%%%%Usuales de hopf
\nc{\D}{\Delta} 
\nc{\e}{\varepsilon}
\nc{\adl}{\ad_\ell}
\nc{\ot}{\otimes}
\nc{\Ho}{H_0} 
\nc{\GH}{G(H)} 
\nc{\coM}{\mathcal{M}^\ast(2,k)} 
\nc{\PH}{\cP(H)}
\nc{\Ftwist}{\overset{\curvearrowright}F}
\nc{\rep}{{\mathcal Rep}(H)}
\newcommand{\deltad}{_*\delta}
\newcommand{\B}{\mathfrak{B}}
\newcommand{\wB}{\widehat{\mathfrak{B}}}
\newcommand{\Cg}[1]{C_{G}(#1)}
\nc{\hmh}{{}_H\hspace{-1pt}{\mathcal M}_H}
\nc{\hm}{{}_H\hspace{-1pt}{\mathcal M}}
\renewcommand{\_}[1]{_{\left[ #1 \right]}}
\renewcommand{\^}[1]{^{\left[ #1 \right]}}
%%%%%%%%%%%%%%%%%%%%%%%%%%%%%Usuales de hopf

%%%%%%%%%%%%%%%%%%%%%%%%%%%%%%%%Usuales
\nc{\im}{\mathtt{i}}
\renewcommand{\Re}{{\rm Re}}
\renewcommand{\Im}{{\rm Im}}
\nc{\Tr}{\mathrm{Tr}} 
\nc{\cark}{char\,k} 
\nc{\ku}{\Bbbk} 
\newcommand{\fd}{finite dimensional}
%%%%%%%%%%%%%%%%%%%%%%%%%%%%%%%%Usuales

%%%%%%%%%%%%%%%%%%%%%%%%Especiales para liftings de duales de Sn
\newcommand{\xij}[1]{x_{(#1)}}
\newcommand{\yij}[1]{y_{(#1)}}
\newcommand{\Xij}[1]{X_{(#1)}}
\newcommand{\dij}[1]{\delta_{#1}}
\newcommand{\aij}[1]{a_{(#1)}}
\newcommand{\hij}[1]{h_{(#1)}}
\newcommand{\gij}[1]{g_{(#1)}}
\newcommand{\eij}[1]{e_{(#1)}}
\newcommand{\fij}[1]{f_{#1}}
\newcommand{\tij}[1]{t_{(#1)}}
\newcommand{\Tij}[1]{T_{(#1)}}
\newcommand{\mij}[1]{m_{(#1)}}
\newcommand{\Lij}[1]{L_{(#1)}}
\newcommand{\mdos}[2]{m_{(#1)(#2)}}
\newcommand{\mtres}[3]{m_{(#1)(#2)(#3)}}
\newcommand{\mcuatro}{m_{\textsf{top}}}
\newcommand{\trid}{\triangleright}
\newcommand{\link}{\sim_{\ba}}
%%%%%%%%%%%%%%%%%%%%%%%%Especiales para liftings de duales de Sn


%%%%%%%%%%%beaamer%%%%%%%%%%%%%%%%%
% Header: Secciones una arriba de la otra.
% \usetheme{Copenhagen}
% \usetheme{Warsaw}

% Header: Secciones una al lado de la otra. Feo footer.
% Los circulitos del header son medio chotos tambien.
% \usetheme[compress]{Ilmenau}
% Muy bueno: difuminado, sin footer:
%\usetheme{Frankfurt}
% agrego footer como el de Copenhagen.
%\useoutertheme[footline=authortitle,subsection=false]{miniframes}




\title[Clase 4 - Sistemas de ecuaciones lineales]{Álgebra/Álgebra II \\ Clase 4 -Sistemas de ecuaciones lineales 2}

\author[]{}
\institute[]{\normalsize FAMAF / UNC
	\\[\baselineskip] ${}^{}$
	\\[\baselineskip]
}
\date[21/03/2024]{21 de marzo de 2024}

%\titlegraphic{\includegraphics[width=0.2\textwidth]{logo_gimp100.pdf}}

% Converted to PDF using ImageMagick:
% # convert logo_gimp100.png logo_gimp100.pdf


\begin{document}

\begin{frame}
\maketitle
\end{frame}

\begin{frame}
	
%{\footnotesize
En la clase 3  motivamos la idea general del método de Gauss a través de ejemplos y vimos el concepto de sistemas de ecuaciones lineales equivalentes.

\vskip .4cm\pause

En esta clase  presentaremos las nociones de:\pause
 \begin{itemize}
  \item Matriz.\pause
  \item Matriz ampliada.\pause
    \item Operaciones elementales por fila.\pause
  %\item Matriz escalón reducida por fila (MERF).\pause
 \end{itemize}
\vskip .4cm%}

%{\footnotesize
Relacionaremos todos estos conceptos con los sistemas de ecuaciones lineales.\pause
\begin{itemize}
	\item Matriz: representará un sistema de ecuaciones homogéneo.\pause
	 \item Matriz ampliada:  representara un sistema de ecuaciones lineales no homogéneo.\pause
	\item Operaciones elementales por fila: representarán ciertas operaciones entre las diferentes ecuaciones del sistema.\pause
	%\item Matriz escalón reducida por fila (MERF):  representarán sistemas de ecuaciones linales que se resuelven en forma trivial.
\end{itemize}
%}
\end{frame}








\begin{frame}{Definición}


Una \textit{matriz $m\times n$} es un arreglo de números reales de $m$ filas y $n$ columnas.

\ \pause

{$\RR^{m\times n}$} y {$M_{m\times n}(\RR)$} denotan el conjunto de matrices $m\times n$.
 \pause
 \
 \begin{block}{Ejemplos}


\begin{center}
	\begin{tabular}{ccc}	
		$
		\left[
		\begin{matrix}
		2 & 1 & 2 \\
		3 & 0 & \pi
		\end{matrix}
		\right]
		$\qquad\qquad&$
		\left[
		\begin{matrix}
		\sqrt{2} & \frac{1}{2} & 9
		\end{matrix}
		\right]
		$\qquad\qquad&$
		\left[
		\begin{array}{r}
		10 \\ -1 \\ 1
		\end{array}
		\right]
		$
	\end{tabular}
\end{center}
 \end{block}
\end{frame}


\begin{frame}\frametitle{Convenciones}
La notación \textit{$A=[a_{ij}]\in\RR^{m\times n}$} quiere decir que $A$ es una matriz $m\times n$ de la siguiente forma

\begin{align*}
\left[
\begin{array}{ccccccccc}
 a_{11} & a_{12} & \cdots & & & a_{1j} & & \cdots & a_{1n}\\ 
 a_{21} & a_{22} & \cdots & & & a_{2j} & & \cdots & a_{2n}\\
 \vdots & \vdots & \ddots & & & \vdots & & & \vdots\\
  & &  & & & & & \\
 a_{i1} & a_{i2} & \cdots & & &  a_{ij} & & \cdots & a_{in}\\  
 & &  & & & & & \\
 \vdots & \vdots & & & & \vdots  & & \ddots & \vdots\\
 a_{m1} & a_{m2} & \cdots & & & a_{mj} & & \cdots & a_{mn}\\ 
\end{array}
\right] 
\end{align*}

\vspace{1cm}

\end{frame}

\begin{frame}
	
La \textit{fila $i$ } de una matriz es la fila (una $n$-upla) ubicada en la posición $i$ desde arriba:	
	\begin{equation*}
	\begin{array}{r}
	   \\ 
  \\ 
  \\ 
F_i \longrightarrow  \\  
    \\ 
  \\ 
	\end{array}
	\left[
	\begin{array}{cccc}
	a_{11} & a_{12} & \cdots & a_{1n}\\ 
	a_{21} & a_{22} & \cdots & a_{2n}\\
	\vdots & \vdots & & \vdots\\
	\alert{a_{i1}} & \alert{a_{i2}} & \cdots &  \alert{a_{in}}\\  
	\vdots & \vdots & & \vdots\\
	a_{m1} & a_{m2} & \cdots & a_{mn}
	\end{array}
	\right] 
	\end{equation*}
	
\end{frame}


\begin{frame}
	
	La \textit{columna $i$ } de una matriz es la columna (una $m$-upla) ubicada en la posición $i$ desde la izquierda:	
	\begin{align*}
	\begin{array}{l}
	C_i \qquad\qquad\quad\,\,\\ \downarrow  
	\end{array}
\\
	\left[
	\begin{array}{cccccc}
	a_{11} & a_{12} & \cdots & \alert{a_{1i}}& \cdots& a_{1n}\\ 
	a_{21} & a_{22} & \cdots & \alert{a_{2i}}& \cdots& a_{2n}\\ 
	\vdots & \vdots &   & \vdots&  & \vdots\\ 
	a_{m1} & a_{m2} & \cdots & \alert{a_{mi}}& \cdots& a_{mn}
	\end{array}
	\right] 
	\end{align*}
	
\end{frame}


\begin{frame}{Convenciones}

\begin{itemize}
 \item Sea $A=[a_{ij}]\in\RR^{m\times n}$ una matriz. Escribiremos {$[A]_{ij}$} para denotar la entrada $a_{ij}$ de $A$.
\end{itemize}
\vskip .4cm\pause
\begin{itemize}
 \item Dos matrices del mismo tama\~no $A=[a_{ij}]\in\RR^{m\times n}$  y $B=[b_{ij}]\in\RR^{m\times n}$ son iguales si cada una de sus entradas lo son:
 $$
 A=B\quad\Longleftrightarrow\quad a_{ij}=b_{ij}, \quad 1 \le  i \le m,\; 1 \le j
\le n $$
\end{itemize}


\end{frame}




\begin{frame}{Sistemas de ecuaciones}
Usaremos matrices para representar los sistemas de ecuaciones. 

\
\begin{definicion}
Si $A=[a_{ij}]\in\RR^{m\times n}$, $X=[x_i]\in\RR^{n \times 1}$ e 
$Y=[y_i]\in\RR^{m \times 1}$ entonces
$$
{AX=Y}
$$
representa al sistema de ecuaciones
\pause
\begin{equation}\label{sist-eq-hom}
\begin{matrix}
a_{11}x_1& + &a_{12}x_2& + &\cdots& + &a_{1n}x_n &= &y_1\\
a_{21}x_1& + &a_{22}x_2& + &\cdots& + &a_{2n}x_n &= &y_2\\
\vdots&  &\vdots& &&  &\vdots \\
a_{m1}x_1& + &a_{m2}x_2& + &\cdots& + &a_{mn}x_n &=&y_m.
\end{matrix}\tag{E}
\end{equation}
\end{definicion}
{\color{gray}(se usa en la pantalla \ref{def-matriz-ampliada}) }


\end{frame}


\begin{frame}
También lo podemos denotar

\vskip .8cm 

\begin{align*}
\left[
\begin{array}{ccccccc}
 a_{11} & a_{12} & \cdots & & &  & a_{1n}\\ 
 a_{21} & a_{22} & \cdots & & &  & a_{2n}\\
 \vdots & \vdots & \ddots & & &  & \vdots\\
  & &  & & &  \\
 & &  & & &  \\
 a_{m1} & a_{m2} & \cdots & & & & a_{mn}\\ 
\end{array}
\right] 
\quad
\left[
\begin{array}{c}
x_1\\ x_2\\ \vdots \\ \\ x_n
\end{array}
\right]
=
\left[
\begin{array}{c}
y_1\\ y_2\\ \vdots \\ \\ \\ y_m
\end{array}
\right]
\end{align*}


\vskip .8cm \vskip .8cm 
\end{frame}



\begin{frame}{Ejemplo}
El sistema de ecuaciones
\begin{equation*}
\begin{matrix}
x_1 &  & +2x_3 & = 1 \\
x_1& -3x_2 & +3x_3 & =2 \\
2x_1& -3x_2 & +5x_3 & =3 \\
x_1& & +3x_3 & =-1
\end{matrix}
\end{equation*}
es representado de la forma $AX=Y$:
\begin{align*}
\left[
\begin{matrix}
1 & 0& 2 & \\
1& -3& 3 & \\
2& -3& 5 & \\
1&0& 3&
\end{matrix}
\right]
\left[
\begin{matrix}
x_1\\ x_2 \\ x_3 
\end{matrix}
\right]
=
\left[
\begin{matrix}
1\\ 2 \\ 3 \\-1
\end{matrix}
\right]
\end{align*}
\pause


\begin{itemize}
 \item Si una incógnita no aparece en una ecuación, el correspondiente coeficiente de la matriz es $0$.
 \item La cantidad de incógnitas es igual a la cantidad de columnas de la matriz $A$.
\end{itemize}

\end{frame}





\begin{frame}{Operaciones elementales por fila: motivación}
	
	Las \textit{operaciones elementales por fila} son:\pause
	\begin{itemize}
		\item transformaciones con las cuales podemos modificar una matriz de manera tal que los correspondientes sistemas de ecuaciones tengan las mismas soluciones.
		\pause
		\item la versión ``matricial'' de las combinaciones lineales de ecuaciones que hicimos en la clase anterior para encontrar las soluciones de los sistemas.
	\end{itemize}
	\pause
	\begin{block}{Observación}
		Hay tres tipos de operaciones las cuales definiremos a continuación.
	\end{block}
	
\end{frame}



\begin{frame}
	En una matriz $A$ de $m \times n$, cada fila puede ser considerada un vector en $\R^n$.
	\vskip .4cm\pause
	Si la fila $i$  de $A$ es
	$$
	\begin{bmatrix} a_{i1}& a_{i2}& \cdots &a_{in} 	\end{bmatrix},
	$$
	y la denotamos $F_i(A)$ o simplemente $F_i$ si $A$. Si $c\in \K$,  entonces 
		\vskip .4cm
	\begin{itemize}\pause
		\item $cF_i = \begin{bmatrix} ca_{i1}& ca_{i2}& \cdots &ca_{in} 	\end{bmatrix}$.\pause
		\item $F_r + F_s = \begin{bmatrix} a_{r1}+a_{s1}& a_{i2}+a_{s2}& \cdots &a_{in}+a_{sn} 	\end{bmatrix}$.\pause
		\item $F_i = \begin{bmatrix} 0& 0& \cdots &0 	\end{bmatrix}$, la fila nula.
	\end{itemize}
\end{frame}

\begin{frame}{Operaciones elementales por fila: definición}
	\begin{definicion}
		Sea $A = [a_{ij}]$ una matriz $m \times  n$, diremos que $e$ es una  \textit{operación elemental por fila}\index{operación elemental por fila} si aplicada a la matriz $A$ se obtiene  $e(A)$ de la siguiente manera:\pause
		\begin{enumerate}
			\item[E1.]\label{elem-1} multiplicando la fila $r$ por una constante $c\not=0$, o\pause
			\item[E2.]\label{elem-2} cambiando la fila $F_r$ por $F_r + tF_s$ con $r\not=s$, para algún $t \in \K$, o\pause
			\item[E3.]\label{elem-3} permutando la fila $r$ por la fila $s$.   
		\end{enumerate}
	\end{definicion}

\vskip .4cm

		E1, E2 y E3 son tres tipos de operaciones elementales,
\end{frame}







\begin{frame}
	\textbf{E1:} {multiplicar la fila $i$ por un número real $c\neq0$.}
	
	
	{\footnotesize
		\begin{align*}
		\left[
		\begin{array}{cccc}
		a_{11} & a_{12} & \cdots & a_{1n}\\ 
		a_{21} & a_{22} & \cdots & a_{2n}\\
		\vdots & \vdots & & \vdots\\
		\alert{a_{i1}} & \alert{a_{i2}} & \cdots &  \alert{a_{in}}\\  
		\vdots & \vdots & & \vdots\\
		a_{m1} & a_{m2} & \cdots & a_{mn}
		\end{array}
		\right] 
		\qquad\qquad \stackrel{c\,F_i}{\longrightarrow} \qquad\qquad
		\left[
		\begin{array}{cccc}
		a_{11} & a_{12} & \cdots & a_{1n}\\ 
		a_{21} & a_{22} & \cdots & a_{2n}\\
		\vdots & \vdots & & \vdots\\
		\alert{c\,a_{i1}} & \alert{c\,a_{i2}} & \cdots &  \alert{c\,a_{in}}\\  
		\vdots & \vdots & & \vdots\\
		a_{m1} & a_{m2} & \cdots & a_{mn}
		\end{array}
		\right]
		\end{align*} 
	}\pause
	
	\begin{exampleblock}{Ejemplo}
		Multiplicar la primer fila por $-2$:
		\begin{align*}
		\left[
		\begin{array}{rr}
		1&2\\
		3&4\\
		5&6
		\end{array}
		\right] 
		\qquad\qquad\qquad\qquad
		\left[
		\begin{array}{rr}
		-2&-4\\
		3&4\\
		5&6
		\end{array}
		\right]
		\end{align*} 
	\end{exampleblock}
\end{frame}



\begin{frame}
	
	\textbf{E2:} sumar a la fila $r$ un múltiplo de la fila $s$.
	
	
	
	{\footnotesize
		\begin{align*}
		\left[
		\begin{array}{cccc}
		a_{11} & a_{12} & \cdots & a_{1n}\\ 
		a_{21} & a_{22} & \cdots & a_{2n}\\
		\vdots & \vdots & & \vdots\\
		\alert{a_{s1}} & \alert{a_{s2}} & \cdots &  \alert{a_{sn}}\\
		\vdots & \vdots & & \vdots\\
		\alert{a_{r1}} & \alert{a_{r2}} & \cdots &  \alert{a_{rn}}\\  
		\vdots & \vdots & & \vdots\\
		a_{m1} & a_{m2} & \cdots & a_{mn}
		\end{array}
		\right] 
		\quad \stackrel{F_r + t F_s}{\longrightarrow} \quad
		\left[
		\begin{array}{cccc}
		a_{11} & a_{12} & \cdots & a_{1n}\\ 
		a_{21} & a_{22} & \cdots & a_{2n}\\
		\vdots & \vdots & & \vdots\\
		\alert{a_{s1}} & \alert{a_{s2}} & \cdots &  \alert{a_{sn}}\\
		\vdots & \vdots & & \vdots\\
		\alert{a_{r1}+ta_{s1}} & \alert{a_{r2}+ta_{s2}} & \cdots &  \alert{a_{rn}+ta_{sn}}\\  
		\vdots & \vdots & & \vdots\\
		a_{m1} & a_{m2} & \cdots & a_{mn}
		\end{array}\right]
		\end{align*} 
	}
	\pause
	
	\begin{exampleblock}{Ejemplo}
		Sumar a la segunda fila la primer fila multiplicada por $3$:
		{\footnotesize
			\begin{align*}
			\left[
			\begin{array}{rr}
			1&2\\
			3&4\\
			5&6
			\end{array}
			\right] 
			\quad \stackrel{F_2 + 3 F_1}{\longrightarrow} \quad
			\left[
			\begin{array}{cc}
			1&2\\
			3+3\cdot 1&4+3\cdot 2\\
			5&6
			\end{array}
			\right]
			=
			\left[
			\begin{array}{cc}
			1&2\\
			6&10\\
			5&6
			\end{array}
			\right]
			\end{align*} 
		}
	\end{exampleblock}
	
	
\end{frame}

\begin{frame}
	
	\textbf{E3:} intercambiar las fila $r$ y  $s$.
	
	
	{\footnotesize
		\begin{align*}
		\left[
		\begin{array}{cccc}
		a_{11} & a_{12} & \cdots & a_{1n}\\ 
		a_{21} & a_{22} & \cdots & a_{2n}\\
		\vdots & \vdots & & \vdots\\
		\alert{a_{r1}} & \alert{a_{r2}} & \cdots &  \alert{a_{rn}}\\
		\vdots & \vdots & & \vdots\\
		\alert{a_{s1}} & \alert{a_{s2}} & \cdots &  \alert{a_{sn}}\\  
		\vdots & \vdots & & \vdots\\
		a_{m1} & a_{m2} & \cdots & a_{mn}
		\end{array}
		\right] 
		\qquad\stackrel{F_r\leftrightarrow F_s}{\longrightarrow} \qquad
		\left[
		\begin{array}{cccc}
		a_{11} & a_{12} & \cdots & a_{1n}\\ 
		a_{21} & a_{22} & \cdots & a_{2n}\\
		\vdots & \vdots & & \vdots\\
		\alert{a_{s1}} & \alert{a_{s2}} & \cdots &  \alert{a_{sn}}\\
		\vdots & \vdots & & \vdots\\
		\alert{a_{r1}} & \alert{a_{r2}} & \cdots &  \alert{a_{rn}}\\  
		\vdots & \vdots & & \vdots\\
		a_{m1} & a_{m2} & \cdots & a_{mn}
		\end{array}\right]
		\end{align*} 
	}
	
	\pause
	
	\begin{exampleblock}{Ejemplo}
		Intercambiar la segunda y tercer fila:
		{\footnotesize
			\begin{align*}
			\left[
			\begin{array}{rr}
			1&2\\
			3&4\\
			5&6
			\end{array}
			\right] 
			\quad\stackrel{F_3\leftrightarrow F_2}{\longrightarrow} \quad
			\left[
			\begin{array}{cc}
			1&2\\
			5&6\\
			3&4
			\end{array}
			\right]
			\end{align*} 
		}
	\end{exampleblock}
\end{frame}

\begin{frame}{Convenciones}
	
	\begin{itemize}
		\item Si $A$ es una matriz, {$e(A)$} denotará la matriz que obtenemos después de modificar a $A$ por cierta operación elemental {$e$}.
	\end{itemize}
	\pause
	\begin{exampleblock}{Ejemplo}
		Si $e$ es la operación intercambiar la segunda y tercer fila y $A={\footnotesize\left[
			\begin{array}{rr}
			1&2\\
			3&4\\
			5&6
			\end{array}
			\right]}
		$, entonces $e(A)={\footnotesize\left[
			\begin{array}{cc}
			1&2\\
			5&6\\
			3&4
			\end{array}
			\right]}$.
	\end{exampleblock}
	
	\pause
	\begin{itemize}
		\item Como hicimos en los ejemplos, cuando le apliquemos una operación elemental a una matriz especificaremos arriba de una flecha que operación aplicamos:
		\begin{align*}
		A\quad\stackrel{e}{\longrightarrow}\quad e(A) 
		\end{align*}
		
		Esta notación  es obligatoria para la corrección de exámenes. 
	\end{itemize}
\end{frame}


\begin{frame}
			\begin{teorema}
		A cada operación elemental por fila $e$ le corresponde otra operación elemental $e^\prime$ (del mismo tipo que $e$) tal que $e^\prime(e(A)) = A$ y $e(e^\prime(A)) = A$. En otras palabras, la operación inversa de una operación elemental es otra operación elemental del mismo tipo.  
	\end{teorema}
	\begin{proof} \vskip -.2cm\pause
		\begin{enumerate}
			\item[E1.] Para $c\not=0$, la operación inversa de $cF_r$ es $\displaystyle\frac1cF_r$.\pause
			\item[E2.] La operación inversa de $F_r + cF_s$ es $F_r - cF_s$ ($r \ne s$).\pause
			\item[E3.] La operación inversa de permutar la fila $r$ por la fila $s$ es la misma operación.
		\end{enumerate}
	\end{proof} \qed
\end{frame}

\begin{frame}
	
	\begin{observacion}
	
	\end{observacion}
	\begin{itemize}
		\item Las operaciones elementales son operaciones lineales entre filas,  es decir  del tipo $sF +tF'$ donde $s,t \in \R$ y $F,F'$ son filas.
		\vskip .4cm
		\item De una sucesión de operaciones elementales obtenemos una matriz donde cada fila es combinación lineal de las filas de la matriz original. 
	\end{itemize}

\vskip 2.5cm
\end{frame}

\begin{frame}
					
	
	\begin{definicion}\label{def-matriz-ampliada}
		Consideremos un sistema como en {\color{blue}(\ref{sist-eq-hom})} y sea  $A$ la matriz correspondiente al sistema. La \textit{matriz  ampliada}\index{sistema de ecuaciones lineales!matriz  ampliada} del sistema es 
		\begin{equation}\label{mtrx-ampliada}
		A' = \left[\begin{array}{ccc|c}  
		a_{11} & \cdots & a_{1n} &  y_1 \\
		a_{21} & \cdots & a_{2n} &  y_2 \\
		\vdots &  & \vdots  &  \vdots  \\
		a_{m1} & \cdots & a_{mn} &  y_m 
		\end{array}\right]
		\end{equation}
		que también podemos denotar
		\begin{equation*}
		A' = [A | Y].
		\end{equation*}
	\end{definicion}
	\pause
	\begin{observacion}
		Hay  una correspondencia biunívoca entre 
		\begin{center}
			sistemas de ecuaciones lineales\quad $\longleftrightarrow$\quad  matrices ampliadas. 
		\end{center}
		
	\end{observacion}
	
\end{frame}



\begin{frame}
	\begin{ejemplo} Dado  el sistema
		\begin{align*}
		\left[
		\begin{array}{rrr}
		1 & 0 & 2\\
		1& -3& 3\\
		2& -3& 5
		\end{array}
		\right]
		\left[
		\begin{array}{r}
		x_1\\
		x_2 \\
		x_3
		\end{array}
		\right]
		=
		\left[
		\begin{array}{r}
		1 \\
		2 \\
		3
		\end{array}
		\right],
		\end{align*}
		la matriz ampliada es \pause
		$$
		\left[
		\begin{array}{rrr|r}
		1 & 0 & 2 & 1 \\
		1& -3& 3 & 2 \\
		2& -3& 5 & 3
		\end{array}
		\right].
		$$
		
	\end{ejemplo}
\end{frame}


\begin{frame}{Operaciones elementales $\longrightarrow$ operaciones entre ecuaciones}
	
	
	 Sea $AX=Y$ un sistema de ecuaciones lineales y $[A|Y]$  su matriz ampliada.
	\vskip .4cm
	\begin{itemize}
	


	\pause
	\item La matriz ampliada, es una matriz con una columna más. La raya vertical es para distinguir los coeficientes de las variables de las constantes a las que se igualan las ecuaciones.
	
	\vskip .4cm\pause
	\item Es decir,  si el sistema es $AX =Y$ donde $A$  es matriz $m \times n$,  entonces la matriz ampliada es $[A|Y]$ es una matriz $m \times (n+1)$,  es decir de $m$-filas y $n+1$-columnas. 
	
	\vskip .4cm\pause
	
	\item Podemos aplicar la operaciones elementales por fila a una matriz ampliada, y eso es lo que haremos en las próximas pantallas.
		\end{itemize}
	
	
	
	

\end{frame}


\begin{frame}{Operaciones elementales por fila en matrices ampliadas}
	
		La relación biunívoca entre sistemas de ecuaciones lineales y matrices ampliadas, resulta en:
	
	
	
	\begin{equation}
	\begin{array}{c}
	\text{Multiplicar fila $r$ por  $c\not=0$} \\
	\updownarrow\\
	\text{multiplicar ecuación $r$-ésima por  $c\not=0$.}
	\end{array}\tag{E1}
	\end{equation}
	\vskip .3cm
	\pause	\begin{equation}
	\begin{array}{c}
	\text{Cambiar fila $F_r$ por $F_r + tF_s$ con $r\not=s$, para algún $t \in \K$} \\
	\updownarrow\\
	\text{sumar a la ecuación $r$-ésima $t$ veces la ecuación $s$-ésima.}
	\end{array}\tag{E2}
	\end{equation}
	\vskip .3cm
	\pause	 \begin{equation}
	\begin{array}{c}
	\text{Permutar fila $r$ por  fila $s$} \\
	\updownarrow\\
	\text{permutar la ecuación $r$-ésima  por la ecuación  $s$-ésima.}
	\end{array}\tag{E3}
	\end{equation}
\end{frame}

\begin{frame}
			
	\begin{teorema}\label{th-equiv-op-elem} Sea $[A | Y]$ la matriz ampliada de un sistema de ecuaciones lineales y sea $[B | Z]$ una matriz que se obtiene a partir de $[A | Y]$ por medio de operaciones elementales. Entonces, los sistemas  $[A | Y]$ y  $[B | Z]$ tienen las mismas soluciones. 
	\end{teorema}
	\begin{proof}\pause
		\begin{itemize}
			\item $[A | Y] \leadsto [B | Z]\quad  \Rightarrow \quad \operatorname{filas}[B | Z] = \operatorname{c.l.}\operatorname{filas}[A | Y]$.
			\item Luego, $\operatorname{Soluciones}[A | Y]  \Rightarrow  \operatorname{Soluciones}[B | Z]$.
		\end{itemize}
Como toda operación elemental tiene inversa 	$\Rightarrow $
		\begin{itemize}
			\item $[B | Z] \leadsto [A | Y] \quad\Rightarrow\quad \operatorname{filas}[A | Y] = \operatorname{c.l.}\operatorname{filas}[B | Z]$.
			\item Luego, $\operatorname{Soluciones}[B | Z]  \Rightarrow  \operatorname{Soluciones}[A | Y]$.
		\end{itemize}
	\vskip .3cm
	Por lo tanto $\operatorname{Soluciones}[A | Y]  =  \operatorname{Soluciones}[B | Z]$. \qed
	\end{proof}
\end{frame}




\begin{frame}
 	\begin{ejemplo}
 	Resolvamos el siguiente sistema:
 	\begin{align*}
 	\begin{split}
 	2x_1 - 6x_2 + x_3  &= 2 \\
 	x_1 - 4x_2  &=1 \\
 	2x_1 -4x_2 -x_3  &= 0,  
 	\end{split}
 	\end{align*}
 	para  $x_i \in \R$ ($1 \le i \le 4$). 
 	
 	\vskip .4cm \pause
 	La matriz ampliada  correspondiente a este sistema de ecuaciones es 
 	$$
 	\left[\begin{array}{ccc|c} 
 	2& -6&1& 2 \\ 1&-4 &0&1 \\ 2&-4&-1&0 \end{array}\right].
 	$$
 	 \end{ejemplo}
 \end{frame}
 	
 \begin{frame}
  	
 	
 	Encontraremos una matriz que nos dará un sistema de ecuaciones equivalente, pero con soluciones mucho más evidentes:
 \pause
 	%{\footnotesize
 	\begin{multline*}
 	\left[\begin{array}{ccc|c}  2& -6&1& 2 \\ 1&-4 &0&1 \\ 2&-4&-1&0 \end{array}\right]
 	\stackrel{F_1\leftrightarrow F_2}{\longrightarrow} 
 	\left[\begin{array}{ccc|c}  1&-4 &0&1 \\ 2& -6&1&2 \\ 2&-4&-1&0 \end{array}\right]
 	\\
 	\underset{F_3-2 F_1}{\stackrel{F_2-2 F_1}{\longrightarrow} }
 	\left[\begin{array}{ccc|c}  1&-4 &0&1 \\  0& 2&1&0  \\ 0&4&-1&-2 \end{array}\right] 
 	\stackrel{F_2/2}{\longrightarrow} 
 	\left[\begin{array}{ccc|c}  1&-4 &0&1 \\  0& 1&1/2&0  \\ 0&4&-1&-2 \end{array}\right] 
 	\\
 	\underset{F_3 -4F_2}{\stackrel{F_1+4F_2}{\longrightarrow}} 
 	\left[\begin{array}{ccc|c}  1&0 &2&1 \\ 0& 1&1/2&0 \\ 0&0&-3&-2 \end{array}\right] 
 		\stackrel{F_3/(-3)}{\longrightarrow} 
 	\left[\begin{array}{ccc|c}  1&0 &2&1 \\ 0& 1&1/2&0 \\ 0&0&1&2/3 \end{array}\right] 
 	\\
 	\underset{F2 -(1/2)F_3}{\stackrel{F_1-2F_3}{\longrightarrow}} 
 	\left[\begin{array}{ccc|c}  1&0 &0&-1/3 \\ 0& 1&0&-1/3 \\ 0&0&1&2/3 \end{array}\right] .
 	\end{multline*}
% }
\end{frame}

 	\begin{frame}
 		Volvamos a las ecuaciones: el nuevo sistema de ecuaciones, equivalente al original, es
 		\begin{align*}
 		x_1  &= -\frac{1}{3} \\
 		x_2  &=-\frac{1}{3} \\
 		x_3 &= \frac{2}{3}, 
 		\end{align*}
 		
 		Por lo tanto, el sistema tiene una sola solución:
 		$$
 		( -\frac{1}{3}, -\frac{1}{3} ,\frac{2}{3}).
 		$$

 	\end{frame}
 	

\begin{frame}{Sistemas homogéneos}
 Si el sistema de ecuaciones lineales  es homogéneo, es decir del tipo $AX=0$, entonces la matriz ampliada es 
 \begin{equation*}
 	[A|0].
 \end{equation*}
 
 Haciendo operaciones elementales sucesivas llegamos a otra matriz  
  \begin{equation*}
 [B|0].
 \end{equation*}
 
 Luego,  en este caso (sistema homogéneo) la convención es no escribir la matriz ampliada para resolver el sistema, sino trabajar directamente sobre la matriz $A$. 
 
 
\end{frame}




\end{document}
