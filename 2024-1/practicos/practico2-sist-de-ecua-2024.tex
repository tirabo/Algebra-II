% PDFLaTeX
\documentclass[a4paper,12pt,twoside,spanish,reqno]{amsbook}
%%%---------------------------------------------------
\usepackage[math]{kurier}

\usepackage{etex}
\usepackage{t1enc}
\usepackage{latexsym}
\usepackage[utf8]{inputenc}
\usepackage{verbatim}
\usepackage{multicol}
\usepackage{amsgen,amsmath,amstext,amsbsy,amsopn,amsfonts,amssymb}
\usepackage{amsthm}
\usepackage{calc}         % From LaTeX distribution
\usepackage{graphicx}     % From LaTeX distribution
\usepackage{ifthen}
\input{random.tex}   
\usepackage{tikz}
\usetikzlibrary{arrows}
\usetikzlibrary{matrix}
\usepackage{mathtools}
\usepackage{stackrel}
\usepackage{enumitem}
\usepackage{tkz-graph}

\usepackage{enumitem} 
\usepackage[compatibility=false]{caption} % para usar subcaption
\usepackage{subcaption} % para poner varias imagenes juntas
\usetikzlibrary{arrows.meta}
\usepackage{hyperref}
\hypersetup{ 
    colorlinks=true,
    linkcolor=blue,
    filecolor=magenta,      
    urlcolor=cyan,
}
\usepackage{hypcap}
\numberwithin{equation}{section}
% http://www.texnia.com/archive/enumitem.pdf (para las labels de los enumerate)
\renewcommand\labelitemi{$\circ$}
\setlist[enumerate, 1]{label={(\arabic*)}}
\setlist[enumerate, 2]{label=\emph{\alph*)}}


\newcommand{\rta}{\noindent\textsc{Solución: }} 

\newcommand{\img}{\operatorname{Im}}
\newcommand{\nuc}{\operatorname{Nu}}
\newcommand\im{\operatorname{Im}}
\renewcommand\nu{\operatorname{Nu}}
\newcommand{\la}{\langle}
\newcommand{\ra}{\rangle}
\renewcommand{\t}{{\operatorname{t}}}
\renewcommand{\sin}{{\,\operatorname{sen}}}
\newcommand{\Q}{\mathbb Q}
\newcommand{\R}{\mathbb R}
\newcommand{\C}{\mathbb C}
\newcommand{\K}{\mathbb K}
\newcommand{\F}{\mathbb F}
\newcommand{\Z}{\mathbb Z}
\newenvironment{amatrix}[1]{%
  \left[\begin{array}{@{}*{#1}{c}|c@{}}
}{%
  \end{array}\right]
}

%%% FORMATOS %%%%%%%%%%%%%%%%%%%%%%%%%%%%%%%%%%%%%%%%%%%%%%%%%%%%%%%%%%%%%%%%%%%%%
\tolerance=10000
\renewcommand{\baselinestretch}{1.3}
\usepackage[a4paper, top=3cm, left=3cm, right=2cm, bottom=2.5cm]{geometry}
\usepackage{setspace}
%\setlength{\parindent}{0,7cm}% tamaño de sangria.
\setlength{\parskip}{0,4cm} % separación entre parrafos.
\renewcommand{\baselinestretch}{0.90}% separacion del interlineado
%%%%%%%%%%%%%%%%%%%%%%%%%%%%%%%%%%%%%%%%%%%%%%%%%%%%%%%%%%%%%%%%%%%%%%%%%%%%%%%%%%%
%\end{comment}
%%% FIN FORMATOS  %%%%%%%%%%%%%%%%%%%%%%%%%%%%%%%%%%%%%%%%%%%%%%%%%%%%%%%%%%%%%%%%%

\begin{document}
    \baselineskip=0.55truecm %original
    
    
    {\bf \begin{center} Práctico 2 \\ Álgebra  II -- Año 2024/1 \\ FAMAF \end{center}}

\

\centerline{\textsc{Sistemas de ecuaciones}}

\subsection*{Objetivos}

\begin{itemize}
 \item Aprender a plantear y resolver sistemas de ecuaciones lineales.
\end{itemize}

\smallskip

\bigbreak



\begin{enumerate}
\item {\it Juego Suko}. Colocar los números del $1$ al $9$ en las celdas de la siguiente tabla de modo que el número en cada círculo sea igual a la suma de las cuatro celdas adyacentes, y la suma de las celdas del mismo color sea igual al número en el círculo de igual color.

\begin{center}
  \begin{tikzpicture}
    \draw [fill=gray!10] (0,0) rectangle (1,-1); 
    \draw [fill=gray!10] (1,0) rectangle (2,-1); 
    \draw [fill=gray!10] (2,0) rectangle (3,-1); 
    \draw [fill=gray!30] (0,-1) rectangle (1,-2); 
    \draw [fill=gray!30] (1,-1) rectangle (2,-2); 
    \draw [fill=gray!10] (2,-1) rectangle (3,-2); 
    \draw [fill=gray!80] (0,-2) rectangle (1,-3); 
    \draw [fill=gray!80] (1,-2) rectangle (2,-3); 
    \draw [fill=gray!80] (2,-2) rectangle (3,-3); 
    \filldraw[fill=white](1,-1) circle (0.33);
    \filldraw[fill=white](2,-1) circle (0.33);
    \filldraw[fill=white](1,-2) circle (0.33);
    \filldraw[fill=white](2,-2) circle (0.33);
    \node at (1,-1) {12};
    \node at (2,-1) {19};
    \node at (1,-2) {23};
    \node at (2,-2) {26};
    \filldraw[fill=gray!10](0.5,-3.5) circle (0.33);
    \filldraw[fill=gray!30](1.5,-3.5) circle (0.33);
    \filldraw[fill=gray!80](2.5,-3.5) circle (0.33);
    \node at (0.5,-3.5) {14};
    \node at (1.5,-3.5) {9};
    \node at (2.5,-3.5) {22};
    
    \end{tikzpicture}
\end{center}

\vskip .4cm

\item\label{polinomio} Encontrar los coeficientes reales del polinomio $p(x) = ax^2+bx+c$ de manera tal que $p(1)=2$, $p(2)=7$ y $p(3)=14$.

\vskip .4cm

\item Determinar cuáles de las siguientes matrices son MERF.
$$\begin{array}{lccccl}
\begin{bmatrix}1 & 2 & 0 \\0 & 0 & 1 \end{bmatrix}, &
\begin{bmatrix}1 & 0 & 2 \\0 & 1 & -3 \end{bmatrix}, &
\begin{bmatrix}0 & 1 & 0 \\0 & 0 & 1 \end{bmatrix}, &
\begin{bmatrix}0 & 1 & 0 \\0 & 0 & 0 \end{bmatrix}, &
\begin{bmatrix}1 & 0 & 0  \\0 & 0 & 1 \\0 & 0 & 1 \end{bmatrix},&
\begin{bmatrix}1 & 0 & 0  \\0 & 0 & 0 \\0 & 0 & 1 \end{bmatrix}.
\end{array}$$

\vskip .4cm

\item Para cada una de las MERF del ejercicio anterior,

\begin{enumerate}
\item
asumir que es la matriz de un sistema homogéneo, escribir el sistema
y dar las soluciones del sistema.

\item
asumir que es la matriz ampliada de un sistema no homogéneo, escribir el sistema
y dar las soluciones del sistema.
\end{enumerate}

\vskip .4cm

\item\label{sistemas homogeneos} Para cada uno de los siguientes sistemas de ecuaciones, describir explícita o paramétricamente todas las soluciones e indicar cuál es la MERF asociada al sistema.

\begin{multicols}{3}
\begin{enumerate}
\item $\begin{cases}
 -x - y + 4z = 0\\
 x+3y+8z = 0\\
 x+2y + 5z = 0
\end{cases}$
\item $\begin{cases}
 x - 3y + 5z = 0\\
 2x-3y+z = 0\\
 -y + 3z = 0
\end{cases}$

\item $\begin{cases}
x-z+2t = 0\\
-x+2y-z+2t = 0\\
-x+y = 0
\end{cases}$

\end{enumerate}
\end{multicols}

\begin{multicols}{3}
\begin{enumerate}

\item[(d)] $\begin{cases}
 -x - y + 4z = 1\\
 x+3y+8z = 3\\
 x+2y + 5z = 1
\end{cases}$

\item[(e)] $\begin{cases}
 x - 3y + 5z = 1\\
 2x-3y+z = 3\\
 -y + 3z = 1
\end{cases}$


\item[(f)] $\begin{cases}
x-z+2t = 1\\
-x+2y-z+2t = 3\\
-x+y = 1
\end{cases}$

\end{enumerate}
\end{multicols}


\vskip .4cm

\item\label{sistemas con soluciones} Para cada uno de los siguientes sistemas, describir implícitamente el conjunto de los vectores $(b_1,b_2,b_3)$
o $(b_1,b_2,b_3,b_4)$ para los cuales cada sistema tiene solución.

\begin{multicols}{2}
\begin{enumerate}

\item $\begin{cases}
 x - 3y + 5z = b_1\\
 2x-3y+z = b_2\\
 -y + 3z = b_3
\end{cases}$


\item
$\begin{cases}
x-z+2t = b_1\\
-x+2y-z+2t = b_2\\
-x+y = b_3\\
y-z+2t=b_4
\end{cases}$


\item  \vskip .4cm  $\begin{cases}
 - x - y + 4 z = b_1\\
 x+3y+8z = b_2\\
 x + 2y + 5z = b_3
\end{cases}$

\end{enumerate}
\end{multicols}

\vskip .4cm

\item Sea $A=\begin{bmatrix}1 & 2 & 3 & \cdots & 2016 \\ 2 & 3 & 4 & \cdots & 2017 \\ 3&4&5& \cdots & 2018\\ \vdots & &&& \vdots \\ 100 & 101 & 102& \cdots& 2115\end{bmatrix}$.

\

\begin{enumerate}
   \item Encontrar todas las soluciones del sistema $AX=0$.
   \item Encontrar todas las soluciones del sistema $AX=\left[\begin{array}{c}
     1\\\vdots \\1 \end{array}\right]$.
\end{enumerate}



\vskip .4cm

     \item Sea $A=\begin{bmatrix}3 & -1 & 2 \\2 & 1 & 1 \\1&-3&0\end{bmatrix}$. Reduciendo $A$ por filas,
 \begin{enumerate}
   \item encontrar todas las soluciones sobre $\mathbb{R}$ y $\mathbb{C}$ del sistema $AX=0$.
   \item encontrar todas las soluciones sobre $\mathbb{R}$ y $\mathbb{C}$ del sistema $AX=\left[\begin{array}{c}
     1\\i\\0 \end{array}\right]$.
 \end{enumerate}
\vskip.2cm
\noindent\textsc{Solución:}

\begin{enumerate}
\item Reducimos la matriz A aplicando operaciones elementales por filas:

\begin{align*}
A = &\begin{bmatrix}3 & -1 & 2 \\2 & 1 & 1 \\1&-3&0\end{bmatrix}
\stackrel{F_1 \leftrightarrow F_3}{\longrightarrow}
\begin{bmatrix}1&-3&0\\2 & 1 & 1 \\3 & -1 & 2 \end{bmatrix}
\stackrel{F_2 - 2 F_1}{\stackrel{F_3 - 3 F_1}{\longrightarrow}}
\begin{bmatrix} 1 & -3 & 0 \\ 0 & 7 & 1 \\ 0 & 8 & 2 \end{bmatrix}
\stackrel{F_3-F_2}{\longrightarrow}
\begin{bmatrix} 1 & -3 & 0 \\ 0 & 7 & 1 \\ 0 & 1 & 1 \end{bmatrix}
\stackrel{F_3 \leftrightarrow F_2}{\longrightarrow} \\
&\begin{bmatrix} 1 & -3 & 0 \\ 0 & 1 & 1 \\ 0 & 7 & 1 \end{bmatrix}
\stackrel{F_1 + 3 F_2}{\stackrel{F_3-7F_2}{\longrightarrow}}
\begin{bmatrix} 1 & 0 & 3 \\ 0 & 1 & 1 \\ 0 & 0 & -6 \end{bmatrix}
\stackrel{F_3 (-\frac{1}{6}) }{\longrightarrow}
\begin{bmatrix} 1 & 0 & 3 \\ 0 & 1 & 1 \\ 0 & 0 & 1 \end{bmatrix}
\stackrel{F_1 - 3 F_3}{\stackrel{F_2 - F_3 }{\longrightarrow}}
\begin{bmatrix} 1 & 0 & 0 \\ 0 & 1 & 0 \\ 0 & 0 & 1 \end{bmatrix} = R_A
\end{align*}

Luego el sistema tiene solución trivial $(0,0,0)$ y la MERF es la matriz $\operatorname{Id}_3$. Notar que todas las operaciones realizadas valen tanto para $\mathbb{R}$ como para $\mathbb{C}$, por lo que $(0,0,0)$ es la solución para ambos casos.

\

\item Análogamente a lo realizado en el  ejercicio $(7.b)$, podemos repetir la secuencia de operaciones elementales sobre el vector $\left[\begin{array}{c}1\\i\\0 \end{array}\right]$ :

\begin{align*}
& \left[\begin{array}{c}1\\i\\0 \end{array}\right]
\stackrel{F_1 \leftrightarrow F_3}{\longrightarrow}
\left[\begin{array}{c}0\\i\\1 \end{array}\right]
\stackrel{F_2 - 2 F_1}{\stackrel{F_3 - 3 F_1}{\longrightarrow}}
\left[\begin{array}{c}0\\i\\1 \end{array}\right]
\stackrel{F_3-F_2}{\longrightarrow}
\left[\begin{array}{c}0\\i\\1-i \end{array}\right]
\stackrel{F_3 \leftrightarrow F_2}{\longrightarrow} 
\left[\begin{array}{c}0\\1-i\\i \end{array}\right]
\stackrel{F_1 + 3 F_2}{\stackrel{F_3-7F_2}{\longrightarrow}} \\
&\left[\begin{array}{c} 3 - 3i \\1-i\\ -7 + 8i \end{array}\right]
\stackrel{F_3 (-\frac{1}{6}) }{\longrightarrow}
\left[\begin{array}{c} 3 - 3i \\1-i\\ \frac{7}{6} - \frac{4}{3} i \end{array}\right]
\stackrel{F_1 - 3 F_3}{\stackrel{F_2 - F_3 }{\longrightarrow}}
\left[\begin{array}{c} -\frac{1}{2}+i \\ -\frac{1}{6}+\frac{1}{3}i \\ \frac{7}{6}-\frac{4}{3} i \end{array}\right]
\end{align*}

En este punto, si bien las operaciones propiamente dichas sólo involucraron números reales, tenemos que la solución tiene números complejos, por lo que el sistema no tiene solución en $\mathbb{R}$, pero si tiene solución en $\mathbb{C}$, y es $X = \left[\begin{array}{c} -\frac{1}{2}+i \\ -\frac{1}{6}+\frac{1}{3}i \\ \frac{7}{6}-\frac{4}{3} i \end{array}\right]$ 

\end{enumerate}

% \item >Para qué valores de $a$ el siguiente sistema tiene única o infinitas soluciones?
% \begin{align*}
% \left\{\begin{array}{l}ax-y+z=2 \\ x-y+z = 2  \\ 2x -2y + (2-a)z = 4a\end{array}\right.
% \end{align*}

\

\item Suponga que tiene que resolver un sistema de ecuaciones lineales homogéneo y que tras hacer algunas operaciones elementales por fila a la matriz asociada obtiene una matriz con la siguiente forma
\begin{align*}
\left(
\begin{array}{cccc}
a & * & * & *\\
0 & b & * & *\\
0 & 0 & c & *\\
0 & 0 & 0 & d
\end{array}
\right)
\end{align*}
donde $a,b,c,d\in\R$ y $*$ son algunos números reales.
?`Qué conclusiones puede inferir acerca del conjunto de soluciones a partir de los valores de $a$, $b$, $c$ y $d$?
\vskip.2cm
\noindent\textsc{Solución:}

Lo primero que podemos observar es que si $a$, $b$, $c$ y $d$ son todos no nulos entonces podemos aplicar las operaciones elementales por fila de multiplicar cada fila por $a^{-1}$, $b^{-1}$, $c^{-1}$ y $d^{-1}$. Luego de esto nos quedar\'ia una matriz con la siguiente forma
\begin{align*}
\left(
\begin{array}{cccc}
1 & * & * & *\\
0 & 1 & * & *\\
0 & 0 & 1 & *\\
0 & 0 & 0 & 1
\end{array}
\right).
\end{align*}
Luego podemos usando esos 1's principales podemos eliminar las entradas por encima de ellos obteniendo la matriz identindad
\begin{align*}
\left(
\begin{array}{cccc}
1 & 0 & 0 & 0\\
0 & 1 & 0 & 0\\
0 & 0 & 1 & 0\\
0 & 0 & 0 & 1
\end{array}
\right).
\end{align*}
En conclusión si $a$, $b$, $c$ y $d$ son todos no nulos podemos llegar mediante operaciones elementales por filas a la identidad y por lo tanto la \'unica soluci\'on del sistema es la trivial $(0,0,0,0)$, recordar el Teorema 2.4.5.

En cambio, si alguno de los escalares $a$, $b$, $c$ y $d$ es nulo, entonces no podemos obtener un 1 principal en su lugar. M\'as a\'un, la MERF a la que lleguemos tendr\'a una fila nula y por lo tanto el sistema tendr\'a infinitas soluciones, recordar el Teorema 2.4.2. 

Por ejemplo, si $d=0$ esto es claro pues la matriz ser\'ia
\begin{align*}
\left(
\begin{array}{cccc}
a & * & * & *\\
0 & b & * & *\\
0 & 0 & c & *\\
0 & 0 & 0 & 0
\end{array}
\right).
\end{align*}
Si $c=0$ y $d\neq0$, entonces la matriz es 
\begin{align*}
\left(
\begin{array}{cccc}
a & * & * & *\\
0 & b & * & *\\
0 & 0 & 0 & *\\
0 & 0 & 0 & d
\end{array}
\right).
\end{align*}
Luego, podemos multiplicar por $d^{-1}$ la \'ultima fila y luego anular la entrada por arriba del 1 que nos quede y as\'i obtener la matriz
\begin{align*}
\left(
\begin{array}{cccc}
a & * & * & *\\
0 & b & * & *\\
0 & 0 & 0 & 0\\
0 & 0 & 0 & 1
\end{array}
\right).
\end{align*}
Un razonamiento similar podr\'iamos hacer con las dem\'as posibilidades.

Moraleja: para saber si un sistema homog\'eneo tiene una o infinitas soluciones no es necesario reducir la matriz hasta llegar a una MERF basta con llegar a una triangular superior. Pero para calcular de forma param\'etrica el conjunto de soluciones si es necesario llegar a una MERF.

\

\item Suponga que tiene que resolver un sistema de ecuaciones lineales y que tras hacer algunas operaciones elementales por fila a la matriz ampliada obtiene una matriz con la siguiente forma
\begin{align*}
\left(
\begin{array}{cccc|c}
a & * & * & * & *\\
0 & b & * & * & *\\
0 & 0 & 0 & 0 & c\\
0 & 0 & 0 & d & *
\end{array}
\right)
\end{align*}
donde $a,b,c,d\in\R$ y $*$ son algunos números reales.
?`Qué conclusiones puede inferir acerca del conjunto de soluciones a partir de los valores de $a$, $b$, $c$ y $d$?
\vskip.2cm
\noindent\textsc{Solución:}

Lo primero que podemos notar es que si $c$ es no nulo el sistema no tiene soluci\'on. Pues sería equivalente a un sistema cuya ecuaci\'on $0=c$ es falsa.

Asumamos ahora que $c=0$. Si $a$, $b$ y $d$ son no nulos, entonces como antes podemos simplificarlos aplicando la operaci\'on elemental multiplicar la respectiva fila por $a^{-1}$, $b^{-1}$ y $d^{-1}$. Luego intercambiar la tercer y cuarta fila para obtener la matriz
\begin{align*}
\left(
\begin{array}{cccc|c}
1 & * & * & * & *\\
0 & 1 & * & * & *\\
0 & 0 & 0 & 1 & *\\
0 & 0 & 0 & 0 & 0
\end{array}
\right)
\end{align*}
Razonando como en el ejercicio anterior podemos transforma la matriz en una MERF que va a tener una fila nula. Adem\'as, este sistema tiene una soluci\'on. En efecto, para fijar ideas supongamos que $z_1$, $z_2$ y $z_3$ son las entradas de la \'ultima columna de la matriz ampliada entonces $(z_1,z_2,0,z_3)$ es una soluci\'on. Por el Teorema 2.4.2, en este caso el sistema tiene infinitas soluciones.

Hay otros varios casos para analizar de manera similar. 

Moraleja: al igual que antes no es necesario llegar a una MERF para saber si el sistema tendr\'a o no soluci\'on, una o infinitas. Pero para calcular de forma param\'etrica el conjunto de soluciones si es necesario llegar a una MERF.

\

\item Suponga que tiene que resolver un sistema de $m$ ecuaciones lineales con $n$ incógnitas. Antes de empezar a hacer cuentas y apelando a la teoría, ?`Qué puede afirmar acerca del conjunto de soluciones en base a $m$ y $n$? ?`Cómo saber si es vacío o no vacío? ?`Si tiene una o varias soluciones?
\vskip.2cm
\noindent\textsc{Solución:}

No hay una respuesta concluyente a este ejercicio pero nos sirve para pensar un poco y repasar la teor\'ia. Algunos razonamientos que podemos hacer son los siguientes.

Si el sistema tiene menos inc\'ognitas que ecuaciones hay chances de que no tenga soluci\'on. En cierto sentido, cada ecuaci\'on es una condici\'on para el conjunto de soluciones y entonces podr\'ia ser que estemos poniendo demasiadas condiciones y que sean contradictorias entre ellas y as\'i no habr\'ia una soluci\'on com\'un a todas (ver la página 29 de la Clase 08 Teórica - Sistemas de ecuaciones 3 (17-09-20) del turno ma\~nana).

Si el sistema tiene m\'as inc\'ognitas que ecuaciones y el sistema tiene soluci\'on entonces el sistema tiene infinitas soluciones. Esto es porque hay inc\'ognitas que no van a ser 1 principal y entonces ser\'ian variables libres (ver la p\'agina 35 de la Clase 08 Teórica - Sistemas de ecuaciones 3 (17-09-20) del turno ma\~nana).

\

\item\label{polinomios} $\textcircled{a}$ Sean $\lambda_1, ..., \lambda_n\in\R$ y $b_1, ..., b_n\in\R$.

\

\begin{enumerate}
 \item Para cada $n\in\{1,2,3,4,5\}$, plantear un sistema de ecuaciones lineales que le permita encontrar un polinomio $p(x)$ con coeficientes reales de grado $n-1$ tal que
 $$
 p(\lambda_1)=b_1, \dots, p(\lambda_n)=b_n.
 $$

 \

\item ?`Se le ocurre alguna condición con la cual pueda afirmar que el sistema anterior no tiene solución?

\

\item  ?`Puede dar una forma general del sistema para cualquier $n$?

\end{enumerate}


\vskip.2cm
\noindent\textsc{Solución:}
 (a) En el ejercicio (\ref{polinomio}) hicimos  $n=3$   para un caso concreto ($p(1)=2$, $p(2) = 7$ y $p(3)=14$). como en ese caso,  una forma de resolver el problema es plantear un sistema de ecuaciones donde los coeficientes del polinomio sean la incógnitas. 
 Sea 
 $$p_n(x) = a_0 + a_1x + a_2x^2 + \cdots a_{n-1}x^{n-1}, $$
 entonces, $p_n(\lambda_i) = b_i$ se traduce en la ecuación
 \begin{equation*}\label{eq-fila-vd}
 a_0 + a_1\lambda_i + a_2\lambda_i^2 + \cdots a_{n-1}\lambda_i^{n-1} = b_i.    
 \end{equation*}
 Las matrices ampliadas de los sistemas de ecuaciones para $n=4$ y $n=5$, son
 \begin{equation*}
     \left[\begin{array}{cccc|c}
     1 &\lambda_1&\lambda_1^2&\lambda_1^3 &b_1 \\
     1 &\lambda_2&\lambda_2^2&\lambda_2^3 &b_2 \\
     1 &\lambda_3&\lambda_3^2&\lambda_3^3 &b_3 \\
     1 &\lambda_4&\lambda_4^2&\lambda_4^3 &b_4
     \end{array}\right], \qquad
     \left[\begin{array}{ccccc|c}
     1 &\lambda_1&\lambda_1^2&\lambda_1^3 &\lambda_1^4 &b_1 \\
     1 &\lambda_2&\lambda_2^2&\lambda_2^3&\lambda_2^4  &b_2 \\
     1 &\lambda_3&\lambda_3^2&\lambda_3^3 &\lambda_3^4 &b_3 \\
     1 &\lambda_4&\lambda_4^2&\lambda_4^3&\lambda_4^4  &b_4\\
     1 &\lambda_5&\lambda_5^2&\lambda_5^3&\lambda_5^4  &b_5
     \end{array}\right], 
 \end{equation*}
 respectivamente. Para $n=1,2,3$ es claro como son los sistemas.
 
 \vskip .2cm
 
 (b) Si sobrentendemos  que todos los $\lambda_i$ son distintos entre si, la respuesta es \textit{no.} 
 \vskip .2cm
 Obviamente si $\lambda_i = \lambda_j$ y $b_i \ne b_j$,  entonces $p(\lambda_i) = b_i \ne b_j = p(\lambda_j) = p(\lambda_i)$,  es decir llegamos a la conclusión que $p(\lambda_i)  \ne p(\lambda_i)$, lo cual es absurdo. 
 
 (Veremos más adelante, usando determinantes,  que si $\lambda_i \ne \lambda_j$ para $i \ne j$,  entonces siempre encontraremos un polinomio que satisfaga las condiciones del ejercicio). 
  \vskip .2cm
 (c) No es difícil generalizar (a) para cualquier $n$: la matriz ampliada del sistema de ecuaciones  correspondiente al caso  $n$ es
 $$[V|Y] =
 \left[\begin{array}{ccccc|c}
     1 &\lambda_1&\lambda_1^2&\cdots &\lambda_1^{n-1} &b_1 \\
     1 &\lambda_2&\lambda_2^2&\cdots&\lambda_2^{n-1}  &b_2 \\
     \vdots & \vdots& \vdots& & \vdots & \vdots \\
    \vdots & \vdots& \vdots& & \vdots & \vdots \\
     1 &\lambda_n&\lambda_n^2&\cdots&\lambda_n^{n-1}  &b_n
     \end{array}\right].
 $$
 (La matriz $V$ es  llamada \textit{la matriz de Vandermonde.})
 
\end{enumerate}



\end{document}












