% PDFLaTeX
\documentclass[a4paper,12pt,twoside,spanish,reqno]{amsbook}
%%%---------------------------------------------------
\usepackage[math]{kurier}

\usepackage{etex}
\usepackage{t1enc}
\usepackage{latexsym}
\usepackage[utf8]{inputenc}
\usepackage{verbatim}
\usepackage{multicol}
\usepackage{amsgen,amsmath,amstext,amsbsy,amsopn,amsfonts,amssymb}
\usepackage{amsthm}
\usepackage{calc}         % From LaTeX distribution
\usepackage{graphicx}     % From LaTeX distribution
\usepackage{ifthen}
\input{random.tex}   
\usepackage{tikz}
\usetikzlibrary{arrows}
\usetikzlibrary{matrix}
\usepackage{mathtools}
\usepackage{stackrel}
\usepackage{enumitem}
\usepackage{tkz-graph}

\usepackage{enumitem} 
\usepackage[compatibility=false]{caption} % para usar subcaption
\usepackage{subcaption} % para poner varias imagenes juntas
\usetikzlibrary{arrows.meta}
\usepackage{hyperref}
\hypersetup{ 
    colorlinks=true,
    linkcolor=blue,
    filecolor=magenta,      
    urlcolor=cyan,
}
\usepackage{hypcap}
\numberwithin{equation}{section}
% http://www.texnia.com/archive/enumitem.pdf (para las labels de los enumerate)
\renewcommand\labelitemi{$\circ$}
\setlist[enumerate, 1]{label={(\arabic*)}}
\setlist[enumerate, 2]{label=\emph{\alph*)}}


\newcommand{\rta}{\noindent\textsc{Solución: }} 
\newcommand{\Id}{\operatorname{Id}}
\newcommand{\img}{\operatorname{Im}}
\newcommand{\nuc}{\operatorname{Nu}}
\newcommand\im{\operatorname{Im}}
\renewcommand\nu{\operatorname{Nu}}
\newcommand{\la}{\langle}
\newcommand{\ra}{\rangle}
\renewcommand{\t}{{\operatorname{t}}}
\renewcommand{\sin}{{\,\operatorname{sen}}}
\newcommand{\Q}{\mathbb Q}
\newcommand{\R}{\mathbb R}
\newcommand{\C}{\mathbb C}
\newcommand{\K}{\mathbb K}
\newcommand{\F}{\mathbb F}
\newcommand{\Z}{\mathbb Z}
\newcommand{\sen}{\operatorname{sen}}
\newenvironment{amatrix}[1]{%
  \left[\begin{array}{@{}*{#1}{c}|c@{}}
}{%
  \end{array}\right]
}

%%% FORMATOS %%%%%%%%%%%%%%%%%%%%%%%%%%%%%%%%%%%%%%%%%%%%%%%%%%%%%%%%%%%%%%%%%%%%%
\tolerance=10000
\renewcommand{\baselinestretch}{1.2}
\usepackage[a4paper, top=3cm, left=3cm, right=2cm, bottom=2.5cm]{geometry}
\usepackage{setspace}
%\setlength{\parindent}{0,7cm}% tamaño de sangria.
\setlength{\parskip}{0,4cm} % separación entre parrafos.
%\renewcommand{\baselinestretch}{0.90}% separacion del interlineado
\renewcommand{\chaptername}{Práctico}
%%%%%%%%%%%%%%%%%%%%%%%%%%%%%%%%%%%%%%%%%%%%%%%%%%%%%%%%%%%%%%%%%%%%%%%%%%%%%%%%%%%
%\end{comment}
%%% FIN FORMATOS  %%%%%%%%%%%%%%%%%%%%%%%%%%%%%%%%%%%%%%%%%%%%%%%%%%%%%%%%%%%%%%%%%

\begin{document}
    \baselineskip=0.55truecm %original

%%% CAP1 =================
    \input{practico-1.tex}
%%%=======================
%%%=======================
%%% CAP2 =================
    \chapter{Sistemas de ecuaciones\\Álgebra  II -- Año 2024/1 -- FAMAF}\label{practico-2}

\centerline{\textsc{}}

\subsection*{Objetivos}

\begin{itemize}
 \item Aprender a plantear y resolver sistemas de ecuaciones lineales.
\end{itemize}

\begin{enumerate}[topsep=6pt, itemsep=.4cm]
\item {\it Juego Suko}. Colocar los números del $1$ al $9$ en las celdas de la siguiente tabla de modo que el número en cada círculo sea igual a la suma de las cuatro celdas adyacentes, y la suma de las celdas del mismo color sea igual al número en el círculo de igual color.

\begin{center}
  \begin{tikzpicture}
    \draw [fill=gray!10] (0,0) rectangle (1,-1); 
    \draw [fill=gray!10] (1,0) rectangle (2,-1); 
    \draw [fill=gray!10] (2,0) rectangle (3,-1); 
    \draw [fill=gray!30] (0,-1) rectangle (1,-2); 
    \draw [fill=gray!30] (1,-1) rectangle (2,-2); 
    \draw [fill=gray!10] (2,-1) rectangle (3,-2); 
    \draw [fill=gray!80] (0,-2) rectangle (1,-3); 
    \draw [fill=gray!80] (1,-2) rectangle (2,-3); 
    \draw [fill=gray!80] (2,-2) rectangle (3,-3); 
    \filldraw[fill=white](1,-1) circle (0.33);
    \filldraw[fill=white](2,-1) circle (0.33);
    \filldraw[fill=white](1,-2) circle (0.33);
    \filldraw[fill=white](2,-2) circle (0.33);
    \node at (1,-1) {12};
    \node at (2,-1) {19};
    \node at (1,-2) {23};
    \node at (2,-2) {26};
    \filldraw[fill=gray!10](0.5,-3.5) circle (0.33);
    \filldraw[fill=gray!30](1.5,-3.5) circle (0.33);
    \filldraw[fill=gray!80](2.5,-3.5) circle (0.33);
    \node at (0.5,-3.5) {14};
    \node at (1.5,-3.5) {9};
    \node at (2.5,-3.5) {22};
    
    \end{tikzpicture}
\end{center}



\item\label{polinomio} Encontrar los coeficientes reales del polinomio $p(x) = ax^2+bx+c$ de manera tal que $p(1)=2$, $p(2)=7$ y $p(3)=14$.



\item Determinar cuáles de las siguientes matrices son MERF.
$$\begin{array}{lccccl}
\begin{bmatrix}1 & 2 & 0 \\0 & 0 & 1 \end{bmatrix}, &
\begin{bmatrix}1 & 0 & 2 \\0 & 1 & -3 \end{bmatrix}, &
\begin{bmatrix}0 & 1 & 0 \\0 & 0 & 1 \end{bmatrix}, &
\begin{bmatrix}0 & 1 & 0 \\0 & 0 & 0 \end{bmatrix}, &
\begin{bmatrix}1 & 0 & 0  \\0 & 0 & 1 \\0 & 0 & 1 \end{bmatrix},&
\begin{bmatrix}1 & 0 & 0  \\0 & 0 & 0 \\0 & 0 & 1 \end{bmatrix}.
\end{array}$$



\item Para cada una de las MERF del ejercicio anterior,

\begin{enumerate}
\item
asumir que es la matriz de un sistema homogéneo, escribir el sistema
y dar las soluciones del sistema.

\item
asumir que es la matriz ampliada de un sistema no homogéneo, escribir el sistema
y dar las soluciones del sistema.
\end{enumerate}



\item\label{sistemas homogeneos} Para cada uno de los siguientes sistemas de ecuaciones, describir explícita o paramétricamente todas las soluciones e indicar cuál es la MERF asociada al sistema.

\begin{multicols}{2}
\begin{enumerate}
\item $\begin{cases}
 -x - y + 4z = 0\\
 x+3y+8z = 0\\
 x+2y + 5z = 0
\end{cases}$
\item $\begin{cases}
 x - 3y + 5z = 0\\
 2x-3y+z = 0\\
 -y + 3z = 0
\end{cases}$

\item $\begin{cases}
x-z+2t = 0\\
-x+2y-z+2t = 0\\
-x+y = 0
\end{cases}$

\item[(d)] $\begin{cases}
 -x - y + 4z = 1\\
 x+3y+8z = 3\\
 x+2y + 5z = 1
\end{cases}$

\item[(e)] $\begin{cases}
 x - 3y + 5z = 1\\
 2x-3y+z = 3\\
 -y + 3z = 1
\end{cases}$


\item[(f)] $\begin{cases}
x-z+2t = 1\\
-x+2y-z+2t = 3\\
-x+y = 1
\end{cases}$

\end{enumerate}
\end{multicols}



\item\label{sistemas con soluciones} Para cada uno de los siguientes sistemas, describir implícitamente el conjunto de los vectores $(b_1,b_2,b_3)$
o $(b_1,b_2,b_3,b_4)$ para los cuales cada sistema tiene solución.
\begin{multicols}{2}
\begin{enumerate}

\item $\begin{cases}
 x - 3y + 5z = b_1\\
 2x-3y+z = b_2\\
 -y + 3z = b_3
\end{cases}$


\item
$\begin{cases}
x-z+2t = b_1\\
-x+2y-z+2t = b_2\\
-x+y = b_3\\
y-z+2t=b_4
\end{cases}$


\item  \vskip .4cm  $\begin{cases}
 - x - y + 4 z = b_1\\
 x+3y+8z = b_2\\
 x + 2y + 5z = b_3
\end{cases}$

\end{enumerate}
\end{multicols}



\item Sea $A=\begin{bmatrix}1 & 2 & 3 & \cdots & 2016 \\ 2 & 3 & 4 & \cdots & 2017 \\ 3&4&5& \cdots & 2018\\ \vdots & &&& \vdots \\ 100 & 101 & 102& \cdots& 2115\end{bmatrix}$.


\begin{enumerate}
   \item Encontrar todas las soluciones del sistema $AX=0$.
   \item Encontrar todas las soluciones del sistema $AX=\left[\begin{array}{c}
     1\\\vdots \\1 \end{array}\right]$.
\end{enumerate}



     \item Sea $A=\begin{bmatrix}3 & -1 & 2 \\2 & 1 & 1 \\1&-3&0\end{bmatrix}$. Reduciendo $A$ por filas,
 \begin{enumerate}
   \item encontrar todas las soluciones sobre $\mathbb{R}$ y $\mathbb{C}$ del sistema $AX=0$.
   \item encontrar todas las soluciones sobre $\mathbb{R}$ y $\mathbb{C}$ del sistema $AX=\left[\begin{array}{c}
     1\\i\\0 \end{array}\right]$.
 \end{enumerate}


\item Suponga que tiene que resolver un sistema de ecuaciones lineales homogéneo y que tras hacer algunas operaciones elementales por fila a la matriz asociada obtiene una matriz con la siguiente forma
\begin{align*}
\left(
\begin{array}{cccc}
a & * & * & *\\
0 & b & * & *\\
0 & 0 & c & *\\
0 & 0 & 0 & d
\end{array}
\right)
\end{align*}
donde $a,b,c,d\in\R$ y $*$ son algunos números reales.
¿Qué conclusiones puede inferir acerca del conjunto de soluciones a partir de los valores de $a$, $b$, $c$ y $d$?



\item Suponga que tiene que resolver un sistema de ecuaciones lineales y que tras hacer algunas operaciones elementales por fila a la matriz ampliada obtiene una matriz con la siguiente forma
\begin{align*}
\left(
\begin{array}{cccc|c}
a & * & * & * & *\\
0 & b & * & * & *\\
0 & 0 & 0 & 0 & c\\
0 & 0 & 0 & d & *
\end{array}
\right)
\end{align*}
donde $a,b,c,d\in\R$ y $*$ son algunos números reales.
¿Qué conclusiones puede inferir acerca del conjunto de soluciones a partir de los valores de $a$, $b$, $c$ y $d$?



\item\label{soluciones-ecuaciones} Suponga que tiene que resolver un sistema de $m$ ecuaciones lineales con $n$ incógnitas. Antes de empezar a hacer cuentas y apelando a la teoría, ¿Qué puede afirmar acerca del conjunto de soluciones en base a $m$ y $n$? ¿Cómo saber si es vacío o no vacío? ¿Si tiene una o varias soluciones?




\item\label{polinomios} $\textcircled{a}$ Sean $\lambda_1, ..., \lambda_n\in\R$ y $b_1, ..., b_n\in\R$.


\begin{enumerate}
 \item\label{polinomios-a} Para cada $n\in\{1,2,3,4,5\}$, plantear un sistema de ecuaciones lineales que le permita encontrar un polinomio $p(x)$ con coeficientes reales de grado $n-1$ tal que
 $$
 p(\lambda_1)=b_1, \dots, p(\lambda_n)=b_n.
 $$
\item\label{polinomios-b} ¿Se le ocurre alguna condición con la cual pueda afirmar que el sistema anterior no tiene solución?
\item\label{polinomios-c}  ¿Puede dar una forma general del sistema para cualquier $n$?
\end{enumerate}


\end{enumerate}















%%%=======================
%%%=======================
%%% CAP3 =================
    \chapter{Álgebra de matrices\\Álgebra  II -- Año 2024/1 -- FAMAF}\label{practico-3}
    



\subsection*{Objetivos}

\begin{itemize}
 \item Familiarizarse con las matrices y sus operaciones de suma y multiplicación, Ejercicios \ref{ej} -- \ref{ej:multiplicar por columna}.
 \vskip .2cm
 \item Familiarizarse con la notación de subíndices para las entradas de matrices, Ejercicios \ref{ej:multiplicar por columna} y \ref{traza}.
 \vskip .2cm
 \item Aprender la noción de matriz inversa y cómo cálcularla, Ejercicios \ref{ej:inversas} -- \ref{nilpotene - id}.
 \vskip .2cm
 \item Usar matrices para la resolución de sistemas de ecuaciones, Ejercicios \ref{sol homog es subesp} -- \ref{ej:sistemas ABX}.
\end{itemize}

\bigbreak

\subsection*{Ejercicios}

Los ejercicios con el símbolo \textcircled{a} tienen una ayuda al final del archivo para que recurran a ella después de pensar un poco.

\begin{enumerate}[topsep=6pt,itemsep=.4cm]

%%%%%%%%%%%%%%%%%%%%%%%%%%%%%%%%%%%%%%%%%%%%%%%%%%%%%%%%%%%%%%%%%%%%%%%%%%%%%%%%%%%%%%%%%%%%%%%%%%%%%%%%%%%%%%%%%%%%%%%%%%%%%%%%%%%%%%%%%%%%%%%%%%%%%%%%

\item\label{ej} Sean
$$
A= \begin{bmatrix} 1&-2&0\\ 1&-2&1\\ 1&-2&-1\end{bmatrix},\quad
\quad B= \begin{bmatrix}1&1&2\\ -2&0&-1\\ 1&3&5 \end{bmatrix},
\quad\quad C=\begin{bmatrix}1&-1&1\\ 2&0&1\\ 3&0&1 \end{bmatrix}.
$$

Verificar que $A(BC)=(AB)C$, es decir que vale la asociatividad del producto.


\item\label{ej2} Determinar cuál de las siguientes matrices es $A$, cuál es $B$ y cuál es $C$ de modo tal que sea posible realizar el producto $ABC$ y verificar que $A(BC)=(AB)C$.
\begin{equation*}
\begin{bmatrix} 2 & -1 & 1 \\ 1 & 2 &
1\end{bmatrix},\qquad
\begin{bmatrix} 3 \\ 1 \\ -1\end{bmatrix}, \qquad
\begin{bmatrix} 1 & -1 \end{bmatrix}.
\end{equation*}


\item Calcular $A^2$ y $A^3$ para la matriz \
$
A=\begin{bmatrix}
3 & 4\\ 6 & 8
\end{bmatrix}.
$


\item\label{ejemplos 2x2} \textcircled{a} Dar ejemplos de matrices no nulas $A$ y $B$ de orden $2\times2$ tales que
\begin{multicols}{2}
\begin{enumerate}[topsep=5pt,itemsep=5pt]
 \item $A^2=0$ (dar dos ejemplos).
 \item $AB\neq BA$.
 \item $A^2=-\operatorname{I}_2$.
 \item $A^2=A\neq\operatorname{I}_2$.
\end{enumerate}
\end{multicols}


\item\label{2x2 central} \textcircled{a} Sea $A \in\mathbb{R}^{2\times 2}$ tal que $AB=BA$ para toda $B\in\mathbb{R}^{2\times 2}$. Probar que $A$ es un múltiplo de $\operatorname{I}_2$.


\item  Para cada $n\in\mathbb{N}$, con $n\geq 2$, hallar una matriz no nula $A\in\mathbb{R}^{n\times n}$ tal que $A^n=0$ pero $A^{n-1}\neq0$.


\item\label{eq:binomio} \textcircled{a} Dar condiciones necesarias y suficientes sobre matrices $A$ y $B$ de tama\~{n}o $n\times n$ para que
\begin{multicols}{2}
    \begin{enumerate}
        \item $(A + B)^2 = A^2 + 2AB + B^2$.
        \item $A^2 - B^2 = (A - B)(A + B)$.
    \end{enumerate}
\end{multicols}

%%%%%%%%%%%%%%%%%%%%%%%%%%%%%%%%%%%%%%%%%%%%%%%%%%%%%%%%%%%%%%%%%%%%%%%%%%%%%%%%%%%%%%%%%%%%%%%%%%%%%%%%%%%%%%%%%%%%%%%%%%%%%%%%%%%%%%%%%%%%%%%%%%%%%%%%

\item\label{ej:multiplicar por columna} $\textcircled{a}$ Sean
\begin{align*}
v=\begin{bmatrix} v_1 \\ \vdots \\ v_n
\end{bmatrix}\in\mathbb{R}^{n\times1}
\quad\mbox{y}\quad A=\begin{bmatrix} \mid& \mid& &\mid\\ C_1 & C_2 & \cdots &C_n\\ \mid& \mid& &\mid\end{bmatrix}
\in\mathbb{R}^{m\times n},
\end{align*}
es decir, $C_1, ..., C_n$ denotan las columnas de $A$. Probar que $Av=\sum_{j=1}^nv_jC_j$.


\item\label{traza} Si $A$ es una matriz cuadrada $n\times n$, se define la {\it \textit{traza}} de $A$
como $\operatorname{Tr}(A)=\displaystyle{\sum_{i=1}^n} a_{ii}$.
\begin{enumerate}[topsep=5pt,itemsep=5pt]
 \item Calcular la traza de las matrices del ejercicio  \ref{ej:inversas}.
 \item\label{ej:traza}\textcircled{a} Probar que si $A,B\in\mathbb{R}^{n\times n}$ y $c\in\mathbb{R}$ entonces
 \begin{align*}
 \operatorname{Tr}(A+cB)=\operatorname{Tr}(A)+c\operatorname{Tr}(B)
 \quad\mbox{y}\quad
 \operatorname{Tr}(AB)=\operatorname{Tr}(BA).
 \end{align*}
\end{enumerate}

%%%%%%%%%%%%%%%%%%%%%%%%%%%%%%%%%%%%%%%%%%%%%%%%%%%%%%%%%%%%%%%%%%%%%%%%%%%%%%%%%%%%%%%%%%%%%%%%%%%%%%%%%%%%%%%%%%%%%%%%%%%%%%%%%%%%%%%%%%%%%%%%%%%%%%%%


\item\label{ej:inversas} Para cada una de las siguientes matrices, usar operaciones elementales
por fila para decidir si son invertibles y hallar la matriz inversa cuando sea posible.
\begin{equation*}
\begin{bmatrix} 3 & -1 & 2 \\ 2 & 1 & 1 \\ 1 & -3 & 0\end{bmatrix},\qquad
\begin{bmatrix} -1 & -1 &4 \\ 1 & 3 & 8 \\ 1 & 2 & 5\end{bmatrix},\qquad
\begin{bmatrix} 1 & 1 & 1 & 2 \\ 1 & -3 & 3 & -8 \\ -2 & 1 & 2 & -2 \\ 1 & 2 & 1 & 4 \end{bmatrix},\qquad
\begin{bmatrix} 1 & -3 & 5 \\ 2 & -3 & 1 \\ 0 & -1 & 3 \end{bmatrix}.
\end{equation*}
(para que hagan menos cuentas: las matrices $3\times3$ aparecieron en el Práctico \ref{practico-2}).


\item Sea $A$ la primera matriz del ejercicio anterior.
Hallar matrices elementales $E_1,E_2,\dots,E_k$ tales que $E_kE_{k-1}\cdots E_2E_1A=\operatorname{I}_3$.


\item\label{suma-de-invertibles} ¿Es cierto que si $A$ y $B$ son matrices invertibles entonces $A+B$ es una matriz invertible? Justificar su respuesta.

\item\label{nilpotene - id} \textcircled{a} Una matriz $A\in\mathbb{R}^{n\times n}$ se dice \emph{nilpotente} si $A^k=0$ para algún $k\in\mathbb{N}$.
Probar que si una matriz $A$ es nilpotente, entonces  $\operatorname{I}_n - A$  es invertible.

%%%%%%%%%%%%%%%%%%%%%%%%%%%%%%%%%%%%%%%%%%%%%%%%%%%%%%%%%%%%%%%%%%%%%%%%%%%%%%%%%%%%%%%%%%%%%%%%%%%%%%%%%%%%%%%%%%%%%%%%%%%%%%%%%%%%%%%%%%%%%%%%%%%%%%%%

\item\label{sol homog es subesp} Sean  $v$ y $w$ dos soluciones del sistema homogéneo $AX=0$. Probar que $v+tw$ también es solución para todo $t\in\mathbb{K}$.

\item Sea $v$ una solución del sistema $AX=Y$ y $w$ una solución del sistema homogéneo $AX=0$. Probar que $v+tw$ también es solución del sistema $AX=Y$ para todo $t\in\mathbb{K}$.

\item Probar que si el sistema homogéneo  $AX=0$ posee alguna solución no trivial, entonces el sistema $AX=Y$ no tiene
solución o tiene al menos dos soluciones distintas.

\item Supongamos que los sistemas $AX=Y$ y $AX=Z$ tienen solución. Probar que el sistema $AX=Y+tZ$ también tiene solución para todo $t\in\mathbb{K}$.

\item Sean $A$ una matriz invertible $n\times n$, y $B$ una matriz $n\times m$.  Probar que los sistemas $BX=Y$ y $ABX=AY$ tienen las mismas soluciones.

\item\label{ej:sistemas ABX} \textcircled{a}
Sean $A$ y $B$ matrices $r\times n$ y $n\times m$ respectivamente.
Probar que:
\begin{enumerate}[topsep=5pt,itemsep=5pt]
    \item  Si $m>n$, entonces el sistema $ABX=0$ tiene soluciones no triviales.
    \item  Si $r>n$, entonces existe un $Y$, $r\times 1$, tal que $ABX=Y$
    no tiene solución.
\end{enumerate}

%%%%%%%%%%%%%%%%%%%%%%%%%%%%%%%%%%%%%%%%%%%%%%%%%%%%%%%%%%%%%%%%%%%%%%%%%%%%%%%%%%%%%%%%%%%%%%%%%%%%%%%%%%%%%%%%%%%%%%%%%%%%%%%%%%%%%%%%%%%%%%%%%%%%%%%%

\end{enumerate}




\subsection*{Ejercicios de repaso}
Si ya hizo los ejercicios anteriores continue con la siguiente guía. Los ejercicios que siguen son similares y le pueden servir para practicar antes de los exámenes.



\begin{enumerate}[resume, topsep=6pt, itemsep=.4cm]


\item\label{ej: distributiva} \textcircled{a} Probar que si $A\in\mathbb{R}^{m\times n}$ y $B,C\in\mathbb{R}^{n\times p}$ entonces
$A(B+ C)=AB + AC$.

\item Probar que si $A,B\in\mathbb{R}^{m\times n}$ y $C\in\mathbb{R}^{n\times p}$ entonces
$(A+B)C = AC + BC$.

\item Sea $v=[v_1 \cdots v_m]\in\mathbb{R}^{1\times m}$ y $A\in\mathbb{R}^{m\times n}$. Probar que $vA=\sum_{i=1}^m v_iF_i$, donde $F_1, ..., F_m$ denotan las filas de $A$.

\item Sea $D=(d_{ij})\in\mathbb{R}^{n\times n}$ una matriz diagonal y $A=(a_{ij})\in\mathbb{R}^{m\times n}$. Probar que \\ $AD=(d_{jj}a_{ij})\in\mathbb{R}^{m\times n}$.

\item
Probar las siguientes afirmaciones:
\begin{enumerate}[topsep=5pt,itemsep=5pt]
\item Si $A,B\in\mathbb{R}^{n\times n}$ son matrices diagonales, entonces $AB=BA$.
\item Si $A=c \operatorname{I}_n$ para algún $c \in \mathbb{R}$, entonces $AB=BA$ para toda $B\in\mathbb{R}^{n\times n}$.
\end{enumerate}

\item Probar que si $A\in\mathbb{R}^{n\times n}$ es una matriz diagonal tal que $\operatorname{Tr} (A^2)=0$, entonces $A=0$.

\item Sea $A$ matriz $2\times 2$  tal que $\operatorname{Tr}(A)=0$ y $\operatorname{Tr}(A^2)=0$.

\begin{enumerate}[topsep=5pt,itemsep=5pt]
    \item  Probar que $A^2 = 0$.
    \item  ¿Es cierta la recíproca?
    %    \item  >Puede generalizar el resultado?
\end{enumerate}


\item Probar que si $A$ y $B$ son matrices $n \times n$ \emph{que conmutan entre s\' \i}, entonces para todo $k \in \mathbb N \cup \{0\}$ se cumple que:
$$(A+B)^k = \sum_{j = 0}^k \binom{k}{j} \, A^j \, B^{k-j}.$$

\item Sea $A\in\mathbb{R}^{m\times n}$. La \emph{matriz traspuesta} de $A$ es la matriz $A^t\in\mathbb{R}^{n\times m}$ tal que
$(A^t)_{ij}=A_{ji}$, $1\le i\le n$, $1\le j\le m$.
\begin{enumerate}[topsep=5pt,itemsep=5pt]
 \item Dar las matrices traspuestas de las matrices $A$, $B$ y $C$ de los ejercicios \ref{ej} y \ref{ej2}.
 \item Probar que si $A,B\in\mathbb{R}^{m\times n}$, $C\in\mathbb{R}^{n\times p}$ y $c\in\mathbb{R}$ entonces
$$(A+cB)^t=A^{t} + c B^t, \quad\quad (BC)^t= C^t B^t.$$

\item Probar que si $D\in\mathbb{R}^{n\times n}$ es invertible, entonces $D^t$ también lo es y $(D^{t})^{-1}=(D^{-1})^{t}$.
\end{enumerate}

\item Una matriz $A$ se dice {\it simétrica} si $A^t=A$. Una matriz $B$ se dice {\it antisimétrica} si $B^t=-B$. Probar que toda matriz se puede expresar como la suma de una matriz simétrica y una antisimétrica.

\item Decidir si las siguientes afirmaciones son verdaderas o falsas. Justificar.
\begin{enumerate}[topsep=5pt,itemsep=5pt]

\item Si $A$ y $B$ son matrices cuadradas tales $AB=BA$ pero ninguna es múltiplo de la otra, entonces $A$ o $B$ es diagonal.

\item Existen una matriz $3\times 2$, $A$,  y una matriz $2\times 3$, $B$, tales que $AB$ es una matriz invertible.

\item Existen una matriz $2\times 3$, $A$,  y una matriz $3\times 2$, $B$, tales que $AB$ es una matriz invertible.

\end{enumerate}

\end{enumerate}



\subsection*{Ayudas}

\vskip .4cm\

\ref{ejemplos 2x2} Probar con algunos $0$ y $1$ en las entradas.

\vskip .4cm

\ref{2x2 central} Elegir matrices $B$ apropiadas con muchos ceros y \textbf{un} $1$.

\vskip .4cm

\ref{eq:binomio} El objetivo del ejercicio es completar los puntos suspensivos en la siguiente frase:


``$(A+B)^2=A^2+2AB+B^2$ si y sólo si $A$ y $B$ satisfacen ..... ''

Desarrollen el cuadrado de la suma $A+B$ usando que el producto de matrices es distributivo y vean que les ``sobra'' para obtener la fórmula del binomio.  Misma idea para el item (b).

\vskip .4cm

\ref{ej:multiplicar por columna} Usar la notación de subíndices para las entradas de matrices.

\vskip .4cm

\ref{traza}\,\ref{ej:traza} Usar la notación de subíndices para las entradas de matrices.

\vskip .4cm

\ref{nilpotene - id} Pensar en la fórmula de $\sum_{i=0}^na^i$ vista en álgebra I/Matemática Discreta I.

\vskip .4cm

\ref{ej:sistemas ABX} Recordar el ejercicio \ref{soluciones-ecuaciones} del Práctico \ref{practico-2}.

\vskip .4cm

\ref{ej: distributiva} Usar la notación de subíndices para las entradas de matrices.


%%%=======================
%%%=======================
%%% CAP4 =================
    \chapter{Determinantes\\Álgebra  II -- Año 2024/1 -- FAMAF}\label{practico-4}
    


\subsection*{Objetivos.}

\begin{itemize}
\item Aprender a calcular el determinante de una matriz.

\item Aprender a utilizar operaciones elementales por filas y/o columnas para calcular el determinante.

\item Aplicar las propiedades del determinante para calcular el determinante de un producto de matrices, y para decidir si una matriz cuadrada es o no invertible.
\end{itemize}
        

\noindent \textbf{Ejercicios.} Los ejercicios con el símbolo \textcircled{a} tienen una ayuda al final del archivo para que recurran a ella después de pensar un poco.

\begin{enumerate}[topsep=6pt,itemsep=.4cm]
\item Calcular el determinante de las siguientes matrices.
    \begin{align*}
    &A=\begin{bmatrix} 4&7\\ 5&3\end{bmatrix},
    &&B=\begin{bmatrix} -3&2&4\\ 1&-1&2\\ -1&4&0\end{bmatrix},
    &&
    C=\begin{bmatrix} 2&3&1&1\\ 0&2&-1&3 \\ 0&5&1&1 \\1&1&2&5\end{bmatrix}.
    \end{align*}

    
\item Sean
        $$A=
    \begin{bmatrix}
        1&3&2 \\
        3&0&2 \\
        1&1&1
    \end{bmatrix}, \qquad
    B =
    \begin{bmatrix}
        1&-1&2\\
        1&1&1 \\
        -1&-1&3
    \end{bmatrix}.
    $$
    Calcular:
    \begin{multicols}{3}
    \begin{enumerate}
        \item $\det(AB)$.
        \item $\det(BA)$.
        \item $\det(A^{-1})$.
        \item $\det(A^{4})$.
        \item $\det(A+B)$.
        \item $\det(A+tB)$, con $t \in \mathbb{R}$.
    \end{enumerate}
\end{multicols}


\item Calcular el determinante de las siguientes matrices haciendo la reducción a matrices triangulares superiores.

        $$A =
        \begin{bmatrix}
            a&1&1&1 \\
            1&a&1&1 \\
            1&1&a&1 \\
            1&1&1&a \\
        \end{bmatrix}, \qquad    
        B =
        \begin{bmatrix}
            1&1&1&1&1 \\
            1&3&3&3&3 \\
            1&3&5&5&5 \\
            1&3&5&7&7 \\
            1&3&5&7&9 \\
        \end{bmatrix}.
        $$

\item Sean $A$, $B$ y $C$ matrices $n\times n$, tales que $\det A=-1$, $\det B=2$ y $\det C=3$.
Calcular:

\begin{enumerate}
\item $\det(PQR)$, donde $P$, $Q$ y $R$ son las matrices que se obtienen a partir de $A$, $B$ y $C$ mediante operaciones elementales por filas de la siguiente manera
 \begin{align*}
 A\overset{F_1+2F_2}{\longrightarrow} P,\quad
 B\overset{3F_3}{\longrightarrow} Q
 \quad\mbox{y}\quad
 C\overset{F_1\leftrightarrow F_4}{\longrightarrow} R.
 \end{align*}
 Es decir,
 \begin{itemize}
  \item[$\circ$] $P$ se obtiene a partir de $A$ sumando a la fila $1$ la fila $2$ multiplicada por $2$.
  \item[$\circ$] $Q$ se obtiene a partir de $B$ multiplicando la fila $3$ por $3$.
  \item[$\circ$] $R$ se obtiene a partir de $C$ intercambiando las filas $1$ y $4$.
 \end{itemize}
    \item $\det(A^2BC^tB^{-1})$ \ y \ $\det(B^2C^{-1}AB^{-1}C^{t})$.
\end{enumerate}


\item  Sea
$$A=
\begin{bmatrix}
    x&y&z \\
    3&0&2\\
    1&1&1
\end{bmatrix}.$$
Sabiendo que $\det(A) = 5$, calcular el determinante de las siguientes matrices.
$$
B = \begin{bmatrix}
2x&2y&2z \\
3/2&0&1\\
1&1&1
\end{bmatrix}, \qquad
C=
\begin{bmatrix}
    x&y&z \\
    3x+3&3y&3z+2\\
    x+1&y+1&z+1
\end{bmatrix}.
$$


\item Determinar todos los valores de $c\in\mathbb{R}$ tales que las siguientes matrices sean invertibles.
\begin{align*}
A=\begin{bmatrix}4& c&3\\c&2&c\\ 5&c&4 \end{bmatrix},\qquad
B=\begin{bmatrix} 1&c&-1\\ c&1&1\\0&1&c\end{bmatrix}.
\end{align*}


\item Calcular el determinante de las siguientes matrices, usando operaciones elementales por fila y/o columnas u otras propiedades del determinante. Determinar cuáles de ellas son invertibles.
\begin{equation*}
A=
\begin{bmatrix}-2&3&2&-6\\ 0&4&4&-5\\ 5&-6&-3&2\\ -3&7&0&0 \end{bmatrix},\quad
B=\begin{bmatrix} 2&0&0&0\\ 0&0&3&0\\ 0&-1&0&0\\ 0&0&0&4\end{bmatrix},\quad
\end{equation*}
\begin{equation*}
    C=\begin{bmatrix} -2&3&2&-6&0\\ 0&4&4&-5&0\\ 5&-6&-3&2&0\\ -3&7&0&0&0\\ 1&1&1&1&1\end{bmatrix},\quad
D=\begin{bmatrix}1&2&3&0&0\\-1&2&-13&6&\frac{1}{3}\\2&0&0&0&0\\11&1&0&0&0\\\sqrt{2}&2&1&\pi&0\end{bmatrix},
\end{equation*}
\begin{equation*}
E=\begin{bmatrix}1&-1&2&0&0\\ 3&1&4&0&0\\ 2&-1&5&0&0 \\0&0&0&2&1\\ 0&0&0&-1&4    \end{bmatrix}.
\end{equation*}

\item Sean $A$ y  $B$ matrices $n \times n$. Probar que:


\begin{enumerate}
    \item $\det(AB) = \det (BA)$.
    \item Si $B$ es invertible, entonces $\det(B A B^{-1}) = \det (A)$.
    \item\label{-A} $\textcircled{a}$ $\det(-A) = (-1)^n\det (A)$.
\end{enumerate}


\item\label{vandermonde} Sean $\lambda_1, \lambda_2, \dots, \lambda_n$ escalares, la matriz de \emph{Vandermonde} asociada es
\begin{align*}
\mathtt V = \begin{bmatrix}
1 & \lambda_1 & \lambda_1^2 & \cdots & \lambda_1^{n-1}\\
1 & \lambda_2 & \lambda_2^2 & \cdots & \lambda_2^{n-1}\\
\vdots &\vdots &\vdots & &\vdots\\
1 & \lambda_n & \lambda_n^2 & \cdots & \lambda_n^{n-1}\\
\end{bmatrix}.
\end{align*}
Esta es la matriz del sistema de ecuaciones del ejercicio \ref{polinomios}\,\ref{polinomios-c} del Práctico \ref{practico-2}.

\begin{enumerate}
    \item\label{vandermonde 2} Si $n=2$, probar que $\det(\mathtt V_n) = \lambda_2-\lambda_1$.
    \item\label{vandermonde 3} Si $n=3$, probar que $\det(\mathtt V_n) = (\lambda_3-\lambda_2) (\lambda_3-\lambda_1) (\lambda_2-\lambda_1)$.
    \item\label{vandermonde gral} Probar que $\det(\mathtt V_n) = \prod_{1\leq i< j \leq n}(\lambda_j-\lambda_i)$ para todo $n\in\mathbb{N}$.
    \item\label{vandermonde inv} Dar una condición necesaria y suficiente para que la matriz de Vandermonde sea invertible.
    \item\label{vandermonde sol} Dados $b_1, \ldots, b_n$  y $\lambda_1, \ldots, \lambda_n$ secuencias de números reales,  dar una condición suficiente para que exista un  polinomio de grado $n$, digamos $p$, tal que 
    $$
    p(\lambda_1)=b_1, \ldots, p(\lambda_n)=b_n.
    $$
    (ver ejercicio \ref{polinomios} del Práctico \ref{practico-2}).
\end{enumerate}


\item Decidir si las siguientes afirmaciones son verdaderas o falsas. Justificar con una demostración o con un contraejemplo, según corresponda.
    \begin{enumerate}
    \item Sean $A$ y $B$ matrices $n \times n$. Entonces $\det(A + B) = \det (A) + \det(B)$.
    \item Existen una matriz $3\times 2$, $A$, y una matriz $2\times 3$, $B$, tales que $\det(AB) \neq 0$.
    \item Sea $A$ una matriz $n\times n$. Si $A^n$ es no invertible, entonces $A$ es no invertible.
    \end{enumerate}

\end{enumerate}



\subsection*{Ejercicios de repaso}
Si ya hizo los ejercicios anteriores continue con la siguiente guía. Los ejercicios que siguen son similares y le pueden servir para practicar antes de los exámenes.


\begin{enumerate}[resume, topsep=6pt,itemsep=.4cm]
\item Determinar todos los valores de $c\in\mathbb{K}$ tales que la siguiente matriz sea invertible.
$$A=\begin{bmatrix} 0&c&-c\\ -1&2&-1\\c&-c&c\end{bmatrix}.$$


\item Sabiendo que \
$
\det \begin{bmatrix} a&b&c\\ p&q&r\\
x&y&z\end{bmatrix}=-1
$,
calcular \
$
\det \begin{bmatrix} -2a&-2b&-2c\\ 2p+x&2q+y&2r+z\\
3x&3y&3z\end{bmatrix}.
$

\item Probar que
$$
\det\begin{bmatrix}
1+x_1 & x_2 & x_3 & \cdots & x_n \\
x_1 & 1+x_2 & x_3 & \cdots & x_n \\
x_1 & x_2 & 1+x_3 & \cdots & x_n \\
\vdots & \vdots & \vdots &\ddots& \vdots \\
x_1 & x_2 & x_3 & \cdots & 1+x_n
\end{bmatrix}
= 1+x_1+x_2 + \cdots + x_n.
$$


\item
Una matriz $A$ $n \times n$ se dice {\it antisimétrica}
si $A^t=-A$.


\begin{enumerate}
\item\label{anti a} $\textcircled{a}$ Probar que si $n$ es impar y $A$ es antisimétrica, entonces
$\det(A)=0$.


\item\label{anti b} $\textcircled{a}$ Para cada $n$ par, encontrar una matriz $A$ antisimétrica
$n \times n$ tal que $\det(A)\not=0$.
\end{enumerate}
    
    
\end{enumerate}


\subsection*{Ayudas}

\

\ref{-A} Analizar primero los casos $n=2,3$.


\ref{vandermonde}\,\ref{vandermonde gral} En Wikipedia hay una posible demostración.


\ref{anti a} Usar el ejercicio \ref{-A}.


\ref{anti b} Encontrar primero una matriz $A_0$ para el caso $2\times 2$. Para $n = 2m$ considerar la matriz $2m \times 2m$ formada por $m$ bloques diagonales iguales a $A_0$.


%%%=======================
%%%=======================
%%% CAP5 =================
    \input{practico-5.tex}
%%%=======================
%%% CAP6 =================
    \chapter{Espacios y subespacios vectoriales \\Álgebra  II -- Año 2024/1 -- FAMAF}\label{practico-6}


    

\subsection*{Objetivos}
    
\begin{itemize}
\item Familiarizarse con los conceptos de espacio y subespacio vectorial.
\item Familiarizarse con los conceptos de conjunto de generadores e independencia lineal, base y dimensión de un espacio vectorial.
        
\item Aprender a caracterizar los subespacios de $\mathbb K^n$ por generadores y de manera implícita.

\item Dado un subespacio $W$ de $\mathbb K^n$, aprender a extraer una base de cualquier conjunto de generadores de $W$, y a completar cualquier subconjunto linealmente independiente de $W$ a una base.

\end{itemize}
    
    
\subsection*{Ejercicios} Los ejercicios con el símbolo $\textcircled{a}$ tienen una ayuda al final del archivo para que recurran a ella después de pensar un poco.

\begin{enumerate}[topsep=6pt, itemsep=.4cm]

    
% \item\label{func} Sea $F$ un cuerpo. Si $(V,\oplus,\odot)$ es un $F$-espacio vectorial y $S$ un conjunto cualquiera, entonces
% $$V^S=\{f:S\to V: \, f \ \text{es una función}\},$$
% denota al conjunto de todas las funciones de $S$ en $V$.  Definimos en $V^S$ la suma y el producto por escalares de la siguiente manera:
% Si $f,g \in V^S$ y $c\in F$ entonces $f + g: S \rightarrow V $ y $c\cdot f: S \rightarrow V$ están dadas por
% $$
% (f + g)(x) = f(x) \oplus g(x), \quad  (c\cdot f)(x) = c\odot f(x), \qquad  \forall\, x \,\in S.
% $$
% Probar que $(V^S,+,\cdot)$ es un $F$-espacio vectorial.


\item\label{sub Rn} Decidir si los siguientes subconjuntos de $\mathbb{R}^3$ son subespacios vectoriales.


    \begin{enumerate}
%         \item $\{(x_1, \ldots ,x_n) \in \mathbb{R}^n \ : \ x_1 = x_n\}$.
        \item\label{sub Rn 1} $A=\{(x_1, x_2 ,x_3) \in \mathbb{R}^3 \ : \ x_1 + x_2 + x_3=1\}$.
        \item\label{sub Rn 0} $B=\{(x_1, x_2 ,x_3) \in \mathbb{R}^3 \ : \ x_1 + x_2 + x_3=0\}$.
        \item\label{sub Rn geq} $C=\{(x_1, x_2 ,x_3) \in \mathbb{R}^3 \ : \ x_1 + x_2 + x_3 \geq 0\}$.
%         \item $\{(x_1, \ldots ,x_n) \in \mathbb{R}^n \ : \ x_n=1\}$.
        \item\label{sub Rn 1 30} $D=\{(x_1, x_2 ,x_3) \in \mathbb{R}^3 \ : \ x_3=0\}$.
        \item\label{sub Rn cup} $B\cup D$.
        \item\label{sub Rn cap} $B\cap D$.
        \item\label{sub Rn q} $G=\{(x_1, x_2 ,x_3) \in \mathbb{R}^3 \ :\ x_1, x_2, x_3\in\mathbb{Q}\}$.
    \end{enumerate}
\end{enumerate}

\textbf{Observación} En los items \ref{sub Rn 1}, \ref{sub Rn 0} y \ref{sub Rn geq} del ejercicio \ref{sub Rn} podemos apreciar como un simple cambio en la condición que define al subconjunto hace que dicho subconjunto sea o no un subespacio vectorial. Este es un fenómeno que pasa en general. De hecho podríamos haber definido subconjuntos similares para todo $\mathbb{R}^n$. Lo mismo sucede en los ejercicios \ref{sub funciones} y \ref{sub polinomios}.
En {\bf Ayudas},  al final del práctico, están las respuestas a los ejercicios \ref{sub Rn}, \ref{sub matrices} y \ref{sub funciones}. 
        
\begin{enumerate}[resume, topsep=6pt, itemsep=.4cm]

\item\label{sub matrices} Decidir en cada caso si el conjunto dado es un subespacio vectorial de $M_{n\times n}(\mathbb{K})$.
\begin{enumerate}
    \item\label{sub matrices invertibles} El conjunto de matrices  invertibles.
%     \item El conjunto de matrices no inversibles.
    \item\label{sub matrices AB} El conjunto de matrices $A$ tales que $AB = BA$, donde $B$ es una matriz fija.
%     \item El conjunto de matrices simétricas $\{A\in M_n(\mathbb{R})\ :\ A=A^t\}$
    \item\label{sub matrices triangulares} El conjunto de matrices triangulares superiores.
%     \item El conjunto de matrices de traza cero $\{A\in M_n(\mathbb{R})\ :\ \operatorname{tr}(A)=0\}$
%     \item $\{A\in M_n(\mathbb{R})\ :\ \operatorname{tr}(A)=1\}$
\end{enumerate}


\item\label{rectas} $\textcircled{a}$ Sea $L$ una recta en $\mathbb{R}^2$. Dar una condición necesaria y suficiente para que $L$ sea un subespacio vectorial de $\mathbb{R}^2$.


\item Sean $V$ un $\mathbb{K}$-espacio vectorial, $v\in V$ no nulo y $\lambda,\mu\in\mathbb{K}$ tales que $\lambda v=\mu v$. Probar que $\lambda=\mu$.


\item Sean $W_1, W_2$ subespacios de un espacio vectorial $V$. Probar que $W_1 \cup W_2$ es un subespacio
    de $V$ si y sólo si $W_1 \subseteq W_2$ o $W_2 \subseteq W_1$.
    

\item Sean $u=(1,1)$, $v=(1,0)$, $w=(0,1)$ y $z=(3,4)$ vectores de $\mathbb{R}^2$.
\begin{enumerate}
\item Escribir $z$ como combinación lineal de $u,v$ y $w$, con coeficientes todos no nulos.
\item Escribir $z$ como combinación lineal de $u$ y $v$.
\item Escribir $z$ como combinación lineal de $u$ y $w$.
\item Escribir $z$ como combinación lineal de $v$ y $w$.
\end{enumerate}

\vskip .4cm
\textbf{Observación.} En este ejercicio vemos como un vector se puede escribir de muchas maneras como combinación lineal de vectores dados. Esto pasa porque $\{u,v,w\}$ es LD.



\item Sean $p(x)=(1-x)(x+2)$, $q(x)=x^2-1$ y $r(x)=x(x^2-1)$ en $\mathbb{R}[x]$.

\begin{enumerate}
\item Describir en forma implícita  todos los polinomios de grado menor o igual que $3$ que son combinación lineal de $p,q$ y $r$.

\item Elegir $a$ tal que el polinomio $x$ se pueda escribir como combinación lineal de $p,q$ y $2x^2+a$.
 \end{enumerate}


\item\label{practicos anteriores} Dar un conjunto de generadores para los siguientes subespacios vectoriales.

\begin{enumerate}
\item Los conjuntos de soluciones de los sistemas homogéneos del ejercicio \ref{sistemas homogeneos} del Práctico \ref{practico-2}.
\item Los conjuntos descriptos en el ejercicio \ref{sistemas con soluciones} del Práctico  \ref{practico-2}.
\end{enumerate}


\item\label{caracterizar}  En cada caso, caracterizar con ecuaciones al subespacio vectorial dado por generadores.

\begin{enumerate}
\item ${\left\langle(1,0,3),(0,1,-2)\right\rangle}\subseteq \mathbb{R}^3$.
\item ${\left\langle(1,2,0,1),(0,-1,-1,0),(2,3,-1,4)\right\rangle}\subseteq \mathbb{R}^4$.
\end{enumerate}


\item\label{son LI} En cada caso, determinar si el subconjunto indicado es linealmente independiente.

\begin{enumerate}
    \item $\{ (1,0,-1), (1,2,1), (0,-3,2) \}\subseteq \mathbb{R}^3$.

    \vspace{0.2cm}

    \item $\left\{  \begin{bmatrix} 1 & 0 & 2 \\ 0 & -1 & -3 \\ \end{bmatrix}, \quad
    \begin{bmatrix} 1 & 0 & 1 \\ -2 & 1 & 0 \\ \end{bmatrix}, \quad
    \begin{bmatrix} 1 & 2 & 3 \\ 3 & 2 & 1 \\ \end{bmatrix} \right\}\subseteq M_{2\times 3}(\mathbb{R})$.
\end{enumerate}


\item Dar un ejemplo de un conjunto de 3 vectores en $\mathbb{R}^3$ que sean LD, y tales que dos cualesquiera de ellos sean LI.


\item  Probar que si $\alpha$, $\beta$ y $\gamma$ son vectores LI en el $\mathbb{R}$-espacio vectorial $V$, entonces $\alpha +\beta$, $\alpha +\gamma$ y $\beta +\gamma $ también son LI.


\item Extender, de ser posible, los siguientes conjuntos a una base de los respectivos espacios vectoriales.

\begin{enumerate}
    \item Los conjuntos del ejercicio \ref{son LI}.
    \item\label{10b} $\{ (1,2,0,0),(1,0,1,0) \}\subseteq\mathbb{R}^4$.
    \item\label{10c} $\{ (1,2,1,1),(1,0,1,1),(3,2,3,3)\}\subseteq\mathbb{R}^4$.
\end{enumerate}


\item Dar subespacios vectoriales $W_0$, $W_1$, $W_2$ y $W_3$ de $\mathbb{R}^3$ tales que $W_0\subset W_1\subset W_2\subset W_3$ y $\dim W_0=0$, $\dim W_1=1$, $\dim W_2=2$ y $\dim W_3=3$.


\item Sea $V$ un espacio vectorial de dimensión $n$ y $\mathcal{B}=\{v_1, ..., v_n\}$ una base de $V$.
\begin{enumerate}
 \item Probar que cualquier subconjunto no vacío de $\mathcal{B}$ es LI.
 \item Para cada $k\in\mathbb{N}_0$,  con $0\leq k\leq n$, dar un subespacio vectorial de $V$ de dimensión $k$.
\end{enumerate}


\item Dar una base y calcular la dimensión de $\mathbb{C}^n$ como $\mathbb{C}$-espacio vectorial y como $\mathbb{R}$-espacio vectorial.


\item  Exhibir una base y calcular la dimensión de los siguientes subespacios.

 \begin{enumerate}
    \item Los subespacios del ejercicio \ref{practicos anteriores}.
    \item $W = \{(x,y,z,w,u) \in \mathbb{R}^5 \ : \ y = x - z,\, w = x + z,\,  u = 2x - 3z \}$.
    \item $W = \langle (1, 0, -1, 1),  (1, 2, 1, 1), (0, 1, 1, 0), (0, -2, -2, 0) \rangle \subseteq \mathbb R^4$.
    \item Matrices triangulares superiores $2\times 2$ y $3\times 3$.
    \item Matrices triangulares superiores $n\times n$ para cualquier $n\in\mathbb{N}$, $n\geq 2$.
\end{enumerate}

\item Sean $W_1$ y $W_2$ los siguientes subespacios de $\mathbb{R}^3$:
    \begin{align*}
    W_1 &= \{ (x,y,z)\in\mathbb{R}^3\ : \ x+y-2z=0\},  \\
    W_2 &= {\left\langle(1,-1,1),(2,1,-2),(3,0,-1)\right\rangle}.
    \end{align*}
    \begin{enumerate}
        \item  Determinar $W_1 \cap W_2$, y describirlo por generadores y con ecuaciones.
        \item  Determinar $W_1+W_2$, y describirlo por generadores y con ecuaciones.
        %\item  ?`Es la suma $W_1+W_2$ directa?
%        \item  Dar un complemento de $W_1$.
%        \item  Dar un complemento de $W_2$.
    \end{enumerate}


\item\label{verdadero o falso} Decidir si las siguientes afirmaciones son verdaderas o falsas. Justificar.

\begin{enumerate}
 \item Si $W_1$ y $W_2$ son subespacios vectoriales de $\mathbb{K}^8$ de dimensión $5$, entonces $W_1\cap W_2=0$.

  \item Si $W$ es un subespacio de $\mathbb{K}^{2\times2}$ de dimensión $2$, entonces existe una matriz triangular superior no nula que pertence a $W$.

 \item Sean $v_1, v_2, w\in \mathbb{K}^{n}$ y $A\in\mathbb{K}^{n\times n}$ tales que $Av_1=Av_2=0\neq Aw$. Si $\{v_1, v_2\}$ es LI, entonces $\{v_1,v_2,w\}$ también es LI.

\item\label{cos} $\textcircled{a}$ $\{1,{\rm sen}(x),\cos(x)\}$ es un subconjunto LI del espacio de funciones de $\mathbb{R}$ en $\mathbb{R}$.

\item\label{cos2} $\textcircled{a}$ $\{1,{\rm sen}^2(x),\cos^2(x)\}$ es un subconjunto LI del espacio de funciones $\mathbb{R}$ en $\mathbb{R}$.

\item\label{exponencial} $\textcircled{a}$ $\{e^{\lambda_1x},e^{\lambda_2x},e^{\lambda_3x}\}$ es un subconjunto LI del espacio de funciones de
$\mathbb{R}$ en $\mathbb{R}$, si $\lambda_1$, $\lambda_2$ y $\lambda_3$ son todos distintos.
\end{enumerate}

% \
%
% \item Sean $W_b$ y $W_c$ los subespacios de $\mathbb{R}^4$ generados por los subconjunto dados en los ejercicios \ref{10b} y \ref{10c}. Decidir si $W_b+W_c$ es una suma directa. En caso de que no lo sea dar un complemento de $W_b$ y otro para $W_c$.



\end{enumerate}


\textbf{Ejercicios de repaso}
Si ya hizo los ejercicios anteriores continue con la siguiente guía. Los ejercicios que siguen son similares y le pueden servir para practicar antes de los exámenes.


\begin{enumerate}[resume, topsep=6pt, itemsep=.4cm]

\item Decidir en cada caso si el conjunto dado es un subespacio vectorial de $\mathbb{R}^{n}$.
    \begin{enumerate}
        \item $\{(x_1, \ldots ,x_n) \in \mathbb{R}^n \ : \ \exists \, j > 1, \, x_1 = x_j\}$.
        \item $\{(x_1, \ldots , x_n) \in\mathbb{R}^n \ : \ x_1x_n = 0 \}$.
    \end{enumerate}


\item\label{sub funciones} Sea $F[0,1]$ el espacio de funciones de $[0,1]$ en $\mathbb{R}$. Decidir en cada caso si el conjunto dado es un subespacio vectorial de $F[0,1]$.
    \begin{enumerate}
%     \item $C[0,1] = \{ f : [0,1] \rightarrow \mathbb{R} \ : \ f \ \text{es continua}\}$.
%             \item $C^1[0,1] = \{ f : [0,1] \rightarrow \mathbb{R} \ : \ f \ \text{es  derivable}\}$.
% \item $\{ f : [0,1] \rightarrow \mathbb{R} \ : \ f \ \text{es  derivable y }f'=0\}$.
%                     \item $\{ f \in C[0,1] \ : \ f(1) \geq 0\}$.
    \item\label{sub funciones 1} $\{ f \in F[0,1] \ : \ f(1) = 1 \}$.
    \item\label{sub funciones 0} $\{ f \in F[0,1] \ : \ f(1) = 0\}$.
% %             \item $F=\{f \in C[0,1] \ : \ f(1) = 0\}$.
%     \item $E\cup F$
%     \item $E\cap F$
\end{enumerate}
    
    \item\label{sub polinomios} Decidir si los siguientes subconjuntos de $\mathbb{R}[x]$ son subespacios vectoriales.

    \begin{enumerate}
     \item $\mathbb{R}_{n}[x] := \{ a_0 + \cdots + a_{n-1}x^{n-1} \ : \ a_i \in \mathbb{R}\}$, es decir, el conjunto formado por todos los polinomios de grado estrictamente menor que $n\in\mathbb{N}$.

     \item $B=\{p(x)\in\mathbb{R}_{n}[x] \ : \ a_0 + \cdots + a_{n-1} = 1\}$.
     \item $C=\{p(x)\in\mathbb{R}_{n}[x] \ : \ a_0 + \cdots + a_{n-1} = 0\}$.
     \item $D=\{p(x)\in\mathbb{R}_{n}[x] \ : \ a_{n-1} \le a_{n-2}\}$.
%         \item $\{(x_1, \ldots ,x_n) \in \mathbb{R}^n \ : \ x_n=1\}$.
     \item $E=\{p(x)\in\mathbb{R}_{n}[x] \ : \ a_{n-1}=0\}$.
     \item $C\cup E$.
     \item $C\cap E$.
     \item $F=\{p(x)\in\mathbb{R}_{n}[x] \ : \  a_0, ..., a_{n-1}\in\mathbb{Q}\}$.
%     \item El conjunto $\mathbb{R}_{par}[x]$ formado por los polinomios de grado par, junto
%     con el polinomio nulo.
%     
%     \item $\mathbb{R}_{< n}[x]\cup \mathbb{R}_{par}[x]$
%
%     \item $\mathbb{R}_{< n}[x]\cap \mathbb{R}_{par}[x]$
     \end{enumerate}


\item  Hallar $a, b, c\in \mathbb{R}$ tales que $(-1,2,1)=a(1,1,1)+b(1,-1,0)+c(2,1,-1)$.


\item
\begin{enumerate}
    \item Hallar escalares $a, b \in \mathbb R$ tales que $1+2i=a(1+i)+b(1-i)$.
    \item  Hallar escalares $w, z \in \mathbb C$ tales que $1+2i=z(1+i)+w(1-i)$.
\end{enumerate}
%
% \item Sean $u=(-1,1)$, $v=(i,i)$, $w=(2,-i)$ y $z=(1,1+i)$.
%         \begin{enumerate}
%             \item Escribir $z$ como combinación lineal de $u,v$ y $w$, con coeficientes todos no nulos.
%             \item Escribir $z$ como combinación lineal de $u$ y $v$.
%             \item Escribir $z$ como combinación lineal de $u$ y $w$.
%             \item Escribir $z$ como combinación lineal de $v$ y $w$.
%         \end{enumerate}


\item  Repetir el ejercicio \ref{son LI} con los subespacios:

\begin{enumerate}
    \item ${\left\langle(1,1,0,0),(0,1,1,0),(0,0,1,1)\right\rangle}\subseteq \mathbb{R}^4$.
    \item ${\left\langle 1+x+x^2,\, x-x^2+x^3,\, 1-x,\, 1-x^2,\, x-x^2,\, 1+x^4\right\rangle}\subseteq \mathbb{R}[x]$.
\end{enumerate}


\item En este ejercicio no es necesario hacer ninguna cuenta. Es lógica y comprender bien la definición de LI y LD. Probar las siguientes afirmaciones.
\begin{enumerate}
\item Todo conjunto que contiene un subconjunto LD es también LD.
\item Todo conjunto que contiene al vector 0 es LD.
\item Un conjunto es LI si y sólo si todos sus subconjuntos \emph{finitos} son LI.
\end{enumerate}


\item Sean $\lambda_1, ..., \lambda_n\in\mathbb{R}$ todos distintos. Probar que el conjunto $\{e^{\lambda_1x}, ..., e^{\lambda_nx}\}$ es LI.


\item  Exhibir una base y calcular la dimensión de los siguientes subespacios.

\begin{enumerate}
    \item $W=\{(x,y,z) \in \mathbb{R}^3 \ : \ z = x + y \}$.
    \item $W = \langle (-1, 1, 1, -1, 1),  (0, 0, 1, 0, 0), (2, -1, 0, 2, -1), (1, 0, 1, 1, 0) \rangle \subseteq \mathbb R^5$.
\end{enumerate}


\item Exhibir una base y calcular la dimensión de los siguientes subespacios.
\begin{enumerate}
    \item $W = \{ p(x)=a+bx+cx^2+dx^3\in \mathbb{R}_{4}[x] \ : \ a+d=b+c \}$.
\item $W= \{ p(x)\in \mathbb{R}_{4}[x] \ : \ p'(0)=0 \}$.
    \item $W = \{A \in \mathbb{R}^{n\times n} \ : \ A = A^t\}$.
%     \item $S = \{A \in \mathbb{C}^{n\times n} \ : \ A = \bar{A^t}\}$ (considerado como $\mathbb{R}$-subespacio de $\mathbb{C}^{n\times n}$).
\end{enumerate}

    \item Sea  $S=\{v_1,v_2,v_3,v_4\}\subset\mathbb R^4$, donde
$$v_1=(-1,0,1,2), \quad v_2=(3,4,-2,5), \quad v_3=(0,4,1,11), \quad v_4=(1,4,0,9).$$
\begin{enumerate}
    \item  Describir implícitamente al subespacio  $W= \langle \, S\, \rangle$.
    \item Si $W_1 = \langle \, v_1,v_2,v_3+v_4\, \rangle $ y $W_2 = \langle \, v_3,v_4\, \rangle $,
    describir $W_1\cap W_2$ implícitamente.
\end{enumerate}


\item\label{matrices} Sean
    $
    A_1=\begin{bmatrix}
    1&-2&0&3&7\\
    2&1&-3&1&1
    \end{bmatrix}$ y $A_2=\begin{bmatrix}
    3&2&0&0&3\\
    1&0&-3&1&0 \\
    -1&1&-3&1&-2
    \end{bmatrix}
    $.
    
    \begin{enumerate}
    \item Sean $W_1$ y $W_2$ los espacios solución de los sistemas
    homogéneos asociados a $A_1$ y $A_2$, respectivamente.  Describir implícitamente $W_1\cap W_2$.
    \item Sean $V_1$ y $V_2$ los subespacios de $\mathbb{R}^5$ generado por las filas de $A_1$ y $A_2$, respectivamente. Dar un conjunto de generadores de $V_1+V_2$.
    \end{enumerate}


\item\label{todo} Sean $W_1$ y $W_2$ los siguientes subespacios de $\mathbb{R}^6$:
    \begin{align*}
    W_1 &= \{ (u,v,w,x,y,z)\ : \ u+v+w=0,\, x+y+z=0\},  \\
    W_2 &= \left\langle{(1,-1,1,-1,1,-1),(1,2,3,4,5,6),(1,0,-1,-1,0,1),(2,1,0,0,0,0)}\right\rangle.
    \end{align*}
    \begin{enumerate}
        \item  Determinar $W_1 \cap W_2$, y describirlo por generadores y con ecuaciones.
        \item  Determinar $W_1+W_2$, y describirlo por generadores y con ecuaciones.
        \item  Decir cuáles de los siguientes vectores están en $W_1\cap W_2$ y cuáles en $W_1+W_2$:
        \[ (1,1,-2,-2,1,1),\ (0,0,0,1,0,-1),\ (1,1,1,0,0,0),\ (3,0,0,1,1,3),\ (-1,2,5,6,5,4). \]
        \item Para los vectores $v$ del punto anterior que estén en $W_1+W_2$,  hallar $w_1\in W_1$ y $w_2\in W_2$ tales que $v=w_1+w_2$.
%         \item  ?`Es la suma $W_1+W_2$ directa?
% \item  Dar un complemento de $W_1$.
% \item  Dar un complemento de $W_2$.

    \end{enumerate}


\end{enumerate}

% \textbf{Ejercicios un poco más difíciles}
%
% Si ya hizo los primeros ejercicios ya sabe lo que tiene que saber. Los siguientes ejercicios le pueden servir si esta muy aburridx con la cuarentena.
%
%
% \begin{enumerate}[resume]
% \item Decidir si los siguientes conjuntos son $\mathbb{R}$-espacios vectoriales, con las operaciones abajo definidas.
%
% \
%
% \begin{enumerate}
% \item $\mathbb{R}^n$, con $v\oplus w = v - w$, y el producto por escalares usual.
%         
% \item $\mathbb{R}^2$, con $(x,y)\oplus(x_1,y_2) = (x + x_1, 0), \,\,c\odot(x,y) = (cx,0)$.
%
% \item $\mathbb{R}^{3}$, con:
%         \begin{align*}
%         (x,y,z)\oplus(x',y',z') &=(x + x', y + y' - 1, z + z');\\
%         c\odot(x,y,z) &= (cx,cy + 1 - c, cz).
%         \end{align*}
%         \item El conjunto de polinomios, con el producto por escalares (reales) usual, pero con suma
%         $p(x)\oplus q(x) = p'(x) + q' (x)$ (suma de derivadas).
% \end{enumerate}
%\end{enumerate}

\textbf{Ayudas}

Ejercicio \ref{sub Rn}: 
\ref{sub Rn 1} No. \ref{sub Rn 0} Si. \ref{sub Rn geq} No. \ref{sub Rn 1 30} Si. \ref{sub Rn cup} No. \ref{sub Rn cap} Si. \ref{sub Rn q} No.


Ejercicio \ref{sub matrices}\,\ref{sub matrices invertibles} No; recordar el ejercicio \ref{suma-de-invertibles} del Práctico  \ref{practico-3}. \ref{sub matrices AB} Si. \ref{sub matrices triangulares} Si.

Ejercicio \ref{rectas} Recordar el ejercicio \ref{rectas-por-el-0} del Práctico  \ref{practico-1}.

Ejercicio \ref{verdadero o falso}\, \ref{cos} Verdadero. Plantear una combinación lineal de las funciones que de igual a cero y evaluar en diferentes valores de $x$ para obtener alguna condición sobre los escalares.

Ejercicio \ref{verdadero o falso}\, \ref{cos2} Falso. Utilizar una igualdad trigonométrica.

Ejercicio \ref{verdadero o falso}\, \ref{exponencial} Verdadero. Plantear una combinación lineal de las funciones que de igual a cero. Derivar dos veces la igualdad obteniendo así dos nuevas combinaciones lineales que den cero. Evaluar en cero las tres combinaciones lineales y utilizar la matriz de Vandermonde.

Ejercicio \ref{sub funciones}: \ref{sub funciones 1} No. \ref{sub funciones 0} Si.


%%%======================= 
%%% CAP7 =================
    \chapter{Transformaciones lineales \\ Álgebra  II -- Año 2024/1 -- FAMAF}\label{practico-7}


\subsection*{Objetivos}

\begin{itemize}
 \item Familiarizarse con las transformaciones lineales.
 \item Aprender a decidir si un función es una transformación lineal, monomorfismos, epimorfismo o isomorfismo.
 \item Aprender a calcular la matriz de una transformaci\'con respecto a las bases canónicas.
 \item Aprender a calcular el núcleo y la imagen de una transformación.

 \item Familiarizarse con el teorema sobre la dimensión del núcleo y la imagen.
\end{itemize}



\subsection*{Ejercicios} Los ejercicios con el símbolo \textcircled{a} tienen una ayuda al final del archivo para que recurran a ella después de pensar un poco.

\begin{enumerate}[topsep=6pt, itemsep=.4cm]
\item Decidir si las siguientes funciones son transformaciones lineales entre los respectivos espacios vectoriales sobre $\mathbb{K}$.
\begin{enumerate}[resume, topsep=5pt,itemsep=5pt]
 \item La traza $\operatorname{Tr}:\mathbb{K}^{n\times n}\longrightarrow\mathbb{K}$ (recordar ejercicio \ref{traza}\,\ref{ej:traza} del Práctico  \ref{practico-3}) 
 \item $T:\mathbb{K}[x]\longrightarrow\mathbb{K}[x]$, $T(p(x))=q(x)\,p(x)$ donde $q(x)$ es un polinomio fijo.
 \item $T:\mathbb{K}^2\longrightarrow\mathbb{K}$, $T(x,y)=xy$
 \item $T:\mathbb{K}^2\longrightarrow\mathbb{K}^3$, $T(x,y)=(x,y,1)$
 \item El determinante $\operatorname{det}:\mathbb{K}^{n\times n}\longrightarrow\mathbb{K}$.
\end{enumerate}


\item Sea $T:\mathbb{C}\longrightarrow\mathbb{C}$, $T(z)=\overline{z}$.
\begin{enumerate}
 \item Considerar a $\mathbb{C}$ como un $\mathbb{C}$-espacio vectorial y decidir si $T$ es una transformación lineal.
 \item Considerar a $\mathbb{C}$ como un $\mathbb{R}$-espacio vectorial y decidir si $T$ es una transformación lineal.
\end{enumerate}


\item\label{T en la base} Sea $T:\mathbb{K}^3\longrightarrow\mathbb{K}^3$ una transformación lineal tal que $T(e_1)=(1,2,3)$, $T(e_2)=(-1,0,5)$ y $T(e_3)=(-2,3,1)$. 
    \begin{enumerate}
     \item Calcular $T(2,3,8)$ y $T(0,1,-1)$. 
     \item\label{T en la base b} Calcular $T(x,y,z)$ para todo $(x,y,z)\in\mathbb{K}^3$. Es decir, dar una fórmula para $T$ como la del ejercicio \ref{Txyz}.
     \item\label{matriz otro}  Encontrar una matriz $A\in\mathbb{K}^{3\times3}$ tal que $T(x,y,z)=A\begin{bmatrix}  x\\y\\z \end{bmatrix}$. En esta parte del ejercicio escribiremos/pensaremos a los vectores de $\mathbb{K}^3$ como columnas.
    \end{enumerate}

    \vskip .3cm

\textbf{Observación.} 
En el ejercicio \ref{T en la base}\,\ref{T en la base b} lo que hicimos fue deducir cuánto vale la transformación lineal en todos los vectores de $\mathbb{K}^3$ a partir de saber cuánto vale la transformación lineal en la base canónica. A partir del valor de $T$ en una base vectores podemos saber el valor de $T$ en todo el espacio. Esto vale para cualquier transformación lineal entre espacios vectoriales y cualquier base porque las transformaciones lineales respetan combinaciones lineales y todo vector de un espacio vectorial es combinación lineal de los vectores de una base.

\vskip .3cm

\textbf{Observación.} La matriz del ejercicio  \ref{T en la base}\,\ref{matriz otro} es la matriz de la transformación lineal $T$ con respecto a la base canónica. En el próximo práctico aprenderemos a calcular la matriz de una transformación lineal 
    con respecto a distintas bases.

    
    
    \item\label{Txyz} Sea $T:\mathbb{K}^3\longrightarrow\mathbb{K}^3$ definida por $T(x,y,z)=(x+2y+3z, y-z,x+5y)$.
        \begin{enumerate}
        \item\label{Txyz-vectores-nucleo} Decir cuáles de los siguientes vectores están en el núcleo: $(1,1,1)$, $(-5,1,1)$.
        \item\label{Txyz imagen} Decir cuáles de los siguientes vectores están en la imagen: $(0,1,0)$, $(0,1,3)$.
        \item\label{Txyz nucleo T implicito} Describir mediante ecuaciones (implícitamente) el núcleo de $T$.
        \item\label{Txyz imagen T generadores} Dar un conjunto de generadores de la imagen.
        \item\label{matriz de T} Encontrar una matriz  $A\in\mathbb{K}^{3\times 3}$ tal que $T(x,y,z)=A\begin{bmatrix}
            x\\y\\z \end{bmatrix}
            $.  Como en el ejercicio  \ref{T en la base}\,\ref{matriz otro} pensamos a los vectores como columnas.
        \end{enumerate}


\item Sea $T: \mathbb{K}^4 \to \mathbb{K}^5$ dada por $T(v) = Av$ donde $A$ es la siguiente matriz
    $$
    A=\begin{bmatrix}
    0& 2& 0&1\\   1& 3& 0&1\\  -1&-1&0&0\\3&0&3&0\\2&1&1&0 \end{bmatrix}
    $$
    \begin{enumerate}[topsep=5pt,itemsep=5pt]
        \item Dar una base del núcleo y de la imagen de $T$. 
        \item Dar la dimensión del núcleo y de la imagen de $T$.
        \item Describir mediante ecuaciones (implícitamente) el núcleo y la imagen de $T$.
        \item Decir cuáles de los siguientes vectores están en el núcleo:
        $(1,2,3,4)$, $(1,-1,-1,2)$, $(1,0,2,1)$.
        \item Decir cuáles de los siguientes vectores están en la imagen:
        $(2,3,-1,0,1)$, $(1,1,0,3,1)$, $(1,0,2,1,0)$.
    \end{enumerate}
    

\item Sea $T:\mathbb{K}^{2\times 2}\longrightarrow\mathbb{K}_{4}[x]$ la transformación lineal definida por
    \begin{align*}
    T   \begin{bmatrix}  a&b\\c&d \end{bmatrix} &= (a-c+2d)x^3+(b+2c-d)x^2+ \\
    &\qquad+(-a+2b+5c-4d)x+(2a-b-4c+5d)
    \end{align*}
    \begin{enumerate}
        \item Decir cuáles de los siguientes matrices están en el núcleo:
            \begin{align*}
                A=\begin{bmatrix}
                    2&0\\0&-1
                \end{bmatrix},
            \quad
            B=\begin{bmatrix}
                -1&-1\\1&1
            \end{bmatrix},
            \quad
            C=\begin{bmatrix}
                -1&-1\\1&0
            \end{bmatrix}.
            \end{align*}

        \item Decir cuáles de los siguientes polinomios están en la imagen:
            \begin{align*}
                p(x)=x^3+x^2+x+1,\quad q(x)=x^3, \quad r(x)=(x-1)(x-1) 
            \end{align*}
    \end{enumerate}



\item\label{funcional ej}  Sea $T:\mathbb{K}^3\longrightarrow\mathbb{K}$ definida por $T(x,y,z)=x+2y+3z$.
\begin{enumerate}
    \item Probar que $T$ es un epimorfismo.
    \item Dar la dimensión del núcleo de $T$.
    \item Encontrar una matriz $A$ tal que
        $T(x,y,z)=A\begin{bmatrix}
        x\\y\\z \end{bmatrix}$. ¿De qué tamaño debe ser $A$? Como en el ejercicio \ref{Txyz}\, \ref{matriz} pensamos a los vectores como columnas. 
\end{enumerate}


\item Determinar cuáles transformaciones lineales de los ejercicios anteriores son monomorfismos, epimorfismos y/o isomorfismos.


\item\label{usar-1} Encontrar en cada caso, cuando sea posible, una matriz $A\in\mathbb{K}^{3\times 3}$ tal que la transformación lineal $T:\mathbb{K}^3\longrightarrow\mathbb{K}^3$, $T(v)=Av$, satisfaga las condiciones exigidas (como en el ejercicio  \ref{T en la base}\,\ref{matriz otro} pensamos a los vectores como columnas). Cuando no sea posible, explicar por qué no es posible.
\begin{enumerate}[ topsep=5pt,itemsep=5pt]
    \item $\operatorname{dim} \operatorname{Im}(T)=2$ y $\operatorname{dim}\operatorname{Nu}(T)=2$.
    \item $T$ inyectiva y $T(e_1)=(1,0,0)$, $T(e_2)=(2,1,5)$ y $T(e_3)=(3,-1,0)$.
    \item $T$ sobreyectiva y $T(e_1)=(1,0,0)$, $T(e_2)=(2,1,5)$ y $T(e_3)=(3,-1,0)$.
    
    \item\label{usar Txyz} \textcircled{a} $T(e_1)=(1,0,0)$, $T(e_2)=(2,1,5)$ y $T(e_3)=(3,-1,0)$.
    
    \item $e_1\in\operatorname{Im}(T)$ y $(-5,1,1)\in\operatorname{Nu}(T)$.
    
    \item $\operatorname{dim} \operatorname{Im}(T)=2$.
\end{enumerate}
    

\item Decidir si las siguientes afirmaciones son verdaderas o falsas. Justificar.
\begin{enumerate}
    \item  Si $T : \mathbb R^{13} \to \mathbb R^9$ es una transformación lineal, entonces $\dim \operatorname{Nu}(T) \geq  4$.
    \item Sea $T:\mathbb{K}^{6}\longrightarrow\mathbb{K}^2$ un epimorfismo y $W$ un subespacio de $\mathbb{K}^{6}$ con $\dim W=3$. Entonces existe $0\neq w\in W$ tal que $T(w)=0$.
    \item Existe una transformación lineal $T : \mathbb R^2 \to \mathbb R^4$ tal que los vectores $(1, 0, -1, 2)$, $(0, 1, 2,-1,)$ y $(0, 0, 2, 2)$ pertenecen a la imagen de $T$.
\end{enumerate}

\item \label{funcionales} \textcircled{a} Sea $V$ un espacio vectorial no nulo y $T:V\longrightarrow\mathbb{K}$ probar que $T=0$ ó $T$ es sobreyectiva.

\item Sea $V$ un espacio vectorial de dimensión finita y $T:V\longrightarrow V$ una transformación lineal. Probar las siguientes afirmaciones.
    \begin{multicols}{2}
        \begin{enumerate}
            \item $\operatorname{Nu}(T)\subseteq\operatorname{Nu}(T^2)$
            \item\label{dimV impar} $\operatorname{Nu}(T)\neq\operatorname{Im}(T)$ si $\dim(V)$ es impar.
        \end{enumerate}
    \end{multicols}



\end{enumerate}


\subsection*{Ejercicios de repaso}
Si ya hizo los ejercicios anteriores continue con la siguiente guía. Los ejercicios que siguen son similares y le pueden servir para practicar antes de los exámenes.

\begin{enumerate}[resume, topsep=5pt,itemsep=.4cm]
  \item Sea $T: \mathbb{K}^3\longrightarrow\mathbb{K}[x]$ una transformación lineal tal que $T(e_1)=x^2+2x+3$, $T(e_2)=-x^2+5$ y $T(e_3)=-2x^2+3x+1$. Calcular $T(2,3,8)$ y $T(0,1,-1)$. Más generalmente, calcular $T(a,b,c)$ para todo $(a,b,c)\in\mathbb{K}^3$.
  
  \item Repetir los ejercicios \ref{Txyz}\,\ref{Txyz nucleo} y \ref{Txyz}\,\ref{matriz} con las siguientes transformaciones lineales.
\begin{enumerate}[topsep=5pt,itemsep=5pt]
 \item $T:\mathbb{K}^3\longrightarrow\mathbb{K}^3$, $T(x,y,z)=(x+2y+3z, y-z,0)$.
 \item $T:\mathbb{K}^2 \longrightarrow \mathbb{K}^3$, \ $T(x,y)=(x-y,x+y,2x+3y)$.
\end{enumerate}    


\item Decidir si las siguientes afirmaciones son verdaderas o falsas. Justificar.

\begin{enumerate}
\item Existe una transformación lineal $T : \mathbb R^3 \to \mathbb R^2$ tal que $T(1, 0,-1) = (1, -1)$ y $T(-1, 0, 1) = (1, 0)$.
\item Existe una transformación lineal $T : \mathbb R^3 \to \mathbb R^2$ tal que $T(1, 0,-1) = (1, -1)$ y $T(-1, 0, 1) = (-1, 1)$.
\item  Si $T : \mathbb R^9 \to \mathbb R^7$ es una transformación lineal, entonces $\dim \operatorname{Nu}(T) \geq  2$.
\item Sea $T : V \to W$ una transformación lineal tal que $T(v_i) = w_i$, para $i = 1, \dots , n$. Si $\{w_1, \dots , w_n\}$ genera $W$, entonces
$\{v_1, \dots , v_n\}$ genera $V$.
\item Existe una transformación lineal $T : \mathbb R^2 \to \mathbb R^5$ tal que los vectores $(1, 0, -1, 0, 0)$, $(1, 1, -1, 0, 0)$ y $(1, 0, -1, 2, 1)$ pertenecen a la imagen de $T$.
\item Existe una transformación lineal sobreyectiva $T : \mathbb R^5 \to \mathbb R^4$ tal que los vectores $(1, 0, 1, -1, 0)$ y $(0, 0, 0, -1, 2)$
pertenecen al núcleo de $T$.
\end{enumerate}

 

\end{enumerate}

\subsection*{Ayudas}

\

    Ejercicio \ref{Txyz}\,\ref{matriz}: recordar en el ejercicio \ref{ej:multiplicar por columna} de la Práctica 3 como podemos interpretar el producto de una matriz por un vector columna. 

    Ejercicio \ref{funcionales}: usar como inspiración el ejercicio \ref{funcional ej} que es un caso particular de esta situación.

    Ejercicio \ref{usar-1}\,\ref{usar Txyz} Usar el ejercicio \ref{Txyz}.

    Ejercicio \ref{funcionales}: asumir lo contrario y usar el Teorema de la dimensión del núcleo y la imagen.


%%%=======================
\end{document}