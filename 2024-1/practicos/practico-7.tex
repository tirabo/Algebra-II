\chapter{Transformaciones lineales \\ Álgebra  II -- Año 2024/1 -- FAMAF}\label{practico-7}


\subsection*{Objetivos}

\begin{itemize}
 \item Familiarizarse con las transformaciones lineales.
 \item Aprender a decidir si un función es una transformación lineal, monomorfismos, epimorfismo o isomorfismo.
 \item Aprender a calcular la matriz de una transformaci\'con respecto a las bases canónicas.
 \item Aprender a calcular el núcleo y la imagen de una transformación.

 \item Familiarizarse con el teorema sobre la dimensión del núcleo y la imagen.
\end{itemize}



\subsection*{Ejercicios} Los ejercicios con el símbolo \textcircled{a} tienen una ayuda al final del archivo para que recurran a ella después de pensar un poco.

\begin{enumerate}[topsep=6pt, itemsep=.4cm]
\item Decidir si las siguientes funciones son transformaciones lineales entre los respectivos espacios vectoriales sobre $\mathbb{K}$.
\begin{enumerate}[resume, topsep=5pt,itemsep=5pt]
 \item La traza $\operatorname{Tr}:\mathbb{K}^{n\times n}\longrightarrow\mathbb{K}$ (recordar ejercicio \ref{traza}\,\ref{ej:traza} del Práctico  \ref{practico-3}) 
 \item $T:\mathbb{K}[x]\longrightarrow\mathbb{K}[x]$, $T(p(x))=q(x)\,p(x)$ donde $q(x)$ es un polinomio fijo.
 \item $T:\mathbb{K}^2\longrightarrow\mathbb{K}$, $T(x,y)=xy$
 \item $T:\mathbb{K}^2\longrightarrow\mathbb{K}^3$, $T(x,y)=(x,y,1)$
 \item El determinante $\operatorname{det}:\mathbb{K}^{n\times n}\longrightarrow\mathbb{K}$.
\end{enumerate}


\item Sea $T:\mathbb{C}\longrightarrow\mathbb{C}$, $T(z)=\overline{z}$.
\begin{enumerate}
 \item Considerar a $\mathbb{C}$ como un $\mathbb{C}$-espacio vectorial y decidir si $T$ es una transformación lineal.
 \item Considerar a $\mathbb{C}$ como un $\mathbb{R}$-espacio vectorial y decidir si $T$ es una transformación lineal.
\end{enumerate}


\item\label{T en la base} Sea $T:\mathbb{K}^3\longrightarrow\mathbb{K}^3$ una transformación lineal tal que $T(e_1)=(1,2,3)$, $T(e_2)=(-1,0,5)$ y $T(e_3)=(-2,3,1)$. 
    \begin{enumerate}
     \item Calcular $T(2,3,8)$ y $T(0,1,-1)$. 
     \item\label{T en la base b} Calcular $T(x,y,z)$ para todo $(x,y,z)\in\mathbb{K}^3$. Es decir, dar una fórmula para $T$ como la del ejercicio \ref{Txyz}.
     \item\label{matriz otro}  Encontrar una matriz $A\in\mathbb{K}^{3\times3}$ tal que $T(x,y,z)=A\begin{bmatrix}  x\\y\\z \end{bmatrix}$. En esta parte del ejercicio escribiremos/pensaremos a los vectores de $\mathbb{K}^3$ como columnas.
    \end{enumerate}

    \vskip .3cm

\textbf{Observación.} 
En el ejercicio \ref{T en la base}\,\ref{T en la base b} lo que hicimos fue deducir cuánto vale la transformación lineal en todos los vectores de $\mathbb{K}^3$ a partir de saber cuánto vale la transformación lineal en la base canónica. A partir del valor de $T$ en una base vectores podemos saber el valor de $T$ en todo el espacio. Esto vale para cualquier transformación lineal entre espacios vectoriales y cualquier base porque las transformaciones lineales respetan combinaciones lineales y todo vector de un espacio vectorial es combinación lineal de los vectores de una base.

\vskip .3cm

\textbf{Observación.} La matriz del ejercicio  \ref{T en la base}\,\ref{matriz otro} es la matriz de la transformación lineal $T$ con respecto a la base canónica. En el próximo práctico aprenderemos a calcular la matriz de una transformación lineal 
    con respecto a distintas bases.

    
    
    \item \label{lineales1} Para cada una de las siguientes transformaciones lineales calcular el núcleo y la imagen. Describir ambos subespacios implícitamente y encontrar una base de cada uno de ellos.	\begin{enumerate}
		\item\label{lineales1-a} $T:\R^2 \longrightarrow \R^3$, $T(x,y)=(x-y,x+y,2x+3y)$.
		\item\label{lineales1-b} $S:\R^3 \longrightarrow \R^2$, $S(x,y,z)=(x-y+z,2x-y+2z)$.
	\end{enumerate}

    
    \item\label{Txyz} Sea $T:\mathbb{K}^3\longrightarrow\mathbb{K}^3$ definida por $T(x,y,z)=(x+2y+3z, y-z,x+5y)$.
        \begin{enumerate}
        \item\label{Txyz-vectores-nucleo} Decir cuáles de los siguientes vectores están en el núcleo: $(1,1,1)$, $(-5,1,1)$.
        \item\label{Txyz imagen} Decir cuáles de los siguientes vectores están en la imagen: $(0,1,0)$, $(0,1,3)$.
        \item\label{Txyz nucleo T implicito} Describir mediante ecuaciones (implícitamente) el núcleo de $T$.
        \item\label{Txyz imagen T generadores} Dar un conjunto de generadores de la imagen.
        \item\label{matriz de T} Encontrar una matriz  $A\in\mathbb{K}^{3\times 3}$ tal que $T(x,y,z)=A\begin{bmatrix}
            x\\y\\z \end{bmatrix}
            $.  Como en el ejercicio  \ref{T en la base}\,\ref{matriz otro} pensamos a los vectores como columnas.
        \end{enumerate}


\item Sea $T: \mathbb{K}^4 \to \mathbb{K}^5$ dada por $T(v) = Av$ donde $A$ es la siguiente matriz
    $$
    A=\begin{bmatrix}
    0& 2& 0&1\\   1& 3& 0&1\\  -1&-1&0&0\\3&0&3&0\\2&1&1&0 \end{bmatrix}
    $$
    \begin{enumerate}[topsep=5pt,itemsep=5pt]
        \item Dar una base del núcleo y de la imagen de $T$. 
        \item Dar la dimensión del núcleo y de la imagen de $T$.
        \item Describir mediante ecuaciones (implícitamente) el núcleo y la imagen de $T$.
        \item Decir cuáles de los siguientes vectores están en el núcleo:
        $(1,2,3,4)$, $(1,-1,-1,2)$, $(1,0,2,1)$.
        \item Decir cuáles de los siguientes vectores están en la imagen:
        $(2,3,-1,0,1)$, $(1,1,0,3,1)$, $(1,0,2,1,0)$.
    \end{enumerate}
    

\item Sea $T:\mathbb{K}^{2\times 2}\longrightarrow\mathbb{K}_{4}[x]$ la transformación lineal definida por
    \begin{align*}
    T   \begin{bmatrix}  a&b\\c&d \end{bmatrix} &= (a-c+2d)x^3+(b+2c-d)x^2+ \\
    &\qquad+(-a+2b+5c-4d)x+(2a-b-4c+5d)
    \end{align*}
    \begin{enumerate}
        \item Decir cuáles de los siguientes matrices están en el núcleo:
            \begin{align*}
                A=\begin{bmatrix}
                    2&0\\0&-1
                \end{bmatrix},
            \quad
            B=\begin{bmatrix}
                -1&-1\\1&1
            \end{bmatrix},
            \quad
            C=\begin{bmatrix}
                -1&-1\\1&0
            \end{bmatrix}.
            \end{align*}

        \item Decir cuáles de los siguientes polinomios están en la imagen:
            \begin{align*}
                p(x)=x^3+x^2+x+1,\quad q(x)=x^3, \quad r(x)=(x-1)(x-1) 
            \end{align*}
    \end{enumerate}



\item\label{funcional ej}  Sea $T:\mathbb{K}^3\longrightarrow\mathbb{K}$ definida por $T(x,y,z)=x+2y+3z$.
\begin{enumerate}
    \item Probar que $T$ es un epimorfismo.
    \item Dar la dimensión del núcleo de $T$.
    \item Encontrar una matriz $A$ tal que
        $T(x,y,z)=A\begin{bmatrix}
        x\\y\\z \end{bmatrix}$. ¿De qué tamaño debe ser $A$? Como en el ejercicio \ref{Txyz}\, \ref{matriz} pensamos a los vectores como columnas. 
\end{enumerate}


\item Determinar cuáles transformaciones lineales de los ejercicios anteriores son monomorfismos, epimorfismos y/o isomorfismos.


\item\label{usar-1} Encontrar en cada caso, cuando sea posible, una matriz $A\in\mathbb{K}^{3\times 3}$ tal que la transformación lineal $T:\mathbb{K}^3\longrightarrow\mathbb{K}^3$, $T(v)=Av$, satisfaga las condiciones exigidas (como en el ejercicio  \ref{T en la base}\,\ref{matriz otro} pensamos a los vectores como columnas). Cuando no sea posible, explicar por qué no es posible.
\begin{enumerate}[ topsep=5pt,itemsep=5pt]
    \item $\operatorname{dim} \operatorname{Im}(T)=2$ y $\operatorname{dim}\operatorname{Nu}(T)=2$.
    \item $T$ inyectiva y $T(e_1)=(1,0,0)$, $T(e_2)=(2,1,5)$ y $T(e_3)=(3,-1,0)$.
    \item $T$ sobreyectiva y $T(e_1)=(1,0,0)$, $T(e_2)=(2,1,5)$ y $T(e_3)=(3,-1,0)$.
    
    \item\label{usar Txyz} \textcircled{a} $T(e_1)=(1,0,0)$, $T(e_2)=(2,1,5)$ y $T(e_3)=(3,-1,0)$.
    
    \item $e_1\in\operatorname{Im}(T)$ y $(-5,1,1)\in\operatorname{Nu}(T)$.
    
    \item $\operatorname{dim} \operatorname{Im}(T)=2$.
\end{enumerate}
    

\item Decidir si las siguientes afirmaciones son verdaderas o falsas. Justificar.
\begin{enumerate}
    \item  Si $T : \mathbb R^{13} \to \mathbb R^9$ es una transformación lineal, entonces $\dim \operatorname{Nu}(T) \geq  4$.
    \item Sea $T:\mathbb{K}^{6}\longrightarrow\mathbb{K}^2$ un epimorfismo y $W$ un subespacio de $\mathbb{K}^{6}$ con $\dim W=3$. Entonces existe $0\neq w\in W$ tal que $T(w)=0$.
    \item Existe una transformación lineal $T : \mathbb R^2 \to \mathbb R^4$ tal que los vectores $(1, 0, -1, 2)$, $(0, 1, 2,-1,)$ y $(0, 0, 2, 2)$ pertenecen a la imagen de $T$.
\end{enumerate}

\item \label{funcionales} \textcircled{a} Sea $V$ un espacio vectorial no nulo y $T:V\longrightarrow\mathbb{K}$ probar que $T=0$ ó $T$ es sobreyectiva.

\item Sea $V$ un espacio vectorial de dimensión finita y $T:V\longrightarrow V$ una transformación lineal. Probar las siguientes afirmaciones.
    \begin{multicols}{2}
        \begin{enumerate}
            \item $\operatorname{Nu}(T)\subseteq\operatorname{Nu}(T^2)$
            \item\label{dimV impar} $\operatorname{Nu}(T)\neq\operatorname{Im}(T)$ si $\dim(V)$ es impar.
        \end{enumerate}
    \end{multicols}



\end{enumerate}


\subsection*{Ejercicios de repaso}
Si ya hizo los ejercicios anteriores continue con la siguiente guía. Los ejercicios que siguen son similares y le pueden servir para practicar antes de los exámenes.

\begin{enumerate}[resume, topsep=5pt,itemsep=.4cm]
  \item Sea $T: \mathbb{K}^3\longrightarrow\mathbb{K}[x]$ una transformación lineal tal que $T(e_1)=x^2+2x+3$, $T(e_2)=-x^2+5$ y $T(e_3)=-2x^2+3x+1$. Calcular $T(2,3,8)$ y $T(0,1,-1)$. Más generalmente, calcular $T(a,b,c)$ para todo $(a,b,c)\in\mathbb{K}^3$.
  
  \item Repetir los ejercicios \ref{Txyz}\,\ref{Txyz nucleo} y \ref{Txyz}\,\ref{matriz} con las siguientes transformaciones lineales.
\begin{enumerate}[topsep=5pt,itemsep=5pt]
 \item $T:\mathbb{K}^3\longrightarrow\mathbb{K}^3$, $T(x,y,z)=(x+2y+3z, y-z,0)$.
 \item $T:\mathbb{K}^2 \longrightarrow \mathbb{K}^3$, \ $T(x,y)=(x-y,x+y,2x+3y)$.
\end{enumerate}    


\item Decidir si las siguientes afirmaciones son verdaderas o falsas. Justificar.

\begin{enumerate}
\item Existe una transformación lineal $T : \mathbb R^3 \to \mathbb R^2$ tal que $T(1, 0,-1) = (1, -1)$ y $T(-1, 0, 1) = (1, 0)$.
\item Existe una transformación lineal $T : \mathbb R^3 \to \mathbb R^2$ tal que $T(1, 0,-1) = (1, -1)$ y $T(-1, 0, 1) = (-1, 1)$.
\item  Si $T : \mathbb R^9 \to \mathbb R^7$ es una transformación lineal, entonces $\dim \operatorname{Nu}(T) \geq  2$.
\item Sea $T : V \to W$ una transformación lineal tal que $T(v_i) = w_i$, para $i = 1, \dots , n$. Si $\{w_1, \dots , w_n\}$ genera $W$, entonces
$\{v_1, \dots , v_n\}$ genera $V$.
\item Existe una transformación lineal $T : \mathbb R^2 \to \mathbb R^5$ tal que los vectores $(1, 0, -1, 0, 0)$, $(1, 1, -1, 0, 0)$ y $(1, 0, -1, 2, 1)$ pertenecen a la imagen de $T$.
\item Existe una transformación lineal sobreyectiva $T : \mathbb R^5 \to \mathbb R^4$ tal que los vectores $(1, 0, 1, -1, 0)$ y $(0, 0, 0, -1, 2)$
pertenecen al núcleo de $T$.
\end{enumerate}

 

\end{enumerate}

\subsection*{Ayudas}

\

    Ejercicio \ref{Txyz}\,\ref{matriz}: recordar en el ejercicio \ref{ej:multiplicar por columna} de la Práctica 3 como podemos interpretar el producto de una matriz por un vector columna. 

    Ejercicio \ref{funcionales}: usar como inspiración el ejercicio \ref{funcional ej} que es un caso particular de esta situación.

    Ejercicio \ref{usar-1}\,\ref{usar Txyz} Usar el ejercicio \ref{Txyz}.

    Ejercicio \ref{funcionales}: asumir lo contrario y usar el Teorema de la dimensión del núcleo y la imagen.

