% PDFLaTeX
\documentclass[a4paper,12pt,twoside,spanish,reqno]{amsbook}
%%%---------------------------------------------------
\usepackage[math]{kurier}

\usepackage{etex}
\usepackage{t1enc}
\usepackage{latexsym}
\usepackage[utf8]{inputenc}
\usepackage{verbatim}
\usepackage{multicol}
\usepackage{amsgen,amsmath,amstext,amsbsy,amsopn,amsfonts,amssymb}
\usepackage{amsthm}
\usepackage{calc}         % From LaTeX distribution
\usepackage{ifthen}
\input{random.tex}        % From CTAN/macros/generic
%\usepackage{subfigure} 
\usepackage{tikz}
\usetikzlibrary{arrows}
\usetikzlibrary{matrix}
\usepackage{mathtools}
\usepackage{stackrel}
\usepackage{enumerate}
\usepackage{graphicx}
\usepackage{multicol}
\usepackage{enumitem}

\usepackage{hyperref}
\hypersetup{
    colorlinks=true,
    linkcolor=blue,
    filecolor=magenta,      
    urlcolor=cyan,
}
\usepackage{hypcap}
\numberwithin{equation}{section}
% http://www.texnia.com/archive/enumitem.pdf (para las labels de los enumerate)
\renewcommand\labelitemi{$\circ$}
\setlist[enumerate, 1]{label={(\arabic*)}}
\setlist[enumerate, 2]{label=\emph{\alph*)}}


\usepackage{xcolor} 

\usepackage{wrapfig}
\usepackage[compatibility=false]{caption} % para usar subcaption
\usepackage{subcaption} % para poner varias imagenes juntas
\usepackage[spanish,activeacute,es-lcroman,es-tabla]{babel}
\usetikzlibrary{babel}

%%% FORMATOS %%%%%%%%%%%%%%%%%%%%%%%%%%%%%%%%%%%%%%%%%%%%%%%%%%%%%%%%%%%%%%%%%%%%%
\tolerance=10000
\renewcommand{\baselinestretch}{1.3} 
\usepackage[a4paper, top=3cm, left=3cm, right=2cm, bottom=2.5cm]{geometry}
\usepackage{setspace}
%\setlength{\parindent}{0,7cm}% tamaño de sangria.
\setlength{\parskip}{0,4cm} % separación entre parrafos.
\renewcommand{\baselinestretch}{0.90}% separacion del interlineado
%%%%%%%%%%%%%%%%%%%%%%%%%%%%%%%%%%%%%%%%%%%%%%%%%%%%%%%%%%%%%%%%%%%%%%%%%%%%%%%%%%%
%\end{comment}
%%% FIN FORMATOS  %%%%%%%%%%%%%%%%%%%%%%%%%%%%%%%%%%%%%%%%%%%%%%%%%%%%%%%%%%%%%%%%%

\newcommand{\rta}{\noindent\textsc{Solución: }} 
\newcommand \Z{{\mathbb Z}}
\newcommand \C{{\mathbb C}}
\newcommand \N{{\mathbb N}}
\newcommand \mcd{\operatorname{mcd}}
\newcommand \mcm{\operatorname{mcm}}


\begin{document}
    \baselineskip=0.55truecm %original
    

    
    
    {\bf \begin{center}  Práctico 0 \\ Álgebra  II -- Año 2024/1 \\ FAMAF \end{center}}
    
    {\bf \begin{center} Ejercicios resueltos \end{center}}
    
    
    
    \begin{enumerate}
    \setlength\itemsep{1.1em}

\item Expresar los siguientes números complejos en la forma $a +i b$.
Hallar el módulo y conjugado de cada uno de ellos, y graficarlos.

\begin{multicols}{3}
\begin{enumerate}
\item $(-1+i) (3-2i)$
\item $i^{131} - i^9 +1$
\item $\frac {1+i}{1+2i} + \frac{1-i}{1-2i}$
\end{enumerate}
\end{multicols}

\rta

\

\begin{enumerate}
    \item $(-1+i) (3-2i) = -3 + 3i + 2i - 2i^2 = -3 + 5i + 2 = \boxed{-1 + 5i}$

\

$ | (-1+i) (3-2i) | = | -1 + 5i | = \sqrt{ (-1)^2 + 5^2 } = \sqrt{1 + 25 } = \boxed{\sqrt{ 26}}$

\

$ \overline{ (-1+i) (3-2i) } = \overline { -1 + 5i } = \boxed{ -1 -5i }$

\

\item $i^{131} - i^9 +1 = i^{4 \cdot 32 + 3} - i^{4 \cdot 2 + 1} +1 = 
(i^4)^{32} \cdot i^3 - (i^4)^2 \cdot i^1 + 1 = i^3 - i +1 = -i -i +1 = \boxed{ 1-2i}$

\

$ | i^{131} - i^9 +1 | = | 1-2i | = \sqrt{ 1^2 + (-2)^2 } = \sqrt{1 + 4 } = \boxed{\sqrt{ 5}}$

\

$ \overline{ i^{131} - i^9 +1  } = \overline { 1-2i} = \boxed{ 1+2i}$

\

\item $\dfrac {1+i}{1+2i} + \dfrac{1-i}{1-2i} = \dfrac{(1+i)(1-2i)+(1-i)(1+2i)}{1^2 + 2^2} = \dfrac{2 Re( 1 - 2i + i - 2 i^2) }{5} = \boxed{ \dfrac{6}{5} }$

\

$ \left| \dfrac {1+i}{1+2i} + \dfrac{1-i}{1-2i} \right| = \left| \dfrac{6}{5} \right| = \boxed{ \dfrac{6}{5} }$

\

\

$ \overline{\; \dfrac {1+i}{1+2i} + \dfrac{1-i}{1-2i} \;} = \overline {\; \dfrac{6}{5} \;} = \boxed{ \dfrac{6}{5} }$
\end{enumerate}

\begin{figure}[!h]

\begin{subfigure}{.3\textwidth}
	\begin{tikzpicture}
		\draw[->] (-2.0,0) -- (1.0,0) node[right] {}; % eje x
		\draw[->] (0,-1) -- (0,6) node[above] {}; % eje y
		\draw[fill] (-1,5) circle [radius=0.05];
		\node [above left] at (-1,5) {$-1 + 5 i$};
		\node [below] at (-1,-3pt) {$-1$};
		\node [right] at (3pt,5) {$5$};
		\draw (-1,-3pt) -- (-1,3pt);
		\draw (-3pt, 5) -- (3pt, 5);
		\draw [dashed] (0,5) -- (-1,5);
		\draw [dashed] (-1,0) -- (-1,5);
	\end{tikzpicture}
\caption{}
\end{subfigure}
\begin{subfigure}{.3\textwidth}
	\begin{tikzpicture}
		\draw[->] (-1.0,0) -- (2.0,0) node[right] {}; % eje x
		\draw[->] (0,-3) -- (0,1) node[above] {}; % eje y
		\draw[fill] (1,-2) circle [radius=0.05];
		\node [below right] at (1,-2) {$1 -2 i$};
		\node [above] at (1,3pt) {$1$};
		\node [left] at (-3pt,-2) {$-2$};
		\draw (1,-3pt) -- (1,3pt);
		\draw (-3pt, -2) -- (3pt, -2);
		\draw [dashed] (0,-2) -- (1,-2);
		\draw [dashed] (1,0) -- (1,-2);
	\end{tikzpicture}
\caption{}
\end{subfigure}
\begin{subfigure}{.3\textwidth}
	\begin{tikzpicture}
		\draw[->] (-1.0,0) -- (1+6/5,0) node[right] {}; % eje x
		\draw[->] (0,-1) -- (0,1) node[above] {}; % eje y
		\draw[fill] (6/5,0) circle [radius=0.05];
%		\node [above right] at (6/5,0) {$ \frac{6}{5}$};
		\node [above] at (6/5,3pt) {$\frac{6}{5}$};
%		\node [left] at (-3pt,-2) {$-2$};
		\draw (6/5,-3pt) -- (6/5,3pt);
%		\draw (-3pt, -2) -- (3pt, -2);
%		\draw [dashed] (0,-2) -- (1,-2);
%		\draw [dashed] (1,0) -- (1,-2);
	\end{tikzpicture}
\caption{}
\end{subfigure}
\caption{Ejercicio 1}
\end{figure}

\newpage

\item Encontrar números reales $x$ e $y$ tales que $3x+2yi-xi+5y = 7 + 5i$.

\

\rta Sean $x, y \in \mathbb{R}$, separo las partes real e imaginaria de la ecuación y planteo un sistema de ecuaciones:

\begin{equation*}
\begin{array}{rl}
3x+2yi-xi+5y = 7 + 5i &\implies \left\{ \begin{array}{rl}
\operatorname{Re} (3x+2yi-xi+5y) &= \operatorname{Re} ( 7 + 5i)	\\
\operatorname{Im} (3x+2yi-xi+5y) &= \operatorname{Im}( 7 + 5i)
\end{array}\right. \\
&\implies \left\{ \begin{array}{rl}
3x+ 5y &= 7	\\
2y -x &= 5
\end{array}\right.
\end{array}
\end{equation*}

\begin{equation*}
\begin{array}{l|r}
\begin{array}{rl|rl}
3 (2y -5) + 5y &= 7		 &	2 \cdot 2 - 5 &= x \\
6y - 15 + 5 y &= 7		 &	-1 &= x \\
11 y &= 22	& &  \\
y &= 2		& & 
\end{array} & 
\boxed{\begin{array}{rl}
y &= 2 \\
x &= -1
\end{array}}
\end{array}
\end{equation*}

\



\item Probar que si $z \in \mathbb{C}$ tiene módulo $1$ entonces $z + z^{-1} \in \mathbb{R}$.

\

\rta Sabemos que el inverso de $z$ se puede escribir $z^{-1} = \frac{\overline{z} }{|z|^2}$. Como por hipótesis tenemos que $|z|=1$, resulta $z^{-1} = \overline{z}$. Luego:

$z + z^{-1} = z + \overline{z} = 2 \operatorname{Re} (z) \in \mathbb{R}$ \qed 

\

\item Probar que si $a\in \mathbb{R} \backslash \{0\} $ entonces el polinomio $x^2+a^2$ tiene siempre dos raíces complejas distintas.

\

\rta Se iguala a $0$ el polinomio:

\begin{equation*}
0 = x^2 + a^2 = x^2 - (ia)^2 = (x+ai)(x-ai) \implies \left\{ \begin{array}{rl}
x_1 &= ai \\
x_2 &= -ai
\end{array} \right.
\end{equation*}

Se tendrá $x_1 \neq x_2 \Leftrightarrow a \neq 0$.

\item
Simplificar las siguientes expresiones:
$$\begin{array}{ll}
 \text{a) } \ \left(\dfrac{-3}{\frac{4}{5}+1}\right)^{-1}\cdot\left(\dfrac{4}{5}-1\right) + \dfrac{1}{3}, \quad &
\text{ b)} \ \dfrac{a}{2\pi-6}(\pi-3)^2 -\dfrac{2a(\pi^2-9)}{\pi-3}.
\end{array}$$
	
\ 

\rta 

\ 


\vspace{.5cm}


\item Demostrar que  dados $z$, $z_1$, $z_2$ en $\C$ se cumple:
\[ |\bar z|= |z|, \qquad |z_1 \, z_2|= |z_1| \, |z_2|. \]

\

\rta 

\

Si $z = a + bi$, entonces $\overline{z} = a - bi$. Luego:
$$
|\bar z| = \sqrt{ a^2  + (-b)^2  }  = \sqrt{ a^2  + b^2  } = |z|. 
$$

\	

Si $z_1 = a + bi$, $z_2 = c + di$, entonces $z_1 \, z_2 = (ac - bd) + (ad + bc)i$. Luego:
\begin{align*}
	|z_1 \, z_2| &= \sqrt{ (ac - bd)^2  + (ad + bc)^2  }  \\
	& = \sqrt{ a^2c^2 -2acbd +b^2d^2  + a^2d^2 +2ad bc +b^2c^2  } \\
	& = \sqrt{ a^2c^2 +b^2d^2  + a^2d^2  +b^2c^2  }. 
\end{align*}
Por otro lado,
\begin{align*}
	|z_1| \, |z_2| &= \sqrt{ a^2 + b^2} \, \sqrt{c^2 +d^2 }  \\
	& = \sqrt{ a^2c^2 + a^2d^2  +b^2c^2 +b^2d^2   },
\end{align*}
con lo que  resulta que $|z_1 \, z_2|= |z_1| \, |z_2|$. 

\vspace{.5cm}


\item Sean $z=1+i$ y $w=\sqrt{2}-i$. Calcular:
 \begin{enumerate}
  \item $z^{-1}$; $1/w$; $z/w$; $w/z$.

  \item $1+z+z^2+z^3+\dots+z^{2019}$.

  \item $(z(z+w)^2-iz)/w$.
 \end{enumerate}
	
 \ 

 \rta 
 
 \  

\vspace{.5cm}


\item Sumar y multiplicar los siguientes pares de números complejos
    \begin{enumerate}
        \item $2+ 3i$ y $4$.
        \item $2+ 3i$ y $4i$.
        \item $1 + i$ y $ 1 -i$.
        \item $3-2i$ y $1 +i$. 
    \end{enumerate}

\ 

\rta 

\ 

\vspace{.5cm}


 \item Expresar los siguientes n{ú}meros complejos en la forma $a +i b$.
 Hallar el módulo, argumento y conjugado de cada uno de ellos y graficarlos.

 $$\begin{array}{lll}
 \text{a) }\ 2e^{\mathrm{i}\pi}-i,  \quad & \text{ b)} \  i^3 - 2i^{-7} -1, \quad &\text{ c)}\ (-2+i) (1+2i).
  \end{array}$$
	
  \ 

  \rta 
  
  \ 
    
\vspace{.5cm}

\item Sean $a,b\in\mathbb{C}$. Decidir si existe $z \in \mathbb{C}$ tal que:
\begin{enumerate}
  \item $z^2=b$. ¿Es único? ¿Para qué valores de $b$ resulta $z$ ser un número real?
  \item $z$ es imaginario puro y $z^2=4$.
  \item $z$ es imaginario puro y $z^2=-4$.
\end{enumerate}
	
\ 

\rta 

\ 




\end{enumerate}



\end{document}

