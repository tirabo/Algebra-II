% PDFLaTeX
\documentclass[a4paper,12pt,twoside,spanish,reqno]{amsbook}
%%%---------------------------------------------------

\usepackage[math]{kurier}

\usepackage{etex}
\usepackage{t1enc}
\usepackage{latexsym}
\usepackage[utf8]{inputenc}
\usepackage{verbatim}
\usepackage{multicol}
\usepackage{amsgen,amsmath,amstext,amsbsy,amsopn,amsfonts,amssymb}
\usepackage{amsthm}
\usepackage{calc}         % From LaTeX distribution
\usepackage{graphicx}     % From LaTeX distribution
\usepackage{ifthen}
\input{random.tex}        % From CTAN/macros/generic
\usepackage{subfigure} 
\usepackage{tikz}
\usetikzlibrary{arrows}
\usetikzlibrary{matrix}
\usepackage{mathtools}
\usepackage{stackrel}
\usepackage{enumitem}
\usepackage{tkz-graph}
%\usepackage{makeidx}
\usepackage{hyperref}
\hypersetup{
    colorlinks=true,
    linkcolor=blue,
    filecolor=magenta,      
    urlcolor=cyan,
}
\usepackage{hypcap}
\numberwithin{equation}{section}
% http://www.texnia.com/archive/enumitem.pdf (para las labels de los enumerate)
\renewcommand\labelitemi{$\circ$}
\setlist[enumerate, 1]{label={(\arabic*)}}
\setlist[enumerate, 2]{label=\emph{\alph*)}}


%%% FORMATOS %%%%%%%%%%%%%%%%%%%%%%%%%%%%%%%%%%%%%%%%%%%%%%%%%%%%%%%%%%%%%%%%%%%%%
\tolerance=10000
\renewcommand{\baselinestretch}{1.3}
\usepackage[a4paper, top=3cm, left=3cm, right=2cm, bottom=2.5cm]{geometry}
\usepackage{setspace}
%\setlength{\parindent}{0,7cm}% tamaño de sangria.
\setlength{\parskip}{0,4cm} % separación entre parrafos.
\renewcommand{\baselinestretch}{0.90}% separacion del interlineado
%%%%%%%%%%%%%%%%%%%%%%%%%%%%%%%%%%%%%%%%%%%%%%%%%%%%%%%%%%%%%%%%%%%%%%%%%%%%%%%%%%%
%\end{comment}
%%% FIN FORMATOS  %%%%%%%%%%%%%%%%%%%%%%%%%%%%%%%%%%%%%%%%%%%%%%%%%%%%%%%%%%%%%%%%%

\newcommand{\rta}{\noindent\textit{Rta: }} 
\newcommand \Z{{\mathbb Z}}
\newcommand \C{{\mathbb C}}
\newcommand \N{{\mathbb N}}
\newcommand \mcd{\operatorname{mcd}}
\newcommand \mcm{\operatorname{mcm}}


\begin{document}
    \baselineskip=0.55truecm %original
    
    
    {\bf \begin{center} Práctico 0 \\ Álgebra  II -- Año 2024/1 \\ FAMAF \end{center}}
    


\bigbreak

\subsection*{Objetivos}

\begin{itemize}
 \item Familiarizarse con los números complejos.
 \item Aprender a operar con números complejos (sumar, multiplicar, calcular inversos, conjugados y normas).
\end{itemize}

\subsection*{Ejercicios}


\begin{enumerate}

\item Expresar los siguientes números complejos en la forma $a +i b$.
Hallar el módulo y conjugado de cada uno de ellos, y graficarlos.

\begin{multicols}{3}
\begin{enumerate}
\item $(-1+i) (3-2i)$
\item $i^{131} - i^9 +1$
\item $\frac {1+i}{1+2i} + \frac{1-i}{1-2i}$
\end{enumerate}
\end{multicols}

\vspace{.5cm}


\item Encontrar números reales $x$ e $y$ tales que $3x+2yi-xi+5y = 7 + 5i$


\vspace{.5cm}

\item Probar que si $z \in \mathbb{C}$ tiene módulo $1$ entonces $z + z^{-1} \in \mathbb{R}.$

\vspace{.5cm}

\item Probar que si $a\in \mathbb{R}\backslash \{0\}$ entonces el polinomio $x^2+a^2$ tiene siempre dos raíces complejas distintas.

\vspace{.5cm}

\item
Simplificar las siguientes expresiones:
$$\begin{array}{ll}
 \text{a) } \ \left(\dfrac{-3}{\frac{4}{5}+1}\right)^{-1}\cdot\left(\dfrac{4}{5}-1\right) + \dfrac{1}{3}, \quad &
\text{ b)} \ \dfrac{a}{2\pi-6}(\pi-3)^2 -\dfrac{2a(\pi^2-9)}{\pi-3}.
\end{array}$$

\vspace{.5cm}


\item Demostrar que  dados $z$, $z_1$, $z_2$ en $\C$ se cumple:
\[ |\bar z|= |z|, \qquad |z_1 \, z_2|= |z_1| \, |z_2|. \]

\vspace{.5cm}


\item Sean $z=1+i$ y $w=\sqrt{2}-i$. Calcular:
 \begin{enumerate}
  \item $z^{-1}$; $1/w$; $z/w$; $w/z$.

  \item $1+z+z^2+z^3+\dots+z^{2019}$.

  \item $(z(z+w)^2-iz)/w$.
 \end{enumerate}


\vspace{.5cm}


\item Sumar y multiplicar los siguientes pares de números complejos
    \begin{enumerate}
        \item $2+ 3i$ y $4$.
        \item $2+ 3i$ y $4i$.
        \item $1 + i$ y $ 1 -i$.
        \item $3-2i$ y $1 +i$. 
    \end{enumerate}

\vspace{.5cm}


 \item Expresar los siguientes n{ú}meros complejos en la forma $a +i b$.
 Hallar el módulo, argumento y conjugado de cada uno de ellos y graficarlos.

 $$\begin{array}{lll}
 \text{a) }\ 2e^{\mathrm{i}\pi}-i,  \quad & \text{ b)} \  i^3 - 2i^{-7} -1, \quad &\text{ c)}\ (-2+i) (1+2i).
  \end{array}$$

\vspace{.5cm}

\item Sean $a,b\in\mathbb{C}$. Decidir si existe $z \in \mathbb{C}$ tal que:
\begin{enumerate}
  \item $z^2=b$. ¿Es único? ¿Para qué valores de $b$ resulta $z$ ser un número real?
  \item $z$ es imaginario puro y $z^2=4$.
  \item $z$ es imaginario puro y $z^2=-4$.
\end{enumerate}



\end{enumerate}

%============================================================
\subsection*{Ejercicios de repaso} Si ya hizo los ejercicios anteriores continue a la siguiente guía. Los ejercicios que siguen son similares a los anteriores y le pueden servir para practicar antes de los exámenes.
%============================================================

\

\begin{enumerate}[resume]

\item Expresar los siguientes n{ú}meros complejos en la forma $a +i b$. Hallar el módulo y conjugado de cada uno de ellos, y graficarlos.

$$\textrm{(a)}\; (\cos\theta - i\sin\theta)^{-1},\; 0\leq\theta<2\pi, \quad \qquad
\textrm{(b)} \; 3 i(1 + i)^4, \quad \qquad
\textrm{(c)} \; \dfrac{1+i}{1-i}$$


\

  \item Sea $z=2+\frac{1}{2}i$, calcular
 \begin{multicols}{3}
\begin{enumerate}
\item $\dfrac{(z+i)(z-i)}{z^2+1}$.
\item $z-2 + \dfrac{1}{z-2}$.
\item $\left|\dfrac{1}{z-i}\right|^2$.
\end{enumerate}
\end{multicols}

\

\item Sea $z\in\mathbb{C}$. Calcular $\frac{1}{z}+\frac{1}{\overline{z}} - \frac{1}{|z|^2}$.

%\
%
%\item Mostrar que las soluciones de la ecuación $z^4 + (-4+2i)z^2 - 1 =0$ son exactamente
%$-1+\sqrt {1-i}$, $-1-\sqrt {1-i}$, $1+\sqrt {1-i}$ y $1-\sqrt {1-i}$.

\

\item\label{desT} (Desigualdad triangular) Sean $w$ y $z$ números complejos. Probar que
\begin{eqnarray*}
  |w + z| \leq |w| + |z|,
\end{eqnarray*}
y la igualdad se cumple si y sólo si $w= r\cdot z$ para algún número real $r \geq 0$.
En general, sean $z_1,z_2,\ldots, z_n$ números complejos. Probar que
\begin{eqnarray*}
  \left|\sum_{k=1}^{n} z_k\right| \leq \sum_{k=1}^{n} |z_k|.
\end{eqnarray*}

\

\item Sean $w$ y $z$ números complejos. Entonces
\begin{eqnarray*}
  ||w|-|z|| \leq |w-z|.
\end{eqnarray*}

\end{enumerate}


\end{document}
