\chapter{Coordenadas - Matriz de cambio de base \\ Álgebra  II -- Año 2024/1 -- FAMAF}\label{practico-8}
    

\subsection*{Objetivos}
\begin{itemize}
\item Aprender a determinar las coordenadas de un vector en una base ordenada de un espacio vectorial.
\item Aprender a calcular la matriz de una transformación respecto a las bases canónicas.
\item Dadas dos bases ordenadas, aprender a operar con la matriz de cambio de base.
\end{itemize}
    
    
\subsection*{Ejercicios} 
    
\begin{enumerate}[topsep=6pt, itemsep=.4cm]


\item \label{lineales1-bases} Escribir las matrices de las siguientes transformaciones lineales respecto de las bases canónicas de los espacios involucrados.

\begin{enumerate}
    \item\label{lineales1-bases-a} $T:\R^2 \longrightarrow \R^3$, $T(x,y)=(x-y,x+y,2x+3y)$.
    \item\label{lineales1-bases-b} $S:\R^3 \longrightarrow \R^2$, $S(x,y,z)=(x-y+z,2x-y+2z)$.
    \item\label{lineales1-base-c} $D:P_4  \longrightarrow P_4$, $D(p(x))=p'(x)$.
    \item $T:M_{2\times 2}(\mathbb{K}) \longrightarrow \mathbb{K}$, $T(A)=\operatorname{tr}(A)$.
    \item\label{lineales1-base-d} $L:P_3 \longrightarrow M_{2\times 2}(\R)$, $L(ax^2+bx+c)=\begin{bmatrix} a & b+c \\ b+c & a \end{bmatrix}$.
    \item\label{lineales1-base-e} $Q:P_3 \longrightarrow P_4$, $Q(p(x))=(x+1)p(x)$.
\end{enumerate}

\item Dar las coordenadas del polinomio $2x^2+10x-1\in\mathbb{K}_3[x]$ en la base ordenada $$\mathcal{B}=\{1,x+1,x^2+x+1\}.$$

\item Dar las coordenadas de la matriz 
$A=\begin{bmatrix}
    1&2\\3&4 
    \end{bmatrix}
$ en la base ordenada 
$$
\mathcal{B}=\left\{
\begin{bmatrix}
    0&1\\0&0 
    \end{bmatrix},
\begin{bmatrix}
    0&0\\0&1 
    \end{bmatrix},
    \begin{bmatrix}
    1&0\\0&0 
    \end{bmatrix},
    \begin{bmatrix}
    0&0\\1&0 
    \end{bmatrix}
\right\}.
$$
Más generalmente, dar las coordenadas de cualquier matriz $\begin{bmatrix}
    a&b\\c&d 
    \end{bmatrix}$ en la base $\mathcal{B}$.


\item 
\begin{enumerate}
\item Dar una base del subespacio $W=\{(x,y,z)\in\mathbb{K}^3\mid x-y+2z=0\}$. 


\item Dar las coordenadas de $w=(1,-1,-1)$ en la base que haya dado en el item anterior.


\item Dado $(x,y,z)\in W$, dar las coordenadas de $(x,y,z)$ en la base que haya calculado en el item anterior. 
\end{enumerate}


\item\label{otras bases} Sea $\mathcal{C}$ la base canónica de $\mathbb{K}^2$ y 
    $\mathcal{B}=\{(1,0),(1,1)\}$ otra base de $\mathbb{R}^2$.
    \begin{enumerate}
        \item\label{otras bases-a} Encontrar la matriz de cambio de base $P_{\mathcal{C},\mathcal{B}}$ de $\mathcal{C}$ a $\mathcal{B}$.
        \item\label{otras bases-b} Encontrar la matriz de cambio de base $P_{\mathcal{B},\mathcal{C}}$ de $\mathcal{B}$ a $\mathcal{C}$.
        \item\label{otras bases-c} ¿Qué relación hay entre $P_{\mathcal{C},\mathcal{B}}$ y $P_{\mathcal{B},\mathcal{C}}$?
        \item\label{otras bases-d} Encontrar $(x,y),(z,w)\in\mathbb{K}^2$ tal que $[(x,y)]_{\mathcal{B}}=(1,4)$ y $[(z,w)]_{\mathcal{B}}=(1,-1)$.
        \item\label{otras bases-e} Determinar las coordenadas de $(2,3)$ y $(0,1)$ en las bases $\mathcal{B}_2$.
    \end{enumerate}
    

    \item\label{matriz P} Sea $P=\begin{bmatrix}
        1&1&0\\2&1&1\\3&1&0
        \end{bmatrix}
        \in\mathbb{K}^{3\times 3}$.
        \vskip .2cm
\begin{enumerate}
\item\label{inversa de P} Calcular la inversa de $P$.
\item\label{base de P} $\textcircled{a}$ Dar una base ordenada $\mathcal{B}$ de $\mathbb{K}^3$  
tal que $P$ es la matriz de cambio de coordenadas de la base canónica de $\mathbb{K}^3$ a la
base $\mathcal{B}$.
\item\label{encontrar vector} Encontrar $(x,y,z)\in\mathbb{K}^3$ tal que su vector de coordenadas con respecto a $\mathcal{B}$ es 
$$[(x,y,z)]_{\mathcal{B}}=(2,-1,-1).$$
\end{enumerate}


\item\label{matriz transformaciones ejemplo} Sea 
$T:\mathbb{R}^3\longrightarrow\mathbb{R}^2$ la transformación lineal definida por $$T(x,y,z)=(x-y,x-z).$$ Sean $\mathcal{C}$ la base canónica de $\mathbb{R}^3$ y $\mathcal{B}'=\{(1,1),(1,-1)\}$ base de $\mathbb{R}^2$.
\begin{enumerate}
    \item\label{matriz transformaciones ejemplo-a} Calcular la matriz $[T]_{\mathcal{C}\mathcal{B}'}$, es decir la matriz de $T$ respecto de las bases $\mathcal{C}$ y $\mathcal{B}'$.
    \item\label{matriz transformaciones ejemplo-b} Sea $(x,y,z)\in\mathbb{R}^3$. Dar las coordenadas de $T(x,y,z)$ respecto de la base $\mathcal{B}'$.
    \item\label{matriz transformaciones ejemplo-c} Sea $S:\mathbb{R}^2\longrightarrow\mathbb{R}^3$ una transformación lineal tal que su matriz respecto a las bases $\mathcal{B}'$ y $\mathcal{C}$ es
    \begin{align*}
    [S]_{\mathcal{B}'\mathcal{C}}=\begin{bmatrix}
    1&2\\1&-1\\1&0
    \end{bmatrix}. 
    \end{align*}
    Calcular la matriz de la composición $T\circ S:\mathbb{R}^2\longrightarrow\mathbb{R}^2$ con respecto a la base $\mathcal{B}'$. 
    \item\label{matriz transformaciones ejemplo-d} Calcular la matriz de $T\circ S$ respecto a la base $\mathcal{B}$ del ejercicio \ref{otras bases} usando las matrices de cambio de base calculadas en ese ejercicio.
\end{enumerate}


\item Sea $A$ la matriz  del ejercicio \ref{autovalores}\ref{autovalores-1} del práctico \ref{practico-5} y $T_A:\mathbb{R}^2\longrightarrow\mathbb{R}^2$ la transformación lineal dada por $T_A(v)=Av$. Hallar los autovalores de $T_A$, y para cada uno de ellos, dar una base de autovectores del correspondiente autoespacio. Decidir si $T_A$ es o no diagonalizable. En caso de serlo dar una matriz invertible $P$ tal que $P^{-1}AP$ es diagonal. 

Repetir esto para cada una de las matrices de dicho ejercicio.


\item Repetir el ejercicio anterior para cada matriz del ejercicio \ref{autovalores} del práctico \ref{practico-5} pero ahora consideradando a la transformación como una transformación lineal entre los $\mathbb{C}$-espacios vectoriales $\mathbb{C}^n$.


\item Sea $T:V\longrightarrow V$ una transformación lineal y $v\in V$ un autovector de autovalor $\lambda$. Probar las siguientes afirmaciones.
\begin{enumerate}
    \item\label{autovalor-autovector-a} Si $\lambda=0$, entonces $v\in\operatorname{Nu}(T)$.
    \item\label{autovalor-autovector-b} Si $\lambda\neq0$, entonces $v\in\operatorname{Im}(T)$.
    \item\label{autovalor-autovector-c} Si $T^2=0$, entonces $T-\operatorname{Id}$ es un isomorfismo.
\end{enumerate}


\item\label{base nilp} $\textcircled{a}$ Sea $V$ un espacio vectorial de dimensión $3$ y $T:V\longrightarrow V$ una transformación lineal. Supongamos que existe $v\in V$ tal que $T^3(v)=0$ pero $T^2(v)\neq0$.
\begin{enumerate}
    \item\label{base nilp a} $\textcircled{a}$ Probar que $\mathcal{B}=\{v,T(v),T^2(v)\}$ es una base de $V$.
    \item\label{base nilp b} Calcular la matriz de $T$ respecto de la base $\mathcal{B}$.
    \item\label{base nilp c} Calcular los autovalores de $T$ y sus correspondientes autoespacios. Decidir si $T$ es diagonalizable.
\end{enumerate}

        
\item Definir en cada caso una transformación lineal $T:\mathbb{R}^3\longrightarrow\mathbb{R}^3$ que satisfaga las condiciones requeridas. ¿Es posible definir más de una transformación lineal?
\begin{enumerate}
    \item\label{tl-condiciones-a} $(1,0,0)\in \operatorname{Nu}(T)$ 
    \item\label{tl-condiciones-b} $(1,0,0)\in \operatorname{Im}(T)$ 
    \item\label{tl-condiciones-c} $(1,0,0),(1,2,1)\in\operatorname{Nu}(T)$ y $(1,0,0) \in  \operatorname{Im}(T)$
\end{enumerate}


\item Decidir si las siguientes afirmaciones son verdaderas o falsas. Justificar. 
\begin{enumerate}
    \item\label{tl-V-o-F-a} Existe una transformación lineal $T:\mathbb{R}^3\longrightarrow\mathbb{R}^3$ tal que $\langle(1,2,3),(2,1,-1)\rangle$ es el autoespacio asociado a $0$ y $\langle(3,1,1),(1,1,3)\rangle$ es el autoespacio asociado a $5$.
    \item\label{tl-V-o-F-b} Existe una transformación lineal $T:\mathbb{R}^3\longrightarrow\mathbb{R}^3$ tal que $\langle(1,2,3)\rangle$ es el autoespacio asociado a $0$ y $\langle(3,1,1)\rangle$ es el autoespacio asociado a $5$.
    \item\label{tl-V-o-F-c} Existe una transformación lineal $T:\mathbb{R}^3\longrightarrow\mathbb{R}^3$ tal que $\{(1,0,1), (0,1,0)\}$ es una base de $\operatorname{Nu}(T)$ y  $\{(1,0,-1), (0,1,0)\}$ es una base de la $\operatorname{Im}(T)$.
    \item\label{tl-V-o-F-d} Existe una transformación lineal $T:\mathbb{R}^3\longrightarrow\mathbb{R}^3$ tal que $\{(1,0,1)\}$ es una base de $\operatorname{Nu}(T)$ y  $\{(1,0,-1), (0,1,0)\}$ es una base de la $\operatorname{Im}(T)$. 
\end{enumerate}

\end{enumerate}


\subsection*{Ejercicios de repaso}
Si ya hizo los ejercicios anteriores continue con la siguiente guía. Los ejercicios que siguen son similares y le pueden servir para practicar antes de los exámenes.

\begin{enumerate}[resume, topsep=5pt,itemsep=5pt]

    \item\label{otras bases 3} Repetir el ejercicio \ref{otras bases} con la base canónica de $\mathbb{R}^3$ y la base $\mathcal{B}_3=\{(1,0,0),(1,1,0),(1,1,1)\}$. Considerar las $3$-upla $(1,2,3)$ y $(0,1,2)$ para los últimos dos items.


\item Repetir los últimos items del ejercicio \ref{matriz transformaciones ejemplo} con la transformación lineal $S\circ T$ y la base del ejercicio anterior.

\item\label{cambio de base} $\textcircled{a}$ Sea $V$ un espacio vectorial con base $\mathcal{B}=\{v_1, ..., v_n\}$ y $A=(a_{ij})\in\mathbb{K}^{n\times n}$ una matriz. Sea $\mathcal{B}'=\{v_1', ..., v_n'\}$ donde
\begin{align*}
v_j'=\sum_{i=1}^na_{ij}v_i\,\mbox{ para todo $1\leq j\leq n$}. 
\end{align*}

Probar que $\mathcal{B}'$ es una base de $V$ si y sólo si $A$ es inversible. En tal caso determinar la matriz de cambio de base de la base $\mathcal{B}'$ a la base $\mathcal{B}$ y viceversa.

\item Para cada una de las siguientes transformaciones lineales, hallar sus autovalores,
    y para cada uno de ellos, dar una base de autovectores del espacio propio asociado. Luego, decir si la
    transformación considerada es o no  diagonalizable.
    \begin{enumerate}
        \item $T:\mathbb{R}^2\to \mathbb{R}^2$, \ $T(x,y)=(y,0)$.
        \item $T:\mathbb{R}^3\to \mathbb{R}^3$, \ $T(x,y,z)=(x+2z,-x-y+z,x+2y+z)$.
        \item $T:\mathbb{R}^3\to \mathbb{R}^3$, \ $T(x,y,z)=(4x+y+5z,4x-y+3z,-12x+y-11z)$.
        \item $T:\mathbb{R}^4\to \mathbb{R}^4$, \ $T(x,y,z,w)=(2x-y,x+4y,z+3w,z-w)$.
    \end{enumerate}

\item Repetir el ejercicio \ref{base nilp}  pero para cualquier $n\in\mathbb{N}$ en vez de $3$.

\end{enumerate}

\subsection*{Ayudas}

\

\ref{matriz P}\ref{base de P} Usar que $P_{\mathcal{C},\mathcal{B}}=P_{\mathcal{B},\mathcal{C}}^{-1}$ y recordar como se define $P_{\mathcal{B},\mathcal{C}}$.


\ref{base nilp}\ref{base nilp a} Es suficiente probar que $\mathcal{B}=\{v, T(v), T^2(v)\}$ es LI. Sean $a,b,c$ escalares tales que $av+bT(v)+cT^2(v)=0$. Si aplicamos $T^2$ en ambos lados deducimos que $aT^2(v)=0$ dado que $T^3(v)=0$. Entonces $a=0$ porque (completar argumento). Con un razonamiento similar deducir que $a=b=c=0$.


\ref{cambio de base} Es suficiente probar que $\mathcal{B}'$ es LI si y sólo si $A$ es invertible. Usar una estrategia similar a la demostración del Teorema 3.3.1 para probar esta equivalencia.
