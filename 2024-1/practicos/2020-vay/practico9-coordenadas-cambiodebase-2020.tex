\documentclass[12pt]{amsart}
\usepackage{amssymb}
\usepackage{enumerate}
\usepackage{amsmath}
\usepackage{geometry}
\geometry{ a4paper, total={210mm,297mm}, left=2cm, right=2cm, top=1.5cm, bottom=2.5cm, }
\usepackage{graphicx}
\usepackage{fancyhdr}
\usepackage{multicol}
%\usepackage{enumitem}

\newcommand{\este}{($\star$) \; }

\pagestyle{fancy}



\begin{document}
	
\noindent {\tiny \'Algebra / \'Algebra II \hfill Primer Cuatrimestre 2020}
	
\centerline{\Large{Pr\' actico 8}}
	
\
	
\centerline{\textsc{Coordenadas - Matriz de cambio de base}}
	
\
	
\noindent \textbf{Objetivos.} 
	
\begin{itemize}
		
\item Aprender a determinar las coordenadas de un vector en una base ordenada de un espacio vectorial.

\item Dadas dos bases ordenadas, aprender a operar con la matriz de cambio de base.

\end{itemize}
	
	\
	
	\noindent \textbf{Ejercicios.} 
	
\begin{enumerate}

	
	\item
Probar que los vectores $\;v_1=(1,0,-i),\;
v_2=(1+i,1-i,1),\;v_3=(i,i,i)$ forman una base de $\mathbb{C}^3$,
y dar las coordenadas de un vector $(x,y,z)$ en esta base.

\

\item Dados los siguientes vectores de $\mathbb{R}^4$,
$$
v_1=(1,1,0,0), \quad v_2=(0,0,1,1), \quad v_3=(1,0,0,4),
\quad v_4=(0,0,0,2).
$$
\begin{enumerate}
	\item Demostrar que
	$\mathcal{B}=\{v_1,v_2,v_3,v_4\}$ es una base de
	$\mathbb{R}^{4}$.
	\item Hallar las coordenadas de los vectores de la
	base can\'onica respecto de $\mathcal{B}$.
	\item Hallar las matrices de cambio de base de la base can\'onica
	a $\mathcal{B}$ y viceversa.
\end{enumerate}

\

\item  Sea $V=P_3$.
Sean
$$ g_1=1-x,\quad g_2=x+x^2, \quad g_3=(x+1)^2.$$
\begin{enumerate}
	\item Demostrar que $\mathcal{B}=\{g_1,g_2,g_3\}$ es una base de $V$.
	\item Hallar las matrices de cambio de base con respecto a $\mathcal{B}$
	y a la base can{\'o}nica $\{1,x,x^2\}$.
\end{enumerate}

\

\item Sea $\mathcal{B}=
\left\{
\begin{bmatrix}
2 & 0& 0 \\
0 & 3& 1
\end{bmatrix},
\begin{bmatrix}
0& 2& -1\\
0& 2&-1
\end{bmatrix},
\begin{bmatrix}
-1 &1&1 \\
2 & 0 &2
\end{bmatrix},
\begin{bmatrix}
1 &0 &1\\
0 &0 &0
\end{bmatrix},
\begin{bmatrix}
1& 0& 0\\
0 &2 &0
\end{bmatrix},
\begin{bmatrix}
0 &0 & 0\\
1 & 2&1
\end{bmatrix}
\right\}$.


\begin{enumerate}
	\item Demostrar que
	$\mathcal{B}$ es una base de $M_{2\times3}(\mathbb{R})$.
	\item Hallar las coordenadas de
	$
	\begin{bmatrix}
	1 & 1& 1 \\
	1 & 1& 1
	\end{bmatrix}$ con respecto a la base $\mathcal{B}$.
	\item Hallar las matrices de cambio de base de la base can\'onica
	a $\mathcal{B}$ y viceversa.
\end{enumerate}

\

\item Sea $W=<v_1,v_2>$, el subespacio de $\mathbb{C}^3$
generado por $v_1=(1,0,i)$ y $v_2=(1+i,1,-1)$.
\begin{enumerate}
	\item Demostrar que $\mathcal{B}_1=\{v_1,v_2\}$ es una base de $W$.
	\item Describir $W$ impl{\'\i}citamente.
	\item Demostrar que los vectores $w_1=(1,1,0)$ y
	$w_2=(1,i,1+i)$ pertenecen a $W$ y que $\mathcal{B}_2=\{w_1,w_2\}$
	es otra base de $W$.
	\item  ¿Cu{\'a}les son las coordenadas de $v_1$ y $v_2$ en la
	base ordenada $\mathcal{B}_2$?
	\item Hallar las matrices de cambio de base
	$P_{\mathcal{B}_1,\mathcal{B}_2}$ y $P_{\mathcal{B}_2,\mathcal{B}_1}$.
\end{enumerate}

\

\item Sea $\mathcal{B}=
\left\{
\begin{bmatrix}
2 & 0\\
0 & -1
\end{bmatrix},
\begin{bmatrix}
0& -1\\
0& 0
\end{bmatrix},
\begin{bmatrix}
-1 &0\\
\; 0 & 1
\end{bmatrix},
\begin{bmatrix}
0 &1 \\
1 &0
\end{bmatrix}
\right\} \subseteq M_2(\mathbb R)$.


\


\begin{enumerate}
	\item Demostrar que $\mathcal{B}$ es una base de $M_2(\mathbb R)$.
	

	\item Hallar las matrices de cambio de base de la base ordenada can\'onica de $M_2(\mathbb R)$ a la base ordenada $\mathcal{B}$ y viceversa.
		
	\item Hallar las coordenadas de la matriz
	$
	\begin{bmatrix}
	2 & 1\\
	1 & 2
	\end{bmatrix}$ con respecto a la base ordenada $\mathcal{B}$.
\end{enumerate}


\end{enumerate}

%============================================================
\subsection*{M\'as ejercicios}
%============================================================

\begin{enumerate}

\item Sea $\mathcal{B}=
\{ (1, 1, 0, 0), (0,0,1,1), (1, 0, 0, 2), (0, 0, -1, 1)\} \subseteq \mathbb R^4$.
Hallar la matriz de cambio de base de la base ordenada $\mathcal B' = \{(0, 0, 1, 0), (0, 0, 0, 1), (0, 1, 0, 0), (1, 0, 0, 0)\}$ a la base ordenada $\mathcal{B}$.


\


\item  Sea $W$ el subespacio de $\mathbb R^4$ generado los vectores $\alpha_1 = (-3, 1, 0, 0)$, $\alpha_2 = (-2, 0, 1, 0)$ y $\alpha_3 = (-1,1,-1,1)$.

Probar que $\mathcal B = \{\alpha_1, \alpha_2, \alpha_3\}$ es  base de $W$ y dar las coordenadas de un vector $(x, y, z, t)$ de $W$ en la base ordenada $\mathcal B$.

\

\item Calcular la matriz de cambio de base de la base ordenada $\mathcal B$ a la base ordenada $\mathcal B'$, donde:
\begin{equation*}\mathcal B  = \{ (0, 1, 1, -1), (0, 2, 0, 0), (2, 0, 1,
-1), (1, 1, 1, 1)\}, \quad \mathcal B'  = \{ e_1-e_2, e_1, e_3-e_4,
e_3\},\end{equation*} siendo $\{ e_1, e_2, e_3, e_4\}$ la base
ordenada can\' onica de $\mathbb R^4$.

\end{enumerate}

\end{document} 
