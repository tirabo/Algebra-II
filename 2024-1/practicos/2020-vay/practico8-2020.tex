\documentclass[12pt]{amsart}
\usepackage{amssymb}
\usepackage{enumerate}
\usepackage{amsmath}
\usepackage{geometry}
\geometry{ a4paper, total={210mm,297mm}, left=2cm, right=2cm, top=1.5cm, bottom=2.5cm, }
\usepackage{graphicx}
\usepackage{fancyhdr}
\usepackage{multicol}
\usepackage{enumitem}

\pagestyle{fancy}



\begin{document}

%\title{Pr\'actico 1}

\noindent {\tiny \'Algebra / \'Algebra II \hfill Segundo Cuatrimestre 2020}

%\maketitle

\centerline{\Large{Pr\' actico 9}}

\

\centerline{\textsc{Coordenadas, Matrices de transformaciones lineales y Diagonalizaci\'on}}

\subsection*{Objetivos}

\begin{itemize}
 \item Aprender a calcular coordenadas y la matriz de cambio de base.
 \item Aprender a calcular la matriz de una transformaci\'on lineal.
 \item Saber decidir si una transformaci\'on lineal es diagonalizable.
 \item Aprender a construir transformaciones lineales que satisfagan las propiedades solicitadas.
\end{itemize}


\bigbreak 


\subsection*{Ejercicios} Los ejercicios con el s\'imbolo $\textcircled{a}$ tienen una ayuda al final del archivo para que recurran a ella despu\'es de pensar un poco.


\begin{enumerate}
\item Dar las coordenadas del polinomio $2x^2+10x-1\in\mathbb{K}_3[x]$ en la base ordenada $$\mathcal{B}=\{1,x+1,x^2+x+1\}.$$

\

\item Dar las coordenadas de la matriz 
$A=\left(\begin{array}{cc}
   1&2\\3&4 
   \end{array}
   \right)
$ en la base ordenada 
$$
\mathcal{B}=\left\{
\left(\begin{array}{cc}
   0&1\\0&0 
   \end{array}
\right),
\left(\begin{array}{cc}
   0&0\\0&1 
   \end{array}
   \right),
   \left(\begin{array}{cc}
   1&0\\0&0 
   \end{array}
   \right),
   \left(\begin{array}{cc}
   0&0\\1&0 
   \end{array}
   \right)
\right\}.
$$
M\'as generalmente, dar las coordenadas de cualquier matriz $\left(\begin{array}{cc}
   a&b\\c&d 
   \end{array}
   \right)$ en la base $\mathcal{B}$.

\
 
\item 
\begin{enumerate}
\item Dar una base del subespacio $W=\{(x,y,z)\in\mathbb{K}^3\mid x-y+2z=0\}$. 

\

\item Dar las coordenadas de $w=(1,-1,-1)$ en la base que haya dado en el item anterior.

\

\item Dado $(x,y,z)\in W$, dar las coordenadas de $(x,y,z)$ en la base que haya calculado en el item anterior. 
\end{enumerate}

\

\item\label{otras bases} Sea $\mathcal{C}$ la base can\'onica de $\mathbb{K}^2$ y 
	$\mathcal{B}=\{(1,0),(1,1)\}$ otra base de $\mathbb{R}^2$.
	\begin{enumerate}
		\item Encontrar la matriz de cambio de base $P_{\mathcal{C},\mathcal{B}}$ de $\mathcal{C}$ a $\mathcal{B}$.
		\item Encontrar la matriz de cambio de base $P_{\mathcal{B},\mathcal{C}}$ de $\mathcal{B}$ a $\mathcal{C}$.
		\item ?`Qu\'e relaci\'on hay entre $P_{\mathcal{C},\mathcal{B}}$ y $P_{\mathcal{B},\mathcal{C}}$?
		\item Encontrar $(x,y),(z,w)\in\mathbb{K}^2$ tal que $[(x,y)]_{\mathcal{B}}=(1,4)$ y $[(z,w)]_{\mathcal{B}}=(1,-1)$.
		\item Determinar las coordenadas de $(2,3)$ y $(0,1)$ en las bases $\mathcal{B}_2$.
	\end{enumerate}
	
	\

	\item Sea $P=\left(\begin{array}{ccc}
      1&1&0\\2&1&1\\3&1&0
      \end{array}
\right)\in\mathbb{K}^{3\times 3}
$.

\begin{enumerate}
\item Calcular la inversa de $P$.
\item\label{base de P} $\textcircled{a}$ Dar una base ordenada $\mathcal{B}$ de $\mathbb{K}^3$  
tal que $P$ es la matriz de cambio de coordenadas de la base can\'onica de $\mathbb{K}^3$ a la
base $\mathcal{B}$.

\item Encontrar $(x,y,z)\in\mathbb{K}^3$ tal que su vector de coordenadas con respecto a $\mathcal{B}$ es 
$$[(x,y,z)]_{\mathcal{B}}=(2,-1,-1).$$
\end{enumerate}

\

\item\label{matriz transformaciones ejemplo} Sea 
$T:\mathbb{R}^3\longrightarrow\mathbb{R}^2$ la transformaci\'on lineal definida por $$T(x,y,z)=(x-y,x-z).$$ Sean $\mathcal{C}$ la base can\'onica de $\mathbb{R}^3$ y $\mathcal{B}'=\{(1,1),(1,-1)\}$ base de $\mathbb{R}^2$.

\begin{enumerate}
 \item Calcular la matriz $[T]_{\mathcal{C}\mathcal{B}'}$, es decir la matriz de $T$ respecto de las bases $\mathcal{C}$ y $\mathcal{B}'$.
 \item Sea $(x,y,z)\in\mathbb{R}^3$. Dar las coordenadas de $T(x,y,z)$ respecto de la base $\mathcal{B}'$.
 \item Sea $S:\mathbb{R}^2\longrightarrow\mathbb{R}^3$ una transformaci\'on lineal tal que su matriz respecto a las bases $\mathcal{B}'$ y $\mathcal{C}$ es
 \begin{align*}
[S]_{\mathcal{B}'\mathcal{C}}=\left(\begin{array}{cr}
1&2\\1&-1\\1&0
\end{array}\right). 
 \end{align*}
Calcular la matriz de la composici\'on $T\circ S:\mathbb{R}^2\longrightarrow\mathbb{R}^2$ con respecto a la base $\mathcal{B}'$. 
\item Calcular la matriz de $T\circ S$ respecto a la base $\mathcal{B}$ del Ejercicio \eqref{otras bases} usando las matrices de cambio de base calculadas en ese ejercicio.
\end{enumerate}

\

\item Sea $A$ la primer matriz del Ejercicio 1 del Pr\'actico 5 y $T_A:\mathbb{R}^2\longrightarrow\mathbb{R}^2$ la transformaci\'on lineal dada por $T_A(v)=Av$. Hallar los autovalores de $T_A$, y para cada uno de ellos, dar una base de autovectores del correspondiente autoespacio. Decidir si $T_A$ es o no diagonalizable. En caso de serlo dar una matriz invertible $P$ tal que $P^{-1}AP$ es diagonal. 

Repetir esto para cada una de las matrices de dicho ejercicio.

\

\item Repetir el ejercicio anterior para cada matriz del Ejercicio 1 del Pr\'actico 5 pero ahora consideradando a la transformaci\'on como una transformaci\'on lineal entre los $\mathbb{C}$-espacios vectoriales $\mathbb{C}^n$.

\

\item Sea $T:V\longrightarrow V$ una transformaci\'on lineal y $v\in V$ un autovector de autovalor $\lambda$. Probar las siguientes afirmaciones.
\begin{enumerate}
 \item Si $\lambda=0$, entonces $v\in\operatorname{Nu}(T)$.
 \item Si $\lambda\neq0$, entonces $v\in\operatorname{Im}(T)$.
  \item Si $T^2=0$, entonces $T-\operatorname{Id}$ es un isomorfismo.
\end{enumerate}

\

\item\label{base nilp} $\textcircled{a}$ Sea $V$ un espacio vectorial de dimensi\'on $3$ y $T:V\longrightarrow V$ una transformaci\'on lineal. Supongamos que existe $v\in V$ tal que $T^3(v)=0$ pero $T^2(v)\neq0$.
\begin{enumerate}
 \item\label{base nilp a} $\textcircled{a}$ Probar que $\mathcal{B}=\{v,T(v),T^2(v)\}$ es una base de $V$.
 \item Calcular la matriz de $T$ respecto de la base $\mathcal{B}$.
 \item Calcular los autovalores de $T$ y sus correspondientes autoespacios. Decidir si $T$ es diagonalizable.
\end{enumerate}

\
		
\item Definir en cada caso una transformaci\'on lineal $T:\mathbb{R}^3\longrightarrow\mathbb{R}^3$ que satisfaga las condiciones requeridas. ?`Es posible definir m\'as de una transformaci\'on lineal?
\begin{enumerate}
 \item $(1,0,0)\in \operatorname{Nu}(T)$ 
 \item $(1,0,0)\in \operatorname{Im}(T)$ 
 \item $(1,0,0),(1,2,1)\in\operatorname{Nu}(T)$ y $(1,0,0) \in  \operatorname{Im}(T)$
\end{enumerate}

\

\item Decidir si las siguientes afirmaciones son verdaderas o falsas. Justificar. 

\begin{enumerate}
\item Existe una transformaci\'on lineal $T:\mathbb{R}^3\longrightarrow\mathbb{R}^3$ tal que $\langle(1,2,3),(2,1,-1)\rangle$ es el autoespacio asociado a $0$ y $\langle(3,1,1),(1,1,3)\rangle$ es el autoespacio asociado a $5$.

\

\item Existe una transformaci\'on lineal $T:\mathbb{R}^3\longrightarrow\mathbb{R}^3$ tal que $\langle(1,2,3)\rangle$ es el autoespacio asociado a $0$ y $\langle(3,1,1)\rangle$ es el autoespacio asociado a $5$.

\

\item Existe una transformaci\'on lineal $T:\mathbb{R}^3\longrightarrow\mathbb{R}^3$ tal que $\{(1,0,1), (0,1,0)\}$ es una base de $\operatorname{Nu}(T)$ y  $\{(1,0,-1), (0,1,0)\}$ es una base de la $\operatorname{Im}(T)$.
 
 \
 
 \item Existe una transformaci\'on lineal $T:\mathbb{R}^3\longrightarrow\mathbb{R}^3$ tal que $\{(1,0,1)\}$ es una base de $\operatorname{Nu}(T)$ y  $\{(1,0,-1), (0,1,0)\}$ es una base de la $\operatorname{Im}(T)$.
\end{enumerate}




\end{enumerate}


\subsection*{Ejercicios de repaso}
Si ya hizo los ejercicios anteriores continue con la siguiente gu\'ia. Los ejercicios que siguen son similares y le pueden servir para practicar antes de los ex\'amenes.

\begin{enumerate}[resume, topsep=5pt,itemsep=5pt]
 \item\label{otras bases 3} Repetir el ejercicio \eqref{otras bases} con la base can\'onica de $\mathbb{R}^3$ y la base $\mathcal{B}_3=\{(1,0,0),(1,1,0),(1,1,1)\}$. Considerar las $3$-upla $(1,2,3)$ y $(0,1,2)$ para los \'ultimos dos items.
 
 \
 
 \item Repetir los \'ultimos items del Ejercicio \eqref{matriz transformaciones ejemplo} con la transformaci\'on lineal $S\circ T$ y la base del ejercicio anterior.

 \
 
 \
 
 \item\label{cambio de base} $\textcircled{a}$ Sea $V$ un espacio vectorial con base $\mathcal{B}=\{v_1, ..., v_n\}$ y $A=(a_{ij})\in\mathbb{K}^{n\times n}$ una matriz. Sea $\mathcal{B}'=\{v_1', ..., v_n'\}$ donde
\begin{align*}
v_j'=\sum_{i=1}^na_{ij}v_i\,\mbox{ para todo $1\leq j\leq n$}. 
\end{align*}

Probar que $\mathcal{B}'$ es una base de $V$ si y s\'olo si $A$ es inversible. En tal caso determinar la matriz de cambio de base de la base $\mathcal{B}'$ a la base $\mathcal{B}$ y viceversa.

\


	    \item Para cada una de las siguientes transformaciones lineales, hallar sus autovalores,
	y para cada uno de ellos, dar una base de autovectores del espacio propio asociado. Luego, decir si la
	transformaci\'on considerada es o no  diagonalizable.
	\begin{enumerate}
		\item $T:\mathbb{R}^2\to \mathbb{R}^2$, \ $T(x,y)=(y,0)$.
		\item $T:\mathbb{R}^3\to \mathbb{R}^3$, \ $T(x,y,z)=(x+2z,-x-y+z,x+2y+z)$.
		\item $T:\mathbb{R}^3\to \mathbb{R}^3$, \ $T(x,y,z)=(4x+y+5z,4x-y+3z,-12x+y-11z)$.
        \item $T:\mathbb{R}^4\to \mathbb{R}^4$, \ $T(x,y,z,w)=(2x-y,x+4y,z+3w,z-w)$.
	\end{enumerate}

 \item Repetir el Ejercicio \eqref{base nilp}  pero para cualquier $n\in\mathbb{N}$ en vez de $3$.

\end{enumerate}

\subsection*{Ayudas}

\eqref{base de P} Usar que $P_{\mathcal{C},\mathcal{B}}=P_{\mathcal{B},\mathcal{C}}^{-1}$ y recordar como se define $P_{\mathcal{B},\mathcal{C}}$.

\

\eqref{base nilp a} Es suficiente probar que $\mathcal{B}=\{v, T(v), T^2(v)\}$ es LI. Sean $a,b,c$ escalares tales que $av+bT(v)+cT^2(v)=0$. Si aplicamos $T^2$ en ambos lados deducimos que $aT^2(v)=0$ dado que $T^3(v)=0$. Entonces $a=0$ porque (completar argumento). Con un razonamiento similar deducir que $a=b=c=0$.

\

\eqref{cambio de base} Es suficiente probar que $\mathcal{B}'$ es LI si y s\'olo si $A$ es invertible. Usar una estrategia similar a la demostraci\'on del Teorema 3.3.1 para probar esta equivalencia.


%\

%\subsection*{M\'as ejercicios}
% Si ya hizo los ejercicios anteriores continue con la siguiente gu\'ia. Los ejercicios que siguen son similares y le pueden servir para practicar antes de los ex\'amenes.


%\begin{enumerate}

%\item


%\end{enumerate}


%\subsection*{Ejercicios un poco m\'as dif\'iciles} 

%Si ya hizo los primeros ejercicios ya sabe lo que tiene que saber. Los %siguientes ejercicios le pueden servir si esta muy aburridx con la cuarentena. 

%\


%\begin{enumerate}
%\item 
%\end{enumerate}

%\subsection*{Ayudas}
\end{document} 
