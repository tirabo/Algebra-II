
\chapter{Soluciones\\Álgebra  II -- Año 2024/1 -- FAMAF}\label{practico-7}

\begin{enumerate}[topsep=6pt, itemsep=.4cm]


    \item\label{transf-lineales-incisos} Decidir si las siguientes funciones son transformaciones lineales entre los respectivos espacios vectoriales sobre $\mathbb{K}$.
    \begin{enumerate}[resume, topsep=5pt,itemsep=5pt]
    \item\label{transf-lineales-a} La traza $\operatorname{Tr}:\mathbb{K}^{n\times n}\longrightarrow\mathbb{K}$ (recordar ejercicio \ref{traza}\,\ref{ej:traza} del Práctico  \ref{practico-3}) 
    \item\label{transf-lineales-b} $T:\mathbb{K}[x]\longrightarrow\mathbb{K}[x]$, $T(p(x))=q(x)\,p(x)$ donde $q(x)$ es un polinomio fijo.
    \item\label{transf-lineales-c} $T:\mathbb{K}^2\longrightarrow\mathbb{K}$, $T(x,y)=xy$
    \item\label{transf-lineales-d} $T:\mathbb{K}^2\longrightarrow\mathbb{K}^3$, $T(x,y)=(x,y,1)$
    \item\label{transf-lineales-e} El determinante $\operatorname{det}:\mathbb{K}^{n\times n}\longrightarrow\mathbb{K}$.
    \end{enumerate}

    \rta

    \ref{transf-lineales-a} Sí, es un transformación lineal. En efecto, si $A,B \in \mathbb{K}^{n\times n}$ y $\lambda \in \mathbb{K}$, entonces 
    \begin{align*}
        \operatorname{Tr}(A+\lambda B) &= \sum_{i=1}^n (a_{ii} + \lambda b_{ii}) = \sum_{i=1}^n a_{ii} + \lambda \sum_{i=1}^n b_{ii} = \operatorname{Tr}(A) + \lambda \operatorname{Tr}(B).
    \end{align*}

    \vskip .3cm
    \ref{transf-lineales-b} Sí, es un transformación lineal. En efecto, si $r,s \in \mathbb{K}[x]$ y $\lambda \in \mathbb{K}$, entonces
    \begin{align*}
        T(r+\lambda s) &= q(r+\lambda s) = qr + \lambda qs = T(r) + \lambda T(s).
    \end{align*}

    \vskip .3cm
    \ref{transf-lineales-c} No, no es una transformación lineal. Por ejemplo,  que $T(2(1,1)) \ne 2T(1,1)$. Por un lado, $T(2(1,1))=T(2,2) = 2 \cdot 2 = 4$. Por otro lado, $2T(1,1) = 2 \cdot 1 \cdot 1 = 2$. 

    \vskip .3cm
    \ref{transf-lineales-d} No, no es una transformación lineal. Por ejemplo, veamos que $T(0,0) = (0,0,1)\ne (0,0,0)$.

    \vskip .3cm
    \ref{transf-lineales-e} No, no es una transformación lineal. Por ejemplo,  $\operatorname{det}(2\Id_2) = 4 \ne 2 = 2 \operatorname{det}(\Id_2)$. En general, $\operatorname{det}(\lambda A) = \lambda^n \operatorname{det}(A)$, donde $n$ es el tamaño de la matriz $A$.

    \qed
    
    
    \item Sea $T:\mathbb{C}\longrightarrow\mathbb{C}$, $T(z)=\overline{z}$.
    \begin{enumerate}
    \item\label{conj-C} Considerar a $\mathbb{C}$ como un $\mathbb{C}$-espacio vectorial y decidir si $T$ es una transformación lineal.
    \item\label{conj-R} Considerar a $\mathbb{C}$ como un $\mathbb{R}$-espacio vectorial y decidir si $T$ es una transformación lineal.
    \end{enumerate}
    
    \rta

    \ref{conj-C} No, no es una transformación lineal. Por ejemplo, $T(2i) = -2i \ne 2i = 2T(i)$.
    
    \vskip .3cm
    \ref{conj-R} Sí, es una transformación lineal. En efecto, si $a+bi, a'+b'i \in \mathbb{C}$ y $\lambda \in \mathbb{R}$, entonces  
    \begin{align*}
        T((a+bi) + \lambda (a'+b'i)) &= T((a+\lambda a') + (b+\lambda b')i) = (a+\lambda a') - (b+\lambda b')i \\
        &= (a-bi) + \lambda (a'-b'i) = T(a+bi) + \lambda T(a'+b'i).
    \end{align*}
    \qed

    \vskip .3cm

    \item\label{T en la base} Sea $T:\mathbb{K}^3\longrightarrow\mathbb{K}^3$ una transformación lineal tal que $T(e_1)=(1,2,3)$, $T(e_2)=(-1,0,5)$ y $T(e_3)=(-2,3,1)$. 
    \begin{enumerate}
        \item\label{T en dos vectores} Calcular $T(2,3,8)$ y $T(0,1,-1)$. 
        \item\label{T en la base b} Calcular $T(x,y,z)$ para todo $(x,y,z)\in\mathbb{K}^3$. Es decir, dar una fórmula para $T$ donde en cada coordenada del vector de llegada hay una combinación lineal de $x,y,z$.
        \item\label{matriz otro}  Encontrar una matriz $A\in\mathbb{K}^{3\times3}$ tal que $T(x,y,z)=A\begin{bmatrix}  x\\y\\z \end{bmatrix}$. En esta parte del ejercicio escribiremos/pensaremos a los vectores de $\mathbb{K}^3$ como columnas.
    \end{enumerate}
    
        \rta

    

        \ref{T en dos vectores} 
        \begin{align*}
            T(2,3,8) &= T(2e_1 + 3e_2 + 8e_3) = 2T(e_1) + 3T(e_2) + 8T(e_3)
            \\
            & = 2(1,2,3) + 3(-1,0,5) +8(-2,3,1) \\
            &= (2,4,6) + (-3,0,15) + (-16,24,8) \\
            &= (-17,28,29).
        \end{align*}
        \begin{align*}
            T(0,1,-1) &= T(0e_1 + 1e_2 - 1e_3) = 0T(e_1) + 1T(e_2) - 1T(e_3)\\
            & = 0(1,2,3) + 1(-1,0,5) - 1(-2,3,1)\\ &= (0,0,0)+(-1,0,5)+(2,-3,-1)\\
            & = (1,-3,4).
        \end{align*}

        \vskip .3cm
        \ref{T en la base b} 
        \begin{align*}
            T(x,y,z) &= T(xe_1 + ye_2 + ze_3) = xT(e_1) + yT(e_2) + zT(e_3)
            \\
            & = x(1,2,3) + y(-1,0,5) +z(-2,3,1) \\
            & = (x,2x,3x) + (-y,0,5y) +(-2z,3z,z) \\
            &= (x-y-2z,2x+3z,3x+5y+z).
        \end{align*}

        \vskip .3cm
        \ref{matriz otro} Observar que  
        $$
        \begin{bmatrix} a&b&c \end{bmatrix} \begin{bmatrix} x\\y\\z \end{bmatrix} = ax+by+cz.
        $$ 
        Basándonos en esta observación, obtenemos la matriz
        \begin{align*}
            A = \begin{bmatrix}
                1 & -1 & -2 \\
                2 & 0 & 3 \\
                3 & 5 & 1
            \end{bmatrix}.
        \end{align*}
        Entonces,
        \begin{align*}
            \begin{bmatrix}
                1 & -1 & -2 \\
                2 & 0 & 3 \\
                3 & 5 & 1
            \end{bmatrix} \begin{bmatrix}
                x\\y\\z
            \end{bmatrix} = \begin{bmatrix}
                x-y-2z\\2x+3z\\3x+5y+z
            \end{bmatrix} = T(x,y,z).
        \end{align*}
    \qed
    
    \vskip .3cm
    
    \item\label{Txyz} Sea $T:\mathbb{K}^3\longrightarrow\mathbb{K}^3$ definida por $T(x,y,z)=(x+2y+3z, y-z,x+5y)$.
        \begin{enumerate}
            \item\label{matriz de T} Encontrar una matriz  $A\in\mathbb{K}^{3\times 3}$ tal que $T(x,y,z)=A\begin{bmatrix}
                    x\\y\\z \end{bmatrix}
                    $.  Como en el ejercicio  \ref{T en la base}\,\ref{matriz otro} pensamos a los vectores como columnas.
            \item\label{Txyz-vectores-nucleo} Decir cuáles de los siguientes vectores están en el núcleo: $(1,1,1)$, $(-5,1,1)$.
            \item\label{Txyz nucleo imagen de T implicito} Describir mediante ecuaciones (implícitamente) el núcleo y la imagen de $T$.
            \item\label{Txyz  nucleo imagen T generadores} Dar un conjunto de generadores del núcleo y la imagen de $T$.
            \item\label{Txyz imagen} Decir cuáles de los siguientes vectores están en la imagen: $(0,1,0)$, $(0,1,3)$.
    \end{enumerate}
    
    \rta

    \ref{matriz de T} Observar  que 
    $$
    F_i(A)\cdot \begin{bmatrix} x\\y\\z \end{bmatrix} = T(x,y,z)_i.
    $$
    Es decir la fila $i$  de $A$ por el vector $(x,y,z)$ nos da la coordenada $i$ de $T(x,y,z)$. Como $T(x,y,z) = (x+2y+3z, y-z,x+5y)$, tenemos que
    $$
    A \begin{bmatrix} x\\y\\z \end{bmatrix} = \begin{bmatrix} x+2y+3z\\ y-z\\x+5y \end{bmatrix},
    $$
    y por lo tanto
    \begin{align*}
        A = \begin{bmatrix}
            1 & 2 & 3 \\
            0 & 1 & -1\\
            1 & 5 & 0
        \end{bmatrix}.  
    \end{align*}

    \vskip .3cm

    \ref{Txyz-vectores-nucleo} 
    $$
    T(1,1,1) = (1+2+3,1-1,1+5) = (6,0,6) \ne (0,0,0),
    $$
    por lo tanto $(1,1,1) \notin \operatorname{Nu}(T)$.

    $$
    T(-5,1,1) = (-5+2+3,1-1,-5+5) = (0,0,0),
    $$
    por lo tanto $(-5,1,1) \in \operatorname{Nu}(T)$.

    \vskip .3cm
    

    \ref{Txyz nucleo imagen de T implicito} Debemos resolver la ecuación $AX=b$, entonces
    \begin{itemize}
        \item el núcleo de $T$ es el conjunto de soluciones del sistema homogéneo $AX=0$,
        \item la imagen de $T$ es el conjunto de los $b\in\R^m$ para los cuales el sistema $AX=b$ tiene solución
    \end{itemize} 
    Resolvamos el sistema con matrices aumentadas:
    \begin{align*}
        &\begin{amatrix}{3}
            1 & 2 & 3 &b_1 \\
            0 & 1 & -1 &b_2\\
            1 & 5 & 0&b_3
        \end{amatrix} 
        \stackrel{F_3-F_1}{\longrightarrow}
        \begin{amatrix}{3}
            1 & 2 & 3 &b_1 \\
            0 & 1 & -1 &b_2\\
            0 & 3 & -3&b_3-b_1
        \end{amatrix}
        \stackrel{F_3-3F_2}{\longrightarrow}
        \begin{amatrix}{3}
            1 & 2 & 3 &b_1 \\
            0 & 1 & -1 &b_2\\
            0 & 0 & 0&b_3-b_1-3b_2
        \end{amatrix} \\
        &\stackrel{F_1-2F_2}{\longrightarrow}
        \begin{amatrix}{3}
            1 & 0 & 5 &b_1-2b_2 \\
            0 & 1 & -1 &b_2\\
            0 & 0 & 0&b_3-b_1-3b_2
        \end{amatrix}.
    \end{align*}
    El sistema tiene solución si y solo si $b_3-b_1-3b_2 = 0$. Es decir,
    
    Es decir que 
    \begin{align}\label{Txyz imagen ecuacion}
        \operatorname{Im}(T) &= \{(b_1,b_2,b_3) \in \mathbb{K}^3: b_3-b_1-3b_2 = 0\} 
    \end{align}

    Por otro lado, el sistema homogéneo asociado a $AX=0$ tiene como soluciones los $(x,y,z)$ tales que $x+5z=0$, $y - z=0$. Es decir,
    $$
    \operatorname{Nu}(T) = \{(x,y,z) \in \mathbb{K}^3: x+5z=0, y - z=0\}.
    $$


    \vskip .3cm

    \ref{Txyz  nucleo imagen T generadores} Por lo visto en el inciso anterior
    \begin{align*}
        \operatorname{Im}(T) &= \{(b_1,b_2,b_3) \in \mathbb{K}^3: b_3-b_1-3b_2 = 0\} \\ &= \{(b_1,b_2,b_3) \in \mathbb{K}^3: b_3 = b_1+3b_2\} \\
        &= \{(b_1,b_2,b_1+3b_2) : b_1,b_2 \in \mathbb{K}\} \\
        &= \{(b_1,0,b_1)  + (0,b_2,3b_2) : b_2, b_3 \in \mathbb{K}\} \\
        &= \langle (1,0,1), (0,1,3) \rangle.
    \end{align*}
    Por lo tanto, podemos elegir $(1,0,1), (0,1,3)$ generadores de $\operatorname{Im}(T)$.

    Por otro lado,
    \begin{align*}
        \operatorname{Nu}(T) &= \{(x,y,z)\} \\
        &= \{(x,y,z) \in \mathbb{K}^3: x=-5z, y = z\} \\
        &= \{(-5z,z,z) : z \in \mathbb{K}\} \\
        &= \{(-5,1,1)z : z \in \mathbb{K}\} \\
        &= \langle (-5,1,1) \rangle.
    \end{align*} 
    Por lo tanto, podemos elegir $(-5,1,1)$ como generador de $\operatorname{Nu}(T)$.

    \vskip .3cm

    \ref{Txyz imagen} Debemos comprobar si los vectores  $(0,1,0)$, $(0,1,3)$ satisfacen las ecuaciones que definen la imagen, es decir  las de la fórmula \eqref{Txyz imagen ecuacion}. Explícitamente, $(b_1,b_2,b_3) \in \operatorname{Im}(T)$ si y solo si $b_3-b_1-3b_2 = 0$. 
    
    Ahora bien, $(0,1,0) \in \operatorname{Im}(T)$ si y solo si $0-0-3\cdot 1 = -3 = 0$, lo cual es falso. Por lo tanto, $(0,1,0) \notin \operatorname{Im}(T)$
    
    Por otro lado, $(0,1,3) \in \operatorname{Im}(T)$ si y solo si $3-0-3\cdot 1 = 0$, lo cual es verdadero. Por lo tanto, $(0,1,3) \in \operatorname{Im}(T)$.

\qed


\vskip .4cm

    
    
    \item\label{tl-matriz} Sea $T: \mathbb{K}^4 \to \mathbb{K}^5$ dada por $T(v) = Av$ donde $A$ es la siguiente matriz
        $$
        A=\begin{bmatrix}
        0& 2& 0&1\\   1& 3& 0&1\\  -1&-1&0&0\\3&0&3&0\\2&1&1&0 \end{bmatrix}
        $$
        \begin{enumerate}[topsep=5pt,itemsep=5pt]
            \item\label{tl-matriz-a} Dar una base del núcleo y de la imagen de $T$. 
            \item\label{tl-matriz-b} Dar la dimensión del núcleo y de la imagen de $T$.
            \item\label{tl-matriz-c} Describir mediante ecuaciones (implícitamente) el núcleo y la imagen de $T$.
            \item\label{tl-matriz-d} Decir cuáles de los siguientes vectores están en el núcleo:
            $(1,2,3,4)$, $(1,-1,-1,2)$, $(1,0,2,1)$.
            \item\label{tl-matriz-e} Decir cuáles de los siguientes vectores están en la imagen:
            $(2,3,-1,0,1)$, $(1,1,0,3,1)$, $(1,0,2,1,0)$.
        \end{enumerate}
        
    \rta primero encontremos una fórmula para $T(x,y,z,t)$, luego resolvamos los incisos. 
    $$
    T(x,y,z,t) = \begin{bmatrix}
        0& 2& 0&1\\   1& 3& 0&1\\  -1&-1&0&0\\3&0&3&0\\2&1&1&0 \end{bmatrix} \begin{bmatrix}
        x\\y\\z\\t
    \end{bmatrix} = \begin{bmatrix}
        2y+t\\x+3y+t\\-x-y\\3x+3z\\2x+y+z
    \end{bmatrix}.
    $$
    Por lo tanto 
    \begin{equation}
        T(x,y,z,t) =  .
    \end{equation}
    
    
    $$(-1,1,9,2) -4(0,0,0,1) -8(0,0,1,0) = (-1,1,1,-2)$$ 

    \ref{tl-matriz-a}  Como $T(v) = (0,0,0,0,0)$ si y solo si $Av=0$, el $\operatorname{Nu}(T)$ es el conjunto de soluciones del sistema $Av=0$,  que resolvemos  con Gauss:
    \begin{align*}
        &\begin{bmatrix}
            0 & 2 & 0 &1\\
            1 & 3 & 0 &1\\
            -1 &-1 &0 &0\\
            3 &0 &3 &0\\
            2 &1 &1 &0
        \end{bmatrix}
        \underset{F_5-2F_2}{\underset{F_4-3F_2}{\stackrel{F_3+F_2}{\longrightarrow}}}
        \begin{bmatrix}
            0 & 2 & 0 &1\\
            1 & 3 & 0 &1\\
            0 &2 &0 &1\\
            0 &-9 &3 &-3\\
            0 &-5 &1 &-2
        \end{bmatrix}
        \stackrel{F_4-3F_5}{\longrightarrow}
        \begin{bmatrix}
            0 & 2 & 0 &1\\
            1 & 3 & 0 &1\\
            0 &2 &0 &1\\
            0 &6 &0 &3\\
            0 &-5 &1 &-2
        \end{bmatrix} \\
        &\underset{F_4-3F_1}{\stackrel{F_3-F_1}{\longrightarrow}}
        \begin{bmatrix}
            0 & 2 & 0 &1\\
            1 & 3 & 0 &1\\
            0 &0 &0 &0\\
            0 &0 &0 &0\\
            0 &-5 &1 &-2
        \end{bmatrix}
        \stackrel{F_1/2}{\longrightarrow}
        \begin{bmatrix}
            0 & 1 & 0 &1/2\\
            1 & 3 & 0 &1\\
            0 &0 &0 &0\\
            0 &0 &0 &0\\
            0 &-5 &1 &-2
        \end{bmatrix}
        \underset{F_5+5F_2}{\stackrel{F_2-3F_1}{\longrightarrow}}
        \begin{bmatrix}
            0 & 1 & 0 &1/2\\
            1 & 0 & 0 &-1/2\\
            0 &0 &0 &0\\
            0 &0 &0 &0\\
            0 &0 &1 &1/2
        \end{bmatrix}.
    \end{align*}
    Luego 
    \begin{equation}\label{eq-del-nucleo}
        \operatorname{Nu}(T) = \{(x,y,z,t) \in \mathbb{K}^4: x -\frac{1}{2}t =0, y + \frac{1}{2}t=0, z + \frac{1}{2}t =0\},
    \end{equation}
    Por lo tanto,  
    \begin{align*}
        \operatorname{Nu}(T) &= \{(x,y,z,t) \in \mathbb{K}^4: x =\frac{1}{2}t, y = -\frac{1}{2}t, z = -\frac{1}{2}t\} \\ &= 
        \left\{\left(\frac{1}{2}t,-\frac{1}{2}t,-\frac{1}{2}t,t\right) : t \in \mathbb{K}\right\} \\ &=
        \left\{t\left(\frac{1}{2},-\frac{1}{2},-\frac{1}{2},1\right) : t \in \mathbb{K}\right\} \\ &=
        \langle \left(\frac{1}{2},-\frac{1}{2},-\frac{1}{2},1\right) \rangle.
    \end{align*}
    En  consecuencia, $\{(1,-1,-1,2)\}$ es una base del núcleo de $T$. 

    \vskip .3cm

    Para encontrar una base de la imagen, calculamos $T(e_1),T(e_2),T(e_3),T(e_4)$ que es un sistema de generadores de la imagen. A partir de estos generadores encontramos una base.
    \begin{align*}
        T(e_1) &= (2\cdot 0+0,1+3\cdot 0+0,-1-0,3\cdot 1+3\cdot 0,2\cdot 1+0+0)\\ &= (0,1,-1,3,2), 
    \end{align*}
    \begin{align*}
        T(e_2) &= (2\cdot 1+0,0+3\cdot 1+0,-0-1,3\cdot 0+3\cdot 0,2\cdot 0+1+0)\\ &= (2,3,-1,0,1), 
    \end{align*}
    \begin{align*}
        T(e_3) &= (2\cdot 0+0,0+3\cdot 0+0,-0-0,3\cdot 0+3\cdot 1,2\cdot 0+0+1)\\ &= (0,0,0,3,1), 
    \end{align*}
    \begin{align*}
        T(e_4) &= (2\cdot 0+1,0+3\cdot 0+1,-0-0,3\cdot 0+3\cdot 0,2\cdot 0+0+0)\\ &= (1,1,0,0,0).
    \end{align*}
    Encontramos una base de la imagen haciendo la matriz donde las filas son los vectores anteriores y hacemos Gauss:
    \begin{align*}
        &\begin{bmatrix}
            0 & 1 & -1 &3 &2\\
            2 & 3 & -1 &0 &1\\
            0 & 0 & 0 &3 &1\\
            1 & 1 & 0 &0 &0
        \end{bmatrix}
        \stackrel{F_2-2F_4}{\longrightarrow}
        \begin{bmatrix}
            0 & 1 & -1 &3 &2\\
            0 & 1 & -1 &0 &1\\
            0 & 0 & 0 &3 &1\\
            1 & 1 & 0 &0 &0
        \end{bmatrix}
        \underset{F_4-F_1}{\stackrel{F_2-F_1}{\longrightarrow}}
        \begin{bmatrix}
            0 & 1 & -1 &3 &2\\
            0 & 0 & 0 &-3 &-1\\
            0 & 0 & 0 &3 &1\\
            1 & 0 & 1 &-3 &-2
        \end{bmatrix} \\
        &\underset{F_4+F_3}{\underset{F_2+F_3}{\stackrel{F_1-F_3}{\longrightarrow}}}
        \begin{bmatrix}
            0 & 1 & -1 &0 &1\\
            0 & 0 & 0 &0 &0\\
            0 & 0 & 0 &3 &1\\
            1 & 0 & 1 &0 &-1
        \end{bmatrix}
    \end{align*}
    Luego, una base de la imagen es $\{(1,0,1,0,-1),(0, 1, -1, 0, 1),(0,0,0,3,1)\}$.

    \vskip .3cm
    \ref{tl-matriz-b} Hemos visto más arriba que el núcleo tiene una base de un elemento y la imagen tiene una base de tres elementos, por consiguiente  $\dim \operatorname{Nu}(T) = 1$ y $\dim \operatorname{Im}(T) = 3$. Observar que  $\dim \operatorname{Nu}(T) + \dim \operatorname{Im}(T) = 4$, que es la dimensión del dominio.

    \vskip .3cm
    \ref{tl-matriz-c}  El núcleo está descripto en \eqref{eq-del-nucleo} o,  equivalentemente,
    \begin{equation}\label{eq-del-nucleo-2}
        \operatorname{Nu}(T) = \{(x,y,z,t) \in \mathbb{K}^4: 2x -t =0, 2y + t=0, 2z + t =0\}.
    \end{equation}
    
    Para  la imagen,  debemos plantear la ecuación $T(x,y,z,t) = (b_1,b_2,b_3,b_4,b_5)$ y resolver el sistema. Es decir, debemos resolver
    \begin{align*}
        (2y+t,x+3y+t,-x-y,3x+3z,2x+y+z) &= (b_1,b_2,b_3,b_4,b_5) \\
        \Rightarrow \quad 2y+t &= b_1, \\
        x+3y+t &= b_2, \\
        -x-y &= b_3, \\
        3x+3z &= b_4, \\
        2x+y+z &= b_5.
    \end{align*}
    Resolvamos el sistema con matrices aumentadas:
    \begin{align*}
        &\begin{amatrix}{4}
            0 & 2 & 0 &1 &b_1\\
            1 & 3 & 0 &1 &b_2\\
            -1 &-1 &0 &0 &b_3\\
            3 &0 &3 &0 &b_4\\
            2 &1 &1 &0 &b_5
        \end{amatrix}
        \underset{F_5-2F_2}{\underset{F_4-3F_2}{\stackrel{F_3+F_2}{\longrightarrow}}}
        \begin{amatrix}{4}
            0 & 2 & 0 &1 &b_1\\
            1 & 3 & 0 &1 &b_2\\
            0 &2 &0 &1 &b_2+b_3\\
            0 &-9 &3 &-3 &-3b_2 +b_4\\
            0 &-5 &1 &-2 &+-2b_2b_5
        \end{amatrix} 
    \end{align*}
    \begin{align*}
        &\stackrel{F_3-F_1}{\longrightarrow}
        \begin{amatrix}{4}
            0 & 2 & 0 &1 &b_1\\
            1 & 3 & 0 &1 &b_2\\
            0 &0 &0 &0 &-b_1+b_2+b_3\\
            0 &-9 &3 &-3 &-3b_2 +b_4\\
            0 &-5 &1 &-2 &-2b_2+b_5
        \end{amatrix}
        \stackrel{F_1/2}{\longrightarrow}
        \begin{amatrix}{4}
            0 & 1 & 0 &1/2 &b_1/2\\
            1 & 3 & 0 &1 &b_2\\
            0 &0 &0 &0 &-b_1+b_2+b_3\\
            0 &-9 &3 &-3 &-3b_2 +b_4\\
            0 &-5 &1 &-2 &-2b_2+b_5
        \end{amatrix} 
    \end{align*}
    \begin{align*}
        &\underset{F_5+5F_1}{\underset{F_4+9F_1}{\stackrel{F_2-3F_1}{\longrightarrow}}}
        \begin{amatrix}{4}
            0 & 1 & 0 &1/2 &b_1/2\\
            1 & 0 & 0 &-1/2 &b_2-3b_1/2\\
            0 &0 &0 &0 &-b_1+b_2+b_3\\
            0 &0 &3 &3/2 &-3b_2 +b_4+9b_1/2\\
            0 &0 &1 &1/2 &-2b_2+b_5+5b_1/2
        \end{amatrix} 
    \end{align*}
    \begin{align*}
        &\stackrel{F_4-3F_5}{\longrightarrow}
        \begin{amatrix}{4}
            0 & 1 & 0 &1/2 &b_1/2\\
            1 & 0 & 0 &-1/2 &b_2-3b_1/2\\
            0 &0 &0 &0 &-b_1+b_2+b_3\\
            0 &0 &0 &0 &-6b_1/2+3b_2 +b_4-3b_5\\
            0 &0 &1 &1/2 &-2b_2+b_5+5b_1/2
        \end{amatrix}.
    \end{align*}
    Lo importante de estas ecuaciones son las filas donde  los coeficientes son $0$: $(b_1,b_2.b_3,b_4) \in \operatorname{Im}(T)$ si y solo si $-b_1+b_2+b_3=0$ y $-6b_1/2+3b_2 +b_4-3b_5=0$. Es decir,
    \begin{equation}\label{eq-del-imagen-2}
        \begin{aligned}
        \operatorname{Im}(T) &= \{(b_1,b_2,b_3,b_4,b_5) \in \mathbb{K}^5: \\ 
        &\qquad\qquad  -b_1+b_2+b_3=0 \text{ y } -3b_1+3b_2 +b_4-3b_5=0\}.
        \end{aligned}
    \end{equation}


    \vskip .3cm
    \ref{tl-matriz-d}  Para ver si $(1,2,3,4)$,  $(1,-1,-1,2)$, $(1,0,2,1)$ están en $\operatorname{Nu}(T)$ debemos ver si $T$ de cada vector es $(0,0,0,0,0)$ o si  cumplen con la ecuación \eqref{eq-del-nucleo-2} o si, como vimos en \ref{tl-matriz-a}, son múltiplos de $(1,-1,-1,2)$. El tercer método es el más sencillo y, por lo tanto, lo usaremos. 
    
    Es claro que $(1,2,3,4)$ no es múltiplo de $(1,-1,-1,2)$, por lo tanto, $(1,2,3,4)$ no está en $\operatorname{Nu}(T)$.

    Por  el contrario $(1,-1,-1,2)$ es múltiplo de $(1,-1,-1,2)$, por lo tanto, $(1,-1,-1,2)$ está en $\operatorname{Nu}(T)$.

    Finalmente, $(1,0,2,1)$ no es múltiplo de $(1,-1,-1,2)$, por lo tanto, $(1,0,2,1)$ no está en $\operatorname{Nu}(T)$.


    \vskip .3cm
    \ref{tl-matriz-e}  Para ver si $(2,3,-1,0,1)$, $(1,1,0,3,1)$, $(1,0,2,1,0)$ están en $\operatorname{Im}(T)$ debemos ver si  cumplen con la ecuación \eqref{eq-del-imagen-2}. Es decir 
    $$
    (x_1,x_2,x_3,x_4,x_5) \in \operatorname{Im}(T) \Leftrightarrow -x_1+x_2+x_3=0 \text{\; y } -3x_1+3x_2 +x_4-3x_5=0.
    $$


    Para $(2,3,-1,0,1)$,  $-2+3-1=0$ y $-6+9-3=0$, por lo tanto, $(2,3,-1,0,1)$ está en $\operatorname{Im}(T)$.

    Para $(1,1,0,3,1)$,  $-1+1+0=0$ y $-3+3+3-3=0$, por lo tanto, $(1,1,0,3,1)$ está en $\operatorname{Im}(T)$.

    Para $(1,0,2,1,0)$,  $-1+0+2=1$ y $-3+0+3-0=0$, por lo tanto, $(1,0,2,1,0)$ no está en $\operatorname{Im}(T)$.
    
\qed



\vskip .3cm


        
    \item \label{lineales1} Para cada una de las siguientes transformaciones lineales calcular el núcleo y la imagen. Describir ambos subespacios implícitamente y encontrar una base de cada uno de ellos.    \begin{enumerate}
        \item\label{lineales1-a} $T:\R^2 \longrightarrow \R^3$, $T(x,y)=(x-y,x+y,2x+3y)$.
        \item\label{lineales1-b} $S:\R^3 \longrightarrow \R^2$, $S(x,y,z)=(x-y+z,2x-y+2z)$.
    \end{enumerate}

    \rta 

    \ref{lineales1-a} Como ya dijimos vamos a caracterizar los $(b_1,b_2,b_3) \in \R^3$ tales que $T(x,y) = (b_1,b_2,b_3)$ para $(x,y) \in \R^2$. Es decir, debemos resolver el sistema
    \begin{align*}
        (x-y,x+y,2x+3y) &= (b_1,b_2,b_3) \\
        \Rightarrow \quad x-y &= b_1, \\
        x+y &= b_2, \\
        2x+3y &= b_3.
    \end{align*}
    Resolvamos el sistema con matrices aumentadas:
    \begin{align*}
        &\begin{amatrix}{2}
            1 & -1 & b_1 \\
            1 & 1 & b_2 \\
            2 & 3 & b_3
        \end{amatrix}
        \stackrel{F_3-2F_1}{\stackrel{F_2-F_1}{\longrightarrow}}
        \begin{amatrix}{2}
            1 & -1 & b_1 \\
            0 & 2 & -b_1+b_2 \\
            0 & 5 & -2b_1+b_3
        \end{amatrix}\\
        &\stackrel{F_2/2}{\longrightarrow}
        \begin{amatrix}{2}
            1 & -1 & b_1 \\
            0 & 1 & -b_1/2+b_2/2 \\
            0 & 5 & -2b_1+b_3
        \end{amatrix}
        \stackrel{F_1+F_2}{\stackrel{F_3-5F_2}{\longrightarrow}}
        \begin{amatrix}{2}
            1 & 0 & b_1/2+b_2/2  \\
            0 & 1 & -b_1/2+b_2/2 \\
            0 & 0 & b_1/2-5b_2/2+b_3
        \end{amatrix}. \tag{*}
    \end{align*}

    Luego, el sistema tiene solución si y solo si $b_1/2-5b_2/2+b_3 = 0$. Es decir,
    \begin{equation}\label{lineales1-a-imagen}
        \operatorname{Im}(T) = \{(b_1,b_2,b_3) \in \R^3: b_1-5b_2+2b_3 = 0\}.
    \end{equation}
    Encontrar una base de $\operatorname{Im}(T)$ es fácil:
    \begin{align*}
        \operatorname{Im}(T) &= \{(b_1,b_2,b_3) \in \R^3: b_1-5b_2+2b_3 = 0\} \\
        &= \{(b_1,b_2,b_3)\in \R^3: b_1 =5b_2-2b_3 \} \\
        &= \{(5b_2-2b_3,b_2,b_3): b_2,b_3 \in \R \} \\
        &= \{b_2(5,1,0)+b_3(-2,0,1): b_2,b_3 \in \R \} \\
        &= \langle (5,1,0),(-2,0,1) \rangle.
    \end{align*}
    Es decir podemos considerar como base de $\operatorname{Im}(T)$ a $\{(5,1,0),(-2,0,1)\}$.

    Para encontrar la ecuación implícita del núcleo debemos resolver el sistema homogéneo asociado a $T(x) =0$. Es decir, debemos resolver el sistema
    \begin{align*}
        (x-y,x+y,2x+3y) &= (0,0,0) \\
        \Rightarrow \quad x-y &= 0, \\
        x+y &= 0, \\
        2x+3y &= 0.
    \end{align*}
    Pero esto ya lo hicimos más arriba, si tomamos $b_1=b_2=b_3=0$ en la MRF (*). Claramente, la  matriz nos indica que $x=y=0$. Es decir, $\operatorname{Nu}(T) = \{(0,0)\}$ y  la base es el conjunto $\emptyset$.


    \vskip .3cm
    \ref{lineales1-b} Como ya dijimos vamos a caracterizar los $(b_1,b_2) \in \R^2$ tales que $S(x,y,z) = (b_1,b_2)$ para $(x,y,z) \in \R^3$. Es decir, debemos resolver el sistema
    \begin{align*}
        (x-y+z,2x-y+2z) &= (b_1,b_2) \\
        \Rightarrow \quad x-y+z &= b_1, \\
        2x-y+2z &= b_2.
    \end{align*}
    Resolvamos el sistema con matrices aumentadas:
    \begin{align*}
        &\begin{amatrix}{3}
            1 & -1 & 1 & b_1 \\
            2 & -1 & 2 & b_2
        \end{amatrix}
        \stackrel{F_2-2F_1}{\longrightarrow}
        \begin{amatrix}{3}
            1 & -1 & 1 & b_1 \\
            0 & 1 & 0 & -2b_1+b_2
        \end{amatrix}\\
        &\stackrel{F_1+F_2}{\longrightarrow}
        \begin{amatrix}{3}
            1 & 0 & 1 & -b_1+b_2 \\
            0 & 1 & 0 & b_2-2b_1
        \end{amatrix}. \tag{**}
    \end{align*}
    Como no hay ninguna condición sobre los $b_i$, el sistema tiene solución para todo $(b_1,b_2) \in \R^2$. Es decir, 
    \begin{equation}\label{lineales1-b-imagen}
        \operatorname{Im}(S) = \R^2.    
    \end{equation}
    Claramente,  una base  podría ser $\{(1,0),(0,1)\}$.

    Para encontrar la ecuación implícita del núcleo debemos resolver el sistema homogéneo asociado a $S(x) =0$. Es decir, debemos resolver el sistema
    \begin{align*}
        (x-y+z,2x-y+2z) &= (0,0) \\
        \Rightarrow \quad x-y+z &= 0, \\
        2x-y+2z &= 0.
    \end{align*}
    Pero esto ya lo hicimos más arriba, si tomamos $b_1=b_2=0$ en la MRF (**). Claramente, la  matriz nos indica que $x+z=0$ e $y=0$. Es decir, 
    \begin{align*}
        \operatorname{Nu}(S) &= \{(x,y,z) \in \R^3: x+z=0, y =0 \}  \tag{***}\\ &= \{(-z,0,z) : z \in \R\} = \{z(-1,0,1) : z \in \R\}
        \\&= \langle (1,0,-1) \rangle.
    \end{align*}
    Luego (***) es la ecuación implícita del núcleo y $\{(1,0,-1)\}$ es una base de $\operatorname{Nu}(S)$.
    
    \qed

    \vskip .3cm

    
\item \label{lineales2} Para cada una de las siguientes transformaciones lineales calcular el núcleo y la imagen. Describir ambos subespacios implícitamente y encontrar una base de cada uno de ellos.    
\begin{enumerate}
    \item $D:P_4  \longrightarrow P_4$, $D(p(x))=p'(x)$.
    \item $T:M_{2\times 2}(\mathbb{K}) \longrightarrow \mathbb{K}$, $T(A)=\operatorname{tr}(A)$.
    \item $L:P_3 \longrightarrow M_{2\times 2}(\R)$, $L(ax^2+bx+c)=\begin{bmatrix} a & b+c \\ b+c & a \end{bmatrix}$.
    \item $Q:P_3 \longrightarrow P_4$, $Q(p(x))=(x+1)p(x)$.
\end{enumerate}

\rta
    \vskip .3cm


        \item Sea $T:\mathbb{K}^{2\times 2}\longrightarrow\mathbb{K}_{4}[x]$ la transformación lineal definida por
        \begin{align*}
        T   \begin{bmatrix}  a&b\\c&d \end{bmatrix} &= (a-c+2d)x^3+(b+2c-d)x^2+ \\
        &\qquad+(-a+2b+5c-4d)x+(2a-b-4c+5d)
        \end{align*}
        \begin{enumerate}
            \item\label{tl-matrices-pol-a} Decir cuáles de los siguientes matrices están en el núcleo:
                \begin{align*}
                    A=\begin{bmatrix}
                        2&0\\0&-1
                    \end{bmatrix},
                \quad
                B=\begin{bmatrix}
                    -1&-1\\1&1
                \end{bmatrix},
                \quad
                C=\begin{bmatrix}
                    -1&-1\\1&0
                \end{bmatrix}.
                \end{align*}
            \item\label{tl-matrices-pol-b} Decir cuáles de los siguientes polinomios están en la imagen:
                \begin{align*}
                    p(x)=x^3+x^2+x+1,\quad q(x)=x^3, \quad r(x)=(x-1)(x-1) 
                \end{align*}
        \end{enumerate}
    
    \rta 

    \ref{tl-matrices-pol-a}  Para ver si $A$, $B$ y $C$ están en el núcleo debemos ver si $T$ de cada matriz es el polinomio nulo. Por la definición de $T$,
    $$
    \begin{bmatrix}  a&b\\c&d \end{bmatrix} \in \operatorname{Nu}(T) \Leftrightarrow \begin{cases}
        a-c+2d=0\\
        b+2c-d=0\\
        -a+2b+5c-4d=0\\
        2a-b-4c+5d=0.
    \end{cases}
    $$
    Podemos comprobar en cada matriz estas ecuacines, o podemos simplificar el sistema para que resulta más fácil al comprobación. Haremos esto último con Gauss:
    \begin{equation*}
        \begin{bmatrix}
            1 & 0 & -1 & 2 \\
            0 & 1 & 2 & -1 \\
            -1 & 2 & 5 & -4 \\
            2 & -1 & -4 & 5
        \end{bmatrix}
        \underset{F_4-2F_1}{\stackrel{F_3+F_1}{\longrightarrow}}
        \begin{bmatrix}
            1 & 0 & -1 & 2 \\
            0 & 1 & 2 & -1 \\
            0 & 2 & 4 & -2 \\
            0 & -1 & -2 & 1
        \end{bmatrix}
        \underset{F_4+F_2}{\stackrel{F_3-2F_2}{\longrightarrow}}
        \begin{bmatrix}
            1 & 0 & -1 & 2 \\
            0 & 1 & 2 & -1 \\
            0 & 0 & 0 & 0 \\
            0 & 0 & 0 & 0
        \end{bmatrix}.
    \end{equation*}
    Por lo tanto, 
    $$
    \begin{bmatrix}  a&b\\c&d \end{bmatrix} \in \operatorname{Nu}(T) \Leftrightarrow \begin{cases}
        a-c+2d\\
        b+2c-d.
    \end{cases}
    $$
    Veamos ahora si  las matrices $A$, $B$ y $C$ cumplen con estas ecuaciones:
    \begin{enumerate}
        \item[$A$:] $a=2$, $b=0$, $c=0$, $d=-1$, por lo tanto, $a-c+2d=2-0+2\cdot (-1)=0$ y $b+2c-d=0+2\cdot 0-(-1)=1$, es decir, $A \notin \operatorname{Nu}(T)$.
        \item[$B$:] $a=-1$, $b=-1$, $c=1$, $d=1$, por lo tanto, $a-c+2d=-1-1+2\cdot 1=0$ y $b+2c-d=-1+2\cdot 1-1=0$, es decir, $B \in \operatorname{Nu}(T)$.
        \item[$C$:] $a=-1$, $b=-1$, $c=1$, $d=0$, por lo tanto, $a-c+2d=-1-1+2\cdot 0=0$ y $b+2c-d=-1+2\cdot 1-0=1$, es decir, $C \notin \operatorname{Nu}(T)$. 
    \end{enumerate}

    \vskip .3cm
    \ref{tl-matrices-pol-b}  Para ver si $p(x)$, $q(x)$ y $r(x)$ están en la imagen debemos caracterizar la imagen.

    Por la definición de $T$, $a_3x^3+a_2x^2+a_1x+a_0 \in \operatorname{Im}(T)$ si y solo si existe $A \in \mathbb{K}^{2\times 2}$ tal que $T(A) = a_3x^3+a_2x^2+a_1x+a_0$. Es decir, si y solo si
    \begin{align*}
        \begin{cases}
            a-c+2d=a_3\\
            b+2c-d=a_2\\
            -a+2b+5c-4d=a_1\\
            2a-b-4c+5d=a_0.
        \end{cases}
    \end{align*}
    Resolvamos el sistema con Gauss.
    \begin{align*}
        &\begin{amatrix}{4}
            1 & 0 & -1 & 2 & a_3 \\
            0 & 1 & 2 & -1 & a_2 \\
            -1 & 2 & 5 & -4 & a_1 \\
            2 & -1 & -4 & 5 & a_0
        \end{amatrix}
        \underset{F_4-2F_1}{\stackrel{F_3+F_1}{\longrightarrow}}
        \begin{amatrix}{4}
            1 & 0 & -1 & 2 & a_3 \\
            0 & 1 & 2 & -1 & a_2 \\
            0 & 2 & 4 & -2 & a_1+a_3 \\
            0 & -1 & -2 & 1 & a_0-2a_3
        \end{amatrix} \\
        &\underset{F_4+F_2}{\stackrel{F_3-2F_2}{\longrightarrow}}
        \begin{amatrix}{4}
            1 & 0 & -1 & 2 & a_3 \\
            0 & 1 & 2 & -1 & a_2 \\
            0 & 0 & 0 & 0 & a_1-2a_2+a_3 \\
            0 & 0 & 0 & 0 & a_0+a_2-2a_3
        \end{amatrix}.
    \end{align*}
    Por lo tanto, 
    $$
    a_3x^3+a_2x^2+a_1x+a_0 \in \operatorname{Im}(T) \quad \Leftrightarrow \quad a_1-2a_2+a_3=0 \text{\; y \;} a_0+a_2-2a_3=0.
    $$

    En  el caso de $p(x)=x^3+x^2+x+1$, $a_3=1$, $a_2=1$, $a_1=1$, $a_0=1$, por lo tanto, $a_1-2a_2+a_3=1-2+1=0$ y $a_0+a_2-2a_3=1+1-2=0$, es decir, $p(x) \in \operatorname{Im}(T)$. 
    
    Para $q(x)=x^3$, $a_3=1$, $a_2=0$, $a_1=0$, $a_0=0$, por lo tanto, $a_1-2a_2+a_3=1 \ne 0$, es decir, $q(x) \notin \operatorname{Im}(T)$. 
    
    Finalmente para $r(x)=(x-1)(x-1) = x^2 -2x +1$, $a_3=0$, $a_2=1$, $a_1=-2$, $a_0=1$, por lo tanto, $a_1-2a_2+a_3=-2-2+0=-4 \ne 0$, es decir, $r(x) \notin \operatorname{Im}(T)$.
    
    \qed


    \item\label{funcional ej}  Sea $T:\mathbb{K}^3\longrightarrow\mathbb{K}$ definida por $T(x,y,z)=x+2y+3z$.
    \begin{enumerate}
        \item Probar que $T$ es un epimorfismo.
        \item Dar la dimensión del núcleo de $T$.
        \item Encontrar una matriz $A$ tal que
            $T(x,y,z)=A\begin{bmatrix}
            x\\y\\z \end{bmatrix}$. ¿De qué tamaño debe ser $A$? Como en el ejercicio \ref{Txyz}\, \ref{matriz de T} pensamos a los vectores como columnas. 
    \end{enumerate}
    
    \rta
    
    \item Determinar cuáles transformaciones lineales de los ejercicios  \ref{lineales1} y \ref{lineales2} son monomorfismos, epimorfismos y/o isomorfismos.
    
    \rta
    
    \item\label{usar-1} Encontrar en cada caso, cuando sea posible, una matriz $A\in\mathbb{K}^{3\times 3}$ tal que la transformación lineal $T:\mathbb{K}^3\longrightarrow\mathbb{K}^3$, $T(v)=Av$, satisfaga las condiciones exigidas (como en el ejercicio  \ref{T en la base}\,\ref{matriz otro} pensamos a los vectores como columnas). Cuando no sea posible, explicar por qué no es posible.
    \begin{enumerate}[ topsep=5pt,itemsep=5pt]
        \item $\operatorname{dim} \operatorname{Im}(T)=2$ y $\operatorname{dim}\operatorname{Nu}(T)=2$.
        \item $T$ inyectiva y $T(e_1)=(1,0,0)$, $T(e_2)=(2,1,5)$ y $T(e_3)=(3,-1,0)$.
        \item $T$ sobreyectiva y $T(e_1)=(1,0,0)$, $T(e_2)=(2,1,5)$ y $T(e_3)=(3,-1,0)$.
        
        \item\label{usar Txyz} \textcircled{a} $T(e_1)=(1,0,0)$, $T(e_2)=(2,1,5)$ y $T(e_3)=(3,-1,0)$.
        
        \item $e_1\in\operatorname{Im}(T)$ y $(-5,1,1)\in\operatorname{Nu}(T)$.
        
        \item $\operatorname{dim} \operatorname{Im}(T)=2$.
    \end{enumerate}
    
    \rta
        
    \end{enumerate}