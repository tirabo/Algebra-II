
\chapter{Soluciones\\Álgebra  II -- Año 2024/1 -- FAMAF}\label{practico-7}

\begin{enumerate}[topsep=6pt, itemsep=.4cm]


    \item\label{transf-lineales-incisos} Decidir si las siguientes funciones son transformaciones lineales entre los respectivos espacios vectoriales sobre $\mathbb{K}$.
    \begin{enumerate}[resume, topsep=5pt,itemsep=5pt]
    \item\label{transf-lineales-a} La traza $\operatorname{Tr}:\mathbb{K}^{n\times n}\longrightarrow\mathbb{K}$ (recordar ejercicio \ref{traza}\,\ref{ej:traza} del Práctico  \ref{practico-3}) 
    \item\label{transf-lineales-b} $T:\mathbb{K}[x]\longrightarrow\mathbb{K}[x]$, $T(p(x))=q(x)\,p(x)$ donde $q(x)$ es un polinomio fijo.
    \item\label{transf-lineales-c} $T:\mathbb{K}^2\longrightarrow\mathbb{K}$, $T(x,y)=xy$
    \item\label{transf-lineales-d} $T:\mathbb{K}^2\longrightarrow\mathbb{K}^3$, $T(x,y)=(x,y,1)$
    \item\label{transf-lineales-e} El determinante $\operatorname{det}:\mathbb{K}^{n\times n}\longrightarrow\mathbb{K}$.
    \end{enumerate}

    \rta

    \ref{transf-lineales-a} Sí, es un transformación lineal. En efecto, si $A,B \in \mathbb{K}^{n\times n}$ y $\lambda \in \mathbb{K}$, entonces 
    \begin{align*}
        \operatorname{Tr}(A+\lambda B) &= \sum_{i=1}^n (a_{ii} + \lambda b_{ii}) = \sum_{i=1}^n a_{ii} + \lambda \sum_{i=1}^n b_{ii} = \operatorname{Tr}(A) + \lambda \operatorname{Tr}(B).
    \end{align*}

    \vskip .3cm
    \ref{transf-lineales-b} Sí, es un transformación lineal. En efecto, si $r,s \in \mathbb{K}[x]$ y $\lambda \in \mathbb{K}$, entonces
    \begin{align*}
        T(r+\lambda s) &= q(r+\lambda s) = qr + \lambda qs = T(r) + \lambda T(s).
    \end{align*}

    \vskip .3cm
    \ref{transf-lineales-c} No, no es una transformación lineal. Por ejemplo,  que $T(2(1,1)) \ne 2T(1,1)$. Por un lado, $T(2(1,1))=T(2,2) = 2 \cdot 2 = 4$. Por otro lado, $2T(1,1) = 2 \cdot 1 \cdot 1 = 2$. 

    \vskip .3cm
    \ref{transf-lineales-d} No, no es una transformación lineal. Por ejemplo, veamos que $T(0,0) = (0,0,1)\ne (0,0,0)$.

    \vskip .3cm
    \ref{transf-lineales-e} No, no es una transformación lineal. Por ejemplo,  $\operatorname{det}(2\Id_2) = 4 \ne 2 = 2 \operatorname{det}(\Id_2)$. En general, $\operatorname{det}(\lambda A) = \lambda^n \operatorname{det}(A)$, donde $n$ es el tamaño de la matriz $A$.
    
    
    \item Sea $T:\mathbb{C}\longrightarrow\mathbb{C}$, $T(z)=\overline{z}$.
    \begin{enumerate}
    \item\label{conj-C} Considerar a $\mathbb{C}$ como un $\mathbb{C}$-espacio vectorial y decidir si $T$ es una transformación lineal.
    \item\label{conj-R} Considerar a $\mathbb{C}$ como un $\mathbb{R}$-espacio vectorial y decidir si $T$ es una transformación lineal.
    \end{enumerate}
    
    \rta

    \ref{conj-C} No, no es una transformación lineal. Por ejemplo, $T(2i) = -2i \ne 2i = 2T(i)$.
    
    \vskip .3cm
    \ref{conj-R} Sí, es una transformación lineal. En efecto, si $a+bi, a'+b'i \in \mathbb{C}$ y $\lambda \in \mathbb{R}$, entonces  
    \begin{align*}
        T((a+bi) + \lambda (a'+b'i)) &= T((a+\lambda a') + (b+\lambda b')i) = (a+\lambda a') - (b+\lambda b')i \\
        &= (a-bi) + \lambda (a'-b'i) = T(a+bi) + \lambda T(a'+b'i).
    \end{align*}


    
    
    \item\label{T en la base} Sea $T:\mathbb{K}^3\longrightarrow\mathbb{K}^3$ una transformación lineal tal que $T(e_1)=(1,2,3)$, $T(e_2)=(-1,0,5)$ y $T(e_3)=(-2,3,1)$. 
        \begin{enumerate}
        \item\label{T en dos vectores} Calcular $T(2,3,8)$ y $T(0,1,-1)$. 
        \item\label{T en la base b} Calcular $T(x,y,z)$ para todo $(x,y,z)\in\mathbb{K}^3$. Es decir, dar una fórmula para $T$ como la del ejercicio \ref{Txyz}.
        \item\label{matriz otro}  Encontrar una matriz $A\in\mathbb{K}^{3\times3}$ tal que
        $T(x,y,z)=A\begin{bmatrix}
        x\\y\\z \end{bmatrix}$. En esta parte del ejercicio escribiremos/pensaremos a los vectores de $\mathbb{K}^3$ como columnas.
        \end{enumerate}
    
        \rta

        \ref{T en dos vectores} 
        \begin{align*}
            T(2,3,8) &= T(2e_1 + 3e_2 + 8e_3) = 2T(e_1) + 3T(e_2) + 8T(e_3)
            \\
            & = 2(1,2,3) + 3(-1,0,5) +8(-2,3,1) \\
            &= (2,4,6) + (-3,0,15) + (-16,24,8) \\
            &= (-17,28,29).
        \end{align*}
        \begin{align*}
            T(0,1,-1) &= T(0e_1 + 1e_2 - 1e_3) = 0T(e_1) + 1T(e_2) - 1T(e_3)\\
            & = 0(1,2,3) + 1(-1,0,5) - 1(-2,3,1)\\ &= (0,0,0)+(-1,0,5)+(2,-3,-1)\\
            & = (1,-3,4).
        \end{align*}

        \vskip .3cm
        \ref{T en la base b} 
        \begin{align*}
            T(x,y,z) &= T(xe_1 + ye_2 + ze_3) = xT(e_1) + yT(e_2) + zT(e_3)
            \\
            & = x(1,2,3) + y(-1,0,5) +z(-2,3,1) \\
            & = (x,2x,3x) + (-y,0,5y) +(-2z,3z,z) \\
            &= (x-y-2z,2x+3z,3x+5y+z).
        \end{align*}

        \vskip .3cm
        \ref{matriz otro} Observar que  
        $$
        \begin{bmatrix} a&b&c \end{bmatrix} \begin{bmatrix} x\\y\\z \end{bmatrix} = ax+by+cz.
        $$ 
        Basándonos en esta observación, obtenemos la matrizs
        \begin{align*}
            A = \begin{bmatrix}
                1 & -1 & -2 \\
                2 & 0 & 3 \\
                3 & 5 & 1
            \end{bmatrix}.
        \end{align*}
        Entonces,
        \begin{align*}
            \begin{bmatrix}
                1 & -1 & -2 \\
                2 & 0 & 3 \\
                3 & 5 & 1
            \end{bmatrix} \begin{bmatrix}
                x\\y\\z
            \end{bmatrix} = \begin{bmatrix}
                x-y-2z\\2x+3z\\3x+5y+z
            \end{bmatrix} = T(x,y,z).
        \end{align*}
    
    \vskip .3cm
        
    \item\label{Txyz} Sea $T:\mathbb{K}^3\longrightarrow\mathbb{K}^3$ definida por $T(x,y,z)=(x+2y+3z, y-z,x+5y)$.
    \begin{enumerate}
    \item\label{Txyz-vectores-nucleo} Decir cuáles de los siguientes vectores están en el núcleo: $(1,1,1)$, $(-5,1,1)$.
    \item\label{Txyz imagen} Decir cuáles de los siguientes vectores están en la imagen: $(0,1,0)$, $(0,1,3)$.
    \item\label{Txyz nucleo T implicito} Describir mediante ecuaciones (implícitamente) el núcleo de $T$.
    \item\label{Txyz imagen T generadores} Dar un conjunto de generadores de la imagen.
    \item\label{matriz de T} Encontrar una matriz  $A\in\mathbb{K}^{3\times 3}$ tal que $T(x,y,z)=A\begin{bmatrix}
        x\\y\\z \end{bmatrix}
        $.  Como en el ejercicio  \ref{T en la base}\,\ref{matriz otro} pensamos a los vectores como columnas.
    \end{enumerate}
    
    \rta

    \ref{Txyz-vectores-nucleo} 
    $$
    T(1,1,1) = (1+2+3,1-1,1+5) = (6,0,6) \ne (0,0,0),
    $$
    por lo tanto $(1,1,1) \notin \operatorname{Nu}(T)$.

    $$
    T(-5,1,1) = (-5+2+3,1-1,-5+5) = (0,0,0),
    $$
    por lo tanto $(-5,1,1) \in \operatorname{Nu}(T)$.

    \vskip .3cm
    \ref{Txyz imagen}
    
    Supongamos que $T(x,y,z) = (1,1,1)$. Entonces,
    \begin{align*}
        (x+2y+3z, y-z,x+5y) &= (1,1,1) \\
        \Rightarrow \quad x+2y+3z &= 1, \\
        y-z &= 1, \\
        x+5y &= 1.
    \end{align*} 
    Resolvamos el sistema:
    \begin{align*}
        &\begin{amatrix}{3}
            1 & 2 & 3 &1 \\
            0 & 1 & -1 &1\\
            1 & 5 & 0&1
        \end{amatrix} 
        \stackrel{F_3-F_1}{\longrightarrow}
        \begin{amatrix}{3}
            1 & 2 & 3 &1 \\
            0 & 1 & -1 &1\\
            0 & 3 & -3&0
        \end{amatrix}
        \stackrel{F_3-3F_2}{\longrightarrow}
        \begin{amatrix}{3}
            1 & 2 & 3 &1 \\
            0 & 1 & -1 &1\\
            0 & 0 & 0&-3
        \end{amatrix}.
    \end{align*}
    La última fila de la última matriz nos dice que el sistema no tiene solución y por lo tanto $(1,1,1) \notin \operatorname{Im}(T)$.

    \vskip .3cm
    Supongamos que $T(x,y,z) = (0,1,3)$. Entonces,
    \begin{align*}
        (x+2y+3z, y-z,x+5y) &= (0,1,3) \\
        \Rightarrow \quad x+2y+3z &= 0, \\
        y-z &= 1, \\
        x+5y &= 3.
    \end{align*}
    Resolvamos el sistema:
    \begin{align*}
        &\begin{amatrix}{3}
            1 & 2 & 3 &0 \\
            0 & 1 & -1 &1\\
            1 & 5 & 0&3
        \end{amatrix} 
        \stackrel{F_3-F_1}{\longrightarrow}
        \begin{amatrix}{3}
            1 & 2 & 3 &0 \\
            0 & 1 & -1 &1\\
            0 & 3 & -3&3
        \end{amatrix}
        \stackrel{F_3-3F_2}{\longrightarrow}
        \begin{amatrix}{3}
            1 & 2 & 3 &0 \\
            0 & 1 & -1 &1\\
            0 & 0 & 0&0
        \end{amatrix} \\
        &\stackrel{F_1-2F_2}{\longrightarrow}
        \begin{amatrix}{3}
            1 & 0 & 5 &-2 \\
            0 & 1 & -1 &1\\
            0 & 0 & 0&0
        \end{amatrix}.
    \end{align*}
    Por lo tanto, $x = -2-5z$, $y = 1+z$ y $z$ es libre. Es decir, el sistema tiene infinitas soluciones, por ejemplo, para $z=0$ obtenemos $(x,y,z) = (-2,1,0)$ y es claro entonces que $T(-2,1,0) = (0,1,3)$.  Para $z=1$ obtenemos $(x,y,z) = (-7,2,1)$. En conclusión, $(0,1,3) \in \operatorname{Im}(T)$. 

    \vskip .3cm
    \ref{Txyz nucleo T implicito}   un vector $(x,y,z)$ está en el núcleo de $T$ si y sólo si $T(x,y,z) = (0,0,0)$. Es decir, si y sólo si
    \begin{align*}
        (x+2y+3z, y-z,x+5y) &= (0,0,0) \\
        \Rightarrow \quad x+2y+3z &= 0, \\
        y-z &= 0, \\
        x+5y &= 0.
    \end{align*}
    Resolvamos el sistema:
    \begin{align*}
        &\begin{bmatrix}
            1 & 2 & 3 \\
            0 & 1 & -1\\
            1 & 5 & 0
        \end{bmatrix} 
        \stackrel{F_3-F_1}{\longrightarrow}
        \begin{bmatrix}
            1 & 2 & 3 \\
            0 & 1 & -1\\
            0 & 3 & -3
        \end{bmatrix}
        \stackrel{F_3-3F_2}{\longrightarrow}
        \begin{bmatrix}
            1 & 2 & 3\\
            0 & 1 & -1\\
            0 & 0 & 0
        \end{bmatrix} 
        \stackrel{F_1-2F_2}{\longrightarrow}
        \begin{bmatrix}
            1 & 0 & 5 \\
            0 & 1 & -1\\
            0 & 0 & 0
        \end{bmatrix}.
    \end{align*}
    Por lo tanto, $(x,y,z) \in \operatorname{Nu}(T)$ si y solo si  $x+5z=0$, $y - z=0$. Es decir,
    $$
    \operatorname{Nu}(T) = \{(x,y,z) \in \mathbb{K}^3: x+5z=0, y - z=0\}.
    $$

    \vskip .3cm
    \ref{Txyz imagen T generadores} (primera forma) Debemos averiguar, los vectores $(x,y,z)$ tales que $T(x,y,z) = (b_1,b_2,b_3)$ para algún $ (b_1,b_2,b_3) \in \mathbb{K}^3$. Es decir, debemos averiguar, los vectores $(x,y,z)$ tales que
    \begin{align*}
        (x+2y+3z, y-z,x+5y) &= (b_1,b_2,b_3) \\
        \Rightarrow \quad x+2y+3z &= b_1, \\
        y-z &= b_2, \\
        x+5y &= b_3.
    \end{align*}
    Resolvamos el sistema con matrices aumentadas:
    \begin{align*}
        &\begin{amatrix}{3}
            1 & 2 & 3 &b_1 \\
            0 & 1 & -1 &b_2\\
            1 & 5 & 0&b_3
        \end{amatrix} 
        \stackrel{F_3-F_1}{\longrightarrow}
        \begin{amatrix}{3}
            1 & 2 & 3 &b_1 \\
            0 & 1 & -1 &b_2\\
            0 & 3 & -3&b_3-b_1
        \end{amatrix}
        \stackrel{F_3-3F_2}{\longrightarrow}
        \begin{amatrix}{3}
            1 & 2 & 3 &b_1 \\
            0 & 1 & -1 &b_2\\
            0 & 0 & 0&b_3-b_1-3b_2
        \end{amatrix} \\
        &\stackrel{F_1-2F_2}{\longrightarrow}
        \begin{amatrix}{3}
            1 & 0 & 5 &b_1-2b_2 \\
            0 & 1 & -1 &b_2\\
            0 & 0 & 0&b_3-b_1-3b_2
        \end{amatrix}.
    \end{align*}
    El sistema tiene solución si y solo si $b_3-b_1-3b_2 = 0$. Es decir,
    \begin{align*}
        \operatorname{Im}(T) &= \{(b_1,b_2,b_3) \in \mathbb{K}^3: b_3-b_1-3b_2 = 0\} \\ &= \{(b_1,b_2,b_3) \in \mathbb{K}^3: b_3 = b_1+3b_2\} \\
        &= \{(b_1,b_2,b_1+3b_2) : b_1,b_2 \in \mathbb{K}\} \\
        &= \{(b_1,0,b_1)  + (0,b_2,3b_2) : b_2, b_3 \in \mathbb{K}\} \\
        &= \langle (1,0,1), (0,1,3) \rangle.
    \end{align*}
    Es decir que $(1,0,1), (0,1,3)$ son generadores de $\operatorname{Im}(T)$.

    \ref{Txyz imagen T generadores} (segunda forma, más sencilla) Sabemos que $T$ de una base es un conjunto de generadores de la imagen. Por lo tanto, $\{T(e_1),T(e_2),T(e_3)\}$ es un conjunto de generadores de la imagen. Ahora bien, $T(e_1) = (1,0,1)$, $T(e_2) = (2,1,5)$ y $T(e_3) = (3,-1,0)$. Por lo tanto,  $(1,0,1),(2,1,5), (3,-1,0)$ son generadores de $\operatorname{Im}(T)$.

    La solución de esta parte del ejercicio termina en el párrafo anterior, pero si queremos obtener una base de la imagen, planteamos una matriz donde las filas son las vectores y con Gauss obtenemos una MRF. Las filas no nulas serán una base de la imagen.
    
    En  este caso:
    \begin{align*}
        &\begin{bmatrix}
            1 & 0 & 1 \\
            2 & 1 & 5\\
            3 & -1 & 0
        \end{bmatrix}
        \underset{F_3-3F_1}{\stackrel{F_2-2F_1}{\longrightarrow}}
        \begin{bmatrix}
            1 & 0 & 1 \\
            0 & 1 & 3\\
            0 & -1 & -3
        \end{bmatrix}
        \stackrel{F_3+F_2}{\longrightarrow}
        \begin{bmatrix}
            1 & 0 & 1 \\
            0 & 1 & 3\\
            0 & 0 & 0
        \end{bmatrix}.    
    \end{align*}
    Por lo tanto, $\{(1,0,1),(0,1,3)\}$ es una base de la imagen de $T$.


    \vskip .3cm
    \ref{matriz de T} Observar  que 
    $$
    F_i(A)\cdot \begin{bmatrix} x\\y\\z \end{bmatrix} = T(x,y,z)_i.
    $$
    Es decir la fila $i$  de $A$ por el vector $(x,y,z)$ nos da la coordenada $i$ de $T(x,y,z)$. Como $T(x,y,z) = (x+2y+3z, y-z,x+5y)$, tenemos que
    $$
    A \begin{bmatrix} x\\y\\z \end{bmatrix} = \begin{bmatrix} x+2y+3z\\ y-z\\x+5y \end{bmatrix},
    $$
    y por lo tanto
    \begin{align*}
        A = \begin{bmatrix}
            1 & 2 & 3 \\
            0 & 1 & -1\\
            1 & 5 & 0
        \end{bmatrix}.  
    \end{align*}




    
    
    \item\label{tl-matriz} Sea $T: \mathbb{K}^4 \to \mathbb{K}^5$ dada por $T(v) = Av$ donde $A$ es la siguiente matriz
        $$
        A=\begin{bmatrix}
        0& 2& 0&1\\   1& 3& 0&1\\  -1&-1&0&0\\3&0&3&0\\2&1&1&0 \end{bmatrix}
        $$
        \begin{enumerate}[topsep=5pt,itemsep=5pt]
            \item\label{tl-matriz-a} Dar una base del núcleo y de la imagen de $T$. 
            \item\label{tl-matriz-b} Dar la dimensión del núcleo y de la imagen de $T$.
            \item\label{tl-matriz-c} Describir mediante ecuaciones (implícitamente) el núcleo y la imagen de $T$.
            \item\label{tl-matriz-d} Decir cuáles de los siguientes vectores están en el núcleo:
            $(1,2,3,4)$, $(1,-1,-1,2)$, $(1,0,2,1)$.
            \item\label{tl-matriz-e} Decir cuáles de los siguientes vectores están en la imagen:
            $(2,3,-1,0,1)$, $(1,1,0,3,1)$, $(1,0,2,1,0)$.
        \end{enumerate}
        
    \rta

        \item Sea $T:\mathbb{K}^{2\times 2}\longrightarrow\mathbb{K}_{4}[x]$ la transformación lineal definida por
        \begin{align*}
        T   \begin{bmatrix}  a&b\\c&d \end{bmatrix} &= (a-c+2d)x^3+(b+2c-d)x^2+ \\
        &\qquad+(-a+2b+5c-4d)x+(2a-b-4c+5d)
        \end{align*}
        \begin{enumerate}
            \item\label{tl-matrices-pol-a} Decir cuáles de los siguientes matrices están en el núcleo:
                \begin{align*}
                    A=\begin{bmatrix}
                        2&0\\0&-1
                    \end{bmatrix},
                \quad
                B=\begin{bmatrix}
                    -1&-1\\1&1
                \end{bmatrix},
                \quad
                C=\begin{bmatrix}
                    -1&-1\\1&0
                \end{bmatrix}.
                \end{align*}
            \item\label{tl-matrices-pol-b} Decir cuáles de los siguientes polinomios están en la imagen:
                \begin{align*}
                    p(x)=x^3+x^2+x+1,\quad q(x)=x^3, \quad r(x)=(x-1)(x-1) 
                \end{align*}
        \end{enumerate}
    
    \rta
    
    \item\label{funcional ej}  Sea $T:\mathbb{K}^3\longrightarrow\mathbb{K}$ definida por $T(x,y,z)=x+2y+3z$.
    \begin{enumerate}
        \item\label{funcional ej a} Probar que $T$ es un epimorfismo.
        \item\label{funcional ej b} Dar la dimensión del núcleo de $T$.
        \item\label{funcional ej c} Encontrar una matriz $A$ tal que
            $T(x,y,z)=A\begin{bmatrix}
            x\\y\\z \end{bmatrix}$. ¿De qué tamaño debe ser $A$? Como en el ejercicio \ref{Txyz}\, \ref{matriz} pensamos a los vectores como columnas. 
    \end{enumerate}

    \rta
    
    
    \item Determinar cuáles transformaciones lineales de los ejercicios anteriores son monomorfismos, epimorfismos y/o isomorfismos.
    
    \rta


    \item\label{usar-1} Encontrar en cada caso, cuando sea posible, una matriz $A\in\mathbb{K}^{3\times 3}$ tal que la transformación lineal $T:\mathbb{K}^3\longrightarrow\mathbb{K}^3$, $T(v)=Av$, satisfaga las condiciones exigidas (como en el ejercicio  \ref{T en la base}\,\ref{matriz otro} pensamos a los vectores como columnas). Cuando no sea posible, explicar por qué no es posible.
    \begin{enumerate}[ topsep=5pt,itemsep=5pt]
        \item\label{usar-1 a} $\operatorname{dim} \operatorname{Im}(T)=2$ y $\operatorname{dim}\operatorname{Nu}(T)=2$.
        \item\label{usar-1 b} $T$ inyectiva y $T(e_1)=(1,0,0)$, $T(e_2)=(2,1,5)$ y $T(e_3)=(3,-1,0)$.
        \item\label{usar-1 c} $T$ sobreyectiva y $T(e_1)=(1,0,0)$, $T(e_2)=(2,1,5)$ y $T(e_3)=(3,-1,0)$.
        
        \item\label{usar Txyz} $T(e_1)=(1,0,0)$, $T(e_2)=(2,1,5)$ y $T(e_3)=(3,-1,0)$.
        
        \item\label{usar-1 e} $e_1\in\operatorname{Im}(T)$ y $(-5,1,1)\in\operatorname{Nu}(T)$.
        
        \item\label{usar-1 f} $\operatorname{dim} \operatorname{Im}(T)=2$.
    \end{enumerate}
        
    \rta 


    \item Decidir si las siguientes afirmaciones son verdaderas o falsas. Justificar.
    \begin{enumerate}
        \item\label{tl-VoF-a} Si $T : \mathbb R^{13} \to \mathbb R^9$ es una transformación lineal, entonces $\dim \operatorname{Nu}(T) \geq  4$.
        \item\label{tl-VoF-b} Sea $T:\mathbb{K}^{6}\longrightarrow\mathbb{K}^2$ un epimorfismo y $W$ un subespacio de $\mathbb{K}^{6}$ con $\dim W=3$. Entonces existe $0\neq w\in W$ tal que $T(w)=0$.
        \item\label{tl-VoF-c} Existe una transformación lineal $T : \mathbb R^2 \to \mathbb R^4$ tal que los vectores $(1, 0, -1, 2)$, $(0, 1, 2,-1,)$ y $(0, 0, 2, 2)$ pertenecen a la imagen de $T$.
    \end{enumerate}
    
    \rta



    \item \label{funcionales} Sea $V$ un espacio vectorial no nulo y $T:V\longrightarrow\mathbb{K}$ probar que $T=0$ ó $T$ es sobreyectiva.
    
    \rta


    \item Sea $V$ un espacio vectorial de dimensión finita y $T:V\longrightarrow V$ una transformación lineal. Probar las siguientes afirmaciones.
        \begin{multicols}{2}
            \begin{enumerate}
                \item $\operatorname{Nu}(T)\subseteq\operatorname{Nu}(T^2)$
                \item\label{dimV impar} $\operatorname{Nu}(T)\neq\operatorname{Im}(T)$ si $\dim(V)$ es impar.
            \end{enumerate}
        \end{multicols}
    
    \rta
        
    \end{enumerate}