% PDFLaTeX
\documentclass[a4paper,12pt,twoside,spanish,reqno]{amsbook}
%%%---------------------------------------------------
\usepackage[math]{kurier}

\usepackage{etex}
\usepackage{t1enc}
\usepackage{latexsym}
\usepackage[utf8]{inputenc}
\usepackage{verbatim}
\usepackage{multicol}
\usepackage{amsgen,amsmath,amstext,amsbsy,amsopn,amsfonts,amssymb}
\usepackage{amsthm}
\usepackage{calc}         % From LaTeX distribution
\usepackage{graphicx}     % From LaTeX distribution
\usepackage{ifthen}
\input{random.tex}   
\usepackage{tikz}
\usetikzlibrary{arrows}
\usetikzlibrary{matrix}
\usepackage{mathtools}
\usepackage{stackrel}
\usepackage{enumitem}
\usepackage{tkz-graph}

\usepackage{enumitem} 
\usepackage[compatibility=false]{caption} % para usar subcaption
\usepackage{subcaption} % para poner varias imagenes juntas
\usetikzlibrary{arrows.meta}
\usepackage{hyperref}
\hypersetup{ 
    colorlinks=true,
    linkcolor=blue,
    filecolor=magenta,      
    urlcolor=cyan,
}
\usepackage{hypcap}
\numberwithin{equation}{section}
% http://www.texnia.com/archive/enumitem.pdf (para las labels de los enumerate)
\renewcommand\labelitemi{$\circ$}
\setlist[enumerate, 1]{label={(\arabic*)}}
\setlist[enumerate, 2]{label=\emph{\alph*)}}


\newcommand{\rta}{\noindent\textsc{Solución: }} 

\newcommand{\img}{\operatorname{Im}}
\newcommand{\nuc}{\operatorname{Nu}}
\newcommand\im{\operatorname{Im}}
\renewcommand\nu{\operatorname{Nu}}
\newcommand{\la}{\langle}
\newcommand{\ra}{\rangle}
\renewcommand{\t}{{\operatorname{t}}}
\renewcommand{\sin}{{\,\operatorname{sen}}}
\newcommand{\Q}{\mathbb Q}
\newcommand{\R}{\mathbb R}
\newcommand{\C}{\mathbb C}
\newcommand{\K}{\mathbb K}
\newcommand{\F}{\mathbb F}
\newcommand{\Z}{\mathbb Z}
\newenvironment{amatrix}[1]{%
  \left[\begin{array}{@{}*{#1}{c}|c@{}}
}{%
  \end{array}\right]
}

%%% FORMATOS %%%%%%%%%%%%%%%%%%%%%%%%%%%%%%%%%%%%%%%%%%%%%%%%%%%%%%%%%%%%%%%%%%%%%
\tolerance=10000
\renewcommand{\baselinestretch}{1.2}
\usepackage[a4paper, top=3cm, left=3cm, right=2cm, bottom=2.5cm]{geometry}
\usepackage{setspace}
%\setlength{\parindent}{0,7cm}% tamaño de sangria.
\setlength{\parskip}{0,4cm} % separación entre parrafos.
%\renewcommand{\baselinestretch}{0.90}% separacion del interlineado
%%%%%%%%%%%%%%%%%%%%%%%%%%%%%%%%%%%%%%%%%%%%%%%%%%%%%%%%%%%%%%%%%%%%%%%%%%%%%%%%%%%
%\end{comment}
%%% FIN FORMATOS  %%%%%%%%%%%%%%%%%%%%%%%%%%%%%%%%%%%%%%%%%%%%%%%%%%%%%%%%%%%%%%%%%

\begin{document}
    \baselineskip=0.55truecm %original

    {\bf \begin{center} Práctico 5 \\ Álgebra  II -- Año 2024/1 \\ FAMAF \end{center}}

	
	\centerline{\textsc{Autovalores y autovectores}}
	
	
\subsection*{Objetivos}
	
\begin{itemize}
\item Familiarizarse con las nociones de autovalor y autovector de una matriz cuadrada.

\item Aprender a calcular el polinomio característico, los autovalores, y los autoespacios de una matriz cuadrada.
\end{itemize}
	
	
\subsection*{Ejercicios} Los ejercicios con el símbolo $\textcircled{a}$ tienen una ayuda al final del archivo para que recurran a ella después de pensar un poco.


\begin{enumerate}[topsep=6pt,itemsep=.4cm]


\item\label{autovalores} Para cada una de las siguientes matrices, hallar sus autovalores reales, y para cada autovalor, dar una descripción paramétrica del autoespacio asociado sobre $\mathbb{R}$.

$$
\textrm{(a)\; } \left[\begin{matrix} 2 & 1\\ -1 & 4 \end{matrix} \right],
\quad
\textrm{(b)\; }
\left[\begin{matrix} 1 & 0\\ 1 & -2 \end{matrix} \right],
\quad \textrm{(c)\; } \left[\begin{matrix}2 & 0 & 0\\ -1 & 1& -1\\ 0 & 0 & 2 \end{matrix} \right] ,\quad
\textrm{(d)\; } \begin{bmatrix} 3 & -5 \\ 1 & -1 \end{bmatrix},
$$

$$
\textrm{(e)\; } \begin{bmatrix} \lambda & 0 & 0 \\ 1 & \lambda & 0\\ 0 & 1 & \lambda \\ \end{bmatrix}, \; \lambda\in \mathbb R,\quad
\textrm{(f)\; } \left[\begin{matrix}1 & 0& 0 \\ 0 & \cos\theta & \operatorname{sen}\theta\\ 0 & -\operatorname{sen}\theta & \cos\theta \end{matrix} \right], 0\leq \theta<2\pi.
$$ 


\item\label{autovalores-complejos} Calcular los autovalores complejos de las matrices (d) y (f) del ejercicio anterior, y para cada autovalor, dar una descripción paramétrica del autoespacio asociado sobre $\mathbb{C}$.
\end{enumerate}

\textbf{Observación.} Es oportuno destacar algunos fenómenos que podemos observar en los ejercicios \ref{autovalores}-\ref{autovalores-complejos}.
\begin{itemize}
 \item[(i)] Una matriz con coeficientes reales puede no tener autovalores reales pero sí complejos (matriz $(d)$) o tener ambos (matriz $(f)$).
 \item[(ii)] Para describir paramétricamente los autoespacios podemos necesitar distintas cantidades de parámetros para los distintos autovalores (la matriz $(c)$). Esta cantidad mínima de parámetros es lo que llamaremos {\it dimensión}.
 \item[(iii)] La cantidad de autovalores distintos es menor o igual al tama\~no de la matriz. Incluso puede tener un sólo autovalor (matriz $(d)$ y más generalmente la matriz $(e)$ del Ejercicio \ref{mas}) o tener tantos como el tama\~no (matriz $(b)$ y $(f)$).
\end{itemize}

\begin{enumerate}[resume,topsep=6pt,itemsep=.4cm]

\item Probar que hay una única matriz $A\in\mathbb{R}^{2\times 2}$ tal que $(1,1)$ es autovector de autovalor $2$, y $(-2,1)$ es autovector de autovalor $1$.
    

\item Sea $A\in\mathbb{K}^{n\times n}$, y sea $f(x) = ax^2+bx+c$ un polinomio, con $a,b,c\in\mathbb{K}$. Sea $f(A)$ la matriz $n \times n$ definida por
$$f(A) = a A^2+bA+c\operatorname{Id}_n.$$
Probar que todo autovector de $A$ con autovalor $\lambda$ es autovector de $f(A)$ con autovalor $f(\lambda)$.

	
\item Sea $A\in\mathbb{K}^{2\times 2}$.

	\begin{enumerate} 	
		\item Probar que el polinomio característico de $A$ es \ $\chi_A(x) = x^2-\operatorname{Tr}(A)x+\det(A)$.
		\item Si $A$ no es invertible, probar que los autovalores de  $A$ son $0$ y $\operatorname{Tr}(A)$.
	\end{enumerate}

\item Sea $A\in\mathbb{K}^{n\times n}$. Probar que el polinomio $\tilde\chi_A(x)=\det(x\operatorname{Id}_n-A)$ y el polinomio característico de $A$ tienen las mismas raíces.

\end{enumerate}

\textbf{Observación} Algunos libros definen el polinomio característico de la matriz $A$ como $\tilde\chi_A(x)=\det(x\operatorname{Id}_n-A)$. Como vemos en el ejercicio anterior, ambas definiciones sirven para encontrar autovalores de $A$. El polinomio $\tilde\chi_A(x)$ tiene la particularidad de ser mónico, o sea que el coeficiente del término $x^n$ es $1$.

\begin{enumerate}[resume,topsep=6pt,itemsep=.4cm]

\item Probar que si $A\in\mathbb{K}^{n\times n}$ es una matriz nilpotente entonces $0$ es el único autovalor de $A$. Usar esto para deducir que la matriz $\operatorname{Id}_n-A$ es invertible (esta es otra demostración del Ejercicio (13) del Práctico 3).


\item Decidir si las siguientes afirmaciones son verdaderas o falsas. Justificar.

\begin{enumerate}
	\item Existe una matriz invertible $A$ tal que $0$ es autovalor de $A$.
	\item  Si $A$ es invertible, entonces todo autovector de $A$ es autovector de $A^{-1}$.
\end{enumerate}

\end{enumerate}

\textbf{Ejercicios de repaso}
Si ya hizo los ejercicios anteriores continue con la siguiente guía. Los ejercicios que siguen son similares y le pueden servir para practicar antes de los exámenes.

\begin{enumerate}[resume,topsep=6pt,itemsep=.4cm]

\item\label{mas} Repetir los Ejercicios 1 y 2 con las siguientes matrices.

$$\textrm{(a)\; }\begin{bmatrix} 2 & 3 \\ -1 & 1
\end{bmatrix}, \qquad
\textrm{(b)\; }\begin{bmatrix} -9 & 4 & 4 \\ -8 & 3 & 4 \\ -16 & 8 & 7 \end{bmatrix}, \qquad \textrm{(c)\; } \left[\begin{matrix}4 & 4 & -12\\ 1 & -1 & 1\\ 5 & 3 & -11 \end{matrix} \right],$$

$$
\textrm{(d)\; } \left[\begin{matrix}2 & 1 & 0 & 0\\ -1 & 4 & 0 & 0\\ 0 & 0 & 1 & 1 \\ 0 & 0 & 3 & -1\end{matrix} \right],
\quad \textrm{(e)\; } \begin{bmatrix} \lambda & 0 & 0 & \dots & 0  \\ 1 & \lambda & 0 &\dots & 0  \\ 0 & 1 & \lambda&  \dots & 0  \\ \vdots & \vdots & \quad & \ddots & \vdots\\ 0 &  0&   \dots & 1  & \lambda \end{bmatrix}, \; \lambda\in \mathbb R.
$$


\item Sea $A\in\mathbb{K}^{n\times n}$, y sea $f(x) = a_0 + a_1 x + \dots + a_nx^n$, $n \geq 1$, $a_i\in\mathbb{K}$, $a_n \neq 0$, un polinomio. Sea $f(A)$ la matriz $n \times n$ definida por
$$f(A) = a_0 \operatorname{Id}_n + a_1 A + \dots + a_n A^n.$$
Probar que todo autovector de $A$ con autovalor $\lambda$ es autovector de $f(A)$ con autovalor $f(\lambda)$.


\item\label{caracteristico-otro} {En este ejercicio consideraremos el polinomio $\tilde\chi_A(x)=\det(x\operatorname{Id}_n-A)$.}

\begin{enumerate}
\item  Calcular el polinomio $\tilde\chi_A(x)$ de las siguientes matrices.
\begin{align*}
A_2 := \begin{bmatrix} 0 & -a_0 \\ 1 & -a_1
\end{bmatrix},\quad\quad
A_3 := \begin{bmatrix} 0 & 0 & -a_0 \\ 1 & 0 & -a_1 \\ 0 & 1 & -a_2
\end{bmatrix}.
\end{align*}
donde $a_0, a_1, a_2$ son escalares.

\item\label{matriz de un polinomio} $\textcircled{a}$ Sean $a_0, ..., a_{n-1}$ escalares. Calcular el polinomio $\tilde\chi_A(x)$ de
\begin{align*}
A_n := \begin{bmatrix} 0 & 0 & 0 &\dots & 0 & -a_0 \\ 1 & 0 & 0&  \dots & 0  & -a_1 \\ 0 & 1 & 0&  \dots & 0  & -a_2 \\ \vdots & \vdots & \ddots & \quad  & \vdots\\ 0 & 0 & 0 & \dots & 1  & -a_{n-1}
\end{bmatrix}.		
 \end{align*}

 \item Deducir que dado un polinomio mónico $p(x)$ siempre existe una matriz $A$ tal que $\tilde\chi_A(x)=p(x)$.


\end{enumerate}


\item\label{tr det}$\textcircled{a}$ {\bf En este ejercicio consideraremos el polinomio $\tilde\chi_A(x)=\det(x\operatorname{Id}_n-A)$.}
Sea $A\in\mathbb{K}^{n\times n}$, y \ $\tilde\chi_A(x) = x^n+a_{n-1}x^{n-1}+\cdots+a_0$. Probar que


\begin{enumerate}
	\item $a_0 = (-1)^n \det(A)$.
	\item $a_{n-1} = - \operatorname{Tr}(A)$.
\end{enumerate}


\item\label{complejos} $\textcircled{a}$ {\bf En este ejercicio consideraremos el polinomio $\tilde\chi_A(x)=\det(x\operatorname{Id}_n-A)$.} Sea \\ $A\in\mathbb{C}^{n\times n}$. Probar que si $\lambda_1,\dots,\lambda_n \in \mathbb C$ son los autovalores de $A$
(posiblemente repetidos), entonces se cumple que:


\begin{enumerate}
	\item $\det(A)=\lambda_1\cdots \lambda_n$.
	\item $\operatorname{Tr}(A)=\lambda_1+\cdots+\lambda_n$.
\end{enumerate}

\subsection*{Aclaración}
Los Ejercicios \ref{tr det} (b) y \ref{complejos} (b) no son fáciles de probar y no son evaluables. Pero ya que enunciamos los items (a) es interesante saber que valen los items (b).
% Demostración: Por hipotesis el polinomio característico de $A$ se descompone de la siguiente manera:
% \begin{align*}
% \chi_A(x)=(x-\lambda_1)(x-\lambda_2)\cdots(x-\lambda_n).
% \end{align*}
% Entonces podemos usar esta igualdad para calcular de distintas maneras los términos $0$ y $n-1$ del polinomio característico.
%
% Por el ejercicio anterior el término $0$ de $\chi_A(x)$ es $(-1)^{n}\det(A)$ y por la igualdad anterior es $(-1)^n\lambda_1\cdots\lambda_n$. De lo cual deducimos el item (a).
%
% También por el ejercicio anterior $-\operatorname{Tr}(A)$ es el término $n-1$ del polinomio característico mientras que por la igualdad anterior podemos ver que es igual a $-\lambda_1-\lambda_2-\cdots-\lambda_n$. De lo cual deducimos (b).

\end{enumerate}


\subsection*{Ayudas}

\ref{caracteristico-otro}\,\ref{matriz de un polinomio} Desarrollar el determinante por la primera fila y hacer inducción.


\ref{tr det} (a) Evaluar el polinomio $\tilde\chi_A(x)$ en un valor apropiado para obtener el término independiente $a_0$.


\ref{tr det} (b) Desarrollar el determinante de $x\operatorname{Id}-A$ por la primera columna y hacer inducción en el tama\~no de la matriz. Es decir, primero
\begin{align*}
\tilde\chi_A(x) = \det(x\operatorname{Id}-A)&=\\
(x-a_{11})\det((x\operatorname{Id}-A)(1|1))&+a_{21}\det((x\operatorname{Id}-A)(2|1))+\cdots+(-1)^n a_{n1}\det((x\operatorname{Id}-A)(n|1)).
\end{align*}
De estos sumandos, el único sumando donde hay $x^{n-1}$ es $(x-a_{11})\det((x\operatorname{Id}-A)(1|1))$. Además, $\det((x\operatorname{Id}-A)(1|1))$ es el polinomio característico de la submatriz $A(1|1)$. Podemos aplicar la hipótesis inductiva a esta matriz y deducir que el coeficiente de $x^{n-1}$ en el producto de polinomios $(x-a_{11})\det((x\operatorname{Id}-A)(1|1))$ es $- \operatorname{Tr}(A)$.



\ref{complejos} Sobre $\mathbb{C}$ podemos descomponer el polinomio $\tilde\chi_A(x)$ de la siguiente manera
\begin{align*}
\tilde\chi_A(x)=(x-\lambda_1)(x-\lambda_2)\cdots(x-\lambda_n). \qquad (\diamondsuit)
\end{align*}
Con esta igualdad podemos calcular los términos $a_0$ y $a_{n-1}$ de $\tilde\chi_A(x)$ de dos maneras. La primera es la obtenida en el Ejercicio \ref{tr det}. La segunda es usando la multiplicación del lado derecho de $(\diamondsuit)$. Para el término $a_0$ hay que evaluar en un valor apropiado. Para el término $a_{n-1}$ hay que notar que para obtener $x^{n-1}$ debemos elegir una $x$ de todos los factores salvo en uno y un término del estilo $-\lambda_i$. Igualando lo que obtengamos probamos el ejercicio.
\end{document}
