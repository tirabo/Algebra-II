\documentclass[12pt]{amsart}
\usepackage{amssymb}
\usepackage{enumerate}
\usepackage{amsmath}
\usepackage{geometry}
\geometry{ a4paper, total={210mm,297mm}, left=2cm, right=2cm, top=1.5cm, bottom=2.5cm, }
\usepackage{graphicx}
\usepackage{fancyhdr}
\usepackage{multicol}
\usepackage{enumitem}

\newcommand{\este}{($\star$) \; }

\pagestyle{fancy}



\begin{document}
	
	\noindent {\tiny \'Algebra / \'Algebra II \hfill Segundo Cuatrimestre 2020}
	
	\centerline{\Large{Pr\' actico 5}}
	
	\
	
	\centerline{\textsc{Autovalores y autovectores}}
	
	\
	
\subsection*{Objetivos}
	
\begin{itemize}
\item Familiarizarse con las nociones de autovalor y autovector de una matriz cuadrada.

\item Aprender a calcular el polinomio caracter\' \i stico, los autovalores, y los autoespacios de una matriz cuadrada.
\end{itemize}
	
	\
	
\subsection*{Ejercicios} Los ejercicios con el s\'imbolo $\textcircled{a}$ tienen una ayuda al final del archivo para que recurran a ella despu\'es de pensar un poco.

\

\begin{enumerate}


\item Para cada una de las siguientes matrices, hallar sus autovalores reales, y para cada autovalor, dar una descripci\' on param\'etrica del autoespacio asociado sobre $\mathbb{R}$.


$$\textrm{(a)\; } \left[\begin{matrix} 2 & 1\\ -1 & 4 \end{matrix} \right],
\quad
\textrm{(b)\; }
\left[\begin{matrix} 1 & 0\\ 1 & -2 \end{matrix} \right],
\quad \textrm{(c)\; } \left[\begin{matrix}2 & 0 & 0\\ -1 & 1& -1\\ 0 & 0 & 2 \end{matrix} \right] ,
\quad
\textrm{(d)\; } \begin{bmatrix} \lambda & 0 & 0 \\ 1 & \lambda & 0\\ 0 & 1 & \lambda \\ \end{bmatrix}, \; \lambda\in \mathbb R,$$


$$\textrm{(e)\; } \begin{bmatrix} 3 & -5 \\ 1 & -1 \end{bmatrix},
\quad
\textrm{(f)\; } \left[\begin{matrix}1 & 0& 0 \\ 0 & \cos\theta & \operatorname{sen}\theta\\ 0 & -\operatorname{sen}\theta & \cos\theta \end{matrix} \right], 0\leq \theta<2\pi.
$$

\

\item Calcular los autovalores complejos de las matrices (e) y (f) del ejercicio anterior, y para cada autovalor, dar una descripci\' on param\'etrica del autoespacio asociado sobre $\mathbb{C}$.


\subsection*{Observaci\'{o}n} Es oportuno destacar algunos fen\'omenos que podemos observar en los ejerecicios (1)-(2).

\

\begin{itemize}
 \item[(i)] Una matriz con coeficientes reales puede no tener autovalores reales pero s\'i complejos (matriz $(e)$) o tener ambos (matriz $(f)$).

 \

 \item[(ii)] Para describir param\'etricamente los autoespacios podemos necesitar distintas cantidades de par\'ametros para los distintos autovalores (la matriz $(c)$). Esta cantidad m\'inima de par\'ametros es lo que llamaremos {\it dimensi\'on}.

 \

 \item[(iii)] La cantidad de autovalores distintos es menor o igual al tama\~no de la matriz. Incluso puede tener un s\'olo autovalor (matriz $(d)$ y m\'as generalmente la matriz $(e)$ del Ejercicio \eqref{mas}) o tener tantos como el tama\~no (matriz $(b)$ y $(f)$).
\end{itemize}

\

\item Probar que hay una \'unica matriz $A\in\mathbb{R}^{2\times 2}$ tal que $(1,1)$ es autovector de autovalor $2$, y $(-2,1)$ es autovector de autovalor $1$.
    
\

\item Sea $A\in\mathbb{K}^{n\times n}$, y sea $f(x) = ax^2+bx+c$ un polinomio, con $a,b,c\in\mathbb{K}$. Sea $f(A)$ la matriz $n \times n$ definida por
$$f(A) = a A^2+bA+c\operatorname{Id}_n.$$
Probar que todo autovector de $A$ con autovalor $\lambda$ es autovector de $f(A)$ con autovalor $f(\lambda)$.

\
	
\item Sea $A\in\mathbb{K}^{2\times 2}$.

\

\begin{enumerate} 	
\item Probar que el polinomio caracter\'istico de $A$ es \ $\chi_A(x) = x^2-\operatorname{Tr}(A)x+\det(A)$.

\

\item Si $A$ no es invertible, probar que los autovalores de  $A$ son $0$ y $\operatorname{Tr}(A)$.
\end{enumerate}

\

\item Sea $A\in\mathbb{K}^{n\times n}$. Probar que el polinomio $\tilde\chi_A(x)=\det(x\operatorname{Id}_n-A)$ y el polinomio \linebreak caracter\'istico de $A$ tienen las mismas ra\'ices.

\subsection*{Observaci\'{o}n} Algunos libros definen el polinomio caracter\'istico de la matriz $A$ como $\tilde\chi_A(x)=\det(x\operatorname{Id}_n-A)$. Como vemos en el ejercicio anterior, ambas definiciones sirven para encontrar autovalores de $A$. El polinomio $\tilde\chi_A(x)$ tiene la particularidad de ser m\'onico, o sea que el coeficiente del t\'ermino $x^n$ es $1$.

\

\item Probar que si $A\in\mathbb{K}^{n\times n}$ es una matriz nilpotente entonces $0$ es el \' unico autovalor de $A$. Usar esto para deducir que la matriz $\operatorname{Id}_n-A$ es invertible (esta es otra demostraci\'on del Ejercicio (13) del Pr\'actico 3).

\


\item Decidir si las siguientes afirmaciones son verdaderas o falsas. Justificar.

\

\begin{enumerate}
\item Existe una matriz invertible $A$ tal que $0$ es autovalor de $A$.

\

\item  Si $A$ es invertible, entonces todo autovector de $A$ es autovector de $A^{-1}$.

\end{enumerate}

\end{enumerate}

\

\subsection*{Ejercicios de repaso}
Si ya hizo los ejercicios anteriores continue con la siguiente gu\'ia. Los ejercicios que siguen son similares y le pueden servir para practicar antes de los ex\'amenes.

\

\begin{enumerate}[resume]

\item\label{mas} Repetir los Ejercicios 1 y 2 con las siguientes matrices.

$$\textrm{(a)\; }\begin{bmatrix} 2 & 3 \\ -1 & 1
\end{bmatrix}, \qquad
\textrm{(b)\; }\begin{bmatrix} -9 & 4 & 4 \\ -8 & 3 & 4 \\ -16 & 8 & 7 \end{bmatrix}, \qquad \textrm{(c)\; } \left[\begin{matrix}4 & 4 & -12\\ 1 & -1 & 1\\ 5 & 3 & -11 \end{matrix} \right],$$

$$
\textrm{(d)\; } \left[\begin{matrix}2 & 1 & 0 & 0\\ -1 & 4 & 0 & 0\\ 0 & 0 & 1 & 1 \\ 0 & 0 & 3 & -1\end{matrix} \right],
\quad \textrm{(e)\; } \begin{bmatrix} \lambda & 0 & 0 & \dots & 0  \\ 1 & \lambda & 0 &\dots & 0  \\ 0 & 1 & \lambda&  \dots & 0  \\ \vdots & \vdots & \quad & \ddots & \vdots\\ 0 &  0&   \dots & 1  & \lambda \end{bmatrix}, \; \lambda\in \mathbb R.
$$

\

\item Sea $A\in\mathbb{K}^{n\times n}$, y sea $f(x) = a_0 + a_1 x + \dots + a_nx^n$, $n \geq 1$, $a_i\in\mathbb{K}$, $a_n \neq 0$, un polinomio. Sea $f(A)$ la matriz $n \times n$ definida por
$$f(A) = a_0 \operatorname{Id}_n + a_1 A + \dots + a_n A^n.$$
Probar que todo autovector de $A$ con autovalor $\lambda$ es autovector de $f(A)$ con autovalor $f(\lambda)$.

\

\item {\bf En este ejercicio consideraremos el polinomio $\tilde\chi_A(x)=\det(x\operatorname{Id}_n-A)$.}

\begin{enumerate}

\item  Calcular el polinomio $\tilde\chi_A(x)$ de las siguientes matrices.
\begin{align*}
A_2 := \begin{bmatrix} 0 & -a_0 \\ 1 & -a_1
\end{bmatrix},\quad\quad
A_3 := \begin{bmatrix} 0 & 0 & -a_0 \\ 1 & 0 & -a_1 \\ 0 & 1 & -a_2
\end{bmatrix}.
\end{align*}
donde $a_0, a_1, a_2$ son escalares.

\item\label{matriz de un polinomio} $\textcircled{a}$ Sean $a_0, ..., a_{n-1}$ escalares. Calcular el polinomio $\tilde\chi_A(x)$ de
\begin{align*}
A_n := \begin{bmatrix} 0 & 0 & 0 &\dots & 0 & -a_0 \\ 1 & 0 & 0&  \dots & 0  & -a_1 \\ 0 & 1 & 0&  \dots & 0  & -a_2 \\ \vdots & \vdots & \ddots & \quad  & \vdots\\ 0 & 0 & 0 & \dots & 1  & -a_{n-1}
\end{bmatrix}.		
 \end{align*}

 \item Deducir que dado un polinomio m\'onico $p(x)$ siempre existe una matriz $A$ tal que $\tilde\chi_A(x)=p(x)$.


\end{enumerate}

\

\item\label{tr det}$\textcircled{a}$ {\bf En este ejercicio consideraremos el polinomio $\tilde\chi_A(x)=\det(x\operatorname{Id}_n-A)$.}
Sea $A\in\mathbb{K}^{n\times n}$, y \ $\tilde\chi_A(x) = x^n+a_{n-1}x^{n-1}+\cdots+a_0$. Probar que

\ 

\begin{enumerate}
	\item $a_0 = (-1)^n \det(A)$.

\

	\item $a_{n-1} = - \operatorname{Tr}(A)$.
\end{enumerate}


% \
%
% Demostraci\'on de la parte de la $\operatorname{Tr}$. Notemos que la matriz $x\operatorname{Id}-A$ tiene la siguiente forma
% \begin{align*}
% x\operatorname{Id}-A=
% \left[\begin{matrix} x-a_{11}   & \cdots & -a_{1n}\\ -a_{21}  &  & \\ \vdots   & x\operatorname{Id}-A(1|1) &  \\ -a_{n1} &  &\end{matrix} \right].
% \end{align*}
% Si desarrollamos el determinante de $x\operatorname{Id}-A$ por la primera columna tenemos que
% \begin{align*}
% \det(x\operatorname{Id}-A)&=\\
% (x-a_{11})\det(x\operatorname{Id}-A(1|1))&-a_{21}\det((x\operatorname{Id}-A)(2|1))+\cdots+(-1)^n-a_{n1}\det((x\operatorname{Id}-A)(n|1)).
% \end{align*}
%
% Ahora observemos que $\det((x\operatorname{Id}-A)(2|1))$, ..., $\det((x\operatorname{Id}-A)(n|1))$ son polinomios de grado $n-2$. Entonces no contribuyen en el t\'ermino $n-1$ del polinomio $\det(x\operatorname{Id}-A)$. Entonces este t\'ermino es igual al t\'ermino $n-1$ del polin\'omio
% \begin{align*}
% (x-a_{11})\det(x\operatorname{Id}-A(1|1))=x\det(x\operatorname{Id}-A(1|1))-a_{11}\det(x\operatorname{Id}-A(1|1)).
% \end{align*}
% De esta igualdad podemos ver que el t\'ermino $n-1$ es igual al t\'ermino $n-2$ de $\det(x\operatorname{Id}-A(1|1))$ m\'as el t\'ermino $n-1$ de $\det(x\operatorname{Id}-A(1|1))$ multiplicado por $-a_{11}$.
%
% \
%
% Dado que $\det(x\operatorname{Id}-A(1|1))$ es el polinomio caracter\'istico de la matriz $A(1|1)$ y esta es una matriz cuadrada $n-1$ podemos deducir que el t\'ermino $n$ es $-a_{11}$ y (haciendo inducci\'on) el t\'ermino $n-1$ es $-\operatorname{Tr}(A(1|1))=-a_{22}-\cdots -a_{nn}$. Sumando ambas cosas obtenemos lo que quer\'iamos.

\

\item\label{complejos} $\textcircled{a}$ {\bf En este ejercicio consideraremos el polinomio $\tilde\chi_A(x)=\det(x\operatorname{Id}_n-A)$.} Sea \\ $A\in\mathbb{C}^{n\times n}$. Probar que si $\lambda_1,\dots,\lambda_n \in \mathbb C$ son los autovalores de $A$
(posiblemente repetidos), entonces se cumple que:

\

\begin{enumerate}
	\item $\det(A)=\lambda_1\cdots \lambda_n$.
	
	\
	
	\item $\operatorname{Tr}(A)=\lambda_1+\cdots+\lambda_n$.
\end{enumerate}

\subsection*{Aclaraci\'on}
Los Ejercicios \eqref{tr det} (b) y \eqref{complejos} (b) no son f\'aciles de probar y no son \linebreak evaluables. Pero ya que enunciamos los items (a) es interesante saber que valen los items (b).
% Demostraci\'on: Por hipotesis el polinomio caracter\'istico de $A$ se descompone de la siguiente manera:
% \begin{align*}
% \chi_A(x)=(x-\lambda_1)(x-\lambda_2)\cdots(x-\lambda_n).
% \end{align*}
% Entonces podemos usar esta igualdad para calcular de distintas maneras los t\'erminos $0$ y $n-1$ del polinomio caracter\'istico.
%
% Por el ejercicio anterior el t\'ermino $0$ de $\chi_A(x)$ es $(-1)^{n}\det(A)$ y por la igualdad anterior es $(-1)^n\lambda_1\cdots\lambda_n$. De lo cual deducimos el item (a).
%
% Tambi\'en por el ejercicio anterior $-\operatorname{Tr}(A)$ es el t\'ermino $n-1$ del polinomio caracter\'istico mientras que por la igualdad anterior podemos ver que es igual a $-\lambda_1-\lambda_2-\cdots-\lambda_n$. De lo cual deducimos (b).

\end{enumerate}


\

\subsection*{Ayudas}

\eqref{matriz de un polinomio} Desarrollar el determinante por la primera fila y hacer inducci\'on.

\

\eqref{tr det} (a) Evaluar el polinomio $\tilde\chi_A(x)$ en un valor apropiado para obtener el t\'ermino independiente $a_0$.

\

\eqref{tr det} (b) Desarrollar el determinante de $x\operatorname{Id}-A$ por la primera columna y hacer inducci\'on en el tama\~no de la matriz. Es decir, primero
\begin{align*}
\tilde\chi_A(x) = \det(x\operatorname{Id}-A)&=\\
(x-a_{11})\det((x\operatorname{Id}-A)(1|1))&+a_{21}\det((x\operatorname{Id}-A)(2|1))+\cdots+(-1)^n a_{n1}\det((x\operatorname{Id}-A)(n|1)).
\end{align*}
De estos sumandos, el \'unico sumando donde hay $x^{n-1}$ es $(x-a_{11})\det((x\operatorname{Id}-A)(1|1))$. Adem\'as, $\det((x\operatorname{Id}-A)(1|1))$ es el polinomio caracter\'istico de la submatriz $A(1|1)$. Podemos aplicar la hip\'{o}tesis inductiva a esta matriz y deducir que el coeficiente de $x^{n-1}$ en el producto de polinomios $(x-a_{11})\det((x\operatorname{Id}-A)(1|1))$ es $- \operatorname{Tr}(A)$.


\

\eqref{complejos} Sobre $\mathbb{C}$ podemos descomponer el polinomio $\tilde\chi_A(x)$ de la siguiente manera
\begin{align*}
\tilde\chi_A(x)=(x-\lambda_1)(x-\lambda_2)\cdots(x-\lambda_n). \qquad (\diamondsuit)
\end{align*}
Con esta igualdad podemos calcular los t\'erminos $a_0$ y $a_{n-1}$ de $\tilde\chi_A(x)$ de dos maneras. La primera es la obtenida en el Ejercicio \eqref{tr det}. La segunda es usando la multiplicaci\'on del lado derecho de $(\diamondsuit)$. Para el t\'ermino $a_0$ hay que evaluar en un valor apropiado. Para el t\'ermino $a_{n-1}$ hay que notar que para obtener $x^{n-1}$ debemos elegir una $x$ de todos los factores salvo en uno y un t\'{e}rmino del estilo $-\lambda_i$. Igualando lo que obtengamos probamos el ejercicio.
\end{document}
