
\chapter{Soluciones\\Álgebra  II -- Año 2024/1 -- FAMAF}\label{practico-5}

\begin{enumerate}[topsep=6pt,itemsep=.4cm]


    \item\label{autovalores} Para cada una de las siguientes matrices, hallar sus autovalores reales, y para cada autovalor, dar una descripción paramétrica del autoespacio asociado sobre $\mathbb{R}$.
    \begin{multicols}{2}
    \begin{enumerate}
        \item\label{autovalores-1} $\left[\begin{matrix} 2 & 1\\ -1 & 4 \end{matrix} \right]$,\vskip .6cm 
        \item\label{autovalores-2} $ \left[\begin{matrix} 1 & 0\\ 1 & -2 \end{matrix} \right]$, \vskip .5cm
        \item\label{autovalores-3} $\left[\begin{matrix}2 & 0 & 0\\ -1 & 1& -1\\ 0 & 0 & 2 \end{matrix} \right]$,\vskip .2cm
        \item\label{autovalores-4} $\begin{bmatrix} 3 & -5 \\ 1 & -1 \end{bmatrix}$,\vskip .2cm
        \item\label{autovalores-5}  $\begin{bmatrix} \lambda & 0 & 0 \\ 1 & \lambda & 0\\ 0 & 1 & \lambda \\ \end{bmatrix}$, $\lambda\in \mathbb R$
        \vskip .2cm
        \item\label{autovalores-6} $\left[\begin{matrix}1 & 0& 0 \\ 0 & \cos\theta & \operatorname{sen}\theta\\ 0 & -\operatorname{sen}\theta & \cos\theta \end{matrix} \right]$, $0\leq \theta<2\pi$.
    \end{enumerate}
    \end{multicols}


    \rta


    \vskip .2cm
    
    
    \item\label{autovalores-complejos} Calcular los autovalores complejos de las matrices \ref{autovalores-4} y \ref{autovalores-6} del ejercicio anterior, y para cada autovalor, dar una descripción paramétrica del autoespacio asociado sobre $\mathbb{C}$.
    
    \rta


    \vskip .2cm

    \end{enumerate}
    

    
    
    \begin{enumerate}[resume,topsep=6pt,itemsep=.4cm]
    
    \item Probar que hay una única matriz $A\in\mathbb{R}^{2\times 2}$ tal que $(1,1)$ es autovector de autovalor $2$, y $(-2,1)$ es autovector de autovalor $1$.
    
    \rta


    \vskip .2cm
    
        
    
    \item Sea $A\in\mathbb{K}^{n\times n}$, y sea $f(x) = ax^2+bx+c$ un polinomio, con $a,b,c\in\mathbb{K}$. Sea $f(A)$ la matriz $n \times n$ definida por
    $$f(A) = a A^2+bA+c\operatorname{Id}_n.$$
    Probar que todo autovector de $A$ con autovalor $\lambda$ es autovector de $f(A)$ con autovalor $f(\lambda)$.
    
    \rta


    \vskip .2cm
    
    
        
    \item Sea $A\in\mathbb{K}^{2\times 2}$.
    
        \begin{enumerate}     
            \item Probar que el polinomio característico de $A$ es \ $\chi_A(x) = x^2-\operatorname{Tr}(A)x+\det(A)$.
            \item Si $A$ no es invertible, probar que los autovalores de  $A$ son $0$ y $\operatorname{Tr}(A)$.
        \end{enumerate}
    
    \rta


    \vskip .2cm
        
        
    \item Sea $A\in\mathbb{K}^{n\times n}$. Probar que el polinomio $\tilde\chi_A(x)=\det(x\operatorname{Id}_n-A)$ y el polinomio característico de $A$ tienen las mismas raíces.
    
    \rta


    \vskip .2cm
    
    
    \end{enumerate}
    
    \begin{enumerate}[resume,topsep=6pt,itemsep=.4cm]
    
    \item Probar que si $A\in\mathbb{K}^{n\times n}$ es una matriz nilpotente entonces $0$ es el único autovalor de $A$. Usar esto para deducir que la matriz $\operatorname{Id}_n-A$ es invertible (esta es otra demostración del ejercicio \ref{nilpotene - id} del Práctico \ref{practico-3}).
    
    \rta


    \vskip .2cm
    
    
    
    \item Decidir si las siguientes afirmaciones son verdaderas o falsas. Justificar.
    \begin{enumerate}
        \item Existe una matriz invertible $A$ tal que $0$ es autovalor de $A$.
        \item  Si $A$ es invertible, entonces todo autovector de $A$ es autovector de $A^{-1}$.
    \end{enumerate}
    
    \rta


    \vskip .2cm
    
    
    \end{enumerate}
    
    \begin{enumerate}[resume,topsep=6pt,itemsep=.4cm]
    
    \item\label{mas} Repetir los ejercicios \ref{autovalores} y \ref{autovalores-complejos} con las siguientes matrices.
    \begin{multicols}{2}
        \begin{enumerate}
            \item\label{mas-autovalores-1} $\begin{bmatrix} 2 & 3 \\ -1 & 1
            \end{bmatrix} $,\vskip .6cm 
            \item\label{mas-autovalores-2} $\begin{bmatrix} -9 & 4 & 4 \\ -8 & 3 & 4 \\ -16 & 8 & 7 \end{bmatrix} $, \vskip .5cm
            \item\label{mas-autovalores-3} $\left[\begin{matrix}4 & 4 & -12\\ 1 & -1 & 1\\ 5 & 3 & -11 \end{matrix} \right] $,\vskip .2cm
            \item\label{mas-autovalores-4} $\left[\begin{matrix}2 & 1 & 0 & 0\\ -1 & 4 & 0 & 0\\ 0 & 0 & 1 & 1 \\ 0 & 0 & 3 & -1\end{matrix} \right] $,\vskip .2cm
            \item\label{mas-autovalores-5} $\begin{bmatrix} \lambda & 0 & 0 & \dots & 0  \\ 1 & \lambda & 0 &\dots & 0  \\ 0 & 1 & \lambda&  \dots & 0  \\ \vdots & \vdots & \quad & \ddots & \vdots\\ 0 &  0&   \dots & 1  & \lambda \end{bmatrix}$, $\lambda\in \mathbb R$. \vskip .2cm
        \end{enumerate}
        \end{multicols}

    \rta


    \vskip .2cm
    
    
    

    \end{enumerate}