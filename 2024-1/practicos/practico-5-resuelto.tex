
\chapter{Soluciones\\Álgebra  II -- Año 2024/1 -- FAMAF}\label{practico-5}

\begin{enumerate}[topsep=6pt,itemsep=.4cm]


    \item\label{autovalores} Para cada una de las siguientes matrices, hallar sus autovalores reales, y para cada autovalor, dar una descripción paramétrica del autoespacio asociado sobre $\mathbb{R}$.
    \begin{multicols}{2}
    \begin{enumerate}
        \item\label{autovalores-1} $\left[\begin{matrix} 2 & 1\\ -1 & 4 \end{matrix} \right]$,\vskip .6cm 
        \item\label{autovalores-2} $ \left[\begin{matrix} 1 & 0\\ 1 & -2 \end{matrix} \right]$, \vskip .5cm
        \item\label{autovalores-3} $\left[\begin{matrix}2 & 0 & 0\\ -1 & 1& -1\\ 0 & 0 & 2 \end{matrix} \right]$,\vskip .2cm
        \item\label{autovalores-4} $\begin{bmatrix}-1&0&0\\0& 3 & -5 \\ 0&1 & -1 \end{bmatrix}$,\vskip .2cm
        \item\label{autovalores-5}  $\begin{bmatrix} \lambda & 0 & 0 \\ 1 & \lambda & 0\\ 0 & 1 & \lambda \\ \end{bmatrix}$, $\lambda\in \mathbb R$
        \vskip .2cm
        \item\label{autovalores-6} $\left[\begin{matrix}\cos\theta & \sen\theta\\ -\sen\theta & \cos\theta \end{matrix} \right]$, $0\leq \theta<2\pi$.
    \end{enumerate}
    \end{multicols}


    \rta
    En todos los casos calculamos el polinomio característico, averiguamos sus raíces y luego resolvemos el sistema de ecuaciones que nos da el autoespacio asociado a cada autovalor.
    
    \vskip .2cm
    \ref{autovalores-1}
    Denominamos $A$  a la matriz del enunciado. El polinomio característico de $A$ es 
    \begin{align*}
        \chi_A(x) &= \left|\begin{matrix} x-2 & -1\\ 1 & x-4 \end{matrix} \right| \\ &= (x-2)(x-4)-(-1)(1) = x^2-6x+9 \\
        &= (x-3)^2.
    \end{align*}
    Luego $\lambda=3$ es el único autovalor. Para calcular el autoespacio asociado a $\lambda=3$ resolvemos el sistema  $(3\Id-A)v=0$, es decir
    \begin{align*}
        \begin{bmatrix} 3-2 & -1\\ 1 & 3-4 \end{bmatrix} &= \begin{bmatrix} 1 & -1\\ 1 & -1 \end{bmatrix} \begin{bmatrix} v_1\\ v_2 \end{bmatrix} = \begin{bmatrix} 0\\ 0 \end{bmatrix} \\
        \Rightarrow \begin{bmatrix} v_1-v_2\\ v_1-v_2 \end{bmatrix} = \begin{bmatrix} 0\\ 0 \end{bmatrix} &\Rightarrow v_1=v_2.
    \end{align*}
    Luego,  el único autovalor es $\lambda=3$ y el autoespacio asociado es $\{(t,t): t\in\mathbb{R}\}$.


    
    \vskip .2cm
    \ref{autovalores-2} Denominamos $B$  a la matriz del enunciado. El polinomio característico de $B$ es
    \begin{align*}
        \chi_B(x) &= \left|\begin{matrix} x-1 & 0\\ -1 & x+2 \end{matrix} \right| \\ &= (x-1)(x+2)-(-1)(0) = x^2+x-2 \\
        &= (x+2)(x-1).
    \end{align*}
    Luego $\lambda_1=-2$ y $\lambda_2=1$ son los autovalores. 
    
    Para calcular el autoespacio asociado a $\lambda_1=-2$ resolvemos el sistema  $(-2\Id-B)v=0$, es decir
    \begin{align*}
        \begin{bmatrix} -2-1 & 0\\ -1 & -2+2 \end{bmatrix} &= \begin{bmatrix} -3 & 0\\ -1 & 0 \end{bmatrix} \begin{bmatrix} v_1\\ v_2 \end{bmatrix} = \begin{bmatrix} 0\\ 0 \end{bmatrix} \\
        \Rightarrow \begin{bmatrix} -3v_1\\ -v_1 \end{bmatrix} = \begin{bmatrix} 0\\ 0 \end{bmatrix} &\Rightarrow v_1=0.
    \end{align*}
    Luego,  el autovalor $\lambda_1=-2$ tiene autoespacio asociado $\{(0,t): t\in\mathbb{R}\}$.
    Para calcular el autoespacio asociado a $\lambda_2=1$ resolvemos el sistema  $(\Id-B)v=0$, es decir
    \begin{align*}
        \begin{bmatrix} 1-1 & 0\\ -1 & 1+2 \end{bmatrix} &= \begin{bmatrix} 0 & 0\\ -1 & 3 \end{bmatrix} \begin{bmatrix} v_1\\ v_2 \end{bmatrix} = \begin{bmatrix} 0\\ 0 \end{bmatrix} \\
        \Rightarrow \begin{bmatrix} 0\\ -v_1+3v_2 \end{bmatrix} = \begin{bmatrix} 0\\ 0 \end{bmatrix} &\Rightarrow v_1=3v_2.
    \end{align*}
    Luego,  el autovalor $\lambda_2=1$ tiene autoespacio asociado $\{(3t,t): t\in\mathbb{R}\}$.
    

    \vskip .2cm
    \ref{autovalores-3}
    Denominamos $C$  a la matriz del enunciado. El polinomio característico de $C$ es
    \begin{align*}
        \chi_C(x) &= \left|\begin{matrix} x-2 & 0 & 0\\ 1 & x-1 & 1\\ 0 & 0 & x-2 \end{matrix} \right| \\ &= (x-2)^2(x-1).
    \end{align*}
    Luego $\lambda_1=2$ y $\lambda_2=1$ son los autovalores.

    Para calcular el autoespacio asociado a $\lambda_1=2$ resolvemos el sistema  $(2\Id-C)v=0$, es decir
    \begin{align*}
        \begin{bmatrix} 2-2 & 0 & 0\\ 1 & 2-1 & 1\\ 0 & 0 & 2-2 \end{bmatrix} &= \begin{bmatrix} 0 & 0 & 0\\ 1 & 1 & 1\\ 0 & 0 & 0 \end{bmatrix} \begin{bmatrix} v_1\\ v_2 \\ v_3\end{bmatrix} = \begin{bmatrix} 0\\ 0 \\ 0\end{bmatrix} \\
        \Rightarrow \begin{bmatrix} 0\\ v_1+v_2+v_3 \\ 0 \end{bmatrix} = \begin{bmatrix} 0\\ 0 \\ 0\end{bmatrix} &\Rightarrow v_1=-v_2-v_3.
    \end{align*}
    Luego,  el autovalor $\lambda_1=2$ tiene autoespacio asociado $\{(-t,t,s): t,s\in\mathbb{R}\}$.

    Para calcular el autoespacio asociado a $\lambda_2=1$ resolvemos el sistema  $(\Id-C)v=0$, es decir
    \begin{align*}
        \begin{bmatrix} 1-2 & 0 & 0\\ 1 & 1-1 & 1\\ 0 & 0 & 1-2 \end{bmatrix} &= \begin{bmatrix} -1 & 0 & 0\\ 1 & 0 & 1\\ 0 & 0 & -1 \end{bmatrix} \begin{bmatrix} v_1\\ v_2 \\ v_3\end{bmatrix} = \begin{bmatrix} 0\\ 0 \\ 0\end{bmatrix} \\
        \Rightarrow \begin{bmatrix} -v_1\\ v_1+v_3 \\ -v_3 \end{bmatrix} = \begin{bmatrix} 0\\ 0 \\ 0\end{bmatrix} &\Rightarrow v_1=v_3=0.
    \end{align*}
    Luego,  el autovalor $\lambda_2=1$ tiene autoespacio asociado $\{(0,t,0): t\in\mathbb{R}\}$.

    

    \vskip .2cm
    \ref{autovalores-4} Denominamos $D$  a la matriz del enunciado. El polinomio característico de $D$ es
    \begin{align*}
        \chi_D(x) &= \left|\begin{matrix}x+1&0&0\\0& x-3 & 5\\0& -1 & x+1 \end{matrix} \right| \\ 
        &= (x+1) \left|\begin{matrix} x-3 & 5\\ -1 & x+1 \end{matrix} \right|\\
        &= (x+1)[(x-3)(x+1)-(-1)(5)] \\&
        = (x+1)(x^2-2x+2). 
    \end{align*}
    El polinomio $ x^2-2x+2$ no tiene raíces reales, luego el  único autovalor real de $D$ es $-1$. Averigüemos el autoespacio correspondiente,    resolviendo el sistema  $(-\Id-D)v=0$, es decir el sistema homogéneo
    \begin{align*}
        &\begin{bmatrix} -1+1 & 0 & 0\\ 0 & -1-3 & -5\\ 0 & 1 & -1+1 \end{bmatrix} = \begin{bmatrix} 0 & 0 & 0\\ 0 & -4 & -5\\ 0 & 1 & 0 \end{bmatrix} \\
        &\stackrel{F_2-4F_3}{\longrightarrow} \begin{bmatrix} 0 & 0 & 0\\ 0 & 0& -5\\ 0 & 1 & 0 \end{bmatrix} 
    \end{align*}
    Luego $v_2=0$ y $v_3=0$, es decir $v_2=v_3=0$. Por lo tanto, el autoespacio asociado a $-1$ es $\{(v_1,0,0): v_1\in\mathbb{R}\}$.


    \vskip .2cm
    \ref{autovalores-5} Denominamos $E$  a la matriz del enunciado. En realidad la matriz $E$ depende de $\lambda$ y  debemos analizar la existencia de autovalores y autovectores para diferentes valores de  $\lambda$. El polinomio característico de $E$ es
    \begin{align*}
        \chi_E(x) &= \left|\begin{matrix} x-\lambda & 0 & 0\\ 1 & x-\lambda & 0\\ 0 & 1 & x-\lambda \end{matrix} \right| \\ &= (x-\lambda)^3.
    \end{align*}
    Luego $\lambda$ es el único autovalor. Para calcular el autoespacio asociado a $\lambda$ resolvemos el sistema  $(\lambda\Id-E)v=0$, es decir   
    \begin{align*}
        \begin{bmatrix} \lambda-\lambda & 0 & 0\\ 1 & \lambda-\lambda & 0\\ 0 & 1 & \lambda-\lambda \end{bmatrix} &= \begin{bmatrix} 0 & 0 & 0\\ 1 & 0 & 0\\ 0 & 1 & 0 \end{bmatrix} \begin{bmatrix} v_1\\ v_2 \\ v_3\end{bmatrix} = \begin{bmatrix} 0\\ 0 \\ 0\end{bmatrix} \\
        \Rightarrow \begin{bmatrix} 0\\ v_1 \\ v_2 \end{bmatrix} = \begin{bmatrix} 0\\ 0 \\ 0\end{bmatrix} &\Rightarrow v_1=v_2=0.
    \end{align*}
    Luego,  el único autovalor es $\lambda$ y el autoespacio asociado es $\{(0,0,t): t\in\mathbb{R}\}$.
    

    \vskip .2cm
    \ref{autovalores-6} Denominamos $F$  a la matriz del enunciado. En realidad la matriz $F$ depende de $\theta$ y  debemos analizar la existencia de autovalores y autovectores para diferentes valores de  $\theta$. El polinomio característico de $F$ es
    \begin{align*}
        \chi_F(x) &= \left|\begin{matrix} x-\cos\theta & -\sen\theta\\\sen\theta & x-\cos\theta \end{matrix} \right| \\ 
        &= (x-\cos\theta)^2+\sen(\theta)^2\\
        &= x^2-2x\cos\theta+\cos^2\theta+\sen(\theta)^2 \\
        &= x^2-2x\cos\theta+1.
    \end{align*}
    Las raíces del polinomio $x^2-2x\cos\theta+1$ son
    \begin{align*}
        \lambda_{1,2} &= \frac{2\cos\theta-2\sqrt{\cos^2\theta-1}}{2} = \cos\theta \pm \sqrt{\cos^2\theta-1}\\&= \cos\theta\pm\sqrt{-\sen^2\theta} \tag{*}
    \end{align*}

    Debemos dividir en 3 casos, según el valor de $\theta$.

    \vskip .2cm

    \textit{i}) Cuando $\theta=0$ tenemos que $\lambda_1 = \lambda_2 =1$. En  ese caso, $F$ es la matriz identidad y por lo tanto hay un único autovalor, 1, y  el autoespacio correspondiente es todo $\R^3$.
    
    \vskip .2cm
    \textit{ii}) Cuando $\theta= \pi$, tenemos que $\lambda_1 = \lambda_2 =-1$. En  este caso la matriz es
    $$
    F = \begin{bmatrix} -1 & 0\\  0 & -1 \end{bmatrix}
    $$
    Luego, $F$ es $-\Id$, hay un solo autovalor, $-1$, y  el autoespacio correspondiente es todo $\R^2$
    
    \vskip .2cm
    \textit{iii}) Cuando $\theta\ne 0, \pi$ y  en este caso tenemos que $-\sen^2\theta < 0$ y por lo tanto, por (*), no hay autovalores reales.\qed

    \vskip .2cm
    \item\label{autovalores-complejos} Calcular los autovalores complejos de las matrices \ref{autovalores-4} y \ref{autovalores-6} del ejercicio anterior, y para cada autovalor, dar una descripción paramétrica del autoespacio asociado sobre $\mathbb{C}$.
    
    \rta
    
    En el caso de la matriz \ref{autovalores-4} del ejercicio anterior, el polinomio característico es $x^2-2x+2$ como ya fue calculado. 
    Por lo tanto sus raíces son
    $$
    \lambda_{1,2} = \frac{2\pm\sqrt{4-8}}{2} = 1\pm i.
    $$
    Para calcular el autoespacio asociado a $\lambda_1 = 1+i$ resolvemos el sistema  $((1+i)\Id-D)v=0$, es decir el sistema homogéneo
    \begin{align*}
        &\begin{bmatrix} 1+i-3 & 5\\ -1 & 1+i+1 \end{bmatrix} = \begin{bmatrix} -2+i & 5\\ -1 & 2+i \end{bmatrix} \stackrel{F_2/(-1)}{\longrightarrow} \begin{bmatrix} -2+i & 5\\ 1 & -2-i \end{bmatrix} \\
        &\qquad\stackrel{F_1 + (2-i) F_2}{\longrightarrow} \begin{bmatrix} 0 & 5-(2+i)(2-i)\\ 1 & -2-i \end{bmatrix} =\begin{bmatrix} 0 & 0\\ 1 & -2-i \end{bmatrix}.
    \end{align*}
    Por lo tanto $v_1 +(-2-i)v_2=0$, es decir $v_1 = (2+i)v_2$. Luego, el autoespacio asociado a $\lambda_1 = 1+i$ es $\{((2+i)t,t): t \in \C\}$.

    Para calcular el autoespacio asociado a $\lambda_2 = 1-i$ resolvemos el sistema  $((1-i)\Id-D)v=0$, es decir el sistema homogéneo
    \begin{align*}
        &\begin{bmatrix} 1-i-3 & 5\\ -1 & 1-i+1 \end{bmatrix} = \begin{bmatrix} -2-i & 5\\ -1 & 2-i \end{bmatrix} \stackrel{F_2/(-1)}{\longrightarrow} \begin{bmatrix} -2-i & 5\\ 1 & -2+i \end{bmatrix} \\
        &\qquad\stackrel{F_1 + (2+i) F_2}{\longrightarrow} \begin{bmatrix} 0 & 5-(2-i)(2+i)\\ 1 & -2+i \end{bmatrix} =\begin{bmatrix} 0 & 0\\ 1 & -2+i \end{bmatrix}.
    \end{align*}
    Por lo tanto $v_1 +(-2+i)v_2=0$, es decir $v_1 = (2-i)v_2$. Luego, el autoespacio asociado a $\lambda_2 = 1-i$ es $\{((2-i)t,t): t \in \C\}$.

    \vskip .2cm

    En la resolución del punto \ref{autovalores-6} del ejercicio anterior, obtuvimos  $\lambda_{1,2} = \cos\theta\pm\sqrt{-\sen^2\theta}$. Luego, los autovalores son $\lambda_1 =\cos\theta + i\sen\theta$ y $\lambda_2 =\cos\theta - i\sen\theta$. Veamos ahora  los autoespacios asociados a estos autovalores complejos, es decir cuando $\sen\theta \ne 0$. En  el caso $\lambda_1 =\cos\theta + i\sen\theta$ resolvemos el sistema  $((\cos\theta + i\sen\theta)\Id-F)v=0$, es decir el sistema homogéneo
    \begin{align*}
        &\begin{bmatrix}\cos\theta + i\sen\theta-\cos\theta & -\sen\theta\\ \sen\theta & \cos\theta + i\sen\theta-\cos\theta \end{bmatrix} \\
        &= \begin{bmatrix}  i\sen\theta & -\sen\theta\\ \sen\theta & i\sen\theta \end{bmatrix} \stackrel{F_1\cdot i}{\longrightarrow} \begin{bmatrix} -\sen\theta & -i\sen\theta\\ \sen\theta & i\sen\theta \end{bmatrix} \\
        &\qquad\overset{F_1/\sen\theta}{\stackrel{F_2/\sen\theta}{\longrightarrow}} \begin{bmatrix}-1 & -i\\ 1 & i \end{bmatrix} \stackrel{F_2+F_1}{\longrightarrow}\begin{bmatrix}-1 & -i\\ 0 & 0 \end{bmatrix}
    \end{align*}
    Por lo tanto $-v_1-iv_2=0$, es decir $v_1 = -iv_2$. Luego, el autoespacio asociado a $\lambda_1 =\cos\theta + i\sen\theta$ es $\{t(-i,1): t \in \C\}$.

    \vskip .2cm 

    En el caso $\lambda_2 =\cos\theta - i\sen\theta$ resolvemos el sistema  $((\cos\theta - i\sen\theta)\Id-F)v=0$, es decir el sistema homogéneo
    \begin{align*}
        &\begin{bmatrix} \cos\theta - i\sen\theta-\cos\theta & -\sen\theta\\  \sen\theta & \cos\theta - i\sen\theta-\cos\theta \end{bmatrix} \\
        &= \begin{bmatrix} -i\sen\theta & -\sen\theta\\ \sen\theta & -i\sen\theta \end{bmatrix} \stackrel{F_1\cdot i}{\longrightarrow} \begin{bmatrix} \sen\theta & -i\sen\theta\\\sen\theta & -i\sen\theta \end{bmatrix} \\
        &\qquad\overset{F_1/\sen\theta}{\stackrel{F_2/\sen\theta}{\longrightarrow}} \begin{bmatrix}1 & -i\\ 1 & -i \end{bmatrix} \stackrel{F_2-F_1}{\longrightarrow}\begin{bmatrix} 1 & -i\\  0 & 0 \end{bmatrix} 
    \end{align*}
    Por lo tanto $v_1-iv_2=0$, es decir $v_1 = iv_2$. Luego, el autoespacio asociado a $\lambda_2 =\cos\theta - i\sen\theta$ es $\{t(i,1): t \in \C\}$.

    \vskip .2cm

    \noindent \textbf{Conclusión.} Haremos un resumen de  los resultados acerca de autovalores y autovectores de la matriz del ejercicio \ref{autovalores} \ref{autovalores-6}, es decir de la matriz
    \begin{equation*}
        \left[\begin{matrix} \cos\theta & \sen\theta\\ -\sen\theta & \cos\theta \end{matrix} \right].
    \end{equation*}

    \begin{itemize}
        \item Si $\theta=0$, entonces $F$ tiene un único autovalor, 1, y  el autoespacio correspondiente es todo $\R^2$.
        \item Si $\theta= \pi$, entonces $F$ tiene un único autovalor, -1, y  el autoespacio correspondiente es todo $\R^2$.
        \item Si $\theta \ne 0, \pi$, entonces $F$  tiene dos autovalores complejos, $\lambda_1 =\cos\theta + i\sen\theta$ y $\lambda_2 =\cos\theta - i\sen\theta$, cuyos autoespacios son $\{t(-i,1): t \in \C\}$ y $\{t(i,1): t \in \C\}$, respectivamente.
    \end{itemize}
    \qed

    \end{enumerate}
    

    
    
    \begin{enumerate}[resume,topsep=6pt,itemsep=.4cm]
    
    \item Probar que hay una única matriz $A\in\mathbb{R}^{2\times 2}$ tal que $(1,1)$ es autovector de autovalor $2$, y $(-2,1)$ es autovector de autovalor $1$.
    
    \rta Para encontrar la matriz debemos resolver los sistemas
    \begin{align*}
        \begin{bmatrix} x & y\\ z & w \end{bmatrix} \begin{bmatrix} 1\\ 1 \end{bmatrix} &= \begin{bmatrix} 2\\ 2 \end{bmatrix} \\
        \begin{bmatrix}x &y\\z &w \end{bmatrix} \begin{bmatrix} -2\\ 1 \end{bmatrix} &= \begin{bmatrix} -2\\ 1 \end{bmatrix}.
    \end{align*}
    O, equivalentemente, 
    \begin{align*}
        \begin{bmatrix}x+y\\z+w \end{bmatrix} &= \begin{bmatrix} 2 \\ 2\end{bmatrix} \\
        \begin{bmatrix} -2x +y\\ -2z +w \end{bmatrix} &= \begin{bmatrix} -2 \\ 1\end{bmatrix}.
    \end{align*}
    Son sistemas sencillos de resolver: 
    $$x+y=2,\quad z+w=2,\quad -2x+y=-2,\quad -2z+w=1.$$ 
    
    Luego, $y = 2-x$ y por lo tanto $-2x+2-x=-2$, es decir  $-3x=-4$ $\Rightarrow$ $x = 4/3$, y  en consecuencia $y =2-x = 2- 4/3=2/3$. 

    Por otro lado, $w = 2-z$ y por lo tanto $-2z+2-z=1$, es decir $-3z=-1$ $\Rightarrow$ $z = 1/3$, y  en consecuencia $w =2-z = 2- 1/3=5/3$.

    Luego, la matriz buscada es 
    $$
    A = \begin{bmatrix} 4/3 & 2/3\\1/3  & 5/3 \end{bmatrix}.
    $$

    \qed
    
    
    \item Sea $A\in\mathbb{K}^{n\times n}$, y sea $f(x) = ax^2+bx+c$ un polinomio, con $a,b,c\in\mathbb{K}$. Sea $f(A)$ la matriz $n \times n$ definida por
    $$f(A) = a A^2+bA+c\Id_n.$$
    Probar que todo autovector de $A$ con autovalor $\lambda$ es autovector de $f(A)$ con autovalor $f(\lambda)$.
    
    \rta Sea $v$ un autovector de $A$ con autovalor $\lambda$, es decir $Av=\lambda v$. Entonces 
    \begin{align*}
        f(A)v &= (a A^2+bA+c\Id_n)v \\
        &= a A^2v+bAv+c\Id_nv \\
        &= a A(\lambda v)+b(\lambda v)+c\Id_n v \\
        &=  a \lambda A(v)+\lambda bv+c v \\
        &=a \lambda^2 v+\lambda bv+c v \\
        &= (\lambda^2 a+\lambda b+c)v \\
        &= f(\lambda)v.
    \end{align*}
    \qed

        
    \item Sea $A\in\mathbb{K}^{2\times 2}$.
    
        \begin{enumerate}     
            \item\label{char2x2} Probar que el polinomio característico de $A$ es \ $\chi_A(x) = x^2-\operatorname{Tr}(A)x+\det(A)$.
            \item\label{no_inv2x2} Si $A$ no es invertible, probar que los autovalores de  $A$ son $0$ y $\operatorname{Tr}(A)$.
        \end{enumerate}
    
    \rta \ref{char2x2} Si $A= [a_{ij}]$, entonces
    \begin{align*}
        \chi_A(x) &= \left|\begin{matrix} x-a_{11} & -a_{12}\\ -a_{21} & x-a_{22} \end{matrix} \right| \\ &= (x-a_{11})(x-a_{22})-(-a_{12})(-a_{21})\\& = x^2-(a_{11}+a_{22})x+a_{11}a_{22}-a_{12}a_{21} \\
        &= x^2-\operatorname{Tr}(A)x+\det(A).
    \end{align*}

    \ref{no_inv2x2} Si $A$ no es invertible, entonces $\det(A)=0$ y por lo tanto $\chi_A(x) = x^2-\operatorname{Tr}(A)x$. Luego, los autovalores de $A$ son las raíces de $x^2-\operatorname{Tr}(A)x = x(x-\operatorname{Tr}(A))$, es decir $0$ y $\operatorname{Tr}(A)$. \qed


    
        
        
    \item Sea $A\in\mathbb{K}^{n\times n}$. Probar que el polinomio $\tilde\chi_A(x)=\det(A-x\Id_n)$ y el polinomio característico de $A$ tienen las mismas raíces.
    
    \rta Observar que $A-x\Id_n$ es igual a $-\Id_n(x\Id_n-A)$ y por lo tanto,
    \begin{align*}
        \tilde\chi_A(x) &= \det(A-x\Id_n) \\
        &= \det(-\Id_n(x\Id_n-A)) \\
        &= \det(-\Id_n)\det(x\Id_n-A) \\
        &= (-1)^n\det(x\Id_n-A) \\
        &= (-1)^n\chi_A(x).
    \end{align*}
    Luego, $\tilde\chi_A(x)$ y $\chi_A(x)$ son polinomios que difieren por un factor $(-1)^n$, es decir tienen las mismas raíces. \qed

    
    \end{enumerate}
    
    \begin{enumerate}[resume,topsep=6pt,itemsep=.4cm]
    
    \item Probar que si $A\in\mathbb{K}^{n\times n}$ es una matriz nilpotente entonces $0$ es el único autovalor de $A$. Usar esto para deducir que la matriz $\Id_n-A$ es invertible (esta es otra demostración del ejercicio \ref{nilpotene - id} del Práctico \ref{practico-3}).
    
    \rta Recordemos que $A$ es nilpotente si existe $k \in \mathbb N$ tal que $A^k = 0$. Si $A$ es nilpotente y $v$ autovector con autovalor $\lambda$, entonces $Av = \lambda v$ y por lo tanto $A^2v = A(Av) = A(\lambda v) = \lambda Av = \lambda^2 v$. Iterando este argumento, obtenemos que $A^kv = \lambda^k v$. Pero $A^k = 0$, luego $\lambda^k v = 0$ y por lo tanto $\lambda^k = 0$ y  en consecuencia $\lambda = 0$. Luego, $0$ es el único autovalor de $A$.

    Ahora bien, si $0$ es el único autovalor de $A$, entonces $\chi_A(x) = x^n$ y por lo tanto $1=\chi_A(1) = \det(\Id_n- A)$ como el determinate de $\Id_n-A$ es no nulo,  esa matriz es invertible. \qed    
    
    
    \item Decidir si las siguientes afirmaciones son verdaderas o falsas. Justificar.
    \begin{enumerate}
        \item Existe una matriz invertible $A$ tal que $0$ es autovalor de $A$.
        \item  Si $A$ es invertible, entonces todo autovector de $A$ es autovector de $A^{-1}$.
    \end{enumerate}
    
    \rta


    \vskip .2cm
    
    
    \end{enumerate}
    
    \begin{enumerate}[resume,topsep=6pt,itemsep=.4cm]
    
    \item\label{mas} Repetir los ejercicios \ref{autovalores} y \ref{autovalores-complejos} con las siguientes matrices.
    \begin{multicols}{2}
        \begin{enumerate}
            \item\label{mas-autovalores-1} $\begin{bmatrix} 2 & 3 \\ -1 & 1
            \end{bmatrix} $,\vskip .6cm 
            \item\label{mas-autovalores-2} $\begin{bmatrix} -9 & 4 & 4 \\ -8 & 3 & 4 \\ -16 & 8 & 7 \end{bmatrix} $, \vskip .5cm
            \item\label{mas-autovalores-3} $\left[\begin{matrix}4 & 4 & -12\\ 1 & -1 & 1\\ 5 & 3 & -11 \end{matrix} \right] $,\vskip .2cm
            \item\label{mas-autovalores-4} $\left[\begin{matrix}2 & 1 & 0 & 0\\ -1 & 4 & 0 & 0\\ 0 & 0 & 1 & 1 \\ 0 & 0 & 3 & -1\end{matrix} \right] $,\vskip .2cm
            \item\label{mas-autovalores-5} $\begin{bmatrix} \lambda & 0 & 0 & \dots & 0  \\ 1 & \lambda & 0 &\dots & 0  \\ 0 & 1 & \lambda&  \dots & 0  \\ \vdots & \vdots & \quad & \ddots & \vdots\\ 0 &  0&   \dots & 1  & \lambda \end{bmatrix}$, $\lambda\in \mathbb R$. \vskip .2cm
        \end{enumerate}
        \end{multicols}

    \rta


    \vskip .2cm
    
    
    

    \end{enumerate}