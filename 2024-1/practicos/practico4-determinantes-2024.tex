% PDFLaTeX
\documentclass[a4paper,12pt,twoside,spanish,reqno]{amsbook}
%%%---------------------------------------------------
\usepackage[math]{kurier}

\usepackage{etex}
\usepackage{t1enc}
\usepackage{latexsym}
\usepackage[utf8]{inputenc}
\usepackage{verbatim}
\usepackage{multicol}
\usepackage{amsgen,amsmath,amstext,amsbsy,amsopn,amsfonts,amssymb}
\usepackage{amsthm}
\usepackage{calc}         % From LaTeX distribution
\usepackage{graphicx}     % From LaTeX distribution
\usepackage{ifthen}
\input{random.tex}   
\usepackage{tikz}
\usetikzlibrary{arrows}
\usetikzlibrary{matrix}
\usepackage{mathtools}
\usepackage{stackrel}
\usepackage{enumitem}
\usepackage{tkz-graph}

\usepackage{enumitem} 
\usepackage[compatibility=false]{caption} % para usar subcaption
\usepackage{subcaption} % para poner varias imagenes juntas
\usetikzlibrary{arrows.meta}
\usepackage{hyperref}
\hypersetup{ 
    colorlinks=true,
    linkcolor=blue,
    filecolor=magenta,      
    urlcolor=cyan,
}
\usepackage{hypcap}
\numberwithin{equation}{section}
% http://www.texnia.com/archive/enumitem.pdf (para las labels de los enumerate)
\renewcommand\labelitemi{$\circ$}
\setlist[enumerate, 1]{label={(\arabic*)}}
\setlist[enumerate, 2]{label=\emph{\alph*)}}


\newcommand{\rta}{\noindent\textsc{Solución: }} 

\newcommand{\img}{\operatorname{Im}}
\newcommand{\nuc}{\operatorname{Nu}}
\newcommand\im{\operatorname{Im}}
\renewcommand\nu{\operatorname{Nu}}
\newcommand{\la}{\langle}
\newcommand{\ra}{\rangle}
\renewcommand{\t}{{\operatorname{t}}}
\renewcommand{\sin}{{\,\operatorname{sen}}}
\newcommand{\Q}{\mathbb Q}
\newcommand{\R}{\mathbb R}
\newcommand{\C}{\mathbb C}
\newcommand{\K}{\mathbb K}
\newcommand{\F}{\mathbb F}
\newcommand{\Z}{\mathbb Z}
\newenvironment{amatrix}[1]{%
  \left[\begin{array}{@{}*{#1}{c}|c@{}}
}{%
  \end{array}\right]
}

%%% FORMATOS %%%%%%%%%%%%%%%%%%%%%%%%%%%%%%%%%%%%%%%%%%%%%%%%%%%%%%%%%%%%%%%%%%%%%
\tolerance=10000
\renewcommand{\baselinestretch}{1.3}
\usepackage[a4paper, top=3cm, left=3cm, right=2cm, bottom=2.5cm]{geometry}
\usepackage{setspace}
%\setlength{\parindent}{0,7cm}% tamaño de sangria.
\setlength{\parskip}{0,4cm} % separación entre parrafos.
\renewcommand{\baselinestretch}{0.90}% separacion del interlineado
%%%%%%%%%%%%%%%%%%%%%%%%%%%%%%%%%%%%%%%%%%%%%%%%%%%%%%%%%%%%%%%%%%%%%%%%%%%%%%%%%%%
%\end{comment}
%%% FIN FORMATOS  %%%%%%%%%%%%%%%%%%%%%%%%%%%%%%%%%%%%%%%%%%%%%%%%%%%%%%%%%%%%%%%%%

\begin{document}
    \baselineskip=0.55truecm %original
    
    
    {\bf \begin{center} Práctico 4 \\ Álgebra  II -- Año 2024/1 \\ FAMAF \end{center}}



\centerline{\textsc{Determinantes}}


\noindent \textbf{Objetivos.}

\begin{itemize}
\item Aprender a calcular el determinante de una matriz.

\item Aprender a utilizar operaciones elementales por filas y/o columnas para calcular el determinante.

\item Aplicar las propiedades del determinante para calcular el determinante de un producto de matrices, y para decidir si una matriz cuadrada es o no invertible.
\end{itemize}
		
\

\noindent \textbf{Ejercicios.} Los ejercicios con el símbolo $\textcircled{a}$ tienen una ayuda al final del archivo para que recurran a ella después de pensar un poco.

\begin{enumerate}
\item Calcular el determinante de las siguientes matrices.
	\begin{align*}
	&A=\begin{bmatrix} 4&7\\ 5&3\end{bmatrix},
	&&B=\begin{bmatrix} -3&2&4\\ 1&-1&2\\ -1&4&0\end{bmatrix},
	&&
	C=\begin{bmatrix} 2&3&1&1\\ 0&2&-1&3 \\ 0&5&1&1 \\1&1&2&5\end{bmatrix}.
	\end{align*}

\

\item Sean
		$$A=
	\begin{bmatrix}
		1&3&2 \\
		3&0&2 \\
		1&1&1
	\end{bmatrix}, \qquad
	B =
	\begin{bmatrix}
		1&-1&2\\
		1&1&1 \\
		-1&-1&3
	\end{bmatrix}.
	$$
	Calcular:
	\begin{multicols}{3}
	\begin{enumerate}
		\item $\det(AB)$.
		\item $\det(BA)$.
		\item $\det(A^{-1})$.
		\item $\det(A^{4})$.
		\item $\det(A+B)$.
		\item $\det(A+tB)$, con $t \in \mathbb{R}$.
	\end{enumerate}
\end{multicols}


\

\item Reduciendo a matrices triangulares superiores, calcular el determinante de las siguientes matrices.

		$$A =
		\begin{bmatrix}
			a&1&1&1 \\
			1&a&1&1 \\
			1&1&a&1 \\
			1&1&1&a \\
		\end{bmatrix}, \qquad	
        B =
		\begin{bmatrix}
			1&1&1&1&1 \\
			1&3&3&3&3 \\
			1&3&5&5&5 \\
			1&3&5&7&7 \\
			1&3&5&7&9 \\
		\end{bmatrix}.
		$$

\

\item Sean $A$, $B$ y $C$ matrices $n\times n$, tales que $\det A=-1$, $\det B=2$ y $\det C=3$.
Calcular:

\

\begin{enumerate}
\item $\det(PQR)$, donde $P$, $Q$ y $R$ son las matrices que se obtienen a partir de $A$, $B$ y $C$ mediante operaciones elementales por filas de la siguiente manera
 \begin{align*}
 A\overset{F_1+2F_2}{\longrightarrow} P,\quad
 B\overset{3F_3}{\longrightarrow} Q
 \quad\mbox{y}\quad
 C\overset{F_1\leftrightarrow F_4}{\longrightarrow} R.
 \end{align*}
 Es decir,
 \begin{itemize}
  \item[$\circ$] $P$ se obtiene a partir de $A$ sumando a la fila $1$ la fila $2$ multiplicada por $2$.
  \item[$\circ$] $Q$ se obtiene a partir de $B$ multiplicando la fila $3$ por $3$.
  \item[$\circ$] $R$ se obtiene a partir de $C$ intercambiando las filas $1$ y $4$.
 \end{itemize}

\

\item $\det(A^2BC^tB^{-1})$ \ y \ $\det(B^2C^{-1}AB^{-1}C^{t})$.

\end{enumerate}


\

\item  Sea
$$A=
\begin{bmatrix}
	x&y&z \\
	3&0&2\\
	1&1&1
\end{bmatrix}.$$
Sabiendo que $\det(A) = 5$, calcular el determinante de las siguientes matrices.
$$
B = \begin{bmatrix}
2x&2y&2z \\
3/2&0&1\\
1&1&1
\end{bmatrix}, \qquad
C=
\begin{bmatrix}
	x&y&z \\
	3x+3&3y&3z+2\\
	x+1&y+1&z+1
\end{bmatrix}.
$$

\

\item Determinar todos los valores de $c\in\mathbb{R}$ tales que las siguientes matrices sean invertibles.
\begin{align*}
A=\begin{bmatrix}4& c&3\\c&2&c\\ 5&c&4 \end{bmatrix},\qquad
B=\begin{bmatrix} 1&c&-1\\ c&1&1\\0&1&c\end{bmatrix}.
\end{align*}


\

\item Calcular el determinante de las siguientes matrices, usando operaciones elementales por fila y/o columnas u otras propiedades del determinante. Determinar cuáles de ellas son invertibles.

\begin{align*}
&A=
\begin{bmatrix}
-2&3&2&-6\\ 0&4&4&-5\\ 5&-6&-3&2\\ -3&7&0&0 \end{bmatrix},\quad
&&B=\begin{bmatrix} 2&0&0&0\\ 0&0&3&0\\ 0&-1&0&0\\ 0&0&0&4\end{bmatrix},\quad
&&
C=\begin{bmatrix}
  -2&3&2&-6&0\\
0&4&4&-5&0\\
5&-6&-3&2&0\\
-3&7&0&0&0\\
1&1&1&1&1
  \end{bmatrix},
\end{align*}
\begin{align*}
D=\begin{bmatrix}
1&2&3&0&0\\
-1&2&-13&6&\frac{1}{3}\\
2&0&0&0&0\\
11&1&0&0&0\\
\sqrt{2}&2&1&\pi&0\\
\end{bmatrix},&&
E=\begin{bmatrix}
1&-1&2&0&0\\ 3&1&4&0&0\\ 2&-1&5&0&0 \\0&0&0&2&1\\ 0&0&0&-1&4
\end{bmatrix}.
\end{align*}


\

\item Sean $A$ y  $B$ matrices $n \times n$. Probar que:

\

\begin{enumerate}
	\item $\det(AB) = \det (BA)$.
	
	\
	
	\item Si $B$ es invertible, entonces $\det(B A B^{-1}) = \det (A)$.
	
	\
	
		\item\label{-A} $\textcircled{a}$ $\det(-A) = (-1)^n\det (A)$.
\end{enumerate}

\

\item\label{vandermonde} Sean $\lambda_1, \lambda_2, \dots, \lambda_n$ escalares, la matriz de \emph{Vandermonde} asociada es
\begin{align*}
\mathtt V = \begin{bmatrix}
1 & \lambda_1 & \lambda_1^2 & \cdots & \lambda_1^{n-1}\\
1 & \lambda_2 & \lambda_2^2 & \cdots & \lambda_2^{n-1}\\
\vdots &\vdots &\vdots & &\vdots\\
1 & \lambda_n & \lambda_n^2 & \cdots & \lambda_n^{n-1}\\
\end{bmatrix}.
\end{align*}
Esta es la matriz del sistema de ecuaciones del Ejercicios 12 (c) del Práctico 2.


\begin{enumerate}
\item Si $n=2$, probar que $\det(\mathtt V) = \lambda_2-\lambda_1$.


\item Si $n=3$, probar que $\det(\mathtt V) = (\lambda_3-\lambda_2) (\lambda_3-\lambda_1) (\lambda_2-\lambda_1)$.


\item\label{vandermonde gral} $\textcircled{a}$ Probar que $\det(\mathtt V) = \prod_{1\leq i< j \leq n}(\lambda_j-\lambda_i)$ para todo $n\in\mathbb{N}$.


\item Dar una condición necesaria y suficiente para que la matriz de Vandermonde sea invertible.


\item Usar lo anterior para responder a la pregunta del Ejercicio 12 (b) del Práctico 2.
\end{enumerate}


\item Decidir si las siguientes afirmaciones son verdaderas o falsas. Justificar con una demostración o con un contraejemplo, según corresponda.


\begin{enumerate}
\item Sean $A$ y $B$ matrices $n \times n$. Entonces $\det(A + B) = \det (A) + \det(B)$.


\item Existen una matriz $3\times 2$, $A$, y una matriz $2\times 3$, $B$, tales que $\det(AB) \neq 0$.


\item Sea $A$ una matriz $n\times n$. Si $A^n$ es no invertible, entonces $A$ es no invertible.
\end{enumerate}




\end{enumerate}



\subsection*{Ejercicios de repaso}
Si ya hizo los ejercicios anteriores continue con la siguiente guía. Los ejercicios que siguen son similares y le pueden servir para practicar antes de los exámenes.

\

\begin{enumerate}[resume]
\item Determinar todos los valores de $c\in\mathbb{K}$ tales que la siguiente matriz sea invertible.
$$A=\begin{bmatrix} 0&c&-c\\ -1&2&-1\\c&-c&c\end{bmatrix}.$$

%\
%		
%\item
%En cada caso decidir si la matriz es invertible y si lo es, calcular la inversa usando la matriz de cofactores.
%\[
%A=\begin{bmatrix} -2&3&2\\ 6&0&3\\4&1&-1\end{bmatrix},\qquad
%B=\begin{bmatrix}\cos(t)& 0& \sin(t)\\ 0&1&0 \\ \sin(t)&0&\cos(t) \end{bmatrix}.
%\]

\

\item Sabiendo que \
$
\det \begin{bmatrix} a&b&c\\ p&q&r\\
x&y&z\end{bmatrix}=-1
$,
calcular \
$
\det \begin{bmatrix} -2a&-2b&-2c\\ 2p+x&2q+y&2r+z\\
3x&3y&3z\end{bmatrix}.
$

\

\item Probar que
$$
\det\begin{bmatrix}
1+x_1 & x_2 & x_3 & \cdots & x_n \\
x_1 & 1+x_2 & x_3 & \cdots & x_n \\
x_1 & x_2 & 1+x_3 & \cdots & x_n \\
\vdots & \vdots & \vdots &\ddots& \vdots \\
x_1 & x_2 & x_3 & \cdots & 1+x_n
\end{bmatrix}
= 1+x_1+x_2 + \cdots + x_n.
$$

\

\item
Una matriz $A$ $n \times n$ se dice {\it antisimétrica}
si $A^t=-A$.

\

\begin{enumerate}
\item\label{anti a} $\textcircled{a}$ Probar que si $n$ es impar y $A$ es antisimétrica, entonces
$\det(A)=0$.

\

\item\label{anti b} $\textcircled{a}$ Para cada $n$ par, encontrar una matriz $A$ antisimétrica
$n \times n$ tal que $\det(A)\not=0$.
\end{enumerate}
	
	
\end{enumerate}


\subsection*{Ayudas}

\

\ref{-A} Analizar primero los casos $n=2,3$.


\ref{vandermonde}\,\ref{vandermonde gral} En Wikipedia hay una posible demostración.


\ref{anti a} Usar el Ejercicio \ref{-A}.


\ref{anti b} Encontrar primero una matriz $A_0$ para el caso $2\times 2$. Para $n = 2m$ considerar la matriz $2m \times 2m$ formada por $m$ bloques diagonales iguales a $A_0$.


\end{document}
