\chapter{Soluciones\\Álgebra  II -- Año 2024/1 -- FAMAF}\label{practico-8}

\begin{enumerate}[topsep=6pt, itemsep=.4cm]

\item \label{lineales1-bases} Escribir las matrices de las siguientes transformaciones lineales respecto de las bases canónicas de los espacios involucrados.
\begin{enumerate}
    \item\label{lineales1-bases-a} $T:\R^2 \longrightarrow \R^3$, $T(x,y)=(x-y,x+y,2x+3y)$.
    \item\label{lineales1-bases-b} $S:\R^3 \longrightarrow \R^2$, $S(x,y,z)=(x-y+z,2x-y+2z)$.
    \item\label{lineales1-base-c} $D:P_4  \longrightarrow P_4$, $D(p(x))=p'(x)$.
    \item $T:M_{2\times 2}(\mathbb{K}) \longrightarrow \mathbb{K}$, $T(A)=\operatorname{tr}(A)$.
    \item\label{lineales1-base-d} $L:P_3 \longrightarrow M_{2\times 2}(\R)$, $L(ax^2+bx+c)=\begin{bmatrix} a & b+c \\ b+c & a \end{bmatrix}$.
    \item\label{lineales1-base-e} $Q:P_3 \longrightarrow P_4$, $Q(p(x))=(x+1)p(x)$.
\end{enumerate}

\rta denotemos $\mathcal{C}_n$  a la base canónica de $\R^n$, Denotemos $\mathcal{B}_n$ a la base $\{1,x,x^2,\ldots,x^{n-1}\}$ de $P_n$ y denotemos $\mathcal{M}_{2\times 2}$ a la base$\{E_{11},E_{12},E_{21},E_{22}\}$ de  $M_{2\times 2}(\mathbb{K})$.

\vskip .3cm

\ref{lineales1-bases-a} 
\begin{align*}
    T(e_1) &= T(1,0) = (1,1,2) = 1(1,0,0) + 1(0,1,0) +2(0,0,1)\\
    T(e_2) &= T(0,1) = (-1,1,3) = -1(1,0,0) +1(0,1,0) +3(0,0,1).
\end{align*}
Por lo tanto, la matriz de $T$ respecto de las bases canónicas es
\begin{align*}
    [T]_{\,\mathcal{C}_2\,\mathcal{C}_3} = \begin{bmatrix} 1 & -1 \\ 1 & 1 \\ 2 & 3 \end{bmatrix}.
\end{align*}

\vskip .3cm

\ref{lineales1-bases-b} 
\begin{align*}
    S(e_1) &= S(1,0,0) = (1,2) = 1(1,0) + 2(0,1)\\
    S(e_2) &= S(0,1,0) = (-1,-1) = -1(1,0) -1(0,1)\\
    S(e_3) &= S(0,0,1) = (1,2) = 1(1,0) + 2(0,1).
\end{align*}
Por lo tanto, la matriz de $S$ respecto de las bases canónicas es
\begin{align*}
    [S]_{\,\mathcal{C}_3\,\mathcal{C}_2} = \begin{bmatrix} 1 & -1 & 1 \\ 2 & -1 & 2 \end{bmatrix}.
\end{align*}

\vskip .3cm

\ref{lineales1-base-c} 
\begin{align*}
    D(1) &= 0 = 0 \cdot 1 + 0\cdot x + 0\cdot x^2 + 0\cdot x^3\\
    D(x) &= 1 =1 \cdot 1 + 0\cdot x + 0\cdot x^2 + 0\cdot x^3\\
    D(x^2) &= 2x =0 \cdot 1 + 2\cdot x + 0\cdot x^2 + 0\cdot x^3\\
    D(x^3) &= 3x^2 =0 \cdot 1 + 0\cdot x + 3\cdot x^2 + 0\cdot x^3.
\end{align*}
Por lo tanto, la matriz de $D$ respecto de las bases canónicas es
\begin{align*}
    [D]_{\,\mathcal{B}_4\,\mathcal{B}_4} = \begin{bmatrix} 0 & 1 & 0 & 0 \\ 0 & 0 & 2 & 0 \\ 0 & 0 & 0 & 3 \\ 0 & 0 & 0 & 0 \end{bmatrix}.
\end{align*}


\vskip .3cm

\ref{lineales1-base-d} 
\begin{align*}
    L(1) &= \begin{bmatrix} 1 & 0 \\ 0 & 1 \end{bmatrix} = 1\cdot E_{11} + 0\cdot E_{12} + 0\cdot E_{21} + 1\cdot E_{22}\\
    L(x) &= \begin{bmatrix} 0 & 1 \\ 1 & 0 \end{bmatrix} = 0\cdot E_{11} + 1\cdot E_{12} + 1\cdot E_{21} + 0\cdot E_{22}\\
    L(x^2) &= \begin{bmatrix} 0 & 1 \\ 1 & 0 \end{bmatrix}= 0\cdot E_{11} + 1\cdot E_{12} + 1\cdot E_{21} + 0\cdot E_{22}\\
\end{align*}
Por lo tanto, la matriz de $L$ respecto de las bases canónicas es
\begin{align*}
    [L]_{\,\mathcal{B}_3\,\mathcal{M}_{2\times 2}} = \begin{bmatrix} 1 & 0 & 0 \\ 0 & 1 & 1 \\ 0 & 1 & 1 \\ 1 & 0 & 0 \end{bmatrix}.
\end{align*}

\vskip .3cm

\ref{lineales1-base-e}
\begin{align*}
    Q(1) &= x+1 = 1\cdot 1 + 1\cdot x + 0\cdot x^2 + 0\cdot x^3\\
    Q(x) &= (x+1)x = 0\cdot 1 + 1\cdot x + 1\cdot x^2 + 0\cdot x^3\\
    Q(x^2) &= (x+1)x^2 = 0\cdot 1 + 0\cdot x + 1\cdot x^2 + 1\cdot x^3.
\end{align*}
Por lo tanto, la matriz de $Q$ respecto de las bases canónicas es
\begin{align*}
    [Q]_{\,\mathcal{B}_3\,\mathcal{B}_4} = \begin{bmatrix} 1 & 0 & 0 \\ 1 & 1 & 0 \\ 0 & 1 & 1 \\ 0 & 0 & 1 \end{bmatrix}.
\end{align*}
\qed

\item Dar las coordenadas del polinomio $2x^2+10x-1\in\mathbb{K}_3[x]$ en la base ordenada $$\mathcal{B}=\{1,x+1,x^2+x+1\}.$$

\rta 
\begin{align*}
    [2x^2+10x-1]_{\mathcal{B}}&= (a,b,c) \\
    &\Updownarrow \\
    2x^2+10x-1&=a\cdot 1+b\cdot (x+1)+c\cdot (x^2+x+1) \\
    &\Updownarrow \\
    2x^2+10x-1&=cx^2+(b+c)x+(a+b+c) \\ 
    &\Updownarrow \\
    c=2,\; b+c &= 10, \;a+b+c=-1.
\end{align*}
Este último renglón es un sistema que se resuelve fácilmente por sustitución: $c=2$, $b=8$ y $a=-11$. Por lo tanto,
\begin{align*}
    [2x^2+10x-1]_{\mathcal{B}}&= (-11,8,2).
\end{align*}

\qed


\item Dar las coordenadas de la matriz 
$A=\begin{bmatrix}
    1&2\\3&4 
    \end{bmatrix}
$ en la base ordenada 
$$
\mathcal{B}=\left\{
\begin{bmatrix}
    0&1\\0&0 
    \end{bmatrix},
\begin{bmatrix}
    0&0\\0&1 
    \end{bmatrix},
    \begin{bmatrix}
    1&0\\0&0 
    \end{bmatrix},
    \begin{bmatrix}
    0&0\\1&0 
    \end{bmatrix}
\right\}.
$$
Más generalmente, dar las coordenadas de cualquier matriz $\begin{bmatrix}
    a&b\\c&d 
    \end{bmatrix}$ en la base $\mathcal{B}$.

    \rta
    \begin{align*}
        \begin{bmatrix}
            a&b\\c&d 
            \end{bmatrix} &= b\begin{bmatrix}
            0&1\\0&0 
            \end{bmatrix}+d\begin{bmatrix}
            0&0\\0&1 
            \end{bmatrix}+a\begin{bmatrix}
            1&0\\0&0 
            \end{bmatrix}+c\begin{bmatrix}
            0&0\\1&0 
            \end{bmatrix} \\
            &\Updownarrow \\
            [A]_{\mathcal{B}} &= (b,d,a,c). 
    \end{align*}
    En particular,
    \begin{align*}
        \begin{bmatrix}
            1&2\\3&4 
            \end{bmatrix}_{\mathcal{B}} &= (2,4,1,3).
    \end{align*}


    \qed
    
    

\item 
\begin{enumerate}
    \item\label{subespacio W a} Dar una base del subespacio $W=\{(x,y,z)\in\mathbb{K}^3\mid x-y+2z=0\}$. 
    \item\label{subespacio W b} Dar las coordenadas de $w=(1,-1,-1)$ en la base que haya dado en el item anterior.
    \item\label{subespacio W c} Dado $(x,y,z)\in W$, dar las coordenadas de $(x,y,z)$ en la base que haya calculado en el item anterior. 
\end{enumerate}

\rta

\ref{subespacio W a} 
\begin{align*}
    W &= \{(x,y,z)\in\mathbb{K}^3\mid x-y+2z=0\} \\
    &= \{(x,y,z)\in\mathbb{K}^3\mid x=y-2z\} \\
    &= \{(y-2z,y,z)\mid y,z\in\mathbb{K}\} \\
    &= \{y(1,1,0)+z(-2,0,1)\mid y,z\in\mathbb{K}\}.
\end{align*}
Por lo tanto, $\{(1,1,0),(-2,0,1)\}$ generan $W$. Además, como $(1,1,0)$ y $(-2,0,1)$ son LI, entonces $\mathcal{B}=\{(1,1,0),(-2,0,1)\}$ es una base ordenada de $W$. 

\vskip .3cm

\ref{subespacio W b} Primero,  es claro que $w=(1,-1,-1)\in W$, pues cumple con la ecuación implícita que define $W$. Entonces
\begin{align*}
    [w]_{\mathcal{B}} = (a,b) &\Leftrightarrow (1,-1,-1) = a(1,1,0)+b(-2,0,1) \\
    &\Leftrightarrow 1=a-2b, \;-1=a, \;-1=b.
\end{align*}
Por lo tanto
\begin{align*}
    [w]_{\mathcal{B}} &= (-1,-1).
\end{align*}

\vskip .3cm

\ref{subespacio W c} Sea $(x,y,z)\in W$. Entonces
\begin{align*}
    [(x,y,z)]_{\mathcal{B}} = (a,b)&\Leftrightarrow (x,y,z) = a(1,1,0)+b(-2,0,1) \\
    &\Leftrightarrow x=a-2b, \;y=a, \;z=b.
\end{align*}
Por lo tanto
\begin{align*}
    [(x,y,z)]_{\mathcal{B}} &= (y,z).
\end{align*}


\qed



\item\label{otras bases} Sea $\mathcal{C}$ la base canónica de $\mathbb{K}^2$ y 
    $\mathcal{B}=\{(1,0),(1,1)\}$ otra base de $\mathbb{R}^2$.
    \begin{enumerate}
        \item\label{otras bases-a} Encontrar la matriz de cambio de base $P_{\mathcal{C},\mathcal{B}}$ de $\mathcal{C}$ a $\mathcal{B}$.
        \item\label{otras bases-b} Encontrar la matriz de cambio de base $P_{\mathcal{B},\mathcal{C}}$ de $\mathcal{B}$ a $\mathcal{C}$.
        \item\label{otras bases-c} ¿Qué relación hay entre $P_{\mathcal{C},\mathcal{B}}$ y $P_{\mathcal{B},\mathcal{C}}$?
        \item\label{otras bases-d} Encontrar $(x,y),(z,w)\in\mathbb{K}^2$ tal que $[(x,y)]_{\mathcal{B}}=(1,4)$ y $[(z,w)]_{\mathcal{B}}=(1,-1)$.
        \item\label{otras bases-e} Determinar las coordenadas de $(2,3)$ y $(0,1)$ en las bases $\mathcal{B}_2$.
    \end{enumerate}

    \rta recordar que si $V$ es un espacio vectorial de dimensión finita sobre el cuerpo $\K$ y  $\mathcal{B}$ y $\mathcal{B'}$ son bases ordenadas de $V$, la matriz $[\Id]_{\mathcal{B}\mathcal{B'}}$  es llamada la \textit{matriz de cambio de base} de la base $\mathcal{B}'$  a la base $\mathcal{B}$. 

    \ref{otras bases-a} 
    \begin{align*}
        \Id(1,0) &= 1 \cdot (1,0) + 0 \cdot (1,1) \\
        \Id(0,1) &= (-1) \cdot (1,0) + 1 \cdot (1,1).
    \end{align*}
    Por lo tanto, 
    $$
    P_{\mathcal{C},\mathcal{B}} = \begin{bmatrix} 1 & -1 \\ 0 & 1 \end{bmatrix}.
    $$

    \vskip .3cm

    \ref{otras bases-b} 
    \begin{align*}
        \Id(1,0) &= 1 \cdot (1,0) + 0 \cdot (0,1) \\
        \Id(1,1) &= 1 \cdot (1,0) + 1 \cdot (0,1).
    \end{align*}
    Por lo tanto,
    $$
    P_{\mathcal{B},\mathcal{C}} = \begin{bmatrix} 1 & 1 \\ 0 & 1 \end{bmatrix}.
    $$

    \vskip .3cm

    \ref{otras bases-c}
    $$
    P_{\mathcal{C},\mathcal{B}}  P_{\mathcal{B},\mathcal{C}} = \Id.    
    $$
    Por la teórica sabemos que esta relación vale en general.

    \vskip .3cm

    \ref{otras bases-d} 
    \begin{align*}
        [(x,y)]_{\mathcal{B}} = (1,4)  &\Leftrightarrow (x,y) = 1 \cdot (1,0) + 4 \cdot (1,1) \\
        &\Leftrightarrow x=5, \;y=4.
    \end{align*}
    También es posible hacerlo por la matriz de cambio de base:
    \begin{align*}
        \begin{bmatrix}
            x \\ y
        \end{bmatrix} = P_{\mathcal{B}, \mathcal{C}} [(x,y)]_\mathcal{B} = P_{\mathcal{B}, \mathcal{C}} \begin{bmatrix}
            1 \\ 4
        \end{bmatrix} = \begin{bmatrix}
            1 & 1 \\ 0 & 1
        \end{bmatrix} \begin{bmatrix}
            1 \\ 4
        \end{bmatrix} = \begin{bmatrix}
            5 \\ 4
        \end{bmatrix}.
    \end{align*}

    De forma análoga, 
    \begin{align*}
        [(z,w)]_{\mathcal{B}} = (1,-1)  &\Leftrightarrow (z,w) = 1 \cdot (1,0) + (-1) \cdot (1,1) \\
        &\Leftrightarrow z=0, \;w=-1.
    \end{align*}
    Obviamente, también es posible hacerlo por la matriz de cambio de base:
    \begin{align*}
        \begin{bmatrix}
            z \\ w
        \end{bmatrix} = P_{\mathcal{B}, \mathcal{C}} [(z,w)]_\mathcal{B} = P_{\mathcal{B}, \mathcal{C}} \begin{bmatrix}
            1 \\ -1
        \end{bmatrix} = \begin{bmatrix}
            1 & 1 \\ 0 & 1
        \end{bmatrix} \begin{bmatrix}
            1 \\ -1
        \end{bmatrix} = \begin{bmatrix}
            0 \\ -1
        \end{bmatrix}.
    \end{align*}

    \vskip .3cm

    \ref{otras bases-e}
    \begin{align*}
        [(2,3)]_{\mathcal{B}} = P_{\mathcal{C},\mathcal{B}} [(2,3)]_\mathcal{C} = P_{\mathcal{C},\mathcal{B}} \begin{bmatrix}
            2 \\ 3
        \end{bmatrix} = \begin{bmatrix}
            1 & -1 \\ 0 & 1
        \end{bmatrix} \begin{bmatrix}
            2 \\ 3
        \end{bmatrix} = \begin{bmatrix}
            -1 \\ 3
        \end{bmatrix}.
    \end{align*}
    El caso general es análogo:
    \begin{align*}
        [(a,b)]_{\mathcal{B}} = P_{\mathcal{C},\mathcal{B}} [(a,b)]_\mathcal{C} = P_{\mathcal{C},\mathcal{B}} \begin{bmatrix}
            a \\ b
        \end{bmatrix} = \begin{bmatrix}
            1 & -1 \\ 0 & 1
        \end{bmatrix} \begin{bmatrix}
            a \\ b
        \end{bmatrix} = \begin{bmatrix}
            a-b \\ b
        \end{bmatrix}.
    \end{align*}


    \qed
    
    

    \item\label{matriz P} Sea $P=\begin{bmatrix}
        1&1&0\\2&1&1\\3&1&0
        \end{bmatrix}
        \in\mathbb{K}^{3\times 3}$.
        \vskip .2cm
\begin{enumerate}
\item\label{inversa de P} Calcular la inversa de $P$.
\item\label{base de P} Dar una base ordenada $\mathcal{B}$ de $\mathbb{K}^3$  
tal que $P$ es la matriz de cambio de coordenadas de la base canónica de $\mathbb{K}^3$ a la
base $\mathcal{B}$.
\item\label{encontrar vector} Encontrar $(x,y,z)\in\mathbb{K}^3$ tal que su vector de coordenadas con respecto a $\mathcal{B}$ es 
$$[(x,y,z)]_{\mathcal{B}}=(2,-1,-1).$$
\end{enumerate}

\rta 

\ref{inversa de P} Utilizamos el método de Gauss-Jordan:
\begin{align*}
    &\left[\begin{array}{ccc|ccc}
    1&1&0&1&0&0\\2&1&1&0&1&0\\3&1&0&0&0&1
    \end{array}\right] \underset{F_3-3F_1}{\stackrel{F_2-2F_1}{\longrightarrow}} \left[\begin{array}{ccc|ccc}
    1&1&0&1&0&0\\0&-1&1&-2&1&0\\0&-2&0&-3&0&1
    \end{array}\right] \\
    &\stackrel{-F_2}{\longrightarrow} \left[\begin{array}{ccc|ccc}
    1&1&0&1&0&0\\0&1&-1&2&-1&0\\0&-2&0&-3&0&1
    \end{array}\right] \underset{F_3+2F_2}{\stackrel{F_1-F_2}{\longrightarrow}} 
    \left[\begin{array}{ccc|ccc}
    1&0&1&-1&1&0\\0&1&-1&2&-1&0\\0&0&-2&1&-2&1
    \end{array}\right] \\
    &\underset{-\frac{1}{2}F_3}{\longrightarrow} \left[\begin{array}{ccc|ccc}
    1&0&1&-1&1&0\\0&1&-1&2&-1&0\\0&0&1&-\frac{1}{2}&1&-\frac{1}{2}
    \end{array}\right] \underset{F_2+F_3}{\stackrel{F_1-F_3}{\longrightarrow}}
    \left[\begin{array}{ccc|ccc}
    1&0&0&-\frac{1}{2}&0&\frac{1}{2}\\0&1&0&\frac{3}{2}&0&-\frac{1}{2}\\0&0&1&-\frac{1}{2}&1&-\frac{1}{2}
    \end{array}\right].
\end{align*}
Concluyendo, 
\begin{align*}
    P^{-1} = \begin{bmatrix}
        -\frac{1}{2}&0&\frac{1}{2}\\
        \frac{3}{2}&0&-\frac{1}{2}\\
        -\frac{1}{2}&1&-\frac{1}{2}
    \end{bmatrix} = \frac12 \begin{bmatrix}
        -1&0&1\\
        3&0&-1\\
        -1&2&-1
    \end{bmatrix}.
\end{align*} 

\vskip .3cm

\ref{base de P} Si $\mathcal C$ es la base canónica, queremos encontrar  $\mathcal{B}=\{v_1,v_2,v_3\}$, tal que $[\Id]_{\mathcal{C}\mathcal{B}}=P$. Es decir, queremos encontrar $v_1,v_2,v_3$ tal que
\begin{align*}
    P(e_1) &=  (1,2,3) \\
    v_2 &= \Id(v_2) = P(0,1,0) = (1,1,1) \\
    v_3 &= \Id(v_3) = P(0,0,1) = (0,1,0).
\end{align*}
Por lo tanto, $\mathcal{B}=\{(1,2,3),(1,1,1),(0,1,0)\}$ es una base de $\mathbb{K}^3$ tal que $[\Id]_{\mathcal{C}\mathcal{B}}=P$.



\qed



\item\label{matriz transformaciones ejemplo} Sea 
$T:\mathbb{R}^3\longrightarrow\mathbb{R}^2$ la transformación lineal definida por $$T(x,y,z)=(x-y,x-z).$$ Sean $\mathcal{C}$ la base canónica de $\mathbb{R}^3$ y $\mathcal{B}'=\{(1,1),(1,-1)\}$ base de $\mathbb{R}^2$.
\begin{enumerate}
    \item\label{matriz transformaciones ejemplo-a} Calcular la matriz $[T]_{\mathcal{C}\mathcal{B}'}$, es decir la matriz de $T$ respecto de las bases $\mathcal{C}$ y $\mathcal{B}'$.
    \item\label{matriz transformaciones ejemplo-b} Sea $(x,y,z)\in\mathbb{R}^3$. Dar las coordenadas de $T(x,y,z)$ respecto de la base $\mathcal{B}'$.
    \item\label{matriz transformaciones ejemplo-c} Sea $S:\mathbb{R}^2\longrightarrow\mathbb{R}^3$ una transformación lineal tal que su matriz respecto a las bases $\mathcal{B}'$ y $\mathcal{C}$ es
    \begin{align*}
    [S]_{\mathcal{B}'\mathcal{C}}=\begin{bmatrix}
    1&2\\1&-1\\1&0
    \end{bmatrix}. 
    \end{align*}
    Calcular la matriz de la composición $T\circ S:\mathbb{R}^2\longrightarrow\mathbb{R}^2$ con respecto a la base $\mathcal{B}'$. 
    \item\label{matriz transformaciones ejemplo-d} Calcular la matriz de $T\circ S$ respecto a la base $\mathcal{B}$ del ejercicio \ref{otras bases} usando las matrices de cambio de base calculadas en ese ejercicio.
\end{enumerate}

\rta


\qed



\item Sea $A$ la matriz  del ejercicio \ref{autovalores}\ref{autovalores-1} del práctico \ref{practico-5} y $T_A:\mathbb{R}^2\longrightarrow\mathbb{R}^2$ la transformación lineal dada por $T_A(v)=Av$. Hallar los autovalores de $T_A$, y para cada uno de ellos, dar una base de autovectores del correspondiente autoespacio. Decidir si $T_A$ es o no diagonalizable. En caso de serlo dar una matriz invertible $P$ tal que $P^{-1}AP$ es diagonal. 

Repetir esto para cada una de las matrices de dicho ejercicio.

\rta


\qed



\item Repetir el ejercicio anterior para cada matriz del ejercicio \ref{autovalores} del práctico \ref{practico-5} pero ahora consideradando a la transformación como una transformación lineal entre los $\mathbb{C}$-espacios vectoriales $\mathbb{C}^n$.

\rta


\qed



\item Sea $T:V\longrightarrow V$ una transformación lineal y $v\in V$ un autovector de autovalor $\lambda$. Probar las siguientes afirmaciones.
\begin{enumerate}
    \item\label{autovalor-autovector-a} Si $\lambda=0$, entonces $v\in\operatorname{Nu}(T)$.
    \item\label{autovalor-autovector-b} Si $\lambda\neq0$, entonces $v\in\operatorname{Im}(T)$.
    \item\label{autovalor-autovector-c} Si $T^2=0$, entonces $T-\operatorname{Id}$ es un isomorfismo.
\end{enumerate}

\rta


\qed



\item\label{base nilp}  Sea $V$ un espacio vectorial de dimensión $3$ y $T:V\longrightarrow V$ una transformación lineal. Supongamos que existe $v\in V$ tal que $T^3(v)=0$ pero $T^2(v)\neq0$.
\begin{enumerate}
    \item\label{base nilp a}  Probar que $\mathcal{B}=\{v,T(v),T^2(v)\}$ es una base de $V$.
    \item\label{base nilp b} Calcular la matriz de $T$ respecto de la base $\mathcal{B}$.
    \item\label{base nilp c} Calcular los autovalores de $T$ y sus correspondientes autoespacios. Decidir si $T$ es diagonalizable.
\end{enumerate}

\rta


\qed


        
\item Definir en cada caso una transformación lineal $T:\mathbb{R}^3\longrightarrow\mathbb{R}^3$ que satisfaga las condiciones requeridas. ¿Es posible definir más de una transformación lineal?
\begin{enumerate}
    \item\label{tl-condiciones-a} $(1,0,0)\in \operatorname{Nu}(T)$ 
    \item\label{tl-condiciones-b} $(1,0,0)\in \operatorname{Im}(T)$ 
    \item\label{tl-condiciones-c} $(1,0,0),(1,2,1)\in\operatorname{Nu}(T)$ y $(1,0,0) \in  \operatorname{Im}(T)$
\end{enumerate}

\rta


\qed



\item Decidir si las siguientes afirmaciones son verdaderas o falsas. Justificar. 
\begin{enumerate}
    \item\label{tl-V-o-F-a} Existe una transformación lineal $T:\mathbb{R}^3\longrightarrow\mathbb{R}^3$ tal que $\langle(1,2,3),(2,1,-1)\rangle$ es el autoespacio asociado a $0$ y $\langle(3,1,1),(1,1,3)\rangle$ es el autoespacio asociado a $5$.
    \item\label{tl-V-o-F-b} Existe una transformación lineal $T:\mathbb{R}^3\longrightarrow\mathbb{R}^3$ tal que $\langle(1,2,3)\rangle$ es el autoespacio asociado a $0$ y $\langle(3,1,1)\rangle$ es el autoespacio asociado a $5$.
    \item\label{tl-V-o-F-c} Existe una transformación lineal $T:\mathbb{R}^3\longrightarrow\mathbb{R}^3$ tal que $\{(1,0,1), (0,1,0)\}$ es una base de $\operatorname{Nu}(T)$ y  $\{(1,0,-1), (0,1,0)\}$ es una base de la $\operatorname{Im}(T)$.
    \item\label{tl-V-o-F-d} Existe una transformación lineal $T:\mathbb{R}^3\longrightarrow\mathbb{R}^3$ tal que $\{(1,0,1)\}$ es una base de $\operatorname{Nu}(T)$ y  $\{(1,0,-1), (0,1,0)\}$ es una base de la $\operatorname{Im}(T)$. 
\end{enumerate}

\rta


\qed


\end{enumerate}


