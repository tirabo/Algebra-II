\chapter{Soluciones\\Álgebra  II -- Año 2024/1 -- FAMAF}\label{practico-8}

\begin{enumerate}[topsep=6pt, itemsep=.4cm]

\item
Probar que los vectores $\;v_1=(1,0,-i),\;
v_2=(1+i,1-i,1),\;v_3=(i,i,i)$ forman una base de $\mathbb{C}^3$,
y dar las coordenadas de un vector $(x,y,z)$ en esta base.


\item Dados los siguientes vectores de $\mathbb{R}^4$,
$$
v_1=(1,1,0,0), \quad v_2=(0,0,1,1), \quad v_3=(1,0,0,4),
\quad v_4=(0,0,0,2).
$$
\begin{enumerate}
    \item Demostrar que
    $\mathcal{B}=\{v_1,v_2,v_3,v_4\}$ es una base de
    $\mathbb{R}^{4}$.
    \item Hallar las coordenadas de los vectores de la
    base can\'onica respecto de $\mathcal{B}$.
    \item Hallar las matrices de cambio de base de la base can\'onica
    a $\mathcal{B}$ y viceversa.
\end{enumerate}


\item  Sea $V=P_3$.
Sean
$$ g_1=1-x,\quad g_2=x+x^2, \quad g_3=(x+1)^2.$$
\begin{enumerate}
    \item Demostrar que $\mathcal{B}=\{g_1,g_2,g_3\}$ es una base de $V$.
    \item Hallar las matrices de cambio de base con respecto a $\mathcal{B}$
    y a la base can{\'o}nica $\{1,x,x^2\}$.
\end{enumerate}



\item \label{lineales1-bases} Escribir las matrices de las transformaciones lineales de las siguientes transformaciones lineales respecto de las bases canónicas de los espacios involucrados.

    \begin{enumerate}
        \item\label{lineales1-bases-a} $T:\R^2 \longrightarrow \R^3$, $T(x,y)=(x-y,x+y,2x+3y)$.
        \item\label{lineales1-bases-b} $S:\R^3 \longrightarrow \R^2$, $S(x,y,z)=(x-y+z,2x-y+2z)$.
        \item\label{lineales1-base-c} $D:P_4  \longrightarrow P_4$, $D(p(x))=p'(x)$.
        \item $T:M_{2\times 2}(\mathbb{K}) \longrightarrow \mathbb{K}$, $T(A)=\operatorname{tr}(A)$.
        \item\label{lineales1-base-d} $L:P_3 \longrightarrow M_{2\times 2}(\R)$, $L(ax^2+bx+c)=\begin{bmatrix} a & b+c \\ b+c & a \end{bmatrix}$.
        \item\label{lineales1-base-e} $Q:P_3 \longrightarrow P_4$, $Q(p(x))=(x+1)p(x)$.
    \end{enumerate}



\item Sea $\mathcal{B}$ el conjunto 
$$
\left\{
\begin{bmatrix}
2 & 0& 0 \\
0 & 3& 1
\end{bmatrix},
\begin{bmatrix}
0& 2& -1\\
0& 2&-1
\end{bmatrix},
\begin{bmatrix}
-1 &1&1 \\
2 & 0 &2
\end{bmatrix},
\begin{bmatrix}
1 &0 &1\\
0 &0 &0
\end{bmatrix},
\begin{bmatrix}
1& 0& 0\\
0 &2 &0
\end{bmatrix},
\begin{bmatrix}
0 &0 & 0\\
1 & 2&1
\end{bmatrix}
\right\}.$$


\begin{enumerate}
    \item Demostrar que
    $\mathcal{B}$ es una base de $M_{2\times3}(\mathbb{R})$.
    \item Hallar las coordenadas de
    $
    \begin{bmatrix}
    1 & 1& 1 \\
    1 & 1& 1
    \end{bmatrix}$ con respecto a la base $\mathcal{B}$.
    \item Hallar las matrices de cambio de base de la base can\'onica
    a $\mathcal{B}$ y viceversa.
\end{enumerate}



\item Sea $W=<v_1,v_2>$, el subespacio de $\mathbb{C}^3$
generado por $v_1=(1,0,i)$ y $v_2=(1+i,1,-1)$.
\begin{enumerate}
    \item Demostrar que $\mathcal{B}_1=\{v_1,v_2\}$ es una base de $W$.
    \item Describir $W$ impl{\'\i}citamente.
    \item Demostrar que los vectores $w_1=(1,1,0)$ y
    $w_2=(1,i,1+i)$ pertenecen a $W$ y que $\mathcal{B}_2=\{w_1,w_2\}$
    es otra base de $W$.
    \item  ¿Cu{\'a}les son las coordenadas de $v_1$ y $v_2$ en la
    base ordenada $\mathcal{B}_2$?
    \item Hallar las matrices de cambio de base
    $P_{\mathcal{B}_1,\mathcal{B}_2}$ y $P_{\mathcal{B}_2,\mathcal{B}_1}$.
\end{enumerate}


\item Sea $\mathcal{B}=
\left\{
\begin{bmatrix}
2 & 0\\
0 & -1
\end{bmatrix},
\begin{bmatrix}
0& -1\\
0& 0
\end{bmatrix},
\begin{bmatrix}
-1 &0\\
\; 0 & 1
\end{bmatrix},
\begin{bmatrix}
0 &1 \\
1 &0
\end{bmatrix}
\right\} \subseteq M_2(\mathbb R)$.



\begin{enumerate}
    \item Demostrar que $\mathcal{B}$ es una base de $M_2(\mathbb R)$.
    

    \item Hallar las matrices de cambio de base de la base ordenada can\'onica de $M_2(\mathbb R)$ a la base ordenada $\mathcal{B}$ y viceversa.
        
    \item Hallar las coordenadas de la matriz
    $
    \begin{bmatrix}
    2 & 1\\
    1 & 2
    \end{bmatrix}$ con respecto a la base ordenada $\mathcal{B}$.
\end{enumerate}


\item\label{otras bases} Sean $\mathcal{C}_n$, $n=2,3$, las bases canónica de $\R^2$ y $\R^3$ respectivamente. Sean
$\mathcal{B}_2=\{(1,0),(1,1)\}$ y $\mathcal{B}_3=\{(1,0,0),(1,1,0),(1,1,1)\}$ bases de $\R^2$, $\R^3$, respectivamente.
\begin{enumerate}
    \item Escribir la matriz de cambio de base $P_{{\mathcal{C}_n},{\mathcal{B}_n}}$ de $\mathcal{C}_n$ a $\mathcal{B}_n$, $n=2,3$.
    \item Escribir la matriz de cambio de base $P_{{\mathcal{B}_n},{\mathcal{C}_n}}$ de $\mathcal{B}_n$ a $\mathcal{C}_n$, $n=2,3$.
    \item ¿Qué relación hay entre $P_{{\mathcal{C}_n},{\mathcal{B}_n}}$ y $P_{{\mathcal{B}_n},{\mathcal{C}_n}}$?
\end{enumerate}


\item \label{basesRn} Sean $\mathcal{C}_n, \mathcal{B}_n$ como en el ejercicio anterior.
\begin{enumerate}
    \item Dar las matrices de las transformaciones de los ejercicios  \ref{lineales1-bases}\ref{lineales1-bases-a}  y \ref{lineales1-bases}\ref{lineales1-bases-b} respecto de las bases $\mathcal{B}_n$ y
    $\mathcal{C}_n$.
    \item Dar las matrices de las transformaciones  de los ejercicios  \ref{lineales1-bases}\ref{lineales1-bases-a}  y \ref{lineales1-bases}\ref{lineales1-bases-b} respecto de las bases $\mathcal{C}_n$ y
    $\mathcal{B}_n$.
    \item Dar las matrices de las transformaciones  de los ejercicios  \ref{lineales1-bases}\ref{lineales1-bases-a}  y \ref{lineales1-bases}\ref{lineales1-bases-b} respecto de las bases $\mathcal{B}_n$ y
    $\mathcal{B}_n$.
\end{enumerate}




\end{enumerate}


