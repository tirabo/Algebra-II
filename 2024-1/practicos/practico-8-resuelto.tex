\chapter{Soluciones\\Álgebra  II -- Año 2024/1 -- FAMAF}\label{practico-8}

\begin{enumerate}[topsep=6pt, itemsep=.4cm]

\item Dar las coordenadas del polinomio $2x^2+10x-1\in\mathbb{K}_3[x]$ en la base ordenada $$\mathcal{B}=\{1,x+1,x^2+x+1\}.$$

\rta 
\begin{align*}
    [2x^2+10x-1]_{\mathcal{B}}&= (a,b,c) \\
    &\Updownarrow \\
    2x^2+10x-1&=a\cdot 1+b\cdot (x+1)+c\cdot (x^2+x+1) \\
    &\Updownarrow \\
    2x^2+10x-1&=cx^2+(b+c)x+(a+b+c) \\ 
    &\Updownarrow \\
    c=2,\; b+c &= 10, \;a+b+c=-1.
\end{align*}
Este último renglón es un sistema que se resuelve fácilmente por sustitución: $c=2$, $b=8$ y $a=-11$. Por lo tanto,
\begin{align*}
    [2x^2+10x-1]_{\mathcal{B}}&= (-11,8,2).
\end{align*}

\qed


\item Dar las coordenadas de la matriz 
$A=\begin{bmatrix}
    1&2\\3&4 
    \end{bmatrix}
$ en la base ordenada 
$$
\mathcal{B}=\left\{
\begin{bmatrix}
    0&1\\0&0 
    \end{bmatrix},
\begin{bmatrix}
    0&0\\0&1 
    \end{bmatrix},
    \begin{bmatrix}
    1&0\\0&0 
    \end{bmatrix},
    \begin{bmatrix}
    0&0\\1&0 
    \end{bmatrix}
\right\}.
$$
Más generalmente, dar las coordenadas de cualquier matriz $\begin{bmatrix}
    a&b\\c&d 
    \end{bmatrix}$ en la base $\mathcal{B}$.

    \rta
    \begin{align*}
        \begin{bmatrix}
            a&b\\c&d 
            \end{bmatrix} &= b\begin{bmatrix}
            0&1\\0&0 
            \end{bmatrix}+d\begin{bmatrix}
            0&0\\0&1 
            \end{bmatrix}+a\begin{bmatrix}
            1&0\\0&0 
            \end{bmatrix}+c\begin{bmatrix}
            0&0\\1&0 
            \end{bmatrix} \\
            &\Updownarrow \\
            [A]_{\mathcal{B}} &= (b,d,a,c). 
    \end{align*}
    En particular,
    \begin{align*}
        \begin{bmatrix}
            1&2\\3&4 
            \end{bmatrix}_{\mathcal{B}} &= (2,4,1,3).
    \end{align*}


    \qed
    
    

\item 
\begin{enumerate}
    \item\label{subespacio W a} Dar una base del subespacio $W=\{(x,y,z)\in\mathbb{K}^3\mid x-y+2z=0\}$. 
    \item\label{subespacio W b} Dar las coordenadas de $w=(1,-1,-1)$ en la base que haya dado en el item anterior.
    \item\label{subespacio W c} Dado $(x,y,z)\in W$, dar las coordenadas de $(x,y,z)$ en la base que haya calculado en el item anterior. 
\end{enumerate}

\rta

\ref{subespacio W a} 
\begin{align*}
    W &= \{(x,y,z)\in\mathbb{K}^3\mid x-y+2z=0\} \\
    &= \{(x,y,z)\in\mathbb{K}^3\mid x=y-2z\} \\
    &= \{(y-2z,y,z)\mid y,z\in\mathbb{K}\} \\
    &= \{y(1,1,0)+z(-2,0,1)\mid y,z\in\mathbb{K}\}.
\end{align*}
Por lo tanto, $\{(1,1,0),(-2,0,1)\}$ generan $W$. Además, como $(1,1,0)$ y $(-2,0,1)$ son LI, entonces $\mathcal{B}=\{(1,1,0),(-2,0,1)\}$ es una base ordenada de $W$. 

\vskip .3cm

\ref{subespacio W b} Primero,  es claro que $w=(1,-1,-1)\in W$, pues cumple con la ecuación implícita que define $W$. Entonces
\begin{align*}
    [w]_{\mathcal{B}} = (a,b) &\Leftrightarrow (1,-1,-1) = a(1,1,0)+b(-2,0,1) \\
    &\Leftrightarrow 1=a-2b, \;-1=a, \;-1=b.
\end{align*}
Por lo tanto
\begin{align*}
    [w]_{\mathcal{B}} &= (-1,-1).
\end{align*}

\vskip .3cm

\ref{subespacio W c} Sea $(x,y,z)\in W$. Entonces
\begin{align*}
    [(x,y,z)]_{\mathcal{B}} = (a,b)&\Leftrightarrow (x,y,z) = a(1,1,0)+b(-2,0,1) \\
    &\Leftrightarrow x=a-2b, \;y=a, \;z=b.
\end{align*}
Por lo tanto
\begin{align*}
    [(x,y,z)]_{\mathcal{B}} &= (y,z).
\end{align*}


\qed


\item \label{lineales1-bases} Escribir las matrices de las siguientes transformaciones lineales respecto de las bases canónicas de los espacios involucrados.
\begin{enumerate}
    \item\label{lineales1-bases-a} $T:\R^2 \longrightarrow \R^3$, $T(x,y)=(x-y,x+y,2x+3y)$.
    \item\label{lineales1-bases-b} $S:\R^3 \longrightarrow \R^2$, $S(x,y,z)=(x-y+z,2x-y+2z)$.
    \item\label{lineales1-base-c} $D:P_4  \longrightarrow P_4$, $D(p(x))=p'(x)$.
    \item $T:M_{2\times 2}(\mathbb{K}) \longrightarrow \mathbb{K}$, $T(A)=\operatorname{tr}(A)$.
    \item\label{lineales1-base-d} $L:P_3 \longrightarrow M_{2\times 2}(\R)$, $L(ax^2+bx+c)=\begin{bmatrix} a & b+c \\ b+c & a \end{bmatrix}$.
    \item\label{lineales1-base-e} $Q:P_3 \longrightarrow P_4$, $Q(p(x))=(x+1)p(x)$.
\end{enumerate}

\rta denotemos $\mathcal{C}_n$  a la base canónica de $\R^n$, Denotemos $\mathcal{B}_n$ a la base $\{1,x,x^2,\ldots,x^{n-1}\}$ de $P_n$ y denotemos $\mathcal{M}_{2\times 2}$ a la base$\{E_{11},E_{12},E_{21},E_{22}\}$ de  $M_{2\times 2}(\mathbb{K})$.

\vskip .3cm

\ref{lineales1-bases-a} 
\begin{align*}
    T(e_1) &= T(1,0) = (1,1,2) = 1(1,0,0) + 1(0,1,0) +2(0,0,1)\\
    T(e_2) &= T(0,1) = (-1,1,3) = -1(1,0,0) +1(0,1,0) +3(0,0,1).
\end{align*}
Por lo tanto, la matriz de $T$ respecto de las bases canónicas es
\begin{align*}
    [T]_{\,\mathcal{C}_2\,\mathcal{C}_3} = \begin{bmatrix} 1 & -1 \\ 1 & 1 \\ 2 & 3 \end{bmatrix}.
\end{align*}

\vskip .3cm

\ref{lineales1-bases-b} 
\begin{align*}
    S(e_1) &= S(1,0,0) = (1,2) = 1(1,0) + 2(0,1)\\
    S(e_2) &= S(0,1,0) = (-1,-1) = -1(1,0) -1(0,1)\\
    S(e_3) &= S(0,0,1) = (1,2) = 1(1,0) + 2(0,1).
\end{align*}
Por lo tanto, la matriz de $S$ respecto de las bases canónicas es
\begin{align*}
    [S]_{\,\mathcal{C}_3\,\mathcal{C}_2} = \begin{bmatrix} 1 & -1 & 1 \\ 2 & -1 & 2 \end{bmatrix}.
\end{align*}

\vskip .3cm

\ref{lineales1-base-c} 
\begin{align*}
    D(1) &= 0 = 0 \cdot 1 + 0\cdot x + 0\cdot x^2 + 0\cdot x^3\\
    D(x) &= 1 =1 \cdot 1 + 0\cdot x + 0\cdot x^2 + 0\cdot x^3\\
    D(x^2) &= 2x =0 \cdot 1 + 2\cdot x + 0\cdot x^2 + 0\cdot x^3\\
    D(x^3) &= 3x^2 =0 \cdot 1 + 0\cdot x + 3\cdot x^2 + 0\cdot x^3.
\end{align*}
Por lo tanto, la matriz de $D$ respecto de las bases canónicas es
\begin{align*}
    [D]_{\,\mathcal{B}_4\,\mathcal{B}_4} = \begin{bmatrix} 0 & 1 & 0 & 0 \\ 0 & 0 & 2 & 0 \\ 0 & 0 & 0 & 3 \\ 0 & 0 & 0 & 0 \end{bmatrix}.
\end{align*}


\vskip .3cm

\ref{lineales1-base-d} 
\begin{align*}
    L(1) &= \begin{bmatrix} 1 & 0 \\ 0 & 1 \end{bmatrix} = 1\cdot E_{11} + 0\cdot E_{12} + 0\cdot E_{21} + 1\cdot E_{22}\\
    L(x) &= \begin{bmatrix} 0 & 1 \\ 1 & 0 \end{bmatrix} = 0\cdot E_{11} + 1\cdot E_{12} + 1\cdot E_{21} + 0\cdot E_{22}\\
    L(x^2) &= \begin{bmatrix} 0 & 1 \\ 1 & 0 \end{bmatrix}= 0\cdot E_{11} + 1\cdot E_{12} + 1\cdot E_{21} + 0\cdot E_{22}\\
\end{align*}
Por lo tanto, la matriz de $L$ respecto de las bases canónicas es
\begin{align*}
    [L]_{\,\mathcal{B}_3\,\mathcal{M}_{2\times 2}} = \begin{bmatrix} 1 & 0 & 0 \\ 0 & 1 & 1 \\ 0 & 1 & 1 \\ 1 & 0 & 0 \end{bmatrix}.
\end{align*}

\vskip .3cm

\ref{lineales1-base-e}
\begin{align*}
    Q(1) &= x+1 = 1\cdot 1 + 1\cdot x + 0\cdot x^2 + 0\cdot x^3\\
    Q(x) &= (x+1)x = 0\cdot 1 + 1\cdot x + 1\cdot x^2 + 0\cdot x^3\\
    Q(x^2) &= (x+1)x^2 = 0\cdot 1 + 0\cdot x + 1\cdot x^2 + 1\cdot x^3.
\end{align*}
Por lo tanto, la matriz de $Q$ respecto de las bases canónicas es
\begin{align*}
    [Q]_{\,\mathcal{B}_3\,\mathcal{B}_4} = \begin{bmatrix} 1 & 0 & 0 \\ 1 & 1 & 0 \\ 0 & 1 & 1 \\ 0 & 0 & 1 \end{bmatrix}.
\end{align*}
\qed


\item\label{otras bases} Sea $\mathcal{C}$ la base canónica de $\mathbb{K}^2$ y 
    $\mathcal{B}=\{(1,0),(1,1)\}$ otra base de $\mathbb{R}^2$.
    \begin{enumerate}
        \item\label{otras bases-a} Encontrar la matriz de cambio de base $P_{\mathcal{C},\mathcal{B}}$ de $\mathcal{C}$ a $\mathcal{B}$.
        \item\label{otras bases-b} Encontrar la matriz de cambio de base $P_{\mathcal{B},\mathcal{C}}$ de $\mathcal{B}$ a $\mathcal{C}$.
        \item\label{otras bases-c} ¿Qué relación hay entre $P_{\mathcal{C},\mathcal{B}}$ y $P_{\mathcal{B},\mathcal{C}}$?
        \item\label{otras bases-d} Encontrar $(x,y),(z,w)\in\mathbb{K}^2$ tal que $[(x,y)]_{\mathcal{B}}=(1,4)$ y $[(z,w)]_{\mathcal{B}}=(1,-1)$.
        \item\label{otras bases-e} Determinar las coordenadas de $(2,3)$ y $(0,1)$ en las bases $\mathcal{B}_2$.
    \end{enumerate}

    \rta recordar que si $V$ es un espacio vectorial de dimensión finita sobre el cuerpo $\K$ y  $\mathcal{B}$ y $\mathcal{B'}$ son bases ordenadas de $V$, la matriz $[\Id]_{\mathcal{B}\mathcal{B'}}$  es llamada la \textit{matriz de cambio de base} de la base $\mathcal{B}'$  a la base $\mathcal{B}$. 

    \ref{otras bases-a} 
    \begin{align*}
        \Id(1,0) &= 1 \cdot (1,0) + 0 \cdot (1,1) \\
        \Id(0,1) &= (-1) \cdot (1,0) + 1 \cdot (1,1).
    \end{align*}
    Por lo tanto, 
    $$
    P_{\mathcal{C},\mathcal{B}} = \begin{bmatrix} 1 & -1 \\ 0 & 1 \end{bmatrix}.
    $$

    \vskip .3cm

    \ref{otras bases-b} 
    \begin{align*}
        \Id(1,0) &= 1 \cdot (1,0) + 0 \cdot (0,1) \\
        \Id(1,1) &= 1 \cdot (1,0) + 1 \cdot (0,1).
    \end{align*}
    Por lo tanto,
    $$
    P_{\mathcal{B},\mathcal{C}} = \begin{bmatrix} 1 & 1 \\ 0 & 1 \end{bmatrix}.
    $$

    \vskip .3cm

    \ref{otras bases-c}
    $$
    P_{\mathcal{C},\mathcal{B}}  P_{\mathcal{B},\mathcal{C}} = \Id.    
    $$
    Por la teórica sabemos que esta relación vale en general.

    \vskip .3cm

    \ref{otras bases-d} 
    \begin{align*}
        [(x,y)]_{\mathcal{B}} = (1,4)  &\Leftrightarrow (x,y) = 1 \cdot (1,0) + 4 \cdot (1,1) \\
        &\Leftrightarrow x=5, \;y=4.
    \end{align*}
    También es posible hacerlo por la matriz de cambio de base:
    \begin{align*}
        \begin{bmatrix}
            x \\ y
        \end{bmatrix} = P_{\mathcal{B}, \mathcal{C}} [(x,y)]_\mathcal{B} = P_{\mathcal{B}, \mathcal{C}} \begin{bmatrix}
            1 \\ 4
        \end{bmatrix} = \begin{bmatrix}
            1 & 1 \\ 0 & 1
        \end{bmatrix} \begin{bmatrix}
            1 \\ 4
        \end{bmatrix} = \begin{bmatrix}
            5 \\ 4
        \end{bmatrix}.
    \end{align*}

    De forma análoga, 
    \begin{align*}
        [(z,w)]_{\mathcal{B}} = (1,-1)  &\Leftrightarrow (z,w) = 1 \cdot (1,0) + (-1) \cdot (1,1) \\
        &\Leftrightarrow z=0, \;w=-1.
    \end{align*}
    Obviamente, también es posible hacerlo por la matriz de cambio de base:
    \begin{align*}
        \begin{bmatrix}
            z \\ w
        \end{bmatrix} = P_{\mathcal{B}, \mathcal{C}} [(z,w)]_\mathcal{B} = P_{\mathcal{B}, \mathcal{C}} \begin{bmatrix}
            1 \\ -1
        \end{bmatrix} = \begin{bmatrix}
            1 & 1 \\ 0 & 1
        \end{bmatrix} \begin{bmatrix}
            1 \\ -1
        \end{bmatrix} = \begin{bmatrix}
            0 \\ -1
        \end{bmatrix}.
    \end{align*}

    \vskip .3cm

    \ref{otras bases-e}
    \begin{align*}
        [(2,3)]_{\mathcal{B}} = P_{\mathcal{C},\mathcal{B}} [(2,3)]_\mathcal{C} = P_{\mathcal{C},\mathcal{B}} \begin{bmatrix}
            2 \\ 3
        \end{bmatrix} = \begin{bmatrix}
            1 & -1 \\ 0 & 1
        \end{bmatrix} \begin{bmatrix}
            2 \\ 3
        \end{bmatrix} = \begin{bmatrix}
            -1 \\ 3
        \end{bmatrix}.
    \end{align*}
    El caso general es análogo:
    \begin{align*}
        [(a,b)]_{\mathcal{B}} = P_{\mathcal{C},\mathcal{B}} [(a,b)]_\mathcal{C} = P_{\mathcal{C},\mathcal{B}} \begin{bmatrix}
            a \\ b
        \end{bmatrix} = \begin{bmatrix}
            1 & -1 \\ 0 & 1
        \end{bmatrix} \begin{bmatrix}
            a \\ b
        \end{bmatrix} = \begin{bmatrix}
            a-b \\ b
        \end{bmatrix}.
    \end{align*}


    \qed
    
    

    \item\label{matriz P} Sea $P=\begin{bmatrix}1&1&0\\2&1&1\\3&1&0\end{bmatrix}
        \in\mathbb{K}^{3\times 3}$.
        \vskip .2cm
\begin{enumerate}
\item\label{inversa de P} Calcular la inversa de $P$.
\item\label{base de P} Dar una base ordenada $\mathcal{B}$ de $\mathbb{K}^3$  
tal que $P$ es la matriz de cambio de coordenadas de la base canónica de $\mathbb{K}^3$ a la
base $\mathcal{B}$.
\item\label{encontrar vector} Encontrar $(x,y,z)\in\mathbb{K}^3$ tal que su vector de coordenadas con respecto a $\mathcal{B}$ es 
$$[(x,y,z)]_{\mathcal{B}}=(2,-1,-1).$$
\end{enumerate}

\rta 

\ref{inversa de P} Utilizamos el método de Gauss-Jordan:
\begin{align*}
    &\left[\begin{array}{ccc|ccc}
    1&1&0&1&0&0\\2&1&1&0&1&0\\3&1&0&0&0&1
    \end{array}\right] \underset{F_3-3F_1}{\stackrel{F_2-2F_1}{\longrightarrow}} \left[\begin{array}{ccc|ccc}
    1&1&0&1&0&0\\0&-1&1&-2&1&0\\0&-2&0&-3&0&1
    \end{array}\right] \\
    &\stackrel{-F_2}{\longrightarrow} \left[\begin{array}{ccc|ccc}
    1&1&0&1&0&0\\0&1&-1&2&-1&0\\0&-2&0&-3&0&1
    \end{array}\right] \underset{F_3+2F_2}{\stackrel{F_1-F_2}{\longrightarrow}} 
    \left[\begin{array}{ccc|ccc}
    1&0&1&-1&1&0\\0&1&-1&2&-1&0\\0&0&-2&1&-2&1
    \end{array}\right] \\
    &\underset{-\frac{1}{2}F_3}{\longrightarrow} \left[\begin{array}{ccc|ccc}
    1&0&1&-1&1&0\\0&1&-1&2&-1&0\\0&0&1&-\frac{1}{2}&1&-\frac{1}{2}
    \end{array}\right] \underset{F_2+F_3}{\stackrel{F_1-F_3}{\longrightarrow}}
    \left[\begin{array}{ccc|ccc}
    1&0&0&-\frac{1}{2}&0&\frac{1}{2}\\0&1&0&\frac{3}{2}&0&-\frac{1}{2}\\0&0&1&-\frac{1}{2}&1&-\frac{1}{2}
    \end{array}\right].
\end{align*}
Concluyendo, 
\begin{align*}
    P^{-1} = \begin{bmatrix}
        -\frac{1}{2}&0&\frac{1}{2}\\
        \frac{3}{2}&0&-\frac{1}{2}\\
        -\frac{1}{2}&1&-\frac{1}{2}
    \end{bmatrix} = \frac12 \begin{bmatrix}
        -1&0&1\\
        3&0&-1\\
        -1&2&-1
    \end{bmatrix}.
\end{align*} 

\vskip .3cm

\ref{base de P} Si $\mathcal C$ es la base canónica, queremos encontrar  $\mathcal{B}=\{v_1,v_2,v_3\}$, tal que $[\Id]_{\mathcal{C}\mathcal{B}}=P$. Es decir,  queremos encontrar $v_1,v_2,v_3 \in \R^3$ tal que
\begin{align*}
    e_1 &= \Id(e_1) = 1 \cdot v_1 + 2 \cdot v_2 + 3 \cdot v3\\
    e_2 &= \Id(e_2) = 1 \cdot v_1 + 1 \cdot v_2 + 1 \cdot v3 \\
    e_3 &= \Id(e_3) = 0 \cdot v_1 + 1 \cdot v_2 + 0 \cdot v3.
\end{align*}
Podemos plantear este sistemas de ecuaciones en coordenadas y resolverlo, pero es más fácil observar que $P^{-1} = [\Id]_{\mathcal{B}\mathcal{C}}$ y por lo tanto, $v_1,v_2,v_3$ son las columnas de $P^{-1}$, es decir
\begin{align*}
    \mathcal{B} = \left\{ (-\frac{1}{2},\frac{3}{2},-\frac{1}{2}), (0,0,1), (\frac{1}{2},-\frac{1}{2},-\frac{1}{2}) \right\}.
\end{align*}


\vskip .3cm

\ref{encontrar vector} Recordar que 
$$
[v]_{\mathcal{B}} = (a,b,c) \qquad \Leftrightarrow \qquad v = a \cdot v_1 + b \cdot v_2 + c \cdot v_3.
$$
por lo tanto, $[(x,y,z)]_{\mathcal{B}}=(2,-1,-1)$ significa que
\begin{align*}
    (x,y,z) &= 2 \cdot (-\frac{1}{2},\frac{3}{2},-\frac{1}{2}) + (-1) \cdot (0,0,1) + (-1) \cdot (\frac{1}{2},-\frac{1}{2},-\frac{1}{2}) \\
    &=  (-1,3,-1) +  (0,0,-1) +  (-\frac{1}{2},\frac{1}{2},\frac{1}{2})\\
    &=  (-\frac{3}{2},\frac{5}{2},-\frac{3}{2}).
\end{align*}

\qed


\item\label{otras bases 2} Sean $\mathcal{C}_n$, $n=2,3$, las bases canónica de $\R^2$ y $\R^3$ respectivamente. Sean
$\mathcal{B}_2=\{(1,0),(1,1)\}$ y $\mathcal{B}_3=\{(1,0,0),(1,1,0),(1,1,1)\}$ bases de $\R^2$, $\R^3$, respectivamente.
\begin{enumerate}
    \item\label{otras bases 2a} Escribir la matriz de cambio de base $P_{{\mathcal{B}_n},{\mathcal{C}_n}}$ de $\mathcal{B}_n$ a $\mathcal{C}_n$, $n=2,3$.
    \item\label{otras bases 2b}  Escribir la matriz de cambio de base $P_{{\mathcal{C}_n},{\mathcal{B}_n}}$ de $\mathcal{C}_n$ a $\mathcal{B}_n$, $n=2,3$. 
\end{enumerate}

\rta

\ref{otras bases 2a} Denotemos  $v_1 = (1, 0)$, $v_2=(1 ,1)$. 
Luego, por definición la matriz de cambio de base  de  $\mathcal B_2$ a $\mathcal C_2$ es
    \begin{equation*}
        P_{{\mathcal{B}_2},{\mathcal{C}_2}}=  [\operatorname{Id}]_{\mathcal B_2\mathcal C_2} =
        \begin{bmatrix}  \mid & \mid\\ v_1 & v_2\\ \mid & \mid &\end{bmatrix} =
        \begin{bmatrix}
        1&1 \\
        0&1\end{bmatrix}. \tag{*}
    \end{equation*}

Análogamente, denotemos  $w_1 = (1, 0,0)$, $w_2=(1 ,1,0)$, $w_3=(1 ,1,1)$, por lo tanto
    \begin{equation*}
        P_{{\mathcal{B}_3},{\mathcal{C}_3}}=  [\operatorname{Id}]_{\mathcal B_3\mathcal C_3} =
        \begin{bmatrix}  \mid & \mid & \mid\\ w_1 & w_2 & w_3\\ \mid & \mid & \mid \end{bmatrix} =
        \begin{bmatrix}
        1&1&1 \\
        0&1&1\\
        0&0&1\end{bmatrix}. \tag{**}
    \end{equation*}

\ref{otras bases 2b} La matriz de cambio de base  de  $\mathcal C_2$ a $\mathcal B_2$ es $[\Id]_{\mathcal C_2\mathcal B_2} = [\Id]_{{\mathcal{B}_2},{\mathcal{C}_2}}^{-1}$, por lo tanto debemos calcular la inversa de la matriz (*). Utilizamos el método de Gauss-Jordan:
\begin{align*}
    &\left[\begin{array}{cc|cc}
    1&1&1&0\\0&1&0&1
    \end{array}\right] \underset{F_1-F_2}{\longrightarrow} \left[\begin{array}{cc|cc}
    1&0&1&-1\\0&1&0&1
    \end{array}\right].
\end{align*}
En  consecuencia:
    \begin{equation*}
        P_{{\mathcal{C}_2},{\mathcal{B}_2}}=  [\operatorname{Id}]_{\mathcal C_2\mathcal B_2} =
        \begin{bmatrix}
        1&-1 \\
        0&1\end{bmatrix}. 
    \end{equation*}
    
    Análogamente, la matriz de cambio de base  de  $\mathcal C_3$ a $\mathcal B_3$ es $[\Id]_{\mathcal C_3\mathcal B_3} = [\Id]_{{\mathcal{B}_3},{\mathcal{C}_3}}^{-1}$, por lo tanto debemos calcular la inversa de la matriz (**). Utilizamos el método de Gauss-Jordan:
\begin{align*}
    &\left[\begin{array}{ccc|ccc}
    1&1&1&1&0&0\\0&1&1&0&1&0\\0&0&1&0&0&1
    \end{array}\right] \underset{F_2-F_3}{\stackrel{F_1-F_3}{\longrightarrow}} \left[\begin{array}{ccc|ccc}
    1&1&0&1&0&-1\\0&1&0&0&1&-1\\0&0&1&0&0&1
    \end{array}\right] \\
    &\underset{F_1-F_2}{\longrightarrow} \left[\begin{array}{ccc|ccc}
    1&0&0&1&-1&0\\0&1&0&0&1&-1\\0&0&1&0&0&1
    \end{array}\right].
\end{align*}
En  consecuencia:
    \begin{equation*}
        P_{{\mathcal{C}_3},{\mathcal{B}_3}}=  [\operatorname{Id}]_{\mathcal C_3\mathcal B_3} =
        \begin{bmatrix}
        1&-1&0 \\
        0&1&-1\\
        0&0&1\end{bmatrix}. 
    \end{equation*}
\qed
    
\item \label{basesRn} Sean $\mathcal{C}_n, \mathcal{B}_n$ como en el ejercicio \ref{otras bases 2} y sean las siguientes transformaciones lineales:
\begin{itemize}
    \item\label{lineales1-a-2} $T:\R^2 \longrightarrow \R^3$, $T(x,y)=(x-y,x+y,2x+3y)$.
    \item\label{lineales1-b-2} $S:\R^3 \longrightarrow \R^2$, $S(x,y,z)=(x-y+z,2x-y+2z)$.
\end{itemize}

Entonces, para cada una de las transformaciones lineales anteriores,
\begin{enumerate}
    \item \label{basesRn-a} Dar las matrices respecto a las bases $\mathcal{B}_n$ y $\mathcal{C}_n$.
    \item \label{basesRn-b} Dar las matrices respecto a las bases $\mathcal{C}_n$ y $\mathcal{B}_n$.
    \item \label{basesRn-c} Dar las matrices respecto a las bases $\mathcal{B}_n$ y $\mathcal{B}_n$.
\end{enumerate}


\rta

\ref{basesRn-a} Para calcular $[T]_{\mathcal{B}_2\mathcal{C}_3}$ debemos calcular $T(v_1)$ y $T(v_2)$ en función de la base $\mathcal{C}_3$. Pero como esta última es la canónica debemos calcular $T(1,0)$ y $T(1,1)$ y ponerlos como columnas:
\begin{align*}
    T(1,0) &= (1,1,2) = 1 \cdot (1,0,0) + 1 \cdot (0,1,0) + 2 \cdot (0,0,1) \\
    T(1,1) &= (0,2,5) = 0 \cdot (1,0,0) + 2 \cdot (0,1,0) + 5 \cdot (0,0,1).
\end{align*}
Luego,
$$
[T]_{\mathcal{B}_2\mathcal{C}_3} = \begin{bmatrix}
    1 & 0 \\ 1 & 2 \\ 2 & 5 
\end{bmatrix}.
$$


Análogamente,
\begin{align*}
    S(1,0,0) &= (1,2), \\ S(1,1,0) &= (0,1), \\  S(1,1,1) &= (1,3).
\end{align*}
Luego, 
$$
[S]_{\mathcal{B}_3\mathcal{C}_2} = \begin{bmatrix}
    1 & 0 & 1 \\ 2 & 1 & 3
\end{bmatrix}.
$$

\ref{basesRn-b} Para calcular $[T]_{\mathcal{C}_2\mathcal{B}_3}$ debemos calcular $T(e_1)$ y $T(e_2)$ en función de la base $\mathcal{B}_3$. Ahora bien,
\begin{align*}
    T(1,0) = (1,1,2) &= a_{11} \cdot (1,0,0) + a_{21} \cdot (1,1,0) + a_{31} \cdot (1,1,1) \\
    &=(a_{11} + a_{21} + a_{31}, a_{21} + a_{31}, a_{31}) \\
    T(0,1) = (-1,1,3) &= a_{12} \cdot (1,0,0) + a_{22} \cdot (1,1,0) + a_{32} \cdot (1,1,1)\\
    &=(a_{12} + a_{22} + a_{32}, a_{22} + a_{32}, a_{32}),
\end{align*}
y debemos calcular los $a_{ij}$. Estas ecuaciones son muy sencillas de resolver pues $a_{31} = 2$, $a_{21} = 1 -a_{31} = -1$, $a_{11} = 1 - a_{21} - a_{31} = 0$. Análogamente, $a_{32} = 3$, $a_{22} = 1 - a_{32} = -2$, $a_{12} = -1 - a_{22} - a_{32} = -2$. Por lo tanto,
$$
[T]_{\mathcal{C}_2\mathcal{B}_3} = \begin{bmatrix}
    0 & -2  \\ -1 & -2 \\ 2 & 3
\end{bmatrix}.
$$ 

Análogamente, para calcular $[S]_{\mathcal{C}_3\mathcal{B}_2}$ debemos calcular $S(e_1)$, $S(e_2)$ y $S(e_3)$ en función de la base $\mathcal{B}_2$. Ahora bien,
\begin{align*}
    S(1,0,0) &= (1,2) = a_{11} \cdot (1,0) + a_{21} \cdot (1,1)=(a_{11} + a_{21}, a_{21}) \\
    S(0,1,0) &= (-1,-1) = a_{12} \cdot (1,0) + a_{22} \cdot (1,1)=(a_{12} + a_{22}, a_{22}) \\
    S(0,0,1) &= (1,2) = a_{13} \cdot (1,0) + a_{23} \cdot (1,1)=(a_{13} + a_{23}, a_{23}).
\end{align*}
Por lo tanto, $a_{21} = 2$, $a_{11} = 1 - a_{21} = -1$, $a_{22} = -1$, $a_{12} = -1 - a_{22} = 0$, $a_{23} = 2$, $a_{13} = 1 - a_{23} = -1$. En  consecuencia,
$$
[S]_{\mathcal{C}_3\mathcal{B}_2} = \begin{bmatrix}
    -1 & 0 & -1 \\ 2 & -1 & 2
\end{bmatrix}.
$$

\ref{basesRn-c} Para calcular $[T]_{\mathcal{B}_2\mathcal{B}_3}$ debemos calcular $T(1,0)$ y $T(1,1)$ en función de la base $\{(1,0,0), (1,1,0), (1,1,1)\}$. Ahora bien,
\begin{align*}
    T(1,0) &= (1,1,2) = 0 \cdot (1,0,0) + (-1) \cdot (1,1,0) + 2 \cdot (1,1,1) \\
    T(1,1) &= (0,2,5) = (-2) \cdot (1,0,0) + (-3) \cdot (1,1,0) + 5 \cdot (1,1,1).
\end{align*}
Esto sale por cálculos sencillos. Por lo tanto, 
$$
[T]_{\mathcal{B}_2\mathcal{B}_3} = \begin{bmatrix}
    0 & -2 \\ -1 & -3 \\ 2 & 5
\end{bmatrix}.
$$

Análogamente, para calcular $[S]_{\mathcal{B}_3\mathcal{B}_2}$ debemos calcular $S(1,0,0)$, $S(1,1,0)$ y $S(1,1,1)$ en función de la base $\{(1,0), (1,1)\}$. Ahora bien,
\begin{align*}
    S(1,0,0) &= (1,2) = (-1) \cdot (1,0) + 2 \cdot (1,1) \\
    S(1,1,0) &= (0,1) = (-1) \cdot (1,0) + 1 \cdot (1,1) \\
    S(1,1,1) &= (1,3) = (-2) \cdot (1,0) + 3 \cdot (1,1).
\end{align*}
Por lo tanto,
$$
[S]_{\mathcal{B}_3\mathcal{B}_2} = \begin{bmatrix}
    -1 & -1 & -2 \\ 2 & 1 & 3
\end{bmatrix}.
$$

\vskip .3cm

\noindent\textbf{Observación.} Es posible hacer los incisos \ref{basesRn-b} y \ref{basesRn-c} de otra forma, utilizando las matrices de cambio de base. En estos casos quizás no valga la pena, pues la cuentas ``directas'' son muy sencillas, pero de cualquier forma lo haremos a modo ilustrativo.

\vskip .3cm
\ref{basesRn-b} (Otra forma) Observemos que
\begin{equation*}
    [T]_{\mathcal{C}_2\mathcal{B}_3} = [\Id]_{\mathcal{C}_3\mathcal{B}_3} [T]_{\mathcal{C}_2\mathcal{C}_3} = [\Id]_{\mathcal{B}_3\mathcal{C}_3}^{-1} [T]_{\mathcal{C}_2\mathcal{C}_3}.
\end{equation*}
Luego debemos primero calcular  $[\Id]_{\mathcal{B}_3\mathcal{C}_3}$ y $[T]_{\mathcal{C}_2\mathcal{C}_3}$, después la inversa de la primera matriz y finalmente multiplicar matrices. La matriz $[\Id]_{\mathcal{B}_3\mathcal{C}_3}$ esta formada por los vectores de la base $\mathcal{B}_3$ como columnas:
\begin{equation*}
    [\Id]_{\mathcal{B}_3\mathcal{C}_3} = \begin{bmatrix}
        1 & 1 & 1 \\ 0 & 1 & 1 \\ 0 & 0 & 1
    \end{bmatrix}.
\end{equation*}
Ya vimos que
\begin{align*}
    T(1,0) &= (1,1,2) \\
    T(0,1) &= (-1,1,3),
\end{align*}
y por lo tanto 
\begin{equation*}
    [T]_{\mathcal{C}_2\mathcal{C}_3} = \begin{bmatrix}
        1 & -1 \\ 1 & 1 \\ 2 & 3
    \end{bmatrix}.
\end{equation*}
Calculemos ahora la inversa de $[\Id]_{\mathcal{B}_3\mathcal{C}_3}$:
\begin{align*}
    &\left[\begin{array}{ccc|ccc}
    1&1&1&1&0&0\\0&1&1&0&1&0\\0&0&1&0&0&1
    \end{array}\right] \underset{F_2-F_3}{\stackrel{F_1-F_3}{\longrightarrow}} \left[\begin{array}{ccc|ccc}
    1&1&0&1&0&-1\\0&1&0&0&1&-1\\0&0&1&0&0&1
    \end{array}\right] \\
    &\underset{F_1-F_2}{\longrightarrow} \left[\begin{array}{ccc|ccc}
    1&0&0&1&-1&0\\0&1&0&0&1&-1\\0&0&1&0&0&1
    \end{array}\right].
\end{align*}
Por lo tanto,
\begin{equation*}
    [\Id]_{\mathcal{B}_3\mathcal{C}_3}^{-1} = \begin{bmatrix}
        1 & -1 & 0 \\ 0 & 1 & -1 \\ 0 & 0 & 1
    \end{bmatrix}.
\end{equation*}
Finalmente,
\begin{equation*}
    [T]_{\mathcal{C}_2\mathcal{B}_3} = \begin{bmatrix}
        1 & -1 & 0 \\ 0 & 1 & -1 \\ 0 & 0 & 1
    \end{bmatrix} \begin{bmatrix}
        1 & -1 \\ 1 & 1 \\ 2 & 3
    \end{bmatrix} = \begin{bmatrix}
        0 & -2 \\ -1 & -3 \\ 2 & 3
    \end{bmatrix},
\end{equation*}
matriz que, obviamente,  ya habíamos obtenido. 

\vskip .3cm

Para calcular, $[S]_{\mathcal{C}_3\mathcal{B}_2}$ usamos la fórmula:
$$
[S]_{\mathcal{C}_3\mathcal{B}_2} = [\Id]_{\mathcal{C}_2\mathcal{B}_2} [S]_{\mathcal{C}_3\mathcal{C}_2} = [\Id]_{\mathcal{B}_2\mathcal{C}_2}^{-1} [S]_{\mathcal{C}_3\mathcal{C}_2}.
$$

\vskip .3cm

\ref{basesRn-c} (Otra forma) Observemos que
\begin{equation*}
    [T]_{\mathcal{B}_2\mathcal{B}_3} = [\Id]_{\mathcal{C}_3\mathcal{B}_3} [T]_{\mathcal{C}_2\mathcal{C}_3}  [\Id]_{\mathcal{B}_2\mathcal{C}_2} = [\Id]_{\mathcal{B}_3\mathcal{C}_3}^{-1} [T]_{\mathcal{C}_2\mathcal{C}_3} [\Id]_{\mathcal{B}_2\mathcal{C}_2}.
\end{equation*}
Ahora bien, 
$$
[\Id]_{\mathcal{B}_3\mathcal{C}_3}^{-1} = \begin{bmatrix}
    1 & -1 & 0 \\ 0 & 1 & -1 \\ 0 & 0 & 1
\end{bmatrix}, \quad 
[T]_{\mathcal{C}_2\mathcal{C}_3} = \begin{bmatrix}
    1 & -1 \\ 1 & 1 \\ 2 & 3
\end{bmatrix}, \quad 
[\Id]_{\mathcal{B}_2\mathcal{C}_2} = \begin{bmatrix}
    1 & 1 \\ 0 & 1
\end{bmatrix}.
$$
Las dos primeras matrices fueron calculadas en el inciso anterior y la tercera es obvia. Por lo tanto,
\begin{equation*}
    [T]_{\mathcal{B}_2\mathcal{B}_3} = \begin{bmatrix}
        1 & -1 & 0 \\ 0 & 1 & -1 \\ 0 & 0 & 1
    \end{bmatrix} \begin{bmatrix}
        1 & -1 \\ 1 & 1 \\ 2 & 3
    \end{bmatrix} \begin{bmatrix}
        1 & 1 \\ 0 & 1
    \end{bmatrix} = \begin{bmatrix}
        0 & -2 \\ -1 & -3 \\ 2 & 5
    \end{bmatrix}.
\end{equation*}

\vskip .3cm
Finalmente, para calcular $[S]_{\mathcal{B}_3\mathcal{B}_2}$ usamos la fórmula:
$$
[S]_{\mathcal{B}_3\mathcal{B}_2} = [\Id]_{\mathcal{C}_2\mathcal{B}_2} [S]_{\mathcal{C}_3\mathcal{C}_2} [\Id]_{\mathcal{B}_3\mathcal{C}_3} = [\Id]_{\mathcal{B}_2\mathcal{C}_2}^{-1} [S]_{\mathcal{C}_3\mathcal{C}_2} [\Id]_{\mathcal{B}_3\mathcal{C}_3}.
$$
\qed


\item\label{matriz transformaciones ejemplo} Sea 
$T:\mathbb{R}^3\longrightarrow\mathbb{R}^2$ la transformación lineal definida por $$T(x,y,z)=(x-y,x-z).$$ Sean $\mathcal{C}$ la base canónica de $\mathbb{R}^3$ y $\mathcal{B}'=\{(1,1),(1,-1)\}$ base de $\mathbb{R}^2$.
\begin{enumerate}
    \item\label{matriz transformaciones ejemplo-a} Calcular la matriz $[T]_{\mathcal{C}\mathcal{B}'}$, es decir la matriz de $T$ respecto de las bases $\mathcal{C}$ y $\mathcal{B}'$.
    \item\label{matriz transformaciones ejemplo-b} Sea $(x,y,z)\in\mathbb{R}^3$. Dar las coordenadas de $T(x,y,z)$ respecto de la base $\mathcal{B}'$.
    \item\label{matriz transformaciones ejemplo-c} Sea $S:\mathbb{R}^2\longrightarrow\mathbb{R}^3$ una transformación lineal tal que su matriz respecto a las bases $\mathcal{B}'$ y $\mathcal{C}$ es
    \begin{align*}
    [S]_{\mathcal{B}'\mathcal{C}}=\begin{bmatrix}
    1&2\\1&-1\\1&0
    \end{bmatrix}. 
    \end{align*}
    Calcular la matriz de la composición $T\circ S:\mathbb{R}^2\longrightarrow\mathbb{R}^2$ con respecto a la base $\mathcal{B}'$. 
\end{enumerate}

\rta

\ref{matriz transformaciones ejemplo-a} 
$$
\begin{array}{cclcccccccc}
    T(1,0,0) &= &(1,1)  &= &1 &\cdot &(1,1) &+ &0 &\cdot &(1,-1) \\
    T(0,1,0) &= &(-1,0) &= &(-\frac12) &\cdot &(1,1) &+ &(-\frac12) &\cdot &(1,-1) \\
    T(0,0,1) &= &(0,-1) &= &(-\frac12) &\cdot &(1,1) &+  &\frac12 &\cdot &(1,-1).
\end{array}
$$
Por lo tanto, 
$$
[T]_{\mathcal{C}\mathcal{B}'} = \begin{bmatrix}
    0 & -1/2 & -1/2 \\ 1 & -1/2 & 1/2
\end{bmatrix}.
$$

\vskip .3cm
\ref{matriz transformaciones ejemplo-b} Dado el vector $(x,y,z)$, debemos resolver el sistema
\begin{align*}
    (x-y,x-z) &= a \cdot (1,1) + b \cdot (1,-1) \quad \Rightarrow\\
    (x-y,x-z) &= (a+b,a-b).
\end{align*}
Por lo tanto
\begin{align*}
    (x-y) + (x-z) &= 2a \;\Rightarrow\;  2x-y-z = 2 \;\Rightarrow\; a = \frac12 (2x-y-z) \\
    (x-y) - (x-z) &= 2b \;\Rightarrow\;  z-y = 2b \;\Rightarrow\; b = \frac12 (z-y).
\end{align*} 
Es decir,
$$
[T(x,y,z)]_{\mathcal{B}'} = \frac12 (2x-y-z, z-y).
$$

\vskip .3cm
\ref{matriz transformaciones ejemplo-c} Calculamos en en inciso \ref{matriz transformaciones ejemplo-a} la matriz  $[T]_{\mathcal{C}\mathcal{B}'}$,  entonces
\begin{align*}
    [T \circ S]_{\mathcal{B}'\mathcal{B}'} &= [T]_{\mathcal{C}\mathcal{B}'} [S]_{\mathcal{B}'\mathcal{C}} \\
    &= \begin{bmatrix}
        0 & -1/2 & -1/2 \\
        1 & -1/2 & 1/2
    \end{bmatrix} \begin{bmatrix}
        1 & 2 \\ 1 & -1 \\ 1 & 0
    \end{bmatrix} \\
    &= \begin{bmatrix}
        -1 & 1/2 \\ 1 & 5/2
    \end{bmatrix}.
\end{align*}

\qed



\item Sea $A$ la matriz  del ejercicio \ref{autovalores}\ref{autovalores-1} del práctico \ref{practico-5} y $T_A:\mathbb{R}^2\longrightarrow\mathbb{R}^2$ la transformación lineal dada por $T_A(v)=Av$. Hallar los autovalores de $T_A$, y para cada uno de ellos, dar una base de autovectores del correspondiente autoespacio. Decidir si $T_A$ es o no diagonalizable. En caso de serlo dar una matriz invertible $P$ tal que $P^{-1}AP$ es diagonal. 

Repetir esto para cada una de las matrices de dicho ejercicio.

\rta Resolvemos el ejercicio para cada una de las matrices del ejercicio  \ref{autovalores} del práctico \ref{practico-5}. Recordemos que para que una transformación lineal $T$ sea diagonalizable, es necesario y suficiente que exista una base de autovectores. En  caso de existir, la matriz de $T$ en dicha base, digamos $\mathcal{B}$,  es diagonal y los valores de la diagonal son los autovalores de $T$. Más aún, si $P$ es la matriz de cambio de base de la base canónica a $\mathcal{B}$, entonces $P^{-1}AP$ es la matriz diagonal de $T$ en la base canónica. 

    \vskip .2cm

\ref{autovalores-1}
    Denominamos $A$  a la matriz del enunciado \ref{autovalores-1} del práctico \ref{practico-5}. entonces,
    $$
    T_A(x,y)=A\begin{bmatrix} x\\ y \end{bmatrix}= \begin{bmatrix} 2 & -1\\ 1 & 4 \end{bmatrix} \begin{bmatrix} x\\ y \end{bmatrix} = \begin{bmatrix} 2x-y\\ x+4y \end{bmatrix}.
    $$ 
    Como ya fue calculado hay un solo autovalor $\lambda = 3$ y el autoespacio asociado es $\{(t,t): t\in\mathbb{R}\}$. Por lo tanto, $T_A$ no es diagonalizable. 

    
    \vskip .2cm
    \ref{autovalores-2} Denominamos $B$  a la matriz del enunciado. Luego, 
    $$
    T_B(x,y) = \begin{bmatrix} 1&0\\ 1&-2 \end{bmatrix}\begin{bmatrix} x\\ y \end{bmatrix} = \begin{bmatrix} x\\ x-2y \end{bmatrix}.
    $$
    Los autovalores son  $\lambda_1=-2$ y $\lambda_2=1$ y los autoespacios son  $V_{-2}=\{(0,t): t\in\mathbb{R}\}$ y $V_1 = \{(3t,t): t\in\mathbb{R}\}$. Por lo tanto una base  de autovectores es $\mathcal{B} = \{(0,1),(3,1)\}$ y $T_B$ es diagonalizable. La matriz de cambio de base de la base canónica a $\mathcal{B}$ se calcula resolviendo el sistema
    \begin{align*}
        (1,0) &= a_{11} \cdot (0,1) + a_{21} \cdot (3,1) \\
        (0,1) &= a_{12} \cdot (0,1) + a_{22} \cdot (3,1),
    \end{align*}
    el cual se soluciona fácilmente: $a_{11} = -1/3$, $a_{21} = 1/3$, $a_{12} = 1$, $a_{22} = 0$. Por lo tanto,
    $$
    P = \begin{bmatrix}
        -1/3 & 1 \\ 1/3 & 0
    \end{bmatrix}.
    $$
    

    \vskip .2cm
    \ref{autovalores-3}
    Denominamos $C$  a la matriz del enunciado. Por lo tanto
    $$
    T_C(x,y,z) = \begin{bmatrix} 2&0&0\\ -1&1&-1\\ 0&0&2 \end{bmatrix}\begin{bmatrix} x\\ y \\ z\end{bmatrix} = \begin{bmatrix} 2x\\ -x+y-z \\ 2z \end{bmatrix}.
    $$
    Los autovalores son  $\lambda_1=2$ y $\lambda_2=1$  y los autoespacios son $V_2=\{(-t,t,s): t,s\in\mathbb{R}\}$ y $V_1 =\{(0,t,0): t\in\mathbb{R}\}$, por lo tanto, podemos elegir como base $\mathcal{B} = \{(-1,1,0),(0,0,1),(0,1,0)\}$ y $T_C$ es diagonalizable. La matriz de cambio de base de la base canónica a $\mathcal{B}$ se calcula resolviendo el sistema
    \begin{align*}
        (1,0,0) &= a_{11} \cdot (-1,1,0) + a_{21} \cdot (0,0,1) + a_{31} \cdot (0,1,0) \\
        (0,1,0) &= a_{12} \cdot (-1,1,0) + a_{22} \cdot (0,0,1) + a_{32} \cdot (0,1,0) \\
        (0,0,1) &= a_{13} \cdot (-1,1,0) + a_{23} \cdot (0,0,1) + a_{33} \cdot (0,1,0).
    \end{align*}
    Las soluciones son, $a_{11} = -1$, $a_{21} = 0$, $a_{31} = 1$, $a_{12} = 0$, $a_{22} = 0$, $a_{32} = 1$, $a_{13} = 0$, $a_{23} = 1$, $a_{33} = 0$. Por lo tanto,
    $$
    P = \begin{bmatrix}
        -1 & 0 & 0 \\ 0 & 0 & 1 \\ 1 & 1 & 0
    \end{bmatrix}.
    $$
    

    \vskip .2cm
    \ref{autovalores-4} Denominamos $D$  a la matriz del enunciado. Entonces, 
    $$
    T_D(x,y,z) = \begin{bmatrix} -1&0&0\\ 0&3&-5\\ 0&1&-1 \end{bmatrix}\begin{bmatrix} x\\ y \\ z\end{bmatrix} = \begin{bmatrix} -x\\ 3y-5z \\ y-z \end{bmatrix}.
    $$
    El  único autovalor real de $T_D$ es $-1$ y el autoespacio asociado es $\{(t,0,0): t\in\mathbb{R}\}$, por lo tanto el operador no es diagonalizable.


    \vskip .2cm
    \ref{autovalores-5} Denominamos $E$  a la matriz del enunciado. Entonces,
    $$
    T_E(x,y,z) = \begin{bmatrix} \lambda&0&0\\ 1&\lambda&0\\ 0&1&\lambda \end{bmatrix}\begin{bmatrix} x\\ y \\ z\end{bmatrix} = \begin{bmatrix} \lambda x\\ x+\lambda y \\ y+\lambda z \end{bmatrix}.
    $$
    Hay un único autovalor,  que es $\lambda$ y $V_\lambda = \{(0,0,t): t\in\mathbb{R}\}$, por lo tanto el operador no es diagonalizable.
    

    \vskip .2cm
    \ref{autovalores-6} Denominamos $F$  a la matriz del enunciado. Entonces
    $$
    T_F(x,y) = \begin{bmatrix}\cos\theta & \sen\theta\\ -\sen\theta & \cos\theta \end{bmatrix}\begin{bmatrix} x\\ y\end{bmatrix} = \begin{bmatrix} x\cos\theta + y\sen\theta\\ -x\sen\theta + y\cos\theta \end{bmatrix}.
    $$ 

    Como vimos cuando resolvimos el ejercicio \ref{autovalores-6} del práctico \ref{practico-5}, hay tres casos para analizar según el valor de $\theta$.

    \vskip .2cm

    \textit{i}) Cuando $\theta=0$ tenemos que $\lambda_1 = \lambda_2 =1$. En  ese caso, $T_F$ es la identidad y por lo tanto hay un único autovalor, $1$, y  el autoespacio correspondiente es todo $\R^2$. Por lo tanto, $T_F$ es diagonalizable y podemos tomar como base la base canónica, por lo tanto $P = \Id$.
    
    \vskip .2cm
    \textit{ii}) Cuando $\theta= \pi$, tenemos que $\lambda_1 = \lambda_2 =-1$. En  este caso $T_F = -\Id$  y hay un solo autovalor, $-1$, y  el autoespacio correspondiente es todo $\R^2$.  Por lo tanto, $T_F$ es diagonalizable y podemos tomar como base la base canónica, por lo tanto $P = \Id$.
    
    \vskip .2cm
    \textit{iii}) Cuando $\theta\ne 0, \pi$ y  en este caso  no hay autovalores reales. Por lo tanto, $T_F$ no es diagonalizable.

\qed



\item Repetir el ejercicio anterior para cada matriz del ejercicio \ref{autovalores} del práctico \ref{practico-5} pero ahora consideradando a la transformación como una transformación lineal entre los $\mathbb{C}$-espacios vectoriales $\mathbb{C}^n$.

\rta Los incisos \ref{autovalores-1}, \ref{autovalores-2}, \ref{autovalores-3} y \ref{autovalores-5} tienen la misma respuesta que en el ejercicio anterior, pues los polinomios característicos no tienen raíces complejas. Para resolver los casos \ref{autovalores-4} y \ref{autovalores-6} usamos los resultados del ejercicio \ref{autovalores-complejos} del práctico \ref{practico-5}.

\vskip .3cm

\ref{autovalores-4}  Los autovalores de $T_D$ son  $\lambda_1 = 1+i$, $\lambda_2 = 1-i$ y los autoespacios son $V_{\lambda_1} =\{((2+i)t,t): t \in \C\}$, $V_{\lambda_2} = \{((2-i)t,t): t \in \C\}$. Por lo tanto, $T_D$ es diagonalizable. Una base de autovectores es $\mathcal{B} = \{(2+i,1),(2-i,1)\}$. La matriz de cambio de base de la base canónica a $\mathcal{B}$ se calcula resolviendo el sistema
\begin{align*}
    (1,0) &= a_{11} \cdot (2+i,1) + a_{21} \cdot (2-i,1) \\
    (0,1) &= a_{12} \cdot (2+i,1) + a_{22} \cdot (2-i,1),
\end{align*}
Por lo tanto $a_{11}(2+i) +a_{21}(2-i) = 1$, $a_{11} +a_{21} = 0$. Resolviendo este sistema obtenemos $a_{11} = -i/2$, $a_{21} = i/2$. De forma análoga, $a_{12} + a_{22} = 1$, $a_{12}(2+i) +a_{22}(2-i) = 0$ y resolviendo este sistema obtenemos $a_{12} = i/2$, $a_{22} = -i/2$. Por lo tanto,  
$$
P = \begin{bmatrix}
    -i/2 & i/2 \\ i/2 & -i/2
\end{bmatrix}.
$$


\vskip .2cm
\ref{autovalores-6} Los casos  $\theta=0, \pi$ ya se hicieron en el ejercicio  anterior. Cuando $\theta \ne 0, \pi$ los autovalores de $T_F$ son  $\lambda_1 =\cos\theta + i\sen\theta$ y $\lambda_2 =\cos\theta - i\sen\theta$ y los autoespacios son  $V_{\lambda_1} = \{t(-i,1): t \in \C\}$, $V_{\lambda_2} = \{t(i,1): t \in \C\}$. Por lo tanto, $T_F$ es diagonalizable. Una base de autovectores es $\mathcal{B} = \{(-i,1),(i,1)\}$. La matriz de cambio de base de la base canónica a $\mathcal{B}$ se calcula resolviendo el sistema
\begin{align*}
    (1,0) &= a_{11} \cdot (-i,1) + a_{21} \cdot (i,1) \\
    (0,1) &= a_{12} \cdot (-i,1) + a_{22} \cdot (i,1),
\end{align*}
cuyas  soluciones son $a_{11} = i/2$, $a_{21} = -i/2$, $a_{12} = 1/2$, $a_{22} = 1/2$. Por lo tanto,
$$
P = \begin{bmatrix}
    i/2 & 1/2 \\ -i/2 & 1/2
\end{bmatrix}.
$$


\qed



\item Sea $T:V\longrightarrow V$ una transformación lineal y $v\in V$ un autovector de autovalor $\lambda$. Probar las siguientes afirmaciones.
\begin{enumerate}
    \item\label{autovalor-autovector-a} Si $\lambda=0$, entonces $v\in\operatorname{Nu}(T)$.
    \item\label{autovalor-autovector-b} Si $\lambda\neq0$, entonces $v\in\operatorname{Im}(T)$.
    \item\label{autovalor-autovector-c} Si $T^2=0$, entonces $T-\operatorname{Id}$ es un isomorfismo.
\end{enumerate}

\rta

\ref{autovalor-autovector-a} Como $v$ es autovector de autovalor $0$, entonces $T(v) = 0 \cdot v =0$. Por lo tanto, $v \in \operatorname{Nu}(T)$.

\vskip .3cm

\ref{autovalor-autovector-b} Como $v$ es autovector de autovalor $\lambda$, entonces $T(v) = \lambda v$. Como $\lambda \ne 0$, podemos dividir por $\lambda$ y en consecuencia $T(v/\lambda) = \lambda v/\lambda = v$.  Por lo tanto, $v \in \operatorname{Im}(T)$.

\vskip .3cm

\ref{autovalor-autovector-c} Observemos que 
$$
(T- \operatorname{Id})(T+\operatorname{Id}) = T^2 - \operatorname{Id} \circ T + T \circ \operatorname{Id} - \operatorname{Id}^2  = T^2 - \operatorname{Id}^2 = - \operatorname{Id}.
$$
Por lo tanto, $(T- \operatorname{Id})(-T-\operatorname{Id}) = \operatorname{Id}$, análogamente $(-T-\operatorname{Id})(T- \operatorname{Id}) = \operatorname{Id}$. Por lo tanto, $T- \operatorname{Id}$ es invertible y su inversa es $-T-\operatorname{Id}$.
\qed


\item\label{base nilp}  Sea $V$ un espacio vectorial de dimensión $3$ y $T:V\longrightarrow V$ una transformación lineal. Supongamos que existe $v\in V$ tal que $T^3(v)=0$ pero $T^2(v)\neq0$.
\begin{enumerate}
    \item\label{base nilp a}  Probar que $\mathcal{B}=\{v,T(v),T^2(v)\}$ es una base de $V$.
    \item\label{base nilp b} Calcular la matriz de $T$ respecto de la base $\mathcal{B}$.
    \item\label{base nilp c} Calcular los autovalores de $T$ y sus correspondientes autoespacios. Decidir si $T$ es diagonalizable.
\end{enumerate}

\rta

\ref{base nilp a} Alcanza con probar que $v,T(v),T^2(v)$ son LI. Si $a,b,c \in \K$ son tales que $av + bT(v) + cT^2(v) = 0$, entonces 
\begin{equation*}
    0=T^2(av + bT(v) + cT^2(v)) = aT^2(v) + bT^3(v)+cT^4(v) , \tag{*}    
\end{equation*}
por hipótesis $T^3(v) = 0$, luego también $T^4(v)=0$ y por lo tanto la ecuación (*) $\Rightarrow$ $aT^2(v) =0$. Por hipótesis, $T^2(v) \ne 0$ y por lo tanto $a=0$, lo cual implica que $bT(v) + cT^2(v) = 0$. Aplicando $T$ a esto último,
\begin{equation*}
    0=T(bT(v) + cT^2(v)) = bT^2(v) + cT^3(v) = bT^2(v).
\end{equation*}
Como $T^2(v) \ne 0$, entonces $b=0$ y por lo tanto $cT^2(v) = 0$. Como $T^2(v) \ne 0$, entonces $c=0$. Por lo tanto, $a=b=c=0$ y $\mathcal{B}$ es LI. Como $\dim V = 3$, $\mathcal{B}$ es una base de $V$.

\vskip .3cm

\ref{base nilp b} La base es  $\mathcal{B}=\{v,T(v),T^2(v)\}$ y apliquemos $T$ a cada uno de los vectores de la base:
\begin{align*}
    T(v) &= T(v) = 0 \cdot v + 1 \cdot T(v) + 0 \cdot T^2(v) \ \\
    T(T(v)) &= T^2(v) = 0 \cdot v + 0 \cdot T(v) + 1 \cdot T^2(v) \\
    T(T^2(v)) &= T^3(v) =0 = 0 \cdot v + 0 \cdot T(v) + 0 \cdot T^2(v).
\end{align*}
Por lo tanto, la matriz de $T$ respecto de la base $\mathcal{B}$ es
\begin{equation*}
    [T]_{\mathcal{B}} = \begin{bmatrix}
        0 & 0 & 0 \\
        1 & 0 & 0 \\
        0 & 1 & 0
    \end{bmatrix}.
\end{equation*}

\vskip .3cm

\ref{base nilp c} El polinomio característico de $T$ es
\begin{equation*}
    p_T(\lambda) = \det \begin{bmatrix}
        \lambda & 0 & 0 \\
        -1 & \lambda & 0 \\
        0 & -1 & \lambda
    \end{bmatrix} = \lambda^3.
\end{equation*}
En  consecuencia, el único autovalor de $T$ es $0$ y el autoespacio asociado es $V_0 = \{v \in V: T(v) = 0\} = \operatorname{Nu}(T)$. Ahora bien, $\operatorname{Im}(T) = \langle T(v), T^2(v), T^3(v) \rangle = \langle T(v), T^2(v) \rangle$ y como $T(v), T^2(v)$ son LI entonces $\dim \operatorname{Im}(T) = 2$. Por el teorema de la dimensión sabemos que $\dim \operatorname{Nu}(T) + \dim \operatorname{Im}(T) = \dim V = 3$, por lo tanto $\dim \operatorname{Nu}(T) = 1$. Como $\dim \operatorname{Nu}(T) = 1$ y $\operatorname{Nu}(T) = V_0$, se concluye que $T$ no es diagonalizable.



\qed




\end{enumerate}


