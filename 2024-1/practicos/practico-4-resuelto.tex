
\chapter{Soluciones\\Álgebra  II -- Año 2024/1 -- FAMAF}\label{practico-4}

\begin{enumerate}[topsep=6pt,itemsep=.4cm]
    \item Calcular el determinante de las siguientes matrices.
        \begin{align*}
        &A=\begin{bmatrix} 4&7\\ 5&3\end{bmatrix},
        &&B=\begin{bmatrix} -3&2&4\\ 1&-1&2\\ -1&4&0\end{bmatrix},
        &&
        C=\begin{bmatrix} 2&3&1&1\\ 0&2&-1&3 \\ 0&5&1&1 \\1&1&2&5\end{bmatrix}.
        \end{align*}
        \rta

        \qed
    
    \item Sean
            $$A=
        \begin{bmatrix}
            1&3&2 \\
            3&0&2 \\
            1&1&1
        \end{bmatrix}, \qquad
        B =
        \begin{bmatrix}
            1&-1&2\\
            1&1&1 \\
            -1&-1&3
        \end{bmatrix}.
        $$
        Calcular:
        \begin{multicols}{3}
        \begin{enumerate}
            \item $\det(AB)$.
            \item $\det(BA)$.
            \item $\det(A^{-1})$.
            \item $\det(A^{4})$.
            \item $\det(A+B)$.
            \item $\det(A+tB)$, con $t \in \mathbb{R}$.
        \end{enumerate}
    \end{multicols}
    \rta

    \qed
    
    \item Reduciendo a matrices triangulares superiores, calcular el determinante de las siguientes matrices.
    
            $$A =
            \begin{bmatrix}
                a&1&1&1 \\
                1&a&1&1 \\
                1&1&a&1 \\
                1&1&1&a \\
            \end{bmatrix}, \qquad	
            B =
            \begin{bmatrix}
                1&1&1&1&1 \\
                1&3&3&3&3 \\
                1&3&5&5&5 \\
                1&3&5&7&7 \\
                1&3&5&7&9 \\
            \end{bmatrix}.
            $$
            \rta

            \qed

    \item Sean $A$, $B$ y $C$ matrices $n\times n$, tales que $\det A=-1$, $\det B=2$ y $\det C=3$.
    Calcular:
    
    \begin{enumerate}
    \item $\det(PQR)$, donde $P$, $Q$ y $R$ son las matrices que se obtienen a partir de $A$, $B$ y $C$ mediante operaciones elementales por filas de la siguiente manera
     \begin{align*}
     A\overset{F_1+2F_2}{\longrightarrow} P,\quad
     B\overset{3F_3}{\longrightarrow} Q
     \quad\mbox{y}\quad
     C\overset{F_1\leftrightarrow F_4}{\longrightarrow} R.
     \end{align*}
     Es decir,
     \begin{itemize}
      \item[$\circ$] $P$ se obtiene a partir de $A$ sumando a la fila $1$ la fila $2$ multiplicada por $2$.
      \item[$\circ$] $Q$ se obtiene a partir de $B$ multiplicando la fila $3$ por $3$.
      \item[$\circ$] $R$ se obtiene a partir de $C$ intercambiando las filas $1$ y $4$.
     \end{itemize}
        \item $\det(A^2BC^tB^{-1})$ \ y \ $\det(B^2C^{-1}AB^{-1}C^{t})$.
    \end{enumerate}
    \rta

    \qed
    
    \item  Sea
    $$A=
    \begin{bmatrix}
        x&y&z \\
        3&0&2\\
        1&1&1
    \end{bmatrix}.$$
    Sabiendo que $\det(A) = 5$, calcular el determinante de las siguientes matrices.
    $$
    B = \begin{bmatrix}
    2x&2y&2z \\
    3/2&0&1\\
    1&1&1
    \end{bmatrix}, \qquad
    C=
    \begin{bmatrix}
        x&y&z \\
        3x+3&3y&3z+2\\
        x+1&y+1&z+1
    \end{bmatrix}.
    $$
    \rta

    \qed
    
    \item Determinar todos los valores de $c\in\mathbb{R}$ tales que las siguientes matrices sean invertibles.
    \begin{align*}
    A=\begin{bmatrix}4& c&3\\c&2&c\\ 5&c&4 \end{bmatrix},\qquad
    B=\begin{bmatrix} 1&c&-1\\ c&1&1\\0&1&c\end{bmatrix}.
    \end{align*}
    \rta

    \qed
    
    \item Calcular el determinante de las siguientes matrices, usando operaciones elementales por fila y/o columnas u otras propiedades del determinante. Determinar cuáles de ellas son invertibles.
    
    \begin{align*}
    &A=
    \begin{bmatrix}
    -2&3&2&-6\\ 0&4&4&-5\\ 5&-6&-3&2\\ -3&7&0&0 \end{bmatrix},\quad
    &&B=\begin{bmatrix} 2&0&0&0\\ 0&0&3&0\\ 0&-1&0&0\\ 0&0&0&4\end{bmatrix},\quad
    &&
    C=\begin{bmatrix}
      -2&3&2&-6&0\\
    0&4&4&-5&0\\
    5&-6&-3&2&0\\
    -3&7&0&0&0\\
    1&1&1&1&1
      \end{bmatrix},
    \end{align*}
    \begin{align*}
    D=\begin{bmatrix}
    1&2&3&0&0\\
    -1&2&-13&6&\frac{1}{3}\\
    2&0&0&0&0\\
    11&1&0&0&0\\
    \sqrt{2}&2&1&\pi&0\\
    \end{bmatrix},&&
    E=\begin{bmatrix}
    1&-1&2&0&0\\ 3&1&4&0&0\\ 2&-1&5&0&0 \\0&0&0&2&1\\ 0&0&0&-1&4
    \end{bmatrix}.
    \end{align*}
    \rta

    \qed
    
    \item Sean $A$ y  $B$ matrices $n \times n$. Probar que:
    \begin{enumerate}
        \item $\det(AB) = \det (BA)$.
        \item Si $B$ es invertible, entonces $\det(B A B^{-1}) = \det (A)$.
        \item\label{-A} $\textcircled{a}$ $\det(-A) = (-1)^n\det (A)$.
    \end{enumerate}
    \rta

    \qed
    
    \item\label{vandermonde} Sean $\lambda_1, \lambda_2, \dots, \lambda_n$ escalares, la matriz de \emph{Vandermonde} asociada es
    \begin{align*}
    \mathtt V = \begin{bmatrix}
    1 & \lambda_1 & \lambda_1^2 & \cdots & \lambda_1^{n-1}\\
    1 & \lambda_2 & \lambda_2^2 & \cdots & \lambda_2^{n-1}\\
    \vdots &\vdots &\vdots & &\vdots\\
    1 & \lambda_n & \lambda_n^2 & \cdots & \lambda_n^{n-1}\\
    \end{bmatrix}.
    \end{align*}
    Esta es la matriz del sistema de ecuaciones del ejercicio \ref{polinomios}\,\ref{polinomios-c} del Práctico \ref{practico-2}.
    
    
    \begin{enumerate}
        \item Si $n=2$, probar que $\det(\mathtt V) = \lambda_2-\lambda_1$.
        \item Si $n=3$, probar que $\det(\mathtt V) = (\lambda_3-\lambda_2) (\lambda_3-\lambda_1) (\lambda_2-\lambda_1)$.
        \item\label{vandermonde gral} $\textcircled{a}$ Probar que $\det(\mathtt V) = \prod_{1\leq i< j \leq n}(\lambda_j-\lambda_i)$ para todo $n\in\mathbb{N}$.
        \item Dar una condición necesaria y suficiente para que la matriz de Vandermonde sea invertible.
        \item Usar lo anterior para responder a la pregunta del ejercicio \ref{polinomios}\,\ref{polinomios-b} del Práctico \ref{practico-2}.
        \end{enumerate}
    \rta

    \qed
        
    \item Decidir si las siguientes afirmaciones son verdaderas o falsas. Justificar con una demostración o con un contraejemplo, según corresponda.
        \begin{enumerate}
        \item Sean $A$ y $B$ matrices $n \times n$. Entonces $\det(A + B) = \det (A) + \det(B)$.
        \item Existen una matriz $3\times 2$, $A$, y una matriz $2\times 3$, $B$, tales que $\det(AB) \neq 0$.
        \item Sea $A$ una matriz $n\times n$. Si $A^n$ es no invertible, entonces $A$ es no invertible.
    \end{enumerate}
    \rta

    \qed
    
    \end{enumerate}
    