
\chapter{Soluciones\\Álgebra  II -- Año 2024/1 -- FAMAF}\label{practico-4}

\begin{enumerate}[topsep=6pt,itemsep=.4cm]
    \item Calcular el determinante de las siguientes matrices.
        \begin{align*}
        &A=\begin{bmatrix} 4&7\\ 5&3\end{bmatrix},
        &&B=\begin{bmatrix} -3&2&4\\ 1&-1&2\\ -1&4&0\end{bmatrix},
        &&
        C=\begin{bmatrix} 2&3&1&1\\ 0&2&-1&3 \\ 0&5&1&1 \\1&1&2&5\end{bmatrix}.
        \end{align*}
        
        \rta
        \begin{align*}
            \left| A \right| &= 4\cdot 3 - 7\cdot 5 = -23,\\
            \left| B \right| &= -3\cdot \left|\begin{matrix} -1&2\\ 4&0\end{matrix}\right| - 1\cdot \left|\begin{matrix} 2&4\\ 4&0\end{matrix}\right| - 1\cdot \left|\begin{matrix} 2&4\\ -1&2\end{matrix} \right|\\
            &= -3\cdot (-8) - 1\cdot (-16) - 1\cdot 8\\ 
            &= 24 + 16 -8 = 32,\\
            \left| C \right| &= 2 \cdot\left| \begin{matrix} 2&-1&3\\ 5&1&1\\ 1&2&5\end{matrix} \right| - 0 \cdot \left| C(2|1) \right| + 0 \cdot \left| C(3|1) \right| -1 \cdot\left|  \begin{matrix} 3&1&1\\ 2&-1&3 \\ 5&1&1\end{matrix} \right| \tag{*}
        \end{align*}
        Debemos ahora calcular el determinante de las matrices $3 \times 3$ que aparecen en la expresión de $|C|$. 
        \begin{align*}
            \left|\begin{matrix} 2&-1&3\\ 5&1&1\\ 1&2&5\end{matrix} \right| &= 2\cdot  \left|\begin{matrix} 1&1\\ 2&5\end{matrix} \right| - 5\cdot  \left|\begin{matrix} -1&3\\ 2&5\end{matrix} \right| + 1\cdot  \left|\begin{matrix} -1&3\\ 1&1\end{matrix} \right| \\
            &= 2\cdot 3  - 5\cdot (-11) + 1\cdot (-4) = 57,\\
            \left|  \begin{matrix} 3&1&1\\ 2&-1&3 \\ 5&1&1\end{matrix} \right| &= 3\cdot  \left| \begin{matrix} -1&3\\ 1&1\end{matrix} \right| - 2\cdot  \left|  \begin{matrix} 1&1\\ 1&1\end{matrix} \right| + 5 \cdot  \left|  \begin{matrix} 1&1\\ -1&3\end{matrix} \right| \\
            &= 3\cdot (-4) - 2\cdot 0 + 5\cdot 4 = -12 + 20 = 8.
        \end{align*}
        Luego, por (*):
        \begin{align*}
            \left| C \right| &= 2 \cdot 57 -1 \cdot 8 = 106.
        \end{align*} 
        \qed
    
    \item Sean
            $$A=
        \begin{bmatrix} 1&3&2 \\ 3&0&2 \\  1&1&1 \end{bmatrix}, \qquad
        B =
        \begin{bmatrix} 1&-1&2\\ 1&1&1 \\ -1&-1&3 \end{bmatrix}.
        $$
        Calcular:
        \begin{multicols}{3}
        \begin{enumerate}
            \item\label{det-AB} $\det(AB)$.
            \item\label{det-BA} $\det(BA)$.
            \item\label{det-A-1} $\det(A^{-1})$.
            \item\label{det-A4} $\det(A^{4})$.
            \item\label{det-A+B} $\det(A+B)$.
            \item\label{det-A+tB} $\det(A+tB)$, con $t \in \mathbb{R}$.
        \end{enumerate}
    \end{multicols}
    \rta

    Primero nos conviene calcular los determinantes de $A$ y $B$, pues algunos cálculos se reducen a saber estos números. 
    \begin{align*}
        \det(A) =\left|\begin{matrix} 1&3&2 \\ 3&0&2 \\  1&1&1 \end{matrix}\right| &= 1\cdot  \left|\begin{matrix} 0&2 \\  1&1\end{matrix} \right| - 3\cdot  \left|\begin{matrix}3&2 \\ 1&1\end{matrix} \right| + 1\cdot  \left|\begin{matrix} 3&2 \\ 0&2\end{matrix} \right| \\
        &= 1\cdot (-2)  - 3\cdot 1 + 1\cdot 6 = 1,\\
        \det(B) =\left|  \begin{matrix}  1&-1&2\\ 1&1&1 \\ -1&-1&3 \end{matrix} \right| &= 1\cdot  \left| \begin{matrix}  1&1 \\ -1&3 \end{matrix} \right| - 1\cdot  \left|  \begin{matrix} -1&2 \\ -1&3\end{matrix} \right| -1  \cdot  \left|  \begin{matrix} -1&2\\ 1&1\end{matrix} \right| \\
        &= 1\cdot 4 - 1\cdot (- 1)  -1\cdot (-3) = 8.
    \end{align*}

    Resumiendo $\det(A) = 1$ y $\det(B) = 8$. Ahora sí, calculemos los determinantes pedidos.
    
    \vskip .2cm
    \ref{det-AB} $\det(AB) = \det(A)\det(B) = 1 \cdot 8 = 8$.

    \vskip .2cm
    \ref{det-BA} $\det(BA) = \det(B)\det(A) = 8 \cdot 1 = 8$.

    \vskip .2cm
    \ref{det-A-1} $\det(A^{-1}) = 1/ \det(A) = 1/1 = 1$.
    
    \vskip .2cm
    \ref{det-A4} $\det(A^{4}) = \det(A)^{4} = 1^4 = 1$.
    
    \vskip .2cm
    \ref{det-A+tB} El ejercicio \ref{det-A+B} es un caso especial de \ref{det-A+tB} para $t=1$.  Así que haremos este inciso primero. Para ello, antes que nada, calculemos la matriz $A + tB$:
    \begin{align*}
        A+ tB &= \begin{bmatrix} 1&3&2 \\ 3&0&2 \\  1&1&1 \end{bmatrix} + t \begin{bmatrix} 1&-1&2\\ 1&1&1 \\ -1&-1&3 \end{bmatrix} \\
        &= \begin{bmatrix} 1+t&3-t&2+2t \\ 3+t&t&2+t \\  1-t&1-t&1+3t \end{bmatrix}.
    \end{align*}
    Entonces
    \begin{align*}
       \left| \begin{matrix} 1+t&3-t&2+2t \\ 3+t&t&2+t \\  1-t&1-t&1+3t \end{matrix}\right| & = (1+t)\cdot  \left| \begin{matrix} t&2+t \\  1-t&1+3t \end{matrix} \right| - (3+t)\cdot  \left| \begin{matrix} 3-t&2+2t \\  1-t&1+3t \end{matrix} \right|\\
       &\qquad\qquad\qquad\qquad\qquad\qquad + (1-t)\cdot  \left| \begin{matrix} 3-t&2+2t \\ t&2+t  \end{matrix} \right| \\
       &= (1+t)\cdot (t(1+3t) - (2+t)(1-t)) -\\
       &\quad- (3+t)\cdot ((3-t)(1+3t) - (2+2t)(1-t)) +\\
       &\quad + (1-t)\cdot ((3-t)(2+t) - (2+2t)t)\\
         &= (1+t)\cdot (4 t^2 + 2 t - 2) +\\
         &\quad- (3+t)\cdot (-t^2 + 8 t + 1) +\\
            &\quad + (1-t)\cdot (-3 t^2 - t + 6)
            \\
            &= (4 t^3 + 6 t^2 - 2) + (t^3 - 5 t^2 - 25 t - 3) + (3 t^3 - 2 t^2 - 7 t + 6
            )\\
            &=8 t^3 - t^2 - 32 t + 1
    \end{align*}

    \vskip .2cm
    \ref{det-A+B} $\det(A+B) = \det(a + 1\cdot B) \stackrel{\ref{det-A+tB}}{=}8 \cdot 1^3 - 1^2 - 32 \cdot 1 + 1 = -24.$


    \qed
    
    \item Calcular el determinante de las siguientes matrices haciendo la reducción a matrices triangulares superiores.
    
            $$A =
            \begin{bmatrix}
                a&1&1&1 \\
                1&a&1&1 \\
                1&1&a&1 \\
                1&1&1&a \\
            \end{bmatrix}, \qquad    
            B =
            \begin{bmatrix}
                1&1&1&1&1 \\
                1&3&3&3&3 \\
                1&3&5&5&5 \\
                1&3&5&7&7 \\
                1&3&5&7&9 \\
            \end{bmatrix}.
            $$
            \rta Recordemos que si $A$ es una matriz triangular superior, entonces $\det(A)$ es el producto de los elementos de la diagonal principal. Por otro  lado, las operaciones elementales por fila  tienen el siguiente efecto en el cálculo del determinante:
            \begin{itemize}
                \item el multiplicar una fila por un escalar $\lambda$ cambia el determinante por $\lambda$ veces el determinante anterior,
                \item intercambiar dos filas cambia el signo del determinante, y
                \item sumar a una fila un múltiplo de otra fila no cambia el determinante.
            \end{itemize}

            Es claro la matriz original tiene todas las filas iguales y por lo tanto $\det(A) = 0$. 
            
            Analicemos el caso  $a\ne 1$.

            Veamos como reducimos $A$ a triangular superior:
            \begin{align*}
                A \underset{F_4 -F_2}{\underset{F_3 -F_2}{\stackrel{F_1-aF_2}{\longrightarrow}}} &=  \begin{bmatrix}
                    0&1-a^2&1-a&1-a \\ 
                    1&a&1&1 \\
                    0&1-a&a-1&0 \\
                    0&1-a&0&a-1 \\
                \end{bmatrix}
                \underset{F_4 / (1-a)}{\underset{F_3 / (1-a)}{\stackrel{F_1 /(1-a)}{\longrightarrow}}} 
                \begin{bmatrix}
                    0&1+a&1&1 \\ 
                    1&a&1&1 \\
                    0&1&-1&0 \\
                    0&1&0&-1 \\
                \end{bmatrix} \\
                &\underset{F_3 -F_4}{\underset{F_2 -aF_4}{\stackrel{F_1-(1+a)F_4}{\longrightarrow}}}
                \begin{bmatrix}
                    0&0&1&2+a \\ 
                    1&0&1&1+a \\
                    0&0&-1&1 \\
                    0&1&0&-1 \\
                \end{bmatrix} 
                \stackrel{(-1)F_3}{\longrightarrow}
                \begin{bmatrix}
                    0&0&1&2+a \\ 
                    1&0&1&1+a \\
                    0&0&1&-1 \\
                    0&1&0&-1 \\
                \end{bmatrix} 
            \end{align*}
            
           






            \qed

    \item Sean $A$, $B$ y $C$ matrices $n\times n$, tales que $\det A=-1$, $\det B=2$ y $\det C=3$.
    Calcular:
    
    \begin{enumerate}
    \item $\det(PQR)$, donde $P$, $Q$ y $R$ son las matrices que se obtienen a partir de $A$, $B$ y $C$ mediante operaciones elementales por filas de la siguiente manera
     \begin{align*}
     A\overset{F_1+2F_2}{\longrightarrow} P,\quad
     B\overset{3F_3}{\longrightarrow} Q
     \quad\mbox{y}\quad
     C\overset{F_1\leftrightarrow F_4}{\longrightarrow} R.
     \end{align*}
     Es decir,
     \begin{itemize}
      \item[$\circ$] $P$ se obtiene a partir de $A$ sumando a la fila $1$ la fila $2$ multiplicada por $2$.
      \item[$\circ$] $Q$ se obtiene a partir de $B$ multiplicando la fila $3$ por $3$.
      \item[$\circ$] $R$ se obtiene a partir de $C$ intercambiando las filas $1$ y $4$.
     \end{itemize}
        \item $\det(A^2BC^tB^{-1})$ \ y \ $\det(B^2C^{-1}AB^{-1}C^{t})$.
    \end{enumerate}
    \rta

    \qed
    
    \item  Sea
    $$A=
    \begin{bmatrix}
        x&y&z \\
        3&0&2\\
        1&1&1
    \end{bmatrix}.$$
    Sabiendo que $\det(A) = 5$, calcular el determinante de las siguientes matrices.
    $$
    B = \begin{bmatrix}
    2x&2y&2z \\
    3/2&0&1\\
    1&1&1
    \end{bmatrix}, \qquad
    C=
    \begin{bmatrix}
        x&y&z \\
        3x+3&3y&3z+2\\
        x+1&y+1&z+1
    \end{bmatrix}.
    $$
    \rta

    \qed
    
    \item Determinar todos los valores de $c\in\mathbb{R}$ tales que las siguientes matrices sean invertibles.
    \begin{align*}
    A=\begin{bmatrix}4& c&3\\c&2&c\\ 5&c&4 \end{bmatrix},\qquad
    B=\begin{bmatrix} 1&c&-1\\ c&1&1\\0&1&c\end{bmatrix}.
    \end{align*}
    \rta

    \qed
    
    \item Calcular el determinante de las siguientes matrices, usando operaciones elementales por fila y/o columnas u otras propiedades del determinante. Determinar cuáles de ellas son invertibles.
    
    \begin{align*}
    &A=
    \begin{bmatrix}
    -2&3&2&-6\\ 0&4&4&-5\\ 5&-6&-3&2\\ -3&7&0&0 \end{bmatrix},\quad
    &&B=\begin{bmatrix} 2&0&0&0\\ 0&0&3&0\\ 0&-1&0&0\\ 0&0&0&4\end{bmatrix},\quad
    &&
    C=\begin{bmatrix}
      -2&3&2&-6&0\\
    0&4&4&-5&0\\
    5&-6&-3&2&0\\
    -3&7&0&0&0\\
    1&1&1&1&1
      \end{bmatrix},
    \end{align*}
    \begin{align*}
    D=\begin{bmatrix}
    1&2&3&0&0\\
    -1&2&-13&6&\frac{1}{3}\\
    2&0&0&0&0\\
    11&1&0&0&0\\
    \sqrt{2}&2&1&\pi&0\\
    \end{bmatrix},&&
    E=\begin{bmatrix}
    1&-1&2&0&0\\ 3&1&4&0&0\\ 2&-1&5&0&0 \\0&0&0&2&1\\ 0&0&0&-1&4
    \end{bmatrix}.
    \end{align*}
    \rta

    \qed
    
    \item Sean $A$ y  $B$ matrices $n \times n$. Probar que:
    \begin{enumerate}
        \item $\det(AB) = \det (BA)$.
        \item Si $B$ es invertible, entonces $\det(B A B^{-1}) = \det (A)$.
        \item\label{-A} $\textcircled{a}$ $\det(-A) = (-1)^n\det (A)$.
    \end{enumerate}
    \rta

    \qed
    
    \item\label{vandermonde} Sean $\lambda_1, \lambda_2, \dots, \lambda_n$ escalares, la matriz de \emph{Vandermonde} asociada es
    \begin{align*}
    \mathtt V = \begin{bmatrix}
    1 & \lambda_1 & \lambda_1^2 & \cdots & \lambda_1^{n-1}\\
    1 & \lambda_2 & \lambda_2^2 & \cdots & \lambda_2^{n-1}\\
    \vdots &\vdots &\vdots & &\vdots\\
    1 & \lambda_n & \lambda_n^2 & \cdots & \lambda_n^{n-1}\\
    \end{bmatrix}.
    \end{align*}
    Esta es la matriz del sistema de ecuaciones del ejercicio \ref{polinomios}\,\ref{polinomios-c} del Práctico \ref{practico-2}.
    
    
    \begin{enumerate}
        \item Si $n=2$, probar que $\det(\mathtt V) = \lambda_2-\lambda_1$.
        \item Si $n=3$, probar que $\det(\mathtt V) = (\lambda_3-\lambda_2) (\lambda_3-\lambda_1) (\lambda_2-\lambda_1)$.
        \item\label{vandermonde gral} $\textcircled{a}$ Probar que $\det(\mathtt V) = \prod_{1\leq i< j \leq n}(\lambda_j-\lambda_i)$ para todo $n\in\mathbb{N}$.
        \item Dar una condición necesaria y suficiente para que la matriz de Vandermonde sea invertible.
        \item Usar lo anterior para responder a la pregunta del ejercicio \ref{polinomios}\,\ref{polinomios-b} del Práctico \ref{practico-2}.
        \end{enumerate}
    \rta

    \qed
        
    \item Decidir si las siguientes afirmaciones son verdaderas o falsas. Justificar con una demostración o con un contraejemplo, según corresponda.
        \begin{enumerate}
        \item Sean $A$ y $B$ matrices $n \times n$. Entonces $\det(A + B) = \det (A) + \det(B)$.
        \item Existen una matriz $3\times 2$, $A$, y una matriz $2\times 3$, $B$, tales que $\det(AB) \neq 0$.
        \item Sea $A$ una matriz $n\times n$. Si $A^n$ es no invertible, entonces $A$ es no invertible.
    \end{enumerate}
    \rta

    \qed
    
    \end{enumerate}
    