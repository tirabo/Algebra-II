
\chapter{Soluciones\\Álgebra  II -- Año 2024/1 -- FAMAF}\label{practico-4}

\begin{enumerate}[topsep=6pt,itemsep=.4cm]
    \item Calcular el determinante de las siguientes matrices.
        \begin{align*}
        &A=\begin{bmatrix} 4&7\\ 5&3\end{bmatrix},
        &&B=\begin{bmatrix} -3&2&4\\ 1&-1&2\\ -1&4&0\end{bmatrix},
        &&
        C=\begin{bmatrix} 2&3&1&1\\ 0&2&-1&3 \\ 0&5&1&1 \\1&1&2&5\end{bmatrix}.
        \end{align*}
        
        \rta
        \begin{align*}
            \left| A \right| &= 4\cdot 3 - 7\cdot 5 = -23,\\
            \left| B \right| &= -3\cdot \left|\begin{matrix} -1&2\\ 4&0\end{matrix}\right| - 1\cdot \left|\begin{matrix} 2&4\\ 4&0\end{matrix}\right| - 1\cdot \left|\begin{matrix} 2&4\\ -1&2\end{matrix} \right|\\
            &= -3\cdot (-8) - 1\cdot (-16) - 1\cdot 8\\ 
            &= 24 + 16 -8 = 32,\\
            \left| C \right| &= 2 \cdot\left| \begin{matrix} 2&-1&3\\ 5&1&1\\ 1&2&5\end{matrix} \right| - 0 \cdot \left| C(2|1) \right| + 0 \cdot \left| C(3|1) \right| -1 \cdot\left|  \begin{matrix} 3&1&1\\ 2&-1&3 \\ 5&1&1\end{matrix} \right| \tag{*}
        \end{align*}
        Debemos ahora calcular el determinante de las matrices $3 \times 3$ que aparecen en la expresión de $|C|$. 
        \begin{align*}
            \left|\begin{matrix} 2&-1&3\\ 5&1&1\\ 1&2&5\end{matrix} \right| &= 2\cdot  \left|\begin{matrix} 1&1\\ 2&5\end{matrix} \right| - 5\cdot  \left|\begin{matrix} -1&3\\ 2&5\end{matrix} \right| + 1\cdot  \left|\begin{matrix} -1&3\\ 1&1\end{matrix} \right| \\
            &= 2\cdot 3  - 5\cdot (-11) + 1\cdot (-4) = 57,\\
            \left|  \begin{matrix} 3&1&1\\ 2&-1&3 \\ 5&1&1\end{matrix} \right| &= 3\cdot  \left| \begin{matrix} -1&3\\ 1&1\end{matrix} \right| - 2\cdot  \left|  \begin{matrix} 1&1\\ 1&1\end{matrix} \right| + 5 \cdot  \left|  \begin{matrix} 1&1\\ -1&3\end{matrix} \right| \\
            &= 3\cdot (-4) - 2\cdot 0 + 5\cdot 4 = -12 + 20 = 8.
        \end{align*}
        Luego, por (*):
        \begin{align*}
            \left| C \right| &= 2 \cdot 57 -1 \cdot 8 = 106.
        \end{align*} 
        \qed
    
    \item Sean
            $$A=
        \begin{bmatrix} 1&3&2 \\ 3&0&2 \\  1&1&1 \end{bmatrix}, \qquad
        B =
        \begin{bmatrix} 1&-1&2\\ 1&1&1 \\ -1&-1&3 \end{bmatrix}.
        $$
        Calcular:
        \begin{multicols}{3}
        \begin{enumerate}
            \item\label{det-AB} $\det(AB)$.
            \item\label{det-BA} $\det(BA)$.
            \item\label{det-A-1} $\det(A^{-1})$.
            \item\label{det-A4} $\det(A^{4})$.
            \item\label{det-A+B} $\det(A+B)$.
            \item\label{det-A+tB} $\det(A+tB)$, con $t \in \mathbb{R}$.
        \end{enumerate}
    \end{multicols}
    \rta

    Primero nos conviene calcular los determinantes de $A$ y $B$, pues algunos cálculos se reducen a saber estos números. 
    \begin{align*}
        \det(A) =\left|\begin{matrix} 1&3&2 \\ 3&0&2 \\  1&1&1 \end{matrix}\right| &= 1\cdot  \left|\begin{matrix} 0&2 \\  1&1\end{matrix} \right| - 3\cdot  \left|\begin{matrix}3&2 \\ 1&1\end{matrix} \right| + 1\cdot  \left|\begin{matrix} 3&2 \\ 0&2\end{matrix} \right| \\
        &= 1\cdot (-2)  - 3\cdot 1 + 1\cdot 6 = 1,\\
        \det(B) =\left|  \begin{matrix}  1&-1&2\\ 1&1&1 \\ -1&-1&3 \end{matrix} \right| &= 1\cdot  \left| \begin{matrix}  1&1 \\ -1&3 \end{matrix} \right| - 1\cdot  \left|  \begin{matrix} -1&2 \\ -1&3\end{matrix} \right| -1  \cdot  \left|  \begin{matrix} -1&2\\ 1&1\end{matrix} \right| \\
        &= 1\cdot 4 - 1\cdot (- 1)  -1\cdot (-3) = 8.
    \end{align*}

    Resumiendo $\det(A) = 1$ y $\det(B) = 8$. Ahora sí, calculemos los determinantes pedidos.
    
    \vskip .2cm
    \ref{det-AB} $\det(AB) = \det(A)\det(B) = 1 \cdot 8 = 8$.

    \vskip .2cm
    \ref{det-BA} $\det(BA) = \det(B)\det(A) = 8 \cdot 1 = 8$.

    \vskip .2cm
    \ref{det-A-1} $\det(A^{-1}) = 1/ \det(A) = 1/1 = 1$.
    
    \vskip .2cm
    \ref{det-A4} $\det(A^{4}) = \det(A)^{4} = 1^4 = 1$.
    
    \vskip .2cm
    \ref{det-A+tB} El ejercicio \ref{det-A+B} es un caso especial de \ref{det-A+tB} para $t=1$.  Así que haremos este inciso primero. Para ello, antes que nada, calculemos la matriz $A + tB$:
    \begin{align*}
        A+ tB &= \begin{bmatrix} 1&3&2 \\ 3&0&2 \\  1&1&1 \end{bmatrix} + t \begin{bmatrix} 1&-1&2\\ 1&1&1 \\ -1&-1&3 \end{bmatrix} \\
        &= \begin{bmatrix} 1+t&3-t&2+2t \\ 3+t&t&2+t \\  1-t&1-t&1+3t \end{bmatrix}.
    \end{align*}
    Entonces
    \begin{align*}
       \left| \begin{matrix} 1+t&3-t&2+2t \\ 3+t&t&2+t \\  1-t&1-t&1+3t \end{matrix}\right| & = (1+t)\cdot  \left| \begin{matrix} t&2+t \\  1-t&1+3t \end{matrix} \right| - (3+t)\cdot  \left| \begin{matrix} 3-t&2+2t \\  1-t&1+3t \end{matrix} \right|\\
       &\qquad\qquad\qquad\qquad\qquad\qquad + (1-t)\cdot  \left| \begin{matrix} 3-t&2+2t \\ t&2+t  \end{matrix} \right| \\
       &= (1+t)\cdot (t(1+3t) - (2+t)(1-t)) -\\
       &\quad- (3+t)\cdot ((3-t)(1+3t) - (2+2t)(1-t)) +\\
       &\quad + (1-t)\cdot ((3-t)(2+t) - (2+2t)t)\\
         &= (1+t)\cdot (4 t^2 + 2 t - 2) +\\
         &\quad- (3+t)\cdot (-t^2 + 8 t + 1) +\\
            &\quad + (1-t)\cdot (-3 t^2 - t + 6)
            \\
            &= (4 t^3 + 6 t^2 - 2) + (t^3 - 5 t^2 - 25 t - 3) + (3 t^3 - 2 t^2 - 7 t + 6
            )\\
            &=8 t^3 - t^2 - 32 t + 1
    \end{align*}

    \vskip .2cm
    \ref{det-A+B} $\det(A+B) = \det(a + 1\cdot B) \stackrel{\ref{det-A+tB}}{=}8 \cdot 1^3 - 1^2 - 32 \cdot 1 + 1 = -24.$


    \qed
    
    \item Calcular el determinante de las siguientes matrices haciendo la reducción a matrices triangulares superiores.
    
            $$A =
            \begin{bmatrix}
                a&1&1&1 \\
                1&a&1&1 \\
                1&1&a&1 \\
                1&1&1&a \\
            \end{bmatrix}, \qquad    
            B =
            \begin{bmatrix}
                1&1&1&1&1 \\
                1&3&3&3&3 \\
                1&3&5&5&5 \\
                1&3&5&7&7 \\
                1&3&5&7&9 \\
            \end{bmatrix}.
            $$
            \rta Recordemos que si $M$ es una matriz triangular superior, entonces $\det(M)$ es el producto de los elementos de la diagonal principal. Por otro  lado, las operaciones elementales por fila  necesarias para obtener una matriz triangular superior tienen el siguiente efecto en el cálculo del determinante:
            \begin{itemize}
                \item intercambiar dos filas cambia el signo del determinante, y
                \item sumar a una fila un múltiplo de otra fila no cambia el determinante.
            \end{itemize}
            La operación elemental de multiplicar una fila por una constante no es necesaria para conseguir una matriz triangular superior.

            \vskip .2cm
        \textbf{Cálculo de $\det(A)$.} Un  caso particular es cuando $a=0$,  en ese caso la matriz tiene todas las filas iguales y por lo tanto $\det(A) = 0$. 
            
            Analicemos el caso  $a\ne 1$.

            Veamos como reducimos $A$ a triangular superior:
            \begin{align*}
                A \underset{F_4 -F_2}{\underset{F_3 -F_2}{\stackrel{F_1-aF_2}{\longrightarrow}}} &=  \begin{bmatrix}
                    0&1-a^2&1-a&1-a \\ 
                    1&a&1&1 \\
                    0&1-a&a-1&0 \\
                    0&1-a&0&a-1 \\
                \end{bmatrix}
                {\stackrel{F_1 \leftrightarrow F_4}{\longrightarrow}} 
                \begin{bmatrix}
                    0&1-a&0&a-1 \\
                    1&a&1&1 \\
                    0&1-a&a-1&0 \\
                    0&1-a^2&1-a&1-a \\ 
                \end{bmatrix} \\
                &{\stackrel{F_1 \leftrightarrow F_2}{\longrightarrow}}
                \begin{bmatrix}
                    1&a&1&1 \\
                    0&1-a&0&a-1 \\
                    0&1-a&a-1&0 \\
                    0&1-a^2&1-a&1-a \\ 
                \end{bmatrix} 
                {\underset{F_4 -(1+a)F_2}{\stackrel{F_3-F_2}{\longrightarrow}}}
                \begin{bmatrix}
                    1&a&1&1 \\
                    0&1-a&0&a-1 \\
                    0&0&a-1&1-a \\
                    0&0&1-a&2-a-a^2 \\ 
                \end{bmatrix}  \\
                &\stackrel{F_4 + F_3}{\longrightarrow}
                \begin{bmatrix}
                    1&a&1&1 \\
                    0&1-a&0&a-1 \\
                    0&0&a-1&1-a \\
                    0&0&0&3-2a-a^2 \\ 
                \end{bmatrix} 
            \end{align*}
        Como solo hicimos dos permutaciones de filas, el determinante de  $A$ es igual al determinante de la última matriz,  es decir
        $$
        \det(A) = (1-a)(a-1)(3-2a-a^2) = a^4 - 6 a^2 + 8 a - 3.
        $$
        
        \vskip .2cm

        \textbf{Cálculo de $\det(B)$.}
        \begin{align*}
            B \underset{F_5 -F_1}{\underset{F_4 -F_1}{\underset{F_3 -F_1}{\underset{F_2 -F_1}{\longrightarrow}}}} &  \begin{bmatrix}
                1&1&1&1&1 \\
                0&2&2&2&2 \\
                0&2&4&4&4 \\
                0&2&4&6&6 \\
                0&2&4&6&8 \\
            \end{bmatrix}
            \underset{F_5 -F_2}{\underset{F_4 -F_2}{\underset{F_3 -F_2}{\longrightarrow}}}
            \begin{bmatrix}
                1&1&1&1&1 \\
                0&2&2&2&2 \\
                0&0&2&2&2 \\
                0&0&2&4&4 \\
                0&0&2&4&6 \\
            \end{bmatrix} \\
            \underset{F_5 -F_3}{\underset{F_4 -F_3}{\longrightarrow}}&
            \begin{bmatrix}
                1&1&1&1&1 \\
                0&2&2&2&2 \\
                0&0&2&2&2 \\
                0&0&0&2&2 \\
                0&0&0&2&4 \\
            \end{bmatrix}
            \underset{F_5 -F_4}{\longrightarrow}
            \begin{bmatrix}
                1&1&1&1&1 \\
                0&2&2&2&2 \\
                0&0&2&2&2 \\
                0&0&0&2&2 \\
                0&0&0&0&2 \\
            \end{bmatrix}.                
        \end{align*}
        El determinante de  $B$ es igual al determinante de la última matriz,  es decir:
        $$
        \det(B) = 2^4 = 16.
        $$






            \qed

    \item Sean $A$, $B$ y $C$ matrices $n\times n$, tales que $\det A=-1$, $\det B=2$ y $\det C=3$.
    Calcular:
    
    \begin{enumerate}
    \item\label{det pqr} $\det(PQR)$, donde $P$, $Q$ y $R$ son las matrices que se obtienen a partir de $A$, $B$ y $C$ mediante operaciones elementales por filas de la siguiente manera
     \begin{align*}
     A\overset{F_1+2F_2}{\longrightarrow} P,\quad
     B\overset{3F_3}{\longrightarrow} Q
     \quad\mbox{y}\quad
     C\overset{F_1\leftrightarrow F_4}{\longrightarrow} R.
     \end{align*}
     Es decir,
     \begin{itemize}
      \item[$\circ$] $P$ se obtiene a partir de $A$ sumando a la fila $1$ la fila $2$ multiplicada por $2$.
      \item[$\circ$] $Q$ se obtiene a partir de $B$ multiplicando la fila $3$ por $3$.
      \item[$\circ$] $R$ se obtiene a partir de $C$ intercambiando las filas $1$ y $4$.
     \end{itemize}
        \item\label{det a2bc..} $\det(A^2BC^tB^{-1})$ \ y \ $\det(B^2C^{-1}AB^{-1}C^{t})$.
    \end{enumerate}
    \rta

    \ref{det pqr} La matriz $P$ se obtiene de $A$ sumando a la fila $1$ la fila $2$ multiplicada por $2$. Esta operación elemental no cambia el determinante, luego $\det(P) = \det(A) = -1$. La matriz $Q$ se obtiene de $B$ multiplicando la fila $3$ por $3$, luego $\det(Q) = 3\cdot \det(B) = 6$. La matriz $R$ se obtiene de $C$ intercambiando las filas $1$ y $4$, luego $\det(R) = -\det(C) = -3$. 

    \vskip .2cm
    \ref{det a2bc..}    \begin{align*}
        \det(A^2BC^tB^{-1}) &= \det(A^2)\det(B)\det(C^t)\det(B^{-1})\\
        &= \det(A)^2\det(B)\det(C)\det(B)^{-1}\\
        &= (-1)^2\cdot 2\cdot 3\cdot 1/2\\
        &= 3.
    \end{align*}


    \qed
    
    \item  Sea
    $$A=
    \begin{bmatrix}
        x&y&z \\
        3&0&2\\
        1&1&1
    \end{bmatrix}.$$
    Sabiendo que $\det(A) = 5$, calcular el determinante de las siguientes matrices.
    $$
    B = \begin{bmatrix}
    2x&2y&2z \\
    3/2&0&1\\
    1&1&1
    \end{bmatrix}, \qquad
    C=
    \begin{bmatrix}
        x&y&z \\
        3x+3&3y&3z+2\\
        x+1&y+1&z+1
    \end{bmatrix}.
    $$
    \rta
    $A \underset{F_2/2}{\overset{2F_1}{\longrightarrow}} B$, luego  $\det(B) = \frac12 \cdot 2\cdot \det(A) = \det(A) = 5$.

    $A \underset{F_3+F_1}{\overset{F_2+3F_1}{\longrightarrow}} C$, luego  $\det(C) = \det(A) = 5$.
    \qed
    
    \item Determinar todos los valores de $c\in\mathbb{R}$ tales que las siguientes matrices sean invertibles.
    \begin{align*}
    A=\begin{bmatrix}4& c&3\\c&2&c\\ 5&c&4 \end{bmatrix},\qquad
    B=\begin{bmatrix} 1&c&-1\\ c&1&1\\0&1&c\end{bmatrix}.
    \end{align*}
    \rta recordar que una matriz es invertible si y solo si  su determinante es no nulo. Luego, debemos calcular el determinante de $A$ y $B$ y ver para qué valores de $c$ son distintos de cero.


    Para el cálculo del determinante desarrollamos por la primera fila.
    \begin{align*}
        \det(A) &= 4\cdot \left|\begin{matrix} 2&c\\ c&4\end{matrix}\right| - c\cdot \left|\begin{matrix} c&c\\ 5&4\end{matrix}\right| + 3\cdot \left|\begin{matrix} c&2\\ 5&c\end{matrix}\right|\\
        &= 4\cdot (8 - c^2) - c\cdot (4c - 5c) + 3\cdot (c^2 - 10)\\
        &= 32 - 4c^2 - 4c^2 + 5c^2 + 3c^2 - 30\\
        &=  2.    
    \end{align*}
    Como $\det(A) = 2 \ne 0$ independientemente del valor de $c$, la matriz $A$ es invertible para todo $c\in\mathbb{R}$.

    Para el cálculo del determinante de $B$ desarrollamos por la primera fila.
    \begin{align*}
        \det(B) &= 1\cdot \left|\begin{matrix} 1&1\\ 1&c\end{matrix}\right| - c\cdot \left|\begin{matrix} c&1\\ 0&c\end{matrix}\right| - 1\cdot \left|\begin{matrix} c&1\\ 0&1\end{matrix}\right|\\
        &= 1\cdot (c - 1) - c\cdot c^2 - 1\cdot c\\
        &= -c^3 -  1.
    \end{align*}
    Luego, $\det(B) = 0$ si y solo si $c^3 = -1$, es decir si y solo si $c = -1$. Por lo tanto, la matriz $B$ es invertible si y solo si $c\ne -1$.
    \qed


    
    \item Calcular el determinante de las siguientes matrices, usando operaciones elementales por fila y/o columnas u otras propiedades del determinante. Determinar cuáles de ellas son invertibles.
    \begin{equation*}
    A=
    \begin{bmatrix}-2&3&2&-6\\ 0&4&4&-5\\ 5&-6&-3&2\\ -3&7&0&0 \end{bmatrix},\quad
    B=\begin{bmatrix} 2&0&0&0\\ 0&0&3&0\\ 0&-1&0&0\\ 0&0&0&4\end{bmatrix},\quad
    \end{equation*}
    \begin{equation*}
        C=\begin{bmatrix} -2&3&2&-6&0\\ 0&4&4&-5&0\\ 5&-6&-3&2&0\\ -3&7&0&0&0\\ 1&1&1&1&1\end{bmatrix},\quad
    D=\begin{bmatrix}1&2&3&0&0\\-1&2&-13&6&\frac{1}{3}\\2&0&0&0&0\\11&1&0&0&0\\\sqrt{2}&2&1&\pi&0\end{bmatrix},
    \end{equation*}
    \begin{equation*}
    E=\begin{bmatrix}1&-1&2&0&0\\ 3&1&4&0&0\\ 2&-1&5&0&0 \\0&0&0&2&1\\ 0&0&0&-1&4    \end{bmatrix}.
    \end{equation*}
    \rta

    \textbf{Cálculo de $\det(A)$.} Calculamos el determinante de $A$ desarrollando por la última fila debido a que tiene dos   $0$. 
    \begin{align*}
        \det(A) &= 3\cdot \left|\begin{matrix} 3&2&-6\\ 4&4&-5\\ -6&-3&2\end{matrix}\right| + 7\cdot \left|\begin{matrix} -2&2&-6\\ 0&4&-5\\ 5&-3&2\end{matrix}\right|\\
        &= 3\cdot (-49) + 7\cdot  84\\
        &= 441.
    \end{align*}
    Como $\det(A) \ne 0$, la matriz $A$ es invertible.


    \vskip .2cm
    \textbf{Cálculo de $\det(B)$.} Permutando  la fila $2$ con la fila $3$ obtenemos una matriz diagonal con $2$, $3$, $-1$, $4$ en la diagonal, luego $\det(B) = -(2\cdot 3\cdot (-1)\cdot 4) = 24$. Por lo tanto, $B$  es invertible.

    \vskip .2cm
    \textbf{Cálculo de $\det(C)$.} En este caso conviene desarrollar por la última columna debido a que tiene cuatro $0$. Entonces
    \begin{equation*}
        \det(C) = 1\cdot \left|\begin{matrix} -2&3&2&-6\\ 0&4&4&-5\\ 5&-6&-3&2\\ -3&7&0&0\end{matrix}\right|
    \end{equation*}
    Observar que la matriz de la derecha es la matriz $A$ del ejercicio anterior. Luego $\det(C) = 1\cdot 441 = 441$. Por lo tanto, $C$  es invertible.

    \vskip .2cm
    \textbf{Cálculo de $\det(D)$.} Desarrollaremos por la última columna y luego de nuevo por la última columna.
    \begin{align*}
        \det(D) &= -\frac13 \cdot  \left|\begin{matrix}1&2&3&0\\2&0&0&0\\11&1&0&0\\\sqrt{2}&2&1&\pi\end{matrix}\right| =  -\frac13 \cdot\pi \cdot  \left|\begin{matrix}1&2&3\\2&0&0\\11&1&0\end{matrix}\right|\\
        &=  -\frac13 \cdot\pi \cdot (-2) \cdot  \left|\begin{matrix}2&3\\1&0\end{matrix}\right| =  -\frac13 \cdot\pi \cdot (-2) \cdot (-3) = -2\pi.
    \end{align*}
    En  el renglón anterior desarrollamos por la fila 2. Luego, $\det(D) = -2\pi$. Por lo tanto, $D$  es invertible.

    \vskip .2cm
    \textbf{Cálculo de $\det(E)$.} Para facilitar la escritura escribamos la matriz como una matriz de bloques, 
    $$
    E = \begin{bmatrix}  E_1&0\\0&E_2 \end{bmatrix},
    $$
    donde $E_1$ es el primer bloque diagonal $3 \times 3$ y $E_2$ es el segundo  bloque diagonal de $2 \times 2$.
    
    Desarrollemos por la última columna dos veces
    \begin{align*}
        \det(E) &= (-1) \cdot \left| \begin{matrix}  E_1&0\\0&1 \end{matrix} \right| + 4 \cdot\left| \begin{matrix}  E_1&0\\0&2 \end{matrix} \right| \\
        &= (-1) \cdot 1 \left|   E_1 \right| + 4 \cdot 2\cdot\left|   E_1 \right| \\
        &= ((-1) \cdot 1 + 4 \cdot 2) \left|   E_1 \right|.
    \end{align*}

    Averigüemos ahora el determinate de $E_1$. Podríamos hacerlo por cálculo directo, pero  lo haremos transformando   $E_1$ en triangular superior por medio de operaciones elementales de fila y luego multiplicando los elementos de la diagonal. 
    \begin{align*}
        E_1 {\underset{F_3 -2F_1}{\stackrel{F_2-3F_1}{\longrightarrow}}} \begin{bmatrix}1&-1&2\\ 0&4&-2\\ 0&1&1  \end{bmatrix} 
        \stackrel{F_2 \leftrightarrow F_3}{\longrightarrow}
        \begin{bmatrix}1&-1&2\\ 0&1&1 \\ 0&4&-2 \end{bmatrix} 
        \stackrel{F_3 -4 F_2}{\longrightarrow}
        \begin{bmatrix}1&-1&2\\ 0&1&1 \\ 0&0&-6 \end{bmatrix}.
    \end{align*}
    Como hay una permutación de filas tenemos que 
    $$\det(E_1) = -(1\cdot 1 \cdot (-6)) = 6.$$

    Luego $\det(E) = (4 \cdot 2 - 1) \cdot 6 = 42$. Por lo tanto, $E$  es invertible.
    \vskip .4cm

    \noindent\textbf{Observación.} Notemos que en este ejercicio 
    $$
    \det(E) = \det(E_1)\det(E_2),
    $$
    y este es un hecho general: si $E$ es una matriz  $n \times n$ de bloques diagonales $E_{1}$ y $E_{2}$, es decir
    $$
    E = \begin{bmatrix}  E_1&0\\0&E_2 \end{bmatrix},
    $$
    entonces 
    $$
    \det(E) = \det(E_1)\det(E_2).
    $$
    También vale  el resultado para un número arbitrario de bloques diagonales. Dejamos al lector interesado la demostración de este hecho.
    \qed
    
    \item Sean $A$ y  $B$ matrices $n \times n$. Probar que:
    \begin{enumerate}
        \item\label{det A.B} $\det(AB) = \det (BA)$.
        \item\label{det B.A.B1} Si $B$ es invertible, entonces $\det(B A B^{-1}) = \det (A)$.
        \item\label{-A} $\det(-A) = (-1)^n\det (A)$.
    \end{enumerate} 
    \rta En todos los incisos usamos la propiedad de que el determinate del producto de matrices es el producto de los determinantes de cada matriz.


    \ref{det A.B} 
    $$
    \det(AB) = \det(A)\det(B) = \det(B)\det(A) = \det(BA).
    $$

    \ref{det B.A.B1} Como $BB^{-1} = \Id$, entonces $\det(B^{-1})\det(B) = \det(\Id) = 1$. Luego, $\det(B^{-1}) = 1/\det(B)$. Por lo tanto,
    \begin{align*}
        \det(B A B^{-1}) &= \det(B)\det(A)\det(B^{-1}) = \det(B)\det(B^{-1})\det(A)\\ &=  \det(B)\frac1{\det(B)}\det(A)=\det(A).
    \end{align*}
    

    \ref{-A} Sea $\Id$ es la matriz identidad $n \times n$. Observar que $-A = (-\Id)A$. Luego, $\det(-A) = \det((-\Id)A) = \det(-\Id)\det(A)$. Como $-\Id$ es una matriz diagonal con $-1$ en la diagonal, entonces $\det(-\Id) = (-1)^n$. Por lo tanto, $\det(-A) = (-1)^n\det(A)$.

    \qed
    
    \item\label{vandermonde} Sean $\lambda_1, \lambda_2, \dots, \lambda_n$ escalares, la matriz de \emph{Vandermonde} asociada es
    \begin{align*}
    \mathtt V_n = \begin{bmatrix}
    1 & \lambda_1 & \lambda_1^2 & \cdots & \lambda_1^{n-1}\\
    1 & \lambda_2 & \lambda_2^2 & \cdots & \lambda_2^{n-1}\\
    \vdots &\vdots &\vdots & &\vdots\\
    1 & \lambda_n & \lambda_n^2 & \cdots & \lambda_n^{n-1}\\
    \end{bmatrix}.
    \end{align*}
    Esta es la matriz del sistema de ecuaciones del ejercicio \ref{polinomios}\,\ref{polinomios-c} del Práctico \ref{practico-2}.
    
    
    \begin{enumerate}
        \item\label{vandermonde 2} Si $n=2$, probar que $\det(\mathtt V_n) = \lambda_2-\lambda_1$.
        \item\label{vandermonde 3} Si $n=3$, probar que $\det(\mathtt V_n) = (\lambda_3-\lambda_2) (\lambda_3-\lambda_1) (\lambda_2-\lambda_1)$.
        \item\label{vandermonde gral} Probar que $\det(\mathtt V_n) = \prod_{1\leq i< j \leq n}(\lambda_j-\lambda_i)$ para todo $n\in\mathbb{N}$.
        \item\label{vandermonde inv} Dar una condición necesaria y suficiente para que la matriz de Vandermonde sea invertible.
        \item\label{vandermonde sol} Dados $b_1, \ldots, b_n$  y $\lambda_1, \ldots, \lambda_n$ secuencias de números reales,  dar una condición suficiente para que exista un  polinomio de grado $n$, digamos $p$, tal que 
        $$
        p(\lambda_1)=b_1, \ldots, p(\lambda_n)=b_n.
        $$
        (ver ejercicio \ref{polinomios} del Práctico \ref{practico-2}).
    \end{enumerate}
    \rta

    \ref{vandermonde 2} En  este caso la matriz es  $2 \times 2$ y por lo tanto es fácil calcular el determinante:      
    \begin{align*}
        \det(\mathtt V_2) &= \left|\begin{matrix} 1 & \lambda_1 \\ 1 & \lambda_2 \end{matrix}\right| = \lambda_2 - \lambda_1.
    \end{align*}

    \vskip .2cm
    \ref{vandermonde 3} En  este caso la matriz es  $3 \times 3$ y calcularemos el determinante transformado la matriz por operaciones elementales de columnas y filas. El  método que haremos servirá para generalizarlo en el caso  \ref{vandermonde gral}.
    \begin{align*}
        \det(\mathtt V_3) &= \left|\begin{matrix} 1 & \lambda_1 & \lambda_1^2 \\ 1 & \lambda_2 & \lambda_2^2 \\ 1 & \lambda_3 & \lambda_3^2 \end{matrix}\right| \\
        &\stackrel{C_3-\lambda_1 C_2}{=} \left|\begin{matrix}  1 & \lambda_1 & 0 \\ 1 & \lambda_2 & \lambda_2(\lambda_2-\lambda_1) \\ 1 & \lambda_3 & \lambda_3(\lambda_3-\lambda_1)  \end{matrix}\right| \\
        &\stackrel{C_2-\lambda_1 C_1}{=} \left|\begin{matrix}  1 & 0 & 0 \\ 1 & \lambda_2 -\lambda_1& \lambda_2(\lambda_2-\lambda_1) \\ 1 & \lambda_3-\lambda_1 & \lambda_3(\lambda_3-\lambda_1)  \end{matrix}\right| \\
        &\overset{F_2 /(\lambda_2-\lambda_1)}{\stackrel{F_3 /(\lambda_3-\lambda_1)}{=}}(\lambda_2-\lambda_1)(\lambda_3-\lambda_1)  \left|\begin{matrix}  1 & 0 & 0 \\ 1 & 1& \lambda_2 \\ 1 & 1 & \lambda_3 \end{matrix}\right| \\
        &{=}(\lambda_2-\lambda_1)(\lambda_3-\lambda_1)  \left|\begin{matrix}   1& \lambda_2 \\ 1 & \lambda_3 \end{matrix}\right| \\
        &= (\lambda_2-\lambda_1) (\lambda_3-\lambda_1)(\lambda_3-\lambda_2).
    \end{align*}
    Para el último cálculo desarrollamos por la primera fila. 

    Observar que $\det(V_3)$ es $ (\lambda_2-\lambda_1) (\lambda_3-\lambda_1)$ por una matriz de Vandermonde de dimensión $2 \times 2$. Esto se  puede generalizar para calcular el determinante de una matriz de Vandermonde de mayor dimensión, como veremos a continuación. 

    \vskip .2cm
    \ref{vandermonde gral} Haremos la demostración por inducción. El caso base es $n=2$ y en ese caso el resultado vale por \ref{vandermonde 2}. Supongamos ahora que valga el resultado \ref{vandermonde gral} para $n-1$. 
    
    Fijarse que en el caso  anterior (\ref{vandermonde 3}) hicimos dos operaciones elementales de columna: $C_3-\lambda_1 C_2$ y $C_2-\lambda_1 C_1$,  en ese orden. En general, haremos  $n-1$ operaciones elementales de columnas de la forma $C_i-\lambda_1 C_{i-1}$, desde $i=n$ hasta $i=2$,  en ese orden con lo cual obtenemos la matriz

\begin{equation*}
    \begin{bmatrix}
    1 & 0 & 0 & \dots & 0\\
    1 & \lambda_2-\lambda_1 & \lambda_2(\lambda_2-\lambda_1) & \dots & \lambda_2^{n-2}(\lambda_2-\lambda_1)\\
    1 & \lambda_3-\lambda_1 & \lambda_3(\lambda_3-\lambda_1) & \dots & \lambda_3^{n-2}(\lambda_3-\lambda_1)\\
    \vdots & \vdots & \vdots & \ddots &\vdots \\
    1 & \lambda_n-\lambda_1 & \lambda_n(\lambda_n-\lambda_1) & \dots & \lambda_n^{n-2}(\lambda_n-\lambda_1)\\
    \end{bmatrix} \tag{*}
\end{equation*}
Ninguna de las operaciones elementales que hicimos cambia el valor del determinante. Si calculamos el determinante de la matriz (*) desarrollando por la primera fila, obtenemos
\begin{equation*}
    \begin{vmatrix}V_n \end{vmatrix}=\begin{vmatrix}
        \lambda_2-\lambda_1 & \lambda_2(\lambda_2-\lambda_1) & \dots & \lambda_2^{n-2}(\lambda_2-\lambda_1)\\
        \lambda_3-\lambda_1 & \lambda_3(\lambda_3-\lambda_1) & \dots & \lambda_3^{n-2}(\lambda_3-\lambda_1)\\
        \vdots & \vdots & &\vdots \\
        \lambda_n-\lambda_1 & \lambda_n(\lambda_n-\lambda_1) & \dots & \lambda_n^{n-2}(\lambda_n-\lambda_1)\\
        \end{vmatrix}.\tag{**}.
\end{equation*}

Ahora dividimos la fila $i$ de la matriz de la derecha de (**) por $\lambda_i-\lambda_1$ para $i=2,\dots,n$ y obtenemos
\begin{equation*}
\begin{vmatrix} V_n \end{vmatrix}=
(\lambda_2-\lambda_1)(\lambda_3-\lambda_1)\dots(\lambda_n-\lambda_1)
\begin{vmatrix}
1 & \lambda_2 & \lambda_2^2 & \dots & \lambda_2^{n-2}\\
1 & \lambda_3 & \lambda_3^2 & \dots & \lambda_3^{n-2}\\
1 & \lambda_4 & \lambda_4^2 & \dots & \lambda_4^{n-2}\\
\vdots & \vdots & \vdots & &\vdots \\
1 & \lambda_n & \lambda_n^2 & \dots & \lambda_n^{n-2}\\
\end{vmatrix}. \tag{***}
\end{equation*}
La matriz de la derecha de (***) es una matriz de Vandermonde de dimensión $n-1$ con variables $\lambda_2, \lambda_3,\ldots,\lambda_n$, y por lo tanto su determinante es, por hipótesis inductiva, 
\begin{equation*}
    \prod_{2\leq i< j \leq n}(\lambda_j-\lambda_i).
\end{equation*}
En consecuencia,
\begin{align*}
    \begin{vmatrix} V_n \end{vmatrix}&=
    (\lambda_2-\lambda_1)(\lambda_3-\lambda_1)\dots(\lambda_n-\lambda_1)
    \prod_{2\leq i< j \leq n}(\lambda_j-\lambda_i) \\
    &=\prod_{1\leq i< j \leq n}(\lambda_j-\lambda_i).
\end{align*}
    
\vskip .2cm
\ref{vandermonde inv} Sabemos que una matriz es invertible si y solo si su determinante es no nulo y que el determinante de una matriz de Vandermonde es el producto de los factores $(\lambda_j-\lambda_i)$ con $i<j$. Por lo tanto, la matriz de Vandermonde es invertible si y solo si $\lambda_j-\lambda_i \ne 0$ para $i<j$. Es decir, si y solo si $\lambda_j \ne \lambda_i$ para $i\ne j$. 

\vskip .2cm
\ref{vandermonde sol} Dados $b_1, \ldots, b_n$  y $\lambda_1, \ldots, \lambda_n$ secuencias de números reales,  se plantea se plantea si existe una polinomio de grado $n$, digamos $p$, tal que 
$$
p(\lambda_1)=b_1, \ldots, p(\lambda_n)=b_n.
$$
Los coeficientes de $p$ son las incógnitas del sistema y la matriz de coeficientes es la matriz de Vandermonde. Por lo tanto, el sistema tiene solución única si la matriz de Vandermonde es invertible, es decir si $\lambda_j \ne \lambda_i$ para $i\ne j$. Por lo tanto,  una condición suficiente para que exista el polinomio $p$ es: $\lambda_j \ne \lambda_i$ para $i\ne j$. 

Hay otras situaciones donde el sistema tiene solución, pero no es única. Por  ejemplo,  si $\lambda_1 = \lambda_2$ y $b_1 = b_2$,  y  $\lambda_j \ne \lambda_i$ para $i\ne j$ para $2 \le i,j \le n$, entonces el sistema tiene infinitas soluciones. 



    \qed
        
    \item Decidir si las siguientes afirmaciones son verdaderas o falsas. Justificar con una demostración o con un contraejemplo, según corresponda.
        \begin{enumerate}
        \item\label{det VoF det(A+b)} Sean $A$ y $B$ matrices $n \times n$. Entonces $\det(A + B) = \det (A) + \det(B)$.
        \item\label{det VoF det 3x2}  Existen una matriz $3\times 2$, $A$, y una matriz $2\times 3$, $B$, tales que $\det(AB) \neq 0$.
        \item\label{inv A^n} Sea $A$ una matriz $n\times n$. Si $A^n$ es no invertible, entonces $A$ es no invertible.
    \end{enumerate}
    \rta

    \ref{det VoF det(A+b)} Como ya fue visto en la teórica esta afirmación es falsa y un ejemplo sencillo es el siguiente:   
    \begin{equation*}
        A = \begin{bmatrix} 1&0\\0&0 \end{bmatrix}, \qquad B = \begin{bmatrix} 0&0\\0&1 \end{bmatrix}.
    \end{equation*} En este caso $\det(A) = \det(B) = 0$ y $\det(A+B) = 1$.
    
    \vskip .2cm
    \ref{det VoF det 3x2} Esta afirmación es falsa y la a continuación haremos una demostración de este hecho. Observar que con un contraejemplo no  basta, pues debemos demostrar  que dadas $A$ matriz  $3\times 2$ y $B$ matriz $2\times 3$, \textit{cualesquiera}, entonces  que $\det(AB) = 0$.

    Haremos operaciones elementales por fila en la matriz $A$ para transformarla en una matriz triangular superior. Como en el caso $3 \times 3$  hacer operaciones elementales por fila  es equivalente a multiplicar a izquierda por matrices elementales $3 \times 3$. 
    
    Por lo tanto si $A'$ es una matriz triangular superior obtenida de $A$ por operaciones elementales por fila, entonces $A' = EA$ para alguna matriz $E$ invertible $3 \times 3$ donde  $E$ es producto de matrices elementales. Como $\det(EAB) = \det(E)\det(AB)$, entonces  $\det(EAB) \ne 0 $ si y solo si $\det(AB) \ne 0$.
\vskip .2cm
    Ahora bien,  $A'$ una matriz $3 \times 2$ triangular superior tiene la forma 
    $$
    A' = \begin{bmatrix} a&b\\0&c\\0&0 \end{bmatrix},
    $$
    por lo tanto
    $$
    A'B = \begin{bmatrix} a&b\\0&c\\0&0 \end{bmatrix} \begin{bmatrix} x&y&z\\u&v&w \end{bmatrix} = \begin{bmatrix} ax+bu&ay+bv&az+bw\\0&cy&cz\\0&0&0 \end{bmatrix}.
    $$ 
    La matriz resultante tiene la última fila con todos los coeficientes iguales a $0$, por lo tanto $\det(A'B) = 0$, lo cual implica que $\det(AB) = 0$, como queríamos demostrar.

    \vskip .2cm
    \ref{inv A^n} Esta afirmación es verdadera. Si $A^n$ es no invertible, entonces $\det(A^n) = 0$. Como $\det(A^n) = \det(A)^n$, entonces $\det(A)^n = 0$ y por lo tantos $\det(A) = 0$. En  consecuencia, $A$ es no invertible.


    \qed
    
    \end{enumerate}
    