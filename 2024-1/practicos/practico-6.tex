\chapter{Espacios y subespacios vectoriales \\Álgebra  II -- Año 2024/1 -- FAMAF}\label{practico-6}


    

\subsection*{Objetivos}
    
\begin{itemize}
\item Familiarizarse con los conceptos de espacio y subespacio vectorial.
\item Familiarizarse con los conceptos de conjunto de generadores e independencia lineal, base y dimensión de un espacio vectorial.
        
\item Aprender a caracterizar los subespacios de $\mathbb K^n$ por generadores y de manera implícita.

\item Dado un subespacio $W$ de $\mathbb K^n$, aprender a extraer una base de cualquier conjunto de generadores de $W$, y a completar cualquier subconjunto linealmente independiente de $W$ a una base.

\end{itemize}
    
    
\subsection*{Ejercicios} Los ejercicios con el símbolo $\textcircled{a}$ tienen una ayuda al final del archivo para que recurran a ella después de pensar un poco.

\begin{enumerate}[topsep=6pt, itemsep=.4cm]

    
% \item\label{func} Sea $F$ un cuerpo. Si $(V,\oplus,\odot)$ es un $F$-espacio vectorial y $S$ un conjunto cualquiera, entonces
% $$V^S=\{f:S\to V: \, f \ \text{es una función}\},$$
% denota al conjunto de todas las funciones de $S$ en $V$.  Definimos en $V^S$ la suma y el producto por escalares de la siguiente manera:
% Si $f,g \in V^S$ y $c\in F$ entonces $f + g: S \rightarrow V $ y $c\cdot f: S \rightarrow V$ están dadas por
% $$
% (f + g)(x) = f(x) \oplus g(x), \quad  (c\cdot f)(x) = c\odot f(x), \qquad  \forall\, x \,\in S.
% $$
% Probar que $(V^S,+,\cdot)$ es un $F$-espacio vectorial.


\item\label{sub Rn} Decidir si los siguientes subconjuntos de $\mathbb{R}^3$ son subespacios vectoriales.


    \begin{enumerate}
%         \item $\{(x_1, \ldots ,x_n) \in \mathbb{R}^n \ : \ x_1 = x_n\}$.
        \item\label{sub Rn 1} $A=\{(x_1, x_2 ,x_3) \in \mathbb{R}^3 \ : \ x_1 + x_2 + x_3=1\}$.
        \item\label{sub Rn 0} $B=\{(x_1, x_2 ,x_3) \in \mathbb{R}^3 \ : \ x_1 + x_2 + x_3=0\}$.
        \item\label{sub Rn geq} $C=\{(x_1, x_2 ,x_3) \in \mathbb{R}^3 \ : \ x_1 + x_2 + x_3 \geq 0\}$.
%         \item $\{(x_1, \ldots ,x_n) \in \mathbb{R}^n \ : \ x_n=1\}$.
        \item\label{sub Rn 1 30} $D=\{(x_1, x_2 ,x_3) \in \mathbb{R}^3 \ : \ x_3=0\}$.
        \item\label{sub Rn cup} $B\cup D$.
        \item\label{sub Rn cap} $B\cap D$.
        \item\label{sub Rn q} $G=\{(x_1, x_2 ,x_3) \in \mathbb{R}^3 \ :\ x_1, x_2, x_3\in\mathbb{Q}\}$.
    \end{enumerate}
\end{enumerate}

\textbf{Observación} En los items \ref{sub Rn 1}, \ref{sub Rn 0} y \ref{sub Rn geq} del ejercicio \ref{sub Rn} podemos apreciar como un simple cambio en la condición que define al subconjunto hace que dicho subconjunto sea o no un subespacio vectorial. Este es un fenómeno que pasa en general. De hecho podríamos haber definido subconjuntos similares para todo $\mathbb{R}^n$. Lo mismo sucede en los ejercicios \ref{sub funciones} y \ref{sub polinomios}.
En {\bf Ayudas},  al final del práctico, están las respuestas a los ejercicios \ref{sub Rn}, \ref{sub matrices} y \ref{sub funciones}. 
        
\begin{enumerate}[resume, topsep=6pt, itemsep=.4cm]

\item\label{sub matrices} Decidir en cada caso si el conjunto dado es un subespacio vectorial de $M_{n\times n}(\mathbb{K})$.
\begin{enumerate}
    \item\label{sub matrices invertibles} El conjunto de matrices  invertibles.
%     \item El conjunto de matrices no inversibles.
    \item\label{sub matrices AB} El conjunto de matrices $A$ tales que $AB = BA$, donde $B$ es una matriz fija.
%     \item El conjunto de matrices simétricas $\{A\in M_n(\mathbb{R})\ :\ A=A^t\}$
    \item\label{sub matrices triangulares} El conjunto de matrices triangulares superiores.
%     \item El conjunto de matrices de traza cero $\{A\in M_n(\mathbb{R})\ :\ \operatorname{tr}(A)=0\}$
%     \item $\{A\in M_n(\mathbb{R})\ :\ \operatorname{tr}(A)=1\}$
\end{enumerate}


\item\label{rectas} $\textcircled{a}$ Sea $L$ una recta en $\mathbb{R}^2$. Dar una condición necesaria y suficiente para que $L$ sea un subespacio vectorial de $\mathbb{R}^2$.


\item Sean $V$ un $\mathbb{K}$-espacio vectorial, $v\in V$ no nulo y $\lambda,\mu\in\mathbb{K}$ tales que $\lambda v=\mu v$. Probar que $\lambda=\mu$.


\item Sean $W_1, W_2$ subespacios de un espacio vectorial $V$. Probar que $W_1 \cup W_2$ es un subespacio
    de $V$ si y sólo si $W_1 \subseteq W_2$ o $W_2 \subseteq W_1$.
    

\item Sean $u=(1,1)$, $v=(1,0)$, $w=(0,1)$ y $z=(3,4)$ vectores de $\mathbb{R}^2$.
\begin{enumerate}
\item Escribir $z$ como combinación lineal de $u,v$ y $w$, con coeficientes todos no nulos.
\item Escribir $z$ como combinación lineal de $u$ y $v$.
\item Escribir $z$ como combinación lineal de $u$ y $w$.
\item Escribir $z$ como combinación lineal de $v$ y $w$.
\end{enumerate}

\vskip .4cm
\textbf{Observación.} En este ejercicio vemos como un vector se puede escribir de muchas maneras como combinación lineal de vectores dados. Esto pasa porque $\{u,v,w\}$ es LD.



\item Sean $p(x)=(1-x)(x+2)$, $q(x)=x^2-1$ y $r(x)=x(x^2-1)$ en $\mathbb{R}[x]$.

\begin{enumerate}
\item Describir en forma implícita  todos los polinomios de grado menor o igual que $3$ que son combinación lineal de $p,q$ y $r$.

\item Elegir $a$ tal que el polinomio $x$ se pueda escribir como combinación lineal de $p,q$ y $2x^2+a$.
 \end{enumerate}


\item\label{practicos anteriores} Dar un conjunto de generadores para los siguientes subespacios vectoriales.

\begin{enumerate}
\item Los conjuntos de soluciones de los sistemas homogéneos del ejercicio \ref{sistemas homogeneos} del Práctico \ref{practico-2}.
\item Los conjuntos descriptos en el ejercicio \ref{sistemas con soluciones} del Práctico  \ref{practico-2}.
\end{enumerate}


\item\label{caracterizar}  En cada caso, caracterizar con ecuaciones al subespacio vectorial dado por generadores.

\begin{enumerate}
\item ${\left\langle(1,0,3),(0,1,-2)\right\rangle}\subseteq \mathbb{R}^3$.
\item ${\left\langle(1,2,0,1),(0,-1,-1,0),(2,3,-1,4)\right\rangle}\subseteq \mathbb{R}^4$.
\end{enumerate}


\item\label{son LI} En cada caso, determinar si el subconjunto indicado es linealmente independiente.

\begin{enumerate}
    \item $\{ (1,0,-1), (1,2,1), (0,-3,2) \}\subseteq \mathbb{R}^3$.

    \vspace{0.2cm}

    \item $\left\{  \begin{bmatrix} 1 & 0 & 2 \\ 0 & -1 & -3 \\ \end{bmatrix}, \quad
    \begin{bmatrix} 1 & 0 & 1 \\ -2 & 1 & 0 \\ \end{bmatrix}, \quad
    \begin{bmatrix} 1 & 2 & 3 \\ 3 & 2 & 1 \\ \end{bmatrix} \right\}\subseteq M_{2\times 3}(\mathbb{R})$.
\end{enumerate}


\item Dar un ejemplo de un conjunto de 3 vectores en $\mathbb{R}^3$ que sean LD, y tales que dos cualesquiera de ellos sean LI.


\item  Probar que si $\alpha$, $\beta$ y $\gamma$ son vectores LI en el $\mathbb{R}$-espacio vectorial $V$, entonces $\alpha +\beta$, $\alpha +\gamma$ y $\beta +\gamma $ también son LI.


\item Extender, de ser posible, los siguientes conjuntos a una base de los respectivos espacios vectoriales.

\begin{enumerate}
    \item Los conjuntos del ejercicio \ref{son LI}.
    \item\label{10b} $\{ (1,2,0,0),(1,0,1,0) \}\subseteq\mathbb{R}^4$.
    \item\label{10c} $\{ (1,2,1,1),(1,0,1,1),(3,2,3,3)\}\subseteq\mathbb{R}^4$.
\end{enumerate}


\item Dar subespacios vectoriales $W_0$, $W_1$, $W_2$ y $W_3$ de $\mathbb{R}^3$ tales que $W_0\subset W_1\subset W_2\subset W_3$ y $\dim W_0=0$, $\dim W_1=1$, $\dim W_2=2$ y $\dim W_3=3$.


\item Sea $V$ un espacio vectorial de dimensión $n$ y $\mathcal{B}=\{v_1, ..., v_n\}$ una base de $V$.
\begin{enumerate}
 \item Probar que cualquier subconjunto no vacío de $\mathcal{B}$ es LI.
 \item Para cada $k\in\mathbb{N}_0$,  con $0\leq k\leq n$, dar un subespacio vectorial de $V$ de dimensión $k$.
\end{enumerate}


\item Dar una base y calcular la dimensión de $\mathbb{C}^n$ como $\mathbb{C}$-espacio vectorial y como $\mathbb{R}$-espacio vectorial.


\item  Exhibir una base y calcular la dimensión de los siguientes subespacios.

 \begin{enumerate}
    \item Los subespacios del ejercicio \ref{practicos anteriores}.
    \item $W = \{(x,y,z,w,u) \in \mathbb{R}^5 \ : \ y = x - z,\, w = x + z,\,  u = 2x - 3z \}$.
    \item $W = \langle (1, 0, -1, 1),  (1, 2, 1, 1), (0, 1, 1, 0), (0, -2, -2, 0) \rangle \subseteq \mathbb R^4$.
    \item Matrices triangulares superiores $2\times 2$ y $3\times 3$.
    \item Matrices triangulares superiores $n\times n$ para cualquier $n\in\mathbb{N}$, $n\geq 2$.
\end{enumerate}

\item Sean $W_1$ y $W_2$ los siguientes subespacios de $\mathbb{R}^3$:
    \begin{align*}
    W_1 &= \{ (x,y,z)\in\mathbb{R}^3\ : \ x+y-2z=0\},  \\
    W_2 &= {\left\langle(1,-1,1),(2,1,-2),(3,0,-1)\right\rangle}.
    \end{align*}
    \begin{enumerate}
        \item  Determinar $W_1 \cap W_2$, y describirlo por generadores y con ecuaciones.
        \item  Determinar $W_1+W_2$, y describirlo por generadores y con ecuaciones.
        %\item  ?`Es la suma $W_1+W_2$ directa?
%        \item  Dar un complemento de $W_1$.
%        \item  Dar un complemento de $W_2$.
    \end{enumerate}


\item\label{verdadero o falso} Decidir si las siguientes afirmaciones son verdaderas o falsas. Justificar.

\begin{enumerate}
 \item Si $W_1$ y $W_2$ son subespacios vectoriales de $\mathbb{K}^8$ de dimensión $5$, entonces $W_1\cap W_2=0$.

  \item Si $W$ es un subespacio de $\mathbb{K}^{2\times2}$ de dimensión $2$, entonces existe una matriz triangular superior no nula que pertence a $W$.

 \item Sean $v_1, v_2, w\in \mathbb{K}^{n}$ y $A\in\mathbb{K}^{n\times n}$ tales que $Av_1=Av_2=0\neq Aw$. Si $\{v_1, v_2\}$ es LI, entonces $\{v_1,v_2,w\}$ también es LI.

\item\label{cos} $\textcircled{a}$ $\{1,{\rm sen}(x),\cos(x)\}$ es un subconjunto LI del espacio de funciones de $\mathbb{R}$ en $\mathbb{R}$.

\item\label{cos2} $\textcircled{a}$ $\{1,{\rm sen}^2(x),\cos^2(x)\}$ es un subconjunto LI del espacio de funciones $\mathbb{R}$ en $\mathbb{R}$.

\item\label{exponencial} $\textcircled{a}$ $\{e^{\lambda_1x},e^{\lambda_2x},e^{\lambda_3x}\}$ es un subconjunto LI del espacio de funciones de
$\mathbb{R}$ en $\mathbb{R}$, si $\lambda_1$, $\lambda_2$ y $\lambda_3$ son todos distintos.
\end{enumerate}

% \
%
% \item Sean $W_b$ y $W_c$ los subespacios de $\mathbb{R}^4$ generados por los subconjunto dados en los ejercicios \ref{10b} y \ref{10c}. Decidir si $W_b+W_c$ es una suma directa. En caso de que no lo sea dar un complemento de $W_b$ y otro para $W_c$.



\end{enumerate}


\textbf{Ejercicios de repaso}
Si ya hizo los ejercicios anteriores continue con la siguiente guía. Los ejercicios que siguen son similares y le pueden servir para practicar antes de los exámenes.


\begin{enumerate}[resume, topsep=6pt, itemsep=.4cm]

\item Decidir en cada caso si el conjunto dado es un subespacio vectorial de $\mathbb{R}^{n}$.
    \begin{enumerate}
        \item $\{(x_1, \ldots ,x_n) \in \mathbb{R}^n \ : \ \exists \, j > 1, \, x_1 = x_j\}$.
        \item $\{(x_1, \ldots , x_n) \in\mathbb{R}^n \ : \ x_1x_n = 0 \}$.
    \end{enumerate}


\item\label{sub funciones} Sea $F[0,1]$ el espacio de funciones de $[0,1]$ en $\mathbb{R}$. Decidir en cada caso si el conjunto dado es un subespacio vectorial de $F[0,1]$.
    \begin{enumerate}
%     \item $C[0,1] = \{ f : [0,1] \rightarrow \mathbb{R} \ : \ f \ \text{es continua}\}$.
%             \item $C^1[0,1] = \{ f : [0,1] \rightarrow \mathbb{R} \ : \ f \ \text{es  derivable}\}$.
% \item $\{ f : [0,1] \rightarrow \mathbb{R} \ : \ f \ \text{es  derivable y }f'=0\}$.
%                     \item $\{ f \in C[0,1] \ : \ f(1) \geq 0\}$.
    \item\label{sub funciones 1} $\{ f \in F[0,1] \ : \ f(1) = 1 \}$.
    \item\label{sub funciones 0} $\{ f \in F[0,1] \ : \ f(1) = 0\}$.
% %             \item $F=\{f \in C[0,1] \ : \ f(1) = 0\}$.
%     \item $E\cup F$
%     \item $E\cap F$
\end{enumerate}
    
    \item\label{sub polinomios} Decidir si los siguientes subconjuntos de $\mathbb{R}[x]$ son subespacios vectoriales.

    \begin{enumerate}
     \item $\mathbb{R}_{n}[x] := \{ a_0 + \cdots + a_{n-1}x^{n-1} \ : \ a_i \in \mathbb{R}\}$, es decir, el conjunto formado por todos los polinomios de grado estrictamente menor que $n\in\mathbb{N}$.

     \item $B=\{p(x)\in\mathbb{R}_{n}[x] \ : \ a_0 + \cdots + a_{n-1} = 1\}$.
     \item $C=\{p(x)\in\mathbb{R}_{n}[x] \ : \ a_0 + \cdots + a_{n-1} = 0\}$.
     \item $D=\{p(x)\in\mathbb{R}_{n}[x] \ : \ a_{n-1} \le a_{n-2}\}$.
%         \item $\{(x_1, \ldots ,x_n) \in \mathbb{R}^n \ : \ x_n=1\}$.
     \item $E=\{p(x)\in\mathbb{R}_{n}[x] \ : \ a_{n-1}=0\}$.
     \item $C\cup E$.
     \item $C\cap E$.
     \item $F=\{p(x)\in\mathbb{R}_{n}[x] \ : \  a_0, ..., a_{n-1}\in\mathbb{Q}\}$.
%     \item El conjunto $\mathbb{R}_{par}[x]$ formado por los polinomios de grado par, junto
%     con el polinomio nulo.
%     
%     \item $\mathbb{R}_{< n}[x]\cup \mathbb{R}_{par}[x]$
%
%     \item $\mathbb{R}_{< n}[x]\cap \mathbb{R}_{par}[x]$
     \end{enumerate}


\item  Hallar $a, b, c\in \mathbb{R}$ tales que $(-1,2,1)=a(1,1,1)+b(1,-1,0)+c(2,1,-1)$.


\item
\begin{enumerate}
    \item Hallar escalares $a, b \in \mathbb R$ tales que $1+2i=a(1+i)+b(1-i)$.
    \item  Hallar escalares $w, z \in \mathbb C$ tales que $1+2i=z(1+i)+w(1-i)$.
\end{enumerate}
%
% \item Sean $u=(-1,1)$, $v=(i,i)$, $w=(2,-i)$ y $z=(1,1+i)$.
%         \begin{enumerate}
%             \item Escribir $z$ como combinación lineal de $u,v$ y $w$, con coeficientes todos no nulos.
%             \item Escribir $z$ como combinación lineal de $u$ y $v$.
%             \item Escribir $z$ como combinación lineal de $u$ y $w$.
%             \item Escribir $z$ como combinación lineal de $v$ y $w$.
%         \end{enumerate}


\item  Repetir el ejercicio \ref{son LI} con los subespacios:

\begin{enumerate}
    \item ${\left\langle(1,1,0,0),(0,1,1,0),(0,0,1,1)\right\rangle}\subseteq \mathbb{R}^4$.
    \item ${\left\langle 1+x+x^2,\, x-x^2+x^3,\, 1-x,\, 1-x^2,\, x-x^2,\, 1+x^4\right\rangle}\subseteq \mathbb{R}[x]$.
\end{enumerate}


\item En este ejercicio no es necesario hacer ninguna cuenta. Es lógica y comprender bien la definición de LI y LD. Probar las siguientes afirmaciones.
\begin{enumerate}
\item Todo conjunto que contiene un subconjunto LD es también LD.
\item Todo conjunto que contiene al vector 0 es LD.
\item Un conjunto es LI si y sólo si todos sus subconjuntos \emph{finitos} son LI.
\end{enumerate}


\item Sean $\lambda_1, ..., \lambda_n\in\mathbb{R}$ todos distintos. Probar que el conjunto $\{e^{\lambda_1x}, ..., e^{\lambda_nx}\}$ es LI.


\item  Exhibir una base y calcular la dimensión de los siguientes subespacios.

\begin{enumerate}
    \item $W=\{(x,y,z) \in \mathbb{R}^3 \ : \ z = x + y \}$.
    \item $W = \langle (-1, 1, 1, -1, 1),  (0, 0, 1, 0, 0), (2, -1, 0, 2, -1), (1, 0, 1, 1, 0) \rangle \subseteq \mathbb R^5$.
\end{enumerate}


\item Exhibir una base y calcular la dimensión de los siguientes subespacios.
\begin{enumerate}
    \item $W = \{ p(x)=a+bx+cx^2+dx^3\in \mathbb{R}_{4}[x] \ : \ a+d=b+c \}$.
\item $W= \{ p(x)\in \mathbb{R}_{4}[x] \ : \ p'(0)=0 \}$.
    \item $W = \{A \in \mathbb{R}^{n\times n} \ : \ A = A^t\}$.
%     \item $S = \{A \in \mathbb{C}^{n\times n} \ : \ A = \bar{A^t}\}$ (considerado como $\mathbb{R}$-subespacio de $\mathbb{C}^{n\times n}$).
\end{enumerate}

    \item Sea  $S=\{v_1,v_2,v_3,v_4\}\subset\mathbb R^4$, donde
$$v_1=(-1,0,1,2), \quad v_2=(3,4,-2,5), \quad v_3=(0,4,1,11), \quad v_4=(1,4,0,9).$$
\begin{enumerate}
    \item  Describir implícitamente al subespacio  $W= \langle \, S\, \rangle$.
    \item Si $W_1 = \langle \, v_1,v_2,v_3+v_4\, \rangle $ y $W_2 = \langle \, v_3,v_4\, \rangle $,
    describir $W_1\cap W_2$ implícitamente.
\end{enumerate}


\item\label{matrices} Sean
    $
    A_1=\begin{bmatrix}
    1&-2&0&3&7\\
    2&1&-3&1&1
    \end{bmatrix}$ y $A_2=\begin{bmatrix}
    3&2&0&0&3\\
    1&0&-3&1&0 \\
    -1&1&-3&1&-2
    \end{bmatrix}
    $.
    
    \begin{enumerate}
    \item Sean $W_1$ y $W_2$ los espacios solución de los sistemas
    homogéneos asociados a $A_1$ y $A_2$, respectivamente.  Describir implícitamente $W_1\cap W_2$.
    \item Sean $V_1$ y $V_2$ los subespacios de $\mathbb{R}^5$ generado por las filas de $A_1$ y $A_2$, respectivamente. Dar un conjunto de generadores de $V_1+V_2$.
    \end{enumerate}


\item\label{todo} Sean $W_1$ y $W_2$ los siguientes subespacios de $\mathbb{R}^6$:
    \begin{align*}
    W_1 &= \{ (u,v,w,x,y,z)\ : \ u+v+w=0,\, x+y+z=0\},  \\
    W_2 &= \left\langle{(1,-1,1,-1,1,-1),(1,2,3,4,5,6),(1,0,-1,-1,0,1),(2,1,0,0,0,0)}\right\rangle.
    \end{align*}
    \begin{enumerate}
        \item  Determinar $W_1 \cap W_2$, y describirlo por generadores y con ecuaciones.
        \item  Determinar $W_1+W_2$, y describirlo por generadores y con ecuaciones.
        \item  Decir cuáles de los siguientes vectores están en $W_1\cap W_2$ y cuáles en $W_1+W_2$:
        \[ (1,1,-2,-2,1,1),\ (0,0,0,1,0,-1),\ (1,1,1,0,0,0),\ (3,0,0,1,1,3),\ (-1,2,5,6,5,4). \]
        \item Para los vectores $v$ del punto anterior que estén en $W_1+W_2$,  hallar $w_1\in W_1$ y $w_2\in W_2$ tales que $v=w_1+w_2$.
%         \item  ?`Es la suma $W_1+W_2$ directa?
% \item  Dar un complemento de $W_1$.
% \item  Dar un complemento de $W_2$.

    \end{enumerate}


\end{enumerate}

% \textbf{Ejercicios un poco más difíciles}
%
% Si ya hizo los primeros ejercicios ya sabe lo que tiene que saber. Los siguientes ejercicios le pueden servir si esta muy aburridx con la cuarentena.
%
%
% \begin{enumerate}[resume]
% \item Decidir si los siguientes conjuntos son $\mathbb{R}$-espacios vectoriales, con las operaciones abajo definidas.
%
% \
%
% \begin{enumerate}
% \item $\mathbb{R}^n$, con $v\oplus w = v - w$, y el producto por escalares usual.
%         
% \item $\mathbb{R}^2$, con $(x,y)\oplus(x_1,y_2) = (x + x_1, 0), \,\,c\odot(x,y) = (cx,0)$.
%
% \item $\mathbb{R}^{3}$, con:
%         \begin{align*}
%         (x,y,z)\oplus(x',y',z') &=(x + x', y + y' - 1, z + z');\\
%         c\odot(x,y,z) &= (cx,cy + 1 - c, cz).
%         \end{align*}
%         \item El conjunto de polinomios, con el producto por escalares (reales) usual, pero con suma
%         $p(x)\oplus q(x) = p'(x) + q' (x)$ (suma de derivadas).
% \end{enumerate}
%\end{enumerate}

\textbf{Ayudas}

Ejercicio \ref{sub Rn}: 
\ref{sub Rn 1} No. \ref{sub Rn 0} Si. \ref{sub Rn geq} No. \ref{sub Rn 1 30} Si. \ref{sub Rn cup} No. \ref{sub Rn cap} Si. \ref{sub Rn q} No.


Ejercicio \ref{sub matrices}\,\ref{sub matrices invertibles} No; recordar el ejercicio \ref{suma-de-invertibles} del Práctico  \ref{practico-3}. \ref{sub matrices AB} Si. \ref{sub matrices triangulares} Si.

Ejercicio \ref{rectas} Recordar el ejercicio \ref{rectas-por-el-0} del Práctico  \ref{practico-1}.

Ejercicio \ref{verdadero o falso}\, \ref{cos} Verdadero. Plantear una combinación lineal de las funciones que de igual a cero y evaluar en diferentes valores de $x$ para obtener alguna condición sobre los escalares.

Ejercicio \ref{verdadero o falso}\, \ref{cos2} Falso. Utilizar una igualdad trigonométrica.

Ejercicio \ref{verdadero o falso}\, \ref{exponencial} Verdadero. Plantear una combinación lineal de las funciones que de igual a cero. Derivar dos veces la igualdad obteniendo así dos nuevas combinaciones lineales que den cero. Evaluar en cero las tres combinaciones lineales y utilizar la matriz de Vandermonde.

Ejercicio \ref{sub funciones}: \ref{sub funciones 1} No. \ref{sub funciones 0} Si.

