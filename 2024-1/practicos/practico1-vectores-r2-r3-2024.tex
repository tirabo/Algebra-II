\documentclass[12pt]{amsart}
\usepackage{amssymb}
\usepackage{enumerate}
\usepackage{amsmath}
\usepackage{geometry}
\geometry{ a4paper, total={210mm,297mm}, left=2cm, right=2cm, top=1.5cm, bottom=2.5cm, }
\usepackage{graphicx}
\usepackage{fancyhdr}
\usepackage{multicol}
\usepackage{enumitem}

\pagestyle{fancy}



\begin{document}

%\title{Pr\'actico 1}

\noindent {\tiny \'Algebra / \'Algebra II \hfill Segundo Cuatrimestre 2020}

%\maketitle

\centerline{\Large{Pr\' actico 1}}

\

\centerline{\textsc{Vectores en $\mathbb R^2$ y $\mathbb R^3$}}


\bigbreak

\subsection*{Objetivos}

\begin{itemize}
 \item Aprender las operaciones b\'asicas de $\mathbb R^2$ y $\mathbb R^3$ (suma de vectores, multiplicaci\'on por escalares, producto escalar, calcular normas y \'angulos).
 \item Familiarizarse con los conceptos de ortogonalidad y paralelismo.
 \item Aprender a describir rectas y planos de forma imp\'icita y param\'etrica.
\end{itemize}

\

\section*{Ejercicios}

Los ejercicios con el s\'imbolo $\textcircled{a}$ tiene una ayuda al final del archivo para que recurran a ella despu\'es de pensar un poco.


%============================================================
\subsection*{Vectores y producto escalar}
%============================================================

\begin{enumerate}


\item Dados $v = (-1, 2, 0)$, $w = (2,-3,-1)$ y $u = (1,-1,1)$, calcular:
\begin{enumerate}
% 	\item $v + w$,
% 	\item $v - w$,
	\item $2v + 3w -5u$,
	\item $5(v+w)$,
	\item $5v + 5w$ (y verificar que es igual al vector de arriba).
\end{enumerate}

\

\item Calcular los siguientes productos escalares. %\langle v , w  \rangle
\begin{enumerate}
  \item $\langle (-1, 2, 0) ,(2,-3,-1) \rangle$,
%   \item  $\langle (2,4,-3,-1),(1,-1,2, 1) \rangle$,
  \item  $\langle (4,-1),(-1,2) \rangle$.
\end{enumerate}

\

\item Dados $v = (-1, 2, 0)$, $w = (2,-3,-1)$  y $u = (1,-1,1)$, verificar que:
\begin{equation*}
	\langle 2v + 3w , -u   \rangle = -2\langle v ,u \rangle -3 \langle w , u  \rangle
\end{equation*}

\

\item Probar  que
\begin{enumerate}
	\item $(2,3,-1)$ y $(1, -2, -4)$ son ortogonales.
	\item $(2,-1)$ y $(1,2)$ son ortogonales. Dibujar en el plano.
\end{enumerate}
\

\item Encontrar
\begin{enumerate}
	\item un vector no nulo ortogonal  a $(3,-4)$,
	\item un vector no nulo ortogonal a $(2,-1,4)$,
	\item un vector no nulo ortogonal a $(2,-1,4)$ y $(0,1,-1)$.
\end{enumerate}

\



\
\item Encontrar la longitud de los vectores.
\begin{align*}
&(a) \ (2,3), && (b) \ (t,t^2), & (c) \ (\cos\phi,\operatorname{sen}\phi).
\end{align*}

\

\item Calcular $\langle v , w  \rangle$ y el {\'a}ngulo entre $v$ y $w$  para los siguientes vectores.
\begin{align*}
&(a) \ v=(2,2), w=(1,0), &&  (b) \  v=(-5,3,1), w=(2,-4,-7).
\end{align*}

\

\item Recordar los vectores $e_1$, $e_2$ y $e_3$ dados en la p\'agina 12 del apunte. Sea $v=(x_1,x_2,x_3)\in\mathbb{R}^3$.  Verificar que
$$v=x_1e_1+x_2e_2+x_3e_3=\langle v,e_1\rangle e_1+\langle v,e_2\rangle e_2+\langle v,e_3\rangle e_3.$$

\

\item Probar, usando s\'olo las propiedades \textbf{P1}, \textbf{P2}, y \textbf{P3} del producto escalar, que dados $v, w, u \in \mathbb R^n$ y $\lambda_1, \lambda_2 \in \mathbb R$,
\begin{enumerate}
	\item se cumple:
	\begin{equation*}
	\langle \lambda_1 v + \lambda_2 w , u  \rangle =  \lambda_1\langle v , u  \rangle +   \lambda_2\langle w , u  \rangle.
	\end{equation*}
	\item Si $\langle v , w  \rangle =0$, es decir si $v$ y $w$ son ortogonales,  entonces
	\begin{equation*}
		\langle \lambda_1 v + \lambda_2 w ,  \lambda_1 v + \lambda_2 w   \rangle =
		\lambda_1^2 \langle  v ,  v  \rangle + \lambda_2^2 \langle w,w  \rangle.
	\end{equation*}
\end{enumerate}


\


\item Dados $v, w\in \mathbb R^n$, probar que si  $\langle v , w  \rangle =0$, es decir si $v$ y $w$ son ortogonales,  entonces
	\begin{equation*}
	||v + w||^2 = ||v||^2 + ||w||^2.
	\end{equation*}
	?`Cu\'al es el nombre con que se conoce este resultado en $\mathbb R^2$?
	

\

\item\label{Schwarz} $\textcircled{a}$ Sean $v,w\in \mathbb R^2$, probar usando  solo la definici\'on explícita del producto escalar en $\mathbb R^2$ que
\begin{equation*}
	|\langle v , w  \rangle| \le ||v||\,||w|| \qquad \text{(Desigualdad de Schwarz).}
\end{equation*}

\end{enumerate}
%============================================================
\subsection*{Rectas y planos}
%============================================================

\


\begin{enumerate}[resume]

\item En  cada uno de los siguientes casos determinar si los
vectores  $\overrightarrow{vw}$ y $\overrightarrow{xy}$ son
equivalentes y/o paralelos.
\begin{enumerate}
\item   $v=(1,-1)$,  $w=(4,3)$, $x=(-1,5)$, $y=(5,2)$.
\item   $v=(1,-1,5)$,  $w=(-2,3,-4)$,  $x=(3,1,1)$,  $y=(-3,9,-17)$.
\end{enumerate}

\

\item Sea $R_1$ la recta que pasa por $p_1=(2,0)$ y es ortogonal a $(1,3)$.
\begin{enumerate}
 \item Dar la descripci\'on param{\'e}trica e impl{\'\i}cita de $R_1$.
 \item Graficar en el plano a $R_1$.
 \item Dar un punto $p$ por el que pase $R_1$ distinto a $p_1$.
 \item Verificar si $p+p_1$ y $-p$ pertenecen a $R_1$
\end{enumerate}

\

\item Repetir el ejercicio anterior con las siguientes rectas.
\begin{enumerate}
	\item
	$R_2$: recta que pasa por $p_2=(0,0)$ y es ortogonal a $(1,3)$.
	\item
	$R_3$: recta que pasa por $p_3=(1,0)$ y es paralela a $R_1$.
% 	\item
% 	$R_4$: recta que pasa por los puntos $(-1,5,4)$ y $(0,3,-2)$.
\end{enumerate}

\

\item Calcular, num\'erica y graficamente, las intersecciones $R_1\cap R_2$ y $R_1\cap R_3$.

\

\item Sea $v_0=(2,-1,1)$.
\begin{enumerate}
	\item Describir param{\'e}tricamente el conjunto
	$P_1=\{w\in\mathbb{ R}^3:\langle v_0 , w  \rangle=0\}$.
	\item Describir param{\'e}tricamente el conjunto
	$P_2=\{w\in\mathbb{ R}^3:\langle v_0 , w  \rangle=1\}$.
	\item ?`Qu\'e relaci\'on hay entre $P_1$ y $P_2$?
\end{enumerate}

\


\item\label{ej-planos} Escribir la ecuaci{\'o}n param\'etrica  y la ecuaci{\'o}n normal de los siguientes planos.
\begin{enumerate}
	\item $\pi_1$: el plano que pasa por $(0,0,0)$, $(1,1,0)$, $(1,-2,0)$.
	\item $\pi_2$: el plano que pasa por $(1,2,-2)$ y es perpendicular a la
	recta que pasa por $(2,1,-1)$, $(3,-2,1)$.
	\item\label{ej-planos-c}  $\pi_3=\{w\in\mathbb{R}^3: w=s(1,2,0)+t(2,0,1)+(1,0,0);\,s,t\in \mathbb R\}$.
\end{enumerate}

\


\item ¿Cu\'ales de las siguientes rectas cortan al plano $\pi_3$ del  ejercicio \eqref{ej-planos-c}?
Describir la intersecci{\'o}n en cada caso.
\begin{align*}
&(a) \ \{w: w=(3,2,1)+t(1,1,1)\}, && (b) \  \{w: w=(1,-1,1)+t(1,2,-1)\}, \\
&(c)\  \{w: w=(-1,0,-1)+t(1,2,-1)\}, && (d) \  \{w: w=(1,-2,1)+t(2,-1,1)\}.
\end{align*}

\

\item\label{rectas como subespacio} Sea $L=\{(x,y)\in\mathbb{R}^2 : ax+by=c\}$ una recta en $\mathbb{R}^2$. Sean $p$ y $q$ dos puntos por los que pasa $L$.
\begin{enumerate}
 \item ?`Para qu\'e valores de $c$ puede asegurar que $(0,0)\in L$?
 \item ?`Para qu\'e valores de $c$ puede asegurar que $\lambda q\in L$? donde $\lambda\in\mathbb{R}$.
 \item ?`Para qu\'e valores de $c$ puede asegurar que $p+q\in L$?
\end{enumerate}


\

\item Sea $L$ una recta en $\mathbb{R}^2$. Probar que $L$ pasa por $(0,0)$ si y s\'olo si pasa por $p+\lambda q$ para todo par de puntos $p$ y $q$ de $L$ y para todo $\lambda\in\mathbb{R}$.
\end{enumerate}

\

\section*{Ayudas}
\eqref{Schwarz} Elevar al cuadrado y aplicar la definici\'on.
\end{document}
