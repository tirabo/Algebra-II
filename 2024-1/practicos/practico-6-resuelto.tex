
\chapter{Soluciones\\Álgebra  II -- Año 2024/1 -- FAMAF}\label{practico-6}


\begin{enumerate}[topsep=6pt, itemsep=.4cm]

    
    \item\label{sub Rn} Decidir si los siguientes subconjuntos de $\mathbb{R}^3$ son subespacios vectoriales.
        \begin{enumerate}
            \item\label{sub Rn 1} $A=\{(x_1, x_2 ,x_3) \in \mathbb{R}^3 \ : \ x_1 + x_2 + x_3=1\}$.
            \item\label{sub Rn 0} $B=\{(x_1, x_2 ,x_3) \in \mathbb{R}^3 \ : \ x_1 + x_2 + x_3=0\}$.
            \item\label{sub Rn geq} $C=\{(x_1, x_2 ,x_3) \in \mathbb{R}^3 \ : \ x_1 + x_2 + x_3 \geq 0\}$.
            \item\label{sub Rn 1 30} $D=\{(x_1, x_2 ,x_3) \in \mathbb{R}^3 \ : \ x_3=0\}$.
            \item\label{sub Rn cup} $B\cup D$.
            \item\label{sub Rn cap} $B\cap D$.
            \item\label{sub Rn q} $G=\{(x_1, x_2 ,x_3) \in \mathbb{R}^3 \ :\ x_1, x_2, x_3\in\mathbb{Q}\}$.
        \end{enumerate}
        
    \rta 

    \ref{sub Rn 1} No es subespacio vectorial. Por ejemplo $(1,0,0)$ y $(0,1,0)$ pertenecen a $A$, pero $(1,0,0)+(0,1,0)=(1,1,0)$ no pertenece a $A$.

    \ref{sub Rn 0} Es subespacio vectorial. En efecto, si $(x_1, x_2 ,x_3)$ y $(y_1, y_2 ,y_3)$ pertenecen a $B$ y $\lambda,\mu\in\mathbb{R}$, entonces
    \begin{align*}
        \lambda(x_1, x_2 ,x_3)+\mu(y_1, y_2 ,y_3)&=(\lambda x_1+\mu y_1, \lambda x_2+\mu y_2, \lambda x_3+\mu y_3)\\
        &=(\lambda x_1+\mu y_1)+(\lambda x_2+\mu y_2)+(\lambda x_3+\mu y_3)\\
        &=\lambda(x_1+ x_2 + x_3)+\mu(y_1+ y_2 + y_3)\\
        &=\lambda\cdot 0+\mu\cdot 0=0.
    \end{align*}

    \ref{sub Rn geq} No es subespacio vectorial. Por ejemplo $(1,0,0) \in C$ pero $(-1)(1,0,0) = (-1,0,0) \not\in C$, pues $-1+0+0<0$.

    \ref{sub Rn 1 30} Es subespacio vectorial. En efecto, si $(x_1, x_2 ,0)$ y $(y_1, y_2 ,0)$ pertenecen a $D$ y $\lambda,\mu\in\mathbb{R}$, entonces
    \begin{align*}
        \lambda(x_1, x_2 ,0)+\mu(y_1, y_2 ,0)&=(\lambda x_1+\mu y_1, \lambda x_2+\mu y_2, 0) \in D.
    \end{align*}

    \ref{sub Rn cup} No es subespacio vectorial. Por ejemplo $(1,0,-1) \in B$ y $(0,1,0)$ pertenecen a $B\cup D$, pero $(1,0,-1)+(0,1,0)=(1,1,-1)$ no pertenece a $B\cup D$, pues $(1,1,-1) \not\in B$ y $(1,1,-1) \not\in D$.

    \ref{sub Rn cap} Es subespacio vectorial
    \begin{align*}
        B \cap D &= \{(x_1, x_2 ,x_3) \in \mathbb{R}^3 \ : \ x_1 + x_2 + x_3=0 \text{ y } x_3=0\} \\&= \{(x_1, x_2 ,0) \in \mathbb{R}^3 \ : \ x_1 + x_2 =0\}.    
    \end{align*}
    Luego, si $(x_1, x_2 ,0)$ y $(y_1, y_2 ,0)$ pertenecen a $B\cap D$ y $\lambda,\mu\in\mathbb{R}$, entonces
    \begin{align*}
        \lambda(x_1, x_2 ,0)+\mu(y_1, y_2 ,0)&=(\lambda x_1+\mu y_1, \lambda x_2+\mu y_2, 0) \in B\cap D,
    \end{align*}
    pues $\lambda x_1+\mu y_1 + \lambda x_2+\mu y_2 = \lambda(x_1+ x_2) + \mu(y_1+ y_2) = \lambda(x_1+ x_2) + \mu(y_1+ y_2) = \lambda 0 + \mu 0 = 0$.

    \ref{sub Rn q} No es subespacio vectorial. Por ejemplo $(1,0,0)$ pertenece a $G$, pero $\sqrt{2}(1,0,0)=(\sqrt{2},0 ,0)$ no pertenece a $G$.


    \qed     

    \end{enumerate}

    
    \begin{enumerate}[resume, topsep=6pt, itemsep=.4cm]
    
    \item\label{sub matrices} Decidir en cada caso si el conjunto dado es un subespacio vectorial de $M_{n\times n}(\mathbb{K})$.
    \begin{enumerate}
        \item\label{sub matrices invertibles} El conjunto de matrices  invertibles.
        \item\label{sub matrices AB} El conjunto de matrices $A$ tales que $AB = BA$, donde $B$ es una matriz fija.
        \item\label{sub matrices triangulares} El conjunto de matrices triangulares superiores.
    \end{enumerate}
    
    \rta 

    \ref{sub matrices invertibles} No es subespacio vectorial. Por ejemplo, $\Id_n$ y $-\Id_n$ son matrices invertibles,  pero $\Id_n+(-\Id_n)=0$ no es invertible.

    \vskip .2cm
    \ref{sub matrices AB} Es subespacio vectorial. En efecto, si $A$ y $A'$ pertenecen al conjunto y $\lambda,\mu\in\mathbb{R}$, entonces
    \begin{align*}
        (\lambda A+\mu A')B&=\lambda AB+\mu A'B&&\\
        &=\lambda BA+\mu BA'&&\text{(por hipótesis)}\\
        &=(\lambda A+\mu A')B.
    \end{align*}
    Luego $\lambda A+\mu A'$ pertenece al conjunto.

    \vskip .2cm
    \ref{sub matrices triangulares} Es subespacio vectorial. En efecto, si $A$ y $A'$ son matrices triangulares superiores y $\lambda,\mu\in\mathbb{R}$, entonces
    \begin{align*}
        \lambda A+\mu A' &=\lambda\begin{bmatrix}
            a_{11} & a_{12} & \cdots & a_{1n} \\
            0 & a_{22} & \cdots & a_{2n} \\
            \vdots & \vdots & \ddots & \vdots \\
            0 & 0 & \cdots & a_{nn} \\
        \end{bmatrix}+\mu\begin{bmatrix}
            a'_{11} & a'_{12} & \cdots & a'_{1n} \\
            0 & a'_{22} & \cdots
            & a'_{2n} \\
            \vdots & \vdots & \ddots & \vdots \\
            0 & 0 & \cdots & a'_{nn} \\
        \end{bmatrix}\\
        &=\begin{bmatrix}
            \lambda a_{11} & \lambda a_{12} & \cdots & \lambda a_{1n} \\
            0 & \lambda a_{22} & \cdots & \lambda a_{2n} \\
            \vdots & \vdots & \ddots & \vdots \\
            0 & 0 & \cdots & \lambda a_{nn} \\
        \end{bmatrix}+\begin{bmatrix}
            \mu a'_{11} & \mu a'_{12} & \cdots & \mu a'_{1n} \\
            0 & \mu a'_{22} & \cdots & \mu a'_{2n} \\
            \vdots & \vdots & \ddots & \vdots \\
            0 & 0 & \cdots & \mu a'_{nn} \\
        \end{bmatrix}\\
        &=\begin{bmatrix}
            \lambda a_{11}+\mu a'_{11} & \lambda a_{12}+\mu a'_{12} & \cdots & \lambda a_{1n}+\mu a'_{1n} \\
            0 & \lambda a_{22}+\mu a'_{22} & \cdots & \lambda a_{2n}+\mu a'_{2n} \\
            \vdots & \vdots & \ddots & \vdots \\
            0 & 0 & \cdots & \lambda a_{nn}+\mu a'_{nn} \\
        \end{bmatrix}.
    \end{align*}
    Luego $\lambda A+\mu A'$ es una matriz triangular superior.

    \qed     
    
    
    \item\label{rectas} Sea $L$ una recta en $\mathbb{R}^2$. Dar una condición necesaria y suficiente para que $L$ sea un subespacio vectorial de $\mathbb{R}^2$.
    
    
    \rta Una recta en $\mathbb{R}^2$ es un subespacio vectorial si y sólo si pasa por el origen. 
    
    La ecuación general de la recta en el plano es $ax+by=c$ con $a,b,c\in\mathbb{R}$ y $a,b$ no ambos nulos.

    ($\Rightarrow$) Si $L$ es un subespacio vectorial, entonces $(0,0) \in L$,  es decir la recta pasa por el origen. Además, como  $0 = a\cdot 0 + b \cdot 0= c$, la ecuación de la recta es  $ax+by=0$.

    ($\Leftarrow$) Si la recta pasa por el origen, entonces $0 = a\cdot 0 + b \cdot 0= c$, es decir la ecuación de la recta es  $ax+by=0$. Luego, si $(x_1,y_1)$ y $(x_2,y_2)$ pertenecen a la recta y $\lambda,\mu\in\mathbb{R}$, entonces
    \begin{align*}
        a(\lambda x_1+\mu x_2)+b(\lambda y_1+\mu y_2)&=\lambda(ax_1+by_1)+\mu(ax_2+by_2)\\
        &=\lambda\cdot 0+\mu\cdot 0=0.
    \end{align*}
    Luego $\lambda(x_1,y_1)+\mu(x_2,y_2)$ pertenece a la recta y por lo tanto la recta es un subespacio vectorial.

    \qed     
    
    \item Sean $V$ un $\mathbb{K}$-espacio vectorial, $v\in V$ no nulo y $\lambda,\mu\in\mathbb{K}$ tales que $\lambda v=\mu v$. Probar que $\lambda=\mu$.
    
    
    \rta Si $\lambda v=\mu v$, entonces $(\lambda-\mu)v=0$. Supongamos que $\lambda-\mu\neq 0$, entonces
    \begin{align*}
        v&= 1 \cdot v &&\text{(axioma P1 de esp. vectoriales)}\\
        &=(\lambda-\mu)^{-1}(\lambda-\mu)v&&\\
        &=(\lambda-\mu)^{-1}0&&\text{(por hipótesis)}\\
        &=0.&&\text{(demostrado en la teórica: $0\cdot v = 0$)}
    \end{align*}
    Concluimos que $v=0$, lo cual contradice la hipótesis. El absurdo vino de suponer que $\lambda-\mu\neq 0$, luego $\lambda=\mu$.

    \qed     
    
    \item Sean $W_1, W_2$ subespacios de un espacio vectorial $V$. Probar que $W_1 \cup W_2$ es un subespacio  de $V$ si y sólo si $W_1 \subseteq W_2$ o $W_2 \subseteq W_1$.
        
    
    \rta 

    ($\Rightarrow$)  Si $W_1 \subseteq W_2$ o $W_2 \subseteq W_1$ no hay nada que demostrar. Supongamos entonces que $W_1 \not\subseteq W_2$ y $W_2 \not\subseteq W_1$. Entonces existen $w_1\in W_1$ tal que $w_1\not\in W_2$ y $w_2\in W_2$ tal que $w_2\not\in W_1$. Como $w_1\in W_1$ y $w_2\in W_2$, entonces $w_1+w_2\in W_1\cup W_2$. Como $W_1 \cup W_2$ es un subespacio  de $V$, entonces $w_1+w_2\in W_1\cup W_2$ y por lo tanto $w_1+w_2\in W_1$ o $w_1+w_2\in W_2$. Supongamos que $w_1+w_2\in W_1$, entonces $w_2=w_1+w_2-w_1\in W_1$, lo cual es absurdo.Análogamente,  si $w_1+w_2\in W_2$, entonces $w_1\in W_2$, lo cual es absurdo. El absurdo vino de suponer que $W_1 \not\subseteq W_2$ y $W_2 \not\subseteq W_1$, luego $W_1 \subseteq W_2$ o $W_2 \subseteq W_1$.

    ($\Leftarrow$) Supongamos que $W_1 \subseteq W_2$. Entonces $W_1 \cup W_2 = W_2$ y por lo tanto $W_1 \cup W_2$ es un subespacio  de $V$. Análogamente se demuestra que si $W_2 \subseteq W_1$, entonces $W_1 \cup W_2$ es un subespacio  de $V$.

    \qed     
        
    \item Sean $u=(1,1)$, $v=(1,0)$, $w=(0,1)$ y $z=(3,4)$ vectores de $\mathbb{R}^2$.
    \begin{enumerate}
    \item\label{comb-lin-u-v-w} Escribir $z$ como combinación lineal de $u,v$ y $w$, con coeficientes todos no nulos.
    \item\label{comb-lin-u-v} Escribir $z$ como combinación lineal de $u$ y $v$.
    \item\label{comb-lin-u-w} Escribir $z$ como combinación lineal de $u$ y $w$.
    \item\label{comb-lin-v-w}Escribir $z$ como combinación lineal de $v$ y $w$.
    \end{enumerate}
    
    \rta En general tenemos que resolver la ecuación $z=\lambda u+\mu v+\nu w$, bajo ciertas condiciones sobre $\lambda,\mu,\nu$. En cada caso, las condiciones son distintas. Si escribimos en coordenadas la ecuación es
    \begin{align*}
        (3,4)&=\lambda (1,1)+\mu (1,0)+\nu (0,1)\\
        &=(\lambda+\mu,\lambda+\nu).    \tag{*}
    \end{align*}
    El sistema es sencillo de resolver, pues la segunda coordenada nos dice que $\lambda+\nu=4$, es decir $\nu=4-\lambda$. Reemplazando en la primera coordenada obtenemos $\lambda+\mu=3$, es decir $\mu=3-\lambda$. Por lo tanto, $\lambda$ es libre y
    \begin{equation*}
        z=\lambda u+\mu v+\nu w=\lambda (1,1)+(3-\lambda) (1,0)+(4-\lambda) (0,1).
    \end{equation*}
    
    \ref{comb-lin-u-v-w} Si $\lambda=1$, entonces $\mu=2$ y $\nu=3$ y por lo tanto
    \begin{equation*}
        z=1\cdot u+2\cdot v+3\cdot w.
    \end{equation*}

    \ref{comb-lin-u-v} Si $\lambda=4$, entonces $\mu=-1$ y $\nu=0$ y por lo tanto
    \begin{equation*}
        z=4\cdot u-1\cdot v.
    \end{equation*}

    \ref{comb-lin-u-w} Si $\lambda=3$, entonces $\mu=0$ y $\nu=1$ y por lo tanto
    \begin{equation*}
        z=3\cdot u+1\cdot w.
    \end{equation*}

    \ref{comb-lin-v-w} Si $\lambda=2$, entonces $\mu=1$ y $\nu=2$ y por lo tanto
    \begin{equation*}
        z=2\cdot v+2\cdot w.
    \end{equation*}
    \qed     
    
    \item Sean $p(x)=(x-1)(x+2)$, $q(x)=x^2-1$ y $r(x)=x(x^2-1)$ en $\mathbb{R}[x]$.
        \begin{enumerate}
        \item\label{comb-lineal-pol-a} Describir en forma implícita todos los polinomios de grado menor o igual que $3$ que son combinación lineal de $p,q$ y $r$.
        \item\label{comb-lineal-pol-b} Elegir $a$ tal que el polinomio $x$ se pueda escribir como combinación lineal de $p,q$ y $2x^2+a$.
        \end{enumerate}
    
    \rta 

    \ref{comb-lineal-pol-a} Escribamos la versión expandida de $p$, $q$ y $r$:
    \begin{align*}
        p(x)&=x^2+x-2,\\
        q(x)&=x^2-1,\\
        r(x)&=x^3-x.                
    \end{align*}
    \begin{comment}
    Debemos encontrar el subespacio generado por estos tres polinomios. Primero encontraremos una base del subespacio en término de los generadores canónicos  ($x^n$ con $n\in\mathbb{N}_0$). 
    \begin{align*}
        &\begin{bmatrix}
            0 & 1 & 1 & -2 \\
            0 & 1 & 0 & -1 \\
            1 & 0 & -1 & 0 \\
        \end{bmatrix}
        \stackrel{F_2-F_1}{\longrightarrow}
        \begin{bmatrix}
            0 & 1 & 1 & -2 \\
            0 & 0 & -1 & 1 \\
            1 & 0 & -1 & 0 \\
        \end{bmatrix} \\
        &\underset{F_3-F_2}{\stackrel{F_1+F_2}{\longrightarrow}}
        \begin{bmatrix}
            0 & 1 & 0 & -1 \\
            0 & 0 & -1 & 1 \\
            1 & 0 & 0 & -1 \\
        \end{bmatrix} \stackrel{-F_2}{\longrightarrow}
        \begin{bmatrix}
            0 & 1 & 0 & -1 \\
            0 & 0 & 1 & -1 \\
            1 & 0 & 0 & -1 \\
        \end{bmatrix}.            
    \end{align*}
    Luego  
    $$
    \langle p,q,r\rangle = \langle x^2-1,x-1,x^3-1 \rangle.
    $$
\end{comment}



Ahora, planteemos  la ecuación 
    \begin{equation*}
        \begin{aligned}
        ax^3+bx^2+cx +d &=\lambda p +\mu q +\nu r\\
        &= \lambda(x^2+x-2)+\mu(x^2-1)+\nu(x^3-x)\\
        &= \nu x^3 +(\lambda+\mu)x^2+(\lambda-\nu)x+(-2\lambda-\mu).
        \end{aligned} \tag{*}
    \end{equation*}
    Debemos encontrar todos los $(a,b,c,d)\in\mathbb{R}^4$ tales que existe $\lambda,\mu,\nu\in\mathbb{R}$ que satisfacen la ecuación anterior. Es decir, debemos encontrar todos los $(a,b,c,d)\in\mathbb{R}^4$ tales que existe $\lambda,\mu,\nu\in\mathbb{R}$ que satisfacen el sistema
    \begin{equation*}
        \begin{aligned}
        a&=\nu\\
        b&=\lambda+\mu\\
        c&=-\lambda-\nu\\
        d&=-2\lambda-\mu. 
        \end{aligned} 
    \end{equation*}
    Si consideramos $a,b,c,d$ como constantes y  $\lambda,\mu,\nu$ como incógnitas, entonces el sistema, presentado como matriz aumentada es:

    \begin{align*}
        &\begin{amatrix}{3}
            0 & 0 & 1 & a \\
            1 & 1 & 0 & b \\
            1 & 0 & -1 & c \\
            -2 & -1 & 0 & d
        \end{amatrix}
        \underset{F_4 +2F_2}{\stackrel{F_3-F_2}{\longrightarrow}}
        \begin{amatrix}{3}
            0 & 0 & 1 & a \\
            1 & 1 & 0 & b \\
            0 & -1 & -1 & -b+c \\
            0 & 1 & 0 & 2b+d
        \end{amatrix}\\
        &\underset{F_3+F_4}{\stackrel{F_2-F_4}{\longrightarrow}}
        \begin{amatrix}{3}
            0 & 0 & 1 & a \\
            1 & 0 & 0 & -b-d \\
            0 & 0 & -1 & b+c+d \\
            0 & 1 & 0 & 2b+d
        \end{amatrix}
        \stackrel{F_3+F_1}{\longrightarrow}
        \begin{amatrix}{3}
            0 & 0 & 1 & a \\
            1 & 0 & 0 & -b-d \\
            0 & 0 & 0 & a+b+c+d\\
            0 & 1 & 0 & 2b+d
        \end{amatrix}.
    \end{align*}

    Luego la ecuación (*) solo puede ser satisfecha si y sólo si $a+b+c+d  =0$ por lo tanto, el subespacio de polinomios que obtenemos es
    \begin{equation*}
        \{a x^3 + bx^2 + cx + d: \ a+b+c+d = 0, \ a,b,c,d\in\mathbb{R}\}.
    \end{equation*}
    También lo podríamos describir de la siguiente manera: 
    \begin{equation*}
        \{(-b-c-d) x^3 + bx^2 + cx + d: \ b,c,d\in\mathbb{R}\}.
    \end{equation*}

    \ref{comb-lineal-pol-b} Debemos encontrar $a$ tal que existan $\lambda,\mu,\nu\in\mathbb{R}$ que satisfacen la ecuación
    \begin{align*}
        x&=\lambda p +\mu q +(2x^2+a) \nu \\
        &= \lambda(x^2+x-2)+\mu(x^2-1)+\nu(2x^2+a)\\
        &= (\lambda+\mu+2\nu)x^2+\lambda x+(-2\lambda-\mu-a\nu).
    \end{align*}
    Claramente $\lambda+\mu+2\nu=0$, $\lambda=1$  y $-2\lambda-\mu-a\nu=0$,  en consecuencia
    \begin{align*}
        0&=1+\mu+2\nu\\
        0&=-2-\mu-a\nu,
    \end{align*}
    o, lo que es lo mismo, 
    \begin{align*}
        \mu+2\nu&=-1\\
        -\mu-a\nu&=2.
    \end{align*}
    Planteamos la matriz aumentada correspondiente a este sistema y la escalonamos:
    \begin{align*}
        &\begin{amatrix}{2}
            1 & 2 & -1 \\
            -1 & -a & 2
        \end{amatrix}
        \stackrel{F_2+F_1}{\longrightarrow}
        \begin{amatrix}{2}
            1 & 2 & -1 \\
            0 & 2-a & 1
        \end{amatrix}.
    \end{align*}
    Observamos entonces que si $a=2$ el sistema no tiene solución pues quedaría $\mu \cdot 0 + \nu \cdot 0 = 1$. Cuando $ a \ne 2$, dividimos por $2-a$ y obtenemos $\nu = \frac{1}{2-a}$ y $\mu = -1 - 2\nu = -1 - \frac{2}{2-a} = \frac{a}{2-a}$. Por lo tanto, si $a \ne 2$, el polinomio $x$ se puede escribir como combinación lineal de $p,q$ y $2x^2+a$.  



    \qed     
    
    \item\label{practicos anteriores} Dar un conjunto de generadores para los siguientes subespacios vectoriales.
    \begin{enumerate}
    \item\label{pa-pr2-sistemas-homogeneos} Los conjuntos de soluciones de los sistemas homogéneos del ejercicio \ref{sistemas homogeneos} del Práctico \ref{practico-2}.
    \item\label{pa-pr2-sistemas-con-soluciones} Los conjuntos descriptos en el ejercicio \ref{sistemas con soluciones} del Práctico  \ref{practico-2}.
    \end{enumerate}
    
    \rta 

    \ref{pa-pr2-sistemas-homogeneos} Los sistemas homogéneos del ejercicio \ref{sistemas homogeneos} del Práctico \ref{practico-2} son los tres primeros.

    El primer sistema era 
    $$
    \begin{cases}
        -x - y + 4z = 0\\
        x+3y+8z = 0\\
        x+2y + 5z = 0.
    \end{cases}
    $$
    Este sistema tenía como única solución $x=y=z=0$, luego el conjunto de soluciones es $\{(0,0,0)\}$ y por lo tanto el subespacio generado por las soluciones es $\{0\}$.


    \vskip .2cm

    El segundo era
    $$
    \begin{cases}
        x - 3y + 5z = 0\\
        2x-3y+z = 0\\
        -y + 3z = 0
    \end{cases}
    $$
    Este sistema tiene como soluciones al conjunto  $\{(4t, 3t, t) : t \in \R\} = \{t(4, 3, 1) : t \in \R\}$, y por lo tanto podemos tomar como generador del subespacio al vector $(4, 3, 1)$.

    \vskip .2cm
    El tercero: 
    $$
    \begin{cases}
        x-z+2t = 0\\
        -x+2y-z+2t = 0\\
        -x+y = 0.
    \end{cases}
    $$
    Las soluciones de este sistema eran los vectores del subespacio  $$\{(u - 2v, u - 2v, u, v) : u, v \in \R\} = \{u(1,1,1,0) + v(-2,-2,0,1) : u, v \in \R\}$$
    y por lo tanto podemos tomar como generadores a los vectores $(1,1,1,0)$ y $(-2,-2,0,1)$.

    \vskip .3cm

    \ref{pa-pr2-sistemas-con-soluciones} el primer conjunto era $\{(b_1, b_2, b_3) \in \R^3 : -2b_1 + b_2 + 3b_3 = 0\}$. Podemos despejar una de las variables respecto a las otras, por ejemplo $b_2 = 2b_1 -3b_3$, y en ese caso el conjunto se  puede escribir como
    $$
    \{(b_1, -2b_1 - 3b_3, b_3) : b_1, b_3 \in \R\} = \{b_1(1, -2, 0) + b_3(0, -3, 1) : b_1, b_3 \in \R\}.
    $$
    Por lo tanto, podemos tomar como generadores a los vectores $(1, -2, 0)$ y $(0, -3, 1)$.

    \vskip .4cm

    El segundo conjunto era 
    $$
    \{(b_1,b_2,b_3,b_4)\in \mathbb R^3: \frac12b_1 -\frac12b_2 +b_3 = 0 \wedge -\frac12b_1 -\frac12b_2+b_4=0\}.
    $$
    Plantemos el sistema correspondiente:
    $$
    \begin{cases}
        \frac12b_1 -\frac12b_2 +b_3 = 0\\
        -\frac12b_1 -\frac12b_2+b_4=0.
    \end{cases}
    $$
    Resolvamos el sistema:
    \begin{align*}
    &\begin{bmatrix}
        \frac12 & -\frac12 & 1 & 0\\
        -\frac12 & -\frac12 & 0 & 1
    \end{bmatrix}
    \underset{2\cdot F_2 }{\stackrel{2\cdot F_1 }{\longrightarrow}}
    \begin{bmatrix}
        1 & -1 & 2 & 0\\
        -1 & -1 & 0 & 2
    \end{bmatrix}
    \stackrel{F_2 + F_1}{\longrightarrow}
    \begin{bmatrix}
        1 & -1 & 2 & 0\\
        0 & -2 & 2 & 2
    \end{bmatrix}  \\
    &\stackrel{ F_2/(-2)}{\longrightarrow}
    \begin{bmatrix}
        1 & -1 & 2 & 0\\
        0 & 1 & -1 & -1
    \end{bmatrix}
    \stackrel{F_1 + F_2}{\longrightarrow}
    \begin{bmatrix}
        1 & 0 & 1 & -1\\
        0 & 1 & -1 & -1
    \end{bmatrix}.
    \end{align*}
    Luego, el conjunto de soluciones es
    $$
    \{(b_1,b_2,b_3,b_4)\in \R^4: b_1=-b_3+b_4 \wedge b_2=b_3+b_4\}.
    $$
    Es decir, el conjunto de soluciones es
    $$
    \{(-b_3+b_4,b_3+b_4,b_3,b_4)\in \R^4: b_3,b_4 \in \R\}.
    $$
    Escrito de otra forma, el conjunto de soluciones es
    $$
    \{b_3(-1,1,1,0)+b_4(1,1,0,1)\in \R^4: b_3,b_4 \in \R\}.
    $$
    Por lo tanto, podemos tomar como generadores a los vectores $(-1,1,1,0)$ y $(1,1,0,1)$.

    \vskip .4cm

    El tercer conjunto era $\R^3$ y por lo tanto podemos tomar los generadores canónicos $(1,0,0)$, $(0,1,0)$ y $(0,0,1)$.

    \qed     
    
    \item\label{caracterizar}  En cada caso, caracterizar con ecuaciones al subespacio vectorial dado por generadores.
    \begin{enumerate}
    \item\label{caracterizar-a} ${\left\langle(1,0,3),(0,1,-2)\right\rangle}\subseteq \mathbb{R}^3$.
    \item\label{caracterizar-b} ${\left\langle(1,2,0,1),(0,-1,-1,0),(2,3,-1,4)\right\rangle}\subseteq \mathbb{R}^4$.
    \end{enumerate}
    
    \rta 

    \ref{caracterizar-a} El subespacio es el conjunto de combinaciones lineales de $(1,0,3)$ y $(0,1,-2)$, es decir
    \begin{align*}
        \{\lambda(1,0,3)&+\mu(0,1,-2): \lambda,\mu\in\R\} = \{(\lambda, \mu, 3\lambda -2\mu) : \lambda,\mu\in\R\} \\
        &=  \{(x,y,z) : x=\lambda, y=\mu, z=3\lambda -2, \,\mu\lambda,\mu\in\R\} . 
    \end{align*}
    Planteamos el sistema correspondiente:
    $$
    \begin{cases}
        \lambda = x\\
        \mu=y\\
        3\lambda -2\mu =z.
    \end{cases}
    $$
    Resolvamos el sistema:
    $$
    \begin{amatrix}{2}
        1 & 0 & x\\
        0 & 1 & y\\
        3 & -2 & z
    \end{amatrix}
    \stackrel{F_3-3F_1}{\longrightarrow}
    \begin{amatrix}{2}
        1 & 0 & x\\
        0 & 1 & y\\
        0 & -2 & z-3x
    \end{amatrix}
    \stackrel{F_3+2F_2}{\longrightarrow}
    \begin{amatrix}{2}
        1 & 0 & x\\
        0 & 1 & y\\
        0 & 0 & z-3x+2y
    \end{amatrix}.
    $$
    Luego, el subespacio se caracteriza implícitamente de la siguiente manera:
    $$
    \{(x,y,z)\in \R^3: z-3x+2y=0\}.
    $$

    \vskip .4cm

    \ref{caracterizar-b} El subespacio es el conjunto de combinaciones lineales de $(1,2,0,1)$, $(0,-1,-1,0)$ y $(2,3,-1,4)$, es decir
    \begin{align*}
        &\{\lambda(1,2,0,1)+\mu(0,-1,-1,0)+\nu(2,3,-1,4): \lambda,\mu,\nu\in\R\} =\\
        &= \{(\lambda + 2\nu, 2\lambda-\mu+3\nu, -\mu-\nu, \lambda+4\nu) : \lambda,\mu,\nu\in\R\} \\
        &= \{(x,y,z,t) : x=\lambda + 2\nu, \\
        &\qquad\qquad\qquad y=2\lambda-\mu+3\nu, z=-\mu-\nu, t=\lambda+4\nu, \,\mu,\lambda,\nu\in\R\} .
    \end{align*}
    Planteamos el sistema correspondiente:
    $$
    \begin{cases}
        \lambda + 2\nu = x\\
        2\lambda-\mu+3\nu = y\\
        -\mu-\nu = z\\
        \lambda+4\nu = t
    \end{cases}
    $$
    Resolvamos el sistema:
    $$
    \begin{amatrix}{3}
        1 & 0 & 2 & x\\
        2 & -1 & 3 & y\\
        0 & -1 & -1 & z\\
        1 & 0 & 4 & t
    \end{amatrix}
    \stackrel{F_2-2F_1}{\longrightarrow}
    \begin{amatrix}{3}
        1 & 0 & 2 & x\\
        0 & -1 & -1 & y-2x\\
        0 & -1 & -1 & z\\
        0 & 0 & 2 & t-x
    \end{amatrix}
    \stackrel{F_3-F_2}{\longrightarrow}
    \begin{amatrix}{3}
        1 & 0 & 2 & x\\
        0 & -1 & -1 & y-2x\\
        0 & 0 & 0 & 2x-y+z\\
        0 & 0 & 2 & t-x
    \end{amatrix}.
    $$
    En  realidad, no es necesario resolver el sistema completamente: la última matriz nos dice que el sistema tiene solución si y sólo si $2x-y+z=0$. Luego, el subespacio se caracteriza implícitamente de la siguiente manera:
    $$
    \{(x,y,z,t)\in \R^4: 2x-y+z=0\}.
    $$
    \qed     
    
    \item\label{son LI} En cada caso, determinar si el subconjunto indicado es linealmente independiente.
    \begin{enumerate}
        \item\label{son LI-a} $\{ (1,0,-1), (1,2,1), (0,-3,2) \}\subseteq \mathbb{R}^3$.
        \vskip .3cm
        \item\label{son LI-b} $\left\{  \begin{bmatrix} 1 & 0 & 2 \\ 0 & -1 & -3 \\ \end{bmatrix}, \quad
        \begin{bmatrix} 1 & 0 & 1 \\ -2 & 1 & 0 \\ \end{bmatrix}, \quad
        \begin{bmatrix} 1 & 2 & 3 \\ 3 & 2 & 1 \\ \end{bmatrix} \right\}\subseteq M_{2\times 3}(\mathbb{R})$.
    \end{enumerate}
    
    \rta para determinar si un conjunto de vectores es LI debemos plantear la ecuación $\lambda_1 v_1+\cdots+\lambda_n v_n=0$ y resolverla. Si la única solución es $\lambda_1=\cdots=\lambda_n=0$, entonces el conjunto es LI. En caso contrario, el conjunto es LD.

    \ref{son LI-a} Debemos resolver la ecuación
    $$
    \lambda_1 (1,0,-1)+\lambda_2 (1,2,1)+\lambda_3 (0,-3,2)=(0,0,0),
    $$
    o sea
    $$
    (\lambda_1+\lambda_2,\,2\lambda_2-3\lambda_3,\,-\lambda_1+\lambda_2+2\lambda_3)=(0,0,0).
    $$
    Es decir, debemos resolver el sistema
    $$
    \begin{cases}
        \lambda_1 +\lambda_2 = 0\\
        2\lambda_2 -3\lambda_3 = 0\\
        -\lambda_1 +\lambda_2 +2\lambda_3 = 0.
    \end{cases}
    $$
    Resolvamos el sistema:
    \begin{align*}
    &\begin{bmatrix}
        1 & 1 & 0\\
        0 & 2 & -3\\
        -1 & 1 & 2
    \end{bmatrix}
    \stackrel{F_3+F_1}{\longrightarrow}
    \begin{bmatrix}
        1 & 1 & 0 \\
        0 & 2 & -3 \\
        0 & 2 & 2 
    \end{bmatrix}
    \stackrel{F_3-F_2}{\longrightarrow}
    \begin{bmatrix}
        1 & 1 & 0 \\
        0 & 2 & -3 \\
        0 & 0 & 5 
    \end{bmatrix} \\
    &\stackrel{F_2/2}{\longrightarrow}
    \begin{bmatrix}
        1 & 1 & 0 \\
        0 & 1 & -\frac32 \\
        0 & 0 & 5
    \end{bmatrix}
    \stackrel{F_1-F_2}{\longrightarrow}
    \begin{bmatrix}
        1 & 0 & \frac32 \\
        0 & 1 & -\frac32 \\
        0 & 0 & 5
    \end{bmatrix}
    \stackrel{F_3/5}{\longrightarrow}
    \begin{bmatrix}
        1 & 0 & \frac32 \\
        0 & 1 & -\frac32 \\
        0 & 0 & 1
    \end{bmatrix} \\
    &\underset{F_2+\frac32 F_3}{\stackrel{F_1-\frac32 F_3}{\longrightarrow}}
    \begin{bmatrix}
        1 & 0 & 0 \\
        0 & 1 & 0 \\
        0 & 0 & 1
    \end{bmatrix}.
    \end{align*}
    Luego  como la única solución es la trivial los vectores son LI.

    \vskip .4cm

    \ref{son LI-b} Debemos resolver la ecuación
    $$
    \lambda_1 \begin{bmatrix} 1 & 0 & 2 \\ 0 & -1 & -3 \\ \end{bmatrix}+\lambda_2 \begin{bmatrix} 1 & 0 & 1 \\ -2 & 1 & 0 \\ \end{bmatrix}+\lambda_3 \begin{bmatrix} 1 & 2 & 3 \\ 3 & 2 & 1 \\ \end{bmatrix}=\begin{bmatrix} 0 & 0 & 0 \\ 0 & 0 & 0 \\ \end{bmatrix},
    $$
    o sea
    $$
    \begin{bmatrix} 
        \lambda_1 +\lambda_2 +\lambda_3 & 2\lambda_3 & 2\lambda_1 +\lambda_2+3\lambda_3 
        \\ -2\lambda_2 +3\lambda_3 &-\lambda_1 + \lambda_2 +2\lambda_3 & -3\lambda_1 +\lambda_3 \\ 
    \end{bmatrix}=
    \begin{bmatrix} 0 & 0 & 0 \\ 0 & 0 & 0 \\ \end{bmatrix}.
    $$
    Por la igualdad de fila $1$, columna $2$, claramente, $\lambda_3=0$. Luego, por la igualdad de fila $2$, columna $3$, $\lambda_1=0$. Finalmente, por la igualdad de fila $1$, columna $1$, $\lambda_2=0$. Es decir, la única solución es la trivial y por lo tanto los vectores son LI.
    





    \qed     
    
    \item Dar un ejemplo de un conjunto de 3 vectores en $\mathbb{R}^3$ que sean LD, y tales que dos cualesquiera de ellos sean LI.
    
    
    \rta tomemos $v_1= (1,0,0)$, $ v_2=(0,1,0)$ y el tercer vector la suma de ambos, es decir, $v_3= (1,1,0)$. Por lo tanto, esos tres vectores son LD ($v_1+v_2-v_3=0$). Ahora bien, si quitamos cualquiera de los vectores, los dos restantes son LI. Por ejemplo, si quitamos $v_1$, entonces $v_2$ y $v_3$ son LI, pues si $\lambda v_2+\mu v_3=0$, entonces $(\mu, \lambda+\mu,0)=(0,0,0)$ y por lo tanto $\mu=0$, luego $\lambda=0$.

    \qed     
    
    \item  Probar que si $\alpha$, $\beta$ y $\gamma$ son vectores LI en el $\mathbb{R}$-espacio vectorial $V$, entonces $\alpha +\beta$, $\alpha +\gamma$ y $\beta +\gamma $ también son LI.
    
    
    \rta debemos plantear la ecuación
    $$
    \lambda_1 (\alpha +\beta)+\lambda_2 (\alpha +\gamma)+\lambda_3 (\beta +\gamma)=0,
    $$
    y ver que la única solución es $\lambda_1=\lambda_2=\lambda_3=0$. La ecuación anterior es equivalente a 
    $$
    (\lambda_1+\lambda_2)\alpha+(\lambda_1+\lambda_3)\beta+(\lambda_2+\lambda_3)\gamma=0.
    $$  
    Como $\alpha$, $\beta$ y $\gamma$ son LI por hipótesis, entonces $\lambda_1+\lambda_2=\lambda_1+\lambda_3=\lambda_2+\lambda_3=0$. Resolvamos el sistema:
    $$
    \begin{bmatrix}
        1 & 1 & 0\\
        1 & 0 & 1\\
        0 & 1 & 1
    \end{bmatrix}
    \stackrel{F_2-F_1}{\longrightarrow}
    \begin{bmatrix}
        1 & 1 & 0\\
        0 & -1 & 1\\
        0 & 1 & 1
    \end{bmatrix}
    \stackrel{F_3+F_2}{\longrightarrow}
    \begin{bmatrix}
        1 & 1 & 0\\
        0 & -1 & 1\\
        0 & 0 & 2
    \end{bmatrix}.
    $$
    Por lo tanto, por la tercera fila, $\lambda_3=0$, por la segunda fila se deduce que $\lambda_2=0$ y por la primera fila se deduce que $\lambda_1=0$. Es decir, la única solución es la trivial y por lo tanto los vectores son LI.

    \qed     
    
    \item Extender, de ser posible, los siguientes conjuntos a una base de los respectivos espacios vectoriales.
    
    \begin{enumerate}
        \item\label{extender-a} Los conjuntos del ejercicio \ref{son LI}.
        \item\label{extender-b} $\{ (1,2,0,0),(1,0,1,0) \}\subseteq\mathbb{R}^4$.
        \item\label{extender-c} $\{ (1,2,1,1),(1,0,1,1),(3,2,3,4)\}\subseteq\mathbb{R}^4$.
    \end{enumerate}
    
    
    \rta 

    \ref{extender-a} Los conjuntos del ejercicio \ref{son LI} son LI. Luego el primer subconjunto es base, pues tiene $3$ elementos de  $\R^3$. 
    
    El segundo subconjunto no es base pues tiene $3$ vectores y $M_{2\times 3}(\mathbb{R})$ tiene dimensión $6$. 
    
    Extendamos entonces $$\left\{  \begin{bmatrix} 1 & 0 & 2 \\ 0 & -1 & -3 \\ \end{bmatrix}, \quad
    \begin{bmatrix} 1 & 0 & 1 \\ -2 & 1 & 0 \\ \end{bmatrix}, \quad
    \begin{bmatrix} 1 & 2 & 3 \\ 3 & 2 & 1 \\ \end{bmatrix} \right\}$$ a una base de $M_{2\times 3}(\mathbb{R})$.
    
    Convendrá mirar a las matrices como vectores de $\R^6$ para poder operar con estos vectores. La forma más obvia de hacerlo es construir  un vector de $6$ coordenadas a partir de cada matriz  con la primera fila y a continuación la segunda. Haciendo esto obtenemos los vectores
    $$
    (1, 0, 2, 0, -1, -3),\; (1, 0, 1, -2, 1, 0),\;  (1, 2, 3, 3, 2, 1).
    $$
    
    %\textbf{Primera solución.} 
    Una forna de extender la base es primero construir una matriz con los vectores fila y hallar la MRF. Los vectores que obtenemos en la MRF generan el mismo subespacio que los vectores originales y los podemos completar a una base de $\R^6$ con vectores canónicos. Hagamos el procedimiento: 
    \begin{align*}
    &\begin{bmatrix}
        1 & 0 & 2 & 0 & -1 & -3\\
        1 & 0 & 1 & -2 & 1 & 0\\
        1 & 2 & 3 & 3 & 2 & 1
    \end{bmatrix}
    \underset{F_3-F_1}{\stackrel{F_2-F_1}{\longrightarrow}}
    \begin{bmatrix}
        1 & 0 & 2 & 0 & -1 & -3\\
        0 & 0 & -1 & -2 & 2 & 3\\
        0 & 2 & 1 & 3 & 3 & 4
    \end{bmatrix} \\
    &\underset{F_3/2}{\stackrel{F_2/(-1)}{\longrightarrow}}
    \begin{bmatrix}
        1 & 0 & 2 & 0 & -1 & -3\\
        0 & 0 & 1 & 2 & -2 & -3\\
        0 & 1 & \frac12 & \frac32 & \frac32 & 2
    \end{bmatrix}
    \underset{F_3-\frac12 F_2}{\stackrel{F_1-2F_2}{\longrightarrow}}
    \begin{bmatrix}
        1 & 0 & 0 & -4 & 3 & 3\\
        0 & 0 & 1 & 2 & -2 & -3\\
        0 & 1 & 0 & \frac12 & \frac52 & \frac72
    \end{bmatrix} \\
    &\stackrel{F_2 \leftrightarrow F_3}{\longrightarrow}
    \begin{bmatrix}
        1 & 0 & 0 & -4 & 3 & 3\\
        0 & 1 & 0 & \frac12 & \frac52 & \frac72\\
        0 & 0 & 1 & 2 & -2 & -3
    \end{bmatrix}.
    \end{align*}
    Luego el conjunto LI
    $$
    \left\{(1, 0, 2, 0, -1, -3), (1, 0, 1, -2, 1, 0),  (1, 2, 3, 3, 2, 1)\right\}
    $$ \
    se puede completar a una base con los vectores canónicos 
    $$(0,0,0,1,0,0),\; (0,0,0,0,1,0),\; (0,0,0,0,0,1).$$
    Volviendo a $M_{2\times 3}(\mathbb{R})$, los vectores canónicos son las matrices
    $$  
    \begin{bmatrix} 0 & 0 & 0 \\ 1 & 0 & 0 \\ \end{bmatrix},\quad
    \begin{bmatrix} 0 & 0 & 0 \\ 0 & 1 & 0 \\ \end{bmatrix},\quad
    \begin{bmatrix} 0 & 0 & 0 \\ 0 & 0 & 1 \\ \end{bmatrix},
    $$
    que completan a una base.

\vskip .3cm

\begin{comment}
\textbf{Segunda solución.} Otra forma de hacerlo es encontrando la forma implícita del subespacio y luego agregar vectores con un criterio que explicaremos más adelante. 

Debemos ahora caracterizar implícitamente los $b_1,\ldots, b_6$ tales que
\begin{multline*}
    \lambda_1(1, 0, 2, 0, -1, -3)+\lambda_2(1, 0, 1, -2, 1, 0)+\lambda_3(1, 2, 3, 3, 2, 1)=(b_1,b_2,b_3,b_4,b_6,b_6).
\end{multline*}
Es decir, debemos resolver el sistema
$$
\begin{cases}
    \lambda_1+\lambda_2+\lambda_3=b_1\\
    2\lambda_3=b_2\\
    2\lambda_1+\lambda_2+3\lambda_3=b_3\\
    -2\lambda_2+3\lambda_3=b_4\\
    -\lambda_1+\lambda_2+2\lambda_3=b_5\\
    -3\lambda_1+\lambda_3=b_6
\end{cases}
$$
Planteamos la matriz ampliada y hacemos Gauss:
\begin{align*}
&\begin{amatrix}{3}
    1 & 1 & 1 & b_1\\   
    0 & 0 & 2 & b_2\\
    2 & 1 & 3 & b_3\\
    0 & -2 & 3 & b_4\\
    -1 & 1 & 2 & b_5\\
    -3 & 0 & 1 & b_6
\end{amatrix}
\underset{F_6+3F_1}{\underset{F_5+F_1}{\stackrel{F_2-2F_1}{\longrightarrow}}}
\begin{amatrix}{3}
    1 & 1 & 1 & b_1\\   
    0 & 0 & 2 & b_2\\
    0 & -1 & 1 & b_3-2b_1\\
    0 & -2 & 3 & b_4\\
    0 & 2 & 3 & b_1+b_5\\
    0 & 3 & 4 & 3b_1+b_6
\end{amatrix}\\
&\overset{F_1+F_3}{\underset{F_6+3F_3}{\underset{F_5+2F_3}{\stackrel{F_4-2F_3}{\longrightarrow}}}}
\begin{amatrix}{3}
    1 & 0 & 2 & -b_1+b_3\\   
    0 & 0 & 2 & b_2\\
    0 & -1 & 1 & b_3-2b_1\\
    0 & 0 & 1 &4b_1-2b_3 + b_4\\
    0 & 0 & 5 &-3b_1+2b_3 + b_5\\
    0 & 0 & 7 & -3b_1+3b_3+b_6
\end{amatrix}\\
&\underset{F_6-7F_4}{\underset{F_5-5F_4}{\stackrel{F_2-2F_4}{\longrightarrow}}}
\begin{amatrix}{3}
    1 & 0 & 0 & -b_1+b_3-2b_4\\   
    0 & 0 & 0 &-8b_1+ b_2+4b_3-2b_4\\
    0 & -1 & 0 & b_3-2b_1+b_4\\
    0 & 0 & 1 &4b_1-2b_3 + b_4\\
    0 & 0 & 0 &-23b_1+12b_3 + b_5-5b_4\\
    0 & 0 & 0 & -3b_1+3b_3+b_6+7b_4
\end{amatrix}\\
\end{align*}
\end{comment}    

    \vskip .3cm
    \ref{extender-b} Lo haremos de forma análoga alo que hicimos en \ref{extender-a}. Debemos hallar la MRF de la matriz formada por los vectores dados como filas:
    $$
    \begin{bmatrix}
        1 & 2 & 0 & 0\\
        1 & 0 & 1 & 0
    \end{bmatrix}
    \stackrel{F_2-F_1}{\longrightarrow}
    \begin{bmatrix}
        1 & 2 & 0 & 0\\
        0 & -2 & 1 & 0
    \end{bmatrix}
    \stackrel{F_2/(-2)}{\longrightarrow}
    \begin{bmatrix}
        1 & 2 & 0 & 0\\
        0 & 1 & -\frac12 & 0
    \end{bmatrix}
    \stackrel{F_1-2F_2}{\longrightarrow}
    \begin{bmatrix}
        1 & 0 & 1 & 0\\
        0 & 1 & -\frac12 & 0
    \end{bmatrix}.
    $$
    Luego, los vectores que completan a una base son $(0,0,1,0)$ y $(0,0,0,1)$. Por supuesto, también podemos completar con otros pares de vectores, pero  estos son los más simples. 


    \vskip .3cm
    \ref{extender-c} Este inciso lo haremos de otra forma: primero encontraremos la forma implícita del subespacio y luego agregaremos vectores con un criterio que explicaremos más adelante. Para encontrar la forma implícita del subespacio planteamos la ecuación:
    $$
    \lambda_1(1,2,1,1)+\lambda_2(1,0,1,1)+\lambda_3(3,2,3,4)=(b_1,b_2,b_3,b_4).
    $$
    Es decir, debemos resolver el sistema
    $$
    \begin{cases}
        \lambda_1+\lambda_2+3\lambda_3=b_1\\
        2\lambda_1+2\lambda_3=b_2\\
        \lambda_1+\lambda_2+3\lambda_3=b_3\\
        \lambda_1+\lambda_2+4\lambda_3=b_4
    \end{cases}
    $$
    Planteamos la matriz ampliada y hacemos Gauss:
    \begin{align*}
    &\begin{amatrix}{3}
        1 & 1 & 3 & b_1\\
        2 & 0 & 2 & b_2\\
        1 & 1 & 3 & b_3\\
        1 & 1 & 4 & b_4
    \end{amatrix}
    \underset{F_4-F_1}{\underset{F_3-F_1}{\stackrel{F_2-2F_1}{\longrightarrow}}}
    \begin{amatrix}{3}
        1 & 1 & 3 & b_1\\
        0 & -2 & -4 & b_2-2b_1\\
        0 & 0 & 0 & b_3-b_1\\
        0 & 0 & 1 & b_4-b_1
    \end{amatrix}\\
    \end{align*}
    Luego  el subespacio generado por los vectores del enunciado tiene por forma implícita
    $$
    \{(b_1,b_2,b_3,b_4)\in\R^4: b_3=b_1\}.
    $$
    Cualquier vector que no cumpla esta condición no pertenece al subespacio y por lo tanto completa a una base. Por ejemplo, $(0,0,1,0)$ completa a una base. 


    \qed     
    
    \item Dar subespacios vectoriales $W_0$, $W_1$, $W_2$ y $W_3$ de $\mathbb{R}^3$ tales que $W_0\subset W_1\subset W_2\subset W_3$ y $\dim W_0=0$, $\dim W_1=1$, $\dim W_2=2$ y $\dim W_3=3$.
    
    
    \rta El único subespacio vectorial de dimensión $0$ es $\{0\}$. Por lo tanto, $W_0=\{0\}$. Sea $W_1=\langle (1,0,0)\rangle$ y $W_2=\langle (1,0,0), (0,1,0)\rangle$. El único subespacio de $\R^3$ de dimensión $3$ es $\R^3$. Por lo tanto, $W_3=\R^3$.

    \qed     
    
    \item Sea $V$ un espacio vectorial de dimensión $n$ y $\mathcal{B}=\{v_1, ..., v_n\}$ una base de $V$.
    \begin{enumerate}
    \item\label{subconjunto-li} Probar que cualquier subconjunto no vacío de $\mathcal{B}$ es LI.
    \item\label{subconjunto-dim-k} Para cada $k\in\mathbb{N}_0$,  con $0\leq k\leq n$, dar un subespacio vectorial de $V$ de dimensión $k$.
    \end{enumerate}
    
    
    \rta 

    \ref{subconjunto-li} Sea $W=\{v_{i_1}, ..., v_{i_k}\}$ un subconjunto no vacío de $\mathcal{B}$. Supongamos que $W$ es LD. Entonces existen $\lambda_1, ..., \lambda_k\in\mathbb{K}$, no todos nulos, tales que 
    $$
    \lambda_1 v_{i_1}+\cdots+\lambda_k v_{i_k}=0.
    $$
    Luego existe una combinación lineal no trivial de los elementos de la base que da como resultado el vector nulo.  Más explicitamente si $\mu_i = 0$ para $i \ne i_1, \ldots, i_k$ y $\mu_{i_j} = \lambda_{j}$ para $j=1,\ldots,k$, entonces
    $$
    \mu_1 v_1+\cdots+\mu_n v_n=0,
    $$
    y no todos los $\mu_i$ son nulos. Esto contradice que $\mathcal{B}$ sea una base. Por lo tanto, $W$ es LI.


    \vskip .3cm
    \ref{subconjunto-dim-k} Sea $W_k=\langle v_1, ..., v_k\rangle$. Entonces $W_k$ es un subespacio de $V$ de dimensión $k$. En efecto, como $\mathcal{B}$ es una base de $V$, entonces $W_k$ es un subespacio de $V$. Por otra parte, por \ref{subconjunto-li}, los $v_1, ..., v_k$ son LI. Por lo tanto, $\dim W_k=k$.

    \qed     
    
    \item Dar una base y calcular la dimensión de $\mathbb{C}^n$ como $\mathbb{C}$-espacio vectorial y como $\mathbb{R}$-espacio vectorial.
    
    
    \rta una base de  $\mathbb{C}^n$ como $\mathbb{C}$-espacio vectorial es $\{e_1,\ldots,e_n\}$, donde $e_i$ es el vector cuyas coordenadas son todas nulas excepto la $i$-ésima que es $1$. Por lo tanto, $\dim_{\mathbb{C}} \mathbb{C}^n=n$.

    Una base de $\mathbb{C}^n$ como $\mathbb{R}$-espacio vectorial es $\{e_1,\ldots,e_n,ie_1,\ldots,ie_n\}$, donde $e_i$ es el vector cuyas coordenadas son todas nulas excepto la $i$-ésima que es $1$. Por lo tanto, $\dim_{\mathbb{R}} \mathbb{C}^n=2n$.

    \qed     
    
    \item  Exhibir una base y calcular la dimensión de los siguientes subespacios.
    \begin{enumerate}
        \item\label{dim-subespacios-a} Los subespacios del ejercicio \ref{practicos anteriores}.
        \item\label{dim-subespacios-b} $W = \{(x,y,z,w,u) \in \mathbb{R}^5 \ : \ y = x - z,\, w = x + z,\,  u = 2x - 3z \}$.
        \item\label{dim-subespacios-c} $W = \langle (1, 0, -1, 1),  (1, 2, 1, 1), (0, 1, 1, 0), (0, -2, -2, 0) \rangle \subseteq \mathbb R^4$.
        \item\label{dim-subespacios-d} Matrices triangulares superiores $2\times 2$ y $3\times 3$.
        \item\label{dim-subespacios-e} Matrices triangulares superiores $n\times n$ para cualquier $n\in\mathbb{N}$, $n\geq 2$.
    \end{enumerate}
    
    \rta 

    \ref{dim-subespacios-a} En  el ejercicio \ref{practicos anteriores} ya dimos conjuntos de generadores para cada subespacio y es sencillo comprobar que cada uno  de estos subconjuntos son LI y por lo tanto son base. Solo resta calcular la dimensión. Listemos los subespacios y sus respectivas bases, lo cual nos dirá la dimensión de cada uno.
    \begin{itemize}
        \item $W_1= \{(0,0,0)\}$ es un subespacio de $\mathbb{R}^3$ y $\dim W_1=0$. Luego  su base es $\emptyset$.
        \item $W_2=\langle (4,3,1)\rangle$ es un subespacio de $\mathbb{R}^3$ y $\dim W_2=1$.
        \item $W_3=\langle (1,1,1,0),(-2,-2-,0-1)\rangle$ es un subespacio de $\mathbb{R}^4$ y $\dim W_3=2$.
        \item $W_4=\langle (1,-2,0),(0,-3,1)\rangle$ es un subespacio de $\mathbb{R}^3$ y $\dim W_4=2$.
        \item $W_5 = \langle (-1, 1, 1, 0), (1, 1, 0, 1) \rangle$ es un subespacio de $\mathbb{R}^4$ y $\dim W_5=2$.
        \item $W_6 = \langle (1, 0, 0), (0, 1, 0), (0,0,1) \rangle$ es un subespacio de $\mathbb{R}^3$ y $\dim W_6=3$.
    \end{itemize}

    
    \vskip.3cm
    \ref{dim-subespacios-b} Observar que 
    \begin{align*}
    W &= \{(x,y,z,w,u) \in \mathbb{R}^5 \ : \ y = x - z,\, w = x + z,\,  u = 2x - 3z \}\\
    &= \{(x,x-z,z,x+z,2x-3z) \in \mathbb{R}^5 \ : \ x,z \in \mathbb{R}\}\\
    &= \{(x,x,0,x,2x)+(0,-z,z,z,-3z) \in \mathbb{R}^5 \ : \ x,z \in \mathbb{R}\}\\
    &=\{x(1,1,0,1,2)+z(0,-1,1,1,-3) \in \mathbb{R}^5 \ : \ x,z \in \mathbb{R}\}\\
    &= \langle (1,1,0,1,2), (0,-1,1,1,-3) \rangle.
    \end{align*}
    Por lo tanto, $\{ (1,1,0,1,2), (0,-1,1,1,-3)\}$ es una base y la dimensión del subespacio es $2$.

    \vskip.3cm
    \ref{dim-subespacios-c} Para ver la dimensión  planteamos la matriz donde en cada fila ponemos un vector y  hacemos Gauss:
    \begin{align*}
        &\begin{bmatrix}
            1 & 1 & 0 & 0\\
            0 & 2 & 1 & -2\\
            -1 & 1 & 1 & -2\\
            1 & 1 & 0 & 0
        \end{bmatrix}
        \underset{F_4-F_1}{\stackrel{F_3+F_1}{\longrightarrow}}
        \begin{bmatrix}
            1 & 1 & 0 & 0\\
            0 & 2 & 1 & -2\\
            0 & 2 & 1 & -2\\
            0 & 0 & 0 & 0
        \end{bmatrix}
        \stackrel{F_3-F_2}{\longrightarrow}
        \begin{bmatrix}
            1 & 1 & 0 & 0\\
            0 & 2 & 1 & -2\\
            0 & 0 & 0 & 0\\
            0 & 0 & 0 & 0
        \end{bmatrix} 
        \\
        &\stackrel{F_2/2}{\longrightarrow}
        \begin{bmatrix}
            1 & 1 & 0 & 0\\
            0 & 1 & \frac12 & -1\\
            0 & 0 & 0 & 0\\
            0 & 0 & 0 & 0
        \end{bmatrix}
        \underset{F_1-F_2}{\longrightarrow}
        \begin{bmatrix}
            1 & 0 & -\frac12 & 1\\
            0 & 1 & \frac12 & -1\\
            0 & 0 & 0 & 0\\
            0 & 0 & 0 & 0
        \end{bmatrix}.
    \end{align*}
    Luego  una base del subespacio es $\{(1,0,-\frac12,1), (0,1,\frac12,-1)\}$ y por lo tanto la dimensión es $2$.


    \vskip.3cm
    \ref{dim-subespacios-d} Una base de las matrices triangulares superiores $2\times 2$ es
    $$
    \left\{  \begin{bmatrix} 1 & 0 \\ 0 & 0 \\ \end{bmatrix}, \quad
    \begin{bmatrix} 0 & 1 \\ 0 & 0 \\ \end{bmatrix}, \quad
    \begin{bmatrix} 0 & 0 \\ 0 & 1 \\ \end{bmatrix} \right\},
    $$
    y por lo tanto su dimensión es $3$.

    Una base de las matrices triangulares superiores $3\times 3$ es
    $$
    \left\{  \begin{bmatrix} 1 & 0 & 0 \\ 0 & 0 & 0 \\ 0 & 0 & 0 \\ \end{bmatrix}, 
    \begin{bmatrix} 0 & 1 & 0 \\ 0 & 0 & 0 \\ 0 & 0 & 0 \\ \end{bmatrix}, 
    \begin{bmatrix} 0 & 0 & 1 \\ 0 & 0 & 0 \\ 0 & 0 & 0 \\ \end{bmatrix}, 
    \begin{bmatrix} 0 & 0 & 0 \\ 0 & 1 & 0 \\ 0 & 0 & 0 \\ \end{bmatrix}, 
    \begin{bmatrix} 0 & 0 & 0 \\ 0 & 0 & 1 \\ 0 & 0 & 0 \\ \end{bmatrix}, 
    \begin{bmatrix} 0 & 0 & 0 \\ 0 & 0 & 0 \\ 0 & 0 & 1 \\ \end{bmatrix} \right\},
    $$
    y por lo tanto su dimensión es $6$.


    \vskip.3cm
    \ref{dim-subespacios-e} Este inciso es una generalización del anterior. Denotemos $E_{ij}$ a la matriz cuyas entradas son todas nulas excepto la $i$-ésima fila y la $j$-ésima columna, que es $1$. Es decir, $E_{ij}$ es la matriz que tiene un $1$ en la posición $(i,j)$ y ceros en el resto de las posiciones. Entonces, una base de las matrices triangulares superiores $n\times n$ es
    $$
    \mathcal B= \left\{  E_{11}, E_{12}, \ldots, E_{1n}, E_{22}, E_{23}, \ldots, E_{2n}, \ldots, E_{n-1,n}, E_{nn} \right\},
    $$
    escrito de forma más compacta y precisa,
    $$
    \mathcal B= \left\{  E_{ij} \in M_{n\times n}(\mathbb{R}) \ : \ 1\leq i\leq j\leq n \right\}.
    $$
    Calculemos ahora la dimensión de este subespacio. Para eso debemos ver cuántos elementos tiene $\mathcal B$. Observar que la diagonal tiene $n$ elementos de la base, todos los de la forma $E_{ii}$, $1 \le i \le n$.  Justo encima de la diagonal hay $n-1$ elementos de la base, todos los de la forma $E_{i,i+1}$, $1 \le i \le n-1$. Encima de esta última diagonal ``menor''  hay $n-2$ elementos de la base, todos los de la forma $E_{i,i+2}$, $1 \le i \le n-2$. Y así sucesivamente hasta llegar a la última diagonal menor, donde hay un solo elemento de la base, $E_{n,n}$. Por lo tanto, la cantidad de elementos de la base es 
    $$
    n + (n-1) + (n-2) + \cdots + 1 = \frac{n(n+1)}{2}.
    $$
    Luego la dimensión del subespacio formado por las matrices triangulares superiores es $\displaystyle\frac{n(n+1)}{2}$.


    \qed     
    
    \item Sean $W_1$ y $W_2$ los siguientes subespacios de $\mathbb{R}^3$:
        \begin{align*}
        W_1 &= \{ (x,y,z)\in\mathbb{R}^3\ : \ x+y-2z=0\},  \\
        W_2 &= {\left\langle(1,-1,1),(2,1,-2),(3,0,-1)\right\rangle}.
        \end{align*}
        \begin{enumerate}
            \item  Determinar $W_1 \cap W_2$, y describirlo por generadores y con ecuaciones.
            \item  Determinar $W_1+W_2$, y describirlo por generadores y con ecuaciones.
        \end{enumerate}
    
    
    \rta 

    \qed     
        
    \item\label{verdadero o falso} Decidir si las siguientes afirmaciones son verdaderas o falsas. Justificar.
    
    \begin{enumerate}
    \item Si $W_1$ y $W_2$ son subespacios vectoriales de $\mathbb{K}^8$ de dimensión $5$, entonces $W_1\cap W_2=0$.
    \item Si $W$ es un subespacio de $\mathbb{K}^{2\times2}$ de dimensión $2$, entonces existe una matriz triangular superior no nula que pertence a $W$.
    \item Sean $v_1, v_2, w\in \mathbb{K}^{n}$ y $A\in\mathbb{K}^{n\times n}$ tales que $Av_1=Av_2=0\neq Aw$. Si $\{v_1, v_2\}$ es LI, entonces $\{v_1,v_2,w\}$ también es LI.
    \item\label{cos}  $\{1,{\rm sen}(x),\cos(x)\}$ es un subconjunto LI del espacio de funciones de $\mathbb{R}$ en $\mathbb{R}$.
    \item\label{cos2}  $\{1,{\rm sen}^2(x),\cos^2(x)\}$ es un subconjunto LI del espacio de funciones $\mathbb{R}$ en $\mathbb{R}$.
    \item\label{exponencial}  $\{e^{\lambda_1x},e^{\lambda_2x},e^{\lambda_3x}\}$ es un subconjunto LI del espacio de funciones de
    $\mathbb{R}$ en $\mathbb{R}$, si $\lambda_1$, $\lambda_2$ y $\lambda_3$ son todos distintos.
    \end{enumerate}
    
    
    \rta 

    \qed     
    
    
    
    
    \end{enumerate}
    
    