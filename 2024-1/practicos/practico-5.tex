\chapter{Autovalores y autovectores \\Álgebra  II -- Año 2024/1 -- FAMAF}\label{practico-5}


    
    
\subsection*{Objetivos}
    
\begin{itemize}
\item Familiarizarse con las nociones de autovalor y autovector de una matriz cuadrada.

\item Aprender a calcular el polinomio característico, los autovalores, y los autoespacios de una matriz cuadrada.
\end{itemize}
    
    
\subsection*{Ejercicios} Los ejercicios con el símbolo $\textcircled{a}$ tienen una ayuda al final del archivo para que recurran a ella después de pensar un poco.


\begin{enumerate}[topsep=6pt,itemsep=.4cm]

\item\label{autovalores} Para cada una de las siguientes matrices, hallar sus autovalores reales, y para cada autovalor, dar una descripción paramétrica del autoespacio asociado sobre $\mathbb{R}$.
    \begin{multicols}{2}
    \begin{enumerate}
        \item\label{autovalores-1} $\left[\begin{matrix} 2 & 1\\ -1 & 4 \end{matrix} \right]$,\vskip .6cm 
        \item\label{autovalores-2} $ \left[\begin{matrix} 1 & 0\\ 1 & -2 \end{matrix} \right]$, \vskip .5cm
        \item\label{autovalores-3} $\left[\begin{matrix}2 & 0 & 0\\ -1 & 1& -1\\ 0 & 0 & 2 \end{matrix} \right]$,\vskip .2cm
        \item\label{autovalores-4} $\begin{bmatrix} 3 & -5 \\ 1 & -1 \end{bmatrix}$,\vskip .2cm
        \item\label{autovalores-5}  $\begin{bmatrix} \lambda & 0 & 0 \\ 1 & \lambda & 0\\ 0 & 1 & \lambda \\ \end{bmatrix}$, $\lambda\in \mathbb R$
        \vskip .2cm
        \item\label{autovalores-6} $\left[\begin{matrix}1 & 0& 0 \\ 0 & \cos\theta & \operatorname{sen}\theta\\ 0 & -\operatorname{sen}\theta & \cos\theta \end{matrix} \right]$, $0\leq \theta<2\pi$.
    \end{enumerate}
    \end{multicols}



\item\label{autovalores-complejos} Calcular los autovalores complejos de las matrices \ref{autovalores-4} y \ref{autovalores-6} del ejercicio anterior, y para cada autovalor, dar una descripción paramétrica del autoespacio asociado sobre $\mathbb{C}$.

\end{enumerate}

\textbf{Observación.} Es oportuno destacar algunos fenómenos que podemos observar en los ejercicios \ref{autovalores}-\ref{autovalores-complejos}.
\begin{itemize}
 \item[(i)] Una matriz con coeficientes reales puede no tener autovalores reales pero sí complejos (matriz $(d)$) o tener ambos (matriz $(f)$).
 \item[(ii)] Para describir paramétricamente los autoespacios podemos necesitar distintas cantidades de parámetros para los distintos autovalores (la matriz $(c)$). Esta cantidad mínima de parámetros es lo que llamaremos {\it dimensión}.
 \item[(iii)] La cantidad de autovalores distintos es menor o igual al tama\~no de la matriz. Incluso puede tener un sólo autovalor (matriz $(d)$ y más generalmente la matriz $(e)$ del Ejercicio \ref{mas}) o tener tantos como el tama\~no (matriz $(b)$ y $(f)$).
\end{itemize}

\begin{enumerate}[resume,topsep=6pt,itemsep=.4cm]

\item Probar que hay una única matriz $A\in\mathbb{R}^{2\times 2}$ tal que $(1,1)$ es autovector de autovalor $2$, y $(-2,1)$ es autovector de autovalor $1$.
    

\item Sea $A\in\mathbb{K}^{n\times n}$, y sea $f(x) = ax^2+bx+c$ un polinomio, con $a,b,c\in\mathbb{K}$. Sea $f(A)$ la matriz $n \times n$ definida por
$$f(A) = a A^2+bA+c\operatorname{Id}_n.$$
Probar que todo autovector de $A$ con autovalor $\lambda$ es autovector de $f(A)$ con autovalor $f(\lambda)$.

    
\item Sea $A\in\mathbb{K}^{2\times 2}$.

    \begin{enumerate}     
        \item Probar que el polinomio característico de $A$ es \ $\chi_A(x) = x^2-\operatorname{Tr}(A)x+\det(A)$.
        \item Si $A$ no es invertible, probar que los autovalores de  $A$ son $0$ y $\operatorname{Tr}(A)$.
    \end{enumerate}

    \item Sea $A\in\mathbb{K}^{n\times n}$. Probar que el polinomio $\tilde\chi_A(x)=\det(A-x\operatorname{Id}_n)$ y el polinomio característico de $A$ tienen las mismas raíces.

\end{enumerate}

\textbf{Observación} Algunos libros definen el polinomio característico de la matriz $A$ como $\tilde\chi_A(x)=\det(x\operatorname{Id}_n-A)$. Como vemos en el ejercicio anterior, ambas definiciones sirven para encontrar autovalores de $A$. El polinomio $\tilde\chi_A(x)$ tiene la particularidad de ser mónico, o sea que el coeficiente del término $x^n$ es $1$.

\begin{enumerate}[resume,topsep=6pt,itemsep=.4cm]

\item Probar que si $A\in\mathbb{K}^{n\times n}$ es una matriz nilpotente entonces $0$ es el único autovalor de $A$. Usar esto para deducir que la matriz $\operatorname{Id}_n-A$ es invertible (esta es otra demostración del ejercicio \ref{nilpotene - id} del Práctico \ref{practico-3}).


\item Decidir si las siguientes afirmaciones son verdaderas o falsas. Justificar.

\begin{enumerate}
    \item Existe una matriz invertible $A$ tal que $0$ es autovalor de $A$.
    \item  Si $A$ es invertible, entonces todo autovector de $A$ es autovector de $A^{-1}$.
\end{enumerate}



\item\label{mas} Repetir los ejercicios \ref{autovalores} y \ref{autovalores-complejos} con las siguientes matrices.
\begin{multicols}{2}
    \begin{enumerate}
        \item\label{mas-autovalores-1} $\begin{bmatrix} 2 & 3 \\ -1 & 1
        \end{bmatrix} $,\vskip .6cm 
        \item\label{mas-autovalores-2} $\begin{bmatrix} -9 & 4 & 4 \\ -8 & 3 & 4 \\ -16 & 8 & 7 \end{bmatrix} $, \vskip .5cm
        \item\label{mas-autovalores-3} $\left[\begin{matrix}4 & 4 & -12\\ 1 & -1 & 1\\ 5 & 3 & -11 \end{matrix} \right] $,\vskip .2cm
        \item\label{mas-autovalores-4} $\left[\begin{matrix}2 & 1 & 0 & 0\\ -1 & 4 & 0 & 0\\ 0 & 0 & 1 & 1 \\ 0 & 0 & 3 & -1\end{matrix} \right] $,\vskip .2cm
        \item\label{mas-autovalores-5} $\begin{bmatrix} \lambda & 0 & 0 & \dots & 0  \\ 1 & \lambda & 0 &\dots & 0  \\ 0 & 1 & \lambda&  \dots & 0  \\ \vdots & \vdots & \quad & \ddots & \vdots\\ 0 &  0&   \dots & 1  & \lambda \end{bmatrix}$, $\lambda\in \mathbb R$. \vskip .2cm
    \end{enumerate}
    \end{multicols}


\end{enumerate}

\textbf{Ejercicios de repaso}
Si ya hizo los ejercicios anteriores continue con la siguiente guía. Los ejercicios que siguen son similares y le pueden servir para practicar antes de los exámenes.

\begin{enumerate}[resume,topsep=6pt,itemsep=.4cm]


\item Sea $A\in\mathbb{K}^{n\times n}$, y sea $f(x) = a_0 + a_1 x + \dots + a_nx^n$, $n \geq 1$, $a_i\in\mathbb{K}$, $a_n \neq 0$, un polinomio. Sea $f(A)$ la matriz $n \times n$ definida por
$$f(A) = a_0 \operatorname{Id}_n + a_1 A + \dots + a_n A^n.$$
Probar que todo autovector de $A$ con autovalor $\lambda$ es autovector de $f(A)$ con autovalor $f(\lambda)$.


\item\label{caracteristico-otro} 

\begin{enumerate}
\item  Calcular el polinomio característico de las siguientes matrices.
\begin{align*}
A_2 := \begin{bmatrix} 0 & -a_0 \\ 1 & -a_1
\end{bmatrix},\quad\quad
A_3 := \begin{bmatrix} 0 & 0 & -a_0 \\ 1 & 0 & -a_1 \\ 0 & 1 & -a_2
\end{bmatrix}.
\end{align*}
donde $a_0, a_1, a_2$ son escalares.

\item\label{matriz de un polinomio} $\textcircled{a}$ Sean $a_0, ..., a_{n-1}$ escalares. Calcular el polinomio característico de
\begin{align*}
A_n := \begin{bmatrix} 0 & 0 & 0 &\dots & 0 & -a_0 \\ 1 & 0 & 0&  \dots & 0  & -a_1 \\ 0 & 1 & 0&  \dots & 0  & -a_2 \\ \vdots & \vdots & \ddots & \quad  & \vdots\\ 0 & 0 & 0 & \dots & 1  & -a_{n-1}
\end{bmatrix}.        
 \end{align*}

 \item Deducir que dado un polinomio mónico $p(x)$ siempre existe una matriz $A$ tal que $\chi_A(x)=p(x)$.
\end{enumerate}


\item\label{tr det}$\textcircled{a}$ 
Sea $A\in\mathbb{K}^{n\times n}$, y \ $\chi_A(x) = x^n+a_{n-1}x^{n-1}+\cdots+a_0$. Probar que
\begin{enumerate}
    \item\label{tr det a} $a_0 = (-1)^n \det(A)$.
    \item\label{tr det b} $a_{n-1} = - \operatorname{Tr}(A)$.
\end{enumerate}


\item\label{complejos} $\textcircled{a}$  Sea $A\in\mathbb{C}^{n\times n}$. Probar que si $\lambda_1,\dots,\lambda_n \in \mathbb C$ son los autovalores de $A$
(posiblemente repetidos), entonces se cumple que:
\begin{enumerate}
    \item\label{complejos a} $\det(A)=\lambda_1\cdots \lambda_n$.
    \item\label{complejos b} $\operatorname{Tr}(A)=\lambda_1+\cdots+\lambda_n$.
\end{enumerate}

\subsection*{Aclaración}
Los ejercicios \ref{tr det}  \ref{tr det b} y \ref{complejos}  \ref{complejos b} no son fáciles de probar y no sería posible que estuvieran en una evaluación. Pero dado que enunciamos los items  \ref{tr det}  \ref{tr det a} y \ref{complejos}  \ref{complejos a}, es interesante saber que valen los items  \ref{tr det}  \ref{tr det b} y \ref{complejos}  \ref{complejos b}, respectivamente.
% Demostración: Por hipotesis el polinomio característico de $A$ se descompone de la siguiente manera:
% \begin{align*}
% \chi_A(x)=(x-\lambda_1)(x-\lambda_2)\cdots(x-\lambda_n).
% \end{align*}
% Entonces podemos usar esta igualdad para calcular de distintas maneras los términos $0$ y $n-1$ del polinomio característico.
%
% Por el ejercicio anterior el término $0$ de $\chi_A(x)$ es $(-1)^{n}\det(A)$ y por la igualdad anterior es $(-1)^n\lambda_1\cdots\lambda_n$. De lo cual deducimos el item (a).
%
% También por el ejercicio anterior $-\operatorname{Tr}(A)$ es el término $n-1$ del polinomio característico mientras que por la igualdad anterior podemos ver que es igual a $-\lambda_1-\lambda_2-\cdots-\lambda_n$. De lo cual deducimos (b).

\end{enumerate}


\subsection*{Ayudas}

\ref{caracteristico-otro}\,\ref{matriz de un polinomio} Desarrollar el determinante por la primera fila y hacer inducción.


\ref{tr det} \ref{tr det a}\ref{tr det a} Evaluar el polinomio $\chi_A(x)$ en un valor apropiado para obtener el término independiente $a_0$.


\ref{tr det} \ref{tr det b} Desarrollar el determinante de $x\operatorname{Id}-A$ por la primera columna y hacer inducción en el tamaño de la matriz. Es decir, primero
\begin{align*}
\chi_A(x) &= \det(x\operatorname{Id}-A)\\
&=(x-a_{11})\det((x\operatorname{Id}-A)(1|1))+a_{21}\det((x\operatorname{Id}-A)(2|1))+\cdots\\
&\qquad+(-1)^n a_{n1}\det((x\operatorname{Id}-A)(n|1)).
\end{align*}
De estos sumandos, el único sumando donde hay $x^{n-1}$ es $(x-a_{11})\det((x\operatorname{Id}-A)(1|1))$. Además, $\det((x\operatorname{Id}-A)(1|1))$ es el polinomio característico de la submatriz $A(1|1)$. Podemos aplicar la hipótesis inductiva a esta matriz y deducir que el coeficiente de $x^{n-1}$ en el producto de polinomios $(x-a_{11})\det((x\operatorname{Id}-A)(1|1))$ es $- \operatorname{Tr}(A)$.



\ref{complejos} Sobre $\mathbb{C}$ podemos descomponer el polinomio $\chi_A(x)$ de la siguiente manera
\begin{align*}
\chi_A(x)=(x-\lambda_1)(x-\lambda_2)\cdots(x-\lambda_n). \qquad (\diamondsuit)
\end{align*}
Con esta igualdad podemos calcular los términos $a_0$ y $a_{n-1}$ de $\chi_A(x)$ de dos maneras. La primera es la obtenida en el ejercicio \ref{tr det}. La segunda es usando la multiplicación del lado derecho de $(\diamondsuit)$. Para el término $a_0$ hay que evaluar en un valor apropiado. Para el término $a_{n-1}$ hay que notar que para obtener $x^{n-1}$ debemos elegir una $x$ de todos los factores salvo en uno y un término del estilo $-\lambda_i$. Igualando lo que obtengamos probamos el ejercicio.

