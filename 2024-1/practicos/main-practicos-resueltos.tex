% PDFLaTeX
\documentclass[a4paper,12pt,twoside,spanish,reqno]{amsbook}
%%%---------------------------------------------------
\usepackage[math]{kurier}

\usepackage{etex}
\usepackage{t1enc}
\usepackage{latexsym}
\usepackage[utf8]{inputenc}
\usepackage{verbatim}
\usepackage{multicol}
\usepackage{amsgen,amsmath,amstext,amsbsy,amsopn,amsfonts,amssymb}
\usepackage{amsthm}
\usepackage{calc}         % From LaTeX distribution
\usepackage{graphicx}     % From LaTeX distribution
\usepackage{ifthen}
\input{random.tex}   
\usepackage{tikz}
\usetikzlibrary{arrows}
\usetikzlibrary{matrix}
\usepackage{mathtools}
\usepackage{stackrel}
\usepackage{enumitem}
\usepackage{tkz-graph}

\usepackage{enumitem} 
\usepackage[compatibility=false]{caption} % para usar subcaption
\usepackage{subcaption} % para poner varias imagenes juntas
\usetikzlibrary{arrows.meta}
\usepackage{hyperref}
\hypersetup{ 
    colorlinks=true,
    linkcolor=blue,
    filecolor=magenta,      
    urlcolor=cyan,
}
\usepackage{hypcap}
\numberwithin{equation}{section}
% http://www.texnia.com/archive/enumitem.pdf (para las labels de los enumerate)
\renewcommand\labelitemi{$\circ$}
\setlist[enumerate, 1]{label={(\arabic*)}}
\setlist[enumerate, 2]{label=\emph{\alph*)}}


\newcommand{\rta}{\vskip.2cm\noindent\textsc{Solución: }} 

\newcommand{\img}{\operatorname{Im}}
\newcommand{\nuc}{\operatorname{Nu}}
\newcommand\im{\operatorname{Im}}
\renewcommand\nu{\operatorname{Nu}}
\newcommand{\la}{\langle}
\newcommand{\ra}{\rangle}
\renewcommand{\t}{{\operatorname{t}}}
\renewcommand{\sin}{{\,\operatorname{sen}}}
\newcommand{\Q}{\mathbb Q}
\newcommand{\R}{\mathbb R}
\newcommand{\C}{\mathbb C}
\newcommand{\K}{\mathbb K}
\newcommand{\F}{\mathbb F}
\newcommand{\Z}{\mathbb Z}
\newenvironment{amatrix}[1]{%
  \left[\begin{array}{@{}*{#1}{c}|c@{}}
}{%
  \end{array}\right]
}
\renewcommand{\qed}{\hfill$\square$\vskip.6cm}
\renewcommand{\theequation}{\arabic{chapter}.\arabic{equation}} 
%%% FORMATOS %%%%%%%%%%%%%%%%%%%%%%%%%%%%%%%%%%%%%%%%%%%%%%%%%%%%%%%%%%%%%%%%%%%%%
\tolerance=10000
\renewcommand{\baselinestretch}{1.2}
\usepackage[a4paper, top=3cm, left=3cm, right=2cm, bottom=2.5cm]{geometry}
\usepackage{setspace}
%\setlength{\parindent}{0,7cm}% tamaño de sangria.
\setlength{\parskip}{0,4cm} % separación entre parrafos.
%\renewcommand{\baselinestretch}{0.90}% separacion del interlineado
\renewcommand{\chaptername}{Práctico}
%%%%%%%%%%%%%%%%%%%%%%%%%%%%%%%%%%%%%%%%%%%%%%%%%%%%%%%%%%%%%%%%%%%%%%%%%%%%%%%%%%%
%\end{comment}
%%% FIN FORMATOS  %%%%%%%%%%%%%%%%%%%%%%%%%%%%%%%%%%%%%%%%%%%%%%%%%%%%%%%%%%%%%%%%%

\begin{document}
    \baselineskip=0.55truecm %original

%%% CAP1 =================
    \input{practico-1-resuelto.tex}
%%%=======================
%%%=======================
%%% CAP2 =================
    \input{practico-2-resuelto.tex}
%%%=======================
%%%=======================
%%% CAP3 =================
    % PDFLaTeX
\documentclass[a4paper,12pt,twoside,spanish,reqno]{amsbook}
%%%---------------------------------------------------
\usepackage[math]{kurier}

\usepackage{etex}
\usepackage{t1enc}
\usepackage{latexsym}
\usepackage[utf8]{inputenc}
\usepackage{verbatim}
\usepackage{multicol}
\usepackage{amsgen,amsmath,amstext,amsbsy,amsopn,amsfonts,amssymb}
\usepackage{amsthm}
\usepackage{calc}         % From LaTeX distribution
\usepackage{graphicx}     % From LaTeX distribution
\usepackage{ifthen}
\input{random.tex}   
\usepackage{tikz}
\usetikzlibrary{arrows}
\usetikzlibrary{matrix}
\usepackage{mathtools}
\usepackage{stackrel}
\usepackage{enumitem}
\usepackage{tkz-graph}

\usepackage{enumitem} 
\usepackage[compatibility=false]{caption} % para usar subcaption
\usepackage{subcaption} % para poner varias imagenes juntas
\usetikzlibrary{arrows.meta}
\usepackage{hyperref}
\hypersetup{ 
    colorlinks=true,
    linkcolor=blue,
    filecolor=magenta,      
    urlcolor=cyan,
}
\usepackage{hypcap}
\numberwithin{equation}{section}
% http://www.texnia.com/archive/enumitem.pdf (para las labels de los enumerate)
\renewcommand\labelitemi{$\circ$}
\setlist[enumerate, 1]{label={(\arabic*)}}
\setlist[enumerate, 2]{label=\emph{\alph*)}}


\newcommand{\rta}{\noindent\textsc{Solución: }} 

\newcommand{\img}{\operatorname{Im}}
\newcommand{\nuc}{\operatorname{Nu}}
\newcommand\im{\operatorname{Im}}
\renewcommand\nu{\operatorname{Nu}}
\newcommand{\la}{\langle}
\newcommand{\ra}{\rangle}
\renewcommand{\t}{{\operatorname{t}}}
\renewcommand{\sin}{{\,\operatorname{sen}}}
\newcommand{\Q}{\mathbb Q}
\newcommand{\R}{\mathbb R}
\newcommand{\C}{\mathbb C}
\newcommand{\K}{\mathbb K}
\newcommand{\F}{\mathbb F}
\newcommand{\Z}{\mathbb Z}
\newenvironment{amatrix}[1]{%
  \left[\begin{array}{@{}*{#1}{c}|c@{}}
}{%
  \end{array}\right]
}

%%% FORMATOS %%%%%%%%%%%%%%%%%%%%%%%%%%%%%%%%%%%%%%%%%%%%%%%%%%%%%%%%%%%%%%%%%%%%%
\tolerance=10000
\renewcommand{\baselinestretch}{1.3}
\usepackage[a4paper, top=3cm, left=3cm, right=2cm, bottom=2.5cm]{geometry}
\usepackage{setspace}
%\setlength{\parindent}{0,7cm}% tamaño de sangria.
\setlength{\parskip}{0,4cm} % separación entre parrafos.
\renewcommand{\baselinestretch}{0.90}% separacion del interlineado
%%%%%%%%%%%%%%%%%%%%%%%%%%%%%%%%%%%%%%%%%%%%%%%%%%%%%%%%%%%%%%%%%%%%%%%%%%%%%%%%%%%
%\end{comment}
%%% FIN FORMATOS  %%%%%%%%%%%%%%%%%%%%%%%%%%%%%%%%%%%%%%%%%%%%%%%%%%%%%%%%%%%%%%%%%

\begin{document}
    \baselineskip=0.55truecm %original
    
    
    {\bf \begin{center} Práctico 3 \\ Álgebra  II -- Año 2024/1 \\ FAMAF \end{center}}

%\title{Pr\'actico 1}


\centerline{\textsc{\'Algebra de matrices}}
\subsection*{Ejercicios resueltos}

\begin{enumerate}[topsep=6pt,itemsep=.4cm]

%%%%%%%%%%%%%%%%%%%%%%%%%%%%%%%%%%%%%%%%%%%%%%%%%%%%%%%%%%%%%%%%%%%%%%%%%%%%%%%%%%%%%%%%%%%%%%%%%%%%%%%%%%%%%%%%%%%%%%%%%%%%%%%%%%%%%%%%%%%%%%%%%%%%%%%%

\item\label{ej} Sean
$$
A= \begin{bmatrix} 1&-2&0\\ 1&-2&1\\ 1&-2&-1\end{bmatrix},\quad
\quad B= \begin{bmatrix}1&1&2\\ -2&0&-1\\ 1&3&5 \end{bmatrix},
\quad\quad C=\begin{bmatrix}1&-1&1\\ 2&0&1\\ 3&0&1 \end{bmatrix}.
$$

Verificar que $A(BC)=(AB)C$, es decir que vale la asociatividad del producto.
\vskip .2cm
\noindent\textsc{Solución:}
\begin{equation*}
	BC =  \begin{bmatrix}1&1&2\\ -2&0&-1\\ 1&3&5 \end{bmatrix} \begin{bmatrix}1&-1&1\\ 2&0&1\\ 3&0&1 \end{bmatrix}=\begin{bmatrix}9 & -1 & 4 \\-5 & 2 & -3 \\22 & -1 & 9 \end{bmatrix}
\end{equation*}
\begin{equation*}
	A(BC) =\begin{bmatrix} 1&-2&0\\ 1&-2&1\\ 1&-2&-1\end{bmatrix}\begin{bmatrix}9 & -1 & 4 \\-5 & 2 & -3 \\22 & -1 & 9 \end{bmatrix} =
	\begin{bmatrix}19 & -5 & 10 \\41 & -6 & 19 \\-3 & -4 & 1\end{bmatrix} \tag{*}
\end{equation*}
\begin{equation*}
	AB = \begin{bmatrix} 1&-2&0\\ 1&-2&1\\ 1&-2&-1\end{bmatrix} \begin{bmatrix}1&1&2\\ -2&0&-1\\ 1&3&5 \end{bmatrix} =
	\begin{bmatrix}5 &1 &4\\ 6&4&9\\ 4&-2&-1 \end{bmatrix}
\end{equation*}
\begin{equation*}
	(AB)C = \begin{bmatrix}5 &1 &4\\ 6&4&9\\ 4&-2&-1 \end{bmatrix}\begin{bmatrix}1&-1&1\\ 2&0&1\\ 3&0&1 \end{bmatrix} =
	\begin{bmatrix}19 & -5 & 10 \\41 & -6 & 19 \\-3 & -4 & 1\end{bmatrix} \tag{**}
\end{equation*}
Luego (*) = (**) y el resultado queda verificado. \qed


\item\label{ej2} Determinar cuál de las siguientes matrices es $A$, cuál es $B$ y cuál es $C$ de modo tal que sea posible realizar el producto $ABC$ y verificar que $A(BC)=(AB)C$.
\begin{equation*}
\begin{bmatrix} 2 & -1 & 1 \\ 1 & 2 &
1\end{bmatrix},\qquad
\begin{bmatrix} 3 \\ 1 \\ -1\end{bmatrix}, \qquad
\begin{bmatrix} 1 & -1 \end{bmatrix}.
\end{equation*}
\vskip .2cm
\noindent\textsc{Solución:} la  primera matriz es $2 \times 3$, la segunda es $3 \times 1$ y la tercera es $1 \times 2$. Entonces, podemos multiplicar la 1º por  la 2º y  queda una matrix $2 \times 1$ que es posible multiplicar por la 3º matriz,  que es $1 \times 2$,  y así obtenemos una matriz $2 \times 2$.  Es decir,
\begin{equation*}
	A=\begin{bmatrix} 2 & -1 & 1 \\ 1 & 2 &
	1\end{bmatrix},\qquad
	B= \begin{bmatrix} 3 \\ 1 \\ -1\end{bmatrix}, \qquad
	C= \begin{bmatrix} 1 & -1 \end{bmatrix}.
\end{equation*}
Ahora bien, 
\begin{equation*}
BC= \begin{bmatrix} 3 \\ 1 \\ -1\end{bmatrix}\begin{bmatrix} 1 & -1 \end{bmatrix} =  \begin{bmatrix} 3& -3\\ 1&-1 \\-1 &1\end{bmatrix}
\end{equation*}
\begin{equation*}
	A(BC)=\begin{bmatrix} 2 & -1 & 1 \\ 1 & 2 &	1\end{bmatrix}\begin{bmatrix} 3& -3\\ 1&-1 \\-1 &1\end{bmatrix} =  \begin{bmatrix} 4&-4 \\ 4&-4 \end{bmatrix}
\end{equation*}
\begin{equation*}
	AB=\begin{bmatrix} 2 & -1 & 1 \\ 1 & 2 &
	1\end{bmatrix}\begin{bmatrix} 3 \\ 1 \\ -1\end{bmatrix} = \begin{bmatrix} 4 \\ 4 \end{bmatrix}.
\end{equation*}
\begin{equation*}
	(AB)C= \begin{bmatrix} 4 \\ 4 \end{bmatrix}\begin{bmatrix} 1 & -1 \end{bmatrix} = \begin{bmatrix} 4&-4 \\ 4&-4 \end{bmatrix}.
\end{equation*}\qed


\item Calcular $A^2$ y $A^3$ para la matriz \
$
A=\begin{bmatrix}
3 & 4\\ 6 & 8
\end{bmatrix}.
$
\vskip .2cm
\noindent\textsc{Solución:}
\begin{align*}
	A^2 &= \begin{bmatrix}
		3 & 4\\ 6 & 8
		\end{bmatrix}\begin{bmatrix}
			3 & 4\\ 6 & 8
			\end{bmatrix} = \begin{bmatrix}
				33 & 44\\ 66 & 88
				\end{bmatrix}  = 11\begin{bmatrix}
					3 & 4\\ 6 & 8
					\end{bmatrix} = 11A\\
	A^3 &= A\cdot A^2 = A\cdot 11A = 11A^2 = 11^2A. 
\end{align*}

\noindent \textbf{Observación.} Este es un caso muy particular. En  el caso de una matriz escalar $c\operatorname{Id}$,  tenemos que  $(c\operatorname{Id})^n= c^n \operatorname{Id}$. Aquí tenemos una matriz $A$ tal que $A^n = 11^{n-1}A$.

\qed



\item\label{ejemplos 2x2}  Dar ejemplos de matrices no nulas $A$ y $B$ de orden $2\times2$ tales que
\begin{multicols}{2}
\begin{enumerate}[topsep=5pt,itemsep=5pt]
 \item $A^2=0$ (dar dos ejemplos).
 \item $AB\neq BA$.
 \item $A^2=-\operatorname{I}_2$.
 \item $A^2=A\neq\operatorname{I}_2$.
\end{enumerate}
\end{multicols}
\vskip .2cm
\noindent\textsc{Solución:}
\vskip .2cm
 (a)
\begin{equation*}
	A = \begin{bmatrix}
		0 & 1\\ 0 & 0
		\end{bmatrix}, \qquad A = \begin{bmatrix}
			0 & 0\\ 1 & 0
			\end{bmatrix}.
\end{equation*}
\vskip .2cm
 (b)
 \begin{equation*}
	A = \begin{bmatrix}
		0 & 1\\ 0 & 0
		\end{bmatrix}, \qquad B = 
		\begin{bmatrix}
		0 & 0\\ 1 & 0
		\end{bmatrix}.
\end{equation*}
\begin{align*}
	AB &= \begin{bmatrix}
		0 & 1\\ 0 & 0
		\end{bmatrix} \begin{bmatrix}
			0 & 0\\ 1 & 0
			\end{bmatrix} = \begin{bmatrix}
				1 & 0\\ 0 & 0
				\end{bmatrix} \\
	BA &=  
		\begin{bmatrix}
		0 & 0\\ 1 & 0
		\end{bmatrix}
		\begin{bmatrix}
		0 & 1\\ 0 & 0
		\end{bmatrix} = \begin{bmatrix}
		0 & 0\\ 0 & 1
		\end{bmatrix} 
\end{align*}
\vskip .2cm
 (c)
 \begin{equation*}
	A = \begin{bmatrix}
		0 & 1\\ -1 & 0
		\end{bmatrix}, \qquad \Rightarrow \qquad A^2 = 
		\begin{bmatrix}
			0 & 1\\ -1 & 0
			\end{bmatrix}
			\begin{bmatrix}
				0 & 1\\ -1 & 0
				\end{bmatrix}
			= \begin{bmatrix}
				-1 & 0\\ 0 & -1
				\end{bmatrix}.
\end{equation*}
\vskip .2cm
(d)
\begin{equation*}
	A = \begin{bmatrix}
		1 & 0\\ 0 & 0
		\end{bmatrix}.
\end{equation*}
\qed


\item\label{2x2 central}  Sea $A \in\mathbb{R}^{2\times 2}$ tal que $AB=BA$ para toda $B\in\mathbb{R}^{2\times 2}$. Probar que $A$ es un múltiplo de $\operatorname{I}_2$.
\vskip .2cm
\noindent\textsc{Solución:} Sea
\begin{equation*}
	A = \begin{bmatrix} a_{11} & a_{12}\\ a_{21} & a_{22} \end{bmatrix},
\end{equation*}
y sean 
\begin{equation*}
	E_{11} = \begin{bmatrix} 1 & 0\\ 0 & 0\end{bmatrix},
	\quad
	E_{12} = \begin{bmatrix} 0 & 1\\ 0 & 0\end{bmatrix},
	\quad
	E_{21} = \begin{bmatrix} 0 & 0\\ 1 & 0\end{bmatrix},
	\quad
	E_{22} = \begin{bmatrix} 0 & 0\\ 0 & 1\end{bmatrix}. 
\end{equation*}
Entonces 
\begin{align*}
	AE_{11} = \begin{bmatrix} a_{11} & a_{12}\\ a_{21} & a_{22} \end{bmatrix} \begin{bmatrix} 1 & 0\\ 0 & 0\end{bmatrix} =
	\begin{bmatrix} a_{11} &0\\ a_{21} & 0\end{bmatrix}, \quad
	E_{11}A =  \begin{bmatrix} 1 & 0\\ 0 & 0\end{bmatrix} \begin{bmatrix} a_{11} & a_{12}\\ a_{21} & a_{22} \end{bmatrix}=
	\begin{bmatrix} a_{11} & a_{12}\\ 0 & 0 \end{bmatrix},
\end{align*}
Como $AE_{11} = E_{11}A$, tenemos que $a_{12} = a_{21} =0$.

Probemos ahora con $E_{12}$:
\begin{align*}
	AE_{12} = \begin{bmatrix} a_{11} & a_{12}\\ a_{21} & a_{22} \end{bmatrix} \begin{bmatrix} 0 & 1\\ 0 & 0\end{bmatrix} =
	\begin{bmatrix} 0 &a_{11}\\ 0 & a_{21}\end{bmatrix}, \quad
	E_{12}A =  \begin{bmatrix} 0 & 1\\ 0 & 0\end{bmatrix} \begin{bmatrix} a_{11} & a_{12}\\ a_{21} & a_{22} \end{bmatrix}=
	\begin{bmatrix} a_{21} & a_{22}\\ 0 & 0 \end{bmatrix},
\end{align*}
Como $AE_{12} = E_{12}A$, tenemos que $a_{11} = a_{22}$. 

Ya no hace falta hacer más ensayos, pues hemos probado que $a_{12} = a_{21} =0$ y $a_{11} = a_{22}$, es decir $A$ es una matriz escalar y  al ser escalar sabemos que conmuta con todas las matrices. 

\vskip .2cm
\noindent \textbf{Observación.} El resultado es cierto también para matrices $n \times  n$ con $n \in \mathbb N$. Es decir,  si $A \in\mathbb{R}^{n\times n}$ tal que $AB=BA$ para toda $B\in\mathbb{R}^{n\times n}$,  entonces  $A$ es un múltiplo de $\operatorname{I}_n$. La estrategia para probar este resultado es la misma que para el caso $2 \times 2$: probar con las matrices $E_{ij}$ que son aquellas que tiene un 1 en la entrada $ij$ y  0  en las demás entradas. 

\qed

\item  Para cada $n\in\mathbb{N}$, con $n\geq 2$, hallar una matriz no nula $A\in\mathbb{R}^{n\times n}$ tal que $A^n=0$ pero $A^{n-1}\neq0$.
\vskip .2cm
\noindent\textsc{Solución:} Las matrices  triangulares estrictas, superiores o inferiores, satisfacen la propiedad de que $A^n=0$ y  muchas de ellas satisfacen que $A^{n-1} \ne 0$. Probemos con una matriz triangular superior estricta particular 
\begin{equation*}
	A = \begin{bmatrix}
		0&1&0&0&\ldots&0 \\
		0&0&1&0&\ldots&0 \\
		0&0&0&1&\ldots&0 \\
		\vdots&&&&\ddots&\vdots \\
		0&0&0&0&\ldots &1 \\
		0&0&0&0&\ldots &0  
	\end{bmatrix} = \begin{bmatrix}
		e_2 \\ e_3 \\ e_4 \\ \vdots \\ e_n \\0
	\end{bmatrix} = \begin{bmatrix} \mid& \mid& \mid& &\mid\\ 0 & e_1 & e_2 & \cdots &e_{n-1}\\ \mid&\mid & \mid& &\mid\end{bmatrix}
\end{equation*}
Es decir $A$  es una matriz $n \times n$ con $1$ encima de la diagonal y $0$  en todas las demás entradas. Más formalmente: $[A]_{i,i+1} = 1$ y $[A]_{i,j} = 0$ si $j \ne i+1$. 
\vskip .2cm
Probaremos por inducción que para $k< n$,  $[A^k]_{i,i+k} = 1$ y $[A]_{i,j} = 0$ si $j \ne i+k$. Es decir,
\begin{equation*}
	A^k  = \begin{bmatrix} 
		\mid& \cdots & \mid& \mid& \mid&   &\mid\\ 
		0 & \cdots & 0 & e_{1}& e_2& \cdots &e_{n-k}\\ 
		\mid& \cdots& \mid& \mid& \mid&   &\mid
	\end{bmatrix},
\end{equation*}
donde $e_1$  está en la columna $k+1$. 


Si $k=1$ el resultado vale por definición de $A$. Supongamos que el resultado es cierto para $k-1$,   luego \begin{equation*}
	A^{k-1}  = \begin{bmatrix} 
		\mid& \cdots & \mid& \mid& \mid&   &\mid\\ 
		0 & \cdots & 0 & e_{1}& e_2& \cdots &e_{n-k+1}\\ 
		\mid& \cdots& \mid& \mid& \mid&   &\mid
	\end{bmatrix},\tag{HI}
\end{equation*}
donde $e_1$  está en la columna $k$. Por lo tanto,
\begin{equation*}
	A \cdot A^{k-1}  = \begin{bmatrix}
		e_2 \\ e_3 \\ e_4 \\ \vdots \\ e_n \\0
	\end{bmatrix} \begin{bmatrix} 
		\mid& \cdots & \mid& \mid& \mid&   &\mid\\ 
		0 & \cdots & 0 & e_{1}& e_2& \cdots &e_{n-k+1}\\ 
		\mid& \cdots& \mid& \mid& \mid&   &\mid
	\end{bmatrix},
\end{equation*}
Como $[A\cdot A^{k-1}]_{ij} = F_i(A) \cdot C_j(A^{k-1})$ (donde $\cdot$ indica el producto escalar), los únicos productos no nulos son $ F_1(A) \cdot C_{k+1}(A^{k-1}) = 1$,  $ F_2(A) \cdot C_{k +2}(A^{k-1}) = 1$, $ F_3(A) \cdot C_{k +3}(A^{k-1}) = 1$, etc. Es decir, las entradas de $A^k$ valen $1$ en $(1,k+1)$, $(2,k+2)$, $(3,k+3)$, etc. lo cual prueba el resultado. 


Probado esto, tenemos $A^{n-1} = [0 \;\cdots\; 0\; e_1] \ne 0$ y $A^n = A \cdot A^{n-1}=0$. 

\qed


\item\label{eq:binomio}  Dar condiciones necesarias y suficientes sobre matrices $A$ y $B$ de tama\~{n}o $n\times n$ para que
\begin{multicols}{2}
	\begin{enumerate}
		\item\label{ej-p3-8-a} $(A + B)^2 = A^2 + 2AB + B^2$.
		\item\label{ej-p3-8-b} $A^2 - B^2 = (A - B)(A + B)$.
	\end{enumerate}
\end{multicols}
\vskip .2cm
\noindent\textsc{Solución:}

(a)  Como
\begin{align*}
&(A + B)^2 = (A+B)(A+B) = AA +AB + BA +BB = A^2 + AB +BA + B^2  \; \text{ y}\\
&A^2 + 2AB + B^2 = A^2 + AB + AB + B^2, 
\end{align*}
tenemos que 
\begin{align*}
	(A + B)^2 = A^2 + 2AB + B^2 \quad &\Leftrightarrow \quad  A^2 + AB +BA + B^2  = A^2 +  AB + AB + B^2 \\
	&\Leftrightarrow \quad AB + BA = AB + AB \\
	&\Leftrightarrow \quad BA = AB.
\end{align*}

\vskip .4cm

(b) Como
\begin{align*}
	&A^2 - B^2  = AA -BB& \text{ y} &\qquad\qquad \\
	&(A - B)(A + B) = AA +AB -BA -BB,&& 
\end{align*}
tenemos que 
\begin{align*}
	A^2 - B^2 = (A - B)(A + B) \quad &\Leftrightarrow \quad AA -BB  = AA +AB -BA -BB \\
	&\Leftrightarrow \quad 0 = AB -BA \\
	&\Leftrightarrow \quad BA = AB.
\end{align*}
\qed




\item\label{ej:multiplicar por columna}  Sean
\begin{align*}
v=\begin{bmatrix} v_1 \\ \vdots \\ v_n
\end{bmatrix}\in\mathbb{R}^{n\times1}
\quad\mbox{y}\quad A=\begin{bmatrix} \mid& \mid& &\mid\\ C_1 & C_2 & \cdots &C_n\\ \mid& \mid& &\mid\end{bmatrix}
\in\mathbb{R}^{m\times n},
\end{align*}
es decir, $C_1, ..., C_n$ denotan las columnas de $A$. Probar que $Av=\sum_{j=1}^nv_jC_j$.
\vskip .2cm
\noindent\textsc{Solución:} Sea 
\begin{equation*}
	C_j = \begin{bmatrix}
		c_{1j} \\c_{2j} \\ \vdots \\ c_{mj} 
	\end{bmatrix}, \quad \text{para $1 \le j \le n$}.
\end{equation*}
$Av$  es una matriz  $m\times 1$ cuya coordenada $i,1$  es $[Av]_{i1} = F_i(C) \cdot v$ donde $\cdot$  es el producto escalar. Es  decir 
\begin{equation*}
	[Av]_{i1} = \sum_{j = 1}^n c_{ij}v_j, \qquad (1 \le i \le m).
\end{equation*}
Por lo tanto
\begin{equation*}
 Av = \begin{bmatrix}
	\sum_{j = 1}^n c_{1j}v_j \\ \vdots \\ \sum_{j = 1}^n c_{ij}v_j \\ \vdots \\ \sum_{j = 1}^n c_{mj}v_j
 \end{bmatrix} =
 \sum_{j = 1}^n
 \begin{bmatrix}
	 c_{1j}v_j \\ \vdots \\  c_{ij}v_j \\ \vdots \\ c_{mj}v_j
 \end{bmatrix} =
 \sum_{j = 1}^n v_j
 \begin{bmatrix}
	 c_{1j}\\ \vdots \\  c_{ij} \\ \vdots \\ c_{mj}
 \end{bmatrix} =
 \sum_{j = 1}^n v_j C_j.
\end{equation*}
\qed



\item\label{traza} Si $A$ es una matriz cuadrada $n\times n$, se define la {\it \textbf{traza}} de $A$ 
como $\operatorname{Tr}(A)=\displaystyle{\sum_{i=1}^n} a_{ii}$.
\begin{enumerate}[topsep=5pt,itemsep=5pt]
 \item Calcular la traza de las matrices del ejercicio  \ref{ej:inversas}. 
 \item\label{ej:traza} Probar que si $A,B\in\mathbb{R}^{n\times n}$ y $c\in\mathbb{R}$ entonces
 \begin{align*}
 \operatorname{Tr}(A+cB)=\operatorname{Tr}(A)+c\operatorname{Tr}(B)
 \quad\mbox{y}\quad
 \operatorname{Tr}(AB)=\operatorname{Tr}(BA).
 \end{align*}
\end{enumerate}
\vskip .2cm
\noindent\textsc{Solución:}
 (a) \begin{align*}
	A&= \begin{bmatrix*}[r] 3 & -1 & 2 \\ 2 & 1 & 1 \\ 1 & -3 & 0\end{bmatrix*} &\Rightarrow& \qquad \operatorname{Tr}(A)= 3+1+0 =4, \\
	B &= \begin{bmatrix*}[r] -1 & -1 &4 \\ 1 & 3 & 8 \\ 1 & 2 & 5\end{bmatrix*} &\Rightarrow& \qquad \operatorname{Tr}(B)=\ -1 +3 + 5 = 7,\\
	C &= \begin{bmatrix*}[r] 1 & 1 & 1 & 2 \\ 1 & -3 & 3 & -8 \\ -2 & 1 & 2 & -2 \\ 1 & 2 & 1 & 4 \end{bmatrix*} &\Rightarrow& \qquad \operatorname{Tr}(C)= 1 +(-3)+2+4 = 4,\\
	D &= \begin{bmatrix*}[r] 1 & -3 & 5 \\ 2 & -3 & 1 \\ 0 & -1 & 3 \end{bmatrix*}&\Rightarrow& \qquad \operatorname{Tr}(D)=1 + (-3) + 3 = 1.
	\end{align*}

(b) Sea $A = [a_{ij}]$ y $B = [b_{ij}]$,  entonces
\begin{align*}
	\operatorname{Tr}(A+cB)&=   \sum_{i=1}^n [A+cB]_{ii}\\
	&= \sum_{i=1}^n (a_{ii} + c b_{ii}) = \sum_{i=1}^n a_{ii} + c \sum_{i=1}^n  b_{ii}\\
	&=\operatorname{Tr}(A)+c\operatorname{Tr}(B)
\end{align*}

Veamos la segunda afirmación de (b):
\begin{align*}
	\operatorname{Tr}(AB)&=   \sum_{i=1}^n [AB]_{ii} = \sum_{i=1}^n (\sum_{j=1}^n  a_{ij}b_{ji}) \\
	&= \sum_{i,j=1}^n  a_{ij}b_{ji} = \sum_{j=1}^n (\sum_{i=1}^n  a_{ij}b_{ji}) \\
	&=  \sum_{j=1}^n (\sum_{i=1}^n  b_{ji}a_{ij}) =  \sum_{j=1}^n [BA]_{jj} \\
	&= \operatorname{Tr}(BA).
\end{align*}

\qed 

\item\label{ej:inversas} Para cada una de las siguientes matrices, usar operaciones elementales
por fila para decidir si son invertibles y hallar la matriz inversa cuando sea posible.
\begin{equation*}
\begin{bmatrix} 3 & -1 & 2 \\ 2 & 1 & 1 \\ 1 & -3 & 0\end{bmatrix},\qquad
\begin{bmatrix} -1 & -1 &4 \\ 1 & 3 & 8 \\ 1 & 2 & 5\end{bmatrix},\qquad
\begin{bmatrix} 1 & 1 & 1 & 2 \\ 1 & -3 & 3 & -8 \\ -2 & 1 & 2 & -2 \\ 1 & 2 & 1 & 4 \end{bmatrix},\qquad
\begin{bmatrix} 1 & -3 & 5 \\ 2 & -3 & 1 \\ 0 & -1 & 3 \end{bmatrix}.
\end{equation*}
(para que hagan menos cuentas: las matrices $3\times3$ aparecieron en el Práctico 2).
\vskip .2cm
\noindent\textsc{Solución:}

\begin{align*}
	[A|\operatornamewithlimits{Id}] &= 
	\begin{bmatrix}3 & -1 & 2&\bigm| &1 &0 & 0\\2 & 1 & 1&\bigm|& 0 &1 &0 \\1&-3&0&\bigm| &0 &0 &1\end{bmatrix}
	\stackrel{F_1 \leftrightarrow F_3}{\longrightarrow}
	\begin{bmatrix}1&-3&0&\bigm| &0 &0 &1\\2 & 1 & 1&\bigm|& 0 &1 &0 \\3 & -1 & 2&\bigm| &1 &0 & 0\end{bmatrix} \\
	&\stackrel{F_2 - 2 F_1}{\stackrel{F_3 - 3 F_1}{\longrightarrow}}
	\begin{bmatrix} 1 & -3 & 0  &\bigm|&0 &0 &1\\ 0 & 7 & 1 &\bigm|& 0 &1 &-2 \\ 0 & 8 & 2 &\bigm| &1 &0 & -3\end{bmatrix}
	\stackrel{F_3-F_2}{\longrightarrow}
	\begin{bmatrix} 1 & -3 & 0  &\bigm|&0 &0 &1\\ 0 & 7 & 1 &\bigm|& 0 &1 &-2 \\  0 & 1 & 1  &\bigm| &1 &-1 & -1\end{bmatrix}\\
	&\stackrel{F_3 \leftrightarrow F_2}{\longrightarrow} 
	\begin{bmatrix} 1 & -3 & 0  &\bigm|&0 &0 &1\\  0 & 1 & 1  &\bigm| &1 &-1 & -1\\ 0 & 7 & 1 &\bigm|& 0 &1 &-2 \end{bmatrix}
	\stackrel{F_1 + 3 F_2}{\stackrel{F_3-7F_2}{\longrightarrow}}
	\begin{bmatrix} 1 & 0 & 3 &\bigm|&3 & -3 &-2\\ 0 & 1 & 1 &\bigm| &1 &-1 & -1\\ 0 & 0 & -6 &\bigm|&-7&8&5\end{bmatrix} \\
	&\stackrel{F_3 (-\frac{1}{6}) }{\longrightarrow}
	\begin{bmatrix} 1 & 0 & 3 &\bigm|&3 & -3 &-2\\ 0 & 1 & 1 &\bigm| &1 &-1 & -1\\ 0 & 0 & 1 &\bigm|&7/6&-4/3&-5/6\end{bmatrix} \\
	&\stackrel{F_1 - 3 F_3}{\stackrel{F_2 - F_3 }{\longrightarrow}}
	\begin{bmatrix} 1 & 0 & 0&\bigm|& -1/2&1&1/2\\ 0 & 1 & 0&\bigm|&-1/6 &1/3&-1/6\\ 0 & 0 & 1&\bigm|&7/6&-4/3&-5/6 \end{bmatrix}
	\end{align*}
	Luego  
	\begin{equation*}
		\begin{bmatrix} 3 & -1 & 2 \\ 2 & 1 & 1 \\ 1 & -3 & 0\end{bmatrix}^{-1} = 
		\begin{bmatrix}-1/2&1&1/2\\ -1/6 &1/3&-1/6\\7/6&-4/3&-5/6 \end{bmatrix}. 
	\end{equation*}
\vskip .4cm
Siguiendo un procedimiento análogo al anterior se obtiene
\begin{align*}
	\begin{bmatrix} -1 & -1 &4 \\ 1 & 3 & 8 \\ 1 & 2 & 5\end{bmatrix}^{-1} = \begin{bmatrix} 1/6& -13/6& 10/3\\-1/2& 3/2& -2\\1/6& -1/6& 1/3\end{bmatrix}.
	\end{align*}

	\vskip .4cm
La tercera matriz tiene una MERF con una fila nula. Por lo tanto no es invertible:
\begin{align*}
	&\begin{bmatrix} 1 & 1 & 1 & 2 \\ 1 & -3 & 3 & -8 \\ -2 & 1 & 2 & -2 \\ 1 & 2 & 1 & 4 \end{bmatrix} 
	\stackrel{F_2 -  F_1}{\stackrel{F_3+2F_1}{\stackrel{F_4 - F_1 }{\longrightarrow}}}
	\begin{bmatrix} 1 & 1 & 1 & 2 \\ 0 & -4 & 2 & -10 \\ 0 & 3 & 4 & 2 \\ 0 & 1 & 0 & 2 \end{bmatrix} 
	\stackrel{F_1 -  F_4}{\stackrel{F_2+4F_4}{\stackrel{F_3 - 3F_1 }{\longrightarrow}}}
	\begin{bmatrix} 1 & 0 & 1 & 0 \\ 0 & 0 & 2 & -2 \\ 0 & 0 & 4 & -4 \\ 0 & 1 & 0 & 2 \end{bmatrix} \\
	&\quad
	\stackrel{F_2/2}{\stackrel{F_3/4}{\longrightarrow}}
	\begin{bmatrix} 1 & 0 & 1 & 0 \\ 0 & 0 & 1 & -1 \\ 0 & 0 & 1 & -1 \\ 0 & 1 & 0 & 2 \end{bmatrix}
	\stackrel{F_1-F_3}{\stackrel{F_2-F_3}{\longrightarrow}}
	\begin{bmatrix} 1 & 0 & 0 & 1 \\ 0 & 0 & 0 & 0 \\ 0 & 0 & 1 & -1 \\ 0 & 1 & 0 & 2 \end{bmatrix}
	\stackrel{F_2 \leftrightarrow F_4}{\longrightarrow}
	\begin{bmatrix} 1 & 0 & 0 & 1 \\ 0 & 1 & 0 & 2 \\ 0 & 0 & 1 & -1 \\ 0 & 0 & 0 & 0 \end{bmatrix}.
\end{align*}

\vskip .4cm
Finalmente, la última matriz  tiene una MERF con una fila nula. Por lo tanto no es invertible:
\begin{align*}
	&\left[\begin{array}{ccc}
	 1&-3&5\\
	 2&-3&1\\
	 0&-1&3\end{array}\right] 
	 \stackrel{F_2-2F_1}{\longrightarrow}   
	 \left[\begin{array}{ccc}
	1&-3&5\\
	 0&3&-9\\
	 0&-1&3\end{array}\right]
	 \stackrel{F_1-3F_3}{\stackrel{F_2+3F_3}{\longrightarrow}}  
	 \left[\begin{array}{ccc}
	 1&0&-4\\
	 0&0&0\\
	 0&-1&3\end{array}\right]  \\
	 &{\stackrel{-F_3}{\longrightarrow}}  \quad
	 \left[\begin{array}{ccc}
	 1&0&-4\\
	 0&0&0\\
	 0&1&-3\end{array}\right]{\stackrel{F_2 \leftrightarrow F_3}{\longrightarrow}}   
	  \left[\begin{array}{ccc}
	1&0&-4\\
	 0&1&-3\\
	0&0&0 \end{array}\right].
	\end{align*}

	\qed 


\item Sea $A$ la primera matriz del ejercicio anterior.
Hallar matrices elementales $E_1,E_2,\dots,E_k$ tales que $E_kE_{k-1}\cdots E_2E_1A=\operatorname{I}_3$.
\vskip .2cm
\noindent\textsc{Solución:} En  la primera matriz del ejercicio (\ref{ej:inversas}) realizamos 10 operaciones elementales de fila para llevar la matriz a la identidad. Si llamamos a la matriz $A$, entonces
$$
\operatorname{Id}_3= E_{10}E_9E_8E_7E_6E_5E_4E_3E_2E_1A,
$$
Donde 
\begin{align*}
	E_1 &= \begin{bmatrix} 0&0&1 \\	0&1&0 \\ 1&0&0\end{bmatrix}\;(F_1 \leftrightarrow F_3) ,\qquad
	E_2 = \begin{bmatrix} 1&0&0 \\	-2&1&0 \\ 0&0&1\end{bmatrix}\; (F_2-2F_1),\qquad \\
	E_3 &= \begin{bmatrix} 1&0&0 \\	0&1&0 \\ -3&0&1\end{bmatrix}\; (F_3-3F_1),\qquad 
	E_4 = \begin{bmatrix} 1&0&0 \\	0&1&0 \\ 0&-1&1\end{bmatrix}\; (F_3-F_2),\qquad  \\
	E_5 &= \begin{bmatrix} 1&0&0 \\	0&0&1 \\ 0&1&0\end{bmatrix}\; (F_3 \leftrightarrow F_2),\qquad 
	E_6 = \begin{bmatrix} 1&3&0 \\	0&1&0 \\ 0&0&1\end{bmatrix}\; (F_1 +3F_2),\qquad \\
	E_7 &= \begin{bmatrix} 1&0&0 \\	0&1&0 \\ 0&-7&1\end{bmatrix}\; (F_3-7F_2),\qquad 
	E_8 = \begin{bmatrix} 1&0&0 \\	0&1&0 \\ 0&0&-1/6\end{bmatrix}\; (-(1/6)F_3),\qquad \\
	E_9 &= \begin{bmatrix} 1&0&-3 \\	0&1&0 \\ 0&0&1\end{bmatrix}\; (F_1-3F_3),\qquad 
	E_{10} = \begin{bmatrix} 1&0&0 \\	0&1&-1 \\ 0&0&1\end{bmatrix}\; (F_2-F_3).\qquad 
\end{align*}

\qed



\item ¿Es cierto que si $A$ y $B$ son matrices invertibles entonces $A+B$ es una matriz invertible? Justificar su respuesta.
\vskip .2cm
\noindent\textsc{Solución:} Es  falso, el ejemplo más sencillo es $\operatorname{Id} +(-\operatorname{Id}) =0$. Otro ejemplo, 
\begin{equation*}
	\begin{bmatrix}  1&0\\0&1 \end{bmatrix} + \begin{bmatrix}  1&1\\0&-1 \end{bmatrix} = \begin{bmatrix}  2&1\\0&0 \end{bmatrix}. 
\end{equation*}\qed



\item\label{nilpotene - id} Una matriz $A\in\mathbb{R}^{n\times n}$ se dice \emph{nilpotente} si $A^k=0$ para algún $k\in\mathbb{N}$.
Probar que si una matriz $A$ es nilpotente, entonces  $\operatorname{I}_n - A$  es invertible.
\vskip .2cm
\noindent\textsc{Solución:} Supongamos que $A^k=0$ para algún $k\in\mathbb{N}$, y probemos que: $$(\operatorname{I}_n - A)^{-1} = \operatorname{I}_n + A + A^2 +\cdots + A^{k-1} = \sum_{i=0}^{k-1} A^{i}.$$
En efecto,
\begin{align*}
& (\operatorname{I}_n - A)(\operatorname{I}_n + A + A^2 +\cdots + A^{k-1})\\
&= \operatorname{I}_n (\operatorname{I}_n + A + A^2 +\cdots + A^{k-1}) - A (\operatorname{I}_n + A + A^2 +\cdots + A^{k-1})\\
&= (\operatorname{I}_n + A + A^2 +\cdots + A^{k-1}) - (A + A^2 + A^3 +\cdots + A^{k})\\
&= \operatorname{I}_n - A^{k} = \operatorname{I}_n - 0 = \operatorname{I}_n.
\end{align*}

\qed


\item\label{sol homog es subesp} Sean  $v$ y $w$ dos soluciones del sistema homogéneo $AX=0$. Probar que $v+tw$ también es solución para todo $t\in\mathbb{K}$.
\vskip .2cm
\noindent\textsc{Solución:} Como $v,w$  soluciones de $AX=0$, tenemos que $Av=0$ y $Aw=0$.  Luego, por las propiedades que cumple el producto y suma de matrices:
\begin{eqnarray*}
	A(v + tw) = Av +A(tw) = 0 + t(Aw) = 0 + 0 = 0.
\end{eqnarray*}
Es decir, $v +tw$  es solución de $AX=0$.
\qed


\item\label{homogeneo+no-homogeneo} Sea $v$ una solución del sistema $AX=Y$ y $w$ una solución del sistema homogéneo $AX=0$. Probar que $v+tw$ también es solución del sistema $AX=Y$ para todo $t\in\mathbb{K}$.
\vskip .2cm
\noindent\textsc{Solución:}  Como $v$  solución de $AX=Y$, tenemos que $Av=Y$. Al ser $w$  solución del sistema $AX=0$, se cumple $Aw=0$. Luego, por las propiedades que cumple el producto y suma de matrices:
\begin{eqnarray*}
	A(v + tw) = Av +A(tw) = Y + t(Aw) = Y + 0 = Y.
\end{eqnarray*}
Es decir, $v +tw$  es solución de $AX=Y$.
\qed


\item Probar que si el sistema homogéneo  $AX=0$ posee alguna solución no trivial, entonces el sistema $AX=Y$ no tiene
solución o tiene al menos dos soluciones distintas.
\vskip .2cm
\noindent\textsc{Solución:} Si $AX=Y$ no tiene solución, listo, pues es uno de los casos. Si $AX=Y$ tiene solución,  sea entonces $w$ alguna solución del sistema, es decir $Aw=Y$. Por hipótesis $AX=0$ tiene soluciones no triviales,  es decir existe $v \ne 0$ tal que $Av =0$. Por el ejercicio  (\ref{homogeneo+no-homogeneo}): $v+ w$ es solución del sistema $AX=Y$, como $v \ne 0$, los vectores $w$ y $v+w$ son distintos y ambos son solución del sistema  $AX=Y$.
\vskip .2cm
\noindent \textbf{Observación.} En  realidad,  la existencia de una solución $w$ del sistema $AX=Y$ implica la existencia de infinitas soluciones, pues por ejercicio   (\ref{homogeneo+no-homogeneo}) los vectores $w + tv$ son soluciones de $AX=Y$ para todo $t \in \mathbb K$. 

\qed


\item Supongamos que los sistemas $AX=Y$ y $AX=Z$ tienen solución. Probar que el sistema $AX=Y+tZ$ también tiene solución para todo $t\in\mathbb{K}$.
\vskip .2cm
\noindent\textsc{Solución:} Sea $v$ solución de $AX=Y$, es decir $Av = Y$ y sea $w$ solución del sistema $AX=Z$,  es decir $Aw=Z$. Entonces, dado $t \in \mathbb K$
\begin{equation*}
	A(v + tw) = Av + Atw = Y +tAw = Y + tZ.
\end{equation*}
Es decir $v +tw$ es solución de $AX=Y+tZ$. 

\qed


\item \label{sist-por-invertible}Sean $A$ una matriz invertible $n\times n$, y $B$ una matriz $n\times m$.  Probar que los sistemas $BX=Y$ y $ABX=AY$ tienen las mismas soluciones.
\vskip .2cm
\noindent\textsc{Solución:} 
\begin{align*}
	\text{$v$ es vsol. de $BX=Y$} \; &\Rightarrow \; Bv = Y&& \\
	&\Rightarrow \;  ABv = AY &&(\text{mult. por $A$  a izq.}) \\
	&\Rightarrow \;   \text{$v$ es sol. de $ABX=AY$}&&
\end{align*}
Esta ``ida'' vale para cualquier matriz $A$,  sea invertible o no. Para la vuelta debemos utilizar la existencia de $A^{-1}$:
\begin{align*}
	\text{$v$ es sol. de $ABX=AY$} \; &\Rightarrow \; ABv = AY&& \\
	&\Rightarrow \;  A^{-1}ABv = A^{-1}AY &&(\text{mult. por $A^{-1}$  a izq.}) \\
	&\Rightarrow \;  \operatorname{Id} Bv = \operatorname{Id}Y &&(\text{ $A^{-1}A= \operatorname{Id}$}) \\
	&\Rightarrow \;   Bv =Y&&(\text{$\operatorname{Id}$ es neutro del prod.}) \\
	&\Rightarrow \;   \text{$v$ es sol. de $BX=Y$}&&
\end{align*}

\qed


\item\label{ej:sistemas ABX} 
Sean $A$ y $B$ matrices $r\times n$ y $n\times m$ respectivamente.
Probar que:
\begin{enumerate}[topsep=5pt,itemsep=5pt]
	\item  Si $m>n$, entonces el sistema $ABX=0$ tiene soluciones no triviales.
	\item  Si $r>n$, entonces existe un $Y$, $r\times 1$, tal que $ABX=Y$
	no tiene solución.
\end{enumerate}
\vskip .2cm
\noindent\textsc{Solución:}

 (a) Como $m> n$ el sistema $BX =0$ tiene más incógnitas que ecuaciones, por lo tanto tiene soluciones no triviales. Sea $v \ne 0$ solución de $BX=0$,  es decir $Bv=0$. Entonces $ABv = A(Bv) = A0 =0$, por lo tanto $v$  es solución del sistema $ABX=0$.  

 \vskip .2cm
(b) Sea $P$  matriz $r \times r$ inversible tal que $PA$ es MERF. Como $r > n$,  la  matriz $PA$  tiene  más filas que columnas y como es MERF la última fila debe ser nula.  

Ahora bien, por el ejercicio (\ref{sist-por-invertible}), los sistemas $PABX=PY$ y $ABX=Y$ tiene las mismas soluciones, por lo tanto si existe $Y$ tal que $PABX=PY$ no tiene solución, entonces el sistema $ABX=Y$ tampoco tiene solución. 

Demostremos, entonces, que existe $Y$  tal que $PABX=PY$ no tiene solución: como $PA$ tiene la última fila nula, $PABX$ también tiene la última fila nula. Sea $e_r$ la matriz $r \times 1$ con $1$ en la coordenada $r$ y $0$  en las otras coordenadas. Entonces $e_r = P(P^{-1}e_r)$  tiene la última fila no nula, por lo tanto el sistema $PABX=P(P^{-1}e_r)$ no tiene solución y, por lo dicho anteriormente, el sistema   $ABX=P^{-1}e_r$ no tiene solución.

\qed 

%%%%%%%%%%%%%%%%%%%%%%%%%%%%%%%%%%%%%%%%%%%%%%%%%%%%%%%%%%%%%%%%%%%%%%%%%%%%%%%%%%%%%%%%%%%%%%%%%%%%%%%%%%%%%%%%%%%%%%%%%%%%%%%%%%%%%%%%%%%%%%%%%%%%%%%%

\end{enumerate}




\end{document}

%%%=======================
%%%=======================
%%% CAP4 =================
    
\chapter{Soluciones\\Álgebra  II -- Año 2024/1 -- FAMAF}\label{practico-4}

\begin{enumerate}[topsep=6pt,itemsep=.4cm]
    \item Calcular el determinante de las siguientes matrices.
        \begin{align*}
        &A=\begin{bmatrix} 4&7\\ 5&3\end{bmatrix},
        &&B=\begin{bmatrix} -3&2&4\\ 1&-1&2\\ -1&4&0\end{bmatrix},
        &&
        C=\begin{bmatrix} 2&3&1&1\\ 0&2&-1&3 \\ 0&5&1&1 \\1&1&2&5\end{bmatrix}.
        \end{align*}
        
        \rta
        \begin{align*}
            \left| A \right| &= 4\cdot 3 - 7\cdot 5 = -23,\\
            \left| B \right| &= -3\cdot \left|\begin{matrix} -1&2\\ 4&0\end{matrix}\right| - 1\cdot \left|\begin{matrix} 2&4\\ 4&0\end{matrix}\right| - 1\cdot \left|\begin{matrix} 2&4\\ -1&2\end{matrix} \right|\\
            &= -3\cdot (-8) - 1\cdot (-16) - 1\cdot 8\\ 
            &= 24 + 16 -8 = 32,\\
            \left| C \right| &= 2 \cdot\left| \begin{matrix} 2&-1&3\\ 5&1&1\\ 1&2&5\end{matrix} \right| - 0 \cdot \left| C(2|1) \right| + 0 \cdot \left| C(3|1) \right| -1 \cdot\left|  \begin{matrix} 3&1&1\\ 2&-1&3 \\ 5&1&1\end{matrix} \right| \tag{*}
        \end{align*}
        Debemos ahora calcular el determinante de las matrices $3 \times 3$ que aparecen en la expresión de $|C|$. 
        \begin{align*}
            \left|\begin{matrix} 2&-1&3\\ 5&1&1\\ 1&2&5\end{matrix} \right| &= 2\cdot  \left|\begin{matrix} 1&1\\ 2&5\end{matrix} \right| - 5\cdot  \left|\begin{matrix} -1&3\\ 2&5\end{matrix} \right| + 1\cdot  \left|\begin{matrix} -1&3\\ 1&1\end{matrix} \right| \\
            &= 2\cdot 3  - 5\cdot (-11) + 1\cdot (-4) = 57,\\
            \left|  \begin{matrix} 3&1&1\\ 2&-1&3 \\ 5&1&1\end{matrix} \right| &= 3\cdot  \left| \begin{matrix} -1&3\\ 1&1\end{matrix} \right| - 2\cdot  \left|  \begin{matrix} 1&1\\ 1&1\end{matrix} \right| + 5 \cdot  \left|  \begin{matrix} 1&1\\ -1&3\end{matrix} \right| \\
            &= 3\cdot (-4) - 2\cdot 0 + 5\cdot 4 = -12 + 20 = 8.
        \end{align*}
        Luego, por (*):
        \begin{align*}
            \left| C \right| &= 2 \cdot 57 -1 \cdot 8 = 106.
        \end{align*} 
        \qed
    
    \item Sean
            $$A=
        \begin{bmatrix} 1&3&2 \\ 3&0&2 \\  1&1&1 \end{bmatrix}, \qquad
        B =
        \begin{bmatrix} 1&-1&2\\ 1&1&1 \\ -1&-1&3 \end{bmatrix}.
        $$
        Calcular:
        \begin{multicols}{3}
        \begin{enumerate}
            \item\label{det-AB} $\det(AB)$.
            \item\label{det-BA} $\det(BA)$.
            \item\label{det-A-1} $\det(A^{-1})$.
            \item\label{det-A4} $\det(A^{4})$.
            \item\label{det-A+B} $\det(A+B)$.
            \item\label{det-A+tB} $\det(A+tB)$, con $t \in \mathbb{R}$.
        \end{enumerate}
    \end{multicols}
    \rta

    Primero nos conviene calcular los determinantes de $A$ y $B$, pues algunos cálculos se reducen a saber estos números. 
    \begin{align*}
        \det(A) =\left|\begin{matrix} 1&3&2 \\ 3&0&2 \\  1&1&1 \end{matrix}\right| &= 1\cdot  \left|\begin{matrix} 0&2 \\  1&1\end{matrix} \right| - 3\cdot  \left|\begin{matrix}3&2 \\ 1&1\end{matrix} \right| + 1\cdot  \left|\begin{matrix} 3&2 \\ 0&2\end{matrix} \right| \\
        &= 1\cdot (-2)  - 3\cdot 1 + 1\cdot 6 = 1,\\
        \det(B) =\left|  \begin{matrix}  1&-1&2\\ 1&1&1 \\ -1&-1&3 \end{matrix} \right| &= 1\cdot  \left| \begin{matrix}  1&1 \\ -1&3 \end{matrix} \right| - 1\cdot  \left|  \begin{matrix} -1&2 \\ -1&3\end{matrix} \right| -1  \cdot  \left|  \begin{matrix} -1&2\\ 1&1\end{matrix} \right| \\
        &= 1\cdot 4 - 1\cdot (- 1)  -1\cdot (-3) = 8.
    \end{align*}

    Resumiendo $\det(A) = 1$ y $\det(B) = 8$. Ahora sí, calculemos los determinantes pedidos.
    
    \vskip .2cm
    \ref{det-AB} $\det(AB) = \det(A)\det(B) = 1 \cdot 8 = 8$.

    \vskip .2cm
    \ref{det-BA} $\det(BA) = \det(B)\det(A) = 8 \cdot 1 = 8$.

    \vskip .2cm
    \ref{det-A-1} $\det(A^{-1}) = 1/ \det(A) = 1/1 = 1$.
    
    \vskip .2cm
    \ref{det-A4} $\det(A^{4}) = \det(A)^{4} = 1^4 = 1$.
    
    \vskip .2cm
    \ref{det-A+tB} El ejercicio \ref{det-A+B} es un caso especial de \ref{det-A+tB} para $t=1$.  Así que haremos este inciso primero. Para ello, antes que nada, calculemos la matriz $A + tB$:
    \begin{align*}
        A+ tB &= \begin{bmatrix} 1&3&2 \\ 3&0&2 \\  1&1&1 \end{bmatrix} + t \begin{bmatrix} 1&-1&2\\ 1&1&1 \\ -1&-1&3 \end{bmatrix} \\
        &= \begin{bmatrix} 1+t&3-t&2+2t \\ 3+t&t&2+t \\  1-t&1-t&1+3t \end{bmatrix}.
    \end{align*}
    Entonces
    \begin{align*}
       \left| \begin{matrix} 1+t&3-t&2+2t \\ 3+t&t&2+t \\  1-t&1-t&1+3t \end{matrix}\right| & = (1+t)\cdot  \left| \begin{matrix} t&2+t \\  1-t&1+3t \end{matrix} \right| - (3+t)\cdot  \left| \begin{matrix} 3-t&2+2t \\  1-t&1+3t \end{matrix} \right|\\
       &\qquad\qquad\qquad\qquad\qquad\qquad + (1-t)\cdot  \left| \begin{matrix} 3-t&2+2t \\ t&2+t  \end{matrix} \right| \\
       &= (1+t)\cdot (t(1+3t) - (2+t)(1-t)) -\\
       &\quad- (3+t)\cdot ((3-t)(1+3t) - (2+2t)(1-t)) +\\
       &\quad + (1-t)\cdot ((3-t)(2+t) - (2+2t)t)\\
         &= (1+t)\cdot (4 t^2 + 2 t - 2) +\\
         &\quad- (3+t)\cdot (-t^2 + 8 t + 1) +\\
            &\quad + (1-t)\cdot (-3 t^2 - t + 6)
            \\
            &= (4 t^3 + 6 t^2 - 2) + (t^3 - 5 t^2 - 25 t - 3) + (3 t^3 - 2 t^2 - 7 t + 6
            )\\
            &=8 t^3 - t^2 - 32 t + 1
    \end{align*}

    \vskip .2cm
    \ref{det-A+B} $\det(A+B) = \det(a + 1\cdot B) \stackrel{\ref{det-A+tB}}{=}8 \cdot 1^3 - 1^2 - 32 \cdot 1 + 1 = -24.$


    \qed
    
    \item Calcular el determinante de las siguientes matrices haciendo la reducción a matrices triangulares superiores.
    
            $$A =
            \begin{bmatrix}
                a&1&1&1 \\
                1&a&1&1 \\
                1&1&a&1 \\
                1&1&1&a \\
            \end{bmatrix}, \qquad    
            B =
            \begin{bmatrix}
                1&1&1&1&1 \\
                1&3&3&3&3 \\
                1&3&5&5&5 \\
                1&3&5&7&7 \\
                1&3&5&7&9 \\
            \end{bmatrix}.
            $$
            \rta Recordemos que si $A$ es una matriz triangular superior, entonces $\det(A)$ es el producto de los elementos de la diagonal principal. Por otro  lado, las operaciones elementales por fila  tienen el siguiente efecto en el cálculo del determinante:
            \begin{itemize}
                \item el multiplicar una fila por un escalar $\lambda$ cambia el determinante por $\lambda$ veces el determinante anterior,
                \item intercambiar dos filas cambia el signo del determinante, y
                \item sumar a una fila un múltiplo de otra fila no cambia el determinante.
            \end{itemize}

            Es claro la matriz original tiene todas las filas iguales y por lo tanto $\det(A) = 0$. 
            
            Analicemos el caso  $a\ne 1$.

            Veamos como reducimos $A$ a triangular superior:
            \begin{align*}
                A \underset{F_4 -F_2}{\underset{F_3 -F_2}{\stackrel{F_1-aF_2}{\longrightarrow}}} &=  \begin{bmatrix}
                    0&1-a^2&1-a&1-a \\ 
                    1&a&1&1 \\
                    0&1-a&a-1&0 \\
                    0&1-a&0&a-1 \\
                \end{bmatrix}
                \underset{F_4 / (1-a)}{\underset{F_3 / (1-a)}{\stackrel{F_1 /(1-a)}{\longrightarrow}}} 
                \begin{bmatrix}
                    0&1+a&1&1 \\ 
                    1&a&1&1 \\
                    0&1&-1&0 \\
                    0&1&0&-1 \\
                \end{bmatrix} \\
                &\underset{F_3 -F_4}{\underset{F_2 -aF_4}{\stackrel{F_1-(1+a)F_4}{\longrightarrow}}}
                \begin{bmatrix}
                    0&0&1&2+a \\ 
                    1&0&1&1+a \\
                    0&0&-1&1 \\
                    0&1&0&-1 \\
                \end{bmatrix} 
                \stackrel{(-1)F_3}{\longrightarrow}
                \begin{bmatrix}
                    0&0&1&2+a \\ 
                    1&0&1&1+a \\
                    0&0&1&-1 \\
                    0&1&0&-1 \\
                \end{bmatrix} 
            \end{align*}
            
           






            \qed

    \item Sean $A$, $B$ y $C$ matrices $n\times n$, tales que $\det A=-1$, $\det B=2$ y $\det C=3$.
    Calcular:
    
    \begin{enumerate}
    \item $\det(PQR)$, donde $P$, $Q$ y $R$ son las matrices que se obtienen a partir de $A$, $B$ y $C$ mediante operaciones elementales por filas de la siguiente manera
     \begin{align*}
     A\overset{F_1+2F_2}{\longrightarrow} P,\quad
     B\overset{3F_3}{\longrightarrow} Q
     \quad\mbox{y}\quad
     C\overset{F_1\leftrightarrow F_4}{\longrightarrow} R.
     \end{align*}
     Es decir,
     \begin{itemize}
      \item[$\circ$] $P$ se obtiene a partir de $A$ sumando a la fila $1$ la fila $2$ multiplicada por $2$.
      \item[$\circ$] $Q$ se obtiene a partir de $B$ multiplicando la fila $3$ por $3$.
      \item[$\circ$] $R$ se obtiene a partir de $C$ intercambiando las filas $1$ y $4$.
     \end{itemize}
        \item $\det(A^2BC^tB^{-1})$ \ y \ $\det(B^2C^{-1}AB^{-1}C^{t})$.
    \end{enumerate}
    \rta

    \qed
    
    \item  Sea
    $$A=
    \begin{bmatrix}
        x&y&z \\
        3&0&2\\
        1&1&1
    \end{bmatrix}.$$
    Sabiendo que $\det(A) = 5$, calcular el determinante de las siguientes matrices.
    $$
    B = \begin{bmatrix}
    2x&2y&2z \\
    3/2&0&1\\
    1&1&1
    \end{bmatrix}, \qquad
    C=
    \begin{bmatrix}
        x&y&z \\
        3x+3&3y&3z+2\\
        x+1&y+1&z+1
    \end{bmatrix}.
    $$
    \rta

    \qed
    
    \item Determinar todos los valores de $c\in\mathbb{R}$ tales que las siguientes matrices sean invertibles.
    \begin{align*}
    A=\begin{bmatrix}4& c&3\\c&2&c\\ 5&c&4 \end{bmatrix},\qquad
    B=\begin{bmatrix} 1&c&-1\\ c&1&1\\0&1&c\end{bmatrix}.
    \end{align*}
    \rta

    \qed
    
    \item Calcular el determinante de las siguientes matrices, usando operaciones elementales por fila y/o columnas u otras propiedades del determinante. Determinar cuáles de ellas son invertibles.
    
    \begin{align*}
    &A=
    \begin{bmatrix}
    -2&3&2&-6\\ 0&4&4&-5\\ 5&-6&-3&2\\ -3&7&0&0 \end{bmatrix},\quad
    &&B=\begin{bmatrix} 2&0&0&0\\ 0&0&3&0\\ 0&-1&0&0\\ 0&0&0&4\end{bmatrix},\quad
    &&
    C=\begin{bmatrix}
      -2&3&2&-6&0\\
    0&4&4&-5&0\\
    5&-6&-3&2&0\\
    -3&7&0&0&0\\
    1&1&1&1&1
      \end{bmatrix},
    \end{align*}
    \begin{align*}
    D=\begin{bmatrix}
    1&2&3&0&0\\
    -1&2&-13&6&\frac{1}{3}\\
    2&0&0&0&0\\
    11&1&0&0&0\\
    \sqrt{2}&2&1&\pi&0\\
    \end{bmatrix},&&
    E=\begin{bmatrix}
    1&-1&2&0&0\\ 3&1&4&0&0\\ 2&-1&5&0&0 \\0&0&0&2&1\\ 0&0&0&-1&4
    \end{bmatrix}.
    \end{align*}
    \rta

    \qed
    
    \item Sean $A$ y  $B$ matrices $n \times n$. Probar que:
    \begin{enumerate}
        \item $\det(AB) = \det (BA)$.
        \item Si $B$ es invertible, entonces $\det(B A B^{-1}) = \det (A)$.
        \item\label{-A} $\textcircled{a}$ $\det(-A) = (-1)^n\det (A)$.
    \end{enumerate}
    \rta

    \qed
    
    \item\label{vandermonde} Sean $\lambda_1, \lambda_2, \dots, \lambda_n$ escalares, la matriz de \emph{Vandermonde} asociada es
    \begin{align*}
    \mathtt V = \begin{bmatrix}
    1 & \lambda_1 & \lambda_1^2 & \cdots & \lambda_1^{n-1}\\
    1 & \lambda_2 & \lambda_2^2 & \cdots & \lambda_2^{n-1}\\
    \vdots &\vdots &\vdots & &\vdots\\
    1 & \lambda_n & \lambda_n^2 & \cdots & \lambda_n^{n-1}\\
    \end{bmatrix}.
    \end{align*}
    Esta es la matriz del sistema de ecuaciones del ejercicio \ref{polinomios}\,\ref{polinomios-c} del Práctico \ref{practico-2}.
    
    
    \begin{enumerate}
        \item Si $n=2$, probar que $\det(\mathtt V) = \lambda_2-\lambda_1$.
        \item Si $n=3$, probar que $\det(\mathtt V) = (\lambda_3-\lambda_2) (\lambda_3-\lambda_1) (\lambda_2-\lambda_1)$.
        \item\label{vandermonde gral} $\textcircled{a}$ Probar que $\det(\mathtt V) = \prod_{1\leq i< j \leq n}(\lambda_j-\lambda_i)$ para todo $n\in\mathbb{N}$.
        \item Dar una condición necesaria y suficiente para que la matriz de Vandermonde sea invertible.
        \item Usar lo anterior para responder a la pregunta del ejercicio \ref{polinomios}\,\ref{polinomios-b} del Práctico \ref{practico-2}.
        \end{enumerate}
    \rta

    \qed
        
    \item Decidir si las siguientes afirmaciones son verdaderas o falsas. Justificar con una demostración o con un contraejemplo, según corresponda.
        \begin{enumerate}
        \item Sean $A$ y $B$ matrices $n \times n$. Entonces $\det(A + B) = \det (A) + \det(B)$.
        \item Existen una matriz $3\times 2$, $A$, y una matriz $2\times 3$, $B$, tales que $\det(AB) \neq 0$.
        \item Sea $A$ una matriz $n\times n$. Si $A^n$ es no invertible, entonces $A$ es no invertible.
    \end{enumerate}
    \rta

    \qed
    
    \end{enumerate}
    
%%%=======================
%%%=======================
%%% CAP5 =================
    
\chapter{Soluciones\\Álgebra  II -- Año 2024/1 -- FAMAF}\label{practico-5}

Completar.
%%%=======================
%%% CAP6 =================
    
\chapter{Soluciones\\Álgebra  II -- Año 2024/1 -- FAMAF}\label{practico-6}


\begin{enumerate}[topsep=6pt, itemsep=.4cm]

    
    \item\label{sub Rn} Decidir si los siguientes subconjuntos de $\mathbb{R}^3$ son subespacios vectoriales.
        \begin{enumerate}
            \item\label{sub Rn 1} $A=\{(x_1, x_2 ,x_3) \in \mathbb{R}^3 \ : \ x_1 + x_2 + x_3=1\}$.
            \item\label{sub Rn 0} $B=\{(x_1, x_2 ,x_3) \in \mathbb{R}^3 \ : \ x_1 + x_2 + x_3=0\}$.
            \item\label{sub Rn geq} $C=\{(x_1, x_2 ,x_3) \in \mathbb{R}^3 \ : \ x_1 + x_2 + x_3 \geq 0\}$.
            \item\label{sub Rn 1 30} $D=\{(x_1, x_2 ,x_3) \in \mathbb{R}^3 \ : \ x_3=0\}$.
            \item\label{sub Rn cup} $B\cup D$.
            \item\label{sub Rn cap} $B\cap D$.
            \item\label{sub Rn q} $G=\{(x_1, x_2 ,x_3) \in \mathbb{R}^3 \ :\ x_1, x_2, x_3\in\mathbb{Q}\}$.
        \end{enumerate}
        
    \rta 

    \ref{sub Rn 1} No es subespacio vectorial. Por ejemplo $(1,0,0)$ y $(0,1,0)$ pertenecen a $A$, pero $(1,0,0)+(0,1,0)=(1,1,0)$ no pertenece a $A$.

    \ref{sub Rn 0} Es subespacio vectorial. En efecto, si $(x_1, x_2 ,x_3)$ y $(y_1, y_2 ,y_3)$ pertenecen a $B$ y $\lambda,\mu\in\mathbb{R}$, entonces
    \begin{align*}
        \lambda(x_1, x_2 ,x_3)+\mu(y_1, y_2 ,y_3)&=(\lambda x_1+\mu y_1, \lambda x_2+\mu y_2, \lambda x_3+\mu y_3)\\
        &=(\lambda x_1+\mu y_1)+(\lambda x_2+\mu y_2)+(\lambda x_3+\mu y_3)\\
        &=\lambda(x_1+ x_2 + x_3)+\mu(y_1+ y_2 + y_3)\\
        &=\lambda\cdot 0+\mu\cdot 0=0.
    \end{align*}

    \ref{sub Rn geq} No es subespacio vectorial. Por ejemplo $(1,0,0) \in C$ pero $(-1)(1,0,0) = (-1,0,0) \not\in C$, pues $-1+0+0<0$.

    \ref{sub Rn 1 30} Es subespacio vectorial. En efecto, si $(x_1, x_2 ,0)$ y $(y_1, y_2 ,0)$ pertenecen a $D$ y $\lambda,\mu\in\mathbb{R}$, entonces
    \begin{align*}
        \lambda(x_1, x_2 ,0)+\mu(y_1, y_2 ,0)&=(\lambda x_1+\mu y_1, \lambda x_2+\mu y_2, 0) \in D.
    \end{align*}

    \ref{sub Rn cup} No es subespacio vectorial. Por ejemplo $(1,0,-1) \in B$ y $(0,1,0)$ pertenecen a $B\cup D$, pero $(1,0,-1)+(0,1,0)=(1,1,-1)$ no pertenece a $B\cup D$, pues $(1,1,-1) \not\in B$ y $(1,1,-1) \not\in D$.

    \ref{sub Rn cap} Es subespacio vectorial
    \begin{align*}
        B \cap D &= \{(x_1, x_2 ,x_3) \in \mathbb{R}^3 \ : \ x_1 + x_2 + x_3=0 \text{ y } x_3=0\} \\&= \{(x_1, x_2 ,0) \in \mathbb{R}^3 \ : \ x_1 + x_2 =0\}.    
    \end{align*}
    Luego, si $(x_1, x_2 ,0)$ y $(y_1, y_2 ,0)$ pertenecen a $B\cap D$ y $\lambda,\mu\in\mathbb{R}$, entonces
    \begin{align*}
        \lambda(x_1, x_2 ,0)+\mu(y_1, y_2 ,0)&=(\lambda x_1+\mu y_1, \lambda x_2+\mu y_2, 0) \in B\cap D,
    \end{align*}
    pues $\lambda x_1+\mu y_1 + \lambda x_2+\mu y_2 = \lambda(x_1+ x_2) + \mu(y_1+ y_2) = \lambda(x_1+ x_2) + \mu(y_1+ y_2) = \lambda 0 + \mu 0 = 0$.

    \ref{sub Rn q} No es subespacio vectorial. Por ejemplo $(1,0,0)$ pertenece a $G$, pero $\sqrt{2}(1,0,0)=(\sqrt{2},0 ,0)$ no pertenece a $G$.


    \qed     

    \end{enumerate}

    
    \begin{enumerate}[resume, topsep=6pt, itemsep=.4cm]
    
    \item\label{sub matrices} Decidir en cada caso si el conjunto dado es un subespacio vectorial de $M_{n\times n}(\mathbb{K})$.
    \begin{enumerate}
        \item\label{sub matrices invertibles} El conjunto de matrices  invertibles.
        \item\label{sub matrices AB} El conjunto de matrices $A$ tales que $AB = BA$, donde $B$ es una matriz fija.
        \item\label{sub matrices triangulares} El conjunto de matrices triangulares superiores.
    \end{enumerate}
    
    \rta 

    \ref{sub matrices invertibles} No es subespacio vectorial. Por ejemplo, $\Id_n$ y $-\Id_n$ son matrices invertibles,  pero $\Id_n+(-\Id_n)=0$ no es invertible.

    \vskip .2cm
    \ref{sub matrices AB} Es subespacio vectorial. En efecto, si $A$ y $A'$ pertenecen al conjunto y $\lambda,\mu\in\mathbb{R}$, entonces
    \begin{align*}
        (\lambda A+\mu A')B&=\lambda AB+\mu A'B&&\\
        &=\lambda BA+\mu BA'&&\text{(por hipótesis)}\\
        &=(\lambda A+\mu A')B.
    \end{align*}
    Luego $\lambda A+\mu A'$ pertenece al conjunto.

    \vskip .2cm
    \ref{sub matrices triangulares} Es subespacio vectorial. En efecto, si $A$ y $A'$ son matrices triangulares superiores y $\lambda,\mu\in\mathbb{R}$, entonces
    \begin{align*}
        \lambda A+\mu A' &=\lambda\begin{bmatrix}
            a_{11} & a_{12} & \cdots & a_{1n} \\
            0 & a_{22} & \cdots & a_{2n} \\
            \vdots & \vdots & \ddots & \vdots \\
            0 & 0 & \cdots & a_{nn} \\
        \end{bmatrix}+\mu\begin{bmatrix}
            a'_{11} & a'_{12} & \cdots & a'_{1n} \\
            0 & a'_{22} & \cdots
            & a'_{2n} \\
            \vdots & \vdots & \ddots & \vdots \\
            0 & 0 & \cdots & a'_{nn} \\
        \end{bmatrix}\\
        &=\begin{bmatrix}
            \lambda a_{11} & \lambda a_{12} & \cdots & \lambda a_{1n} \\
            0 & \lambda a_{22} & \cdots & \lambda a_{2n} \\
            \vdots & \vdots & \ddots & \vdots \\
            0 & 0 & \cdots & \lambda a_{nn} \\
        \end{bmatrix}+\begin{bmatrix}
            \mu a'_{11} & \mu a'_{12} & \cdots & \mu a'_{1n} \\
            0 & \mu a'_{22} & \cdots & \mu a'_{2n} \\
            \vdots & \vdots & \ddots & \vdots \\
            0 & 0 & \cdots & \mu a'_{nn} \\
        \end{bmatrix}\\
        &=\begin{bmatrix}
            \lambda a_{11}+\mu a'_{11} & \lambda a_{12}+\mu a'_{12} & \cdots & \lambda a_{1n}+\mu a'_{1n} \\
            0 & \lambda a_{22}+\mu a'_{22} & \cdots & \lambda a_{2n}+\mu a'_{2n} \\
            \vdots & \vdots & \ddots & \vdots \\
            0 & 0 & \cdots & \lambda a_{nn}+\mu a'_{nn} \\
        \end{bmatrix}.
    \end{align*}
    Luego $\lambda A+\mu A'$ es una matriz triangular superior.

    \qed     
    
    
    \item\label{rectas} Sea $L$ una recta en $\mathbb{R}^2$. Dar una condición necesaria y suficiente para que $L$ sea un subespacio vectorial de $\mathbb{R}^2$.
    
    
    \rta Una recta en $\mathbb{R}^2$ es un subespacio vectorial si y sólo si pasa por el origen. 
    
    La ecuación general de la recta en el plano es $ax+by=c$ con $a,b,c\in\mathbb{R}$ y $a,b$ no ambos nulos.

    ($\Rightarrow$) Si $L$ es un subespacio vectorial, entonces $(0,0) \in L$,  es decir la recta pasa por el origen. Además, como  $0 = a\cdot 0 + b \cdot 0= c$, la ecuación de la recta es  $ax+by=0$.

    ($\Leftarrow$) Si la recta pasa por el origen, entonces $0 = a\cdot 0 + b \cdot 0= c$, es decir la ecuación de la recta es  $ax+by=0$. Luego, si $(x_1,y_1)$ y $(x_2,y_2)$ pertenecen a la recta y $\lambda,\mu\in\mathbb{R}$, entonces
    \begin{align*}
        a(\lambda x_1+\mu x_2)+b(\lambda y_1+\mu y_2)&=\lambda(ax_1+by_1)+\mu(ax_2+by_2)\\
        &=\lambda\cdot 0+\mu\cdot 0=0.
    \end{align*}
    Luego $\lambda(x_1,y_1)+\mu(x_2,y_2)$ pertenece a la recta y por lo tanto la recta es un subespacio vectorial.

    \qed     
    
    \item Sean $V$ un $\mathbb{K}$-espacio vectorial, $v\in V$ no nulo y $\lambda,\mu\in\mathbb{K}$ tales que $\lambda v=\mu v$. Probar que $\lambda=\mu$.
    
    
    \rta Si $\lambda v=\mu v$, entonces $(\lambda-\mu)v=0$. Supongamos que $\lambda-\mu\neq 0$, entonces
    \begin{align*}
        v&= 1 \cdot v &&\text{(axioma P1 de esp. vectoriales)}\\
        &=(\lambda-\mu)^{-1}(\lambda-\mu)v&&\\
        &=(\lambda-\mu)^{-1}0&&\text{(por hipótesis)}\\
        &=0.&&\text{(demostrado en la teórica: $0\cdot v = 0$)}
    \end{align*}
    Concluimos que $v=0$, lo cual contradice la hipótesis. El absurdo vino de suponer que $\lambda-\mu\neq 0$, luego $\lambda=\mu$.

    \qed     
    
    \item Sean $W_1, W_2$ subespacios de un espacio vectorial $V$. Probar que $W_1 \cup W_2$ es un subespacio  de $V$ si y sólo si $W_1 \subseteq W_2$ o $W_2 \subseteq W_1$.
        
    
    \rta 

    ($\Rightarrow$)  Si $W_1 \subseteq W_2$ o $W_2 \subseteq W_1$ no hay nada que demostrar. Supongamos entonces que $W_1 \not\subseteq W_2$ y $W_2 \not\subseteq W_1$. Entonces existen $w_1\in W_1$ tal que $w_1\not\in W_2$ y $w_2\in W_2$ tal que $w_2\not\in W_1$. Como $w_1\in W_1$ y $w_2\in W_2$, entonces $w_1+w_2\in W_1\cup W_2$. Como $W_1 \cup W_2$ es un subespacio  de $V$, entonces $w_1+w_2\in W_1\cup W_2$ y por lo tanto $w_1+w_2\in W_1$ o $w_1+w_2\in W_2$. Supongamos que $w_1+w_2\in W_1$, entonces $w_2=w_1+w_2-w_1\in W_1$, lo cual es absurdo.Análogamente,  si $w_1+w_2\in W_2$, entonces $w_1\in W_2$, lo cual es absurdo. El absurdo vino de suponer que $W_1 \not\subseteq W_2$ y $W_2 \not\subseteq W_1$, luego $W_1 \subseteq W_2$ o $W_2 \subseteq W_1$.

    ($\Leftarrow$) Supongamos que $W_1 \subseteq W_2$. Entonces $W_1 \cup W_2 = W_2$ y por lo tanto $W_1 \cup W_2$ es un subespacio  de $V$. Análogamente se demuestra que si $W_2 \subseteq W_1$, entonces $W_1 \cup W_2$ es un subespacio  de $V$.

    \qed     
        
    \item Sean $u=(1,1)$, $v=(1,0)$, $w=(0,1)$ y $z=(3,4)$ vectores de $\mathbb{R}^2$.
    \begin{enumerate}
    \item\label{comb-lin-u-v-w} Escribir $z$ como combinación lineal de $u,v$ y $w$, con coeficientes todos no nulos.
    \item\label{comb-lin-u-v} Escribir $z$ como combinación lineal de $u$ y $v$.
    \item\label{comb-lin-u-w} Escribir $z$ como combinación lineal de $u$ y $w$.
    \item\label{comb-lin-v-w}Escribir $z$ como combinación lineal de $v$ y $w$.
    \end{enumerate}
    
    \rta En general tenemos que resolver la ecuación $z=\lambda u+\mu v+\nu w$, bajo ciertas condiciones sobre $\lambda,\mu,\nu$. En cada caso, las condiciones son distintas. Si escribimos en coordenadas la ecuación es
    \begin{align*}
        (3,4)&=\lambda (1,1)+\mu (1,0)+\nu (0,1)\\
        &=(\lambda+\mu,\lambda+\nu).    \tag{*}
    \end{align*}
    El sistema es sencillo de resolver, pues la segunda coordenada nos dice que $\lambda+\nu=4$, es decir $\nu=4-\lambda$. Reemplazando en la primera coordenada obtenemos $\lambda+\mu=3$, es decir $\mu=3-\lambda$. Por lo tanto, $\lambda$ es libre y
    \begin{equation*}
        z=\lambda u+\mu v+\nu w=\lambda (1,1)+(3-\lambda) (1,0)+(4-\lambda) (0,1).
    \end{equation*}
    
    \ref{comb-lin-u-v-w} Si $\lambda=1$, entonces $\mu=2$ y $\nu=3$ y por lo tanto
    \begin{equation*}
        z=1\cdot u+2\cdot v+3\cdot w.
    \end{equation*}

    \ref{comb-lin-u-v} Si $\lambda=4$, entonces $\mu=-1$ y $\nu=0$ y por lo tanto
    \begin{equation*}
        z=4\cdot u-1\cdot v.
    \end{equation*}

    \ref{comb-lin-u-w} Si $\lambda=3$, entonces $\mu=0$ y $\nu=1$ y por lo tanto
    \begin{equation*}
        z=3\cdot u+1\cdot w.
    \end{equation*}

    \ref{comb-lin-v-w} Si $\lambda=2$, entonces $\mu=1$ y $\nu=2$ y por lo tanto
    \begin{equation*}
        z=2\cdot v+2\cdot w.
    \end{equation*}
    \qed     
    
    \item Sean $p(x)=(x-1)(x+2)$, $q(x)=x^2-1$ y $r(x)=x(x^2-1)$ en $\mathbb{R}[x]$.
        \begin{enumerate}
        \item\label{comb-lineal-pol-a} Describir en forma implícita todos los polinomios de grado menor o igual que $3$ que son combinación lineal de $p,q$ y $r$.
        \item\label{comb-lineal-pol-b} Elegir $a$ tal que el polinomio $x$ se pueda escribir como combinación lineal de $p,q$ y $2x^2+a$.
        \end{enumerate}
    
    \rta 

    \ref{comb-lineal-pol-a} Escribamos la versión expandida de $p$, $q$ y $r$:
    \begin{align*}
        p(x)&=x^2+x-2,\\
        q(x)&=x^2-1,\\
        r(x)&=x^3-x.                
    \end{align*}
    \begin{comment}
    Debemos encontrar el subespacio generado por estos tres polinomios. Primero encontraremos una base del subespacio en término de los generadores canónicos  ($x^n$ con $n\in\mathbb{N}_0$). 
    \begin{align*}
        &\begin{bmatrix}
            0 & 1 & 1 & -2 \\
            0 & 1 & 0 & -1 \\
            1 & 0 & -1 & 0 \\
        \end{bmatrix}
        \stackrel{F_2-F_1}{\longrightarrow}
        \begin{bmatrix}
            0 & 1 & 1 & -2 \\
            0 & 0 & -1 & 1 \\
            1 & 0 & -1 & 0 \\
        \end{bmatrix} \\
        &\underset{F_3-F_2}{\stackrel{F_1+F_2}{\longrightarrow}}
        \begin{bmatrix}
            0 & 1 & 0 & -1 \\
            0 & 0 & -1 & 1 \\
            1 & 0 & 0 & -1 \\
        \end{bmatrix} \stackrel{-F_2}{\longrightarrow}
        \begin{bmatrix}
            0 & 1 & 0 & -1 \\
            0 & 0 & 1 & -1 \\
            1 & 0 & 0 & -1 \\
        \end{bmatrix}.            
    \end{align*}
    Luego  
    $$
    \langle p,q,r\rangle = \langle x^2-1,x-1,x^3-1 \rangle.
    $$
\end{comment}



Ahora, planteemos  la ecuación 
    \begin{equation*}
        \begin{aligned}
        ax^3+bx^2+cx +d &=\lambda p +\mu q +\nu r\\
        &= \lambda(x^2+x-2)+\mu(x^2-1)+\nu(x^3-x)\\
        &= \nu x^3 +(\lambda+\mu)x^2+(\lambda-\nu)x+(-2\lambda-\mu).
        \end{aligned} \tag{*}
    \end{equation*}
    Debemos encontrar todos los $(a,b,c,d)\in\mathbb{R}^4$ tales que existe $\lambda,\mu,\nu\in\mathbb{R}$ que satisfacen la ecuación anterior. Es decir, debemos encontrar todos los $(a,b,c,d)\in\mathbb{R}^4$ tales que existe $\lambda,\mu,\nu\in\mathbb{R}$ que satisfacen el sistema
    \begin{equation*}
        \begin{aligned}
        a&=\nu\\
        b&=\lambda+\mu\\
        c&=-\lambda-\nu\\
        d&=-2\lambda-\mu. 
        \end{aligned} 
    \end{equation*}
    Si consideramos $a,b,c,d$ como constantes y  $\lambda,\mu,\nu$ como incógnitas, entonces el sistema, presentado como matriz aumentada es:

    \begin{align*}
        &\begin{amatrix}{3}
            0 & 0 & 1 & a \\
            1 & 1 & 0 & b \\
            1 & 0 & -1 & c \\
            -2 & -1 & 0 & d
        \end{amatrix}
        \underset{F_4 +2F_2}{\stackrel{F_3-F_2}{\longrightarrow}}
        \begin{amatrix}{3}
            0 & 0 & 1 & a \\
            1 & 1 & 0 & b \\
            0 & -1 & -1 & -b+c \\
            0 & 1 & 0 & 2b+d
        \end{amatrix}\\
        &\underset{F_3+F_4}{\stackrel{F_2-F_4}{\longrightarrow}}
        \begin{amatrix}{3}
            0 & 0 & 1 & a \\
            1 & 0 & 0 & -b-d \\
            0 & 0 & -1 & b+c+d \\
            0 & 1 & 0 & 2b+d
        \end{amatrix}
        \stackrel{F_3+F_1}{\longrightarrow}
        \begin{amatrix}{3}
            0 & 0 & 1 & a \\
            1 & 0 & 0 & -b-d \\
            0 & 0 & 0 & a+b+c+d\\
            0 & 1 & 0 & 2b+d
        \end{amatrix}.
    \end{align*}

    Luego la ecuación (*) solo puede ser satisfecha si y sólo si $a+b+c+d  =0$ por lo tanto, el subespacio de polinomios que obtenemos es
    \begin{equation*}
        \{a x^3 + bx^2 + cx + d: \ a+b+c+d = 0, \ a,b,c,d\in\mathbb{R}\}.
    \end{equation*}
    También lo podríamos describir de la siguiente manera: 
    \begin{equation*}
        \{(-b-c-d) x^3 + bx^2 + cx + d: \ b,c,d\in\mathbb{R}\}.
    \end{equation*}

    \ref{comb-lineal-pol-b} Debemos encontrar $a$ tal que existan $\lambda,\mu,\nu\in\mathbb{R}$ que satisfacen la ecuación
    \begin{align*}
        x&=\lambda p +\mu q +(2x^2+a) \nu \\
        &= \lambda(x^2+x-2)+\mu(x^2-1)+\nu(2x^2+a)\\
        &= (\lambda+\mu+2\nu)x^2+\lambda x+(-2\lambda-\mu-a\nu).
    \end{align*}
    Claramente $\lambda+\mu+2\nu=0$, $\lambda=1$  y $-2\lambda-\mu-a\nu=0$,  en consecuencia
    \begin{align*}
        0&=1+\mu+2\nu\\
        0&=-2-\mu-a\nu,
    \end{align*}
    o, lo que es lo mismo, 
    \begin{align*}
        \mu+2\nu&=-1\\
        -\mu-a\nu&=2.
    \end{align*}
    Planteamos la matriz aumentada correspondiente a este sistema y la escalonamos:
    \begin{align*}
        &\begin{amatrix}{2}
            1 & 2 & -1 \\
            -1 & -a & 2
        \end{amatrix}
        \stackrel{F_2+F_1}{\longrightarrow}
        \begin{amatrix}{2}
            1 & 2 & -1 \\
            0 & 2-a & 1
        \end{amatrix}.
    \end{align*}
    Observamos entonces que si $a=2$ el sistema no tiene solución pues quedaría $\mu \cdot 0 + \nu \cdot 0 = 1$. Cuando $ a \ne 2$, dividimos por $2-a$ y obtenemos $\nu = \frac{1}{2-a}$ y $\mu = -1 - 2\nu = -1 - \frac{2}{2-a} = \frac{a}{2-a}$. Por lo tanto, si $a \ne 2$, el polinomio $x$ se puede escribir como combinación lineal de $p,q$ y $2x^2+a$.  



    \qed     
    
    \item\label{practicos anteriores} Dar un conjunto de generadores para los siguientes subespacios vectoriales.
    \begin{enumerate}
    \item\label{pa-pr2-sistemas-homogeneos} Los conjuntos de soluciones de los sistemas homogéneos del ejercicio \ref{sistemas homogeneos} del Práctico \ref{practico-2}.
    \item\label{pa-pr2-sistemas-con-soluciones} Los conjuntos descriptos en el ejercicio \ref{sistemas con soluciones} del Práctico  \ref{practico-2}.
    \end{enumerate}
    
    \rta 

    \ref{pa-pr2-sistemas-homogeneos} Los sistemas homogéneos del ejercicio \ref{sistemas homogeneos} del Práctico \ref{practico-2} son los tres primeros.

    El primer sistema era 
    $$
    \begin{cases}
        -x - y + 4z = 0\\
        x+3y+8z = 0\\
        x+2y + 5z = 0.
    \end{cases}
    $$
    Este sistema tenía como única solución $x=y=z=0$, luego el conjunto de soluciones es $\{(0,0,0)\}$ y por lo tanto el subespacio generado por las soluciones es $\{0\}$.


    \vskip .2cm

    El segundo era
    $$
    \begin{cases}
        x - 3y + 5z = 0\\
        2x-3y+z = 0\\
        -y + 3z = 0
    \end{cases}
    $$
    Este sistema tiene como soluciones al conjunto  $\{(4t, 3t, t) : t \in \R\} = \{t(4, 3, 1) : t \in \R\}$, y por lo tanto podemos tomar como generador del subespacio al vector $(4, 3, 1)$.

    \vskip .2cm
    El tercero: 
    $$
    \begin{cases}
        x-z+2t = 0\\
        -x+2y-z+2t = 0\\
        -x+y = 0.
    \end{cases}
    $$
    Las soluciones de este sistema eran los vectores del subespacio  $$\{(u - 2v, u - 2v, u, v) : u, v \in \R\} = \{u(1,1,1,0) + v(-2,-2,0,1) : u, v \in \R\}$$
    y por lo tanto podemos tomar como generadores a los vectores $(1,1,1,0)$ y $(-2,-2,0,1)$.

    \vskip .3cm

    \ref{pa-pr2-sistemas-con-soluciones} el primer conjunto era $\{(b_1, b_2, b_3) \in \R^3 : -2b_1 + b_2 + 3b_3 = 0\}$. Podemos despejar una de las variables respecto a las otras, por ejemplo $b_2 = 2b_1 -3b_3$, y en ese caso el conjunto se  puede escribir como
    $$
    \{(b_1, -2b_1 - 3b_3, b_3) : b_1, b_3 \in \R\} = \{b_1(1, -2, 0) + b_3(0, -3, 1) : b_1, b_3 \in \R\}.
    $$
    Por lo tanto, podemos tomar como generadores a los vectores $(1, -2, 0)$ y $(0, -3, 1)$.

    \vskip .4cm

    El segundo conjunto era 
    $$
    \{(b_1,b_2,b_3,b_4)\in \mathbb R^3: \frac12b_1 -\frac12b_2 +b_3 = 0 \wedge -\frac12b_1 -\frac12b_2+b_4=0\}.
    $$
    Plantemos el sistema correspondiente:
    $$
    \begin{cases}
        \frac12b_1 -\frac12b_2 +b_3 = 0\\
        -\frac12b_1 -\frac12b_2+b_4=0.
    \end{cases}
    $$
    Resolvamos el sistema:
    \begin{align*}
    &\begin{bmatrix}
        \frac12 & -\frac12 & 1 & 0\\
        -\frac12 & -\frac12 & 0 & 1
    \end{bmatrix}
    \underset{2\cdot F_2 }{\stackrel{2\cdot F_1 }{\longrightarrow}}
    \begin{bmatrix}
        1 & -1 & 2 & 0\\
        -1 & -1 & 0 & 2
    \end{bmatrix}
    \stackrel{F_2 + F_1}{\longrightarrow}
    \begin{bmatrix}
        1 & -1 & 2 & 0\\
        0 & -2 & 2 & 2
    \end{bmatrix}  \\
    &\stackrel{ F_2/(-2)}{\longrightarrow}
    \begin{bmatrix}
        1 & -1 & 2 & 0\\
        0 & 1 & -1 & -1
    \end{bmatrix}
    \stackrel{F_1 + F_2}{\longrightarrow}
    \begin{bmatrix}
        1 & 0 & 1 & -1\\
        0 & 1 & -1 & -1
    \end{bmatrix}.
    \end{align*}
    Luego, el conjunto de soluciones es
    $$
    \{(b_1,b_2,b_3,b_4)\in \R^4: b_1=-b_3+b_4 \wedge b_2=b_3+b_4\}.
    $$
    Es decir, el conjunto de soluciones es
    $$
    \{(-b_3+b_4,b_3+b_4,b_3,b_4)\in \R^4: b_3,b_4 \in \R\}.
    $$
    Escrito de otra forma, el conjunto de soluciones es
    $$
    \{b_3(-1,1,1,0)+b_4(1,1,0,1)\in \R^4: b_3,b_4 \in \R\}.
    $$
    Por lo tanto, podemos tomar como generadores a los vectores $(-1,1,1,0)$ y $(1,1,0,1)$.

    \vskip .4cm

    El tercer conjunto era $\R^3$ y por lo tanto podemos tomar los generadores canónicos $(1,0,0)$, $(0,1,0)$ y $(0,0,1)$.

    \qed     
    
    \item\label{caracterizar}  En cada caso, caracterizar con ecuaciones al subespacio vectorial dado por generadores.
    \begin{enumerate}
    \item\label{caracterizar-a} ${\left\langle(1,0,3),(0,1,-2)\right\rangle}\subseteq \mathbb{R}^3$.
    \item\label{caracterizar-b} ${\left\langle(1,2,0,1),(0,-1,-1,0),(2,3,-1,4)\right\rangle}\subseteq \mathbb{R}^4$.
    \end{enumerate}
    
    \rta 

    \ref{caracterizar-a} El subespacio es el conjunto de combinaciones lineales de $(1,0,3)$ y $(0,1,-2)$, es decir
    \begin{align*}
        \{\lambda(1,0,3)&+\mu(0,1,-2): \lambda,\mu\in\R\} = \{(\lambda, \mu, 3\lambda -2\mu) : \lambda,\mu\in\R\} \\
        &=  \{(x,y,z) : x=\lambda, y=\mu, z=3\lambda -2, \,\mu\lambda,\mu\in\R\} . 
    \end{align*}
    Planteamos el sistema correspondiente:
    $$
    \begin{cases}
        \lambda = x\\
        \mu=y\\
        3\lambda -2\mu =z.
    \end{cases}
    $$
    Resolvamos el sistema:
    $$
    \begin{amatrix}{2}
        1 & 0 & x\\
        0 & 1 & y\\
        3 & -2 & z
    \end{amatrix}
    \stackrel{F_3-3F_1}{\longrightarrow}
    \begin{amatrix}{2}
        1 & 0 & x\\
        0 & 1 & y\\
        0 & -2 & z-3x
    \end{amatrix}
    \stackrel{F_3+2F_2}{\longrightarrow}
    \begin{amatrix}{2}
        1 & 0 & x\\
        0 & 1 & y\\
        0 & 0 & z-3x+2y
    \end{amatrix}.
    $$
    Luego, el subespacio se caracteriza implícitamente de la siguiente manera:
    $$
    \{(x,y,z)\in \R^3: z-3x+2y=0\}.
    $$

    \vskip .4cm

    \ref{caracterizar-b} El subespacio es el conjunto de combinaciones lineales de $(1,2,0,1)$, $(0,-1,-1,0)$ y $(2,3,-1,4)$, es decir
    \begin{align*}
        &\{\lambda(1,2,0,1)+\mu(0,-1,-1,0)+\nu(2,3,-1,4): \lambda,\mu,\nu\in\R\} =\\
        &= \{(\lambda + 2\nu, 2\lambda-\mu+3\nu, -\mu-\nu, \lambda+4\nu) : \lambda,\mu,\nu\in\R\} \\
        &= \{(x,y,z,t) : x=\lambda + 2\nu, \\
        &\qquad\qquad\qquad y=2\lambda-\mu+3\nu, z=-\mu-\nu, t=\lambda+4\nu, \,\mu,\lambda,\nu\in\R\} .
    \end{align*}
    Planteamos el sistema correspondiente:
    $$
    \begin{cases}
        \lambda + 2\nu = x\\
        2\lambda-\mu+3\nu = y\\
        -\mu-\nu = z\\
        \lambda+4\nu = t
    \end{cases}
    $$
    Resolvamos el sistema:
    $$
    \begin{amatrix}{3}
        1 & 0 & 2 & x\\
        2 & -1 & 3 & y\\
        0 & -1 & -1 & z\\
        1 & 0 & 4 & t
    \end{amatrix}
    \stackrel{F_2-2F_1}{\longrightarrow}
    \begin{amatrix}{3}
        1 & 0 & 2 & x\\
        0 & -1 & -1 & y-2x\\
        0 & -1 & -1 & z\\
        0 & 0 & 2 & t-x
    \end{amatrix}
    \stackrel{F_3-F_2}{\longrightarrow}
    \begin{amatrix}{3}
        1 & 0 & 2 & x\\
        0 & -1 & -1 & y-2x\\
        0 & 0 & 0 & 2x-y+z\\
        0 & 0 & 2 & t-x
    \end{amatrix}.
    $$
    En  realidad, no es necesario resolver el sistema completamente: la última matriz nos dice que el sistema tiene solución si y sólo si $2x-y+z=0$. Luego, el subespacio se caracteriza implícitamente de la siguiente manera:
    $$
    \{(x,y,z,t)\in \R^4: 2x-y+z=0\}.
    $$
    \qed     
    
    \item\label{son LI} En cada caso, determinar si el subconjunto indicado es linealmente independiente.
    \begin{enumerate}
        \item\label{son LI-a} $\{ (1,0,-1), (1,2,1), (0,-3,2) \}\subseteq \mathbb{R}^3$.
        \vskip .3cm
        \item\label{son LI-b} $\left\{  \begin{bmatrix} 1 & 0 & 2 \\ 0 & -1 & -3 \\ \end{bmatrix}, \quad
        \begin{bmatrix} 1 & 0 & 1 \\ -2 & 1 & 0 \\ \end{bmatrix}, \quad
        \begin{bmatrix} 1 & 2 & 3 \\ 3 & 2 & 1 \\ \end{bmatrix} \right\}\subseteq M_{2\times 3}(\mathbb{R})$.
    \end{enumerate}
    
    \rta para determinar si un conjunto de vectores es LI debemos plantear la ecuación $\lambda_1 v_1+\cdots+\lambda_n v_n=0$ y resolverla. Si la única solución es $\lambda_1=\cdots=\lambda_n=0$, entonces el conjunto es LI. En caso contrario, el conjunto es LD.

    \ref{son LI-a} Debemos resolver la ecuación
    $$
    \lambda_1 (1,0,-1)+\lambda_2 (1,2,1)+\lambda_3 (0,-3,2)=(0,0,0),
    $$
    o sea
    $$
    (\lambda_1+\lambda_2,\,2\lambda_2-3\lambda_3,\,-\lambda_1+\lambda_2+2\lambda_3)=(0,0,0).
    $$
    Es decir, debemos resolver el sistema
    $$
    \begin{cases}
        \lambda_1 +\lambda_2 = 0\\
        2\lambda_2 -3\lambda_3 = 0\\
        -\lambda_1 +\lambda_2 +2\lambda_3 = 0.
    \end{cases}
    $$
    Resolvamos el sistema:
    \begin{align*}
    &\begin{bmatrix}
        1 & 1 & 0\\
        0 & 2 & -3\\
        -1 & 1 & 2
    \end{bmatrix}
    \stackrel{F_3+F_1}{\longrightarrow}
    \begin{bmatrix}
        1 & 1 & 0 \\
        0 & 2 & -3 \\
        0 & 2 & 2 
    \end{bmatrix}
    \stackrel{F_3-F_2}{\longrightarrow}
    \begin{bmatrix}
        1 & 1 & 0 \\
        0 & 2 & -3 \\
        0 & 0 & 5 
    \end{bmatrix} \\
    &\stackrel{F_2/2}{\longrightarrow}
    \begin{bmatrix}
        1 & 1 & 0 \\
        0 & 1 & -\frac32 \\
        0 & 0 & 5
    \end{bmatrix}
    \stackrel{F_1-F_2}{\longrightarrow}
    \begin{bmatrix}
        1 & 0 & \frac32 \\
        0 & 1 & -\frac32 \\
        0 & 0 & 5
    \end{bmatrix}
    \stackrel{F_3/5}{\longrightarrow}
    \begin{bmatrix}
        1 & 0 & \frac32 \\
        0 & 1 & -\frac32 \\
        0 & 0 & 1
    \end{bmatrix} \\
    &\underset{F_2+\frac32 F_3}{\stackrel{F_1-\frac32 F_3}{\longrightarrow}}
    \begin{bmatrix}
        1 & 0 & 0 \\
        0 & 1 & 0 \\
        0 & 0 & 1
    \end{bmatrix}.
    \end{align*}
    Luego  como la única solución es la trivial los vectores son LI.

    \vskip .4cm

    \ref{son LI-b} Debemos resolver la ecuación
    $$
    \lambda_1 \begin{bmatrix} 1 & 0 & 2 \\ 0 & -1 & -3 \\ \end{bmatrix}+\lambda_2 \begin{bmatrix} 1 & 0 & 1 \\ -2 & 1 & 0 \\ \end{bmatrix}+\lambda_3 \begin{bmatrix} 1 & 2 & 3 \\ 3 & 2 & 1 \\ \end{bmatrix}=\begin{bmatrix} 0 & 0 & 0 \\ 0 & 0 & 0 \\ \end{bmatrix},
    $$
    o sea
    $$
    \begin{bmatrix} 
        \lambda_1 +\lambda_2 +\lambda_3 & 2\lambda_3 & 2\lambda_1 +\lambda_2+3\lambda_3 
        \\ -2\lambda_2 +3\lambda_3 &-\lambda_1 + \lambda_2 +2\lambda_3 & -3\lambda_1 +\lambda_3 \\ 
    \end{bmatrix}=
    \begin{bmatrix} 0 & 0 & 0 \\ 0 & 0 & 0 \\ \end{bmatrix}.
    $$
    Por la igualdad de fila $1$, columna $2$, claramente, $\lambda_3=0$. Luego, por la igualdad de fila $2$, columna $3$, $\lambda_1=0$. Finalmente, por la igualdad de fila $1$, columna $1$, $\lambda_2=0$. Es decir, la única solución es la trivial y por lo tanto los vectores son LI.
    





    \qed     
    
    \item Dar un ejemplo de un conjunto de 3 vectores en $\mathbb{R}^3$ que sean LD, y tales que dos cualesquiera de ellos sean LI.
    
    
    \rta tomemos $v_1= (1,0,0)$, $ v_2=(0,1,0)$ y el tercer vector la suma de ambos, es decir, $v_3= (1,1,0)$. Por lo tanto, esos tres vectores son LD ($v_1+v_2-v_3=0$). Ahora bien, si quitamos cualquiera de los vectores, los dos restantes son LI. Por ejemplo, si quitamos $v_1$, entonces $v_2$ y $v_3$ son LI, pues si $\lambda v_2+\mu v_3=0$, entonces $(\mu, \lambda+\mu,0)=(0,0,0)$ y por lo tanto $\mu=0$, luego $\lambda=0$.

    \qed     
    
    \item  Probar que si $\alpha$, $\beta$ y $\gamma$ son vectores LI en el $\mathbb{R}$-espacio vectorial $V$, entonces $\alpha +\beta$, $\alpha +\gamma$ y $\beta +\gamma $ también son LI.
    
    
    \rta debemos plantear la ecuación
    $$
    \lambda_1 (\alpha +\beta)+\lambda_2 (\alpha +\gamma)+\lambda_3 (\beta +\gamma)=0,
    $$
    y ver que la única solución es $\lambda_1=\lambda_2=\lambda_3=0$. La ecuación anterior es equivalente a 
    $$
    (\lambda_1+\lambda_2)\alpha+(\lambda_1+\lambda_3)\beta+(\lambda_2+\lambda_3)\gamma=0.
    $$  
    Como $\alpha$, $\beta$ y $\gamma$ son LI por hipótesis, entonces $\lambda_1+\lambda_2=\lambda_1+\lambda_3=\lambda_2+\lambda_3=0$. Resolvamos el sistema:
    $$
    \begin{bmatrix}
        1 & 1 & 0\\
        1 & 0 & 1\\
        0 & 1 & 1
    \end{bmatrix}
    \stackrel{F_2-F_1}{\longrightarrow}
    \begin{bmatrix}
        1 & 1 & 0\\
        0 & -1 & 1\\
        0 & 1 & 1
    \end{bmatrix}
    \stackrel{F_3+F_2}{\longrightarrow}
    \begin{bmatrix}
        1 & 1 & 0\\
        0 & -1 & 1\\
        0 & 0 & 2
    \end{bmatrix}.
    $$
    Por lo tanto, por la tercera fila, $\lambda_3=0$, por la segunda fila se deduce que $\lambda_2=0$ y por la primera fila se deduce que $\lambda_1=0$. Es decir, la única solución es la trivial y por lo tanto los vectores son LI.

    \qed     
    
    \item Extender, de ser posible, los siguientes conjuntos a una base de los respectivos espacios vectoriales.
    
    \begin{enumerate}
        \item\label{extender-a} Los conjuntos del ejercicio \ref{son LI}.
        \item\label{extender-b} $\{ (1,2,0,0),(1,0,1,0) \}\subseteq\mathbb{R}^4$.
        \item\label{extender-c} $\{ (1,2,1,1),(1,0,1,1),(3,2,3,4)\}\subseteq\mathbb{R}^4$.
    \end{enumerate}
    
    
    \rta 

    \ref{extender-a} Los conjuntos del ejercicio \ref{son LI} son LI. Luego el primer subconjunto es base, pues tiene $3$ elementos de  $\R^3$. 
    
    El segundo subconjunto no es base pues tiene $3$ vectores y $M_{2\times 3}(\mathbb{R})$ tiene dimensión $6$. 
    
    Extendamos entonces $$\left\{  \begin{bmatrix} 1 & 0 & 2 \\ 0 & -1 & -3 \\ \end{bmatrix}, \quad
    \begin{bmatrix} 1 & 0 & 1 \\ -2 & 1 & 0 \\ \end{bmatrix}, \quad
    \begin{bmatrix} 1 & 2 & 3 \\ 3 & 2 & 1 \\ \end{bmatrix} \right\}$$ a una base de $M_{2\times 3}(\mathbb{R})$.
    
    Convendrá mirar a las matrices como vectores de $\R^6$ para poder operar con estos vectores. La forma más obvia de hacerlo es construir  un vector de $6$ coordenadas a partir de cada matriz  con la primera fila y a continuación la segunda. Haciendo esto obtenemos los vectores
    $$
    (1, 0, 2, 0, -1, -3),\; (1, 0, 1, -2, 1, 0),\;  (1, 2, 3, 3, 2, 1).
    $$
    
    %\textbf{Primera solución.} 
    Una forna de extender la base es primero construir una matriz con los vectores fila y hallar la MRF. Los vectores que obtenemos en la MRF generan el mismo subespacio que los vectores originales y los podemos completar a una base de $\R^6$ con vectores canónicos. Hagamos el procedimiento: 
    \begin{align*}
    &\begin{bmatrix}
        1 & 0 & 2 & 0 & -1 & -3\\
        1 & 0 & 1 & -2 & 1 & 0\\
        1 & 2 & 3 & 3 & 2 & 1
    \end{bmatrix}
    \underset{F_3-F_1}{\stackrel{F_2-F_1}{\longrightarrow}}
    \begin{bmatrix}
        1 & 0 & 2 & 0 & -1 & -3\\
        0 & 0 & -1 & -2 & 2 & 3\\
        0 & 2 & 1 & 3 & 3 & 4
    \end{bmatrix} \\
    &\underset{F_3/2}{\stackrel{F_2/(-1)}{\longrightarrow}}
    \begin{bmatrix}
        1 & 0 & 2 & 0 & -1 & -3\\
        0 & 0 & 1 & 2 & -2 & -3\\
        0 & 1 & \frac12 & \frac32 & \frac32 & 2
    \end{bmatrix}
    \underset{F_3-\frac12 F_2}{\stackrel{F_1-2F_2}{\longrightarrow}}
    \begin{bmatrix}
        1 & 0 & 0 & -4 & 3 & 3\\
        0 & 0 & 1 & 2 & -2 & -3\\
        0 & 1 & 0 & \frac12 & \frac52 & \frac72
    \end{bmatrix} \\
    &\stackrel{F_2 \leftrightarrow F_3}{\longrightarrow}
    \begin{bmatrix}
        1 & 0 & 0 & -4 & 3 & 3\\
        0 & 1 & 0 & \frac12 & \frac52 & \frac72\\
        0 & 0 & 1 & 2 & -2 & -3
    \end{bmatrix}.
    \end{align*}
    Luego el conjunto LI
    $$
    \left\{(1, 0, 2, 0, -1, -3), (1, 0, 1, -2, 1, 0),  (1, 2, 3, 3, 2, 1)\right\}
    $$ \
    se puede completar a una base con los vectores canónicos 
    $$(0,0,0,1,0,0),\; (0,0,0,0,1,0),\; (0,0,0,0,0,1).$$
    Volviendo a $M_{2\times 3}(\mathbb{R})$, los vectores canónicos son las matrices
    $$  
    \begin{bmatrix} 0 & 0 & 0 \\ 1 & 0 & 0 \\ \end{bmatrix},\quad
    \begin{bmatrix} 0 & 0 & 0 \\ 0 & 1 & 0 \\ \end{bmatrix},\quad
    \begin{bmatrix} 0 & 0 & 0 \\ 0 & 0 & 1 \\ \end{bmatrix},
    $$
    que completan a una base.

\vskip .3cm

\begin{comment}
\textbf{Segunda solución.} Otra forma de hacerlo es encontrando la forma implícita del subespacio y luego agregar vectores con un criterio que explicaremos más adelante. 

Debemos ahora caracterizar implícitamente los $b_1,\ldots, b_6$ tales que
\begin{multline*}
    \lambda_1(1, 0, 2, 0, -1, -3)+\lambda_2(1, 0, 1, -2, 1, 0)+\lambda_3(1, 2, 3, 3, 2, 1)=(b_1,b_2,b_3,b_4,b_6,b_6).
\end{multline*}
Es decir, debemos resolver el sistema
$$
\begin{cases}
    \lambda_1+\lambda_2+\lambda_3=b_1\\
    2\lambda_3=b_2\\
    2\lambda_1+\lambda_2+3\lambda_3=b_3\\
    -2\lambda_2+3\lambda_3=b_4\\
    -\lambda_1+\lambda_2+2\lambda_3=b_5\\
    -3\lambda_1+\lambda_3=b_6
\end{cases}
$$
Planteamos la matriz ampliada y hacemos Gauss:
\begin{align*}
&\begin{amatrix}{3}
    1 & 1 & 1 & b_1\\   
    0 & 0 & 2 & b_2\\
    2 & 1 & 3 & b_3\\
    0 & -2 & 3 & b_4\\
    -1 & 1 & 2 & b_5\\
    -3 & 0 & 1 & b_6
\end{amatrix}
\underset{F_6+3F_1}{\underset{F_5+F_1}{\stackrel{F_2-2F_1}{\longrightarrow}}}
\begin{amatrix}{3}
    1 & 1 & 1 & b_1\\   
    0 & 0 & 2 & b_2\\
    0 & -1 & 1 & b_3-2b_1\\
    0 & -2 & 3 & b_4\\
    0 & 2 & 3 & b_1+b_5\\
    0 & 3 & 4 & 3b_1+b_6
\end{amatrix}\\
&\overset{F_1+F_3}{\underset{F_6+3F_3}{\underset{F_5+2F_3}{\stackrel{F_4-2F_3}{\longrightarrow}}}}
\begin{amatrix}{3}
    1 & 0 & 2 & -b_1+b_3\\   
    0 & 0 & 2 & b_2\\
    0 & -1 & 1 & b_3-2b_1\\
    0 & 0 & 1 &4b_1-2b_3 + b_4\\
    0 & 0 & 5 &-3b_1+2b_3 + b_5\\
    0 & 0 & 7 & -3b_1+3b_3+b_6
\end{amatrix}\\
&\underset{F_6-7F_4}{\underset{F_5-5F_4}{\stackrel{F_2-2F_4}{\longrightarrow}}}
\begin{amatrix}{3}
    1 & 0 & 0 & -b_1+b_3-2b_4\\   
    0 & 0 & 0 &-8b_1+ b_2+4b_3-2b_4\\
    0 & -1 & 0 & b_3-2b_1+b_4\\
    0 & 0 & 1 &4b_1-2b_3 + b_4\\
    0 & 0 & 0 &-23b_1+12b_3 + b_5-5b_4\\
    0 & 0 & 0 & -3b_1+3b_3+b_6+7b_4
\end{amatrix}\\
\end{align*}
\end{comment}    

    \vskip .3cm
    \ref{extender-b} Lo haremos de forma análoga alo que hicimos en \ref{extender-a}. Debemos hallar la MRF de la matriz formada por los vectores dados como filas:
    $$
    \begin{bmatrix}
        1 & 2 & 0 & 0\\
        1 & 0 & 1 & 0
    \end{bmatrix}
    \stackrel{F_2-F_1}{\longrightarrow}
    \begin{bmatrix}
        1 & 2 & 0 & 0\\
        0 & -2 & 1 & 0
    \end{bmatrix}
    \stackrel{F_2/(-2)}{\longrightarrow}
    \begin{bmatrix}
        1 & 2 & 0 & 0\\
        0 & 1 & -\frac12 & 0
    \end{bmatrix}
    \stackrel{F_1-2F_2}{\longrightarrow}
    \begin{bmatrix}
        1 & 0 & 1 & 0\\
        0 & 1 & -\frac12 & 0
    \end{bmatrix}.
    $$
    Luego, los vectores que completan a una base son $(0,0,1,0)$ y $(0,0,0,1)$. Por supuesto, también podemos completar con otros pares de vectores, pero  estos son los más simples. 


    \vskip .3cm
    \ref{extender-c} Este inciso lo haremos de otra forma: primero encontraremos la forma implícita del subespacio y luego agregaremos vectores con un criterio que explicaremos más adelante. Para encontrar la forma implícita del subespacio planteamos la ecuación:
    $$
    \lambda_1(1,2,1,1)+\lambda_2(1,0,1,1)+\lambda_3(3,2,3,4)=(b_1,b_2,b_3,b_4).
    $$
    Es decir, debemos resolver el sistema
    $$
    \begin{cases}
        \lambda_1+\lambda_2+3\lambda_3=b_1\\
        2\lambda_1+2\lambda_3=b_2\\
        \lambda_1+\lambda_2+3\lambda_3=b_3\\
        \lambda_1+\lambda_2+4\lambda_3=b_4
    \end{cases}
    $$
    Planteamos la matriz ampliada y hacemos Gauss:
    \begin{align*}
    &\begin{amatrix}{3}
        1 & 1 & 3 & b_1\\
        2 & 0 & 2 & b_2\\
        1 & 1 & 3 & b_3\\
        1 & 1 & 4 & b_4
    \end{amatrix}
    \underset{F_4-F_1}{\underset{F_3-F_1}{\stackrel{F_2-2F_1}{\longrightarrow}}}
    \begin{amatrix}{3}
        1 & 1 & 3 & b_1\\
        0 & -2 & -4 & b_2-2b_1\\
        0 & 0 & 0 & b_3-b_1\\
        0 & 0 & 1 & b_4-b_1
    \end{amatrix}\\
    \end{align*}
    Luego  el subespacio generado por los vectores del enunciado tiene por forma implícita
    $$
    \{(b_1,b_2,b_3,b_4)\in\R^4: b_3=b_1\}.
    $$
    Cualquier vector que no cumpla esta condición no pertenece al subespacio y por lo tanto completa a una base. Por ejemplo, $(0,0,1,0)$ completa a una base. 


    \qed     
    
    \item Dar subespacios vectoriales $W_0$, $W_1$, $W_2$ y $W_3$ de $\mathbb{R}^3$ tales que $W_0\subset W_1\subset W_2\subset W_3$ y $\dim W_0=0$, $\dim W_1=1$, $\dim W_2=2$ y $\dim W_3=3$.
    
    
    \rta El único subespacio vectorial de dimensión $0$ es $\{0\}$. Por lo tanto, $W_0=\{0\}$. Sea $W_1=\langle (1,0,0)\rangle$ y $W_2=\langle (1,0,0), (0,1,0)\rangle$. El único subespacio de $\R^3$ de dimensión $3$ es $\R^3$. Por lo tanto, $W_3=\R^3$.

    \qed     
    
    \item Sea $V$ un espacio vectorial de dimensión $n$ y $\mathcal{B}=\{v_1, ..., v_n\}$ una base de $V$.
    \begin{enumerate}
    \item\label{subconjunto-li} Probar que cualquier subconjunto no vacío de $\mathcal{B}$ es LI.
    \item\label{subconjunto-dim-k} Para cada $k\in\mathbb{N}_0$,  con $0\leq k\leq n$, dar un subespacio vectorial de $V$ de dimensión $k$.
    \end{enumerate}
    
    
    \rta 

    \ref{subconjunto-li} Sea $W=\{v_{i_1}, ..., v_{i_k}\}$ un subconjunto no vacío de $\mathcal{B}$. Supongamos que $W$ es LD. Entonces existen $\lambda_1, ..., \lambda_k\in\mathbb{K}$, no todos nulos, tales que 
    $$
    \lambda_1 v_{i_1}+\cdots+\lambda_k v_{i_k}=0.
    $$
    Luego existe una combinación lineal no trivial de los elementos de la base que da como resultado el vector nulo.  Más explicitamente si $\mu_i = 0$ para $i \ne i_1, \ldots, i_k$ y $\mu_{i_j} = \lambda_{j}$ para $j=1,\ldots,k$, entonces
    $$
    \mu_1 v_1+\cdots+\mu_n v_n=0,
    $$
    y no todos los $\mu_i$ son nulos. Esto contradice que $\mathcal{B}$ sea una base. Por lo tanto, $W$ es LI.


    \vskip .3cm
    \ref{subconjunto-dim-k} Sea $W_k=\langle v_1, ..., v_k\rangle$. Entonces $W_k$ es un subespacio de $V$ de dimensión $k$. En efecto, como $\mathcal{B}$ es una base de $V$, entonces $W_k$ es un subespacio de $V$. Por otra parte, por \ref{subconjunto-li}, los $v_1, ..., v_k$ son LI. Por lo tanto, $\dim W_k=k$.

    \qed     
    
    \item Dar una base y calcular la dimensión de $\mathbb{C}^n$ como $\mathbb{C}$-espacio vectorial y como $\mathbb{R}$-espacio vectorial.
    
    
    \rta una base de  $\mathbb{C}^n$ como $\mathbb{C}$-espacio vectorial es $\{e_1,\ldots,e_n\}$, donde $e_i$ es el vector cuyas coordenadas son todas nulas excepto la $i$-ésima que es $1$. Por lo tanto, $\dim_{\mathbb{C}} \mathbb{C}^n=n$.

    Una base de $\mathbb{C}^n$ como $\mathbb{R}$-espacio vectorial es $\{e_1,\ldots,e_n,ie_1,\ldots,ie_n\}$, donde $e_i$ es el vector cuyas coordenadas son todas nulas excepto la $i$-ésima que es $1$. Por lo tanto, $\dim_{\mathbb{R}} \mathbb{C}^n=2n$.

    \qed     
    
    \item  Exhibir una base y calcular la dimensión de los siguientes subespacios.
    \begin{enumerate}
        \item\label{dim-subespacios-a} Los subespacios del ejercicio \ref{practicos anteriores}.
        \item\label{dim-subespacios-b} $W = \{(x,y,z,w,u) \in \mathbb{R}^5 \ : \ y = x - z,\, w = x + z,\,  u = 2x - 3z \}$.
        \item\label{dim-subespacios-c} $W = \langle (1, 0, -1, 1),  (1, 2, 1, 1), (0, 1, 1, 0), (0, -2, -2, 0) \rangle \subseteq \mathbb R^4$.
        \item\label{dim-subespacios-d} Matrices triangulares superiores $2\times 2$ y $3\times 3$.
        \item\label{dim-subespacios-e} Matrices triangulares superiores $n\times n$ para cualquier $n\in\mathbb{N}$, $n\geq 2$.
    \end{enumerate}
    
    \rta 

    \ref{dim-subespacios-a} En  el ejercicio \ref{practicos anteriores} ya dimos conjuntos de generadores para cada subespacio y es sencillo comprobar que cada uno  de estos subconjuntos son LI y por lo tanto son base. Solo resta calcular la dimensión. Listemos los subespacios y sus respectivas bases, lo cual nos dirá la dimensión de cada uno.
    \begin{itemize}
        \item $W_1= \{(0,0,0)\}$ es un subespacio de $\mathbb{R}^3$ y $\dim W_1=0$. Luego  su base es $\emptyset$.
        \item $W_2=\langle (4,3,1)\rangle$ es un subespacio de $\mathbb{R}^3$ y $\dim W_2=1$.
        \item $W_3=\langle (1,1,1,0),(-2,-2-,0-1)\rangle$ es un subespacio de $\mathbb{R}^4$ y $\dim W_3=2$.
        \item $W_4=\langle (1,-2,0),(0,-3,1)\rangle$ es un subespacio de $\mathbb{R}^3$ y $\dim W_4=2$.
        \item $W_5 = \langle (-1, 1, 1, 0), (1, 1, 0, 1) \rangle$ es un subespacio de $\mathbb{R}^4$ y $\dim W_5=2$.
        \item $W_6 = \langle (1, 0, 0), (0, 1, 0), (0,0,1) \rangle$ es un subespacio de $\mathbb{R}^3$ y $\dim W_6=3$.
    \end{itemize}

    
    \vskip.3cm
    \ref{dim-subespacios-b} Observar que 
    \begin{align*}
    W &= \{(x,y,z,w,u) \in \mathbb{R}^5 \ : \ y = x - z,\, w = x + z,\,  u = 2x - 3z \}\\
    &= \{(x,x-z,z,x+z,2x-3z) \in \mathbb{R}^5 \ : \ x,z \in \mathbb{R}\}\\
    &= \{(x,x,0,x,2x)+(0,-z,z,z,-3z) \in \mathbb{R}^5 \ : \ x,z \in \mathbb{R}\}\\
    &=\{x(1,1,0,1,2)+z(0,-1,1,1,-3) \in \mathbb{R}^5 \ : \ x,z \in \mathbb{R}\}\\
    &= \langle (1,1,0,1,2), (0,-1,1,1,-3) \rangle.
    \end{align*}
    Por lo tanto, $\{ (1,1,0,1,2), (0,-1,1,1,-3)\}$ es una base y la dimensión del subespacio es $2$.

    \vskip.3cm
    \ref{dim-subespacios-c} Para ver la dimensión  planteamos la matriz donde los vectores fila son los vectores del enunciado y  hacemos Gauss para obtener una MRF:
    \begin{align*}
        &\begin{bmatrix}
            1& 0& -1& 1 \\  
            1& 2& 1& 1 \\ 
            0& 1& 1& 0 \\ 
            0& -2& -2& 0
        \end{bmatrix}
        \stackrel{F_2-F_1}{\longrightarrow}
        \begin{bmatrix}
            1& 0& -1& 1 \\  
            0& 2& 2& 0 \\ 
            0& 1& 1& 0 \\ 
            0& -2& -2& 0
        \end{bmatrix}
        \underset{F_4+2F_3}{\stackrel{F_2-2F_3}{\longrightarrow}}
        \begin{bmatrix}
            1& 0& -1& 1 \\  
            0& 0& 0& 0 \\ 
            0& 1& 1& 0 \\ 
            0& 0& 0& 0
        \end{bmatrix} .
    \end{align*}
    Luego  una base del subespacio es $\{(1,0,-1,1), (0,1,1,0)\}$ y por lo tanto la dimensión es $2$.


    \vskip.3cm
    \ref{dim-subespacios-d} Una base de las matrices triangulares superiores $2\times 2$ es
    $$
    \left\{  \begin{bmatrix} 1 & 0 \\ 0 & 0 \\ \end{bmatrix}, \quad
    \begin{bmatrix} 0 & 1 \\ 0 & 0 \\ \end{bmatrix}, \quad
    \begin{bmatrix} 0 & 0 \\ 0 & 1 \\ \end{bmatrix} \right\},
    $$
    y por lo tanto su dimensión es $3$.

    Una base de las matrices triangulares superiores $3\times 3$ es
    $$
    \left\{  \begin{bmatrix} 1 & 0 & 0 \\ 0 & 0 & 0 \\ 0 & 0 & 0 \\ \end{bmatrix}, 
    \begin{bmatrix} 0 & 1 & 0 \\ 0 & 0 & 0 \\ 0 & 0 & 0 \\ \end{bmatrix}, 
    \begin{bmatrix} 0 & 0 & 1 \\ 0 & 0 & 0 \\ 0 & 0 & 0 \\ \end{bmatrix}, 
    \begin{bmatrix} 0 & 0 & 0 \\ 0 & 1 & 0 \\ 0 & 0 & 0 \\ \end{bmatrix}, 
    \begin{bmatrix} 0 & 0 & 0 \\ 0 & 0 & 1 \\ 0 & 0 & 0 \\ \end{bmatrix}, 
    \begin{bmatrix} 0 & 0 & 0 \\ 0 & 0 & 0 \\ 0 & 0 & 1 \\ \end{bmatrix} \right\},
    $$
    y por lo tanto su dimensión es $6$.


    \vskip.3cm
    \ref{dim-subespacios-e} Este inciso es una generalización del anterior. Denotemos $E_{ij}$ a la matriz cuyas entradas son todas nulas excepto la $i$-ésima fila y la $j$-ésima columna, que es $1$. Es decir, $E_{ij}$ es la matriz que tiene un $1$ en la posición $(i,j)$ y ceros en el resto de las posiciones. Entonces, una base de las matrices triangulares superiores $n\times n$ es
    $$
    \mathcal B= \left\{  E_{11}, E_{12}, \ldots, E_{1n}, E_{22}, E_{23}, \ldots, E_{2n}, \ldots, E_{n-1,n}, E_{nn} \right\},
    $$
    escrito de forma más compacta y precisa,
    $$
    \mathcal B= \left\{  E_{ij} \in M_{n\times n}(\mathbb{R}) \ : \ 1\leq i\leq j\leq n \right\}.
    $$
    Calculemos ahora la dimensión de este subespacio. Para eso debemos ver cuántos elementos tiene $\mathcal B$. Observar que la diagonal tiene $n$ elementos de la base, todos los de la forma $E_{ii}$, $1 \le i \le n$.  Justo encima de la diagonal hay $n-1$ elementos de la base, todos los de la forma $E_{i,i+1}$, $1 \le i \le n-1$. Encima de esta última diagonal ``menor''  hay $n-2$ elementos de la base, todos los de la forma $E_{i,i+2}$, $1 \le i \le n-2$. Y así sucesivamente hasta llegar a la última diagonal menor, donde hay un solo elemento de la base, $E_{n,n}$. Por lo tanto, la cantidad de elementos de la base es 
    $$
    n + (n-1) + (n-2) + \cdots + 1 = \frac{n(n+1)}{2}.
    $$
    Luego la dimensión del subespacio formado por las matrices triangulares superiores es $\displaystyle\frac{n(n+1)}{2}$.


    \qed     
    
    \item Sean $W_1$ y $W_2$ los siguientes subespacios de $\mathbb{R}^3$:
        \begin{align*}
        W_1 &= \{ (x,y,z)\in\mathbb{R}^3\ : \ x+y-2z=0\},  \\
        W_2 &= {\left\langle(1,-1,1),(2,1,-2),(3,0,-1)\right\rangle}.
        \end{align*}
        \begin{enumerate}
            \item\label{interseccion-subespacios}  Determinar $W_1 \cap W_2$, y describirlo por generadores y con ecuaciones.
            \item\label{suma-subespacios}  Determinar $W_1+W_2$, y describirlo por generadores y con ecuaciones.
        \end{enumerate}
    
    
    \rta 
    
    \ref{interseccion-subespacios} La forma más sencilla de hacer este ejercicio  es encontrar la descripción de manera implícita de $W_2$ y  agregando estas ecuaciones a las de $W_1$  encontramos la forma implícita de  $W_1 \cap W_2$. A partir de esta forma implícita se encuentran los generadores.

    Los vectores de $W_2$ son los $(b_1,b_2,b_3)$ tales que
    $$
    \begin{bmatrix} b_1 \\ b_2 \\ b_3 \\ \end{bmatrix} = \lambda_1 \begin{bmatrix} 1 \\ -1 \\ 1 \\ \end{bmatrix} + \lambda_2 \begin{bmatrix} 2 \\ 1 \\ -2 \\ \end{bmatrix} + \lambda_3 \begin{bmatrix} 3 \\ 0 \\ -1 \\ \end{bmatrix} = \begin{bmatrix} \lambda_1 + 2\lambda_2 + 3\lambda_3 \\ -\lambda_1 + \lambda_2 \\ \lambda_1 - 2\lambda_2 - \lambda_3 \\ \end{bmatrix}.
    $$
    Plantemos la matriz ampliado del sistema y  hallamos una MRF:
    \begin{align*}
        &\begin{amatrix}{3}
            1 & 2 & 3 & b_1\\
            -1 & 1 & 0 & b_2\\
            1 & -2 & -1 & b_3
        \end{amatrix}
        \stackrel{F_3-F_1}{\stackrel{F_2+F_1}{\longrightarrow}}
        \begin{amatrix}{3}
            1 & 2 & 3 & b_1\\
            0 & 3 & 3 & b_1+b_2\\
            0 & -4 & -4 & -b_1+b_3
        \end{amatrix} \\
        &\stackrel{F_2/3}{\longrightarrow}
        \begin{amatrix}{3}
            1 & 2 & 3 & b_1\\
            0 & 1 & 1 & \frac13(b_1+b_2)\\
            0 & -4 & -4 & -b_1+b_3
        \end{amatrix}
        \stackrel{F_3+4F_2}{\longrightarrow}
        \begin{amatrix}{3}
            1 & 2 & 3 & b_1\\
            0 & 1 & 1 & \frac13(b_1+b_2)\\
            0 & 0 & 0 & \frac13b_1+\frac43b_2+b_3
        \end{amatrix}.
    \end{align*}
    Luego
    $$
    W_2 = \left\{ (b_1,b_2,b_3) \in \mathbb{R}^3 \ : \ \frac13b_1+\frac43b_2+b_3 = 0 \right\},
    $$
    o equivalentemente,
    $$
    W_2 = \left\{ (x,y,z) \in \mathbb{R}^3 \ : x +4y +3z =0\right\}.
    $$
    Por lo tanto, 
    $$
    W_1 \cap W_2 = \left\{ (x,y,z) \in \mathbb{R}^3 \ : \ x+y-2z=0 \text{ y } x +4y +3z =0\right\}.
    $$
    Para encontrar una base de $W_1 \cap W_2$ debemos resolver el sistema   
    $$
    \begin{cases}
        x+y-2z=0\\
        x +4y +3z =0.
    \end{cases}
    $$
    Planteamos la matriz correspondiente al sistema y hacemos Gauss:
    $$
    \begin{bmatrix}
        1 & 1 & -2\\
        1 & 4 & 3
    \end{bmatrix}
    \stackrel{F_2-F_1}{\longrightarrow}
    \begin{bmatrix}
        1 & 1 & -2\\
        0 & 3 & 5
    \end{bmatrix}
    \stackrel{F_2/3}{\longrightarrow}
    \begin{bmatrix}
        1 & 1 & -2\\
        0 & 1 & \frac53
    \end{bmatrix}
    \stackrel{F_1-F_2}{\longrightarrow}
    \begin{bmatrix}
        1 & 0 & -\frac{11}3\\
        0 & 1 & \frac53
    \end{bmatrix}.
    $$
    Es decir que la solución del sistema es
    $$
    \begin{cases}
        x = -\frac{11}3z\\
        y = \frac53z.
    \end{cases}
    $$

    Por lo tanto, una base de $W_1 \cap W_2$ es $\{(-11,5,3)\}$ y la dimensión es $1$.

    \vskip .3cm
    \ref{suma-subespacios} Para encontrar vectores que generan $W_1+W_2$ unimos el conjunto de vectores que generan $W_1$ con el conjunto de vectores que generan $W_2$. Ahora bien, 
    \begin{align*}
        W_1 &= \{ (x,y,z)\in\mathbb{R}^3\ : \ x+y-2z=0\}  \\
        &=  \{ (x,y,z)\in\mathbb{R}^3\ : \ x=-y+2z\}  \\
        &=  \{ (-y+2z,y,z)\in\mathbb{R}^3\ : \ y,z\in\mathbb{R}\}  \\
        &=  \{ y(-1,1,0)+z(2,0,1)\in\mathbb{R}^3\ : \ y,z\in\mathbb{R}\}.
    \end{align*}
    Luego $\{(-1,1,0),(2,0,1)\}$ es un conjunto de generadores de $W_1$. Por otra parte, $\{(1,-1,1),(2,1,-2),(3,0,-1)\}$ es un conjunto de generadores de $W_2$. Por lo tanto, $\{(-1,1,0),(2,0,1),(1,-1,1),(2,1,-2),(3,0,-1)\}$ es un conjunto de generadores de $W_1+W_2$.

    Para encontrar una base de $W_1 +W_2$ plantemos la matriz cuyas filas son los vectores del conjunto de generadores y encontramos una MRF:
    \begin{align*}
        &\begin{bmatrix}
            -1 & 1 & 0\\
            2 & 0 & 1\\
            1 & -1 & 1\\
            2 & 1 & -2\\
            3 & 0 & -1
        \end{bmatrix}
        \stackrel{F_2+2F_1}{\stackrel{F_3+F_1}{\stackrel{F_4+2F_1}{\stackrel{F_5+3F_1}{\longrightarrow}}}}
        \begin{bmatrix}
            -1 & 1 & 0\\
            0 & 2 & 1\\
            0 & 0 & 1\\
            0 & 3 & -2\\
            0 & 3 & -1
        \end{bmatrix} \\
        &\stackrel{F_5-F_3}{\stackrel{F_4-F_3}{\stackrel{F_2/2}{\stackrel{F_5-F_4}{\longrightarrow}}}}
        \begin{bmatrix}
            -1 & 1 & 0\\
            0 & 1 & \frac12\\
            0 & 0 & 1\\
            0 & 0 & -\frac52\\
            0 & 0 & \frac12
        \end{bmatrix}
        \stackrel{F_1-F_2}{\stackrel{F_4+\frac52F_3}{\stackrel{F_5-\frac12F_3}{\longrightarrow}}}
        \begin{bmatrix}
            -1 & 0 & -\frac12\\
            0 & 1 & \frac12\\
            0 & 0 & 1\\
            0 & 0 & 0\\
            0 & 0 & 0
        \end{bmatrix}.
    \end{align*}
    Luego $\{(-1,0,-\frac12),(0,1,\frac12),(0,0,1)\}$ es una base de $W_1+W_2$ y por lo tanto la dimensión es $3$, es decir el  subespacio es $\R^3$.

    \qed     
        
    \item\label{verdadero o falso} Decidir si las siguientes afirmaciones son verdaderas o falsas. Justificar.
    
    \begin{enumerate}
    \item\label{VoF-a} Si $W_1$ y $W_2$ son subespacios vectoriales de $\mathbb{K}^8$ de dimensión $5$, entonces $W_1\cap W_2=0$.
    \item\label{VoF-b} Si $W$ es un subespacio de $\mathbb{K}^{2\times2}$ de dimensión $2$, entonces existe una matriz triangular superior no nula que pertence a $W$.
    \item\label{VoF-c} Sean $v_1, v_2, w\in \mathbb{K}^{n}$ y $A\in\mathbb{K}^{n\times n}$ tales que $Av_1=Av_2=0\neq Aw$. Si $\{v_1, v_2\}$ es LI, entonces $\{v_1,v_2,w\}$ también es LI.
    \item\label{cos}  $\{1,{\rm sen}(x),\cos(x)\}$ es un subconjunto LI del espacio de funciones de $\mathbb{R}$ en $\mathbb{R}$.
    \item\label{cos2}  $\{1,{\rm sen}^2(x),\cos^2(x)\}$ es un subconjunto LI del espacio de funciones $\mathbb{R}$ en $\mathbb{R}$.
    \item\label{exponencial}  $\{e^{\lambda_1x},e^{\lambda_2x},e^{\lambda_3x}\}$ es un subconjunto LI del espacio de funciones de
    $\mathbb{R}$ en $\mathbb{R}$, si $\lambda_1$, $\lambda_2$ y $\lambda_3$ son todos distintos.
    \end{enumerate}
    
    
    \rta 

    \ref{VoF-a} Falso. Por ejemplo, si $W_1=\langle e_1, e_2, e_3, e_4, e_5\rangle$ y $W_2=\langle e_4, e_5, e_6, e_7, e_8\rangle$, entonces $W_1\cap W_2=\langle e_4, e_5\rangle$. En este ejemplo $\dim W_1=\dim W_2=5$ y $\dim W_1\cap W_2=2$.

    \vskip .3cm
    \ref{VoF-b} Verdadero. Sean 
    $$
    A=\begin{bmatrix}
        a & b\\
        c & d
    \end{bmatrix}
    \quad\text{y}\quad
    B=\begin{bmatrix}
        a' & b'\\
        c' & d'
    \end{bmatrix}
    $$
    matrices que son base del subespacio de dimensión $2$. Si $c=0$, entonces $A$ es triangular superior. Si $c'=0$, entonces $B$ es triangular superior. Si $c\neq 0$ y $c'\neq 0$, entonces $\frac1cA-\frac1{c'}B$ es cero en fila $2$ y columna $1$ y por lo tanto es triangular superior y no nula (pues $A$ y $B$ son LI). Es decir en cualquier caso encontramos una matriz no nula y triangular superior.


    \vskip .3cm
    \ref{VoF-c} Verdadero. Seaan $\lambda_1,\lambda_2,\lambda_3\in\mathbb{K}$ tales que $\lambda_1v_1+\lambda_2v_2+\lambda_3w=0$. 

    Ahora bien,
    $$
    0=A(\lambda_1v_1+\lambda_2v_2+\lambda_3w)=\lambda_1 Av_1+\lambda_2 Av_2+\lambda_3 Aw= \lambda_3Aw.
    $$
    Como $Aw\neq 0$, entonces $\lambda_3=0$. Por lo tanto, $\lambda_1v_1+\lambda_2v_2=0$. Como $\{v_1,v_2\}$ es LI, entonces $\lambda_1=\lambda_2=0$. Es decir, $\{v_1,v_2,w\}$ es LI.


    \vskip .3cm
    \ref{cos} Verdadero. Supongamos  que 
    $$
    a + b\sen(x) + c\cos(x) = 0.
    $$
    Evaluando en $x=0$ obtenemos $a+c=0$.  Evaluando en $x=\pi$ obtenemos $a-c=0$. Sumando ambas ecuaciones obtenemos $a=0$. Luego $c=0$ y por lo tanto $b=0$. Es decir, $\{1,{\rm sen}(x),\cos(x)\}$ es LI. 

    \vskip .3cm
    \ref{cos2} Falso. Por ejemplo, $\cos^2(x)=1-{\rm sen}^2(x)$, luego $\cos^2(x)$ es combinación lineal de $1$ y ${\rm sen}^2(x)$.

    \vskip .3cm
    \ref{exponencial} Verdadero. Supongamos que 
    \begin{equation}
        a e^{\lambda_1x} + b e^{\lambda_2x} + c e^{\lambda_3x} = 0.
    \label{eq-1}
    \end{equation}
    Si derivamos la ecuación \eqref{eq-1} obtenemos
    \begin{equation}
        a \lambda_1 e^{\lambda_1x} + b \lambda_2 e^{\lambda_2x} + c \lambda_3 e^{\lambda_3x} = 0.
    \label{eq-2}
    \end{equation}
    Derivemos nuevamente:
    \begin{equation}
        a \lambda_1^2 e^{\lambda_1x} + b \lambda_2^2 e^{\lambda_2x} + c \lambda_3^2 e^{\lambda_3x} = 0.
    \label{eq-3}    
    \end{equation}
    Especializando  \eqref{eq-1}, \eqref{eq-2}  y \eqref{eq-3} en $x=0$ obtenemos   
    \begin{align*}
        a + b + c &= 0,\\
        a \lambda_1 + b \lambda_2 + c \lambda_3 &= 0,\\
        a \lambda_1^2 + b \lambda_2^2 + c \lambda_3^2 &= 0.
    \end{align*}
    Es decir, el sistema
    $$
    \begin{bmatrix}
        1 & 1 & 1\\
        \lambda_1 & \lambda_2 & \lambda_3\\
        \lambda_1^2 & \lambda_2^2 & \lambda_3^2
    \end{bmatrix}
    \begin{bmatrix}
        a\\
        b\\
        c
    \end{bmatrix}
    =
    \begin{bmatrix}
        0\\
        0\\
        0
    \end{bmatrix}.
    $$
    Observar que la matriz de la izquierda es similar a la matriz de Vandermonde y es fácil probar que tiene el mismo determinante, es decir $(\lambda_3-\lambda_2) (\lambda_3-\lambda_1) (\lambda_2-\lambda_1)$ (ver práctico \ref{practico-4}, ejercicio \ref{vandermonde}). Como los $\lambda_i$ son distintos, entonces el determinante es distinto de cero y por lo tanto la única solución del sistema es $a=b=c=0$. Es decir, $\{e^{\lambda_1x},e^{\lambda_2x},e^{\lambda_3x}\}$ es LI.



    \qed     
    
    
    
    
    \end{enumerate}
    
    
%%%======================= 
%%% CAP7 =================
    
\chapter{Soluciones\\Álgebra  II -- Año 2024/1 -- FAMAF}\label{practico-7}

\begin{enumerate}[topsep=6pt, itemsep=.4cm]


    \item\label{transf-lineales-incisos} Decidir si las siguientes funciones son transformaciones lineales entre los respectivos espacios vectoriales sobre $\mathbb{K}$.
    \begin{enumerate}[resume, topsep=5pt,itemsep=5pt]
    \item\label{transf-lineales-a} La traza $\operatorname{Tr}:\mathbb{K}^{n\times n}\longrightarrow\mathbb{K}$ (recordar ejercicio \ref{traza}\,\ref{ej:traza} del Práctico  \ref{practico-3}) 
    \item\label{transf-lineales-b} $T:\mathbb{K}[x]\longrightarrow\mathbb{K}[x]$, $T(p(x))=q(x)\,p(x)$ donde $q(x)$ es un polinomio fijo.
    \item\label{transf-lineales-c} $T:\mathbb{K}^2\longrightarrow\mathbb{K}$, $T(x,y)=xy$
    \item\label{transf-lineales-d} $T:\mathbb{K}^2\longrightarrow\mathbb{K}^3$, $T(x,y)=(x,y,1)$
    \item\label{transf-lineales-e} El determinante $\operatorname{det}:\mathbb{K}^{n\times n}\longrightarrow\mathbb{K}$.
    \end{enumerate}

    \rta

    \ref{transf-lineales-a} Sí, es un transformación lineal. En efecto, si $A,B \in \mathbb{K}^{n\times n}$ y $\lambda \in \mathbb{K}$, entonces 
    \begin{align*}
        \operatorname{Tr}(A+\lambda B) &= \sum_{i=1}^n (a_{ii} + \lambda b_{ii}) = \sum_{i=1}^n a_{ii} + \lambda \sum_{i=1}^n b_{ii} = \operatorname{Tr}(A) + \lambda \operatorname{Tr}(B).
    \end{align*}

    \vskip .3cm
    \ref{transf-lineales-b} Sí, es un transformación lineal. En efecto, si $r,s \in \mathbb{K}[x]$ y $\lambda \in \mathbb{K}$, entonces
    \begin{align*}
        T(r+\lambda s) &= q(r+\lambda s) = qr + \lambda qs = T(r) + \lambda T(s).
    \end{align*}

    \vskip .3cm
    \ref{transf-lineales-c} No, no es una transformación lineal. Por ejemplo,  que $T(2(1,1)) \ne 2T(1,1)$. Por un lado, $T(2(1,1))=T(2,2) = 2 \cdot 2 = 4$. Por otro lado, $2T(1,1) = 2 \cdot 1 \cdot 1 = 2$. 

    \vskip .3cm
    \ref{transf-lineales-d} No, no es una transformación lineal. Por ejemplo, veamos que $T(0,0) = (0,0,1)\ne (0,0,0)$.

    \vskip .3cm
    \ref{transf-lineales-e} No, no es una transformación lineal. Por ejemplo,  $\operatorname{det}(2\Id_2) = 4 \ne 2 = 2 \operatorname{det}(\Id_2)$. En general, $\operatorname{det}(\lambda A) = \lambda^n \operatorname{det}(A)$, donde $n$ es el tamaño de la matriz $A$.

    \qed
    
    
    \item Sea $T:\mathbb{C}\longrightarrow\mathbb{C}$, $T(z)=\overline{z}$.
    \begin{enumerate}
    \item\label{conj-C} Considerar a $\mathbb{C}$ como un $\mathbb{C}$-espacio vectorial y decidir si $T$ es una transformación lineal.
    \item\label{conj-R} Considerar a $\mathbb{C}$ como un $\mathbb{R}$-espacio vectorial y decidir si $T$ es una transformación lineal.
    \end{enumerate}
    
    \rta

    \ref{conj-C} No, no es una transformación lineal. Por ejemplo, $T(2i) = -2i \ne 2i = 2T(i)$.
    
    \vskip .3cm
    \ref{conj-R} Sí, es una transformación lineal. En efecto, si $a+bi, a'+b'i \in \mathbb{C}$ y $\lambda \in \mathbb{R}$, entonces  
    \begin{align*}
        T((a+bi) + \lambda (a'+b'i)) &= T((a+\lambda a') + (b+\lambda b')i) = (a+\lambda a') - (b+\lambda b')i \\
        &= (a-bi) + \lambda (a'-b'i) = T(a+bi) + \lambda T(a'+b'i).
    \end{align*}
    \qed

    
    
    \item\label{T en la base} Sea $T:\mathbb{K}^3\longrightarrow\mathbb{K}^3$ una transformación lineal tal que $T(e_1)=(1,2,3)$, $T(e_2)=(-1,0,5)$ y $T(e_3)=(-2,3,1)$. 
        \begin{enumerate}
        \item\label{T en dos vectores} Calcular $T(2,3,8)$ y $T(0,1,-1)$. 
        \item\label{T en la base b} Calcular $T(x,y,z)$ para todo $(x,y,z)\in\mathbb{K}^3$. Es decir, dar una fórmula para $T$ como la del ejercicio \ref{Txyz}.
        \item\label{matriz otro}  Encontrar una matriz $A\in\mathbb{K}^{3\times3}$ tal que
        $T(x,y,z)=A\begin{bmatrix}
        x\\y\\z \end{bmatrix}$. En esta parte del ejercicio escribiremos/pensaremos a los vectores de $\mathbb{K}^3$ como columnas.
        \end{enumerate}
    
        \rta

        \ref{T en dos vectores} 
        \begin{align*}
            T(2,3,8) &= T(2e_1 + 3e_2 + 8e_3) = 2T(e_1) + 3T(e_2) + 8T(e_3)
            \\
            & = 2(1,2,3) + 3(-1,0,5) +8(-2,3,1) \\
            &= (2,4,6) + (-3,0,15) + (-16,24,8) \\
            &= (-17,28,29).
        \end{align*}
        \begin{align*}
            T(0,1,-1) &= T(0e_1 + 1e_2 - 1e_3) = 0T(e_1) + 1T(e_2) - 1T(e_3)\\
            & = 0(1,2,3) + 1(-1,0,5) - 1(-2,3,1)\\ &= (0,0,0)+(-1,0,5)+(2,-3,-1)\\
            & = (1,-3,4).
        \end{align*}

        \vskip .3cm
        \ref{T en la base b} 
        \begin{align*}
            T(x,y,z) &= T(xe_1 + ye_2 + ze_3) = xT(e_1) + yT(e_2) + zT(e_3)
            \\
            & = x(1,2,3) + y(-1,0,5) +z(-2,3,1) \\
            & = (x,2x,3x) + (-y,0,5y) +(-2z,3z,z) \\
            &= (x-y-2z,2x+3z,3x+5y+z).
        \end{align*}

        \vskip .3cm
        \ref{matriz otro} Observar que  
        $$
        \begin{bmatrix} a&b&c \end{bmatrix} \begin{bmatrix} x\\y\\z \end{bmatrix} = ax+by+cz.
        $$ 
        Basándonos en esta observación, obtenemos la matrizs
        \begin{align*}
            A = \begin{bmatrix}
                1 & -1 & -2 \\
                2 & 0 & 3 \\
                3 & 5 & 1
            \end{bmatrix}.
        \end{align*}
        Entonces,
        \begin{align*}
            \begin{bmatrix}
                1 & -1 & -2 \\
                2 & 0 & 3 \\
                3 & 5 & 1
            \end{bmatrix} \begin{bmatrix}
                x\\y\\z
            \end{bmatrix} = \begin{bmatrix}
                x-y-2z\\2x+3z\\3x+5y+z
            \end{bmatrix} = T(x,y,z).
        \end{align*}
    \qed
    \vskip .3cm


        
    \item \label{lineales1} Para cada una de las siguientes transformaciones lineales calcular el núcleo y la imagen. Describir ambos subespacios implícitamente y encontrar una base de cada uno de ellos.	\begin{enumerate}
		\item\label{lineales1-a} $T:\R^2 \longrightarrow \R^3$, $T(x,y)=(x-y,x+y,2x+3y)$.
		\item\label{lineales1-b} $S:\R^3 \longrightarrow \R^2$, $S(x,y,z)=(x-y+z,2x-y+2z)$.
	\end{enumerate}

    \rta el método más general para resolver este tipo de ejercicios es el siguiente:
    dada $T: \R^n \to \R^m$ transformación lineal,  calculamos la imagen encontrando la ecuación implícita de los $(b_1,\ldots,b_m) \in \R^m$ tales que $T(x_1,\ldots,x_n) = (b_1,\ldots,b_m)$ para $(x_1,\ldots,x_n) \in \R^n$. Esto se hará planteando las ecuaciones lineales y resolviendo por Gauss obtenemos una  MRF $[A|b]$. Las filas de $A$ que son cero  nos darán las condiciones que deben cumplir los $b_i$.  Resolviendo  las ecuaciones que deben cumplir los $b_i$ obtenemos una base de la imagen de $T$. Por otro lado, con las ecuaciones ya resueltas y especializando en  $b_1= \cdots = b_m = 0$ obtenemos las ecuaciones implícitas del núcleo en las filas no nulas de $A$. Una base de  núcleo se obtiene despejando los pivotes.

    \ref{lineales1-a} Como ya dijimos vamos a caracterizar los $(b_1,b_2,b_3) \in \R^3$ tales que $T(x,y) = (b_1,b_2,b_3)$ para $(x,y) \in \R^2$. Es decir, debemos resolver el sistema
    \begin{align*}
        (x-y,x+y,2x+3y) &= (b_1,b_2,b_3) \\
        \Rightarrow \quad x-y &= b_1, \\
        x+y &= b_2, \\
        2x+3y &= b_3.
    \end{align*}
    Resolvamos el sistema con matrices aumentadas:
    \begin{align*}
        &\begin{amatrix}{2}
            1 & -1 & b_1 \\
            1 & 1 & b_2 \\
            2 & 3 & b_3
        \end{amatrix}
        \stackrel{F_3-2F_1}{\stackrel{F_2-F_1}{\longrightarrow}}
        \begin{amatrix}{2}
            1 & -1 & b_1 \\
            0 & 2 & -b_1+b_2 \\
            0 & 5 & -2b_1+b_3
        \end{amatrix}\\
        &\stackrel{F_2/2}{\longrightarrow}
        \begin{amatrix}{2}
            1 & -1 & b_1 \\
            0 & 1 & -b_1/2+b_2/2 \\
            0 & 5 & -2b_1+b_3
        \end{amatrix}
        \stackrel{F_1+F_2}{\stackrel{F_3-5F_2}{\longrightarrow}}
        \begin{amatrix}{2}
            1 & 0 & b_1/2+b_2/2  \\
            0 & 1 & -b_1/2+b_2/2 \\
            0 & 0 & b_1/2-5b_2/2+b_3
        \end{amatrix}. \tag{*}
    \end{align*}

    Luego, el sistema tiene solución si y solo si $b_1/2-5b_2/2+b_3 = 0$. Es decir,
    \begin{equation}\label{lineales1-a-imagen}
        \operatorname{Im}(T) = \{(b_1,b_2,b_3) \in \R^3: b_1-5b_2+2b_3 = 0\}.
    \end{equation}
    Encontrar una base de $\operatorname{Im}(T)$ es fácil:
    \begin{align*}
        \operatorname{Im}(T) &= \{(b_1,b_2,b_3) \in \R^3: b_1-5b_2+2b_3 = 0\} \\
        &= \{(b_1,b_2,b_3)\in \R^3: b_1 =5b_2-2b_3 \} \\
        &= \{(5b_2-2b_3,b_2,b_3): b_2,b_3 \in \R \} \\
        &= \{b_2(5,1,0)+b_3(-2,0,1): b_2,b_3 \in \R \} \\
        &= \langle (5,1,0),(-2,0,1) \rangle.
    \end{align*}
    Es decir podemos considerar como base de $\operatorname{Im}(T)$ a $\{(5,1,0),(-2,0,1)\}$.

    Para encontrar la ecuación implícita del núcleo debemos resolver el sistema homogéneo asociado a $T(x) =0$. Es decir, debemos resolver el sistema
    \begin{align*}
        (x-y,x+y,2x+3y) &= (0,0,0) \\
        \Rightarrow \quad x-y &= 0, \\
        x+y &= 0, \\
        2x+3y &= 0.
    \end{align*}
    Pero esto ya lo hicimos más arriba, si tomamos $b_1=b_2=b_3=0$ en la MRF (*). Claramente, la  matriz nos indica que $x=y=0$. Es decir, $\operatorname{Nu}(T) = \{(0,0)\}$ y  la base es el conjunto $\emptyset$.


    \vskip .3cm
    \ref{lineales1-b} Como ya dijimos vamos a caracterizar los $(b_1,b_2) \in \R^2$ tales que $S(x,y,z) = (b_1,b_2)$ para $(x,y,z) \in \R^3$. Es decir, debemos resolver el sistema
    \begin{align*}
        (x-y+z,2x-y+2z) &= (b_1,b_2) \\
        \Rightarrow \quad x-y+z &= b_1, \\
        2x-y+2z &= b_2.
    \end{align*}
    Resolvamos el sistema con matrices aumentadas:
    \begin{align*}
        &\begin{amatrix}{3}
            1 & -1 & 1 & b_1 \\
            2 & -1 & 2 & b_2
        \end{amatrix}
        \stackrel{F_2-2F_1}{\longrightarrow}
        \begin{amatrix}{3}
            1 & -1 & 1 & b_1 \\
            0 & 1 & 0 & -2b_1+b_2
        \end{amatrix}\\
        &\stackrel{F_1+F_2}{\longrightarrow}
        \begin{amatrix}{3}
            1 & 0 & 1 & -b_1+b_2 \\
            0 & 1 & 0 & b_2-2b_1
        \end{amatrix}. \tag{**}
    \end{align*}
    Como no hay ninguna condición sobre los $b_i$, el sistema tiene solución para todo $(b_1,b_2) \in \R^2$. Es decir, 
    \begin{equation}\label{lineales1-b-imagen}
        \operatorname{Im}(S) = \R^2.    
    \end{equation}
    Claramente,  una base  podría ser $\{(1,0),(0,1)\}$.

    Para encontrar la ecuación implícita del núcleo debemos resolver el sistema homogéneo asociado a $S(x) =0$. Es decir, debemos resolver el sistema
    \begin{align*}
        (x-y+z,2x-y+2z) &= (0,0) \\
        \Rightarrow \quad x-y+z &= 0, \\
        2x-y+2z &= 0.
    \end{align*}
    Pero esto ya lo hicimos más arriba, si tomamos $b_1=b_2=0$ en la MRF (**). Claramente, la  matriz nos indica que $x+z=0$ e $y=0$. Es decir, 
    \begin{align*}
        \operatorname{Nu}(S) &= \{(x,y,z) \in \R^3: x+z=0, y =0 \}  \tag{***}\\ &= \{(-z,0,z) : z \in \R\} = \{z(-1,0,1) : z \in \R\}
        \\&= \langle (1,0,-1) \rangle.
    \end{align*}
    Luego (***) es la ecuación implícita del núcleo y $\{(1,0,-1)\}$ es una base de $\operatorname{Nu}(S)$.
    
    \qed

        
    \item\label{Txyz} Sea $T:\mathbb{K}^3\longrightarrow\mathbb{K}^3$ definida por $T(x,y,z)=(x+2y+3z, y-z,x+5y)$.
        \begin{enumerate}
        \item\label{Txyz-vectores-nucleo} Decir cuáles de los siguientes vectores están en el núcleo: $(1,1,1)$, $(-5,1,1)$.
        \item\label{Txyz imagen} Decir cuáles de los siguientes vectores están en la imagen: $(0,1,0)$, $(0,1,3)$.
        \item\label{Txyz nucleo T implicito} Describir mediante ecuaciones (implícitamente) el núcleo de $T$.
        \item\label{Txyz imagen T generadores} Dar un conjunto de generadores de la imagen.
        \item\label{matriz de T} Encontrar una matriz  $A\in\mathbb{K}^{3\times 3}$ tal que $T(x,y,z)=A\begin{bmatrix}
            x\\y\\z \end{bmatrix}
            $.  Como en el ejercicio  \ref{T en la base}\,\ref{matriz otro} pensamos a los vectores como columnas.
        \end{enumerate}
    
    \rta

    \ref{Txyz-vectores-nucleo} 
    $$
    T(1,1,1) = (1+2+3,1-1,1+5) = (6,0,6) \ne (0,0,0),
    $$
    por lo tanto $(1,1,1) \notin \operatorname{Nu}(T)$.

    $$
    T(-5,1,1) = (-5+2+3,1-1,-5+5) = (0,0,0),
    $$
    por lo tanto $(-5,1,1) \in \operatorname{Nu}(T)$.

    \vskip .3cm
    \ref{Txyz imagen}
    
    Supongamos que $T(x,y,z) = (1,1,1)$. Entonces,
    \begin{align*}
        (x+2y+3z, y-z,x+5y) &= (1,1,1) \\
        \Rightarrow \quad x+2y+3z &= 1, \\
        y-z &= 1, \\
        x+5y &= 1.
    \end{align*} 
    Resolvamos el sistema:
    \begin{align*}
        &\begin{amatrix}{3}
            1 & 2 & 3 &1 \\
            0 & 1 & -1 &1\\
            1 & 5 & 0&1
        \end{amatrix} 
        \stackrel{F_3-F_1}{\longrightarrow}
        \begin{amatrix}{3}
            1 & 2 & 3 &1 \\
            0 & 1 & -1 &1\\
            0 & 3 & -3&0
        \end{amatrix}
        \stackrel{F_3-3F_2}{\longrightarrow}
        \begin{amatrix}{3}
            1 & 2 & 3 &1 \\
            0 & 1 & -1 &1\\
            0 & 0 & 0&-3
        \end{amatrix}.
    \end{align*}
    La última fila de la última matriz nos dice que el sistema no tiene solución y por lo tanto $(1,1,1) \notin \operatorname{Im}(T)$.

    \vskip .3cm
    Supongamos que $T(x,y,z) = (0,1,3)$. Entonces,
    \begin{align*}
        (x+2y+3z, y-z,x+5y) &= (0,1,3) \\
        \Rightarrow \quad x+2y+3z &= 0, \\
        y-z &= 1, \\
        x+5y &= 3.
    \end{align*}
    Resolvamos el sistema:
    \begin{align*}
        &\begin{amatrix}{3}
            1 & 2 & 3 &0 \\
            0 & 1 & -1 &1\\
            1 & 5 & 0&3
        \end{amatrix} 
        \stackrel{F_3-F_1}{\longrightarrow}
        \begin{amatrix}{3}
            1 & 2 & 3 &0 \\
            0 & 1 & -1 &1\\
            0 & 3 & -3&3
        \end{amatrix}
        \stackrel{F_3-3F_2}{\longrightarrow}
        \begin{amatrix}{3}
            1 & 2 & 3 &0 \\
            0 & 1 & -1 &1\\
            0 & 0 & 0&0
        \end{amatrix} \\
        &\stackrel{F_1-2F_2}{\longrightarrow}
        \begin{amatrix}{3}
            1 & 0 & 5 &-2 \\
            0 & 1 & -1 &1\\
            0 & 0 & 0&0
        \end{amatrix}.
    \end{align*}
    Por lo tanto, $x = -2-5z$, $y = 1+z$ y $z$ es libre. Es decir, el sistema tiene infinitas soluciones, por ejemplo, para $z=0$ obtenemos $(x,y,z) = (-2,1,0)$ y es claro entonces que $T(-2,1,0) = (0,1,3)$.  Para $z=1$ obtenemos $(x,y,z) = (-7,2,1)$. En conclusión, $(0,1,3) \in \operatorname{Im}(T)$. 

    \vskip .3cm
    \ref{Txyz nucleo T implicito}   un vector $(x,y,z)$ está en el núcleo de $T$ si y sólo si $T(x,y,z) = (0,0,0)$. Es decir, si y sólo si
    \begin{align*}
        (x+2y+3z, y-z,x+5y) &= (0,0,0) \\
        \Rightarrow \quad x+2y+3z &= 0, \\
        y-z &= 0, \\
        x+5y &= 0.
    \end{align*}
    Resolvamos el sistema:
    \begin{align*}
        &\begin{bmatrix}
            1 & 2 & 3 \\
            0 & 1 & -1\\
            1 & 5 & 0
        \end{bmatrix} 
        \stackrel{F_3-F_1}{\longrightarrow}
        \begin{bmatrix}
            1 & 2 & 3 \\
            0 & 1 & -1\\
            0 & 3 & -3
        \end{bmatrix}
        \stackrel{F_3-3F_2}{\longrightarrow}
        \begin{bmatrix}
            1 & 2 & 3\\
            0 & 1 & -1\\
            0 & 0 & 0
        \end{bmatrix} 
        \stackrel{F_1-2F_2}{\longrightarrow}
        \begin{bmatrix}
            1 & 0 & 5 \\
            0 & 1 & -1\\
            0 & 0 & 0
        \end{bmatrix}.
    \end{align*}
    Por lo tanto, $(x,y,z) \in \operatorname{Nu}(T)$ si y solo si  $x+5z=0$, $y - z=0$. Es decir,
    $$
    \operatorname{Nu}(T) = \{(x,y,z) \in \mathbb{K}^3: x+5z=0, y - z=0\}.
    $$

    \vskip .3cm
    \ref{Txyz imagen T generadores} (primera forma) Debemos averiguar, los vectores $(x,y,z)$ tales que $T(x,y,z) = (b_1,b_2,b_3)$ para algún $ (b_1,b_2,b_3) \in \mathbb{K}^3$. Es decir, debemos averiguar, los vectores $(x,y,z)$ tales que
    \begin{align*}
        (x+2y+3z, y-z,x+5y) &= (b_1,b_2,b_3) \\
        \Rightarrow \quad x+2y+3z &= b_1, \\
        y-z &= b_2, \\
        x+5y &= b_3.
    \end{align*}
    Resolvamos el sistema con matrices aumentadas:
    \begin{align*}
        &\begin{amatrix}{3}
            1 & 2 & 3 &b_1 \\
            0 & 1 & -1 &b_2\\
            1 & 5 & 0&b_3
        \end{amatrix} 
        \stackrel{F_3-F_1}{\longrightarrow}
        \begin{amatrix}{3}
            1 & 2 & 3 &b_1 \\
            0 & 1 & -1 &b_2\\
            0 & 3 & -3&b_3-b_1
        \end{amatrix}
        \stackrel{F_3-3F_2}{\longrightarrow}
        \begin{amatrix}{3}
            1 & 2 & 3 &b_1 \\
            0 & 1 & -1 &b_2\\
            0 & 0 & 0&b_3-b_1-3b_2
        \end{amatrix} \\
        &\stackrel{F_1-2F_2}{\longrightarrow}
        \begin{amatrix}{3}
            1 & 0 & 5 &b_1-2b_2 \\
            0 & 1 & -1 &b_2\\
            0 & 0 & 0&b_3-b_1-3b_2
        \end{amatrix}.
    \end{align*}
    El sistema tiene solución si y solo si $b_3-b_1-3b_2 = 0$. Es decir,
    \begin{align*}
        \operatorname{Im}(T) &= \{(b_1,b_2,b_3) \in \mathbb{K}^3: b_3-b_1-3b_2 = 0\} \\ &= \{(b_1,b_2,b_3) \in \mathbb{K}^3: b_3 = b_1+3b_2\} \\
        &= \{(b_1,b_2,b_1+3b_2) : b_1,b_2 \in \mathbb{K}\} \\
        &= \{(b_1,0,b_1)  + (0,b_2,3b_2) : b_2, b_3 \in \mathbb{K}\} \\
        &= \langle (1,0,1), (0,1,3) \rangle.
    \end{align*}
    Es decir que $(1,0,1), (0,1,3)$ son generadores de $\operatorname{Im}(T)$.

    \ref{Txyz imagen T generadores} (segunda forma, más sencilla) Sabemos que $T$ de una base es un conjunto de generadores de la imagen. Por lo tanto, $\{T(e_1),T(e_2),T(e_3)\}$ es un conjunto de generadores de la imagen. Ahora bien, $T(e_1) = (1,0,1)$, $T(e_2) = (2,1,5)$ y $T(e_3) = (3,-1,0)$. Por lo tanto,  $(1,0,1),(2,1,5), (3,-1,0)$ son generadores de $\operatorname{Im}(T)$.

    La solución de esta parte del ejercicio termina en el párrafo anterior, pero si queremos obtener una base de la imagen, planteamos una matriz donde las filas son las vectores y con Gauss obtenemos una MRF. Las filas no nulas serán una base de la imagen.
    
    En  este caso:
    \begin{align*}
        &\begin{bmatrix}
            1 & 0 & 1 \\
            2 & 1 & 5\\
            3 & -1 & 0
        \end{bmatrix}
        \underset{F_3-3F_1}{\stackrel{F_2-2F_1}{\longrightarrow}}
        \begin{bmatrix}
            1 & 0 & 1 \\
            0 & 1 & 3\\
            0 & -1 & -3
        \end{bmatrix}
        \stackrel{F_3+F_2}{\longrightarrow}
        \begin{bmatrix}
            1 & 0 & 1 \\
            0 & 1 & 3\\
            0 & 0 & 0
        \end{bmatrix}.    
    \end{align*}
    Por lo tanto, $\{(1,0,1),(0,1,3)\}$ es una base de la imagen de $T$.


    \vskip .3cm
    \ref{matriz de T} Observar  que 
    $$
    F_i(A)\cdot \begin{bmatrix} x\\y\\z \end{bmatrix} = T(x,y,z)_i.
    $$
    Es decir la fila $i$  de $A$ por el vector $(x,y,z)$ nos da la coordenada $i$ de $T(x,y,z)$. Como $T(x,y,z) = (x+2y+3z, y-z,x+5y)$, tenemos que
    $$
    A \begin{bmatrix} x\\y\\z \end{bmatrix} = \begin{bmatrix} x+2y+3z\\ y-z\\x+5y \end{bmatrix},
    $$
    y por lo tanto
    \begin{align*}
        A = \begin{bmatrix}
            1 & 2 & 3 \\
            0 & 1 & -1\\
            1 & 5 & 0
        \end{bmatrix}.  
    \end{align*}
\qed

\vskip .4cm

    
    
    \item\label{tl-matriz} Sea $T: \mathbb{K}^4 \to \mathbb{K}^5$ dada por $T(v) = Av$ donde $A$ es la siguiente matriz
        $$
        A=\begin{bmatrix}
        0& 2& 0&1\\   1& 3& 0&1\\  -1&-1&0&0\\3&0&3&0\\2&1&1&0 \end{bmatrix}
        $$
        \begin{enumerate}[topsep=5pt,itemsep=5pt]
            \item\label{tl-matriz-a} Dar una base del núcleo y de la imagen de $T$. 
            \item\label{tl-matriz-b} Dar la dimensión del núcleo y de la imagen de $T$.
            \item\label{tl-matriz-c} Describir mediante ecuaciones (implícitamente) el núcleo y la imagen de $T$.
            \item\label{tl-matriz-d} Decir cuáles de los siguientes vectores están en el núcleo:
            $(1,2,3,4)$, $(1,-1,-1,2)$, $(1,0,2,1)$.
            \item\label{tl-matriz-e} Decir cuáles de los siguientes vectores están en la imagen:
            $(2,3,-1,0,1)$, $(1,1,0,3,1)$, $(1,0,2,1,0)$.
        \end{enumerate}
        
    \rta primero encontremos una fórmula para $T(x,y,z,t)$, luego resolvamos los incisos. 
    $$
    T(x,y,z,t) = \begin{bmatrix}
        0& 2& 0&1\\   1& 3& 0&1\\  -1&-1&0&0\\3&0&3&0\\2&1&1&0 \end{bmatrix} \begin{bmatrix}
        x\\y\\z\\t
    \end{bmatrix} = \begin{bmatrix}
        2y+t\\x+3y+t\\-x-y\\3x+3z\\2x+y+z
    \end{bmatrix}.
    $$
    Por lo tanto 
    \begin{equation}
        T(x,y,z,t) =  .
    \end{equation}
    
    
    $$(-1,1,9,2) -4(0,0,0,1) -8(0,0,1,0) = (-1,1,1,-2)$$ 

    \ref{tl-matriz-a}  Como $T(v) = (0,0,0,0,0)$ si y solo si $Av=0$, el $\operatorname{Nu}(T)$ es el conjunto de soluciones del sistema $Av=0$,  que resolvemos  con Gauss:
    \begin{align*}
        &\begin{bmatrix}
            0 & 2 & 0 &1\\
            1 & 3 & 0 &1\\
            -1 &-1 &0 &0\\
            3 &0 &3 &0\\
            2 &1 &1 &0
        \end{bmatrix}
        \underset{F_5-2F_2}{\underset{F_4-3F_2}{\stackrel{F_3+F_2}{\longrightarrow}}}
        \begin{bmatrix}
            0 & 2 & 0 &1\\
            1 & 3 & 0 &1\\
            0 &2 &0 &1\\
            0 &-9 &3 &-3\\
            0 &-5 &1 &-2
        \end{bmatrix}
        \stackrel{F_4-3F_5}{\longrightarrow}
        \begin{bmatrix}
            0 & 2 & 0 &1\\
            1 & 3 & 0 &1\\
            0 &2 &0 &1\\
            0 &6 &0 &3\\
            0 &-5 &1 &-2
        \end{bmatrix} \\
        &\underset{F_4-3F_1}{\stackrel{F_3-F_1}{\longrightarrow}}
        \begin{bmatrix}
            0 & 2 & 0 &1\\
            1 & 3 & 0 &1\\
            0 &0 &0 &0\\
            0 &0 &0 &0\\
            0 &-5 &1 &-2
        \end{bmatrix}
        \stackrel{F_1/2}{\longrightarrow}
        \begin{bmatrix}
            0 & 1 & 0 &1/2\\
            1 & 3 & 0 &1\\
            0 &0 &0 &0\\
            0 &0 &0 &0\\
            0 &-5 &1 &-2
        \end{bmatrix}
        \underset{F_5+5F_2}{\stackrel{F_2-3F_1}{\longrightarrow}}
        \begin{bmatrix}
            0 & 1 & 0 &1/2\\
            1 & 0 & 0 &-1/2\\
            0 &0 &0 &0\\
            0 &0 &0 &0\\
            0 &0 &1 &1/2
        \end{bmatrix}.
    \end{align*}
    Luego 
    \begin{equation}\label{eq-del-nucleo}
        \operatorname{Nu}(T) = \{(x,y,z,t) \in \mathbb{K}^4: x -\frac{1}{2}t =0, y + \frac{1}{2}t=0, z + \frac{1}{2}t =0\},
    \end{equation}
    Por lo tanto,  
    \begin{align*}
        \operatorname{Nu}(T) &= \{(x,y,z,t) \in \mathbb{K}^4: x =\frac{1}{2}t, y = -\frac{1}{2}t, z = -\frac{1}{2}t\} \\ &= 
        \left\{\left(\frac{1}{2}t,-\frac{1}{2}t,-\frac{1}{2}t,t\right) : t \in \mathbb{K}\right\} \\ &=
        \left\{t\left(\frac{1}{2},-\frac{1}{2},-\frac{1}{2},1\right) : t \in \mathbb{K}\right\} \\ &=
        \langle \left(\frac{1}{2},-\frac{1}{2},-\frac{1}{2},1\right) \rangle.
    \end{align*}
    En  consecuencia, $\{(1,-1,-1,2)\}$ es una base del núcleo de $T$. 

    \vskip .3cm

    Para encontrar una base de la imagen, calculamos $T(e_1),T(e_2),T(e_3),T(e_4)$ que es un sistema de generadores de la imagen. A partir de estos generadores encontramos una base.
    \begin{align*}
        T(e_1) &= (2\cdot 0+0,1+3\cdot 0+0,-1-0,3\cdot 1+3\cdot 0,2\cdot 1+0+0)\\ &= (0,1,-1,3,2), 
    \end{align*}
    \begin{align*}
        T(e_2) &= (2\cdot 1+0,0+3\cdot 1+0,-0-1,3\cdot 0+3\cdot 0,2\cdot 0+1+0)\\ &= (2,3,-1,0,1), 
    \end{align*}
    \begin{align*}
        T(e_3) &= (2\cdot 0+0,0+3\cdot 0+0,-0-0,3\cdot 0+3\cdot 1,2\cdot 0+0+1)\\ &= (0,0,0,3,1), 
    \end{align*}
    \begin{align*}
        T(e_4) &= (2\cdot 0+1,0+3\cdot 0+1,-0-0,3\cdot 0+3\cdot 0,2\cdot 0+0+0)\\ &= (1,1,0,0,0).
    \end{align*}
    Encontramos una base de la imagen haciendo la matriz donde las filas son los vectores anteriores y hacemos Gauss:
    \begin{align*}
        &\begin{bmatrix}
            0 & 1 & -1 &3 &2\\
            2 & 3 & -1 &0 &1\\
            0 & 0 & 0 &3 &1\\
            1 & 1 & 0 &0 &0
        \end{bmatrix}
        \stackrel{F_2-2F_4}{\longrightarrow}
        \begin{bmatrix}
            0 & 1 & -1 &3 &2\\
            0 & 1 & -1 &0 &1\\
            0 & 0 & 0 &3 &1\\
            1 & 1 & 0 &0 &0
        \end{bmatrix}
        \underset{F_4-F_1}{\stackrel{F_2-F_1}{\longrightarrow}}
        \begin{bmatrix}
            0 & 1 & -1 &3 &2\\
            0 & 0 & 0 &-3 &-1\\
            0 & 0 & 0 &3 &1\\
            1 & 0 & 1 &-3 &-2
        \end{bmatrix} \\
        &\underset{F_4+F_3}{\underset{F_2+F_3}{\stackrel{F_1-F_3}{\longrightarrow}}}
        \begin{bmatrix}
            0 & 1 & -1 &0 &1\\
            0 & 0 & 0 &0 &0\\
            0 & 0 & 0 &3 &1\\
            1 & 0 & 1 &0 &-1
        \end{bmatrix}
    \end{align*}
    Luego, una base de la imagen es $\{(1,0,1,0,-1),(0, 1, -1, 0, 1),(0,0,0,3,1)\}$.

    \vskip .3cm
    \ref{tl-matriz-b} Hemos visto más arriba que el núcleo tiene una base de un elemento y la imagen tiene una base de tres elementos, por consiguiente  $\dim \operatorname{Nu}(T) = 1$ y $\dim \operatorname{Im}(T) = 3$. Observar que  $\dim \operatorname{Nu}(T) + \dim \operatorname{Im}(T) = 4$, que es la dimensión del dominio.

    \vskip .3cm
    \ref{tl-matriz-c}  El núcleo está descripto en \eqref{eq-del-nucleo} o,  equivalentemente,
    \begin{equation}\label{eq-del-nucleo-2}
        \operatorname{Nu}(T) = \{(x,y,z,t) \in \mathbb{K}^4: 2x -t =0, 2y + t=0, 2z + t =0\}.
    \end{equation}
    
    Para  la imagen,  debemos plantear la ecuación $T(x,y,z,t) = (b_1,b_2,b_3,b_4,b_5)$ y resolver el sistema. Es decir, debemos resolver
    \begin{align*}
        (2y+t,x+3y+t,-x-y,3x+3z,2x+y+z) &= (b_1,b_2,b_3,b_4,b_5) \\
        \Rightarrow \quad 2y+t &= b_1, \\
        x+3y+t &= b_2, \\
        -x-y &= b_3, \\
        3x+3z &= b_4, \\
        2x+y+z &= b_5.
    \end{align*}
    Resolvamos el sistema con matrices aumentadas:
    \begin{align*}
        &\begin{amatrix}{4}
            0 & 2 & 0 &1 &b_1\\
            1 & 3 & 0 &1 &b_2\\
            -1 &-1 &0 &0 &b_3\\
            3 &0 &3 &0 &b_4\\
            2 &1 &1 &0 &b_5
        \end{amatrix}
        \underset{F_5-2F_2}{\underset{F_4-3F_2}{\stackrel{F_3+F_2}{\longrightarrow}}}
        \begin{amatrix}{4}
            0 & 2 & 0 &1 &b_1\\
            1 & 3 & 0 &1 &b_2\\
            0 &2 &0 &1 &b_2+b_3\\
            0 &-9 &3 &-3 &-3b_2 +b_4\\
            0 &-5 &1 &-2 &+-2b_2b_5
        \end{amatrix} 
    \end{align*}
    \begin{align*}
        &\stackrel{F_3-F_1}{\longrightarrow}
        \begin{amatrix}{4}
            0 & 2 & 0 &1 &b_1\\
            1 & 3 & 0 &1 &b_2\\
            0 &0 &0 &0 &-b_1+b_2+b_3\\
            0 &-9 &3 &-3 &-3b_2 +b_4\\
            0 &-5 &1 &-2 &-2b_2+b_5
        \end{amatrix}
        \stackrel{F_1/2}{\longrightarrow}
        \begin{amatrix}{4}
            0 & 1 & 0 &1/2 &b_1/2\\
            1 & 3 & 0 &1 &b_2\\
            0 &0 &0 &0 &-b_1+b_2+b_3\\
            0 &-9 &3 &-3 &-3b_2 +b_4\\
            0 &-5 &1 &-2 &-2b_2+b_5
        \end{amatrix} 
    \end{align*}
    \begin{align*}
        &\underset{F_5+5F_1}{\underset{F_4+9F_1}{\stackrel{F_2-3F_1}{\longrightarrow}}}
        \begin{amatrix}{4}
            0 & 1 & 0 &1/2 &b_1/2\\
            1 & 0 & 0 &-1/2 &b_2-3b_1/2\\
            0 &0 &0 &0 &-b_1+b_2+b_3\\
            0 &0 &3 &3/2 &-3b_2 +b_4+9b_1/2\\
            0 &0 &1 &1/2 &-2b_2+b_5+5b_1/2
        \end{amatrix} 
    \end{align*}
    \begin{align*}
        &\stackrel{F_4-3F_5}{\longrightarrow}
        \begin{amatrix}{4}
            0 & 1 & 0 &1/2 &b_1/2\\
            1 & 0 & 0 &-1/2 &b_2-3b_1/2\\
            0 &0 &0 &0 &-b_1+b_2+b_3\\
            0 &0 &0 &0 &-6b_1/2+3b_2 +b_4-3b_5\\
            0 &0 &1 &1/2 &-2b_2+b_5+5b_1/2
        \end{amatrix}.
    \end{align*}
    Lo importante de estas ecuaciones son las filas donde  los coeficientes son $0$: $(b_1,b_2.b_3,b_4) \in \operatorname{Im}(T)$ si y solo si $-b_1+b_2+b_3=0$ y $-6b_1/2+3b_2 +b_4-3b_5=0$. Es decir,
    \begin{equation}\label{eq-del-imagen-2}
        \begin{aligned}
        \operatorname{Im}(T) &= \{(b_1,b_2,b_3,b_4,b_5) \in \mathbb{K}^5: \\ 
        &\qquad\qquad  -b_1+b_2+b_3=0 \text{ y } -3b_1+3b_2 +b_4-3b_5=0\}.
        \end{aligned}
    \end{equation}


    \vskip .3cm
    \ref{tl-matriz-d}  Para ver si $(1,2,3,4)$,  $(1,-1,-1,2)$, $(1,0,2,1)$ están en $\operatorname{Nu}(T)$ debemos ver si $T$ de cada vector es $(0,0,0,0,0)$ o si  cumplen con la ecuación \eqref{eq-del-nucleo-2} o si, como vimos en \ref{tl-matriz-a}, son múltiplos de $(1,-1,-1,2)$. El tercer método es el más sencillo y, por lo tanto, lo usaremos. 
    
    Es claro que $(1,2,3,4)$ no es múltiplo de $(1,-1,-1,2)$, por lo tanto, $(1,2,3,4)$ no está en $\operatorname{Nu}(T)$.

    Por  el contrario $(1,-1,-1,2)$ es múltiplo de $(1,-1,-1,2)$, por lo tanto, $(1,-1,-1,2)$ está en $\operatorname{Nu}(T)$.

    Finalmente, $(1,0,2,1)$ no es múltiplo de $(1,-1,-1,2)$, por lo tanto, $(1,0,2,1)$ no está en $\operatorname{Nu}(T)$.


    \vskip .3cm
    \ref{tl-matriz-e}  Para ver si $(2,3,-1,0,1)$, $(1,1,0,3,1)$, $(1,0,2,1,0)$ están en $\operatorname{Im}(T)$ debemos ver si  cumplen con la ecuación \eqref{eq-del-imagen-2}. Es decir 
    $$
    (x_1,x_2,x_3,x_4,x_5) \in \operatorname{Im}(T) \Leftrightarrow -x_1+x_2+x_3=0 \text{\; y } -3x_1+3x_2 +x_4-3x_5=0.
    $$


    Para $(2,3,-1,0,1)$,  $-2+3-1=0$ y $-6+9-3=0$, por lo tanto, $(2,3,-1,0,1)$ está en $\operatorname{Im}(T)$.

    Para $(1,1,0,3,1)$,  $-1+1+0=0$ y $-3+3+3-3=0$, por lo tanto, $(1,1,0,3,1)$ está en $\operatorname{Im}(T)$.

    Para $(1,0,2,1,0)$,  $-1+0+2=1$ y $-3+0+3-0=0$, por lo tanto, $(1,0,2,1,0)$ no está en $\operatorname{Im}(T)$.
    
\qed





        \item Sea $T:\mathbb{K}^{2\times 2}\longrightarrow\mathbb{K}_{4}[x]$ la transformación lineal definida por
        \begin{align*}
        T   \begin{bmatrix}  a&b\\c&d \end{bmatrix} &= (a-c+2d)x^3+(b+2c-d)x^2+ \\
        &\qquad+(-a+2b+5c-4d)x+(2a-b-4c+5d)
        \end{align*}
        \begin{enumerate}
            \item\label{tl-matrices-pol-a} Decir cuáles de los siguientes matrices están en el núcleo:
                \begin{align*}
                    A=\begin{bmatrix}
                        2&0\\0&-1
                    \end{bmatrix},
                \quad
                B=\begin{bmatrix}
                    -1&-1\\1&1
                \end{bmatrix},
                \quad
                C=\begin{bmatrix}
                    -1&-1\\1&0
                \end{bmatrix}.
                \end{align*}
            \item\label{tl-matrices-pol-b} Decir cuáles de los siguientes polinomios están en la imagen:
                \begin{align*}
                    p(x)=x^3+x^2+x+1,\quad q(x)=x^3, \quad r(x)=(x-1)(x-1) 
                \end{align*}
        \end{enumerate}
    
    \rta 

    \ref{tl-matrices-pol-a}  Para ver si $A$, $B$ y $C$ están en el núcleo debemos ver si $T$ de cada matriz es el polinomio nulo. Por la definición de $T$,
    $$
    \begin{bmatrix}  a&b\\c&d \end{bmatrix} \in \operatorname{Nu}(T) \Leftrightarrow \begin{cases}
        a-c+2d=0\\
        b+2c-d=0\\
        -a+2b+5c-4d=0\\
        2a-b-4c+5d=0.
    \end{cases}
    $$
    Podemos comprobar en cada matriz estas ecuacines, o podemos simplificar el sistema para que resulta más fácil al comprobación. Haremos esto último con Gauss:
    \begin{equation*}
        \begin{bmatrix}
            1 & 0 & -1 & 2 \\
            0 & 1 & 2 & -1 \\
            -1 & 2 & 5 & -4 \\
            2 & -1 & -4 & 5
        \end{bmatrix}
        \underset{F_4-2F_1}{\stackrel{F_3+F_1}{\longrightarrow}}
        \begin{bmatrix}
            1 & 0 & -1 & 2 \\
            0 & 1 & 2 & -1 \\
            0 & 2 & 4 & -2 \\
            0 & -1 & -2 & 1
        \end{bmatrix}
        \underset{F_4+F_2}{\stackrel{F_3-2F_2}{\longrightarrow}}
        \begin{bmatrix}
            1 & 0 & -1 & 2 \\
            0 & 1 & 2 & -1 \\
            0 & 0 & 0 & 0 \\
            0 & 0 & 0 & 0
        \end{bmatrix}.
    \end{equation*}
    Por lo tanto, 
    $$
    \begin{bmatrix}  a&b\\c&d \end{bmatrix} \in \operatorname{Nu}(T) \Leftrightarrow \begin{cases}
        a-c+2d\\
        b+2c-d.
    \end{cases}
    $$
    Veamos ahora si  las matrices $A$, $B$ y $C$ cumplen con estas ecuaciones:
    \begin{enumerate}
        \item[$A$:] $a=2$, $b=0$, $c=0$, $d=-1$, por lo tanto, $a-c+2d=2-0+2\cdot (-1)=0$ y $b+2c-d=0+2\cdot 0-(-1)=1$, es decir, $A \notin \operatorname{Nu}(T)$.
        \item[$B$:] $a=-1$, $b=-1$, $c=1$, $d=1$, por lo tanto, $a-c+2d=-1-1+2\cdot 1=0$ y $b+2c-d=-1+2\cdot 1-1=0$, es decir, $B \in \operatorname{Nu}(T)$.
        \item[$C$:] $a=-1$, $b=-1$, $c=1$, $d=0$, por lo tanto, $a-c+2d=-1-1+2\cdot 0=0$ y $b+2c-d=-1+2\cdot 1-0=1$, es decir, $C \notin \operatorname{Nu}(T)$. 
    \end{enumerate}

    \vskip .3cm
    \ref{tl-matrices-pol-b}  Para ver si $p(x)$, $q(x)$ y $r(x)$ están en la imagen debemos caracterizar la imagen.

    Por la definición de $T$, $a_3x^3+a_2x^2+a_1x+a_0 \in \operatorname{Im}(T)$ si y solo si existe $A \in \mathbb{K}^{2\times 2}$ tal que $T(A) = a_3x^3+a_2x^2+a_1x+a_0$. Es decir, si y solo si
    \begin{align*}
        \begin{cases}
            a-c+2d=a_3\\
            b+2c-d=a_2\\
            -a+2b+5c-4d=a_1\\
            2a-b-4c+5d=a_0.
        \end{cases}
    \end{align*}
    Resolvamos el sistema con Gauss.
    \begin{align*}
        &\begin{amatrix}{4}
            1 & 0 & -1 & 2 & a_3 \\
            0 & 1 & 2 & -1 & a_2 \\
            -1 & 2 & 5 & -4 & a_1 \\
            2 & -1 & -4 & 5 & a_0
        \end{amatrix}
        \underset{F_4-2F_1}{\stackrel{F_3+F_1}{\longrightarrow}}
        \begin{amatrix}{4}
            1 & 0 & -1 & 2 & a_3 \\
            0 & 1 & 2 & -1 & a_2 \\
            0 & 2 & 4 & -2 & a_1+a_3 \\
            0 & -1 & -2 & 1 & a_0-2a_3
        \end{amatrix} \\
        &\underset{F_4+F_2}{\stackrel{F_3-2F_2}{\longrightarrow}}
        \begin{amatrix}{4}
            1 & 0 & -1 & 2 & a_3 \\
            0 & 1 & 2 & -1 & a_2 \\
            0 & 0 & 0 & 0 & a_1-2a_2+a_3 \\
            0 & 0 & 0 & 0 & a_0+a_2-2a_3
        \end{amatrix}.
    \end{align*}
    Por lo tanto, 
    $$
    a_3x^3+a_2x^2+a_1x+a_0 \in \operatorname{Im}(T) \quad \Leftrightarrow \quad a_1-2a_2+a_3=0 \text{\; y \;} a_0+a_2-2a_3=0.
    $$

    En  el caso de $p(x)=x^3+x^2+x+1$, $a_3=1$, $a_2=1$, $a_1=1$, $a_0=1$, por lo tanto, $a_1-2a_2+a_3=1-2+1=0$ y $a_0+a_2-2a_3=1+1-2=0$, es decir, $p(x) \in \operatorname{Im}(T)$. 
    
    Para $q(x)=x^3$, $a_3=1$, $a_2=0$, $a_1=0$, $a_0=0$, por lo tanto, $a_1-2a_2+a_3=1 \ne 0$, es decir, $q(x) \notin \operatorname{Im}(T)$. 
    
    Finalmente para $r(x)=(x-1)(x-1) = x^2 -2x +1$, $a_3=0$, $a_2=1$, $a_1=-2$, $a_0=1$, por lo tanto, $a_1-2a_2+a_3=-2-2+0=-4 \ne 0$, es decir, $r(x) \notin \operatorname{Im}(T)$.
    
    \qed



    \item\label{funcional ej}  Sea $T:\mathbb{K}^3\longrightarrow\mathbb{K}$ definida por $T(x,y,z)=x+2y+3z$.
    \begin{enumerate}
        \item\label{funcional ej a} Probar que $T$ es un epimorfismo.
        \item\label{funcional ej b} Dar la dimensión del núcleo de $T$.
        \item\label{funcional ej c} Encontrar una matriz $A$ tal que
            $T(x,y,z)=A\begin{bmatrix}
            x\\y\\z \end{bmatrix}$. ¿De qué tamaño debe ser $A$? Como en el ejercicio \ref{T en la base}\,\ref{matriz otro} pensamos a los vectores como columnas. 
    \end{enumerate}

    \rta
    
    
    \item Determinar cuáles transformaciones lineales de los ejercicios anteriores son monomorfismos, epimorfismos y/o isomorfismos.
    
    \rta


    \item\label{usar-1} Encontrar en cada caso, cuando sea posible, una matriz $A\in\mathbb{K}^{3\times 3}$ tal que la transformación lineal $T:\mathbb{K}^3\longrightarrow\mathbb{K}^3$, $T(v)=Av$, satisfaga las condiciones exigidas (como en el ejercicio  \ref{T en la base}\,\ref{matriz otro} pensamos a los vectores como columnas). Cuando no sea posible, explicar por qué no es posible.
    \begin{enumerate}[ topsep=5pt,itemsep=5pt]
        \item\label{usar-1 a} $\operatorname{dim} \operatorname{Im}(T)=2$ y $\operatorname{dim}\operatorname{Nu}(T)=2$.
        \item\label{usar-1 b} $T$ inyectiva y $T(e_1)=(1,0,0)$, $T(e_2)=(2,1,5)$ y $T(e_3)=(3,-1,0)$.
        \item\label{usar-1 c} $T$ sobreyectiva y $T(e_1)=(1,0,0)$, $T(e_2)=(2,1,5)$ y $T(e_3)=(3,-1,0)$.
        
        \item\label{usar Txyz} $T(e_1)=(1,0,0)$, $T(e_2)=(2,1,5)$ y $T(e_3)=(3,-1,0)$.
        
        \item\label{usar-1 e} $e_1\in\operatorname{Im}(T)$ y $(-5,1,1)\in\operatorname{Nu}(T)$.
        
        \item\label{usar-1 f} $\operatorname{dim} \operatorname{Im}(T)=2$.
    \end{enumerate}
        
    \rta 


    \item Decidir si las siguientes afirmaciones son verdaderas o falsas. Justificar.
    \begin{enumerate}
        \item\label{tl-VoF-a} Si $T : \mathbb R^{13} \to \mathbb R^9$ es una transformación lineal, entonces $\dim \operatorname{Nu}(T) \geq  4$.
        \item\label{tl-VoF-b} Sea $T:\mathbb{K}^{6}\longrightarrow\mathbb{K}^2$ un epimorfismo y $W$ un subespacio de $\mathbb{K}^{6}$ con $\dim W=3$. Entonces existe $0\neq w\in W$ tal que $T(w)=0$.
        \item\label{tl-VoF-c} Existe una transformación lineal $T : \mathbb R^2 \to \mathbb R^4$ tal que los vectores $(1, 0, -1, 2)$, $(0, 1, 2,-1,)$ y $(0, 0, 2, 2)$ pertenecen a la imagen de $T$.
    \end{enumerate}
    
    \rta



    \item \label{funcionales} Sea $V$ un espacio vectorial no nulo y $T:V\longrightarrow\mathbb{K}$ probar que $T=0$ ó $T$ es sobreyectiva.
    
    \rta


    \item Sea $V$ un espacio vectorial de dimensión finita y $T:V\longrightarrow V$ una transformación lineal. Probar las siguientes afirmaciones.
        \begin{multicols}{2}
            \begin{enumerate}
                \item $\operatorname{Nu}(T)\subseteq\operatorname{Nu}(T^2)$
                \item\label{dimV impar} $\operatorname{Nu}(T)\neq\operatorname{Im}(T)$ si $\dim(V)$ es impar.
            \end{enumerate}
        \end{multicols}
    
    \rta
        
    \end{enumerate}
%%%=======================
\end{document}