% PDFLaTeX
\documentclass[a4paper,12pt,twoside,spanish,reqno]{amsbook}
%%%---------------------------------------------------
\usepackage[math]{kurier}

\usepackage{etex}
\usepackage{t1enc}
\usepackage{latexsym}
\usepackage[utf8]{inputenc}
\usepackage{verbatim}
\usepackage{multicol}
\usepackage{amsgen,amsmath,amstext,amsbsy,amsopn,amsfonts,amssymb}
\usepackage{amsthm}
\usepackage{calc}         % From LaTeX distribution
\usepackage{graphicx}     % From LaTeX distribution
\usepackage{ifthen}
\input{random.tex}   
\usepackage{tikz}
\usetikzlibrary{arrows}
\usetikzlibrary{matrix}
\usepackage{mathtools}
\usepackage{stackrel}
\usepackage{enumitem}
\usepackage{tkz-graph}

\usepackage{enumitem} 
\usepackage[compatibility=false]{caption} % para usar subcaption
\usepackage{subcaption} % para poner varias imagenes juntas
\usetikzlibrary{arrows.meta}
\usepackage{hyperref}
\hypersetup{ 
    colorlinks=true,
    linkcolor=blue,
    filecolor=magenta,      
    urlcolor=cyan,
}
\usepackage{hypcap}
\numberwithin{equation}{section}
% http://www.texnia.com/archive/enumitem.pdf (para las labels de los enumerate)
\renewcommand\labelitemi{$\circ$}
\setlist[enumerate, 1]{label={(\arabic*)}}
\setlist[enumerate, 2]{label=\emph{\alph*)}}


\newcommand{\rta}{\vskip.2cm\noindent\textsc{Solución: }} 

\newcommand{\img}{\operatorname{Im}}
\newcommand{\nuc}{\operatorname{Nu}}
\newcommand\im{\operatorname{Im}}
\renewcommand\nu{\operatorname{Nu}}
\newcommand{\la}{\langle}
\newcommand{\ra}{\rangle}
\renewcommand{\t}{{\operatorname{t}}}
\renewcommand{\sin}{{\,\operatorname{sen}}}
\newcommand{\Q}{\mathbb Q}
\newcommand{\R}{\mathbb R}
\newcommand{\C}{\mathbb C}
\newcommand{\K}{\mathbb K}
\newcommand{\F}{\mathbb F}
\newcommand{\Z}{\mathbb Z}
\newenvironment{amatrix}[1]{%
  \left[\begin{array}{@{}*{#1}{c}|c@{}}
}{%
  \end{array}\right]
}
\renewcommand{\qed}{\hfill$\square$\vskip.6cm}
\renewcommand{\theequation}{\arabic{chapter}.\arabic{equation}} 
%%% FORMATOS %%%%%%%%%%%%%%%%%%%%%%%%%%%%%%%%%%%%%%%%%%%%%%%%%%%%%%%%%%%%%%%%%%%%%
\tolerance=10000
\renewcommand{\baselinestretch}{1.2}
\usepackage[a4paper, top=3cm, left=3cm, right=2cm, bottom=2.5cm]{geometry}
\usepackage{setspace}
%\setlength{\parindent}{0,7cm}% tamaño de sangria.
\setlength{\parskip}{0,4cm} % separación entre parrafos.
%\renewcommand{\baselinestretch}{0.90}% separacion del interlineado
\renewcommand{\chaptername}{Práctico}
%%%%%%%%%%%%%%%%%%%%%%%%%%%%%%%%%%%%%%%%%%%%%%%%%%%%%%%%%%%%%%%%%%%%%%%%%%%%%%%%%%%
%\end{comment}
%%% FIN FORMATOS  %%%%%%%%%%%%%%%%%%%%%%%%%%%%%%%%%%%%%%%%%%%%%%%%%%%%%%%%%%%%%%%%%

\begin{document}
    \baselineskip=0.55truecm %original

%%% CAP1 =================
    \chapter{Soluciones\\Álgebra  II -- Año 2024/1 -- FAMAF}\label{practico-1}
    


%============================================================
\subsection*{Vectores y producto escalar}
%============================================================

\begin{enumerate}[topsep=6pt, itemsep=.4cm]


    \item Dados $v = (-1, 2, 0)$, $w = (2,-3,-1)$ y $u = (1,-1,1)$, calcular:
    \begin{enumerate}
    %     \item $v + w$,
    %     \item $v - w$,
        \item\label{comb-lin-a} $2v + 3w -5u$,
        \item\label{comb-lin-b} $5(v+w)$,
        \item\label{comb-lin-c} $5v + 5w$ (y verificar que es igual al vector de arriba).
    \end{enumerate}

\rta

\noindent  \ref{comb-lin-a} $2v + 3w -5u = 2 \cdot (-1, 2, 0) + 3 \cdot (2,-3,-1) - 5 \cdot (1,-1,1) $

$ = (-2, 4, 0) + (6,-9,-3) + (-5,5,-5) = \boxed{(-1,0,-8)}$

\noindent\ref{comb-lin-b} $5(v+w) = 5 \cdot ( (-1, 2, 0) + (2,-3,-1) ) = 5 \cdot (1,-1,-1) = \boxed{(5,-5,-5)} $


\noindent\ref{comb-lin-c} $5v + 5w = 5 \cdot (-1, 2, 0) + 5 \cdot (2,-3,-1) = (-5, 10, 0) + (10,-15,-5)$

$= \boxed{(5,-5,-5)}$


\qed


\item Calcular los siguientes productos escalares. %\langle v , w  \rangle
\begin{enumerate}
  \item\label{prod-esc-a} $\langle (-1, 2, 0) ,(2,-3,-1) \rangle$,
%   \item  $\langle (2,4,-3,-1),(1,-1,2, 1) \rangle$,
  \item\label{prod-esc-b}  $\langle (4,-1),(-1,2) \rangle$.
\end{enumerate}

\rta


\noindent\ref{prod-esc-a} $\langle (-1, 2, -0) ,(2,-3,-1) \rangle = (-1) \cdot 2 + 2 \cdot (-3) + 0 \cdot (-1) = -2 + (-6) + 0 = \boxed{-8}$ 
%   \item  $\langle (2,4,-3,-1),(1,-1,2, 1) \rangle$,}

\noindent\ref{prod-esc-b}  $\langle (4,-1),(-1,2) \rangle = 4 \cdot (-1) + (-1) \cdot 2 = -4 + (-2) = \boxed{-6}$

\qed

\item Dados $v = (-1, 2,0)$, $w = (2,-3,-1)$  y $u = (1,-1,1)$, verificar que:
\begin{equation*}
    \langle 2v + 3w , -u   \rangle = -2\langle v ,u \rangle -3 \langle w , u  \rangle
\end{equation*}

\rta Calculamos ambos miembros por separado.


Miembro izquierdo: $\langle 2v + 3w , -u   \rangle = \langle 2 \cdot (-1, 2,0) + 3 \cdot (2,-3,-1) , -  (1,-1,1) \rangle $

$= \langle (-2, 4,0) + (6,-9,-3) , (-1,1,-1) \rangle = \langle (4,-5,-3) , (-1,1,-1) \rangle $

$= 4 \cdot (-1) + (-5) \cdot 1 + (-3) \cdot (-1) = -4 + (-5) + 3 = \boxed{-6}$

Miembro derecho: $-2\langle v ,u \rangle -3 \langle w , u  \rangle = -2 \langle (-1, 2,0) ,(1,-1,1) \rangle -3 \langle (2,-3,-1) , (1,-1,1)  \rangle $

$= -2 \cdot ( -1 \cdot 1 + 2 \cdot (-1) + 0 \cdot 1 ) - 3 \cdot ( 2 \cdot 1 + (-3) \cdot (-1) + (-1) \cdot 1 ) $

$= -2 \cdot ( -1 + (-2) + 0) - 3 \cdot ( 2 + 3 + (-1)) = -2 \cdot (-3) - 3 \cdot  4 = 6 - 12 = \boxed{-6} $

\qed

\item Probar  que 
\begin{enumerate}
    \item\label{otrogonales-a} $(2,3,-1)$ y $(1, -2, -4)$ son ortogonales.
    \item\label{otrogonales-b} $(2,-1)$ y $(1,2)$ son ortogonales. Dibujar en el plano. 
\end{enumerate}

\rta Calculamos su producto interno para ver si es nulo.


\noindent\ref{otrogonales-a} $ \langle (2,3,-1) , (1, -2, -4) \rangle = 2 \cdot 1 + 3 \cdot (-2) + (-1) \cdot (-4) = 2 + (-6) + 4 = \boxed{0}$. Por lo tanto, son vectores ortogonales. 
\vskip .3cm
\noindent\ref{otrogonales-b}$ \langle (2,-1) , (1,2) \rangle = 2 \cdot 1 + (-1) \cdot 2 = 2 - 2 = \boxed{0}$. Por lo tanto, son vectores ortogonales y  su gráfica es:
\begin{center}
    \begin{tikzpicture}
        \draw[->] (-1.0,0) -- (3.0,0) node[right] {}; % eje x
        \draw[->] (0,-2) -- (0,3) node[above] {}; % eje y
%        \draw[fill] (-1,5) circle [radius=0.05];
        \node [above right] at (1,2) {$(1,2)$};
        \node [below right] at (2,-1) {$(2,-1)$};
        \node [above right] at (1,3pt) {$1$};
        \node [above right] at (2,3pt) {$2$};
        \node [left] at (-3pt,2) {$2$};
        \node [left] at (-3pt,-1) {$-1$};
        \draw (1,-3pt) -- (1,3pt);
        \draw (2,-3pt) -- (2,3pt);
        \draw (-3pt, 2) -- (3pt, 2);
        \draw (-3pt, -1) -- (3pt, -1);
        \draw [dashed] (1,0) -- (1,2);
        \draw [dashed] (0,2) -- (1,2);
        \draw [dashed] (2,0) -- (2,-1);
        \draw [dashed] (0,-1) -- (2,-1);
        \draw[-{Stealth[length=5pt,width=3pt]}] (0,0) -- (1,2) node[right] {};
        \draw[-{Stealth[length=5pt,width=3pt]}] (0,0) -- (2,-1) node[right] {};
        \draw (0.15,0.3) -- (0.45, 0.15) -- (0.3,-0.15);
    \end{tikzpicture}
\end{center}
    
\qed

\item Encontrar 
\begin{enumerate}
    \item\label{no-nulo-ortogonal-a} un vector no nulo ortogonal  a $(3,-4)$,
    \item\label{no-nulo-ortogonal-b} un vector no nulo ortogonal a $(2,-1,4)$,
    \item\label{no-nulo-ortogonal-c} un vector no nulo ortogonal a $(2,-1,4)$ y $(0,1,-1)$,
\end{enumerate}

\rta 


\noindent\ref{no-nulo-ortogonal-a} $(4,3)$ es un vector no nulo ortogonal  a $(3,-4)$, pues:
$$\langle (3,-4),(4,3) \rangle = 3 \cdot 4 + (-4) \cdot 3 = 12 - 12 = \boxed{0}.$$

\noindent\ref{no-nulo-ortogonal-b} $(1,2,0)$ es un vector no nulo ortogonal a $(2,-1,4)$, pues:
$$\langle (2,-1,4),(1,2,0) \rangle = 2 \cdot 1 + (-1) \cdot 2 + 4 \cdot 0 = 2 - 2 + 0 = \boxed{0}.$$    
    
\noindent\ref{no-nulo-ortogonal-c} Primero notar que cualquier vector de la pinta $(a,b,b)$ será ortogonal a $(0,1,-1)$, pues:
$$\langle (0,1,-1),(a,b,b)\rangle = 0 \cdot a + 1 \cdot b + (-1) \cdot b = 0 + b - b = \boxed{0}.$$
Si ahora multiplicamos nuestro candidato $(a,b,b)$ con $(2,-1,4)$ tenemos:
$$\langle (2,-1,4) , (a,b,b) \rangle = 2 \cdot a + (-1) \cdot b + 4 \cdot b = \boxed{2a + 3 b}.$$

Luego, si elegimos por ejemplo $a=-3$ y $b=2$ vamos a tener a nuestro candidato ortogonal a ambos vectores. Es decir, $(-3,2,2)$ cumple lo requerido.


\qed

\item Encontrar la longitud de los vectores.
\begin{multicols}{3}
    \begin{enumerate}
        \item\label{long-a} $(2,3)$,
        \item\label{long-b} $(t,t^2)$,
        \item\label{long-c} $(\cos\phi,\operatorname{sen}\phi)$.
    \end{enumerate}
\end{multicols}


\rta 

\noindent\ref{long-a} $||(2,3)|| = \sqrt{2^2 + 3^2} = \sqrt{4+9} = \boxed{\sqrt{13}}$

\noindent\ref{long-b} $||(t,t^2)|| = \sqrt{t^2 + (t^2)^2} = \sqrt{t^2+t^4} = \boxed{|t|\sqrt{1+t^2}}$

\noindent\ref{long-c} $||(\cos\phi,\operatorname{sen}\phi)|| = \sqrt{\cos^2\phi + \operatorname{sen}^2\phi} = \sqrt{1} = \boxed{1}$


\qed

\item Calcular $\langle v , w  \rangle$ y el ángulo entre $v$ y $w$  para los siguientes vectores.
\begin{multicols}{2}
    \begin{enumerate}
        \item\label{angulo-a} $v=(2,2)$, $w=(1,0)$,
        \item\label{angulo-b} $v=(-5,3,1)$, $w=(2,-4,-7)$.
    \end{enumerate}
\end{multicols}

\rta Para encontrar el ángulo se deben calcular además las normas de los vectores:

\noindent\ref{angulo-a} $\langle v , w  \rangle = \langle (2,2) , (1,0)  \rangle = 2b\cdot 1 + 1 \cdot 0 = 2 + 0 = \boxed{2}$

$||v||=||(2,2)|| = \sqrt{2^2 + 2^2} = \sqrt{4+4} = \sqrt{8} = 2 \sqrt{2}$

$||w||=||(1,0)|| = \sqrt{1^2 + 0 ^2} = \sqrt{1+0} = \sqrt{1} = 1 $

$\theta = \cos^{-1} \left( \dfrac{\langle v,w \rangle}{||v|| \; ||w||} \right) = \cos^{-1} \left( \dfrac{ 2 }{2\sqrt{2} \cdot 1 } \right) = \cos^{-1} \left( \dfrac{ 1 }{\sqrt{2}} \right) = \boxed{45^\circ}$

\noindent\ref{angulo-b} $\langle v , w  \rangle = \langle (-5,3,1) , (2,-4,-7)  \rangle = -5 \cdot 2 + 3 \cdot (-4) + 1 \cdot (-7) = -10 -12-7 = \boxed{-29}$

$||v||=||(-5,3,1)|| = \sqrt{(-5)^2 + 3^2 + 1^2} = \sqrt{25+9+1} = \sqrt{35}$

$||w||=||(2,-4,-7)|| = \sqrt{2^2 + (-4)^2 + (-7)^2} = \sqrt{4+16+49} = \sqrt{69}$

$\theta = \cos^{-1} \left( \dfrac{\langle v,w \rangle}{||v|| \; ||w||} \right) = \cos^{-1} \left( \dfrac{ -29 }{ \sqrt{35} \sqrt{69} } \right) = \boxed{126^\circ 9'55.57''}$

\qed

\item Recordar los vectores $e_1$, $e_2$ y $e_3$ dados en la página 12 del apunte. Sea $v=(x_1,x_2,x_3)\in\mathbb{R}^3$.  Verificar que 
$$v=x_1e_1+x_2e_2+x_3e_3=\langle v,e_1\rangle e_1+\langle v,e_2\rangle e_2+\langle v,e_3\rangle e_3.$$

\rta Podemos empezar desde el miembro de la derecha, pasar por el del medio y llegar al de la izquierda aplicando las definiciones y propiedades conocidas:

$ \langle v,e_1\rangle e_1+\langle v,e_2\rangle e_2+\langle v,e_3\rangle e_3 = $

$= \langle (x_1,x_2,x_3),(1,0,0)\rangle e_1+\langle (x_1,x_2,x_3),(0,1,0)\rangle e_2+\langle (x_1,x_2,x_3),(0,0,1)\rangle e_3 $

$= (x_1 \cdot 1 + x_2 \cdot 0 + x_3 \cdot 0) e_1+ (x_1 \cdot 0 + x_2 \cdot 1 + x_3 \cdot 0) e_2+ (x_1 \cdot 0 + x_2 \cdot 0 + x_3 \cdot 1) e_3 $

$= (x_1 + 0 + 0) e_1+ (0 + x_2 + 0) e_2+ (0 + 0 + x_3) e_3 = \boxed{x_1 e_1+ x_2 e_2 + x_3 e_3}$

$x_1 e_1+ x_2 e_2 + x_3 e_3 = x_1 (1,0,0) + x_2 (0,1,0) + x_3 (0,0,1) = $

$= (x_1 \cdot 1,x_1 \cdot 0,x_1 \cdot 0) + (x_2 \cdot 0, x_2 \cdot 1 , x_2 \cdot 0 ) + (x_3 \cdot 0 , x_3 \cdot 0 , x_3 \cdot 1) $

$= (x_1,0,0)+(0,x_2,0)+(0,0,x_3) = (x_1+0+0,0+x_2+0,0+0+x_3) = (x_1,x_2,x_3) = \boxed{v}$

\qed

\item Probar, usando sólo las propiedades \textbf{P1}, \textbf{P2}, y \textbf{P3} del producto escalar, que dados $v, w, u \in \mathbb R^n$ y $\lambda_1, \lambda_2 \in \mathbb R$, 
\begin{enumerate}
    \item se cumple:
    \begin{equation*}
    \langle \lambda_1 v + \lambda_2 w , u  \rangle =  \lambda_1\langle v , u  \rangle +   \lambda_2\langle w , u  \rangle.
    \end{equation*}
    \item Si $\langle v , w  \rangle =0$, es decir si $v$ y $w$ son ortogonales,  entonces
    \begin{equation*}
        \langle \lambda_1 v + \lambda_2 w ,  \lambda_1 v + \lambda_2 w   \rangle =
        \lambda_1^2 \langle  v ,  v  \rangle + \lambda_2^2 \langle w,w  \rangle.
    \end{equation*}
\end{enumerate}

\rta 

\begin{enumerate}
\item $\langle \lambda_1 v + \lambda_2 w , u  \rangle \overset{\textbf{P2}}{=} \langle u , \lambda_1 v \rangle + \langle u , \lambda_2 w \rangle \overset{\textbf{P3}}{=} \lambda_1 \langle u,v \rangle + \lambda_2 \langle u,w \rangle \overset{\textbf{P1}}{=} \lambda_1 \langle v,u \rangle + \lambda_2 \langle w,u \rangle$

\item $ \langle \lambda_1 v + \lambda_2 w, \lambda_1 v + \lambda_2 w \rangle \overset{\textbf{P2}}{=} \langle \lambda_1 v + \lambda_2 w, \lambda_1 v \rangle + \langle \lambda_1 v + \lambda_2 w, \lambda_2 w \rangle \overset{\textbf{P2}}{=} $

$ \overset{\textbf{P2}}{=} \langle \lambda_1 v , \lambda_1 v \rangle + \langle \lambda_1 v, \lambda_2 w \rangle + \langle \lambda_2 w, \lambda_1 v \rangle + \langle \lambda_2 w, \lambda_2 w \rangle \overset{\textbf{P3}}{=} $

$ \overset{\textbf{P3}}{=} \lambda_1^2 \langle v , v \rangle + \lambda_1 \lambda_2 \langle v, w \rangle + \lambda_2 \lambda_1 \langle w, v \rangle + \lambda_2^2 \langle w, w \rangle \overset{\textbf{HIP}}{=} \lambda_1^2 \langle v , v \rangle + \lambda_2^2 \langle w, w \rangle $

En el último paso se utilizó la hipótesis $\langle v , w  \rangle =0$.

\end{enumerate}

\qed

\item Dados $v, w\in \mathbb R^n$, probar que si  $\langle v , w  \rangle =0$, es decir si $v$ y $w$ son ortogonales,  entonces
    \begin{equation*}
    ||v + w||^2 = ||v||^2 + ||w||^2.
    \end{equation*}
    ¿Cuál es el nombre con que se conoce este resultado en $\mathbb R^2$?
    
\rta Vamos a usar la definición de norma y el inciso b) del ejercicio anterior, tomando $\lambda_1 = \lambda_2 = 1$:
$$||v + w||^2 \overset{def}{=} \langle v+w,v+w \rangle \overset{9.b)}{=} \langle v,v \rangle + \langle w,w \rangle \overset{def}{=} ||v||^2 + ||w||^2.$$


En $\mathbb R^2$ esta igualdad es el \emph{Teorema de Pitágoras}.

\qed
 
\item\label{Schwarz} $\textcircled{a}$ Sean $v,w\in \mathbb R^2$, probar usando  solo la definición explícita del producto escalar en $\mathbb R^2$ que 
\begin{equation*}
    |\langle v , w  \rangle| \le ||v||\,||w|| \qquad \text{(Desigualdad de Schwarz).}
\end{equation*}

\rta Vamos a escribir $v = (v_1 , v_2)$ y $w=(w_1,w_2) $. Veamos la pinta del cuadrado del lado izquierdo:

\begin{equation}\label{cuad_izq}
\langle v,w \rangle^2 = \langle (v_1,v_2) , (w_1,w_2) \rangle^2 =  (v_1 w_1 + v_2 w_2 )^2
\end{equation}
Ahora comenzamos por el cuadrado del lado derecho con el objetivo de llegar a (\ref{cuad_izq}):
\begin{equation*}
||v||^2||w||^2 = (v_1^2 + v_2^2)(w_1^2 + w_2^2) = (v_1 w_1)^2 + (v_1 w_2)^2 + (v_2 w_1)^2 + (v_2 w_2)^2.
\end{equation*}
Mirando el primer y último término tenemos que si completamos ese cuadrado obtendríamos (\ref{cuad_izq}). Sumamos y restamos $2(v_1w_1)(v_2w_2)$ y agrupamos:
\begin{align*}
    ||v||^2||w||^2 &= (v_1 w_1)^2 + (v_1 w_2)^2 + (v_2 w_1)^2 + (v_2 w_2)^2+ \\
                & \qquad\qquad\qquad\qquad+ 2(v_1w_1)(v_2w_2) - 2(v_1w_1)(v_2w_2) = \\
                = [(v_1 &w_1)^2 + 2(v_1w_1)(v_2w_2) + (v_2 w_2)^2 ] + [(v_2 w_1)^2 - 2v_1w_1v_2w_2 + (v_1 w_2)^2 ]
\end{align*}
El segundo grupo de términos también forma un cuadrado perfecto. Escribimos ambos como cuadrados y acotamos:
\begin{equation*}
    ||v||^2||w||^2 = \underset{= \langle v,w \rangle^2}{ \underbrace{ (v_1w_1 + v_2w_2)^2} } + \underset{\geq 0}{ \underbrace{ (v_2w_1 - v_1w_2)^2}} \geq \langle v,w \rangle^2.
\end{equation*}
  \qed

\end{enumerate}


\begin{comment}
%============================================================
\subsection*{Rectas y planos}
%============================================================

\

\begin{enumerate}[resume,topsep=6pt, itemsep=.4cm]
 
\item En  cada uno de los siguientes casos determinar si los
vectores  $\overrightarrow{vw}$ y $\overrightarrow{xy}$ son
equivalentes y/o paralelos.
\begin{enumerate}
\item   $v=(1,-1)$,  $w=(4,3)$, $x=(-1,5)$, $y=(5,2)$. 
\item   $v=(1,-1,5)$,  $w=(-2,3,-4)$,  $x=(3,1,1)$,  $y=(-3,9,-17)$.
\end{enumerate}


\rta Calculamos las diferencias correspondientes y las analizamos:
\begin{enumerate}
\item $w - v = (4,3) - (1,-1) = (4-1,3-(-1)) = (3,4)$

$ y-x = (5,2) - (-1,5) = (5-(-1),2-5) = (6,-3)$

No son equivalentes ni paralelos.

\item $w-v = (-2,3,-4) - (1,-1,5) = (-2-1,3-(-1),-4-5) = (-3,4,-9)$

$ y-x = (-3,9,-17) - (3,1,1) = (-3-3,9-1,-17-1) = (-6,8,-18)$.

No son equivalente pero si paralelos. Tomando $\lambda=2$ se tiene que $y-x = \lambda (w-v)$.

\end{enumerate}

\qed

\item Sea $R_1$ la recta que pasa por $p_1=(2,0)$ y es ortogonal a $(1,3)$.
\begin{enumerate}
 \item Dar la descripción paramétrica e implícita de $R_1$.
 \item Graficar en el plano a $R_1$.
 \item Dar un punto $p$ por el que pase $R_1$ distinto a $p_1$.
 \item Verificar si $p+p_1$ y $-p$ pertenecen a $R_1$
\end{enumerate}

\rta

\begin{enumerate}
 \item Para la descripción paramétrica necesitamos un vector paralelo a $R_1$, es decir, ortogonal a $(1,3)$. Un vector así puede ser el $(3,-1)$, con el que tenemos:
 
Descripción paramétrica: $\boxed {R_1 = \{ (2,0) + t(3,-1) \; | \; t \in \mathbb{R} \} }$

Para la descripción implícita simplemente reemplazamos todos los datos dados en la ecuación $ax + by = \langle (x_0,y_0) , (a,b) \rangle $ y tenemos:

Descripción implícita: $\boxed {R_1 = \{ (x,y) \; | \; x+3y = 2 \} }$

\item ver figura \ref{ej13by14}

\setcounter{enumii}{2}
 \item Para dar un punto sobre la recta conviene usar la descripción paramétrica. En este caso debe ser distinto a $p_1$, con lo que cualquier valor de $t \neq 0$ va a servir. Si tomamos por ejemplo $t=-1$ vamos a tener $\boxed {p = (-1,1) }$.
 \item Para verificar si un punto pertenece, conviene usar la descripción implícita. Calculamos cada punto y reemplazamos en la ecuación:

\begin{equation*}
\begin{array}{ll|ll}
p+p_1 = (-1,1) + (2,0) = (1,1) &&& -p = (1,-1) \\
(1) + 3 \cdot (1) = 4 \neq 2    &&& (1) + 3 \cdot (-1) = -2 \neq 2 \\
\therefore  p+p_1 \notin R_1   &&& \therefore  -p \notin R_1
\end{array}
\end{equation*}

\end{enumerate}

\qed

\item Repetir el ejercicio anterior con las siguientes rectas.
\begin{enumerate}
    \item
    $R_2$: recta que pasa por $p_2=(0,0)$ y es ortogonal a $(1,3)$.
    \item
    $R_3$: recta que pasa por $p_3=(1,0)$ y es paralela a $R_1$.
%     \item
%     $R_4$: recta que pasa por los puntos $(-1,5,4)$ y $(0,3,-2)$.
\end{enumerate}

\rta Los procedimientos son análogos a los del ejercicio 13. Las gráficas están en la figura \ref{ej13by14}

\begin{enumerate}
    \item 
Descripción paramétrica: $\boxed {R_2 = \{ t(3,-1) \; | \; t \in \mathbb{R} \} }$

Descripción implícita: $\boxed {R_2 = \{ (x,y) \; | \; x+3y = 0 \} }$

Tomando $t=-1$ tenemos $\boxed {p = (-3,1) }$.

\begin{equation*}
\begin{array}{ll|ll}
p+p_2 = (-3,1) + (0,0) = (-3,1)    &&& -p = (3,-1) \\
(-3) + 3 \cdot (1) = -3 + 3 = 0    &&& (3) + 3 \cdot (-1) = 3 -3 = 0 \\
\therefore  p+p_2 \in R_2           &&& \therefore  -p \in R_2
\end{array}
\end{equation*}

    \item
    
Descripción paramétrica: $\boxed {R_3 = \{ (1,0) + t(3,-1) \; | \; t \in \mathbb{R} \} }$


Descripción implícita: $\boxed {R_3 = \{ (x,y) \; | \; x+3y = 1 \} }$

Tomando $t=-1$ tenemos $\boxed {p = (-2,1) }$.
\begin{equation*}
\begin{array}{ll|ll}
p+p_3 = (-2,1) + (1,0) = (-1,1)    &&& -p = (2,-1) \\
(-1) + 3 \cdot (1) = 2 \neq 1        &&& (2) + 3 \cdot (-1) = -1 \neq 1 \\
\therefore  p+p_3 \notin R_3          &&& \therefore  -p \notin R_3
\end{array}
\end{equation*}


\end{enumerate}

\begin{figure}[!h]
\begin{subfigure}{.3\textwidth} 
    \begin{tikzpicture}[scale=0.8]
        \draw[->] (-1.0,0) -- (4.0,0) node[right] {}; % eje x
        \draw[->] (0,-2) -- (0,2) node[above] {}; % eje y
        \draw[fill] (2,0) circle [radius=0.05];
        \node [above] at (2,3pt) {$p_1$};
        \node [below] at (2,-3pt) {$2$};
        \node [above right] at (3pt,2/3) {$\frac{2}{3}$};
        \node [above] at (-1,1) {$R_1$};
        \draw (2,-3pt) -- (2,3pt);
        \draw (-3pt,2/3) -- (3pt,2/3);
        \draw (-1,1) -- (4, -2/3);
    \end{tikzpicture}
\caption{Ejercicio 13.b}
\end{subfigure}
\begin{subfigure}{.3\textwidth}
    \begin{tikzpicture}[scale=0.8]
        \draw[->] (-1.0,0) -- (4.0,0) node[right] {}; % eje x
        \draw[->] (0,-2) -- (0,2) node[above] {}; % eje y
        \draw[fill] (0,0) circle [radius=0.05];
        \draw[fill] (3,-1) circle [radius=0.05];
        \node [above right] at (0,3pt) {$p_2$};
        \node [above] at (3,3pt) {$3$};
        \node [left] at (-3pt,-1) {$-1$};
        \node [above] at (-1,1/3) {$R_2$};
        \draw[dashed] (0,-1) -- (3,-1);
        \draw[dashed] (3,0) -- (3,-1);
        \draw (3,-3pt) -- (3,3pt);
        \draw (-3pt,-1) -- (3pt,-1);
        \draw (-1,1/3) -- (4, -4/3);
    \end{tikzpicture}
\caption{Ejercicio 14.a}
\end{subfigure}
\begin{subfigure}{.3\textwidth}
    \begin{tikzpicture}[scale=0.8]
        \draw[->] (-1.0,0) -- (4.0,0) node[right] {}; % eje x
        \draw[->] (0,-2) -- (0,2) node[above] {}; % eje y
        \draw[fill] (1,0) circle [radius=0.05];
        \node [above right] at (1,3pt) {$p_3$};
        \node [below] at (1,-13pt) {$1$};
        \node [above right] at (3pt,1/3) {$\frac{1}{3}$};
        \node [above] at (-1,2/3) {$R_3$};
        \draw (1,-3pt) -- (1,3pt);
        \draw (-3pt,1/3) -- (3pt,1/3);
        \draw (-1,2/3) -- (4,-1);
    \end{tikzpicture}
\caption{Ejercicio 14.b}
\end{subfigure} \hfill
\caption{}\label{ej13by14}
\end{figure}

\qed

\item Calcular, numérica y graficamente, las intersecciones $R_1\cap R_2$ y $R_1\cap R_3$. 

\rta Para el cálculo numérico, notar que las ecuaciones de las tres rectas son de la forma $x+3y = c$ donde $c$ vale 2, 0 y 1 para $R_1$, $R_2$ y $R_3$ respectivamente. Así tendremos por ejemplo que para calcular la intersección $R_1\cap R_2$ tendremos que resolver el sistema:

\begin{equation*}
\left\{\begin{array}{l}
x+3y=2 \\
x+3y=0
\end{array} \right.
\end{equation*}

Este sistema no tiene solución, pues para cualquier valores de $x$ e $y$ que elijamos, no puede suceder que al hacer la cuenta $x+3y$ obtengamos simultáneamente el resultado 2 y el resultado 0. El caso $R_1\cap R_3$ es análogo.  

Para la determinación gráfica, se pueden observar los gráficos de la figura \ref{ej13by14} y notar que ambas parejas son paralelas, y por lo tanto no tienen intersección.

En conclusión, tenemos $\boxed{ R_1 \cap R_2 = R_1 \cap R_3 = \emptyset }$

\qed

\item Sea $v_0=(2,-1,1)$.
\begin{enumerate}
    \item Describir param{é}tricamente el conjunto
    $P_1=\{w\in\mathbb{ R}^3:\langle v_0 , w  \rangle=0\}$.
    \item Describir param{é}tricamente el conjunto
    $P_2=\{w\in\mathbb{ R}^3:\langle v_0 , w  \rangle=1\}$.
    \item ?`Qué relación hay entre $P_1$ y $P_2$?
\end{enumerate}


\rta

\begin{enumerate}
    \item Debemos despejar la ecuación implícita y reemplazarla en el vector:

$ (x,y,z) \in P_1 \iff \langle (2,-1,1) , (x,y,z) \rangle = 0$
    
$ (x,y,z) \in P_1 \iff 2x-y+z = 0$    

$ (x,y,z) \in P_1 \iff 2x+z = y$    

$ (x,y,z) \in P_1 \iff (x,y,z) = (x,2x+z,z) = x (1,2,0) + z (0,1,1)$    

$ \therefore \boxed{ P_1 = \{ s (1,2,0) + t (0,1,1) \; | \; s,t, \in \mathbb{R} \} }$
    
    \item Análogo al item anterior:

$ (x,y,z) \in P_2 \iff \langle (2,-1,1) , (x,y,z) \rangle = 1$
    
$ (x,y,z) \in P_2 \iff 2x-y+z = 1$    

$ (x,y,z) \in P_2 \iff 2x+z-1 = y$    

$ (x,y,z) \in P_2 \iff (x,y,z) = (x,2x+z-1,z) = (0,-1,0) + x (1,2,0) + z (0,1,1)$    

$ \therefore \boxed{ P_2 = \{ (0,-1,0) + s (1,2,0) + t (0,1,1) \; | \; s,t, \in \mathbb{R} \} }$

    \item Los planos $P_1$ y $P_2$ son paralelos.
\end{enumerate}

\qed

\item\label{ej-planos} Escribir la ecuación paramétrica  y la ecuación normal de los siguientes planos.
\begin{enumerate}
    \item $\pi_1$: el plano que pasa por $(0,0,0)$, $(1,1,0)$, $(1,-2,0)$.
    \item $\pi_2$: el plano que pasa por $(1,2,-2)$ y es perpendicular a la
    recta que pasa por $(2,1,-1)$, $(3,-2,1)$.
    \item\label{ej-planos-c}  $\pi_3=\{w\in\mathbb{R}^3: w=s(1,2,0)+t(2,0,1)+(1,0,0);\,s,t\in \mathbb R\}$.
\end{enumerate}

\rta

\begin{enumerate}
    \item Llamemos $p_0=(0,0,0)$, $p_1=(1,1,0)$ y $p_2=(1,-2,0)$ a los puntos involucrados. Como $p_0$ es el origen y $p_2$ no es un múltiplo de $p_1$, tenemos que los puntos no son colineales. Luego para la descripción paramétrica basta con elegir uno de ellos y dos parejas distintas cualquiera. Así, por ejemplo podríamos escribir:
    
$\pi_1 = \{ p_0 + s \; \overrightarrow{p_0 p_1} + t \; \overrightarrow{p_0 p_2} \; | \; s,t \in \mathbb{R} \}  = \boxed{ \{ s (1,1,0) + t (1,-2,0) \; | \; s,t \in \mathbb{R} \} }$

Notar que cualquier otra elección para el primer punto y las dos parejas da lugar a parametrizaciones diferentes, pero equivalentes, de $\pi_1$.

Para la ecuación normal vamos a necesitar un vector que sea ortogonal a ambas direcciones, $\overrightarrow{p_0 p_1}$ y $\overrightarrow{p_0 p_2}$. A simple vista puede verse que un vector que cumple eso es $e_3 = (0,0,1)$. Luego reemplazamos eso en la ecuación normal $\langle v,e_3 \rangle = \langle p_0, e_3 \rangle$. Notar que podríamos haber elegido cualquier punto en $\pi_1$ en lugar de $p_0$, y todos deberían dar el mismo resultado. La ecuación normal sería entonces:

$\pi_1 = \boxed{ \{ w \in \mathbb{R}^3 \; | \; \langle w,e_3 \rangle = 0  \} }$

    \item Llamemos $p_0 = (1,2,-2)$, $p_1 = (2,1,-1)$ y $p_2=(3,-2,1)$. En este caso conviene empezar con la ecuación normal pues contamos con una dirección perpendicular al plano: $\overrightarrow{p_1 p_2} = p_2 - p_1 = (1,-3,2)$. Reemplazamos en la ecuación normal y tenemos:
    
$\pi_2 = \{ w \in \mathbb{R}^3 \; | \; \langle w, \overrightarrow{p_1 p_2} \rangle =  \langle p_0, \overrightarrow{p_1 p_2} \rangle \} = \boxed{ \{ w \in \mathbb{R}^3 \; | \; \langle w, \overrightarrow{p_1 p_2} \rangle =  -9 \} }$

Para encontrar la forma paramétrica se siguen los mismos pasos que en el ejercicio 16.a) y 16.b):

$ (x,y,z) \in \pi_2 \iff \langle (1,-3,2) , (x,y,z) \rangle = -9$

$ (x,y,z) \in \pi_2 \iff x -3y +2z = -9$

$ (x,y,z) \in \pi_2 \iff x = -9 + 3y -2z$

$ (x,y,z) \in \pi_2 \iff (x,y,z) = (-9 + 3y -2z ,y,z) = (-9,0,0) + y (3,1,0) + z (-2,1,0)$

$ \therefore \boxed{ \pi_2 = \{ (-9,0,0) + s (3,1,0) + t (-2,1,0) \; | \; s,t, \in \mathbb{R} \} }$


    \item El plano ya viene dado en forma paramétrica, por lo que sólo resta expresarlo en forma normal. Para ello es necesario encontrar un vector $(x,y,z)$ que sea perpendicular a $(1,2,0)$ y a $(2,0,1)$. Como en este caso no es obvio, podemos plantear ambos productos escalares y despejar:
    
\begin{equation*}
\left\{ \begin{array}{rl}
x + 2 y &= 0 \\
2x+z &= 0
\end{array} \right. \implies
\left\{ \begin{array}{rl}
x &= -2y \\
z &= -2x = -2 (-2y) = 4y
\end{array} \right. \implies
\left\{ \begin{array}{rl}
x &= -2y \\
z &= 4y
\end{array} \right.
\end{equation*}

Es decir que el vector buscado es de la pinta $(-2y,y,4y) = y(-2,1,4)$ o, lo que es lo mismo, cualquier múltplo de $(-2,1,4)$ será perpendicular al plano. La forma normal es entonces:

$\pi_3 = \{ w \in \mathbb{R}^3 \; | \; \langle w, (-2,1,4) \rangle =  \langle (1,0,0), (-2,1-4) \rangle \} = \boxed{  \{ w \in \mathbb{R}^3 \; | \; \langle w, (-2,1,4) \rangle =  -2 }$
    
\end{enumerate}

\qed

\item ¿Cuáles de las siguientes rectas cortan al plano $\pi_3$ del  ejercicio \ref{ej-planos-c}?
Describir la intersecci{ó}n en cada caso.
\begin{align*}
&(a) \ \{w: w=(3,2,1)+t(1,1,1)\}, && (b) \  \{w: w=(1,-1,1)+t(1,2,-1)\}, \\
&(c)\  \{w: w=(-1,0,-1)+t(1,2,-1)\}, && (d) \  \{w: w=(1,-2,1)+t(2,-1,1)\}.
\end{align*}


\rta La manera más directa de chequear si una recta interseca a un plano es con la forma normal del plano. Si la dirección de la recta es perpendicular a la dirección normal del plano, la recta es paralela al plano. Luego, o bien toda la recta está contenida en el plano, o bien la recta y el plano tienen intersección vacía.

Si una recta no es paralela a un plano, lo corta en un único punto. La manera más fácil de hallar ese punto es reemplazar la parametrización de la recta en la ecuación normal y despejar $t$. Luego, reemplazando $t$ en la parametrización de la recta se encuentra el punto.

\begin{enumerate}
\item Como $\langle (1,1,1),(-2,1,4) \rangle  = 3 \neq 0 $, la recta corta al plano $\pi_3$. Encuentro el punto de intersección:

\begin{equation*}
\begin{array}{rl}
-2 (3+t) + (2+t) + 4 (1+t) &= -2 \\
-6 - 2t + 2+t + 4 + 4t  &= -2 \\
3t  &= -2 \implies \boxed{t=-\frac{2}{3}}
\end{array}
\end{equation*}

El punto de intersección es $(3,2,1) - \frac{2}{3} (1,1,1) = \boxed{ \left( \frac{7}{3} , \frac{4}{3} , \frac{1}{3} \right) }$

\item Como $\langle (1,2,-1),(-2,1,4) \rangle  = -4 \neq 0 $, la recta corta al plano $\pi_3$. Encuentro el punto de intersección:

\begin{equation*}
\begin{array}{rl}
-2 (1+t) + (-1+2t) + 4 (1-t) &= -2 \\
-2 -2t -1 +2t +4 -4t  &= -2 \\
3 &= 4t \implies \boxed{t=\frac{3}{4}}
\end{array}
\end{equation*}

El punto de intersección es $(1,-1,1) + \frac{3}{4} (1,2,-1) = \boxed{ \left( \frac{7}{4} , \frac{1}{2} , \frac{1}{4} \right) }$

\item Como $\langle (1,2,-1),(-2,1,4) \rangle = -4 \neq 0 $, la recta corta al plano $\pi_3$. Encuentro el punto de intersección:

\begin{equation*}
\begin{array}{rl}
-2 (-1+t) + (2t) + 4 (-1-t) &= -2 \\
2-2t+2t-4-4t &= -2 \\
-4t &= 0 \implies \boxed{t=0}
\end{array}
\end{equation*}

El punto de intersección es $(-1,0,-1) + 0 \cdot (1,2,-1) = \boxed{ (-1,0,-1) }$

\item Como $\langle (2,-1,1),(-2,1,4) \rangle  = -1 \neq 0 $, la recta corta al plano $\pi_3$. Encuentro el punto de intersección:

\begin{equation*}
\begin{array}{rl}
-2 (1+2t) + (-2-t) + 4 (1+t) &= -2 \\
-2-4t-2-t+4 + 4t &= -2 \\
-t &= -2 \implies \boxed{t=2}
\end{array}
\end{equation*}

El punto de intersección es $(1,-2,1) + 2 (2,-1,1) = \boxed{ ( 5,-4,3 ) }$

\end{enumerate}

\qed

\item\label{rectas como subespacio} Sea $L=\{(x,y)\in\mathbb{R}^2 : ax+by=c\}$ una recta en $\mathbb{R}^2$. Sean $p$ y $q$ dos puntos por los que pasa $L$.
\begin{enumerate}
 \item ?`Para qué valores de $c$ puede asegurar que $(0,0)\in L$?
 \item ?`Para qué valores de $c$ puede asegurar que $\lambda q\in L$? donde $\lambda\in\mathbb{R}$.
 \item ?`Para qué valores de $c$ puede asegurar que $p+q\in L$?
\end{enumerate}

\rta 

\begin{enumerate}
 \item Si $(0,0) \in L$, entonces esos valores de $x$ e $y$ deben verificar la ecuación normal de la recta. Es decir, debe suceder $c = ax + by = a\cdot 0 + b \cdot 0 = 0$. Con lo cual debe ser $c=0$ y por lo tanto es el único valor de $c$ con esta propiedad.

 \item Llamemos $q=(x_q , y_q)$. Como $q \in L$, sabemos que se cumple 
\begin{equation}\label{q_in_L}
a x_q + b y_q = c
\end{equation} 

Ahora supongamos que además $ \lambda q = (\lambda x_q, \lambda y_q) \in L$. Vamos a tener entonces:
\begin{equation*}
\begin{array}{rl}
a (\lambda x_q ) + b (\lambda y_q) &= c \\
\lambda a x_q  + \lambda b y_q &= c \\
\lambda ( a x_q  + b y_q ) &= c \; \; \text{(Reemplazamos la ecuación \ref{q_in_L})}\\
\lambda c &= c \\
(\lambda -1 ) c &= 0
\end{array}
\end{equation*}

Luego tenemos dos casos: Si $\lambda = 1$, entonces $c$ puede tomar cualquier valor. Si $\lambda \neq 1$ entonces sólo puede ser $c=0$. En particular, si $c=0$, $\lambda$ puede tener cualquier valor.
 
 \item Llamemos $p=(x_p,y_p)$. Como $p \in L$ vamos a tener el análogo a la ecuación \ref{q_in_L} para $p$:
\begin{equation}\label{p_in_L}
a x_p + b y_p = c
\end{equation} 

Ahora suponemos que además $p+q = (x_p + x_q,y_p + y_q) \in L$ y tenemos:
\begin{equation*}
\begin{array}{rl}
a (x_p + x_q ) + b (y_p + y_q) &= c \\
a x_p + a x_q + b y_p + b y_q &= c \\
(a x_p + b y_p) + (a x_q + b y_q) &= c \\
c + c &= c \; \; \text{(reemplazamos las ecuaciones \ref{q_in_L} y \ref{p_in_L})}\\
2 c &= c \\
c &= 0
\end{array}
\end{equation*}

Por lo tanto debe ser $c=0$ y es el único valor con esta propiedad.
\end{enumerate}

\qed

\item Sea $L$ una recta en $\mathbb{R}^2$. Probar que $L$ pasa por $(0,0)$ si y sólo si pasa por $p+\lambda q$ para todo par de puntos $p$ y $q$ de $L$ y para todo $\lambda\in\mathbb{R}$.
\end{enumerate}

\rta 

$\boxed{ \implies }$ Supongo que $(0,0) \in L$, entonces por el ejercicio 19.a) tengo que $c=0$. Si $c=0$, por ejercicio 19.b) tengo que como $q \in L$ entonces $\lambda q \in L$. Luego, por ejercicio 19.c) tengo que como $p \in L$ y $\lambda q \in L$ entonces su suma también: $p + \lambda q \in L$.

$\boxed{ \impliedby }$ Considero un $p \in L$ cualquiera, y tomo $\lambda = -1$ y $q = p$. Tengo entonces por hipótesis que $p + \lambda q \in L$, pero $p + \lambda q = p + (-1) p = p - p = (0,0)$ y por lo tanto $(0,0) \in L$. \qed

\end{comment}
%%%=======================
%%%=======================
%%% CAP2 =================
    % PDFLaTeX
\documentclass[a4paper,12pt,twoside,spanish,reqno]{amsbook}
%%%---------------------------------------------------
\usepackage[math]{kurier}

\usepackage{etex}
\usepackage{t1enc}
\usepackage{latexsym}
\usepackage[utf8]{inputenc}
\usepackage{verbatim}
\usepackage{multicol}
\usepackage{amsgen,amsmath,amstext,amsbsy,amsopn,amsfonts,amssymb}
\usepackage{amsthm}
\usepackage{calc}         % From LaTeX distribution
\usepackage{graphicx}     % From LaTeX distribution
\usepackage{ifthen}
\input{random.tex}   
\usepackage{tikz}
\usetikzlibrary{arrows}
\usetikzlibrary{matrix}
\usepackage{mathtools}
\usepackage{stackrel}
\usepackage{enumitem}
\usepackage{tkz-graph}

\usepackage{enumitem} 
\usepackage[compatibility=false]{caption} % para usar subcaption
\usepackage{subcaption} % para poner varias imagenes juntas
\usetikzlibrary{arrows.meta}
\usepackage{hyperref}
\hypersetup{ 
    colorlinks=true,
    linkcolor=blue,
    filecolor=magenta,      
    urlcolor=cyan,
}
\usepackage{hypcap}
\numberwithin{equation}{section}
% http://www.texnia.com/archive/enumitem.pdf (para las labels de los enumerate)
\renewcommand\labelitemi{$\circ$}
\setlist[enumerate, 1]{label={(\arabic*)}}
\setlist[enumerate, 2]{label=\emph{\alph*)}}


\newcommand{\rta}{\noindent\textsc{Solución: }} 

\newcommand{\img}{\operatorname{Im}}
\newcommand{\nuc}{\operatorname{Nu}}
\newcommand\im{\operatorname{Im}}
\renewcommand\nu{\operatorname{Nu}}
\newcommand{\la}{\langle}
\newcommand{\ra}{\rangle}
\renewcommand{\t}{{\operatorname{t}}}
\renewcommand{\sin}{{\,\operatorname{sen}}}
\newcommand{\Q}{\mathbb Q}
\newcommand{\R}{\mathbb R}
\newcommand{\C}{\mathbb C}
\newcommand{\K}{\mathbb K}
\newcommand{\F}{\mathbb F}
\newcommand{\Z}{\mathbb Z}
\newenvironment{amatrix}[1]{%
  \left[\begin{array}{@{}*{#1}{c}|c@{}}
}{%
  \end{array}\right]
}

%%% FORMATOS %%%%%%%%%%%%%%%%%%%%%%%%%%%%%%%%%%%%%%%%%%%%%%%%%%%%%%%%%%%%%%%%%%%%%
\tolerance=10000
\renewcommand{\baselinestretch}{1.3}
\usepackage[a4paper, top=3cm, left=3cm, right=2cm, bottom=2.5cm]{geometry}
\usepackage{setspace}
%\setlength{\parindent}{0,7cm}% tamaño de sangria.
\setlength{\parskip}{0,4cm} % separación entre parrafos.
\renewcommand{\baselinestretch}{0.90}% separacion del interlineado
%%%%%%%%%%%%%%%%%%%%%%%%%%%%%%%%%%%%%%%%%%%%%%%%%%%%%%%%%%%%%%%%%%%%%%%%%%%%%%%%%%%
%\end{comment}
%%% FIN FORMATOS  %%%%%%%%%%%%%%%%%%%%%%%%%%%%%%%%%%%%%%%%%%%%%%%%%%%%%%%%%%%%%%%%%

\begin{document}
    \baselineskip=0.55truecm %original
    
    
    {\bf \begin{center} Práctico 2 \\ Álgebra  II -- Año 2024/1 \\ FAMAF \end{center}}

\

\centerline{\textsc{Sistemas de ecuaciones}}

\centerline{\textsc{Soluciones}}

\bigbreak



\begin{enumerate}
\item {\it Juego Suko}. Colocar los números del $1$ al $9$ en las celdas de la siguiente tabla de modo que el número en cada círculo sea igual a la suma de las cuatro celdas adyacentes, y la suma de las celdas del mismo color sea igual al número en el círculo de igual color.

\begin{center}
  \begin{tikzpicture}
    \draw [fill=gray!10] (0,0) rectangle (1,-1); 
    \draw [fill=gray!10] (1,0) rectangle (2,-1); 
    \draw [fill=gray!10] (2,0) rectangle (3,-1); 
    \draw [fill=gray!30] (0,-1) rectangle (1,-2); 
    \draw [fill=gray!30] (1,-1) rectangle (2,-2); 
    \draw [fill=gray!10] (2,-1) rectangle (3,-2); 
    \draw [fill=gray!80] (0,-2) rectangle (1,-3); 
    \draw [fill=gray!80] (1,-2) rectangle (2,-3); 
    \draw [fill=gray!80] (2,-2) rectangle (3,-3); 
    \filldraw[fill=white](1,-1) circle (0.33);
    \filldraw[fill=white](2,-1) circle (0.33);
    \filldraw[fill=white](1,-2) circle (0.33);
    \filldraw[fill=white](2,-2) circle (0.33);
    \node at (1,-1) {12};
    \node at (2,-1) {19};
    \node at (1,-2) {23};
    \node at (2,-2) {26};
    \filldraw[fill=gray!10](0.5,-3.5) circle (0.33);
    \filldraw[fill=gray!30](1.5,-3.5) circle (0.33);
    \filldraw[fill=gray!80](2.5,-3.5) circle (0.33);
    \node at (0.5,-3.5) {14};
    \node at (1.5,-3.5) {9};
    \node at (2.5,-3.5) {22};
    
    \end{tikzpicture}
\end{center}
\vskip.2cm
\noindent\textsc{Solución:} Queremos ver que valor toma cada celda y, por lo tanto,  a cada celda le asignamos una variable:
\begin{center}
  \begin{tikzpicture}
    \draw [fill=gray!10] (0,0) rectangle (1,-1); 
    \draw [fill=gray!10] (1,0) rectangle (2,-1); 
    \draw [fill=gray!10] (2,0) rectangle (3,-1); 
    \draw [fill=gray!30] (0,-1) rectangle (1,-2); 
    \draw [fill=gray!30] (1,-1) rectangle (2,-2); 
    \draw [fill=gray!10] (2,-1) rectangle (3,-2); 
    \draw [fill=gray!80] (0,-2) rectangle (1,-3); 
    \draw [fill=gray!80] (1,-2) rectangle (2,-3); 
    \draw [fill=gray!80] (2,-2) rectangle (3,-3); 
    \node at (0.5,-0.5) {$x_1$};
    \node at (1.5,-0.5) {$x_2$};
    \node at (2.5,-0.5) {$x_3$};
    \node at (0.5,-1.5) {$x_4$};
    \node at (1.5,-1.5) {$x_5$};
    \node at (2.5,-1.5) {$x_6$};
    \node at (0.5,-2.5) {$x_7$};
    \node at (1.5,-2.5) {$x_8$};
    \node at (2.5,-2.5) {$x_9$};
    \end{tikzpicture}
\end{center}
Tenemos entonces las 9 incógnitas que debemos resolver y la información del Suko original nos dice que se deben cumplir las siguientes ecuaciones:
\begin{align}
&x_1 + x_2 +x_4 +x_5 =12,  \\
&x_2+ x_3+x_5 +x_6 = 19, \\
&x_4 +x_5 +x_7 + x_8 = 23, \\
&x_5 +x_6 +x_8 + x_9 = 26, \\
&x_4 + x_5 = 9, \\
&x_1+x_2 +x_3 +x_6 = 14, \\
&x_7+ x_8 + x_9 = 22
\end{align}


Primero vamos a tratar de resolver el sistema de ecuaciones. Podríamos plantear  la matriz ampliada del sistema y  reducir la matriz, pero en este caso va a resultar más corto trabajar con las ecuaciones directamente. Hay 9 incógnitas y 7 ecuaciones, entonces en general es razonable que queden 7 variables dependientes y 2 variables libres. Con esto en mente trataremos de despejar todo en función de las variables $x_1$ y $x_3$ (esto fue elegido arbitrariamente). 

\vskip .2cm 

Como $x_4 + x_5 = 9$ $\;\stackrel{(1)}{\Rightarrow}\;$ $x_1 + x_2 + 9 =12$,  es decir $x_1 + x_2  =3$ o \colorbox{green!20}{$x_2 = 3 -x_1$ ($a$)}.


Como $x_4 + x_5 = 9$ 
$\;\stackrel{(3)}{\Rightarrow}\;$ 
$9+x_7 + x_8 = 23$ 
$\;\stackrel{}{\Rightarrow}\;$
$x_7 + x_8 = 14$ 
$\;\stackrel{(7)}{\Rightarrow}\;$
$14 + x_9 = 22$ $\;\stackrel{}{\Rightarrow}\;$ \colorbox{green!20}{$x_9 = 8$ ($b$)}.

Ahora, como $x_1 + x_2 = 3$ $\;\stackrel{(6)}{\Rightarrow}\;$ $3 +x_3 +x_6 = 14$,  es decir $x_3 +x_6 = 11$ o \colorbox{green!20}{$x_6 = 11 -x_3$ ($c$)}. 

Si en la ecuación (2) reemplazamos $x_2$ y $x_6$, obtenemos
$$
19 = x_2+ x_3+x_5 +x_6 = (3 -x_1) + x_3 + x_5 +(11 -x_3) = 14 -x_1 +x_5,
$$
o \colorbox{green!20}{$x_5 = 5 +x_1$ ($d$)}.

Con todo lo que hemos averiguado hacemos reemplazos en (4):
$$
26 = x_5 +x_6 +x_8 + x_9 = ( 5 +x_1) +( 11 -x_3) +x_8 + 8 = 24 +x_1 - x_3 + x_8, 
$$ 
luego \colorbox{green!20}{$x_8 = 2 -x_1 + x_3$ ($e$)}.

De la fórmula (7), de ($e$) y de ($b$), obtenemos 
$$
22 = x_7+ x_8 + x_9 = x_7+ (2 -x_1 + x_3) + 8 =  10 + x_7 - x_1 +x_3, 
$$
luego \colorbox{green!20}{$x_7 = 12 + x_1 - x_3$ ($f$)}.

Utilizando ($d$) y la fórmula (5):


$$
9 =x_4+ x_5 = x_4 + (5 +x_1) = 5 + x_1 +x_4,
$$ 
luego  \colorbox{green!20}{$x_4 = 4 - x_1$ ($g$)}.


Teníamos 7 ecuaciones y 9  incógnitas, entonces. como ya dijimos,  era de esperarse  que queden 7 variables dependientes y 2 variables libres,  en este caso $x_1$ y $x_3$. Si  no tuviéramos más restricciones la cantidad de soluciones sería infinita, pero debemos considerar que
\begin{equation*}
  x_i \in \mathbb N \quad\wedge \quad 1 \le x_i \le 9  \quad\wedge \quad x_i \ne x_i \text{ si $i\ne j$.} \tag{*}
\end{equation*}
($1 \le i,j \le 7$).  
\vskip .3cm 
Debido a (*) y ($a$), $x_1$ solo puede ser $1$ o $2$. 
\vskip .3cm 

\noindent{\bf Caso $x_1=1$.} por las ecuaciones  ($a$),$\ldots$,($f$) obtenemos:
\begin{equation*}
  \begin{array}{rlrlrl}
    x_1 &= 1,        & x_2 &= 2,         &x_4 &= 3, \\
    x_5 &= 6,        & x_9 &= 8,         &   &\\ 
    x_6 &= 11 - x_3,\quad & x_7 &= 13 - x_3,\quad   &x_8 &= x_3 +1, 
\end{array} \tag{**}
\end{equation*}
y $x_3$ libre (con las restricciones de (*)). Luego, $x_3$ tampoco puede tomar los valores $1,2,3,6,8$ (pues ya los tienen otras variables),  así que $x_3= 4, 5, 7 ,9$. 

{\bf Subcaso $x_1=1$, $x_3 =4$.}  En  este caso, por (**): $x_6=7$, $x_7=9$, $x_8 =5$,  y estas serían soluciones admisibles.

{\bf Subcaso $x_1=1$, $x_3 =5$.}  En  este caso, por (**), $x_6 = 6 = x_5$, lo cual no es admisible (debe ser $x_6 \ne x_5$. )

{\bf Subcaso $x_1=1$, $x_3 =7$.} En  este caso, por (**), $x_8 = 7 + 1  = 8 = x_9$, lo cual no es admisible. 


{\bf Subcaso $x_1=1$, $x_3 =9$.} En  este caso, por (**), $x_8 = 10$, lo cual no es admisible ($x_i \le 9$ para todo $i$). 

\vskip .2cm 

Falta ver

\vskip .3cm 

\noindent{\bf Caso $x_1=2$.}  Por la ecuación ($g$)   obtenemos $x_4 = 4 -2 = 2 = x_1$, lo cual no es admisible.

\vskip .3cm 

Es decir,  hay  una única solución para estas ecuaciones con las restricciones mencionadas:
\begin{equation*}
  \begin{array}{rlrlrl}
    x_1 &= 1,        & x_2 &= 2,         &x_3 &= 4, \\
    x_4 &= 3,        & x_5 &= 6,         &x_6 &= 7\\ 
    x_7 &= 9,\quad & x_8 &= 5,\quad   &x_9 &= 8. 
  \end{array} 
\end{equation*}
\qed
\vskip .6cm

\item\label{polinomio} Encontrar los coeficientes reales del polinomio $p(x) = ax^2+bx+c$ de manera tal que $p(1)=2$, $p(2)=7$ y $p(3)=14$.
\vskip.2cm
\noindent\textsc{Solución:} Planteemos a nivel de los coeficientes del polinomio las condiciones:
\begin{equation*}
 \begin{array}{rrrrrrrrr}
 P(1) =2 &\quad \Rightarrow \quad& a\cdot 1^2 &+ & b \cdot 1 &+ & c &= & 2 \\   
 P(2) =7 &\quad \Rightarrow \quad& a\cdot 2^2 &+ & b \cdot 2 &+ & c &= & 7 \\
 P(3) =14 &\quad \Rightarrow \quad& a\cdot 3^2 &+ & b \cdot 3 &+ & c &= & 14.    
\end{array} \tag{*}
\end{equation*}
Nuestro objetivo es averiguar $a$, $b$ y $c$,  que como vemos, se podrían obtener resolviendo el sistema de ecuaciones lineales planteado en (*). La matriz ampliada de este sistema es 
$$
[A|Y] = \left[\begin{array}{ccc|c}
1&1&1&2 \\
4&2&1&7\\
9&3&1&14.
\end{array}\right]
$$
Resolvamos el sistema.
\begin{multline*}\qquad
\left[\begin{array}{ccc|c}
1&1&1&2 \\
4&2&1&7\\
9&3&1&4
\end{array}\right] \stackrel{F_3 - 9F_1}{\stackrel{F_2 - 4 F_1}{\longrightarrow}}
\left[\begin{array}{ccc|c}
1&1&1&2 \\
0&-2&-3&-1\\
0&-6&-8&-4
\end{array}\right]\stackrel{F_3 - 3 F_2}{\longrightarrow}
\left[\begin{array}{ccc|c}
1&1&1&2 \\
0&-2&-3&-1\\
0&0&1&-1
\end{array}\right] \\
\stackrel{F_1 - F_3}{\stackrel{F_2 + 3 F_3}{\longrightarrow}}
\left[\begin{array}{ccc|c}
1&1&0&3 \\
0&-2&0&-4\\
0&0&1&-1
\end{array}\right]\stackrel{F_2/(-2)}{\longrightarrow}
\left[\begin{array}{ccc|c}
1&1&0&3 \\
0&1&0&2\\
0&0&1&-1
\end{array}\right]\stackrel{F_1 -F_2}{\longrightarrow}
\left[\begin{array}{ccc|c}
1&0&0&1 \\
0&1&0&2\\
0&0&1&-1.
\end{array}\right]
\qquad
\end{multline*}
Luego $a = 1$, $b =2$, $c=-1$. Es decir el polinomio que satisface las hipótesis es:
$$
x^2 + 2x -1.
$$



\vskip .6cm
\item Determinar cuáles de las siguientes matrices son MERF.
$$\begin{array}{lccccl}
\begin{bmatrix}1 & 2 & 0 \\0 & 0 & 1 \end{bmatrix}, &
\begin{bmatrix}1 & 0 & 2 \\0 & 1 & -3 \end{bmatrix}, &
\begin{bmatrix}0 & 1 & 0 \\0 & 0 & 1 \end{bmatrix}, &
\begin{bmatrix}0 & 1 & 0 \\0 & 0 & 0 \end{bmatrix}, &
\begin{bmatrix}1 & 0 & 0  \\0 & 0 & 1 \\0 & 0 & 1 \end{bmatrix},&
\begin{bmatrix}1 & 0 & 0  \\0 & 0 & 0 \\0 & 0 & 1 \end{bmatrix}.
\end{array}$$
\vskip.2cm
\noindent\textsc{Solución:} Recordemos que una matriz MERF debe satisfacer:
\begin{enumerate}
\item[\textit{a})] la primera entrada no nula de una fila es 1 (el 1 principal).
\item[\textit{b})] Cada columna que contiene un  1 principal tiene todos los otros elementos iguales a 0. 
\item[\textit{c})] todas las filas cuyas entradas son todas iguales a cero están al final de la matriz, y
\item[\textit{d})] en dos filas consecutivas no nulas el 1 principal de la fila inferior está más a la derecha que el 1 principal de la fila superior. 
\end{enumerate}
\vskip .3cm

Las primeras cuatro matrices satisfacen la definición de MERF. 

La 5º matriz no satisface \textit{b}), pues el 1 principal es la segunda fila está en la columna 3  y  en esa columna hay otro elemento no nulo (en la posición $33$ hay un 1).

La 6º matriz tampoco es MERF pues la fila 2 es nula y la fila 3 no lo es. Luego  no satisface \textit{c}).  \qed

\vskip .5cm

\item Para cada una de las MERF del ejercicio anterior,

\

\begin{enumerate}
\item
asumir que es la matriz de un sistema homogéneo, escribir el sistema
y dar las soluciones del sistema.

\

\item
asumir que es la matriz ampliada de un sistema no homogéneo, escribir el sistema
y dar las soluciones del sistema.
\end{enumerate}
\vskip.2cm
\noindent\textsc{Solución:}
Como ya vimos en el ejercicio anterior las MERF son

$$
(i)\;\begin{bmatrix}1 & 2 & 0 \\0 & 0 & 1 \end{bmatrix}, \quad
(ii)\; \begin{bmatrix}1 & 0 & 2 \\0 & 1 & -3 \end{bmatrix}, \quad
(iii)\;\begin{bmatrix}0 & 1 & 0 \\0 & 0 & 1 \end{bmatrix}, \quad
(iv)\;\begin{bmatrix}0 & 1 & 0 \\0 & 0 & 0 \end{bmatrix}.$$

\noindent (a) 

(\textit{i}) El sistema homogéneo correspondiente  es 
$$\begin{array}{rl}
x_1 +  2x_2 &= 0 \\ x_3&=0, 
\end{array}$$
luego $x_1= -2x_2$ y las soluciones del sistema son $\{ (-2t, t,0): t\in \mathbb R \}$.

(\textit{ii}) El sistema homogéneo correspondiente  es 
$$\begin{array}{rl}
x_1+ 2x_3 &= 0 \\x_2  -3x_3 &= 0,
\end{array}$$
luego $x_1 = -2x_3$, $x_2 = 3x_3$ y las soluciones son $\{ (-2t, 3t,t): t\in \mathbb R \}$.


(\textit{iii}) El sistema homogéneo correspondiente es 
$$\begin{array}{rl}
x_2 &= 0 \\ x_3&=0,
\end{array}$$
luego  las soluciones son $\{ (t, 0,0): t\in \mathbb R \}$.


(\textit{iv}) El sistema homogéneo correspondiente  es 
$$\begin{array}{rl}
x_2&=0,
\end{array}$$
luego  las soluciones son $\{ (t,0,s): t, s\in \mathbb R \}$.


\vskip .3cm
\noindent (b) 
Las matrices ampliadas son
$$
(i)\;\begin{amatrix}{2}1 & 2 & 0 \\0 & 0 & 1\end{amatrix}, \quad
(ii)\;\begin{amatrix}{2}1 & 0 & 2 \\0 & 1 & -3 \end{amatrix},\quad 
(iii)\;\begin{amatrix}{2}0 & 1 & 0 \\0 & 0 & 1 \end{amatrix}, \quad
(iv)\;\begin{amatrix}{2}0 & 1 & 0 \\0 & 0 & 0 \end{amatrix}.
$$



(\textit{i}) El sistema no homogéneo correspondiente  es 
$$\begin{array}{rl}
x_1 +  2x_2 &= 0 \\ 0&=1.
\end{array}$$
Por lo tanto no tiene solución.

(\textit{ii}) El sistema  correspondiente  es 
$$\begin{array}{rl}
x_1&= 2 \\x_2 &= -3,
\end{array}$$
luego la solución es $(2,-3)$.


(\textit{iii}) El sistema  correspondiente  es 
$$\begin{array}{rl}
x_2 &= 0 \\ 0&=1.
\end{array}$$
Por lo tanto no tiene solución.


(\textit{iv}) El sistema correspondiente  es 
$$\begin{array}{rl}
x_2&=0,
\end{array}$$
luego  las soluciones son $\{ (t,0): t\in \mathbb R \}$.


\vskip .5cm

\item\label{sistemas homogeneos} Para cada uno de los siguientes sistemas de ecuaciones, describir explícita o paramétricamente todas las soluciones e indicar cuál es la MERF asociada al sistema.

\begin{multicols}{3}
\begin{enumerate}
\item $\begin{cases}
 -x - y + 4z = 0\\
 x+3y+8z = 0\\
 x+2y + 5z = 0
\end{cases}$
\item $\begin{cases}
 x - 3y + 5z = 0\\
 2x-3y+z = 0\\
 -y + 3z = 0
\end{cases}$

\item $\begin{cases}
x-z+2t = 0\\
-x+2y-z+2t = 0\\
-x+y = 0
\end{cases}$

\end{enumerate}
\end{multicols}

\begin{multicols}{3}
\begin{enumerate}

\item[(d)] $\begin{cases}
 -x - y + 4z = 1\\
 x+3y+8z = 3\\
 x+2y + 5z = 1
\end{cases}$

\item[(e)] $\begin{cases}
 x - 3y + 5z = 1\\
 2x-3y+z = 3\\
 -y + 3z = 1
\end{cases}$


\item[(f)] $\begin{cases}
x-z+2t = 1\\
-x+2y-z+2t = 3\\
-x+y = 1
\end{cases}$

\end{enumerate}
\end{multicols}
\vskip.2cm
\noindent\textsc{Solución:}

(a) La matriz asociada a este sistema es la matriz $A_1$ que escribimos más abajo y a la cual luego reducimos a una MERF.


\begin{align*}
A_1 &= \left[\begin{array}{ccc}
 -1&-1&4\\
 1&3&8\\
 1&2&5\end{array}\right] \stackrel{F_2+F_1}{\stackrel{F_3+F_1}{\longrightarrow}}   
 \left[\begin{array}{ccc}
 -1&-1&4\\
 0&2&12\\
 0&1&9\end{array}\right]\stackrel{F_1+F_3}{\stackrel{F_2-2F_3}{\longrightarrow}}  
 \left[\begin{array}{ccc}
 -1&0&13\\
 0&0&-6\\
 0&1&9\end{array}\right]  \\
 &\stackrel{F_1/(-1)}{\stackrel{F_2/(-6)}{\longrightarrow}}  \quad
 \left[\begin{array}{ccc}
 1&0&-13\\
 0&0&1\\
 0&1&9\end{array}\right]\stackrel{F_1+13F_2}{\stackrel{F_3-9F_2}{\longrightarrow}}   
 \left[\begin{array}{ccc}
1&0&0\\
 0&0&1\\
 0&1&0\end{array}\right]{\stackrel{F_2 \leftrightarrow F_3}{\longrightarrow}}   
  \left[\begin{array}{ccc}
1&0&0\\
 0&1&0\\
 0&0&1\end{array}\right] =R_1.
\end{align*}
Luego el sistema tiene solución trivial $(0,0,0)$ y la MERF es la matriz $\operatorname{Id}_3$.

\vskip .3cm
(b)
\begin{align*}
A_2 &= \left[\begin{array}{ccc}
 1&-3&5\\
 2&-3&1\\
 0&-1&3\end{array}\right] \stackrel{F_2-2F_1}{\longrightarrow}   
 \left[\begin{array}{ccc}
1&-3&5\\
 0&3&-9\\
 0&-1&3\end{array}\right]\stackrel{F_1-3F_3}{\stackrel{F_2+3F_3}{\longrightarrow}}  
 \left[\begin{array}{ccc}
 1&0&-4\\
 0&0&0\\
 0&-1&3\end{array}\right]  \\
 &{\stackrel{-F_3}{\longrightarrow}}  \quad
 \left[\begin{array}{ccc}
 1&0&-4\\
 0&0&0\\
 0&1&-3\end{array}\right]{\stackrel{F_2 \leftrightarrow F_3}{\longrightarrow}}   
  \left[\begin{array}{ccc}
1&0&-4\\
 0&1&-3\\
0&0&0 \end{array}\right]=R_2.
\end{align*}
Luego la MERF asociada al sistema es $R_2$ y ahora el nuevo sistema es
$$
\begin{cases}
x -4 z = 0 \\
y -3 z =0
\end{cases} \quad \Rightarrow\quad 
\begin{cases}
x = 4 z \\
y =3 z. 
\end{cases}
$$
Por lo tanto las soluciones del sistema son $\{(4t,3t,t): t\in \mathbb R \}$.

\vskip .3cm
(c) Este es un sistema homogéneo de 3 ecuaciones con  4 incógnitas ($x$, $y$, $z$, $t$). La matriz del sistema es la $A_3$ que mostramos en la siguiente fila y luego la reducimos a MERF:
\begin{align*}
A_3 &= \begin{bmatrix}
 1&0&-1&2\\
 -1&2&-1&2\\
 -1&1&0&0\end{bmatrix} \stackrel{F_2+F_1}{\stackrel{F_3+F_1}{\longrightarrow}}   
 \begin{bmatrix}
 1&0&-1&2\\
 0&2&-2&4\\
 0&1&-1&2\end{bmatrix}{\stackrel{F_2/2}{\longrightarrow}}   
 \begin{bmatrix}
 1&0&-1&2\\
 0&1&-1&2\\
 0&1&-1&2\end{bmatrix}{\stackrel{F_3-F_2}{\longrightarrow}}  \\
 &\quad
 \begin{bmatrix}
 1&0&-1&2\\
 0&1&-1&2\\
 0&0&0&0\end{bmatrix} = R_3.
\end{align*}
Luego $R_3$ es la MERF asociada al sistema. El sistema ahora es 
\begin{equation*}
    \begin{cases}
 x - z + 2t = 0\\
 y - z + 2t = 0
\end{cases}
\quad \Rightarrow \quad
 \begin{cases}
 x = z - 2t \\
 y = z - 2t.
\end{cases}
\end{equation*}
Por lo tanto,  las soluciones del sistema son  $\{ (u -2v, u-2v,u,v): u,v\in \mathbb R \}$.


\vskip .3cm
(d) Este es un sistema no homogéneo de 3 ecuaciones y 3 incógnitas. La matriz ampliada del sistema es $[A_1|Y]$, donde $A_1$ es la matriz del inciso (a). Luego  para reducir $A_1$  a MERF hacemos los mismos pasos que en (a):
\begin{align*}
[A_1|Y] &= \left[\begin{array}{ccc|c}
 -1&-1&4&1\\
 1&3&8&3\\
 1&2&5&1\end{array}\right] \stackrel{F_2+F_1}{\stackrel{F_3+F_1}{\longrightarrow}}   
 \left[\begin{array}{ccc|c}
 -1&-1&4&1\\
 0&2&12&4\\
 0&1&9&2\end{array}\right]\stackrel{F_1+F_3}{\stackrel{F_2-2F_3}{\longrightarrow}}  
 \left[\begin{array}{ccc|c}
 -1&0&13&3\\
 0&0&-6&0\\
 0&1&9&2\end{array}\right]  \\
 &\stackrel{F_1/(-1)}{\stackrel{F_2/(-6)}{\longrightarrow}}  \quad
 \left[\begin{array}{ccc|c}
 1&0&-13&-3\\
 0&0&1&0\\
 0&1&9&2\end{array}\right]\stackrel{F_1+13F_2}{\stackrel{F_3-9F_2}{\longrightarrow}}   
 \left[\begin{array}{ccc|c}
1&0&0&-3\\
 0&0&1&0\\
 0&1&0&2\end{array}\right]{\stackrel{F_2 \leftrightarrow F_3}{\longrightarrow}}   
  \left[\begin{array}{ccc|c}
1&0&0&-3\\
 0&1&0&2\\
 0&0&1&0\end{array}\right].
\end{align*}
Luego la MERF asociada al sistema es $\operatorname{Id}_3$ (con en (a),  obviamente) y  la solución del sistema es $(-3,2,0)$.


\vskip .3cm
(e) Este es un sistema no homogéneo de 3 ecuaciones y 3 incógnitas. La matriz ampliada del sistema es $[A_2|Y]$ donde $A_2$ es la matriz del inciso (b). Luego  para reducir $A_2$  a MERF hacemos los mismos pasos que en (b):
\begin{align*}
[A_2|Y] &= \left[\begin{array}{ccc|c}
 1&-3&5&1\\
 2&-3&1&3\\
 0&-1&3&1\end{array}\right] \stackrel{F_2-2F_1}{\longrightarrow}   
 \left[\begin{array}{ccc|c}
1&-3&5&1\\
 0&3&-9&1\\
 0&-1&3&1\end{array}\right]\stackrel{F_1-3F_3}{\stackrel{F_2+3F_3}{\longrightarrow}}  
 \left[\begin{array}{ccc|c}
 1&0&-4&-2\\
 0&0&0&4\\
 0&-1&3&1\end{array}\right]  \\
 &{\stackrel{-F_3}{\longrightarrow}}  \quad
 \left[\begin{array}{ccc|c}
 1&0&-4&-2\\
 0&0&0&4\\
 0&1&-3&-1\end{array}\right]{\stackrel{F_2 \leftrightarrow F_3}{\longrightarrow}}   
  \left[\begin{array}{ccc|c}
1&0&-4&-2\\
 0&1&-3&-1\\
0&0&0&4 \end{array}\right].
\end{align*}
Luego la MERF asociada al sistema es $R_2$ del ejercicio (b)  y  como en el nuevo sistema tenemos la ecuación $0=4$,  el sistema no tiene solución.


\vskip .3cm
(f) Este es un sistema  no homogéneo de 3 ecuaciones con  4 incógnitas. La matriz ampliada del sistema es $[A_3|Y]$ donde $A_3$ es la matriz del inciso (c). Luego  para reducir $A_3$  a MERF hacemos los mismos pasos que en (c):
\begin{align*}
[A_3|Y] &= \left[\begin{array}{cccc|c}
 1&0&-1&2&1\\
 -1&2&-1&2&3\\
 -1&1&0&0&1\end{array}\right] \stackrel{F_2+F_1}{\stackrel{F_3+F_1}{\longrightarrow}}   
 \left[\begin{array}{cccc|c}
 1&0&-1&2&1\\
 0&2&-2&4&4\\
 0&1&-1&2&2\end{array}\right]{\stackrel{F_2/2}{\longrightarrow}}  \\
  &\quad
 \left[\begin{array}{cccc|c}
 1&0&-1&2&1\\
 0&1&-1&2&2\\
 0&1&-1&2&2\end{array}\right]{\stackrel{F_3-F_2}{\longrightarrow}}  
 \left[\begin{array}{cccc|c}
 1&0&-1&2&2\\
 0&1&-1&2&2\\
 0&0&0&0&0\end{array}\right].
\end{align*}
Luego $R_3$ (de (c)) es la MERF asociada al sistema. El sistema ahora es 
\begin{equation*}
    \begin{cases}
 x - z + 2t = 2\\
 y - z + 2t = 2
\end{cases}
\quad \Rightarrow \quad
 \begin{cases}
 x = z - 2t +2 \\
 y = z - 2t +2.
\end{cases}
\end{equation*}
Por lo tanto,  las soluciones del sistema son  $\{ (u -2v +2, u-2v +2 ,u,v): u,v\in \mathbb R \}$.


\qed
\vskip .5cm

\item\label{sistemas con soluciones} Para cada uno de los siguientes sistemas, describir implícitamente el conjunto de los vectores $(b_1,b_2,b_3)$
o $(b_1,b_2,b_3,b_4)$ para los cuales cada sistema tiene solución.

\begin{multicols}{3}
\begin{enumerate}

\item $\begin{cases}
 x - 3y + 5z = b_1\\
 2x-3y+z = b_2\\
 -y + 3z = b_3
\end{cases}$


\item $\begin{cases}
x-z+2t = b_1\\
-x+2y-z+2t = b_2\\
-x+y = b_3\\
y-z+2t=b_4
\end{cases}$


\item $\begin{cases}
 - x - y + 4 z = b_1\\
 x+3y+8z = b_2\\
 x + 2y + 5z = b_3
\end{cases}$

\end{enumerate}
\end{multicols}
\vskip.2cm
\noindent\textsc{Solución:}
\vskip .3cm
(a) La matriz ampliada del sistema es  
\begin{equation*}
[A_2|Y] = \left[\begin{array}{ccc|c}
 1&-3&5&b_1\\
 2&-3&1&b_2\\
 0&-1&3&b_3\end{array}\right] .
\end{equation*}
Observar que $A_2$ es la misma matriz que la de los ejercicios (\ref{sistemas homogeneos}) b) y e) y, por lo tanto, los pasos para reducir la matriz asociada al sistema serán los mismos. 
\begin{align*}
[A_2|Y] &= \left[\begin{array}{ccc|c}
 1&-3&5&b_1\\
 2&-3&1&b_2\\
 0&-1&3&b_3\end{array}\right] \stackrel{F_2-2F_1}{\longrightarrow}   
 \left[\begin{array}{ccc|c}
 1&-3&5&b_1\\
 0&3&-9&b_2-2b_1\\
 0&-1&3&b_3\end{array}\right]\stackrel{F_1-3F_3}{\stackrel{F_2+3F_3}{\longrightarrow}}  \\
 &\left[\begin{array}{ccc|c}
 1&0&-4&b_1-3b_3\\
 0&0&0&-2b_1 +b_2 + 3b_3\\
 0&-1&3&b_3\end{array}\right] 
 {\stackrel{-F_3}{\longrightarrow}}  \quad
 \left[\begin{array}{ccc|c}
 1&0&-4&b_1-3b_3\\
 0&0&0&-2b_1 +b_2 + 3b_3\\
 0&1&-3&-b_3\end{array} \right]
  \\
& {\stackrel{F_2 \leftrightarrow F_3}{\longrightarrow}}   
  \left[\begin{array}{ccc|c}
1&0&-4&b_1-3b_3\\
 0&1&-3&-b_3\\
 0&0&0&-2b_1 +b_2 + 3b_3 \end{array}\right].
\end{align*}
Luego el conjunto de $b_i$'s para los cuales el sistema tiene solución es $$\{(b_1,b_2,b_3)\in \mathbb R^3: -2b_1 +b_2 + 3b_3 = 0\}.$$

\vskip .3cm
(b) La matriz ampliada del sistema es  
\begin{equation*}
[B|Y] = \left[\begin{array}{cccc|c}
 1&0&-1&2&b_1\\
 -1&2&-1&2&b_2\\
 -1&1&0&0&b_3\\
 0&1&-1&2&b_4\end{array}\right] .
\end{equation*}
Ahora, reducimos  $B$  a una MERF:. 
\begin{align*}
[B|Y] &= \left[\begin{array}{cccc|c}
1&0&-1&2&b_1\\
 -1&2&-1&2&b_2\\
 -1&1&0&0&b_3\\
 0&1&-1&2&b_4\end{array}\right] \stackrel{F_2+F_1}{\stackrel{F_3+F_1}{\longrightarrow}}   
 \left[\begin{array}{cccc|c}
 1&0&-1&2&b_1\\
 0&2&-2&4&b_1 +b_2\\
 0&1&-1&2&b_1 +b_3\\
 0&1&-1&2&b_4\end{array}\right]{\stackrel{F_2/2}{\longrightarrow}}  \\
  &\quad
 \left[\begin{array}{cccc|c}
 1&0&-1&2&b_1\\
 0&1&-1&2&\frac12b_1 +\frac12b_2\\
 0&1&-1&2&b_1 +b_3\\
 0&1&-1&2&b_4\end{array}\right]\stackrel{F_3-F_2}{\stackrel{F_4-F_2}{\longrightarrow}}  
 \left[\begin{array}{cccc|c}
 1&0&-1&2&b_1\\
 0&1&-1&2&\frac12b_1 +\frac12b_2\\
 0&0&0&0&b_1 +b_3 -\frac12b_1 -\frac12b_2\\
 0&0&0&0&b_4 -\frac12b_1 -\frac12b_2\end{array}\right] \\
 &= \left[\begin{array}{cccc|c}
 1&0&-1&2&b_1\\
 0&1&-1&2&\frac12b_1 +\frac12b_2\\
 0&0&0&0&\frac12b_1 -\frac12b_2 +b_3\\
 0&0&0&0&-\frac12b_1 -\frac12b_2+b_4 \end{array}\right].
\end{align*}
Luego el conjunto de $b_i$'s para los cuales el sistema tiene solución es 
$$\{(b_1,b_2,b_3,b_4)\in \mathbb R^3: \frac12b_1 -\frac12b_2 +b_3 = 0 \wedge -\frac12b_1 -\frac12b_2+b_4=0\}.$$

\vskip .3cm
(c) La matriz ampliada del sistema es  
\begin{equation*}
[A_1|Y] = \left[\begin{array}{ccc|c}
 -1&-1&4&b_1\\
 1&3&8&b_2\\
 1&2&5&b_3\end{array}\right] .
\end{equation*}
Observar que $A_1$ es la misma matriz que la de los ejercicios (\ref{sistemas homogeneos}) a) y d) y, por lo tanto, los pasos para reducir la matriz asociada al sistema serán los mismos. 

\begin{align*}
A_1 &= \left[\begin{array}{ccc|c}
 -1&-1&4&b_1\\
 1&3&8&b_2\\
 1&2&5&b_3\end{array}\right] \stackrel{F_2+F_1}{\stackrel{F_3+F_1}{\longrightarrow}}   
 \left[\begin{array}{ccc|c}
 -1&-1&4&b_1\\
 0&2&12&b_1+b_2\\
 0&1&9&b_1+b_3\end{array}\right]\stackrel{F_1+F_3}{\stackrel{F_2-2F_3}{\longrightarrow}}  \\
 &\left[\begin{array}{ccc|c}
 -1&0&13&2b_1+b_3\\
 0&0&-6&-b_1 + b_2 -2b_3 \\
 0&1&9&b_1+b_3\end{array}\right] 
 \stackrel{F_1/(-1)}{\stackrel{F_2/(-6)}{\longrightarrow}}  \quad
 \left[\begin{array}{ccc|c}
 1&0&-13&-2b_1-b_3\\
 0&0&1&\frac16b_1 -\frac16 b_2 +\frac13b_3 \\
 0&1&9&b_1+b_3\end{array}\right]\stackrel{F_1+13F_2}{\stackrel{F_3-9F_2}{\longrightarrow}}   \\
 &\left[\begin{array}{ccc|c}
1&0&0&\frac16b_1 -\frac{13}6b_2 +\frac{10}3b_3\\
&&&\\
 0&0&1&\frac16b_1 -\frac16 b_2 +\frac13b_3 \\
&&&\\
 0&1&0&-\frac12b_1 +\frac32 b_2  -2b_3\end{array}\right]{\stackrel{F_2 \leftrightarrow F_3}{\longrightarrow}}   
\left[\begin{array}{ccc|c}
 1&0&0&\frac16b_1 -\frac{13}6b_2 +\frac{10}3b_3\\
&&&\\
 0&1&0&-\frac12b_1 +\frac32 b_2  -2b_3 \\
&&&\\
 0&0&1&\frac16b_1 -\frac16 b_2 +\frac13b_3 \end{array}\right].
\end{align*}
Luego el conjunto de $b_i$'s para los cuales el sistema tiene solución es $\mathbb R^3$.


\qed
\vskip .5cm
\item Sea $A=\begin{bmatrix}1 & 2 & 3 & \cdots & 2016 \\ 2 & 3 & 4 & \cdots & 2017 \\ 3&4&5& \cdots & 2018\\ \vdots & &&& \vdots \\ 100 & 101 & 102& \cdots& 2115\end{bmatrix}$.

\

\begin{enumerate}
   \item Encontrar todas las soluciones del sistema $AX=0$.
   \item Encontrar todas las soluciones del sistema $AX=\left[\begin{array}{c}
     1\\\vdots \\1 \end{array}\right]$.
\end{enumerate}
\vskip.2cm
\noindent\textsc{Solución:}

\begin{enumerate}
   \item Reducimos la matriz A aplicando operaciones elementales por filas:

\begin{align*}
A = &\begin{bmatrix}1 & 2 & 3 & \cdots & 2016 \\ 2 & 3 & 4 & \cdots & 2017 \\ 3&4&5& \cdots & 2018\\ \vdots & &&& \vdots \\ 100 & 101 & 102& \cdots& 2115\end{bmatrix} \stackrel{F_2-F_1}{\stackrel{F_3 - F_1}{\stackrel{\vdots}{\stackrel{F_{100}-F_1}{\longrightarrow}}}}
\begin{bmatrix}1 & 2 & 3 & \cdots & 2016 \\ 1 & 1 & 1 & \cdots & 1 \\ 2&2&2& \cdots & 2\\ \vdots & &&& \vdots \\ 99 & 99 & 99& \cdots& 99\end{bmatrix}
\stackrel{F_3-2F_2}{\stackrel{F_4-3F_2}{\stackrel{\vdots}{\stackrel{F_{100}-99F_2}{\longrightarrow}}}} \\
&\begin{bmatrix}1 & 2 & 3 & \cdots & 2016 \\ 1 & 1 & 1 & \cdots & 1 \\ 0&0&0& \cdots & 0\\ \vdots & &&& \vdots \\ 0 & 0 & 0& \cdots& 0\end{bmatrix}
\stackrel{F_2 \leftrightarrow F_1}{\longrightarrow}
\begin{bmatrix}1 & 1 & 1 & \cdots & 1 \\ 1 & 2 & 3 & \cdots & 2016 \\ 0&0&0& \cdots & 0\\ \vdots & &&& \vdots \\ 0 & 0 & 0& \cdots& 0\end{bmatrix}
\stackrel{F_2-F_1}{\longrightarrow}
\begin{bmatrix}1 & 1 & 1 & 1 & \cdots & 1 \\ 0 & 1 & 2 & 3 & \cdots & 2015 \\ 0&0&0&0& \cdots & 0\\ \vdots & &&&& \vdots \\ 0 & 0 & 0&0& \cdots& 0\end{bmatrix}
\stackrel{F_1-F_2}{\longrightarrow} \\
&\begin{bmatrix}1 & 0 & -1 & -2 & \cdots & -2014 \\ 0 & 1 & 2 & 3 & \cdots & 2015 \\ 0&0&0&0& \cdots & 0\\ \vdots & &&&& \vdots \\ 0 & 0 & 0&0& \cdots& 0\end{bmatrix} = R_A
\end{align*}

Luego $R_A$ es la MERF asociada al sistema. El sistema ahora es 
\begin{equation*}
\begin{cases}
x_1 + (-1)x_3 + (-2)x_4 + \cdots + (-2014)x_{2016} = 0\\
x_2 + 2 x_3 + 3 x_4 + \cdots + 2015 x_{2016} = 0
\end{cases}
\quad \Rightarrow \quad
\begin{cases}
x_1 = \sum_{j=3}^{2016} (j-2) x_j \\
x_2 = \sum_{j=3}^{2016} (1-j) x_j
\end{cases}
\end{equation*}
Por lo tanto,  las soluciones del sistema son:

\

$\{ ( \sum_{j=3}^{2016} (j-2) x_j , \sum_{j=3}^{2016} (1-j) x_j , x_3 , x_4, \cdots, x_{2016}) : x_3,x_4,\cdots, x_{2016} \in \mathbb R \}$.

\

\item Podemos repetir la secuencia de operaciones elementales sobre el vector de unos para obtener las soluciones:

\begin{align*}
&\left[\begin{array}{c} 1\\ 1 \\ 1 \\ \vdots \\1 \end{array}\right]
\stackrel{F_2-F_1}{\stackrel{F_3 - F_1}{\stackrel{\vdots}{\stackrel{F_{100}-F_1}{\longrightarrow}}}}
\left[\begin{array}{c} 1\\ 0 \\ 0 \\ \vdots \\0 \end{array}\right]
\stackrel{F_3-2F_2}{\stackrel{F_4-3F_2}{\stackrel{\vdots}{\stackrel{F_{100}-99F_2}{\longrightarrow}}}}
\left[\begin{array}{c} 1\\ 0 \\ 0 \\ \vdots \\0 \end{array}\right]
\stackrel{F_2 \leftrightarrow F_1}{\longrightarrow}
\left[\begin{array}{c} 0\\ 1 \\ 0 \\ \vdots \\0 \end{array}\right]
\stackrel{F_2-F_1}{\longrightarrow}
\left[\begin{array}{c} 0\\ 1 \\ 0 \\ \vdots \\0 \end{array}\right]
\stackrel{F_1-F_2}{\longrightarrow} 
\left[\begin{array}{c} -1\\ 1 \\ 0 \\ \vdots \\0 \end{array}\right]
\end{align*}

Análogamente al inciso (a), el sistema ahora es:
\begin{equation*}
\begin{cases}
x_1 + (-1)x_3 + (-2)x_4 + \cdots + (-2014)x_{2016} = -1\\
x_2 + 2 x_3 + 3 x_4 + \cdots + 2015 x_{2016} = 1
\end{cases}
\quad \Rightarrow \quad
\begin{cases}
x_1 = -1 + \sum_{j=3}^{2016} (j-2) x_j \\
x_2 = 1 + \sum_{j=3}^{2016} (1-j) x_j
\end{cases}
\end{equation*}
Por lo tanto,  las soluciones del sistema son:

\

$\{ ( -1+\sum_{j=3}^{2016} (j-2) x_j , 1+\sum_{j=3}^{2016} (1-j) x_j , x_3 , x_4, \cdots, x_{2016}) : x_3,x_4,\cdots, x_{2016} \in \mathbb R \}$.

\

\end{enumerate}

\

     \item Sea $A=\begin{bmatrix}3 & -1 & 2 \\2 & 1 & 1 \\1&-3&0\end{bmatrix}$. Reduciendo $A$ por filas,
 \begin{enumerate}
   \item encontrar todas las soluciones sobre $\mathbb{R}$ y $\mathbb{C}$ del sistema $AX=0$.
   \item encontrar todas las soluciones sobre $\mathbb{R}$ y $\mathbb{C}$ del sistema $AX=\left[\begin{array}{c}
     1\\i\\0 \end{array}\right]$.
 \end{enumerate}
\vskip.2cm
\noindent\textsc{Solución:}

\begin{enumerate}
\item Reducimos la matriz A aplicando operaciones elementales por filas:

\begin{align*}
A = &\begin{bmatrix}3 & -1 & 2 \\2 & 1 & 1 \\1&-3&0\end{bmatrix}
\stackrel{F_1 \leftrightarrow F_3}{\longrightarrow}
\begin{bmatrix}1&-3&0\\2 & 1 & 1 \\3 & -1 & 2 \end{bmatrix}
\stackrel{F_2 - 2 F_1}{\stackrel{F_3 - 3 F_1}{\longrightarrow}}
\begin{bmatrix} 1 & -3 & 0 \\ 0 & 7 & 1 \\ 0 & 8 & 2 \end{bmatrix}
\stackrel{F_3-F_2}{\longrightarrow}
\begin{bmatrix} 1 & -3 & 0 \\ 0 & 7 & 1 \\ 0 & 1 & 1 \end{bmatrix}
\stackrel{F_3 \leftrightarrow F_2}{\longrightarrow} \\
&\begin{bmatrix} 1 & -3 & 0 \\ 0 & 1 & 1 \\ 0 & 7 & 1 \end{bmatrix}
\stackrel{F_1 + 3 F_2}{\stackrel{F_3-7F_2}{\longrightarrow}}
\begin{bmatrix} 1 & 0 & 3 \\ 0 & 1 & 1 \\ 0 & 0 & -6 \end{bmatrix}
\stackrel{F_3 (-\frac{1}{6}) }{\longrightarrow}
\begin{bmatrix} 1 & 0 & 3 \\ 0 & 1 & 1 \\ 0 & 0 & 1 \end{bmatrix}
\stackrel{F_1 - 3 F_3}{\stackrel{F_2 - F_3 }{\longrightarrow}}
\begin{bmatrix} 1 & 0 & 0 \\ 0 & 1 & 0 \\ 0 & 0 & 1 \end{bmatrix} = R_A
\end{align*}

Luego el sistema tiene solución trivial $(0,0,0)$ y la MERF es la matriz $\operatorname{Id}_3$. Notar que todas las operaciones realizadas valen tanto para $\mathbb{R}$ como para $\mathbb{C}$, por lo que $(0,0,0)$ es la solución para ambos casos.

\

\item Análogamente a lo realizado en el  ejercicio $(7.b)$, podemos repetir la secuencia de operaciones elementales sobre el vector $\left[\begin{array}{c}1\\i\\0 \end{array}\right]$ :

\begin{align*}
& \left[\begin{array}{c}1\\i\\0 \end{array}\right]
\stackrel{F_1 \leftrightarrow F_3}{\longrightarrow}
\left[\begin{array}{c}0\\i\\1 \end{array}\right]
\stackrel{F_2 - 2 F_1}{\stackrel{F_3 - 3 F_1}{\longrightarrow}}
\left[\begin{array}{c}0\\i\\1 \end{array}\right]
\stackrel{F_3-F_2}{\longrightarrow}
\left[\begin{array}{c}0\\i\\1-i \end{array}\right]
\stackrel{F_3 \leftrightarrow F_2}{\longrightarrow} 
\left[\begin{array}{c}0\\1-i\\i \end{array}\right]
\stackrel{F_1 + 3 F_2}{\stackrel{F_3-7F_2}{\longrightarrow}} \\
&\left[\begin{array}{c} 3 - 3i \\1-i\\ -7 + 8i \end{array}\right]
\stackrel{F_3 (-\frac{1}{6}) }{\longrightarrow}
\left[\begin{array}{c} 3 - 3i \\1-i\\ \frac{7}{6} - \frac{4}{3} i \end{array}\right]
\stackrel{F_1 - 3 F_3}{\stackrel{F_2 - F_3 }{\longrightarrow}}
\left[\begin{array}{c} -\frac{1}{2}+i \\ -\frac{1}{6}+\frac{1}{3}i \\ \frac{7}{6}-\frac{4}{3} i \end{array}\right]
\end{align*}

En este punto, si bien las operaciones propiamente dichas sólo involucraron números reales, tenemos que la solución tiene números complejos, por lo que el sistema no tiene solución en $\mathbb{R}$, pero si tiene solución en $\mathbb{C}$, y es $X = \left[\begin{array}{c} -\frac{1}{2}+i \\ -\frac{1}{6}+\frac{1}{3}i \\ \frac{7}{6}-\frac{4}{3} i \end{array}\right]$ 

\end{enumerate}

% \item >Para qué valores de $a$ el siguiente sistema tiene única o infinitas soluciones?
% \begin{align*}
% \left\{\begin{array}{l}ax-y+z=2 \\ x-y+z = 2  \\ 2x -2y + (2-a)z = 4a\end{array}\right.
% \end{align*}

\

\item Suponga que tiene que resolver un sistema de ecuaciones lineales homogéneo y que tras hacer algunas operaciones elementales por fila a la matriz asociada obtiene una matriz con la siguiente forma
\begin{align*}
\left(
\begin{array}{cccc}
a & * & * & *\\
0 & b & * & *\\
0 & 0 & c & *\\
0 & 0 & 0 & d
\end{array}
\right)
\end{align*}
donde $a,b,c,d\in\R$ y $*$ son algunos números reales.
?`Qué conclusiones puede inferir acerca del conjunto de soluciones a partir de los valores de $a$, $b$, $c$ y $d$?
\vskip.2cm
\noindent\textsc{Solución:}

Lo primero que podemos observar es que si $a$, $b$, $c$ y $d$ son todos no nulos entonces podemos aplicar las operaciones elementales por fila de multiplicar cada fila por $a^{-1}$, $b^{-1}$, $c^{-1}$ y $d^{-1}$. Luego de esto nos quedar\'ia una matriz con la siguiente forma
\begin{align*}
\left(
\begin{array}{cccc}
1 & * & * & *\\
0 & 1 & * & *\\
0 & 0 & 1 & *\\
0 & 0 & 0 & 1
\end{array}
\right).
\end{align*}
Luego podemos usando esos 1's principales podemos eliminar las entradas por encima de ellos obteniendo la matriz identindad
\begin{align*}
\left(
\begin{array}{cccc}
1 & 0 & 0 & 0\\
0 & 1 & 0 & 0\\
0 & 0 & 1 & 0\\
0 & 0 & 0 & 1
\end{array}
\right).
\end{align*}
En conclusión si $a$, $b$, $c$ y $d$ son todos no nulos podemos llegar mediante operaciones elementales por filas a la identidad y por lo tanto la \'unica soluci\'on del sistema es la trivial $(0,0,0,0)$, recordar el Teorema 2.4.5.

En cambio, si alguno de los escalares $a$, $b$, $c$ y $d$ es nulo, entonces no podemos obtener un 1 principal en su lugar. M\'as a\'un, la MERF a la que lleguemos tendr\'a una fila nula y por lo tanto el sistema tendr\'a infinitas soluciones, recordar el Teorema 2.4.2. 

Por ejemplo, si $d=0$ esto es claro pues la matriz ser\'ia
\begin{align*}
\left(
\begin{array}{cccc}
a & * & * & *\\
0 & b & * & *\\
0 & 0 & c & *\\
0 & 0 & 0 & 0
\end{array}
\right).
\end{align*}
Si $c=0$ y $d\neq0$, entonces la matriz es 
\begin{align*}
\left(
\begin{array}{cccc}
a & * & * & *\\
0 & b & * & *\\
0 & 0 & 0 & *\\
0 & 0 & 0 & d
\end{array}
\right).
\end{align*}
Luego, podemos multiplicar por $d^{-1}$ la \'ultima fila y luego anular la entrada por arriba del 1 que nos quede y as\'i obtener la matriz
\begin{align*}
\left(
\begin{array}{cccc}
a & * & * & *\\
0 & b & * & *\\
0 & 0 & 0 & 0\\
0 & 0 & 0 & 1
\end{array}
\right).
\end{align*}
Un razonamiento similar podr\'iamos hacer con las dem\'as posibilidades.

Moraleja: para saber si un sistema homog\'eneo tiene una o infinitas soluciones no es necesario reducir la matriz hasta llegar a una MERF basta con llegar a una triangular superior. Pero para calcular de forma param\'etrica el conjunto de soluciones si es necesario llegar a una MERF.

\

\item Suponga que tiene que resolver un sistema de ecuaciones lineales y que tras hacer algunas operaciones elementales por fila a la matriz ampliada obtiene una matriz con la siguiente forma
\begin{align*}
\left(
\begin{array}{cccc|c}
a & * & * & * & *\\
0 & b & * & * & *\\
0 & 0 & 0 & 0 & c\\
0 & 0 & 0 & d & *
\end{array}
\right)
\end{align*}
donde $a,b,c,d\in\R$ y $*$ son algunos números reales.
?`Qué conclusiones puede inferir acerca del conjunto de soluciones a partir de los valores de $a$, $b$, $c$ y $d$?
\vskip.2cm
\noindent\textsc{Solución:}

Lo primero que podemos notar es que si $c$ es no nulo el sistema no tiene soluci\'on. Pues sería equivalente a un sistema cuya ecuaci\'on $0=c$ es falsa.

Asumamos ahora que $c=0$. Si $a$, $b$ y $d$ son no nulos, entonces como antes podemos simplificarlos aplicando la operaci\'on elemental multiplicar la respectiva fila por $a^{-1}$, $b^{-1}$ y $d^{-1}$. Luego intercambiar la tercer y cuarta fila para obtener la matriz
\begin{align*}
\left(
\begin{array}{cccc|c}
1 & * & * & * & *\\
0 & 1 & * & * & *\\
0 & 0 & 0 & 1 & *\\
0 & 0 & 0 & 0 & 0
\end{array}
\right)
\end{align*}
Razonando como en el ejercicio anterior podemos transforma la matriz en una MERF que va a tener una fila nula. Adem\'as, este sistema tiene una soluci\'on. En efecto, para fijar ideas supongamos que $z_1$, $z_2$ y $z_3$ son las entradas de la \'ultima columna de la matriz ampliada entonces $(z_1,z_2,0,z_3)$ es una soluci\'on. Por el Teorema 2.4.2, en este caso el sistema tiene infinitas soluciones.

Hay otros varios casos para analizar de manera similar. 

Moraleja: al igual que antes no es necesario llegar a una MERF para saber si el sistema tendr\'a o no soluci\'on, una o infinitas. Pero para calcular de forma param\'etrica el conjunto de soluciones si es necesario llegar a una MERF.

\

\item Suponga que tiene que resolver un sistema de $m$ ecuaciones lineales con $n$ incógnitas. Antes de empezar a hacer cuentas y apelando a la teoría, ?`Qué puede afirmar acerca del conjunto de soluciones en base a $m$ y $n$? ?`Cómo saber si es vacío o no vacío? ?`Si tiene una o varias soluciones?
\vskip.2cm
\noindent\textsc{Solución:}

No hay una respuesta concluyente a este ejercicio pero nos sirve para pensar un poco y repasar la teor\'ia. Algunos razonamientos que podemos hacer son los siguientes.

Si el sistema tiene menos inc\'ognitas que ecuaciones hay chances de que no tenga soluci\'on. En cierto sentido, cada ecuaci\'on es una condici\'on para el conjunto de soluciones y entonces podr\'ia ser que estemos poniendo demasiadas condiciones y que sean contradictorias entre ellas y as\'i no habr\'ia una soluci\'on com\'un a todas (ver la página 29 de la Clase 08 Teórica - Sistemas de ecuaciones 3 (17-09-20) del turno ma\~nana).

Si el sistema tiene m\'as inc\'ognitas que ecuaciones y el sistema tiene soluci\'on entonces el sistema tiene infinitas soluciones. Esto es porque hay inc\'ognitas que no van a ser 1 principal y entonces ser\'ian variables libres (ver la p\'agina 35 de la Clase 08 Teórica - Sistemas de ecuaciones 3 (17-09-20) del turno ma\~nana).

\

\item\label{polinomios} $\textcircled{a}$ Sean $\lambda_1, ..., \lambda_n\in\R$ y $b_1, ..., b_n\in\R$.

\

\begin{enumerate}
 \item Para cada $n\in\{1,2,3,4,5\}$, plantear un sistema de ecuaciones lineales que le permita encontrar un polinomio $p(x)$ con coeficientes reales de grado $n-1$ tal que
 $$
 p(\lambda_1)=b_1, \dots, p(\lambda_n)=b_n.
 $$

 \

\item ?`Se le ocurre alguna condición con la cual pueda afirmar que el sistema anterior no tiene solución?

\

\item  ?`Puede dar una forma general del sistema para cualquier $n$?

\end{enumerate}


\vskip.2cm
\noindent\textsc{Solución:}
 (a) En el ejercicio (\ref{polinomio}) hicimos  $n=3$   para un caso concreto ($p(1)=2$, $p(2) = 7$ y $p(3)=14$). como en ese caso,  una forma de resolver el problema es plantear un sistema de ecuaciones donde los coeficientes del polinomio sean la incógnitas. 
 Sea 
 $$p_n(x) = a_0 + a_1x + a_2x^2 + \cdots a_{n-1}x^{n-1}, $$
 entonces, $p_n(\lambda_i) = b_i$ se traduce en la ecuación
 \begin{equation*}\label{eq-fila-vd}
 a_0 + a_1\lambda_i + a_2\lambda_i^2 + \cdots a_{n-1}\lambda_i^{n-1} = b_i.    
 \end{equation*}
 Las matrices ampliadas de los sistemas de ecuaciones para $n=4$ y $n=5$, son
 \begin{equation*}
     \left[\begin{array}{cccc|c}
     1 &\lambda_1&\lambda_1^2&\lambda_1^3 &b_1 \\
     1 &\lambda_2&\lambda_2^2&\lambda_2^3 &b_2 \\
     1 &\lambda_3&\lambda_3^2&\lambda_3^3 &b_3 \\
     1 &\lambda_4&\lambda_4^2&\lambda_4^3 &b_4
     \end{array}\right], \qquad
     \left[\begin{array}{ccccc|c}
     1 &\lambda_1&\lambda_1^2&\lambda_1^3 &\lambda_1^4 &b_1 \\
     1 &\lambda_2&\lambda_2^2&\lambda_2^3&\lambda_2^4  &b_2 \\
     1 &\lambda_3&\lambda_3^2&\lambda_3^3 &\lambda_3^4 &b_3 \\
     1 &\lambda_4&\lambda_4^2&\lambda_4^3&\lambda_4^4  &b_4\\
     1 &\lambda_5&\lambda_5^2&\lambda_5^3&\lambda_5^4  &b_5
     \end{array}\right], 
 \end{equation*}
 respectivamente. Para $n=1,2,3$ es claro como son los sistemas.
 
 \vskip .2cm
 
 (b) Si sobrentendemos  que todos los $\lambda_i$ son distintos entre si, la respuesta es \textit{no.} 
 \vskip .2cm
 Obviamente si $\lambda_i = \lambda_j$ y $b_i \ne b_j$,  entonces $p(\lambda_i) = b_i \ne b_j = p(\lambda_j) = p(\lambda_i)$,  es decir llegamos a la conclusión que $p(\lambda_i)  \ne p(\lambda_i)$, lo cual es absurdo. 
 
 (Veremos más adelante, usando determinantes,  que si $\lambda_i \ne \lambda_j$ para $i \ne j$,  entonces siempre encontraremos un polinomio que satisfaga las condiciones del ejercicio). 
  \vskip .2cm
 (c) No es difícil generalizar (a) para cualquier $n$: la matriz ampliada del sistema de ecuaciones  correspondiente al caso  $n$ es
 $$[V|Y] =
 \left[\begin{array}{ccccc|c}
     1 &\lambda_1&\lambda_1^2&\cdots &\lambda_1^{n-1} &b_1 \\
     1 &\lambda_2&\lambda_2^2&\cdots&\lambda_2^{n-1}  &b_2 \\
     \vdots & \vdots& \vdots& & \vdots & \vdots \\
    \vdots & \vdots& \vdots& & \vdots & \vdots \\
     1 &\lambda_n&\lambda_n^2&\cdots&\lambda_n^{n-1}  &b_n
     \end{array}\right].
 $$
 (La matriz $V$ es  llamada \textit{la matriz de Vandermonde.})
 
\end{enumerate}



\end{document}

%%%=======================
%%%=======================
%%% CAP3 =================
    
\chapter{Soluciones\\Álgebra  II -- Año 2024/1 -- FAMAF}\label{practico-3}

\begin{enumerate}[topsep=6pt,itemsep=.4cm]

%%%%%%%%%%%%%%%%%%%%%%%%%%%%%%%%%%%%%%%%%%%%%%%%%%%%%%%%%%%%%%%%%%%%%%%%%%%%%%%%%%%%%%%%%%%%%%%%%%%%%%%%%%%%%%%%%%%%%%%%%%%%%%%%%%%%%%%%%%%%%%%%%%%%%%%%

\item\label{ej} Sean
$$
A= \begin{bmatrix} 1&-2&0\\ 1&-2&1\\ 1&-2&-1\end{bmatrix},\quad
\quad B= \begin{bmatrix}1&1&2\\ -2&0&-1\\ 1&3&5 \end{bmatrix},
\quad\quad C=\begin{bmatrix}1&-1&1\\ 2&0&1\\ 3&0&1 \end{bmatrix}.
$$
Verificar que $A(BC)=(AB)C$, es decir que vale la asociatividad del producto.
\rta
\begin{equation*}
    BC =  \begin{bmatrix}1&1&2\\ -2&0&-1\\ 1&3&5 \end{bmatrix} \begin{bmatrix}1&-1&1\\ 2&0&1\\ 3&0&1 \end{bmatrix}=\begin{bmatrix}9 & -1 & 4 \\-5 & 2 & -3 \\22 & -1 & 9 \end{bmatrix}
\end{equation*}
\begin{equation*}
    A(BC) =\begin{bmatrix} 1&-2&0\\ 1&-2&1\\ 1&-2&-1\end{bmatrix}\begin{bmatrix}9 & -1 & 4 \\-5 & 2 & -3 \\22 & -1 & 9 \end{bmatrix} =
    \begin{bmatrix}19 & -5 & 10 \\41 & -6 & 19 \\-3 & -4 & 1\end{bmatrix} \tag{*}
\end{equation*}
\begin{equation*}
    AB = \begin{bmatrix} 1&-2&0\\ 1&-2&1\\ 1&-2&-1\end{bmatrix} \begin{bmatrix}1&1&2\\ -2&0&-1\\ 1&3&5 \end{bmatrix} =
    \begin{bmatrix}5 &1 &4\\ 6&4&9\\ 4&-2&-1 \end{bmatrix}
\end{equation*}
\begin{equation*}
    (AB)C = \begin{bmatrix}5 &1 &4\\ 6&4&9\\ 4&-2&-1 \end{bmatrix}\begin{bmatrix}1&-1&1\\ 2&0&1\\ 3&0&1 \end{bmatrix} =
    \begin{bmatrix}19 & -5 & 10 \\41 & -6 & 19 \\-3 & -4 & 1\end{bmatrix} \tag{**}
\end{equation*}
Luego (*) = (**) y el resultado queda verificado. \qed


\item\label{ej2} Determinar cuál de las siguientes matrices es $A$, cuál es $B$ y cuál es $C$ de modo tal que sea posible realizar el producto $ABC$ y verificar que $A(BC)=(AB)C$.
\begin{equation*}
\begin{bmatrix} 2 & -1 & 1 \\ 1 & 2 &
1\end{bmatrix},\qquad
\begin{bmatrix} 3 \\ 1 \\ -1\end{bmatrix}, \qquad
\begin{bmatrix} 1 & -1 \end{bmatrix}.
\end{equation*}
\rta la  primera matriz es $2 \times 3$, la segunda es $3 \times 1$ y la tercera es $1 \times 2$. Entonces, podemos multiplicar la 1º por  la 2º y  queda una matrix $2 \times 1$ que es posible multiplicar por la 3º matriz,  que es $1 \times 2$,  y así obtenemos una matriz $2 \times 2$.  Es decir,
\begin{equation*}
    A=\begin{bmatrix} 2 & -1 & 1 \\ 1 & 2 &
    1\end{bmatrix},\qquad
    B= \begin{bmatrix} 3 \\ 1 \\ -1\end{bmatrix}, \qquad
    C= \begin{bmatrix} 1 & -1 \end{bmatrix}.
\end{equation*}
Ahora bien, 
\begin{equation*}
BC= \begin{bmatrix} 3 \\ 1 \\ -1\end{bmatrix}\begin{bmatrix} 1 & -1 \end{bmatrix} =  \begin{bmatrix} 3& -3\\ 1&-1 \\-1 &1\end{bmatrix}
\end{equation*}
\begin{equation*}
    A(BC)=\begin{bmatrix} 2 & -1 & 1 \\ 1 & 2 &    1\end{bmatrix}\begin{bmatrix} 3& -3\\ 1&-1 \\-1 &1\end{bmatrix} =  \begin{bmatrix} 4&-4 \\ 4&-4 \end{bmatrix}
\end{equation*}
\begin{equation*}
    AB=\begin{bmatrix} 2 & -1 & 1 \\ 1 & 2 &
    1\end{bmatrix}\begin{bmatrix} 3 \\ 1 \\ -1\end{bmatrix} = \begin{bmatrix} 4 \\ 4 \end{bmatrix}.
\end{equation*}
\begin{equation*}
    (AB)C= \begin{bmatrix} 4 \\ 4 \end{bmatrix}\begin{bmatrix} 1 & -1 \end{bmatrix} = \begin{bmatrix} 4&-4 \\ 4&-4 \end{bmatrix}.
\end{equation*}\qed


\item Calcular $A^2$ y $A^3$ para la matriz \
$
A=\begin{bmatrix}
3 & 4\\ 6 & 8
\end{bmatrix}.
$
\rta
\begin{align*}
    A^2 &= \begin{bmatrix}
        3 & 4\\ 6 & 8
        \end{bmatrix}\begin{bmatrix}
            3 & 4\\ 6 & 8
            \end{bmatrix} = \begin{bmatrix}
                33 & 44\\ 66 & 88
                \end{bmatrix}  = 11\begin{bmatrix}
                    3 & 4\\ 6 & 8
                    \end{bmatrix} = 11A\\
    A^3 &= A\cdot A^2 = A\cdot 11A = 11A^2 = 11^2A. 
\end{align*}

\noindent \textbf{Observación.} Este es un caso muy particular. En  el caso de una matriz escalar $c\operatorname{Id}$,  tenemos que  $(c\operatorname{Id})^n= c^n \operatorname{Id}$. Aquí tenemos una matriz $A$ tal que $A^n = 11^{n-1}A$.

\qed



\item\label{ejemplos 2x2}  Dar ejemplos de matrices no nulas $A$ y $B$ de orden $2\times2$ tales que
\begin{multicols}{2}
\begin{enumerate}[topsep=5pt,itemsep=5pt]
 \item\label{ejemplos 2x2-a} $A^2=0$ (dar dos ejemplos).
 \item\label{ejemplos 2x2-b} $AB\neq BA$.
 \item\label{ejemplos 2x2-c} $A^2=-\operatorname{I}_2$.
 \item\label{ejemplos 2x2-d} $A^2=A\neq\operatorname{I}_2$.
\end{enumerate}
\end{multicols}
\rta
 
\ref{ejemplos 2x2-a}
\begin{equation*}
    A = \begin{bmatrix}
        0 & 1\\ 0 & 0
        \end{bmatrix}, \qquad A = \begin{bmatrix}
            0 & 0\\ 1 & 0
            \end{bmatrix}.
\end{equation*}

\ref{ejemplos 2x2-b}
 \begin{equation*}
    A = \begin{bmatrix}
        0 & 1\\ 0 & 0
        \end{bmatrix}, \qquad B = 
        \begin{bmatrix}
        0 & 0\\ 1 & 0
        \end{bmatrix}.
\end{equation*}
\begin{align*}
    AB &= \begin{bmatrix}
        0 & 1\\ 0 & 0
        \end{bmatrix} \begin{bmatrix}
            0 & 0\\ 1 & 0
            \end{bmatrix} = \begin{bmatrix}
                1 & 0\\ 0 & 0
                \end{bmatrix} \\
    BA &=  
        \begin{bmatrix}
        0 & 0\\ 1 & 0
        \end{bmatrix}
        \begin{bmatrix}
        0 & 1\\ 0 & 0
        \end{bmatrix} = \begin{bmatrix}
        0 & 0\\ 0 & 1
        \end{bmatrix} 
\end{align*}

\ref{ejemplos 2x2-c}
 \begin{equation*}
    A = \begin{bmatrix}
        0 & 1\\ -1 & 0
        \end{bmatrix}, \qquad \Rightarrow \qquad A^2 = 
        \begin{bmatrix}
            0 & 1\\ -1 & 0
            \end{bmatrix}
            \begin{bmatrix}
                0 & 1\\ -1 & 0
                \end{bmatrix}
            = \begin{bmatrix}
                -1 & 0\\ 0 & -1
                \end{bmatrix}.
\end{equation*}

\ref{ejemplos 2x2-d}
\begin{equation*}
    A = \begin{bmatrix}
        1 & 0\\ 0 & 0
        \end{bmatrix}.
\end{equation*}
\qed


\item\label{2x2 central}  Sea $A \in\mathbb{R}^{2\times 2}$ tal que $AB=BA$ para toda $B\in\mathbb{R}^{2\times 2}$. Probar que $A$ es un múltiplo de $\operatorname{I}_2$.
\rta Sea
\begin{equation*}
    A = \begin{bmatrix} a_{11} & a_{12}\\ a_{21} & a_{22} \end{bmatrix},
\end{equation*}
y sean 
\begin{equation*}
    E_{11} = \begin{bmatrix} 1 & 0\\ 0 & 0\end{bmatrix},
    \quad
    E_{12} = \begin{bmatrix} 0 & 1\\ 0 & 0\end{bmatrix},
    \quad
    E_{21} = \begin{bmatrix} 0 & 0\\ 1 & 0\end{bmatrix},
    \quad
    E_{22} = \begin{bmatrix} 0 & 0\\ 0 & 1\end{bmatrix}. 
\end{equation*}
Entonces 
\begin{align*}
    AE_{11} &= \begin{bmatrix} a_{11} & a_{12}\\ a_{21} & a_{22} \end{bmatrix} \begin{bmatrix} 1 & 0\\ 0 & 0\end{bmatrix} =
    \begin{bmatrix} a_{11} &0\\ a_{21} & 0\end{bmatrix}, \\
    E_{11}A &=  \begin{bmatrix} 1 & 0\\ 0 & 0\end{bmatrix} \begin{bmatrix} a_{11} & a_{12}\\ a_{21} & a_{22} \end{bmatrix}=
    \begin{bmatrix} a_{11} & a_{12}\\ 0 & 0 \end{bmatrix}.
\end{align*}
Como $AE_{11} = E_{11}A$, tenemos que $a_{12} = a_{21} =0$.

Probemos ahora con $E_{12}$:
\begin{align*}
    AE_{12} &= \begin{bmatrix} a_{11} & a_{12}\\ a_{21} & a_{22} \end{bmatrix} \begin{bmatrix} 0 & 1\\ 0 & 0\end{bmatrix} =
    \begin{bmatrix} 0 &a_{11}\\ 0 & a_{21}\end{bmatrix}, \\
    E_{12}A &=  \begin{bmatrix} 0 & 1\\ 0 & 0\end{bmatrix} \begin{bmatrix} a_{11} & a_{12}\\ a_{21} & a_{22} \end{bmatrix}=
    \begin{bmatrix} a_{21} & a_{22}\\ 0 & 0 \end{bmatrix}.
\end{align*}
Como $AE_{12} = E_{12}A$, tenemos que $a_{11} = a_{22}$. 

Ya no hace falta hacer más ensayos, pues hemos probado que $a_{12} = a_{21} =0$ y $a_{11} = a_{22}$, es decir $A$ es una matriz escalar y  al ser escalar sabemos que conmuta con todas las matrices. 

\vskip .2cm
\noindent \textbf{Observación.} El resultado es cierto también para matrices $n \times  n$ con $n \in \mathbb N$. Es decir,  si $A \in\mathbb{R}^{n\times n}$ tal que $AB=BA$ para toda $B\in\mathbb{R}^{n\times n}$,  entonces  $A$ es un múltiplo de $\operatorname{I}_n$. La estrategia para probar este resultado es la misma que para el caso $2 \times 2$: probar con las matrices $E_{ij}$ que son aquellas que tiene un 1 en la entrada $ij$ y  0  en las demás entradas. 

\qed

\item  Para cada $n\in\mathbb{N}$, con $n\geq 2$, hallar una matriz no nula $A\in\mathbb{R}^{n\times n}$ tal que $A^n=0$ pero $A^{n-1}\neq0$.
\rta las matrices  triangulares estrictas, superiores o inferiores, satisfacen la propiedad de que $A^n=0$ y  muchas de ellas satisfacen que $A^{n-1} \ne 0$. Probemos con una matriz triangular superior estricta particular 
\begin{equation*}
    A = \begin{bmatrix}
        0&1&0&0&\ldots&0 \\
        0&0&1&0&\ldots&0 \\
        0&0&0&1&\ldots&0 \\
        \vdots&&&&\ddots&\vdots \\
        0&0&0&0&\ldots &1 \\
        0&0&0&0&\ldots &0  
    \end{bmatrix} = \begin{bmatrix}
        e_2 \\ e_3 \\ e_4 \\ \vdots \\ e_n \\0
    \end{bmatrix} = \begin{bmatrix} \mid& \mid& \mid& &\mid\\ 0 & e_1 & e_2 & \cdots &e_{n-1}\\ \mid&\mid & \mid& &\mid\end{bmatrix}
\end{equation*}
Es decir $A$  es una matriz $n \times n$ con $1$ encima de la diagonal y $0$  en todas las demás entradas. Más formalmente: $[A]_{i,i+1} = 1$ y $[A]_{i,j} = 0$ si $j \ne i+1$. 
\vskip .2cm
Probaremos por inducción que para $k< n$,  $[A^k]_{i,i+k} = 1$ y $[A]_{i,j} = 0$ si $j \ne i+k$. Es decir,
\begin{equation*}
    A^k  = \begin{bmatrix} 
        \mid& \cdots & \mid& \mid& \mid&   &\mid\\ 
        0 & \cdots & 0 & e_{1}& e_2& \cdots &e_{n-k}\\ 
        \mid& \cdots& \mid& \mid& \mid&   &\mid
    \end{bmatrix},
\end{equation*}
donde $e_1$  está en la columna $k+1$. 


Si $k=1$ el resultado vale por definición de $A$. Supongamos que el resultado es cierto para $k-1$,   luego \begin{equation*}
    A^{k-1}  = \begin{bmatrix} 
        \mid& \cdots & \mid& \mid& \mid&   &\mid\\ 
        0 & \cdots & 0 & e_{1}& e_2& \cdots &e_{n-k+1}\\ 
        \mid& \cdots& \mid& \mid& \mid&   &\mid
    \end{bmatrix},\tag{HI}
\end{equation*}
donde $e_1$  está en la columna $k$. Por lo tanto,
\begin{equation*}
    A \cdot A^{k-1}  = \begin{bmatrix}
        e_2 \\ e_3 \\ e_4 \\ \vdots \\ e_n \\0
    \end{bmatrix} \begin{bmatrix} 
        \mid& \cdots & \mid& \mid& \mid&   &\mid\\ 
        0 & \cdots & 0 & e_{1}& e_2& \cdots &e_{n-k+1}\\ 
        \mid& \cdots& \mid& \mid& \mid&   &\mid
    \end{bmatrix},
\end{equation*}
Como $[A\cdot A^{k-1}]_{ij} = F_i(A) \cdot C_j(A^{k-1})$ (donde $\cdot$ indica el producto escalar), los únicos productos no nulos son $ F_1(A) \cdot C_{k+1}(A^{k-1}) = 1$,  $ F_2(A) \cdot C_{k +2}(A^{k-1}) = 1$, $ F_3(A) \cdot C_{k +3}(A^{k-1}) = 1$, etc. Es decir, las entradas de $A^k$ valen $1$ en $(1,k+1)$, $(2,k+2)$, $(3,k+3)$, etc. lo cual prueba el resultado. 


Probado esto, tenemos $A^{n-1} = [0 \;\cdots\; 0\; e_1] \ne 0$ y $A^n = A \cdot A^{n-1}=0$. 

\qed


\item\label{eq:binomio}  Dar condiciones necesarias y suficientes sobre matrices $A$ y $B$ de tamaño $n\times n$ para que
\begin{multicols}{2}
    \begin{enumerate}
        \item\label{ej-p3-8-a} $(A + B)^2 = A^2 + 2AB + B^2$.
        \item\label{ej-p3-8-b} $A^2 - B^2 = (A - B)(A + B)$.
    \end{enumerate}
\end{multicols}
\rta

\ref{ej-p3-8-a} Como
\begin{align*}
&(A + B)^2 = (A+B)(A+B) = AA +AB + BA +BB = A^2 + AB +BA + B^2  \; \text{ y}\\
&A^2 + 2AB + B^2 = A^2 + AB + AB + B^2, 
\end{align*}
tenemos que 
\begin{align*}
    (A + B)^2 = A^2 + 2AB + B^2 \quad &\Leftrightarrow \quad  A^2 + AB +BA + B^2  = A^2 +  AB + AB + B^2 \\
    &\Leftrightarrow \quad AB + BA = AB + AB \\
    &\Leftrightarrow \quad BA = AB.
\end{align*}

\vskip .2cm

\ref{ej-p3-8-b} Como
\begin{align*}
    &A^2 - B^2  = AA -BB& \text{ y} &\qquad\qquad \\
    &(A - B)(A + B) = AA +AB -BA -BB,&& 
\end{align*}
tenemos que 
\begin{align*}
    A^2 - B^2 = (A - B)(A + B) \quad &\Leftrightarrow \quad AA -BB  = AA +AB -BA -BB \\
    &\Leftrightarrow \quad 0 = AB -BA \\
    &\Leftrightarrow \quad BA = AB.
\end{align*}
\qed

\item\label{ej:multiplicar por columna}  Sean
\begin{align*}
v=\begin{bmatrix} v_1 \\ \vdots \\ v_n
\end{bmatrix}\in\mathbb{R}^{n\times1}
\quad\mbox{y}\quad A=\begin{bmatrix} \mid& \mid& &\mid\\ C_1 & C_2 & \cdots &C_n\\ \mid& \mid& &\mid\end{bmatrix}
\in\mathbb{R}^{m\times n},
\end{align*}
es decir, $C_1, ..., C_n$ denotan las columnas de $A$. Probar que $Av=\sum_{j=1}^nv_jC_j$.
\rta sea 
\begin{equation*}
    C_j = \begin{bmatrix}
        c_{1j} \\c_{2j} \\ \vdots \\ c_{mj} 
    \end{bmatrix}, \quad \text{para $1 \le j \le n$}.
\end{equation*}
$Av$  es una matriz  $m\times 1$ cuya coordenada $i,1$  es $[Av]_{i1} = F_i(C) \cdot v$ donde $\cdot$  es el producto escalar. Es  decir 
\begin{equation*}
    [Av]_{i1} = \sum_{j = 1}^n c_{ij}v_j, \qquad (1 \le i \le m).
\end{equation*}
Por lo tanto
\begin{equation*}
 Av = \begin{bmatrix}
    \sum_{j = 1}^n c_{1j}v_j \\ \vdots \\ \sum_{j = 1}^n c_{ij}v_j \\ \vdots \\ \sum_{j = 1}^n c_{mj}v_j
 \end{bmatrix} =
 \sum_{j = 1}^n
 \begin{bmatrix}
     c_{1j}v_j \\ \vdots \\  c_{ij}v_j \\ \vdots \\ c_{mj}v_j
 \end{bmatrix} =
 \sum_{j = 1}^n v_j
 \begin{bmatrix}
     c_{1j}\\ \vdots \\  c_{ij} \\ \vdots \\ c_{mj}
 \end{bmatrix} =
 \sum_{j = 1}^n v_j C_j.
\end{equation*}
\qed



\item\label{traza} Si $A$ es una matriz cuadrada $n\times n$, se define la {\textit{traza}} de $A$ 
como $\operatorname{Tr}(A)=\displaystyle{\sum_{i=1}^n} a_{ii}$.
\begin{enumerate}[topsep=5pt,itemsep=5pt]
 \item Calcular la traza de las matrices del ejercicio  \ref{ej:inversas}. 
 \item\label{ej:traza} Probar que si $A,B\in\mathbb{R}^{n\times n}$ y $c\in\mathbb{R}$ entonces
 \begin{align*}
 \operatorname{Tr}(A+cB)=\operatorname{Tr}(A)+c\operatorname{Tr}(B)
 \quad\mbox{y}\quad
 \operatorname{Tr}(AB)=\operatorname{Tr}(BA).
 \end{align*}
\end{enumerate}
\vskip .2cm
\noindent\textsc{Solución:}
 (a) \begin{align*}
    A&= \begin{bmatrix*}[r] 3 & -1 & 2 \\ 2 & 1 & 1 \\ 1 & -3 & 0\end{bmatrix*} &\Rightarrow& \qquad \operatorname{Tr}(A)= 3+1+0 =4, \\
    B &= \begin{bmatrix*}[r] -1 & -1 &4 \\ 1 & 3 & 8 \\ 1 & 2 & 5\end{bmatrix*} &\Rightarrow& \qquad \operatorname{Tr}(B)=\ -1 +3 + 5 = 7,\\
    C &= \begin{bmatrix*}[r] 1 & 1 & 1 & 2 \\ 1 & -3 & 3 & -8 \\ -2 & 1 & 2 & -2 \\ 1 & 2 & 1 & 4 \end{bmatrix*} &\Rightarrow& \qquad \operatorname{Tr}(C)= 1 +(-3)+2+4 = 4,\\
    D &= \begin{bmatrix*}[r] 1 & -3 & 5 \\ 2 & -3 & 1 \\ 0 & -1 & 3 \end{bmatrix*}&\Rightarrow& \qquad \operatorname{Tr}(D)=1 + (-3) + 3 = 1.
    \end{align*}

(b) Sea $A = [a_{ij}]$ y $B = [b_{ij}]$,  entonces
\begin{align*}
    \operatorname{Tr}(A+cB)&=   \sum_{i=1}^n [A+cB]_{ii}\\
    &= \sum_{i=1}^n (a_{ii} + c b_{ii}) = \sum_{i=1}^n a_{ii} + c \sum_{i=1}^n  b_{ii}\\
    &=\operatorname{Tr}(A)+c\operatorname{Tr}(B)
\end{align*}

Veamos la segunda afirmación de (b):
\begin{align*}
    \operatorname{Tr}(AB)&=   \sum_{i=1}^n [AB]_{ii} = \sum_{i=1}^n (\sum_{j=1}^n  a_{ij}b_{ji}) \\
    &= \sum_{i,j=1}^n  a_{ij}b_{ji} = \sum_{j=1}^n (\sum_{i=1}^n  a_{ij}b_{ji}) \\
    &=  \sum_{j=1}^n (\sum_{i=1}^n  b_{ji}a_{ij}) =  \sum_{j=1}^n [BA]_{jj} \\
    &= \operatorname{Tr}(BA).
\end{align*}

\qed 

\item\label{ej:inversas} Para cada una de las siguientes matrices, usar operaciones elementales por fila para decidir si son invertibles y hallar la matriz inversa cuando sea posible.
\begin{equation*}
\begin{bmatrix} 3 & -1 & 2 \\ 2 & 1 & 1 \\ 1 & -3 & 0\end{bmatrix},\qquad
\begin{bmatrix} -1 & -1 &4 \\ 1 & 3 & 8 \\ 1 & 2 & 5\end{bmatrix},\qquad
\begin{bmatrix} 1 & 1 & 1 & 2 \\ 1 & -3 & 3 & -8 \\ -2 & 1 & 2 & -2 \\ 1 & 2 & 1 & 4 \end{bmatrix},\qquad
\begin{bmatrix} 1 & -3 & 5 \\ 2 & -3 & 1 \\ 0 & -1 & 3 \end{bmatrix}.
\end{equation*}
(para que hagan menos cuentas: las matrices $3\times3$ aparecieron en el Práctico \ref{practico-2}).
\rta

\begin{align*}
    [A|\operatornamewithlimits{Id}] &= 
    \begin{bmatrix}3 & -1 & 2&\bigm| &1 &0 & 0\\2 & 1 & 1&\bigm|& 0 &1 &0 \\1&-3&0&\bigm| &0 &0 &1\end{bmatrix}
    \stackrel{F_1 \leftrightarrow F_3}{\longrightarrow}
    \begin{bmatrix}1&-3&0&\bigm| &0 &0 &1\\2 & 1 & 1&\bigm|& 0 &1 &0 \\3 & -1 & 2&\bigm| &1 &0 & 0\end{bmatrix} \\
    &\stackrel{F_2 - 2 F_1}{\stackrel{F_3 - 3 F_1}{\longrightarrow}}
    \begin{bmatrix} 1 & -3 & 0  &\bigm|&0 &0 &1\\ 0 & 7 & 1 &\bigm|& 0 &1 &-2 \\ 0 & 8 & 2 &\bigm| &1 &0 & -3\end{bmatrix}
    \stackrel{F_3-F_2}{\longrightarrow}
    \begin{bmatrix} 1 & -3 & 0  &\bigm|&0 &0 &1\\ 0 & 7 & 1 &\bigm|& 0 &1 &-2 \\  0 & 1 & 1  &\bigm| &1 &-1 & -1\end{bmatrix}\\
    &\stackrel{F_3 \leftrightarrow F_2}{\longrightarrow} 
    \begin{bmatrix} 1 & -3 & 0  &\bigm|&0 &0 &1\\  0 & 1 & 1  &\bigm| &1 &-1 & -1\\ 0 & 7 & 1 &\bigm|& 0 &1 &-2 \end{bmatrix}
    \stackrel{F_1 + 3 F_2}{\stackrel{F_3-7F_2}{\longrightarrow}}
    \begin{bmatrix} 1 & 0 & 3 &\bigm|&3 & -3 &-2\\ 0 & 1 & 1 &\bigm| &1 &-1 & -1\\ 0 & 0 & -6 &\bigm|&-7&8&5\end{bmatrix} \\
    &\stackrel{F_3 (-\frac{1}{6}) }{\longrightarrow}
    \begin{bmatrix} 1 & 0 & 3 &\bigm|&3 & -3 &-2\\ 0 & 1 & 1 &\bigm| &1 &-1 & -1\\ 0 & 0 & 1 &\bigm|&7/6&-4/3&-5/6\end{bmatrix} \\
    &\stackrel{F_1 - 3 F_3}{\stackrel{F_2 - F_3 }{\longrightarrow}}
    \begin{bmatrix} 1 & 0 & 0&\bigm|& -1/2&1&1/2\\ 0 & 1 & 0&\bigm|&-1/6 &1/3&-1/6\\ 0 & 0 & 1&\bigm|&7/6&-4/3&-5/6 \end{bmatrix}
    \end{align*}
    Luego  
    \begin{equation*}
        \begin{bmatrix} 3 & -1 & 2 \\ 2 & 1 & 1 \\ 1 & -3 & 0\end{bmatrix}^{-1} = 
        \begin{bmatrix}-1/2&1&1/2\\ -1/6 &1/3&-1/6\\7/6&-4/3&-5/6 \end{bmatrix}. 
    \end{equation*}
\vskip .4cm
Siguiendo un procedimiento análogo al anterior se obtiene
\begin{align*}
    \begin{bmatrix} -1 & -1 &4 \\ 1 & 3 & 8 \\ 1 & 2 & 5\end{bmatrix}^{-1} = \begin{bmatrix} 1/6& -13/6& 10/3\\-1/2& 3/2& -2\\1/6& -1/6& 1/3\end{bmatrix}.
    \end{align*}

    \vskip .4cm
La tercera matriz tiene una MERF con una fila nula. Por lo tanto no es invertible:
\begin{align*}
    &\begin{bmatrix} 1 & 1 & 1 & 2 \\ 1 & -3 & 3 & -8 \\ -2 & 1 & 2 & -2 \\ 1 & 2 & 1 & 4 \end{bmatrix} 
    \stackrel{F_2 -  F_1}{\stackrel{F_3+2F_1}{\stackrel{F_4 - F_1 }{\longrightarrow}}}
    \begin{bmatrix} 1 & 1 & 1 & 2 \\ 0 & -4 & 2 & -10 \\ 0 & 3 & 4 & 2 \\ 0 & 1 & 0 & 2 \end{bmatrix} 
    \stackrel{F_1 -  F_4}{\stackrel{F_2+4F_4}{\stackrel{F_3 - 3F_1 }{\longrightarrow}}}
    \begin{bmatrix} 1 & 0 & 1 & 0 \\ 0 & 0 & 2 & -2 \\ 0 & 0 & 4 & -4 \\ 0 & 1 & 0 & 2 \end{bmatrix} \\
    &\quad
    \stackrel{F_2/2}{\stackrel{F_3/4}{\longrightarrow}}
    \begin{bmatrix} 1 & 0 & 1 & 0 \\ 0 & 0 & 1 & -1 \\ 0 & 0 & 1 & -1 \\ 0 & 1 & 0 & 2 \end{bmatrix}
    \stackrel{F_1-F_3}{\stackrel{F_2-F_3}{\longrightarrow}}
    \begin{bmatrix} 1 & 0 & 0 & 1 \\ 0 & 0 & 0 & 0 \\ 0 & 0 & 1 & -1 \\ 0 & 1 & 0 & 2 \end{bmatrix}
    \stackrel{F_2 \leftrightarrow F_4}{\longrightarrow}
    \begin{bmatrix} 1 & 0 & 0 & 1 \\ 0 & 1 & 0 & 2 \\ 0 & 0 & 1 & -1 \\ 0 & 0 & 0 & 0 \end{bmatrix}.
\end{align*}

\vskip .4cm
Finalmente, la última matriz  tiene una MERF con una fila nula. Por lo tanto no es invertible:
\begin{align*}
    &\left[\begin{array}{ccc}
     1&-3&5\\
     2&-3&1\\
     0&-1&3\end{array}\right] 
     \stackrel{F_2-2F_1}{\longrightarrow}   
     \left[\begin{array}{ccc}
    1&-3&5\\
     0&3&-9\\
     0&-1&3\end{array}\right]
     \stackrel{F_1-3F_3}{\stackrel{F_2+3F_3}{\longrightarrow}}  
     \left[\begin{array}{ccc}
     1&0&-4\\
     0&0&0\\
     0&-1&3\end{array}\right]  \\
     &{\stackrel{-F_3}{\longrightarrow}}  \quad
     \left[\begin{array}{ccc}
     1&0&-4\\
     0&0&0\\
     0&1&-3\end{array}\right]{\stackrel{F_2 \leftrightarrow F_3}{\longrightarrow}}   
      \left[\begin{array}{ccc}
    1&0&-4\\
     0&1&-3\\
    0&0&0 \end{array}\right].
    \end{align*}

    \qed 


\item Sea $A$ la primera matriz del ejercicio anterior.
Hallar matrices elementales $E_1,E_2,\dots,E_k$ tales que $E_kE_{k-1}\cdots E_2E_1A=\operatorname{I}_3$.
\rta en  la primera matriz del ejercicio \ref{ej:inversas} realizamos 10 operaciones elementales de fila para llevar la matriz a la identidad. Si llamamos a la matriz $A$, entonces
$$
\operatorname{Id}_3= E_{10}E_9E_8E_7E_6E_5E_4E_3E_2E_1A,
$$
Donde 
$$
    E_1 = \begin{bmatrix} 0&0&1 \\    0&1&0 \\ 1&0&0\end{bmatrix}\;(F_1 \leftrightarrow F_3) ,\qquad
    E_2 = \begin{bmatrix} 1&0&0 \\    -2&1&0 \\ 0&0&1\end{bmatrix}\; (F_2-2F_1),
$$
$$    
    E_3 = \begin{bmatrix} 1&0&0 \\    0&1&0 \\ -3&0&1\end{bmatrix}\; (F_3-3F_1),\qquad 
    E_4 = \begin{bmatrix} 1&0&0 \\    0&1&0 \\ 0&-1&1\end{bmatrix}\; (F_3-F_2),
$$
$$
    E_5 = \begin{bmatrix} 1&0&0 \\    0&0&1 \\ 0&1&0\end{bmatrix}\; (F_3 \leftrightarrow F_2),\qquad 
    E_6 = \begin{bmatrix} 1&3&0 \\    0&1&0 \\ 0&0&1\end{bmatrix}\; (F_1 +3F_2),
$$
$$
    E_7 = \begin{bmatrix} 1&0&0 \\    0&1&0 \\ 0&-7&1\end{bmatrix}\; (F_3-7F_2),\qquad 
    E_8 = \begin{bmatrix} 1&0&0 \\    0&1&0 \\ 0&0&-1/6\end{bmatrix}\; (-(1/6)F_3),
$$
$$
    E_9 = \begin{bmatrix} 1&0&-3 \\    0&1&0 \\ 0&0&1\end{bmatrix}\; (F_1-3F_3),\qquad 
    E_{10} = \begin{bmatrix} 1&0&0 \\    0&1&-1 \\ 0&0&1\end{bmatrix}\; (F_2-F_3
$$

\qed



\item ¿Es cierto que si $A$ y $B$ son matrices invertibles entonces $A+B$ es una matriz invertible? Justificar su respuesta.
\rta es  falso, el ejemplo más sencillo es $\operatorname{Id} +(-\operatorname{Id}) =0$. Otro ejemplo, 
\begin{equation*}
    \begin{bmatrix}  1&0\\0&1 \end{bmatrix} + \begin{bmatrix}  1&1\\0&-1 \end{bmatrix} = \begin{bmatrix}  2&1\\0&0 \end{bmatrix}. 
\end{equation*}\qed



\item\label{nilpotene - id} Una matriz $A\in\mathbb{R}^{n\times n}$ se dice \emph{nilpotente} si $A^k=0$ para algún $k\in\mathbb{N}$.
Probar que si una matriz $A$ es nilpotente, entonces  $\operatorname{I}_n - A$  es invertible.
\rta supongamos que $A^k=0$ para algún $k\in\mathbb{N}$, y probemos que: $$(\operatorname{I}_n - A)^{-1} = \operatorname{I}_n + A + A^2 +\cdots + A^{k-1} = \sum_{i=0}^{k-1} A^{i}.$$
En efecto,
\begin{align*}
& (\operatorname{I}_n - A)(\operatorname{I}_n + A + A^2 +\cdots + A^{k-1})\\
&= \operatorname{I}_n (\operatorname{I}_n + A + A^2 +\cdots + A^{k-1}) - A (\operatorname{I}_n + A + A^2 +\cdots + A^{k-1})\\
&= (\operatorname{I}_n + A + A^2 +\cdots + A^{k-1}) - (A + A^2 + A^3 +\cdots + A^{k})\\
&= \operatorname{I}_n - A^{k} = \operatorname{I}_n - 0 = \operatorname{I}_n.
\end{align*}

\qed


\item\label{sol homog es subesp} Sean  $v$ y $w$ dos soluciones del sistema homogéneo $AX=0$. Probar que $v+tw$ también es solución para todo $t\in\mathbb{K}$.
\rta como $v,w$  soluciones de $AX=0$, tenemos que $Av=0$ y $Aw=0$.  Luego, por las propiedades que cumple el producto y suma de matrices:
\begin{eqnarray*}
    A(v + tw) = Av +A(tw) = 0 + t(Aw) = 0 + 0 = 0.
\end{eqnarray*}
Es decir, $v +tw$  es solución de $AX=0$.
\qed


\item\label{homogeneo+no-homogeneo} Sea $v$ una solución del sistema $AX=Y$ y $w$ una solución del sistema homogéneo $AX=0$. Probar que $v+tw$ también es solución del sistema $AX=Y$ para todo $t\in\mathbb{K}$.
\rta como $v$  solución de $AX=Y$, tenemos que $Av=Y$. Al ser $w$  solución del sistema $AX=0$, se cumple $Aw=0$. Luego, por las propiedades que cumple el producto y suma de matrices:
\begin{eqnarray*}
    A(v + tw) = Av +A(tw) = Y + t(Aw) = Y + 0 = Y.
\end{eqnarray*}
Es decir, $v +tw$  es solución de $AX=Y$.
\qed


\item Probar que si el sistema homogéneo  $AX=0$ posee alguna solución no trivial, entonces el sistema $AX=Y$ no tiene
solución o tiene al menos dos soluciones distintas.
\rta si $AX=Y$ no tiene solución, listo, pues es uno de los casos. Si $AX=Y$ tiene solución,  sea entonces $w$ alguna solución del sistema, es decir $Aw=Y$. Por hipótesis $AX=0$ tiene soluciones no triviales,  es decir existe $v \ne 0$ tal que $Av =0$. Por el ejercicio  \ref{homogeneo+no-homogeneo}: $v+ w$ es solución del sistema $AX=Y$, como $v \ne 0$, los vectores $w$ y $v+w$ son distintos y ambos son solución del sistema  $AX=Y$.
\vskip .2cm
\noindent \textbf{Observación.} En  realidad,  la existencia de una solución $w$ del sistema $AX=Y$ implica la existencia de infinitas soluciones, pues por ejercicio   \ref{homogeneo+no-homogeneo} los vectores $w + tv$ son soluciones de $AX=Y$ para todo $t \in \mathbb K$. 

\qed


\item Supongamos que los sistemas $AX=Y$ y $AX=Z$ tienen solución. Probar que el sistema $AX=Y+tZ$ también tiene solución para todo $t\in\mathbb{K}$.
\rta sea $v$ solución de $AX=Y$, es decir $Av = Y$ y sea $w$ solución del sistema $AX=Z$,  es decir $Aw=Z$. Entonces, dado $t \in \mathbb K$
\begin{equation*}
    A(v + tw) = Av + Atw = Y +tAw = Y + tZ.
\end{equation*}
Es decir $v +tw$ es solución de $AX=Y+tZ$. 

\qed


\item \label{sist-por-invertible}Sean $A$ una matriz invertible $n\times n$, y $B$ una matriz $n\times m$.  Probar que los sistemas $BX=Y$ y $ABX=AY$ tienen las mismas soluciones.
\rta
\begin{align*}
    \text{$v$ es vsol. de $BX=Y$} \; &\Rightarrow \; Bv = Y&& \\
    &\Rightarrow \;  ABv = AY &&(\text{mult. por $A$  a izq.}) \\
    &\Rightarrow \;   \text{$v$ es sol. de $ABX=AY$}&&
\end{align*}
Esta ``ida'' vale para cualquier matriz $A$,  sea invertible o no. Para la vuelta debemos utilizar la existencia de $A^{-1}$:
\begin{align*}
    \text{$v$ es sol. de $ABX=AY$} \; &\Rightarrow \; ABv = AY&& \\
    &\Rightarrow \;  A^{-1}ABv = A^{-1}AY &&(\text{mult. por $A^{-1}$  a izq.}) \\
    &\Rightarrow \;  \operatorname{Id} Bv = \operatorname{Id}Y &&(\text{ $A^{-1}A= \operatorname{Id}$}) \\
    &\Rightarrow \;   Bv =Y&&(\text{$\operatorname{Id}$ es neutro del prod.}) \\
    &\Rightarrow \;   \text{$v$ es sol. de $BX=Y$}&&
\end{align*}

\qed


\item\label{ej:sistemas ABX} 
Sean $A$ y $B$ matrices $r\times n$ y $n\times m$ respectivamente.
Probar que:
\begin{enumerate}[topsep=5pt,itemsep=5pt]
    \item\label{ej:sistemas ABX-a}  Si $m>n$, entonces el sistema $ABX=0$ tiene soluciones no triviales.
    \item\label{ej:sistemas ABX-b}  Si $r>n$, entonces existe un $Y$, $r\times 1$, tal que $ABX=Y$
    no tiene solución.
\end{enumerate}
\rta

\ref{ej:sistemas ABX-a} Como $m> n$ el sistema $BX =0$ tiene más incógnitas que ecuaciones, por lo tanto tiene soluciones no triviales. Sea $v \ne 0$ solución de $BX=0$,  es decir $Bv=0$. Entonces $ABv = A(Bv) = A0 =0$, por lo tanto $v$  es solución del sistema $ABX=0$.  

 \vskip .2cm
 \ref{ej:sistemas ABX-b} Sea $P$  matriz $r \times r$ invertible tal que $PA$ es MERF. Como $r > n$,  la  matriz $PA$  tiene  más filas que columnas y como es MERF la última fila debe ser nula.  

Ahora bien, por el ejercicio \ref{sist-por-invertible}, los sistemas $PABX=PY$ y $ABX=Y$ tiene las mismas soluciones, por lo tanto si existe $Y$ tal que $PABX=PY$ no tiene solución, entonces el sistema $ABX=Y$ tampoco tiene solución. 

Demostremos, entonces, que existe $Y$  tal que $PABX=PY$ no tiene solución: como $PA$ tiene la última fila nula, $PABX$ también tiene la última fila nula. Sea $e_r$ la matriz $r \times 1$ con $1$ en la coordenada $r$ y $0$  en las otras coordenadas. Entonces $e_r = P(P^{-1}e_r)$  tiene la última fila no nula, por lo tanto el sistema $PABX=P(P^{-1}e_r)$ no tiene solución y, por lo dicho anteriormente, el sistema   $ABX=P^{-1}e_r$ no tiene solución.\qed 

%%%%%%%%%%%%%%%%%%%%%%%%%%%%%%%%%%%%%%%%%%%%%%%%%%%%%%%%%%%%%%%%%%%%%%%%%%%%%%%%%%%%%%%%%%%%%%%%%%%%%%%%%%%%%%%%%%%%%%%%%%%%%%%%%%%%%%%%%%%%%%%%%%%%%%%%

\end{enumerate}



%%%=======================
%%%=======================
%%% CAP4 =================
    
\chapter{Soluciones\\Álgebra  II -- Año 2024/1 -- FAMAF}\label{practico-4}

\begin{enumerate}[topsep=6pt,itemsep=.4cm]
    \item Calcular el determinante de las siguientes matrices.
        \begin{align*}
        &A=\begin{bmatrix} 4&7\\ 5&3\end{bmatrix},
        &&B=\begin{bmatrix} -3&2&4\\ 1&-1&2\\ -1&4&0\end{bmatrix},
        &&
        C=\begin{bmatrix} 2&3&1&1\\ 0&2&-1&3 \\ 0&5&1&1 \\1&1&2&5\end{bmatrix}.
        \end{align*}
        
        \rta
        \begin{align*}
            \left| A \right| &= 4\cdot 3 - 7\cdot 5 = -23,\\
            \left| B \right| &= -3\cdot \left|\begin{matrix} -1&2\\ 4&0\end{matrix}\right| - 1\cdot \left|\begin{matrix} 2&4\\ 4&0\end{matrix}\right| - 1\cdot \left|\begin{matrix} 2&4\\ -1&2\end{matrix} \right|\\
            &= -3\cdot (-8) - 1\cdot (-16) - 1\cdot 8\\ 
            &= 24 + 16 -8 = 32,\\
            \left| C \right| &= 2 \cdot\left| \begin{matrix} 2&-1&3\\ 5&1&1\\ 1&2&5\end{matrix} \right| - 0 \cdot \left| C(2|1) \right| + 0 \cdot \left| C(3|1) \right| -1 \cdot\left|  \begin{matrix} 3&1&1\\ 2&-1&3 \\ 5&1&1\end{matrix} \right| \tag{*}
        \end{align*}
        Debemos ahora calcular el determinante de las matrices $3 \times 3$ que aparecen en la expresión de $|C|$. 
        \begin{align*}
            \left|\begin{matrix} 2&-1&3\\ 5&1&1\\ 1&2&5\end{matrix} \right| &= 2\cdot  \left|\begin{matrix} 1&1\\ 2&5\end{matrix} \right| - 5\cdot  \left|\begin{matrix} -1&3\\ 2&5\end{matrix} \right| + 1\cdot  \left|\begin{matrix} -1&3\\ 1&1\end{matrix} \right| \\
            &= 2\cdot 3  - 5\cdot (-11) + 1\cdot (-4) = 57,\\
            \left|  \begin{matrix} 3&1&1\\ 2&-1&3 \\ 5&1&1\end{matrix} \right| &= 3\cdot  \left| \begin{matrix} -1&3\\ 1&1\end{matrix} \right| - 2\cdot  \left|  \begin{matrix} 1&1\\ 1&1\end{matrix} \right| + 5 \cdot  \left|  \begin{matrix} 1&1\\ -1&3\end{matrix} \right| \\
            &= 3\cdot (-4) - 2\cdot 0 + 5\cdot 4 = -12 + 20 = 8.
        \end{align*}
        Luego, por (*):
        \begin{align*}
            \left| C \right| &= 2 \cdot 57 -1 \cdot 8 = 106.
        \end{align*} 
        \qed
    
    \item Sean
            $$A=
        \begin{bmatrix} 1&3&2 \\ 3&0&2 \\  1&1&1 \end{bmatrix}, \qquad
        B =
        \begin{bmatrix} 1&-1&2\\ 1&1&1 \\ -1&-1&3 \end{bmatrix}.
        $$
        Calcular:
        \begin{multicols}{3}
        \begin{enumerate}
            \item\label{det-AB} $\det(AB)$.
            \item\label{det-BA} $\det(BA)$.
            \item\label{det-A-1} $\det(A^{-1})$.
            \item\label{det-A4} $\det(A^{4})$.
            \item\label{det-A+B} $\det(A+B)$.
            \item\label{det-A+tB} $\det(A+tB)$, con $t \in \mathbb{R}$.
        \end{enumerate}
    \end{multicols}
    \rta

    Primero nos conviene calcular los determinantes de $A$ y $B$, pues algunos cálculos se reducen a saber estos números. 
    \begin{align*}
        \det(A) =\left|\begin{matrix} 1&3&2 \\ 3&0&2 \\  1&1&1 \end{matrix}\right| &= 1\cdot  \left|\begin{matrix} 0&2 \\  1&1\end{matrix} \right| - 3\cdot  \left|\begin{matrix}3&2 \\ 1&1\end{matrix} \right| + 1\cdot  \left|\begin{matrix} 3&2 \\ 0&2\end{matrix} \right| \\
        &= 1\cdot (-2)  - 3\cdot 1 + 1\cdot 6 = 1,\\
        \det(B) =\left|  \begin{matrix}  1&-1&2\\ 1&1&1 \\ -1&-1&3 \end{matrix} \right| &= 1\cdot  \left| \begin{matrix}  1&1 \\ -1&3 \end{matrix} \right| - 1\cdot  \left|  \begin{matrix} -1&2 \\ -1&3\end{matrix} \right| -1  \cdot  \left|  \begin{matrix} -1&2\\ 1&1\end{matrix} \right| \\
        &= 1\cdot 4 - 1\cdot (- 1)  -1\cdot (-3) = 8.
    \end{align*}

    Resumiendo $\det(A) = 1$ y $\det(B) = 8$. Ahora sí, calculemos los determinantes pedidos.
    
    \vskip .2cm
    \ref{det-AB} $\det(AB) = \det(A)\det(B) = 1 \cdot 8 = 8$.

    \vskip .2cm
    \ref{det-BA} $\det(BA) = \det(B)\det(A) = 8 \cdot 1 = 8$.

    \vskip .2cm
    \ref{det-A-1} $\det(A^{-1}) = 1/ \det(A) = 1/1 = 1$.
    
    \vskip .2cm
    \ref{det-A4} $\det(A^{4}) = \det(A)^{4} = 1^4 = 1$.
    
    \vskip .2cm
    \ref{det-A+tB} El ejercicio \ref{det-A+B} es un caso especial de \ref{det-A+tB} para $t=1$.  Así que haremos este inciso primero. Para ello, antes que nada, calculemos la matriz $A + tB$:
    \begin{align*}
        A+ tB &= \begin{bmatrix} 1&3&2 \\ 3&0&2 \\  1&1&1 \end{bmatrix} + t \begin{bmatrix} 1&-1&2\\ 1&1&1 \\ -1&-1&3 \end{bmatrix} \\
        &= \begin{bmatrix} 1+t&3-t&2+2t \\ 3+t&t&2+t \\  1-t&1-t&1+3t \end{bmatrix}.
    \end{align*}
    Entonces
    \begin{align*}
       \left| \begin{matrix} 1+t&3-t&2+2t \\ 3+t&t&2+t \\  1-t&1-t&1+3t \end{matrix}\right| & = (1+t)\cdot  \left| \begin{matrix} t&2+t \\  1-t&1+3t \end{matrix} \right| - (3+t)\cdot  \left| \begin{matrix} 3-t&2+2t \\  1-t&1+3t \end{matrix} \right|\\
       &\qquad\qquad\qquad\qquad\qquad\qquad + (1-t)\cdot  \left| \begin{matrix} 3-t&2+2t \\ t&2+t  \end{matrix} \right| \\
       &= (1+t)\cdot (t(1+3t) - (2+t)(1-t)) -\\
       &\quad- (3+t)\cdot ((3-t)(1+3t) - (2+2t)(1-t)) +\\
       &\quad + (1-t)\cdot ((3-t)(2+t) - (2+2t)t)\\
         &= (1+t)\cdot (4 t^2 + 2 t - 2) +\\
         &\quad- (3+t)\cdot (-t^2 + 8 t + 1) +\\
            &\quad + (1-t)\cdot (-3 t^2 - t + 6)
            \\
            &= (4 t^3 + 6 t^2 - 2) + (t^3 - 5 t^2 - 25 t - 3) + (3 t^3 - 2 t^2 - 7 t + 6
            )\\
            &=8 t^3 - t^2 - 32 t + 1
    \end{align*}

    \vskip .2cm
    \ref{det-A+B} $\det(A+B) = \det(a + 1\cdot B) \stackrel{\ref{det-A+tB}}{=}8 \cdot 1^3 - 1^2 - 32 \cdot 1 + 1 = -24.$


    \qed
    
    \item Calcular el determinante de las siguientes matrices haciendo la reducción a matrices triangulares superiores.
    
            $$A =
            \begin{bmatrix}
                a&1&1&1 \\
                1&a&1&1 \\
                1&1&a&1 \\
                1&1&1&a \\
            \end{bmatrix}, \qquad    
            B =
            \begin{bmatrix}
                1&1&1&1&1 \\
                1&3&3&3&3 \\
                1&3&5&5&5 \\
                1&3&5&7&7 \\
                1&3&5&7&9 \\
            \end{bmatrix}.
            $$
            \rta Recordemos que si $M$ es una matriz triangular superior, entonces $\det(M)$ es el producto de los elementos de la diagonal principal. Por otro  lado, las operaciones elementales por fila  necesarias para obtener una matriz triangular superior tienen el siguiente efecto en el cálculo del determinante:
            \begin{itemize}
                \item intercambiar dos filas cambia el signo del determinante, y
                \item sumar a una fila un múltiplo de otra fila no cambia el determinante.
            \end{itemize}
            La operación elemental de multiplicar una fila por una constante no es necesaria para conseguir una matriz triangular superior.

            \vskip .2cm
        \textbf{Cálculo de $\det(A)$.} Un  caso particular es cuando $a=0$,  en ese caso la matriz tiene todas las filas iguales y por lo tanto $\det(A) = 0$. 
            
            Analicemos el caso  $a\ne 1$.

            Veamos como reducimos $A$ a triangular superior:
            \begin{align*}
                A \underset{F_4 -F_2}{\underset{F_3 -F_2}{\stackrel{F_1-aF_2}{\longrightarrow}}} &=  \begin{bmatrix}
                    0&1-a^2&1-a&1-a \\ 
                    1&a&1&1 \\
                    0&1-a&a-1&0 \\
                    0&1-a&0&a-1 \\
                \end{bmatrix}
                {\stackrel{F_1 \leftrightarrow F_4}{\longrightarrow}} 
                \begin{bmatrix}
                    0&1-a&0&a-1 \\
                    1&a&1&1 \\
                    0&1-a&a-1&0 \\
                    0&1-a^2&1-a&1-a \\ 
                \end{bmatrix} \\
                &{\stackrel{F_1 \leftrightarrow F_2}{\longrightarrow}}
                \begin{bmatrix}
                    1&a&1&1 \\
                    0&1-a&0&a-1 \\
                    0&1-a&a-1&0 \\
                    0&1-a^2&1-a&1-a \\ 
                \end{bmatrix} 
                {\underset{F_4 -(1+a)F_2}{\stackrel{F_3-F_2}{\longrightarrow}}}
                \begin{bmatrix}
                    1&a&1&1 \\
                    0&1-a&0&a-1 \\
                    0&0&a-1&1-a \\
                    0&0&1-a&2-a-a^2 \\ 
                \end{bmatrix}  \\
                &\stackrel{F_4 + F_3}{\longrightarrow}
                \begin{bmatrix}
                    1&a&1&1 \\
                    0&1-a&0&a-1 \\
                    0&0&a-1&1-a \\
                    0&0&0&3-2a-a^2 \\ 
                \end{bmatrix} 
            \end{align*}
        Como solo hicimos dos permutaciones de filas, el determinante de  $A$ es igual al determinante de la última matriz,  es decir
        $$
        \det(A) = (1-a)(a-1)(3-2a-a^2) = a^4 - 6 a^2 + 8 a - 3.
        $$
        
        \vskip .2cm

        \textbf{Cálculo de $\det(B)$.}
        \begin{align*}
            B \underset{F_5 -F_1}{\underset{F_4 -F_1}{\underset{F_3 -F_1}{\underset{F_2 -F_1}{\longrightarrow}}}} &  \begin{bmatrix}
                1&1&1&1&1 \\
                0&2&2&2&2 \\
                0&2&4&4&4 \\
                0&2&4&6&6 \\
                0&2&4&6&8 \\
            \end{bmatrix}
            \underset{F_5 -F_2}{\underset{F_4 -F_2}{\underset{F_3 -F_2}{\longrightarrow}}}
            \begin{bmatrix}
                1&1&1&1&1 \\
                0&2&2&2&2 \\
                0&0&2&2&2 \\
                0&0&2&4&4 \\
                0&0&2&4&6 \\
            \end{bmatrix} \\
            \underset{F_5 -F_3}{\underset{F_4 -F_3}{\longrightarrow}}&
            \begin{bmatrix}
                1&1&1&1&1 \\
                0&2&2&2&2 \\
                0&0&2&2&2 \\
                0&0&0&2&2 \\
                0&0&0&2&4 \\
            \end{bmatrix}
            \underset{F_5 -F_4}{\longrightarrow}
            \begin{bmatrix}
                1&1&1&1&1 \\
                0&2&2&2&2 \\
                0&0&2&2&2 \\
                0&0&0&2&2 \\
                0&0&0&0&2 \\
            \end{bmatrix}.                
        \end{align*}
        El determinante de  $B$ es igual al determinante de la última matriz,  es decir:
        $$
        \det(B) = 2^4 = 16.
        $$






            \qed

    \item Sean $A$, $B$ y $C$ matrices $n\times n$, tales que $\det A=-1$, $\det B=2$ y $\det C=3$.
    Calcular:
    
    \begin{enumerate}
    \item\label{det pqr} $\det(PQR)$, donde $P$, $Q$ y $R$ son las matrices que se obtienen a partir de $A$, $B$ y $C$ mediante operaciones elementales por filas de la siguiente manera
     \begin{align*}
     A\overset{F_1+2F_2}{\longrightarrow} P,\quad
     B\overset{3F_3}{\longrightarrow} Q
     \quad\mbox{y}\quad
     C\overset{F_1\leftrightarrow F_4}{\longrightarrow} R.
     \end{align*}
     Es decir,
     \begin{itemize}
      \item[$\circ$] $P$ se obtiene a partir de $A$ sumando a la fila $1$ la fila $2$ multiplicada por $2$.
      \item[$\circ$] $Q$ se obtiene a partir de $B$ multiplicando la fila $3$ por $3$.
      \item[$\circ$] $R$ se obtiene a partir de $C$ intercambiando las filas $1$ y $4$.
     \end{itemize}
        \item\label{det a2bc..} $\det(A^2BC^tB^{-1})$ \ y \ $\det(B^2C^{-1}AB^{-1}C^{t})$.
    \end{enumerate}
    \rta

    \ref{det pqr} La matriz $P$ se obtiene de $A$ sumando a la fila $1$ la fila $2$ multiplicada por $2$. Esta operación elemental no cambia el determinante, luego $\det(P) = \det(A) = -1$. La matriz $Q$ se obtiene de $B$ multiplicando la fila $3$ por $3$, luego $\det(Q) = 3\cdot \det(B) = 6$. La matriz $R$ se obtiene de $C$ intercambiando las filas $1$ y $4$, luego $\det(R) = -\det(C) = -3$. 

    \vskip .2cm
    \ref{det a2bc..}    \begin{align*}
        \det(A^2BC^tB^{-1}) &= \det(A^2)\det(B)\det(C^t)\det(B^{-1})\\
        &= \det(A)^2\det(B)\det(C)\det(B)^{-1}\\
        &= (-1)^2\cdot 2\cdot 3\cdot 1/2\\
        &= 3.
    \end{align*}


    \qed
    
    \item  Sea
    $$A=
    \begin{bmatrix}
        x&y&z \\
        3&0&2\\
        1&1&1
    \end{bmatrix}.$$
    Sabiendo que $\det(A) = 5$, calcular el determinante de las siguientes matrices.
    $$
    B = \begin{bmatrix}
    2x&2y&2z \\
    3/2&0&1\\
    1&1&1
    \end{bmatrix}, \qquad
    C=
    \begin{bmatrix}
        x&y&z \\
        3x+3&3y&3z+2\\
        x+1&y+1&z+1
    \end{bmatrix}.
    $$
    \rta
    $A \underset{F_2/2}{\overset{2F_1}{\longrightarrow}} B$, luego  $\det(B) = \frac12 \cdot 2\cdot \det(A) = \det(A) = 5$.

    $A \underset{F_3+F_1}{\overset{F_2+3F_1}{\longrightarrow}} C$, luego  $\det(C) = \det(A) = 5$.
    \qed
    
    \item Determinar todos los valores de $c\in\mathbb{R}$ tales que las siguientes matrices sean invertibles.
    \begin{align*}
    A=\begin{bmatrix}4& c&3\\c&2&c\\ 5&c&4 \end{bmatrix},\qquad
    B=\begin{bmatrix} 1&c&-1\\ c&1&1\\0&1&c\end{bmatrix}.
    \end{align*}
    \rta recordar que una matriz es invertible si y solo si  su determinante es no nulo. Luego, debemos calcular el determinante de $A$ y $B$ y ver para qué valores de $c$ son distintos de cero.


    Para el cálculo del determinante desarrollamos por la primera fila.
    \begin{align*}
        \det(A) &= 4\cdot \left|\begin{matrix} 2&c\\ c&4\end{matrix}\right| - c\cdot \left|\begin{matrix} c&c\\ 5&4\end{matrix}\right| + 3\cdot \left|\begin{matrix} c&2\\ 5&c\end{matrix}\right|\\
        &= 4\cdot (8 - c^2) - c\cdot (4c - 5c) + 3\cdot (c^2 - 10)\\
        &= 32 - 4c^2 - 4c^2 + 5c^2 + 3c^2 - 30\\
        &=  2.    
    \end{align*}
    Como $\det(A) = 2 \ne 0$ independientemente del valor de $c$, la matriz $A$ es invertible para todo $c\in\mathbb{R}$.

    Para el cálculo del determinante de $B$ desarrollamos por la primera fila.
    \begin{align*}
        \det(B) &= 1\cdot \left|\begin{matrix} 1&1\\ 1&c\end{matrix}\right| - c\cdot \left|\begin{matrix} c&1\\ 0&c\end{matrix}\right| - 1\cdot \left|\begin{matrix} c&1\\ 0&1\end{matrix}\right|\\
        &= 1\cdot (c - 1) - c\cdot c^2 - 1\cdot c\\
        &= -c^3 -  1.
    \end{align*}
    Luego, $\det(B) = 0$ si y solo si $c^3 = -1$, es decir si y solo si $c = -1$. Por lo tanto, la matriz $B$ es invertible si y solo si $c\ne -1$.
    \qed


    
    \item Calcular el determinante de las siguientes matrices, usando operaciones elementales por fila y/o columnas u otras propiedades del determinante. Determinar cuáles de ellas son invertibles.
    \begin{equation*}
    A=
    \begin{bmatrix}-2&3&2&-6\\ 0&4&4&-5\\ 5&-6&-3&2\\ -3&7&0&0 \end{bmatrix},\quad
    B=\begin{bmatrix} 2&0&0&0\\ 0&0&3&0\\ 0&-1&0&0\\ 0&0&0&4\end{bmatrix},\quad
    \end{equation*}
    \begin{equation*}
        C=\begin{bmatrix} -2&3&2&-6&0\\ 0&4&4&-5&0\\ 5&-6&-3&2&0\\ -3&7&0&0&0\\ 1&1&1&1&1\end{bmatrix},\quad
    D=\begin{bmatrix}1&2&3&0&0\\-1&2&-13&6&\frac{1}{3}\\2&0&0&0&0\\11&1&0&0&0\\\sqrt{2}&2&1&\pi&0\end{bmatrix},
    \end{equation*}
    \begin{equation*}
    E=\begin{bmatrix}1&-1&2&0&0\\ 3&1&4&0&0\\ 2&-1&5&0&0 \\0&0&0&2&1\\ 0&0&0&-1&4    \end{bmatrix}.
    \end{equation*}
    \rta

    \textbf{Cálculo de $\det(A)$.} Calculamos el determinante de $A$ desarrollando por la última fila debido a que tiene dos   $0$. 
    \begin{align*}
        \det(A) &= 3\cdot \left|\begin{matrix} 3&2&-6\\ 4&4&-5\\ -6&-3&2\end{matrix}\right| + 7\cdot \left|\begin{matrix} -2&2&-6\\ 0&4&-5\\ 5&-3&2\end{matrix}\right|\\
        &= 3\cdot (-49) + 7\cdot  84\\
        &= 441.
    \end{align*}
    Como $\det(A) \ne 0$, la matriz $A$ es invertible.


    \vskip .2cm
    \textbf{Cálculo de $\det(B)$.} Permutando  la fila $2$ con la fila $3$ obtenemos una matriz diagonal con $2$, $3$, $-1$, $4$ en la diagonal, luego $\det(B) = -(2\cdot 3\cdot (-1)\cdot 4) = 24$. Por lo tanto, $B$  es invertible.

    \vskip .2cm
    \textbf{Cálculo de $\det(C)$.} En este caso conviene desarrollar por la última columna debido a que tiene cuatro $0$. Entonces
    \begin{equation*}
        \det(C) = 1\cdot \left|\begin{matrix} -2&3&2&-6\\ 0&4&4&-5\\ 5&-6&-3&2\\ -3&7&0&0\end{matrix}\right|
    \end{equation*}
    Observar que la matriz de la derecha es la matriz $A$ del ejercicio anterior. Luego $\det(C) = 1\cdot 441 = 441$. Por lo tanto, $C$  es invertible.

    \vskip .2cm
    \textbf{Cálculo de $\det(D)$.} Desarrollaremos por la última columna y luego de nuevo por la última columna.
    \begin{align*}
        \det(D) &= -\frac13 \cdot  \left|\begin{matrix}1&2&3&0\\2&0&0&0\\11&1&0&0\\\sqrt{2}&2&1&\pi\end{matrix}\right| =  -\frac13 \cdot\pi \cdot  \left|\begin{matrix}1&2&3\\2&0&0\\11&1&0\end{matrix}\right|\\
        &=  -\frac13 \cdot\pi \cdot (-2) \cdot  \left|\begin{matrix}2&3\\1&0\end{matrix}\right| =  -\frac13 \cdot\pi \cdot (-2) \cdot (-3) = -2\pi.
    \end{align*}
    En  el renglón anterior desarrollamos por la fila 2. Luego, $\det(D) = -2\pi$. Por lo tanto, $D$  es invertible.

    \vskip .2cm
    \textbf{Cálculo de $\det(E)$.} Para facilitar la escritura escribamos la matriz como una matriz de bloques, 
    $$
    E = \begin{bmatrix}  E_1&0\\0&E_2 \end{bmatrix},
    $$
    donde $E_1$ es el primer bloque diagonal $3 \times 3$ y $E_2$ es el segundo  bloque diagonal de $2 \times 2$.
    
    Desarrollemos por la última columna dos veces
    \begin{align*}
        \det(E) &= (-1) \cdot \left| \begin{matrix}  E_1&0\\0&1 \end{matrix} \right| + 4 \cdot\left| \begin{matrix}  E_1&0\\0&2 \end{matrix} \right| \\
        &= (-1) \cdot 1 \left|   E_1 \right| + 4 \cdot 2\cdot\left|   E_1 \right| \\
        &= ((-1) \cdot 1 + 4 \cdot 2) \left|   E_1 \right|.
    \end{align*}

    Averigüemos ahora el determinate de $E_1$. Podríamos hacerlo por cálculo directo, pero  lo haremos transformando   $E_1$ en triangular superior por medio de operaciones elementales de fila y luego multiplicando los elementos de la diagonal. 
    \begin{align*}
        E_1 {\underset{F_3 -2F_1}{\stackrel{F_2-3F_1}{\longrightarrow}}} \begin{bmatrix}1&-1&2\\ 0&4&-2\\ 0&1&1  \end{bmatrix} 
        \stackrel{F_2 \leftrightarrow F_3}{\longrightarrow}
        \begin{bmatrix}1&-1&2\\ 0&1&1 \\ 0&4&-2 \end{bmatrix} 
        \stackrel{F_3 -4 F_2}{\longrightarrow}
        \begin{bmatrix}1&-1&2\\ 0&1&1 \\ 0&0&-6 \end{bmatrix}.
    \end{align*}
    Como hay una permutación de filas tenemos que 
    $$\det(E_1) = -(1\cdot 1 \cdot (-6)) = 6.$$

    Luego $\det(E) = (4 \cdot 2 - 1) \cdot 6 = 42$. Por lo tanto, $E$  es invertible.
    \vskip .4cm

    \noindent\textbf{Observación.} Notemos que en este ejercicio 
    $$
    \det(E) = \det(E_1)\det(E_2),
    $$
    y este es un hecho general: si $E$ es una matriz  $n \times n$ de bloques diagonales $E_{1}$ y $E_{2}$, es decir
    $$
    E = \begin{bmatrix}  E_1&0\\0&E_2 \end{bmatrix},
    $$
    entonces 
    $$
    \det(E) = \det(E_1)\det(E_2).
    $$
    También vale  el resultado para un número arbitrario de bloques diagonales. Dejamos al lector interesado la demostración de este hecho.
    \qed
    
    \item Sean $A$ y  $B$ matrices $n \times n$. Probar que:
    \begin{enumerate}
        \item\label{det A.B} $\det(AB) = \det (BA)$.
        \item\label{det B.A.B1} Si $B$ es invertible, entonces $\det(B A B^{-1}) = \det (A)$.
        \item\label{-A} $\det(-A) = (-1)^n\det (A)$.
    \end{enumerate} 
    \rta En todos los incisos usamos la propiedad de que el determinate del producto de matrices es el producto de los determinantes de cada matriz.


    \ref{det A.B} 
    $$
    \det(AB) = \det(A)\det(B) = \det(B)\det(A) = \det(BA).
    $$

    \ref{det B.A.B1} Como $BB^{-1} = \Id$, entonces $\det(B^{-1})\det(B) = \det(\Id) = 1$. Luego, $\det(B^{-1}) = 1/\det(B)$. Por lo tanto,
    \begin{align*}
        \det(B A B^{-1}) &= \det(B)\det(A)\det(B^{-1}) = \det(B)\det(B^{-1})\det(A)\\ &=  \det(B)\frac1{\det(B)}\det(A)=\det(A).
    \end{align*}
    

    \ref{-A} Sea $\Id$ es la matriz identidad $n \times n$. Observar que $-A = (-\Id)A$. Luego, $\det(-A) = \det((-\Id)A) = \det(-\Id)\det(A)$. Como $-\Id$ es una matriz diagonal con $-1$ en la diagonal, entonces $\det(-\Id) = (-1)^n$. Por lo tanto, $\det(-A) = (-1)^n\det(A)$.

    \qed
    
    \item\label{vandermonde} Sean $\lambda_1, \lambda_2, \dots, \lambda_n$ escalares, la matriz de \emph{Vandermonde} asociada es
    \begin{align*}
    \mathtt V_n = \begin{bmatrix}
    1 & \lambda_1 & \lambda_1^2 & \cdots & \lambda_1^{n-1}\\
    1 & \lambda_2 & \lambda_2^2 & \cdots & \lambda_2^{n-1}\\
    \vdots &\vdots &\vdots & &\vdots\\
    1 & \lambda_n & \lambda_n^2 & \cdots & \lambda_n^{n-1}\\
    \end{bmatrix}.
    \end{align*}
    Esta es la matriz del sistema de ecuaciones del ejercicio \ref{polinomios}\,\ref{polinomios-c} del Práctico \ref{practico-2}.
    
    
    \begin{enumerate}
        \item\label{vandermonde 2} Si $n=2$, probar que $\det(\mathtt V_n) = \lambda_2-\lambda_1$.
        \item\label{vandermonde 3} Si $n=3$, probar que $\det(\mathtt V_n) = (\lambda_3-\lambda_2) (\lambda_3-\lambda_1) (\lambda_2-\lambda_1)$.
        \item\label{vandermonde gral} Probar que $\det(\mathtt V_n) = \prod_{1\leq i< j \leq n}(\lambda_j-\lambda_i)$ para todo $n\in\mathbb{N}$.
        \item\label{vandermonde inv} Dar una condición necesaria y suficiente para que la matriz de Vandermonde sea invertible.
        \item\label{vandermonde sol} Dados $b_1, \ldots, b_n$  y $\lambda_1, \ldots, \lambda_n$ secuencias de números reales,  dar una condición suficiente para que exista un  polinomio de grado $n$, digamos $p$, tal que 
        $$
        p(\lambda_1)=b_1, \ldots, p(\lambda_n)=b_n.
        $$
        (ver ejercicio \ref{polinomios} del Práctico \ref{practico-2}).
    \end{enumerate}
    \rta

    \ref{vandermonde 2} En  este caso la matriz es  $2 \times 2$ y por lo tanto es fácil calcular el determinante:      
    \begin{align*}
        \det(\mathtt V_2) &= \left|\begin{matrix} 1 & \lambda_1 \\ 1 & \lambda_2 \end{matrix}\right| = \lambda_2 - \lambda_1.
    \end{align*}

    \vskip .2cm
    \ref{vandermonde 3} En  este caso la matriz es  $3 \times 3$ y calcularemos el determinante transformado la matriz por operaciones elementales de columnas y filas. El  método que haremos servirá para generalizarlo en el caso  \ref{vandermonde gral}.
    \begin{align*}
        \det(\mathtt V_3) &= \left|\begin{matrix} 1 & \lambda_1 & \lambda_1^2 \\ 1 & \lambda_2 & \lambda_2^2 \\ 1 & \lambda_3 & \lambda_3^2 \end{matrix}\right| \\
        &\stackrel{C_3-\lambda_1 C_2}{=} \left|\begin{matrix}  1 & \lambda_1 & 0 \\ 1 & \lambda_2 & \lambda_2(\lambda_2-\lambda_1) \\ 1 & \lambda_3 & \lambda_3(\lambda_3-\lambda_1)  \end{matrix}\right| \\
        &\stackrel{C_2-\lambda_1 C_1}{=} \left|\begin{matrix}  1 & 0 & 0 \\ 1 & \lambda_2 -\lambda_1& \lambda_2(\lambda_2-\lambda_1) \\ 1 & \lambda_3-\lambda_1 & \lambda_3(\lambda_3-\lambda_1)  \end{matrix}\right| \\
        &\overset{F_2 /(\lambda_2-\lambda_1)}{\stackrel{F_3 /(\lambda_3-\lambda_1)}{=}}(\lambda_2-\lambda_1)(\lambda_3-\lambda_1)  \left|\begin{matrix}  1 & 0 & 0 \\ 1 & 1& \lambda_2 \\ 1 & 1 & \lambda_3 \end{matrix}\right| \\
        &{=}(\lambda_2-\lambda_1)(\lambda_3-\lambda_1)  \left|\begin{matrix}   1& \lambda_2 \\ 1 & \lambda_3 \end{matrix}\right| \\
        &= (\lambda_2-\lambda_1) (\lambda_3-\lambda_1)(\lambda_3-\lambda_2).
    \end{align*}
    Para el último cálculo desarrollamos por la primera fila. 

    Observar que $\det(V_3)$ es $ (\lambda_2-\lambda_1) (\lambda_3-\lambda_1)$ por una matriz de Vandermonde de dimensión $2 \times 2$. Esto se  puede generalizar para calcular el determinante de una matriz de Vandermonde de mayor dimensión, como veremos a continuación. 

    \vskip .2cm
    \ref{vandermonde gral} Haremos la demostración por inducción. El caso base es $n=2$ y en ese caso el resultado vale por \ref{vandermonde 2}. Supongamos ahora que valga el resultado \ref{vandermonde gral} para $n-1$. 
    
    Fijarse que en el caso  anterior (\ref{vandermonde 3}) hicimos dos operaciones elementales de columna: $C_3-\lambda_1 C_2$ y $C_2-\lambda_1 C_1$,  en ese orden. En general, haremos  $n-1$ operaciones elementales de columnas de la forma $C_i-\lambda_1 C_{i-1}$, desde $i=n$ hasta $i=2$,  en ese orden con lo cual obtenemos la matriz

\begin{equation*}
    \begin{bmatrix}
    1 & 0 & 0 & \dots & 0\\
    1 & \lambda_2-\lambda_1 & \lambda_2(\lambda_2-\lambda_1) & \dots & \lambda_2^{n-2}(\lambda_2-\lambda_1)\\
    1 & \lambda_3-\lambda_1 & \lambda_3(\lambda_3-\lambda_1) & \dots & \lambda_3^{n-2}(\lambda_3-\lambda_1)\\
    \vdots & \vdots & \vdots & \ddots &\vdots \\
    1 & \lambda_n-\lambda_1 & \lambda_n(\lambda_n-\lambda_1) & \dots & \lambda_n^{n-2}(\lambda_n-\lambda_1)\\
    \end{bmatrix} \tag{*}
\end{equation*}
Ninguna de las operaciones elementales que hicimos cambia el valor del determinante. Si calculamos el determinante de la matriz (*) desarrollando por la primera fila, obtenemos
\begin{equation*}
    \begin{vmatrix}V_n \end{vmatrix}=\begin{vmatrix}
        \lambda_2-\lambda_1 & \lambda_2(\lambda_2-\lambda_1) & \dots & \lambda_2^{n-2}(\lambda_2-\lambda_1)\\
        \lambda_3-\lambda_1 & \lambda_3(\lambda_3-\lambda_1) & \dots & \lambda_3^{n-2}(\lambda_3-\lambda_1)\\
        \vdots & \vdots & &\vdots \\
        \lambda_n-\lambda_1 & \lambda_n(\lambda_n-\lambda_1) & \dots & \lambda_n^{n-2}(\lambda_n-\lambda_1)\\
        \end{vmatrix}.\tag{**}.
\end{equation*}

Ahora dividimos la fila $i$ de la matriz de la derecha de (**) por $\lambda_i-\lambda_1$ para $i=2,\dots,n$ y obtenemos
\begin{equation*}
\begin{vmatrix} V_n \end{vmatrix}=
(\lambda_2-\lambda_1)(\lambda_3-\lambda_1)\dots(\lambda_n-\lambda_1)
\begin{vmatrix}
1 & \lambda_2 & \lambda_2^2 & \dots & \lambda_2^{n-2}\\
1 & \lambda_3 & \lambda_3^2 & \dots & \lambda_3^{n-2}\\
1 & \lambda_4 & \lambda_4^2 & \dots & \lambda_4^{n-2}\\
\vdots & \vdots & \vdots & &\vdots \\
1 & \lambda_n & \lambda_n^2 & \dots & \lambda_n^{n-2}\\
\end{vmatrix}. \tag{***}
\end{equation*}
La matriz de la derecha de (***) es una matriz de Vandermonde de dimensión $n-1$ con variables $\lambda_2, \lambda_3,\ldots,\lambda_n$, y por lo tanto su determinante es, por hipótesis inductiva, 
\begin{equation*}
    \prod_{2\leq i< j \leq n}(\lambda_j-\lambda_i).
\end{equation*}
En consecuencia,
\begin{align*}
    \begin{vmatrix} V_n \end{vmatrix}&=
    (\lambda_2-\lambda_1)(\lambda_3-\lambda_1)\dots(\lambda_n-\lambda_1)
    \prod_{2\leq i< j \leq n}(\lambda_j-\lambda_i) \\
    &=\prod_{1\leq i< j \leq n}(\lambda_j-\lambda_i).
\end{align*}
    
\vskip .2cm
\ref{vandermonde inv} Sabemos que una matriz es invertible si y solo si su determinante es no nulo y que el determinante de una matriz de Vandermonde es el producto de los factores $(\lambda_j-\lambda_i)$ con $i<j$. Por lo tanto, la matriz de Vandermonde es invertible si y solo si $\lambda_j-\lambda_i \ne 0$ para $i<j$. Es decir, si y solo si $\lambda_j \ne \lambda_i$ para $i\ne j$. 

\vskip .2cm
\ref{vandermonde sol} Dados $b_1, \ldots, b_n$  y $\lambda_1, \ldots, \lambda_n$ secuencias de números reales,  se plantea se plantea si existe una polinomio de grado $n$, digamos $p$, tal que 
$$
p(\lambda_1)=b_1, \ldots, p(\lambda_n)=b_n.
$$
Los coeficientes de $p$ son las incógnitas del sistema y la matriz de coeficientes es la matriz de Vandermonde. Por lo tanto, el sistema tiene solución única si la matriz de Vandermonde es invertible, es decir si $\lambda_j \ne \lambda_i$ para $i\ne j$. Por lo tanto,  una condición suficiente para que exista el polinomio $p$ es: $\lambda_j \ne \lambda_i$ para $i\ne j$. 

Hay otras situaciones donde el sistema tiene solución, pero no es única. Por  ejemplo,  si $\lambda_1 = \lambda_2$ y $b_1 = b_2$,  y  $\lambda_j \ne \lambda_i$ para $i\ne j$ para $2 \le i,j \le n$, entonces el sistema tiene infinitas soluciones. 



    \qed
        
    \item Decidir si las siguientes afirmaciones son verdaderas o falsas. Justificar con una demostración o con un contraejemplo, según corresponda.
        \begin{enumerate}
        \item\label{det VoF det(A+b)} Sean $A$ y $B$ matrices $n \times n$. Entonces $\det(A + B) = \det (A) + \det(B)$.
        \item\label{det VoF det 3x2}  Existen una matriz $3\times 2$, $A$, y una matriz $2\times 3$, $B$, tales que $\det(AB) \neq 0$.
        \item\label{inv A^n} Sea $A$ una matriz $n\times n$. Si $A^n$ es no invertible, entonces $A$ es no invertible.
    \end{enumerate}
    \rta

    \ref{det VoF det(A+b)} Como ya fue visto en la teórica esta afirmación es falsa y un ejemplo sencillo es el siguiente:   
    \begin{equation*}
        A = \begin{bmatrix} 1&0\\0&0 \end{bmatrix}, \qquad B = \begin{bmatrix} 0&0\\0&1 \end{bmatrix}.
    \end{equation*} En este caso $\det(A) = \det(B) = 0$ y $\det(A+B) = 1$.
    
    \vskip .2cm
    \ref{det VoF det 3x2} Esta afirmación es falsa y la a continuación haremos una demostración de este hecho. Observar que con un contraejemplo no  basta, pues debemos demostrar  que dadas $A$ matriz  $3\times 2$ y $B$ matriz $2\times 3$, \textit{cualesquiera}, entonces  que $\det(AB) = 0$.

    Haremos operaciones elementales por fila en la matriz $A$ para transformarla en una matriz triangular superior. Como en el caso $3 \times 3$  hacer operaciones elementales por fila  es equivalente a multiplicar a izquierda por matrices elementales $3 \times 3$. 
    
    Por lo tanto si $A'$ es una matriz triangular superior obtenida de $A$ por operaciones elementales por fila, entonces $A' = EA$ para alguna matriz $E$ invertible $3 \times 3$ donde  $E$ es producto de matrices elementales. Como $\det(EAB) = \det(E)\det(AB)$, entonces  $\det(EAB) \ne 0 $ si y solo si $\det(AB) \ne 0$.
\vskip .2cm
    Ahora bien,  $A'$ una matriz $3 \times 2$ triangular superior tiene la forma 
    $$
    A' = \begin{bmatrix} a&b\\0&c\\0&0 \end{bmatrix},
    $$
    por lo tanto
    $$
    A'B = \begin{bmatrix} a&b\\0&c\\0&0 \end{bmatrix} \begin{bmatrix} x&y&z\\u&v&w \end{bmatrix} = \begin{bmatrix} ax+bu&ay+bv&az+bw\\0&cy&cz\\0&0&0 \end{bmatrix}.
    $$ 
    La matriz resultante tiene la última fila con todos los coeficientes iguales a $0$, por lo tanto $\det(A'B) = 0$, lo cual implica que $\det(AB) = 0$, como queríamos demostrar.

    \vskip .2cm
    \ref{inv A^n} Esta afirmación es verdadera. Si $A^n$ es no invertible, entonces $\det(A^n) = 0$. Como $\det(A^n) = \det(A)^n$, entonces $\det(A)^n = 0$ y por lo tantos $\det(A) = 0$. En  consecuencia, $A$ es no invertible.


    \qed
    
    \end{enumerate}
    
%%%=======================
%%%=======================
%%% CAP5 =================
    
\chapter{Soluciones\\Álgebra  II -- Año 2024/1 -- FAMAF}\label{practico-5}

\begin{enumerate}[topsep=6pt,itemsep=.4cm]


    \item\label{autovalores} Para cada una de las siguientes matrices, hallar sus autovalores reales, y para cada autovalor, dar una descripción paramétrica del autoespacio asociado sobre $\mathbb{R}$.
    \begin{multicols}{2}
    \begin{enumerate}
        \item\label{autovalores-1} $\left[\begin{matrix} 2 & 1\\ -1 & 4 \end{matrix} \right]$,\vskip .6cm 
        \item\label{autovalores-2} $ \left[\begin{matrix} 1 & 0\\ 1 & -2 \end{matrix} \right]$, \vskip .5cm
        \item\label{autovalores-3} $\left[\begin{matrix}2 & 0 & 0\\ -1 & 1& -1\\ 0 & 0 & 2 \end{matrix} \right]$,\vskip .2cm
        \item\label{autovalores-4} $\begin{bmatrix} 3 & -5 \\ 1 & -1 \end{bmatrix}$,\vskip .2cm
        \item\label{autovalores-5}  $\begin{bmatrix} \lambda & 0 & 0 \\ 1 & \lambda & 0\\ 0 & 1 & \lambda \\ \end{bmatrix}$, $\lambda\in \mathbb R$
        \vskip .2cm
        \item\label{autovalores-6} $\left[\begin{matrix}1 & 0& 0 \\ 0 & \cos\theta & \sen\theta\\ 0 & -\sen\theta & \cos\theta \end{matrix} \right]$, $0\leq \theta<2\pi$.
    \end{enumerate}
    \end{multicols}


    \rta
    En todos los casos calculamos el polinomio característico, averiguamos sus raíces y luego resolvemos el sistema de ecuaciones que nos da el autoespacio asociado a cada autovalor.
    
    \vskip .2cm
    \ref{autovalores-1}
    Denominamos $A$  a la matriz del enunciado. El polinomio característico de $A$ es 
    \begin{align*}
        \chi_A(x) &= \left|\begin{matrix} x-2 & -1\\ 1 & x-4 \end{matrix} \right| \\ &= (x-2)(x-4)-(-1)(1) = x^2-6x+9 \\
        &= (x-3)^2.
    \end{align*}
    Luego $\lambda=3$ es el único autovalor. Para calcular el autoespacio asociado a $\lambda=3$ resolvemos el sistema  $(3\Id-A)v=0$, es decir
    \begin{align*}
        \begin{bmatrix} 3-2 & -1\\ 1 & 3-4 \end{bmatrix} &= \begin{bmatrix} 1 & -1\\ 1 & -1 \end{bmatrix} \begin{bmatrix} v_1\\ v_2 \end{bmatrix} = \begin{bmatrix} 0\\ 0 \end{bmatrix} \\
        \Rightarrow \begin{bmatrix} v_1-v_2\\ v_1-v_2 \end{bmatrix} = \begin{bmatrix} 0\\ 0 \end{bmatrix} &\Rightarrow v_1=v_2.
    \end{align*}
    Luego,  el único autovalor es $\lambda=3$ y el autoespacio asociado es $\{(t,t): t\in\mathbb{R}\}$.


    
    \vskip .2cm
    \ref{autovalores-2} Denominamos $B$  a la matriz del enunciado. El polinomio característico de $B$ es
    \begin{align*}
        \chi_B(x) &= \left|\begin{matrix} x-1 & 0\\ -1 & x+2 \end{matrix} \right| \\ &= (x-1)(x+2)-(-1)(0) = x^2+x-2 \\
        &= (x+2)(x-1).
    \end{align*}
    Luego $\lambda_1=-2$ y $\lambda_2=1$ son los autovalores. 
    
    Para calcular el autoespacio asociado a $\lambda_1=-2$ resolvemos el sistema  $(-2\Id-B)v=0$, es decir
    \begin{align*}
        \begin{bmatrix} -2-1 & 0\\ -1 & -2+2 \end{bmatrix} &= \begin{bmatrix} -3 & 0\\ -1 & 0 \end{bmatrix} \begin{bmatrix} v_1\\ v_2 \end{bmatrix} = \begin{bmatrix} 0\\ 0 \end{bmatrix} \\
        \Rightarrow \begin{bmatrix} -3v_1\\ -v_1 \end{bmatrix} = \begin{bmatrix} 0\\ 0 \end{bmatrix} &\Rightarrow v_1=0.
    \end{align*}
    Luego,  el autovalor $\lambda_1=-2$ tiene autoespacio asociado $\{(0,t): t\in\mathbb{R}\}$.
    Para calcular el autoespacio asociado a $\lambda_2=1$ resolvemos el sistema  $(\Id-B)v=0$, es decir
    \begin{align*}
        \begin{bmatrix} 1-1 & 0\\ -1 & 1+2 \end{bmatrix} &= \begin{bmatrix} 0 & 0\\ -1 & 3 \end{bmatrix} \begin{bmatrix} v_1\\ v_2 \end{bmatrix} = \begin{bmatrix} 0\\ 0 \end{bmatrix} \\
        \Rightarrow \begin{bmatrix} 0\\ -v_1+3v_2 \end{bmatrix} = \begin{bmatrix} 0\\ 0 \end{bmatrix} &\Rightarrow v_1=3v_2.
    \end{align*}
    Luego,  el autovalor $\lambda_2=1$ tiene autoespacio asociado $\{(3t,t): t\in\mathbb{R}\}$.
    

    \vskip .2cm
    \ref{autovalores-3}
    Denominamos $C$  a la matriz del enunciado. El polinomio característico de $C$ es
    \begin{align*}
        \chi_C(x) &= \left|\begin{matrix} x-2 & 0 & 0\\ 1 & x-1 & 1\\ 0 & 0 & x-2 \end{matrix} \right| \\ &= (x-2)^2(x-1).
    \end{align*}
    Luego $\lambda_1=2$ y $\lambda_2=1$ son los autovalores.

    Para calcular el autoespacio asociado a $\lambda_1=2$ resolvemos el sistema  $(2\Id-C)v=0$, es decir
    \begin{align*}
        \begin{bmatrix} 2-2 & 0 & 0\\ 1 & 2-1 & 1\\ 0 & 0 & 2-2 \end{bmatrix} &= \begin{bmatrix} 0 & 0 & 0\\ 1 & 1 & 1\\ 0 & 0 & 0 \end{bmatrix} \begin{bmatrix} v_1\\ v_2 \\ v_3\end{bmatrix} = \begin{bmatrix} 0\\ 0 \\ 0\end{bmatrix} \\
        \Rightarrow \begin{bmatrix} 0\\ v_1+v_2+v_3 \\ 0 \end{bmatrix} = \begin{bmatrix} 0\\ 0 \\ 0\end{bmatrix} &\Rightarrow v_1=-v_2-v_3.
    \end{align*}
    Luego,  el autovalor $\lambda_1=2$ tiene autoespacio asociado $\{(-t,t,s): t,s\in\mathbb{R}\}$.

    Para calcular el autoespacio asociado a $\lambda_2=1$ resolvemos el sistema  $(\Id-C)v=0$, es decir
    \begin{align*}
        \begin{bmatrix} 1-2 & 0 & 0\\ 1 & 1-1 & 1\\ 0 & 0 & 1-2 \end{bmatrix} &= \begin{bmatrix} -1 & 0 & 0\\ 1 & 0 & 1\\ 0 & 0 & -1 \end{bmatrix} \begin{bmatrix} v_1\\ v_2 \\ v_3\end{bmatrix} = \begin{bmatrix} 0\\ 0 \\ 0\end{bmatrix} \\
        \Rightarrow \begin{bmatrix} -v_1\\ v_1+v_3 \\ -v_3 \end{bmatrix} = \begin{bmatrix} 0\\ 0 \\ 0\end{bmatrix} &\Rightarrow v_1=v_3=0.
    \end{align*}
    Luego,  el autovalor $\lambda_2=1$ tiene autoespacio asociado $\{(0,t,0): t\in\mathbb{R}\}$.

    

    \vskip .2cm
    \ref{autovalores-4} Denominamos $D$  a la matriz del enunciado. El polinomio característico de $D$ es
    \begin{align*}
        \chi_D(x) &= \left|\begin{matrix} x-3 & 5\\ -1 & x+1 \end{matrix} \right| \\ &= (x-3)(x+1)-(-1)(5) = x^2-2x+2 
    \end{align*}
    El polinomio $ x^2-2x+2$ no tiene raíces reales, luego $D$ no tiene autovalores reales y por lo tanto tampoco hay autovectores.


    \vskip .2cm
    \ref{autovalores-5} Denominamos $E$  a la matriz del enunciado. En realidad la matriz $E$ depende de $\lambda$ y  debemos analizar la existencia de autovalores y autovectores para diferentes valores de  $\lambda$. El polinomio característico de $E$ es
    \begin{align*}
        \chi_E(x) &= \left|\begin{matrix} x-\lambda & 0 & 0\\ 1 & x-\lambda & 0\\ 0 & 1 & x-\lambda \end{matrix} \right| \\ &= (x-\lambda)^3.
    \end{align*}
    Luego $\lambda$ es el único autovalor. Para calcular el autoespacio asociado a $\lambda$ resolvemos el sistema  $(\lambda\Id-E)v=0$, es decir   
    \begin{align*}
        \begin{bmatrix} \lambda-\lambda & 0 & 0\\ 1 & \lambda-\lambda & 0\\ 0 & 1 & \lambda-\lambda \end{bmatrix} &= \begin{bmatrix} 0 & 0 & 0\\ 1 & 0 & 0\\ 0 & 1 & 0 \end{bmatrix} \begin{bmatrix} v_1\\ v_2 \\ v_3\end{bmatrix} = \begin{bmatrix} 0\\ 0 \\ 0\end{bmatrix} \\
        \Rightarrow \begin{bmatrix} 0\\ v_1 \\ v_2 \end{bmatrix} = \begin{bmatrix} 0\\ 0 \\ 0\end{bmatrix} &\Rightarrow v_1=v_2=0.
    \end{align*}
    Luego,  el único autovalor es $\lambda$ y el autoespacio asociado es $\{(0,0,t): t\in\mathbb{R}\}$.
    

    \vskip .2cm
    \ref{autovalores-6} Denominamos $F$  a la matriz del enunciado. En realidad la matriz $F$ depende de $\theta$ y  debemos analizar la existencia de autovalores y autovectores para diferentes valores de  $\theta$. El polinomio característico de $F$ es
    \begin{align*}
        \chi_F(x) &= \left|\begin{matrix} x-1 & 0 & 0\\ 0 & x-\cos\theta & -\sen\theta\\ 0 & \sen\theta & x-\cos\theta \end{matrix} \right| \\ 
        &= (x-1)((x-\cos\theta)^2+\sen(\theta)^2) \\
        &= (x-1)(x^2-2x\cos\theta+\cos^2\theta+\sen(\theta)^2) \\
        &= (x-1)(x^2-2x\cos\theta+1).
    \end{align*}
    Por lo tanto $\lambda_1=1$ es un autovalor.  Por otro lado, las raíces del polinomio $x^2-2x\cos\theta+1$ son
    \begin{align*}
        \lambda_{2,3} &= \frac{2\cos\theta-2\sqrt{\cos^2\theta-1}}{2} = \cos\theta \pm \sqrt{\cos^2\theta-1}= \cos\theta\pm\sqrt{-\sen^2\theta}
    \end{align*}

    Debemos dividir en 3 casos, según el valor de $\theta$.

    \vskip .2cm

    \textit{i}) Cuando $\theta=0$ o múltiplo de $2\pi$ tenemos que $\lambda_2 = \lambda_2 =1$. En  ese caso, $F$ es la matriz identidad y por lo tanto hay un único autovalor, 1, y  el autoespacio correspondiente es todo $\R^3$.
    
    \vskip .2cm
    \textit{ii}) Cuando $\theta= \pi + 2k\pi$ con $k\in\mathbb{Z}$ tenemos que $\lambda_2 = \lambda_3 =-1$. En  este caso la matriz es
    $$
    F = \begin{bmatrix}1 & 0& 0 \\ 0 & -1 & 0\\ 0 & 0 & -1 \end{bmatrix}
    $$
    Luego, los autoespacios son $V_1 = \{(t,0,0): t \in \R\}$ y $V_{-1} =\{(0,t,s): t,s \in \R\}$.
    
    \vskip .2cm
    \textit{iii}) Cuando $\theta$ no es múltiplo de $\pi$ tenemos que $-\sen^2\theta < 0$ y por lo tanto el  único autovalor real es $\lambda_1 = 1$. Para calcular el autoespacio asociado a $\lambda_1=1$ resolvemos el sistema  $(\Id-F)v=0$, es decir 
    \begin{align*}
        \begin{bmatrix} 1-1 & 0 & 0\\ 0 & 1-\cos\theta & \sen\theta\\ 0 & -\sen\theta & 1-\cos\theta \end{bmatrix} &= \begin{bmatrix} 0 & 0 & 0\\ 0 & 1-\cos\theta & \sen\theta\\ 0 & -\sen\theta & 1-\cos\theta \end{bmatrix} \begin{bmatrix} v_1\\ v_2 \\ v_3\end{bmatrix} = \begin{bmatrix} 0\\ 0 \\ 0\end{bmatrix} \\
        &\Rightarrow \begin{bmatrix} 0\\ (1-\cos\theta)v_2+\sen\theta v_3 \\ -\sen\theta v_2+(1-\cos\theta)v_3 \end{bmatrix} = \begin{bmatrix} 0\\ 0 \\ 0\end{bmatrix}.
    \end{align*}
    Luego $v_1$ es arbitrario y para ver los valores de $v_2$ y $v_3$ debemos resolver el sistema homogéneo:
    \begin{align*}
        \begin{bmatrix} 0\\ (1-\cos\theta)v_2+\sen\theta v_3 \\ -\sen\theta v_2+(1-\cos\theta)v_3 \end{bmatrix} \underset{F_3/\sen\theta}{\stackrel{F_2/\sen\theta}{\longrightarrow}} \begin{bmatrix} 0\\ \frac{1-\cos\theta}{\sen\theta}v_2+ v_3 \\ - v_2+\frac{1-\cos\theta}{\sen\theta}v_3 \end{bmatrix}
    \end{align*}
    Luego $v_3 = -\frac{1-\cos\theta}{\sen\theta}v_2$ y $v_2 = \frac{1-\cos\theta}{\sen\theta}v_3$, es decir $v_2 = -\frac{1-\cos\theta}{\sen\theta}\frac{1-\cos\theta}{\sen\theta}v_2$ o, equivalentemente $v_2 = -\frac{(1-\cos\theta)^2}{\sen^2\theta}v_2$. Si $v_2 = 0$, entonces $v_3=0$. Si $v_2 \ne 0$, entonces $-\frac{(1-\cos\theta)^2}{\sen^2\theta} = 1$,  lo cual no es posible pues $-\frac{(1-\cos\theta)^2}{\sen^2\theta} <0$.
    \vskip .2cm
    Concluyendo el caso 3): el único autovalor es $\lambda_1 = 1$ y el autoespacio asociado es $\{(t,0,0): t \in \R\}$.

    \vskip .2cm
    \item\label{autovalores-complejos} Calcular los autovalores complejos de las matrices \ref{autovalores-4} y \ref{autovalores-6} del ejercicio anterior, y para cada autovalor, dar una descripción paramétrica del autoespacio asociado sobre $\mathbb{C}$.
    
    \rta
    
    En el caso de la matriz \ref{autovalores-4} del ejercicio anterior, el polinomio característico es $x^2-2x+2$ como ya fue calculado. 
    Por lo tanto sus raíces son
    $$
    \lambda_{1,2} = \frac{2\pm\sqrt{4-8}}{2} = 1\pm i.
    $$
    Para calcular el autoespacio asociado a $\lambda_1 = 1+i$ resolvemos el sistema  $((1+i)\Id-D)v=0$, es decir el sistema homogéneo
    \begin{align*}
        &\begin{bmatrix} 1+i-3 & 5\\ -1 & 1+i+1 \end{bmatrix} = \begin{bmatrix} -2+i & 5\\ -1 & 2+i \end{bmatrix} \stackrel{F_2/(-1)}{\longrightarrow} \begin{bmatrix} -2+i & 5\\ 1 & -2-i \end{bmatrix} \\
        &\qquad\stackrel{F_1 + (2-i) F_2}{\longrightarrow} \begin{bmatrix} 0 & 5-(2+i)(2-i)\\ 1 & -2-i \end{bmatrix} =\begin{bmatrix} 0 & 0\\ 1 & -2-i \end{bmatrix}.
    \end{align*}
    Por lo tanto $v_1 +(-2-i)v_2=0$, es decir $v_1 = (2+i)v_2$. Luego, el autoespacio asociado a $\lambda_1 = 1+i$ es $\{((2+i)t,t): t \in \C\}$.

    Para calcular el autoespacio asociado a $\lambda_2 = 1-i$ resolvemos el sistema  $((1-i)\Id-D)v=0$, es decir el sistema homogéneo
    \begin{align*}
        &\begin{bmatrix} 1-i-3 & 5\\ -1 & 1-i+1 \end{bmatrix} = \begin{bmatrix} -2-i & 5\\ -1 & 2-i \end{bmatrix} \stackrel{F_2/(-1)}{\longrightarrow} \begin{bmatrix} -2-i & 5\\ 1 & -2+i \end{bmatrix} \\
        &\qquad\stackrel{F_1 + (2+i) F_2}{\longrightarrow} \begin{bmatrix} 0 & 5-(2-i)(2+i)\\ 1 & -2+i \end{bmatrix} =\begin{bmatrix} 0 & 0\\ 1 & -2+i \end{bmatrix}.
    \end{align*}
    Por lo tanto $v_1 +(-2+i)v_2=0$, es decir $v_1 = (2-i)v_2$. Luego, el autoespacio asociado a $\lambda_2 = 1-i$ es $\{((2-i)t,t): t \in \C\}$.

    \vskip .2cm

    En la resolución del punto \ref{autovalores-6} del ejercicio anterior, obtuvimos los autovalores $\lambda_1 = 1$ y $\lambda_{2,3} = \cos\theta\pm\sqrt{-\sen^2\theta}$. Luego, $\lambda_2 =\cos\theta + i\sen\theta$ y $\lambda_3 =\cos\theta - i\sen\theta$. Veamos ahora  los autoespacios asociados a los autovalores complejos, es decir cuando $\sen\theta \ne 0$. En  el caso $\lambda_2 =\cos\theta + i\sen\theta$ resolvemos el sistema  $((\cos\theta + i\sen\theta)\Id-F)v=0$, es decir el sistema homogéneo
    \begin{align*}
        &\begin{bmatrix} \cos\theta + i\sen\theta-1 & 0 & 0\\ 0 & \cos\theta + i\sen\theta-\cos\theta & -\sen\theta\\ 0 & \sen\theta & \cos\theta + i\sen\theta-\cos\theta \end{bmatrix} \\
        &= \begin{bmatrix} \cos\theta + i\sen\theta-1 & 0 & 0\\ 0 & i\sen\theta & -\sen\theta\\ 0 & \sen\theta & i\sen\theta \end{bmatrix} \stackrel{F_2\cdot i}{\longrightarrow} \begin{bmatrix} \cos\theta + i\sen\theta-1 & 0 & 0\\ 0 &  -\sen\theta & -i\sen\theta\\ 0 & \sen\theta & i\sen\theta \end{bmatrix} \\
        &\qquad\overset{F_2/\sen\theta}{\stackrel{F_3/\sen\theta}{\longrightarrow}} \begin{bmatrix} \cos\theta + i\sen\theta-1 & 0 & 0\\ 0 & -1 & -i\\ 0 & 1 & i \end{bmatrix} \stackrel{F_3+F_2}{\longrightarrow}\begin{bmatrix} \cos\theta + i\sen\theta-1 & 0 & 0\\ 0 & -1 & -i\\ 0 & 0 & 0 \end{bmatrix}
    \end{align*}
    Por lo tanto $v_1 = 0$, y $-v_2-iv_3=0$, es decir $v_2 = -iv_3$. Luego, el autoespacio asociado a $\lambda_2 =\cos\theta + i\sen\theta$ es $\{t(0,-i,1): t \in \C\}$.

    En el caso $\lambda_3 =\cos\theta - i\sen\theta$ resolvemos el sistema  $((\cos\theta - i\sen\theta)\Id-F)v=0$, es decir el sistema homogéneo
    \begin{align*}
        &\begin{bmatrix} \cos\theta - i\sen\theta-1 & 0 & 0\\ 0 & \cos\theta - i\sen\theta-\cos\theta & -\sen\theta\\ 0 & \sen\theta & \cos\theta - i\sen\theta-\cos\theta \end{bmatrix} \\
        &= \begin{bmatrix} \cos\theta - i\sen\theta-1 & 0 & 0\\ 0 & -i\sen\theta & -\sen\theta\\ 0 & \sen\theta & -i\sen\theta \end{bmatrix} \stackrel{F_2\cdot i}{\longrightarrow} \begin{bmatrix} \cos\theta - i\sen\theta-1 & 0 & 0\\ 0 &  \sen\theta & -i\sen\theta\\ 0 & \sen\theta & -i\sen\theta \end{bmatrix} \\
        &\qquad\overset{F_2/\sen\theta}{\stackrel{F_3/\sen\theta}{\longrightarrow}} \begin{bmatrix} \cos\theta - i\sen\theta-1 & 0 & 0\\ 0 & 1 & -i\\ 0 & 1 & -i \end{bmatrix} \stackrel{F_3-F_2}{\longrightarrow}\begin{bmatrix} \cos\theta - i\sen\theta-1 & 0 & 0\\ 0 & 1 & -i\\ 0 & 0 & 0 \end{bmatrix} 
    \end{align*}
    Por lo tanto $v_1 = 0$, y $v_2-iv_3=0$, es decir $v_2 = iv_3$. Luego, el autoespacio asociado a $\lambda_3 =\cos\theta - i\sen\theta$ es $\{t(0,i,1): t \in \C\}$.

    \vskip .2cm

    \noindent \textbf{Conclusión.} Haremos un resumen de  los resultados acerca de autovalores y autovectores de la matriz del ejercicio \ref{autovalores} \ref{autovalores-6}, es decir de la matriz
    \begin{equation*}
        \left[\begin{matrix}1 & 0& 0 \\ 0 & \cos\theta & \sen\theta\\ 0 & -\sen\theta & \cos\theta \end{matrix} \right].
    \end{equation*}

    \begin{itemize}
        \item Si $\theta=0$ o múltiplo de $2\pi$, entonces $F$ tiene un único autovalor, 1, y  el autoespacio correspondiente es todo $\R^3$.
        \item Si $\theta= \pi + 2k\pi$ con $k\in\mathbb{Z}$, entonces $F$ tiene dos autovalores, 1 y $-1$, y  los autoespacios correspondientes son $V_1 = \{(t,0,0): t \in \R\}$ y $V_{-1} =\{(0,t,s): t,s \in \R\}$.
        \item Si $\theta$ no es múltiplo de $\pi$, entonces $F$ tiene un único autovalor real, 1, y  el autoespacio correspondiente es $\{(t,0,0): t \in \R\}$, y tiene dos autovalores complejos, $\lambda_2 =\cos\theta + i\sen\theta$ y $\lambda_3 =\cos\theta - i\sen\theta$, cuyos autoespacios son $\{t(0,-i,1): t \in \C\}$ y $\{t(0,i,1): t \in \C\}$, respectivamente.
    \end{itemize}


    
    


    \vskip .2cm

    \end{enumerate}
    

    
    
    \begin{enumerate}[resume,topsep=6pt,itemsep=.4cm]
    
    \item Probar que hay una única matriz $A\in\mathbb{R}^{2\times 2}$ tal que $(1,1)$ es autovector de autovalor $2$, y $(-2,1)$ es autovector de autovalor $1$.
    
    \rta


    \vskip .2cm
    
        
    
    \item Sea $A\in\mathbb{K}^{n\times n}$, y sea $f(x) = ax^2+bx+c$ un polinomio, con $a,b,c\in\mathbb{K}$. Sea $f(A)$ la matriz $n \times n$ definida por
    $$f(A) = a A^2+bA+c\operatorname{Id}_n.$$
    Probar que todo autovector de $A$ con autovalor $\lambda$ es autovector de $f(A)$ con autovalor $f(\lambda)$.
    
    \rta


    \vskip .2cm
    
    
        
    \item Sea $A\in\mathbb{K}^{2\times 2}$.
    
        \begin{enumerate}     
            \item Probar que el polinomio característico de $A$ es \ $\chi_A(x) = x^2-\operatorname{Tr}(A)x+\det(A)$.
            \item Si $A$ no es invertible, probar que los autovalores de  $A$ son $0$ y $\operatorname{Tr}(A)$.
        \end{enumerate}
    
    \rta


    \vskip .2cm
        
        
    \item Sea $A\in\mathbb{K}^{n\times n}$. Probar que el polinomio $\tilde\chi_A(x)=\det(A-x\operatorname{Id}_n)$ y el polinomio característico de $A$ tienen las mismas raíces.
    
    \rta


    \vskip .2cm
    
    
    \end{enumerate}
    
    \begin{enumerate}[resume,topsep=6pt,itemsep=.4cm]
    
    \item Probar que si $A\in\mathbb{K}^{n\times n}$ es una matriz nilpotente entonces $0$ es el único autovalor de $A$. Usar esto para deducir que la matriz $\operatorname{Id}_n-A$ es invertible (esta es otra demostración del ejercicio \ref{nilpotene - id} del Práctico \ref{practico-3}).
    
    \rta


    \vskip .2cm
    
    
    
    \item Decidir si las siguientes afirmaciones son verdaderas o falsas. Justificar.
    \begin{enumerate}
        \item Existe una matriz invertible $A$ tal que $0$ es autovalor de $A$.
        \item  Si $A$ es invertible, entonces todo autovector de $A$ es autovector de $A^{-1}$.
    \end{enumerate}
    
    \rta


    \vskip .2cm
    
    
    \end{enumerate}
    
    \begin{enumerate}[resume,topsep=6pt,itemsep=.4cm]
    
    \item\label{mas} Repetir los ejercicios \ref{autovalores} y \ref{autovalores-complejos} con las siguientes matrices.
    \begin{multicols}{2}
        \begin{enumerate}
            \item\label{mas-autovalores-1} $\begin{bmatrix} 2 & 3 \\ -1 & 1
            \end{bmatrix} $,\vskip .6cm 
            \item\label{mas-autovalores-2} $\begin{bmatrix} -9 & 4 & 4 \\ -8 & 3 & 4 \\ -16 & 8 & 7 \end{bmatrix} $, \vskip .5cm
            \item\label{mas-autovalores-3} $\left[\begin{matrix}4 & 4 & -12\\ 1 & -1 & 1\\ 5 & 3 & -11 \end{matrix} \right] $,\vskip .2cm
            \item\label{mas-autovalores-4} $\left[\begin{matrix}2 & 1 & 0 & 0\\ -1 & 4 & 0 & 0\\ 0 & 0 & 1 & 1 \\ 0 & 0 & 3 & -1\end{matrix} \right] $,\vskip .2cm
            \item\label{mas-autovalores-5} $\begin{bmatrix} \lambda & 0 & 0 & \dots & 0  \\ 1 & \lambda & 0 &\dots & 0  \\ 0 & 1 & \lambda&  \dots & 0  \\ \vdots & \vdots & \quad & \ddots & \vdots\\ 0 &  0&   \dots & 1  & \lambda \end{bmatrix}$, $\lambda\in \mathbb R$. \vskip .2cm
        \end{enumerate}
        \end{multicols}

    \rta


    \vskip .2cm
    
    
    

    \end{enumerate}
%%%=======================
%%% CAP6 =================
    
\chapter{Soluciones\\Álgebra  II -- Año 2024/1 -- FAMAF}\label{practico-6}


\begin{enumerate}[topsep=6pt, itemsep=.4cm]

    
    \item\label{sub Rn} Decidir si los siguientes subconjuntos de $\mathbb{R}^3$ son subespacios vectoriales.
        \begin{enumerate}
            \item\label{sub Rn 1} $A=\{(x_1, x_2 ,x_3) \in \mathbb{R}^3 \ : \ x_1 + x_2 + x_3=1\}$.
            \item\label{sub Rn 0} $B=\{(x_1, x_2 ,x_3) \in \mathbb{R}^3 \ : \ x_1 + x_2 + x_3=0\}$.
            \item\label{sub Rn geq} $C=\{(x_1, x_2 ,x_3) \in \mathbb{R}^3 \ : \ x_1 + x_2 + x_3 \geq 0\}$.
            \item\label{sub Rn 1 30} $D=\{(x_1, x_2 ,x_3) \in \mathbb{R}^3 \ : \ x_3=0\}$.
            \item\label{sub Rn cup} $B\cup D$.
            \item\label{sub Rn cap} $B\cap D$.
            \item\label{sub Rn q} $G=\{(x_1, x_2 ,x_3) \in \mathbb{R}^3 \ :\ x_1, x_2, x_3\in\mathbb{Q}\}$.
        \end{enumerate}
        
    \rta 

    \ref{sub Rn 1} No es subespacio vectorial. Por ejemplo $(1,0,0)$ y $(0,1,0)$ pertenecen a $A$, pero $(1,0,0)+(0,1,0)=(1,1,0)$ no pertenece a $A$.

    \ref{sub Rn 0} Es subespacio vectorial. En efecto, si $(x_1, x_2 ,x_3)$ y $(y_1, y_2 ,y_3)$ pertenecen a $B$ y $\lambda,\mu\in\mathbb{R}$, entonces
    \begin{align*}
        \lambda(x_1, x_2 ,x_3)+\mu(y_1, y_2 ,y_3)&=(\lambda x_1+\mu y_1, \lambda x_2+\mu y_2, \lambda x_3+\mu y_3)\\
        &=(\lambda x_1+\mu y_1)+(\lambda x_2+\mu y_2)+(\lambda x_3+\mu y_3)\\
        &=\lambda(x_1+ x_2 + x_3)+\mu(y_1+ y_2 + y_3)\\
        &=\lambda\cdot 0+\mu\cdot 0=0.
    \end{align*}

    \ref{sub Rn geq} No es subespacio vectorial. Por ejemplo $(1,0,0) \in C$ pero $(-1)(1,0,0) = (-1,0,0) \not\in C$, pues $-1+0+0<0$.

    \ref{sub Rn 1 30} Es subespacio vectorial. En efecto, si $(x_1, x_2 ,0)$ y $(y_1, y_2 ,0)$ pertenecen a $D$ y $\lambda,\mu\in\mathbb{R}$, entonces
    \begin{align*}
        \lambda(x_1, x_2 ,0)+\mu(y_1, y_2 ,0)&=(\lambda x_1+\mu y_1, \lambda x_2+\mu y_2, 0) \in D.
    \end{align*}

    \ref{sub Rn cup} No es subespacio vectorial. Por ejemplo $(1,0,-1) \in B$ y $(0,1,0)$ pertenecen a $B\cup D$, pero $(1,0,-1)+(0,1,0)=(1,1,-1)$ no pertenece a $B\cup D$, pues $(1,1,-1) \not\in B$ y $(1,1,-1) \not\in D$.

    \ref{sub Rn cap} Es subespacio vectorial
    \begin{align*}
        B \cap D &= \{(x_1, x_2 ,x_3) \in \mathbb{R}^3 \ : \ x_1 + x_2 + x_3=0 \text{ y } x_3=0\} \\&= \{(x_1, x_2 ,0) \in \mathbb{R}^3 \ : \ x_1 + x_2 =0\}.    
    \end{align*}
    Luego, si $(x_1, x_2 ,0)$ y $(y_1, y_2 ,0)$ pertenecen a $B\cap D$ y $\lambda,\mu\in\mathbb{R}$, entonces
    \begin{align*}
        \lambda(x_1, x_2 ,0)+\mu(y_1, y_2 ,0)&=(\lambda x_1+\mu y_1, \lambda x_2+\mu y_2, 0) \in B\cap D,
    \end{align*}
    pues $\lambda x_1+\mu y_1 + \lambda x_2+\mu y_2 = \lambda(x_1+ x_2) + \mu(y_1+ y_2) = \lambda(x_1+ x_2) + \mu(y_1+ y_2) = \lambda 0 + \mu 0 = 0$.

    \ref{sub Rn q} No es subespacio vectorial. Por ejemplo $(1,0,0)$ pertenece a $G$, pero $\sqrt{2}(1,0,0)=(\sqrt{2},0 ,0)$ no pertenece a $G$.


    \qed     

    \end{enumerate}

    
    \begin{enumerate}[resume, topsep=6pt, itemsep=.4cm]
    
    \item\label{sub matrices} Decidir en cada caso si el conjunto dado es un subespacio vectorial de $M_{n\times n}(\mathbb{K})$.
    \begin{enumerate}
        \item\label{sub matrices invertibles} El conjunto de matrices  invertibles.
        \item\label{sub matrices AB} El conjunto de matrices $A$ tales que $AB = BA$, donde $B$ es una matriz fija.
        \item\label{sub matrices triangulares} El conjunto de matrices triangulares superiores.
    \end{enumerate}
    
    \rta 

    \ref{sub matrices invertibles} No es subespacio vectorial. Por ejemplo, $\Id_n$ y $-\Id_n$ son matrices invertibles,  pero $\Id_n+(-\Id_n)=0$ no es invertible.

    \vskip .2cm
    \ref{sub matrices AB} Es subespacio vectorial. En efecto, si $A$ y $A'$ pertenecen al conjunto y $\lambda,\mu\in\mathbb{R}$, entonces
    \begin{align*}
        (\lambda A+\mu A')B&=\lambda AB+\mu A'B&&\\
        &=\lambda BA+\mu BA'&&\text{(por hipótesis)}\\
        &=(\lambda A+\mu A')B.
    \end{align*}
    Luego $\lambda A+\mu A'$ pertenece al conjunto.

    \vskip .2cm
    \ref{sub matrices triangulares} Es subespacio vectorial. En efecto, si $A$ y $A'$ son matrices triangulares superiores y $\lambda,\mu\in\mathbb{R}$, entonces
    \begin{align*}
        \lambda A+\mu A' &=\lambda\begin{bmatrix}
            a_{11} & a_{12} & \cdots & a_{1n} \\
            0 & a_{22} & \cdots & a_{2n} \\
            \vdots & \vdots & \ddots & \vdots \\
            0 & 0 & \cdots & a_{nn} \\
        \end{bmatrix}+\mu\begin{bmatrix}
            a'_{11} & a'_{12} & \cdots & a'_{1n} \\
            0 & a'_{22} & \cdots
            & a'_{2n} \\
            \vdots & \vdots & \ddots & \vdots \\
            0 & 0 & \cdots & a'_{nn} \\
        \end{bmatrix}\\
        &=\begin{bmatrix}
            \lambda a_{11} & \lambda a_{12} & \cdots & \lambda a_{1n} \\
            0 & \lambda a_{22} & \cdots & \lambda a_{2n} \\
            \vdots & \vdots & \ddots & \vdots \\
            0 & 0 & \cdots & \lambda a_{nn} \\
        \end{bmatrix}+\begin{bmatrix}
            \mu a'_{11} & \mu a'_{12} & \cdots & \mu a'_{1n} \\
            0 & \mu a'_{22} & \cdots & \mu a'_{2n} \\
            \vdots & \vdots & \ddots & \vdots \\
            0 & 0 & \cdots & \mu a'_{nn} \\
        \end{bmatrix}\\
        &=\begin{bmatrix}
            \lambda a_{11}+\mu a'_{11} & \lambda a_{12}+\mu a'_{12} & \cdots & \lambda a_{1n}+\mu a'_{1n} \\
            0 & \lambda a_{22}+\mu a'_{22} & \cdots & \lambda a_{2n}+\mu a'_{2n} \\
            \vdots & \vdots & \ddots & \vdots \\
            0 & 0 & \cdots & \lambda a_{nn}+\mu a'_{nn} \\
        \end{bmatrix}.
    \end{align*}
    Luego $\lambda A+\mu A'$ es una matriz triangular superior.

    \qed     
    
    
    \item\label{rectas} Sea $L$ una recta en $\mathbb{R}^2$. Dar una condición necesaria y suficiente para que $L$ sea un subespacio vectorial de $\mathbb{R}^2$.
    
    
    \rta Una recta en $\mathbb{R}^2$ es un subespacio vectorial si y sólo si pasa por el origen. 
    
    La ecuación general de la recta en el plano es $ax+by=c$ con $a,b,c\in\mathbb{R}$ y $a,b$ no ambos nulos.

    ($\Rightarrow$) Si $L$ es un subespacio vectorial, entonces $(0,0) \in L$,  es decir la recta pasa por el origen. Además, como  $0 = a\cdot 0 + b \cdot 0= c$, la ecuación de la recta es  $ax+by=0$.

    ($\Leftarrow$) Si la recta pasa por el origen, entonces $0 = a\cdot 0 + b \cdot 0= c$, es decir la ecuación de la recta es  $ax+by=0$. Luego, si $(x_1,y_1)$ y $(x_2,y_2)$ pertenecen a la recta y $\lambda,\mu\in\mathbb{R}$, entonces
    \begin{align*}
        a(\lambda x_1+\mu x_2)+b(\lambda y_1+\mu y_2)&=\lambda(ax_1+by_1)+\mu(ax_2+by_2)\\
        &=\lambda\cdot 0+\mu\cdot 0=0.
    \end{align*}
    Luego $\lambda(x_1,y_1)+\mu(x_2,y_2)$ pertenece a la recta y por lo tanto la recta es un subespacio vectorial.

    \qed     
    
    \item Sean $V$ un $\mathbb{K}$-espacio vectorial, $v\in V$ no nulo y $\lambda,\mu\in\mathbb{K}$ tales que $\lambda v=\mu v$. Probar que $\lambda=\mu$.
    
    
    \rta Si $\lambda v=\mu v$, entonces $(\lambda-\mu)v=0$. Supongamos que $\lambda-\mu\neq 0$, entonces
    \begin{align*}
        v&= 1 \cdot v &&\text{(axioma P1 de esp. vectoriales)}\\
        &=(\lambda-\mu)^{-1}(\lambda-\mu)v&&\\
        &=(\lambda-\mu)^{-1}0&&\text{(por hipótesis)}\\
        &=0.&&\text{(demostrado en la teórica: $0\cdot v = 0$)}
    \end{align*}
    Concluimos que $v=0$, lo cual contradice la hipótesis. El absurdo vino de suponer que $\lambda-\mu\neq 0$, luego $\lambda=\mu$.

    \qed     
    
    \item Sean $W_1, W_2$ subespacios de un espacio vectorial $V$. Probar que $W_1 \cup W_2$ es un subespacio  de $V$ si y sólo si $W_1 \subseteq W_2$ o $W_2 \subseteq W_1$.
        
    
    \rta 

    ($\Rightarrow$)  Si $W_1 \subseteq W_2$ o $W_2 \subseteq W_1$ no hay nada que demostrar. Supongamos entonces que $W_1 \not\subseteq W_2$ y $W_2 \not\subseteq W_1$. Entonces existen $w_1\in W_1$ tal que $w_1\not\in W_2$ y $w_2\in W_2$ tal que $w_2\not\in W_1$. Como $w_1\in W_1$ y $w_2\in W_2$, entonces $w_1+w_2\in W_1\cup W_2$. Como $W_1 \cup W_2$ es un subespacio  de $V$, entonces $w_1+w_2\in W_1\cup W_2$ y por lo tanto $w_1+w_2\in W_1$ o $w_1+w_2\in W_2$. Supongamos que $w_1+w_2\in W_1$, entonces $w_2=w_1+w_2-w_1\in W_1$, lo cual es absurdo.Análogamente,  si $w_1+w_2\in W_2$, entonces $w_1\in W_2$, lo cual es absurdo. El absurdo vino de suponer que $W_1 \not\subseteq W_2$ y $W_2 \not\subseteq W_1$, luego $W_1 \subseteq W_2$ o $W_2 \subseteq W_1$.

    ($\Leftarrow$) Supongamos que $W_1 \subseteq W_2$. Entonces $W_1 \cup W_2 = W_2$ y por lo tanto $W_1 \cup W_2$ es un subespacio  de $V$. Análogamente se demuestra que si $W_2 \subseteq W_1$, entonces $W_1 \cup W_2$ es un subespacio  de $V$.

    \qed     
        
    \item Sean $u=(1,1)$, $v=(1,0)$, $w=(0,1)$ y $z=(3,4)$ vectores de $\mathbb{R}^2$.
    \begin{enumerate}
    \item\label{comb-lin-u-v-w} Escribir $z$ como combinación lineal de $u,v$ y $w$, con coeficientes todos no nulos.
    \item\label{comb-lin-u-v} Escribir $z$ como combinación lineal de $u$ y $v$.
    \item\label{comb-lin-u-w} Escribir $z$ como combinación lineal de $u$ y $w$.
    \item\label{comb-lin-v-w}Escribir $z$ como combinación lineal de $v$ y $w$.
    \end{enumerate}
    
    \rta En general tenemos que resolver la ecuación $z=\lambda u+\mu v+\nu w$, bajo ciertas condiciones sobre $\lambda,\mu,\nu$. En cada caso, las condiciones son distintas. Si escribimos en coordenadas la ecuación es
    \begin{align*}
        (3,4)&=\lambda (1,1)+\mu (1,0)+\nu (0,1)\\
        &=(\lambda+\mu,\lambda+\nu).    \tag{*}
    \end{align*}
    El sistema es sencillo de resolver, pues la segunda coordenada nos dice que $\lambda+\nu=4$, es decir $\nu=4-\lambda$. Reemplazando en la primera coordenada obtenemos $\lambda+\mu=3$, es decir $\mu=3-\lambda$. Por lo tanto, $\lambda$ es libre y
    \begin{equation*}
        z=\lambda u+\mu v+\nu w=\lambda (1,1)+(3-\lambda) (1,0)+(4-\lambda) (0,1).
    \end{equation*}
    
    \ref{comb-lin-u-v-w} Si $\lambda=1$, entonces $\mu=2$ y $\nu=3$ y por lo tanto
    \begin{equation*}
        z=1\cdot u+2\cdot v+3\cdot w.
    \end{equation*}

    \ref{comb-lin-u-v} Si $\lambda=4$, entonces $\mu=-1$ y $\nu=0$ y por lo tanto
    \begin{equation*}
        z=4\cdot u-1\cdot v.
    \end{equation*}

    \ref{comb-lin-u-w} Si $\lambda=3$, entonces $\mu=0$ y $\nu=1$ y por lo tanto
    \begin{equation*}
        z=3\cdot u+1\cdot w.
    \end{equation*}

    \ref{comb-lin-v-w} Si $\lambda=2$, entonces $\mu=1$ y $\nu=2$ y por lo tanto
    \begin{equation*}
        z=2\cdot v+2\cdot w.
    \end{equation*}
    \qed     
    
    \item Sean $p(x)=(1-x)(x+2)$, $q(x)=x^2-1$ y $r(x)=x(x^2-1)$ en $\mathbb{R}[x]$.
        \begin{enumerate}
        \item\label{comb-lineal-pol-a} Describir todos los polinomios de grado menor o igual que $3$ que son combinación lineal de $p,q$ y $r$.
        \item\label{comb-lineal-pol-b} Elegir $a$ tal que el polinomio $x$ se pueda escribir como combinación lineal de $p,q$ y $2x^2+a$.
        \end{enumerate}
    
    \rta 

    \ref{comb-lineal-pol-a} Escribamos la versión expandida de $p$, $q$ y $r$:
    \begin{align*}
        p(x)&=x^2+x-2,\\
        q(x)&=x^2-1,\\
        r(x)&=x^3-x.                
    \end{align*}
    Debemos encontrar el subespacio generado por estos tres polinomios. Primero encontraremos una base del subespacio en término de los generadores canónicos  ($x^n$ con $n\in\mathbb{N}_0$). 
    \begin{align*}
        &\begin{bmatrix}
            0 & 1 & 1 & -2 \\
            0 & 1 & 0 & -1 \\
            1 & 0 & -1 & 0 \\
        \end{bmatrix}
        \stackrel{F_2-F_1}{\longrightarrow}
        \begin{bmatrix}
            0 & 1 & 1 & -2 \\
            0 & 0 & -1 & 1 \\
            1 & 0 & -1 & 0 \\
        \end{bmatrix} \\
        &\underset{F_3-F_2}{\stackrel{F_1+F_2}{\longrightarrow}}
        \begin{bmatrix}
            0 & 1 & 0 & -1 \\
            0 & 0 & -1 & 1 \\
            1 & 0 & 0 & -1 \\
        \end{bmatrix} \stackrel{-F_2}{\longrightarrow}
        \begin{bmatrix}
            0 & 1 & 0 & -1 \\
            0 & 0 & 1 & -1 \\
            1 & 0 & 0 & -1 \\
        \end{bmatrix}.            
    \end{align*}
    Luego  
    $$
    \langle p,q,r\rangle = \langle x^2-1,x-1,x^3-1 \rangle.
    $$




    \ref{comb-lineal-pol-a} Debemos estudiar la ecuación 
    \begin{align*}
        ax^3+bx^2+cx +d &=\lambda (1-x)(x+2)+\mu (x^2-1)+\nu x(x^2-1)\\
        &=\lambda(-x^2 -x +2)+\mu (x^2-1)+\nu (x^3-x) \\
        &= \nu x^3 + (\mu-\lambda)x^2 + (-\nu-\lambda)x + (2\lambda-\mu).
    \end{align*}
    Debemos encontrar todos los $(a,b,c,d)\in\mathbb{R}^4$ tales que existe $\lambda,\mu,\nu\in\mathbb{R}$ que satisfacen la ecuación anterior. Es decir, debemos encontrar todos los $(a,b,c,d)\in\mathbb{R}^4$ tales que existe $\lambda,\mu,\nu\in\mathbb{R}$ que satisfacen el sistema
    \begin{align*}
        a&=\nu\\
        b&=\mu-\lambda\\
        c&=-\nu-\lambda\\
        d&=2\lambda-\mu.
    \end{align*}
    Si consideramos $a,b,c,d$ como constantes y  $\lambda,\mu,\nu$ como incógnitas, entonces el sistema, presentado como matriz aumentada es:
    \begin{align*}
        &\begin{amatrix}{3}
            0 & 0 & 1 & a \\
            -1 & 1 & 0 & b \\
            -1 & 0 & -1 & c \\
            2 & -1 & 0 & d
        \end{amatrix}
        \underset{F_4 + 2F_2}{\stackrel{F_3-F_2}{\longrightarrow}}
        \begin{amatrix}{3}
            0 & 0 & 1 & a \\
            -1 & 1 & 0 & b \\
            0 & -1 & -1 & c-b \\
            0 & 1 & 0 & d+2b
        \end{amatrix} \\
        &\underset{F_3 + F_4}{\stackrel{F_2-F_4}{\longrightarrow}}
        \begin{amatrix}{3}
            0 & 0 & 1 & a \\
            -1 & 0 & 0 & -b-d \\
            0 & 0 & -1 & b+c+d \\
            0 & 1 & 0 & d+2b
        \end{amatrix} \underset{F_3\cdot (-1)}{\stackrel{F_2\cdot (-1)}{\longrightarrow}} 
        \begin{amatrix}{3}
            0 & 0 & 1 & a \\
            1 & 0 & 0 & b+d \\
            0 & 0 & 1 & -b-c-d \\
            0 & 1 & 0 & d+2b
        \end{amatrix} \\
    \end{align*}
    Es decir, el sistema se reduce a 
    \begin{align*}
        \lambda &= b+d\\
        \mu &= 2b+d\\
        \nu &= a\\
        \nu&= -b-c-d.
    \end{align*}
    Luego el sistema tiene solución si y sólo si $a=-b-c-d$ por lo tanto, el subespacio de polinomios que obtenemos es
    \begin{equation*}
        \{(-b-c-d) x^3 + bx^2 + cx + d: \ b,c,d\in\mathbb{R}\}.
    \end{equation*}

    \qed     
    
    \item\label{practicos anteriores} Dar un conjunto de generadores para los siguientes subespacios vectoriales.
    \begin{enumerate}
    \item Los conjuntos de soluciones de los sistemas homogéneos del ejercicio \ref{sistemas homogeneos} del Práctico \ref{practico-2}.
    \item Los conjuntos descriptos en el ejercicio \ref{sistemas con soluciones} del Práctico  \ref{practico-2}.
    \end{enumerate}
    
    \rta 

    \qed     
    
    \item\label{caracterizar}  En cada caso, caracterizar con ecuaciones al subespacio vectorial dado por generadores.
    \begin{enumerate}
    \item ${\left\langle(1,0,3),(0,1,-2)\right\rangle}\subseteq \mathbb{R}^3$.
    \item ${\left\langle(1,2,0,1),(0,-1,-1,0),(2,3,-1,4)\right\rangle}\subseteq \mathbb{R}^4$.
    \end{enumerate}
    
    
    \rta 

    \qed     
    
    \item\label{son LI} En cada caso, determinar si el subconjunto indicado es linealmente independiente.
    \begin{enumerate}
        \item $\{ (1,0,-1), (1,2,1), (0,-3,2) \}\subseteq \mathbb{R}^3$.
        \item $\left\{  \begin{bmatrix} 1 & 0 & 2 \\ 0 & -1 & -3 \\ \end{bmatrix}, \quad
        \begin{bmatrix} 1 & 0 & 1 \\ -2 & 1 & 0 \\ \end{bmatrix}, \quad
        \begin{bmatrix} 1 & 2 & 3 \\ 3 & 2 & 1 \\ \end{bmatrix} \right\}\subseteq M_{2\times 3}(\mathbb{R})$.
    \end{enumerate}
    
    
    \rta 

    \qed     
    
    \item Dar un ejemplo de un conjunto de 3 vectores en $\mathbb{R}^3$ que sean LD, y tales que dos cualesquiera de ellos sean LI.
    
    
    \rta 

    \qed     
    
    \item  Probar que si $\alpha$, $\beta$ y $\gamma$ son vectores LI en el $\mathbb{R}$-espacio vectorial $V$, entonces $\alpha +\beta$, $\alpha +\gamma$ y $\beta +\gamma $ también son LI.
    
    
    \rta 

    \qed     
    
    \item Extender, de ser posible, los siguientes conjuntos a una base de los respectivos espacios vectoriales.
    
    \begin{enumerate}
        \item Los conjuntos del ejercicio \ref{son LI}.
        \item\label{10b} $\{ (1,2,0,0),(1,0,1,0) \}\subseteq\mathbb{R}^4$.
        \item\label{10c} $\{ (1,2,1,1),(1,0,1,1),(3,2,3,3)\}\subseteq\mathbb{R}^4$.
    \end{enumerate}
    
    
    \rta 

    \qed     
    
    \item Dar subespacios vectoriales $W_0$, $W_1$, $W_2$ y $W_3$ de $\mathbb{R}^3$ tales que $W_0\subset W_1\subset W_2\subset W_3$ y $\dim W_0=0$, $\dim W_1=1$, $\dim W_2=2$ y $\dim W_3=3$.
    
    
    \rta 

    \qed     
    
    \item Sea $V$ un espacio vectorial de dimensión $n$ y $\mathcal{B}=\{v_1, ..., v_n\}$ una base de $V$.
    \begin{enumerate}
    \item Probar que cualquier subconjunto no vacío de $\mathcal{B}$ es LI.
    \item Para cada $k\in\mathbb{N}_0$,  con $0\leq k\leq n$, dar un subespacio vectorial de $V$ de dimensión $k$.
    \end{enumerate}
    
    
    \rta 

    \qed     
    
    \item Dar una base y calcular la dimensión de $\mathbb{C}^n$ como $\mathbb{C}$-espacio vectorial y como $\mathbb{R}$-espacio vectorial.
    
    
    \rta 

    \qed     
    
    \item  Exhibir una base y calcular la dimensión de los siguientes subespacios.
    \begin{enumerate}
        \item Los subespacios del ejercicio \ref{practicos anteriores}.
        \item $W = \{(x,y,z,w,u) \in \mathbb{R}^5 \ : \ y = x - z,\, w = x + z,\,  u = 2x - 3z \}$.
        \item $W = \langle (1, 0, -1, 1),  (1, 2, 1, 1), (0, 1, 1, 0), (0, -2, -2, 0) \rangle \subseteq \mathbb R^4$.
        \item Matrices triangulares superiores $2\times 2$ y $3\times 3$.
        \item Matrices triangulares superiores $n\times n$ para cualquier $n\in\mathbb{N}$, $n\geq 2$.
    \end{enumerate}
    
    \rta 

    \qed     
    
    \item Sean $W_1$ y $W_2$ los siguientes subespacios de $\mathbb{R}^3$:
        \begin{align*}
        W_1 &= \{ (x,y,z)\in\mathbb{R}^3\ : \ x+y-2z=0\},  \\
        W_2 &= {\left\langle(1,-1,1),(2,1,-2),(3,0,-1)\right\rangle}.
        \end{align*}
        \begin{enumerate}
            \item  Determinar $W_1 \cap W_2$, y describirlo por generadores y con ecuaciones.
            \item  Determinar $W_1+W_2$, y describirlo por generadores y con ecuaciones.
        \end{enumerate}
    
    
    \rta 

    \qed     
        
    \item\label{verdadero o falso} Decidir si las siguientes afirmaciones son verdaderas o falsas. Justificar.
    
    \begin{enumerate}
    \item Si $W_1$ y $W_2$ son subespacios vectoriales de $\mathbb{K}^8$ de dimensión $5$, entonces $W_1\cap W_2=0$.
    \item Si $W$ es un subespacio de $\mathbb{K}^{2\times2}$ de dimensión $2$, entonces existe una matriz triangular superior no nula que pertence a $W$.
    \item Sean $v_1, v_2, w\in \mathbb{K}^{n}$ y $A\in\mathbb{K}^{n\times n}$ tales que $Av_1=Av_2=0\neq Aw$. Si $\{v_1, v_2\}$ es LI, entonces $\{v_1,v_2,w\}$ también es LI.
    \item\label{cos}  $\{1,{\rm sen}(x),\cos(x)\}$ es un subconjunto LI del espacio de funciones de $\mathbb{R}$ en $\mathbb{R}$.
    \item\label{cos2}  $\{1,{\rm sen}^2(x),\cos^2(x)\}$ es un subconjunto LI del espacio de funciones $\mathbb{R}$ en $\mathbb{R}$.
    \item\label{exponencial}  $\{e^{\lambda_1x},e^{\lambda_2x},e^{\lambda_3x}\}$ es un subconjunto LI del espacio de funciones de
    $\mathbb{R}$ en $\mathbb{R}$, si $\lambda_1$, $\lambda_2$ y $\lambda_3$ son todos distintos.
    \end{enumerate}
    
    
    \rta 

    \qed     
    
    
    
    
    \end{enumerate}
    
    
%%%======================= 
%%% CAP7 =================
    
\chapter{Soluciones\\Álgebra  II -- Año 2024/1 -- FAMAF}\label{practico-7}

\begin{enumerate}[topsep=6pt, itemsep=.4cm]
    \item\label{transf-lineales-a} Decidir si las siguientes funciones son transformaciones lineales entre los respectivos espacios vectoriales sobre $\mathbb{K}$.
    \begin{enumerate}[resume, topsep=5pt,itemsep=5pt]
    \item\label{transf-lineales-b} La traza $\operatorname{Tr}:\mathbb{K}^{n\times n}\longrightarrow\mathbb{K}$ (recordar ejercicio \ref{traza}\,\ref{ej:traza} del Práctico  \ref{practico-3}) 
    \item\label{transf-lineales-} $T:\mathbb{K}[x]\longrightarrow\mathbb{K}[x]$, $T(p(x))=q(x)\,p(x)$ donde $q(x)$ es un polinomio fijo.
    \item\label{transf-lineales-c} $T:\mathbb{K}^2\longrightarrow\mathbb{K}$, $T(x,y)=xy$
    \item\label{transf-lineales-d} $T:\mathbb{K}^2\longrightarrow\mathbb{K}^3$, $T(x,y)=(x,y,1)$
    \item\label{transf-lineales-e} El determinante $\operatorname{det}:\mathbb{K}^{n\times n}\longrightarrow\mathbb{K}$.
    \end{enumerate}

    \rta
    
    
    \item Sea $T:\mathbb{C}\longrightarrow\mathbb{C}$, $T(z)=\overline{z}$.
    \begin{enumerate}
    \item\label{conj-C} Considerar a $\mathbb{C}$ como un $\mathbb{C}$-espacio vectorial y decidir si $T$ es una transformación lineal.
    \item\label{conj-R} Considerar a $\mathbb{C}$ como un $\mathbb{R}$-espacio vectorial y decidir si $T$ es una transformación lineal.
    \end{enumerate}
    
    \rta
    
    
    
    \item\label{T en la base} Sea $T:\mathbb{K}^3\longrightarrow\mathbb{K}^3$ una transformación lineal tal que $T(e_1)=(1,2,3)$, $T(e_2)=(-1,0,5)$ y $T(e_3)=(-2,3,1)$. 
        \begin{enumerate}
        \item\label{T en dos vectores} Calcular $T(2,3,8)$ y $T(0,1,-1)$. 
        \item\label{T en la base b} Calcular $T(x,y,z)$ para todo $(x,y,z)\in\mathbb{K}^3$. Es decir, dar una fórmula para $T$ como la del ejercicio \ref{Txyz}.
        \item\label{matriz otro}  Encontrar una matriz $A\in\mathbb{K}^{3\times3}$ tal que
        $T(x,y,z)=A\begin{bmatrix}
        x\\y\\z \end{bmatrix}$. En esta parte del ejercicio escribiremos/pensaremos a los vectores de $\mathbb{K}^3$ como columnas.
        \end{enumerate}
    
        \rta
    
    
        
    \item\label{Txyz} Sea $T:\mathbb{K}^3\longrightarrow\mathbb{K}^3$ definida por $T(x,y,z)=(x+2y+3z, y-z,x+5y)$.
    \begin{enumerate}
    \item\label{Txyz-vectores-nucleo} Decir cuáles de los siguientes vectores están en el núcleo: $(1,1,1)$, $(-5,1,1)$.
    \item\label{Txyz imagen} Decir cuáles de los siguientes vectores están en la imagen: $(0,1,0)$, $(0,1,7)$.
    \item\label{Txyz nucleo imagen} Describir mediante ecuaciones (implícitamente) el núcleo y dar un conjunto de generadores de la imagen.
    \item\label{matriz} Encontrar una matriz  $A\in\mathbb{K}^{3\times 3}$ tal que $T(x,y,z)=A\left(\begin{matrix}
        x\\y\\z \end{matrix}
        \right)$.  Como en el ejercicio  \ref{T en la base}\,\ref{matriz otro} pensamos a los vectores como columnas.
    \end{enumerate}
    
    \rta
    
    
    
    \item\label{tl-matriz} Sea $T: \mathbb{K}^4 \to \mathbb{K}^5$ dada por $T(v) = Av$ donde $A$ es la siguiente matriz
        $$
        A=\begin{bmatrix}
        0& 2& 0&1\\   1& 3& 0&1\\  -1&-1&0&0\\3&0&3&0\\2&1&1&0 \end{bmatrix}
        $$
        \begin{enumerate}[topsep=5pt,itemsep=5pt]
            \item\label{tl-matriz-a} Dar una base del núcleo y de la imagen de $T$. 
            \item\label{tl-matriz-b} Dar la dimensión del núcleo y de la imagen de $T$.
            \item\label{tl-matriz-c} Describir mediante ecuaciones (implícitamente) el núcleo y la imagen de $T$.
            \item\label{tl-matriz-d} Decir cuáles de los siguientes vectores están en el núcleo:
            $(1,2,3,4)$, $(1,-1,-1,2)$, $(1,0,2,1)$.
            \item\label{tl-matriz-e} Decir cuáles de los siguientes vectores están en la imagen:
            $(2,3,-1,0,1)$, $(1,1,0,3,1)$, $(1,0,2,1,0)$.
        \end{enumerate}
        
    \rta

        \item Sea $T:\mathbb{K}^{2\times 2}\longrightarrow\mathbb{K}_{4}[x]$ la transformación lineal definida por
        \begin{align*}
        T   \begin{bmatrix}  a&b\\c&d \end{bmatrix} &= (a-c+2d)x^3+(b+2c-d)x^2+ \\
        &\qquad+(-a+2b+5c-4d)x+(2a-b-4c+5d)
        \end{align*}
        \begin{enumerate}
            \item\label{tl-matrices-pol-a} Decir cuáles de los siguientes matrices están en el núcleo:
                \begin{align*}
                    A=\begin{bmatrix}
                        2&0\\0&-1
                    \end{bmatrix},
                \quad
                B=\begin{bmatrix}
                    -1&-1\\1&1
                \end{bmatrix},
                \quad
                C=\begin{bmatrix}
                    -1&-1\\1&0
                \end{bmatrix}.
                \end{align*}
            \item\label{tl-matrices-pol-b} Decir cuáles de los siguientes polinomios están en la imagen:
                \begin{align*}
                    p(x)=x^3+x^2+x+1,\quad q(x)=x^3, \quad r(x)=(x-1)(x-1) 
                \end{align*}
        \end{enumerate}
    
    \rta
    
    \item\label{funcional ej}  Sea $T:\mathbb{K}^3\longrightarrow\mathbb{K}$ definida por $T(x,y,z)=x+2y+3z$.
    \begin{enumerate}
        \item\label{funcional ej a} Probar que $T$ es un epimorfismo.
        \item\label{funcional ej b} Dar la dimensión del núcleo de $T$.
        \item\label{funcional ej c} Encontrar una matriz $A$ tal que
            $T(x,y,z)=A\begin{bmatrix}
            x\\y\\z \end{bmatrix}$. ¿De qué tamaño debe ser $A$? Como en el ejercicio \ref{Txyz}\, \ref{matriz} pensamos a los vectores como columnas. 
    \end{enumerate}

    \rta
    
    
    \item Determinar cuáles transformaciones lineales de los ejercicios anteriores son monomorfismos, epimorfismos y/o isomorfismos.
    
    \rta


    \item\label{usar-1} Encontrar en cada caso, cuando sea posible, una matriz $A\in\mathbb{K}^{3\times 3}$ tal que la transformación lineal $T:\mathbb{K}^3\longrightarrow\mathbb{K}^3$, $T(v)=Av$, satisfaga las condiciones exigidas (como en el ejercicio  \ref{T en la base}\,\ref{matriz otro} pensamos a los vectores como columnas). Cuando no sea posible, explicar por qué no es posible.
    \begin{enumerate}[ topsep=5pt,itemsep=5pt]
        \item\label{usar-1 a} $\operatorname{dim} \operatorname{Im}(T)=2$ y $\operatorname{dim}\operatorname{Nu}(T)=2$.
        \item\label{usar-1 b} $T$ inyectiva y $T(e_1)=(1,0,0)$, $T(e_2)=(2,1,5)$ y $T(e_3)=(3,-1,0)$.
        \item\label{usar-1 c} $T$ sobreyectiva y $T(e_1)=(1,0,0)$, $T(e_2)=(2,1,5)$ y $T(e_3)=(3,-1,0)$.
        
        \item\label{usar Txyz} $T(e_1)=(1,0,0)$, $T(e_2)=(2,1,5)$ y $T(e_3)=(3,-1,0)$.
        
        \item\label{usar-1 e} $e_1\in\operatorname{Im}(T)$ y $(-5,1,1)\in\operatorname{Nu}(T)$.
        
        \item\label{usar-1 f} $\operatorname{dim} \operatorname{Im}(T)=2$.
    \end{enumerate}
        
    \rta 


    \item Decidir si las siguientes afirmaciones son verdaderas o falsas. Justificar.
    \begin{enumerate}
        \item\label{tl-VoF-a} Si $T : \mathbb R^{13} \to \mathbb R^9$ es una transformación lineal, entonces $\dim \operatorname{Nu}(T) \geq  4$.
        \item\label{tl-VoF-b} Sea $T:\mathbb{K}^{6}\longrightarrow\mathbb{K}^2$ un epimorfismo y $W$ un subespacio de $\mathbb{K}^{6}$ con $\dim W=3$. Entonces existe $0\neq w\in W$ tal que $T(w)=0$.
        \item\label{tl-VoF-c} Existe una transformación lineal $T : \mathbb R^2 \to \mathbb R^4$ tal que los vectores $(1, 0, -1, 2)$, $(0, 1, 2,-1,)$ y $(0, 0, 2, 2)$ pertenecen a la imagen de $T$.
    \end{enumerate}
    
    \rta



    \item \label{funcionales} Sea $V$ un espacio vectorial no nulo y $T:V\longrightarrow\mathbb{K}$ probar que $T=0$ ó $T$ es sobreyectiva.
    
    \rta


    \item Sea $V$ un espacio vectorial de dimensión finita y $T:V\longrightarrow V$ una transformación lineal. Probar las siguientes afirmaciones.
        \begin{multicols}{2}
            \begin{enumerate}
                \item $\operatorname{Nu}(T)\subseteq\operatorname{Nu}(T^2)$
                \item\label{dimV impar} $\operatorname{Nu}(T)\neq\operatorname{Im}(T)$ si $\dim(V)$ es impar.
            \end{enumerate}
        \end{multicols}
    
    \rta
        
    \end{enumerate}
%%%=======================
\end{document}