\begin{chapter}{Vectores en $\mathbb R^2$ y $\mathbb R^3$ (práctico)}\label{practico-6}


%============================================================
\section*{Objetivos}
%============================================================

\begin{itemize}
 \item Aprender las operaciones básicas de $\mathbb R^2$ y $\mathbb R^3$ (suma de vectores, multiplicación por escalares, producto escalar, calcular normas y \'angulos).
 \item Familiarizarse con los conceptos de ortogonalidad y paralelismo.
 \item Aprender a describir rectas y planos de forma imp\'icita y param\'etrica.
\end{itemize}



%============================================================
\section*{Vectores y producto escalar}
%============================================================

\begin{enumerate}


\item Dados $v = (-1, 2-0)$, $w = (2,-3,-1)$ y $u = (1,-1,1)$, calcular:
\begin{enumerate}
% 	\item $v + w$, 
% 	\item $v - w$, 
	\item $2v + 3w -5u$,
	\item $5(v+w)$, 
	\item $5v + 5w$ (y verificar que es igual al vector de arriba).
\end{enumerate}

\

\item Calcular los siguientes productos escalares. %\langle v , w  \rangle
\begin{enumerate}
  \item $\langle (-1, 2-0) ,(2,-3,-1) \rangle$, 
%   \item  $\langle (2,4,-3,-1),(1,-1,2, 1) \rangle$,
  \item  $\langle (4,-1),(-1,2) \rangle$.
\end{enumerate}

\

\item Dados $v = (-1, 2-0)$, $w = (2,-3,-1)$  y $u = (1,-1,1)$, verificar que:
\begin{equation*}
	\langle 2v + 3w , -u   \rangle = -2\langle v ,u \rangle -3 \langle w , u  \rangle
\end{equation*}

\ 

\item Probar  que 
\begin{enumerate}
	\item $(2,3,-1)$ y $(1, -2, -4)$ son ortogonales.
	\item $(2,-1)$ y $(1,2)$ son ortogonales. Dibujar en el plano. 
\end{enumerate}
\ 

\item Encontrar 
\begin{enumerate}
	\item un vector no nulo ortogonal  a $(3,-4)$,
	\item un vector no nulo ortogonal a $(2,-1,4)$,
	\item un vector no nulo ortogonal a $(2,-1,4)$ y $(0,1,-1)$,
\end{enumerate}

\



\ 
\item Encontrar la longitud de los vectores.
\begin{align*}
&(a) \ (2,3), && (b) \ (t,t^2), & (c) \ (\cos\phi,\sen\phi).
\end{align*}

\

\item Calcular $\langle v , w  \rangle$ y el {\'a}ngulo entre $v$ y $w$  para los siguientes vectores.
\begin{align*}
&(a) \ v=(2,2), w=(1,0), &&  (b) \  v=(-5,3,1), w=(2,-4,-7).
\end{align*}

\

\item Sea $v=(x_1,x_2,x_3)\in\mathbb{R}^3$ y recordar los vectores $e_!$, $e_2$ y $e_3$ dados en la p\'agina 12 del apunte. Verificar que 
$$v=x_1e_1+x_2e_2+x_3e_3=\langle v,e_1\rangle e_1+\langle v,e_2\rangle e_2+\langle v,e_3\rangle e_3.$$

\

\item Probar, usando sólo las propiedades \textbf{P1}, \textbf{P2}, y \textbf{P3} del producto escalar, que dados $v, w, u \in \mathbb R^n$ y $\lambda_1, \lambda_2 \in \mathbb R$, 
\begin{enumerate}
	\item se cumple:
	\begin{equation*}
	\langle \lambda_1 v + \lambda_2 w , u  \rangle =  \lambda_1\langle v , u  \rangle +   \lambda_2\langle w , u  \rangle.
	\end{equation*}
	\item Si $\langle v , w  \rangle =0$, es decir si $v$ y $w$ son ortogonales,  entonces
	\begin{equation*}
		\langle \lambda_1 v + \lambda_2 w ,  \lambda_1 v + \lambda_2 w   \rangle =
		\lambda_1^2 \langle  v ,  v  \rangle + \lambda_2^2 \langle w,w  \rangle.
	\end{equation*}
\end{enumerate}


\


\item Dados $v, w,\in \mathbb R^n$, probar que si  $\langle v , w  \rangle =0$, es decir si $v$ y $w$ son ortogonales,  entonces
	\begin{equation*}
	||v + w||^2 = ||v||^2 + ||w||^2.
	\end{equation*}
	¿Cuál es el nombre con que se conoce este resultado en $\mathbb R^2$?
	

\
 
\item Sean $v,w\in \mathbb R^2$, probar usando  solo la definición explícita del producto escalar en $\mathbb R^2$ que 
\begin{equation*}
	|\langle v , w  \rangle| \le ||v||\,||w|| \qquad \text{(Desigualdad de Schwarz).}
\end{equation*}
[Ayuda: elevar al cuadrado y aplicar la definición.]

\vskip .5cm

%============================================================
\section*{Rectas y planos}
%============================================================

\item En  cada uno de los siguientes casos determinar si los
vectores  $\overrightarrow{vw}$ y $\overrightarrow{xy}$ son
equivalentes y/o paralelos.
\begin{enumerate}
\item   $v=(1,-1)$,  $w=(4,3)$, $x=(-1,5)$, $y=(5,2)$. 
\item   $v=(1,-1,5)$,  $w=(-2,3,-4)$,  $x=(3,1,1)$,  $y=(-3,9,-17)$.
\end{enumerate}

\

\item Sea $R_1$ la recta que pasa por $p_1=(2,0)$ y es ortogonal a $(1,3)$.
\begin{enumerate}
 \item Dar la descripci\'on param{\'e}trica e impl{\'\i}cita de $R_1$.
 \item Graficar en el plano a $R_1$.
 \item Dar un punto $p$ por el que pase $R_1$ distinto a $p_1$.
 \item Verificar si $p+p_i$ y $-p$ pertenece a $R_1$
\end{enumerate}

\

\item Repetir el ejercicio anterior con las siguientes rectas.
\begin{enumerate}
	\item
	$R_2$: recta que pasa por $p_2=(0,0)$ y es ortogonal a $(1,3)$.
	\item
	$R_3$: recta que pasa por $p_3=(1,0)$ y es paralela a $R_1$.
% 	\item
% 	$R_4$: recta que pasa por los puntos $(-1,5,4)$ y $(0,3,-2)$.
\end{enumerate}

\

\item Calcular, num\'erica y graficamente, las intersecciones $R_1\cap R_2$ y $R_1\cap R_3$. 

\

\item Sea $v_0=(2,-1,1)$.
\begin{enumerate}
	\item Describir param{\'e}tricamente el conjunto
	$P_1=\{w\in\mathbb{ R}^3:\langle v_0 , w  \rangle=0\}$.
	\item Describir param{\'e}tricamente el conjunto
	$P_2=\{w\in\mathbb{ R}^3:\langle v_0 , w  \rangle=1\}$.
	\item ¿Qu\'e relaci\'on hay entre $P_1$ y $P_2$?
\end{enumerate}

\
\begin{comment}
\item Calcular.
\begin{enumerate}
	\item El {\'a}rea del tri{\'a}ngulo de v{\'e}rtices $(1,2)$, $(-1,2)$, $(2,4)$.
	\item El volumen del paralelep{\'\i}pedo definido por los vectores $(-2,3,1)$, $(1,1,2)$, $(1,2,3)$.
\end{enumerate}

\
\end{comment}



\item\label{ej-planos} Escribir la ecuaci{\'o}n paramétrica  y la ecuaci{\'o}n normal de los siguientes planos.
\begin{enumerate}
	\item $\pi_1$: el plano que pasa por $(0,0,0)$, $(1,1,0)$, $(1,-2,0)$.
	\item $\pi_2$: el plano que pasa por $(1,2,-2)$ y es perpendicular a la
	recta que pasa por $(2,1,-1)$, $(3,-2,1)$.
	\item\label{ej-planos-c}  $\pi_3=\{w\in\mathbb{R}^3: w=s(1,2,0)+t(2,0,1)+(1,0,0);\,s,t\in \mathbb R\}$.
\end{enumerate}

\


\item ¿Cu\'ales de las siguientes rectas cortan al plano $\pi_3$ del  ejercicio \ref{ej-planos-c})?
Describir la intersecci{\'o}n en cada caso.
\begin{align*}
&(a) \ \{w: w=(3,2,1)+t(1,1,1)\}, && (b) \  \{w: w=(1,-1,1)+t(1,2,-1)\}, \\
&(c)\  \{w: w=(-1,0,-1)+t(1,2,-1)\}, && (d) \  \{w: w=(1,-2,1)+t(2,-1,1)\}.
\end{align*}

\

\item Sea $L=\{(x,y)\in\mathbb{R}^2 : ax+by=c\}$ una recta en $\mathbb{R}^2$. Sean $p$ y $q$ dos puntos por los que pasa $L$.
\begin{enumerate}
 \item ¿Para qu\'e valores de $c$ puede asegurar que $(0,0)\in L$?
 \item ¿Para qu\'e valores de $c$ puede asegurar que $\lambda q\in L$?, donde $\lambda\in\mathbb{R}$.
 \item ¿Para qu\'e valores de $c$ puede asegurar que $p+q\in L$?
\end{enumerate}


\

\item Sea $L$ una recta en $\mathbb{R}^2$. Probar que $L$ pasa por $(0,0)$ si y s\'olo si pasa por $p+\lambda q$ para todo par de puntos distintos $p$ y $q$ de $L$ y para todo $\lambda\in\mathbb{R}$.


%============================================================
\section*{Ejercicios de repaso}

 Si ya hizo los ejercicios anteriores continue a la siguiente gu\'ia. Los ejercicios que siguen son similares a los anteriores y le pueden servir para practicar antes de los ex\'amenes.
%============================================================
\item Probar, usando sólo las propiedades \textbf{P1}, \textbf{P2}, y \textbf{P3} del producto escalar, que dados $v, w, u \in \mathbb R^n$ y $\lambda_1, \lambda_2 \in \mathbb R$, 
\begin{enumerate}	
	\item $||\lambda_1 v|| = |\lambda_1|\, ||v||$.
	\item  
	\begin{equation*}
	\langle \lambda_1 v + \lambda_2 w ,  \lambda_1 v + \lambda_2 w   \rangle =
	\lambda_1^2 || v||^2 + 2  \langle v,w  \rangle + \lambda_2^2 ||w||^2 .
	\end{equation*}
\end{enumerate}

\

\item ¿Qu\'e parejas de vectores son perpendiculares entre s{\'\i}?
\begin{align*}
&(a) \ (1, -1,1) \text{ y } (2,1,5), && (b)  \ (1,-1,1) \text{ y } (2,3,1), \\
&(c) \  (-5,2,7)  \text{ y } (3,-1,2) && (d)  \ (\pi,2,1) \text{ y }  (2, -\pi,0).
\end{align*}

\
	
\item Dados $v, w,\in \mathbb R^n$, probar que
\begin{equation} \label{eq-ley-pa}
||v + w||^2  + ||v - w||^2= 2||v||^2 + 2||w||^2 . \tag{*}
\end{equation}
Hay un resultado  clásico de la geometría elemental que dice \textit{``la suma de los cuadrados de las longitudes de los cuatro lados de un paralelogramo es igual a la suma de los cuadrados de las longitudes de las dos diagonales de éste''} (Ley del paralelogramo). Relacione geométricamente el  resultado (\ref{eq-ley-pa}) aplicado  a $\mathbb R^2$  con  la Ley del paralelogramo.


\begin{comment}
	content

\item Hallar la distancia y el punto que realiza la distancia.
\begin{enumerate}
	\item  entre recta $R_2$ y el  punto $Q=(4,3,2)$.
	\item entre plano $\pi_1$ y el punto $Q=(1,1,1)$.
\end{enumerate}

\

\item
Demostrar las siguientes proposiciones.
\begin{enumerate}
	\item ${X} . (Y \times Z)=\det\left(\begin{smallmatrix} X\\Y\\Z\end{smallmatrix}\right)=
	\det \left(\begin{smallmatrix}
	x_1&x_2&x_3\\
	y_1&y_2&y_3\\
	z_1&z_2&z_3
	\end{smallmatrix}\right)$.
	\item $||{X} \times {Y}||^2 = ||X||^2||Y||^2 - (X.Y)^2$.
	\item $||X \times Y|| =||X||\,||Y||\, \sen(\theta)$ donde $\theta$ es el
	\'angulo entre $X$ e $Y$.
	\item $X \times Y=0$ si y s\'olo si $X$ e $Y$ son paralelos.
\end{enumerate}

\end{comment}



\end{enumerate}




\end{chapter}
